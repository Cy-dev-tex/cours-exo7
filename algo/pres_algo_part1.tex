
%%%%%%%%%%%%%%%%%% PREAMBULE %%%%%%%%%%%%%%%%%%

\documentclass[aspectratio=169,utf8]{beamer}
%\documentclass[aspectratio=169,handout]{beamer}

\usetheme{Boadilla}
%\usecolortheme{seahorse}
\usecolortheme[RGB={245,66,24}]{structure}
\useoutertheme{infolines}

% packages
\usepackage{amsfonts,amsmath,amssymb,amsthm}
\usepackage[utf8]{inputenc}
\usepackage[T1]{fontenc}
\usepackage{lmodern}

\usepackage[francais]{babel}
\usepackage{fancybox}
\usepackage{graphicx}

\usepackage{float}
\usepackage{xfrac}

%\usepackage[usenames, x11names]{xcolor}
\usepackage{tikz}
\usepackage{pgfplots}
\usepackage{datetime}



%-----  Package unités -----
\usepackage{siunitx}
\sisetup{locale = FR,detect-all,per-mode = symbol}

%\usepackage{mathptmx}
%\usepackage{fouriernc}
%\usepackage{newcent}
%\usepackage[mathcal,mathbf]{euler}

%\usepackage{palatino}
%\usepackage{newcent}
% \usepackage[mathcal,mathbf]{euler}



% \usepackage{hyperref}
% \hypersetup{colorlinks=true, linkcolor=blue, urlcolor=blue,
% pdftitle={Exo7 - Exercices de mathématiques}, pdfauthor={Exo7}}


%section
% \usepackage{sectsty}
% \allsectionsfont{\bf}
%\sectionfont{\color{Tomato3}\upshape\selectfont}
%\subsectionfont{\color{Tomato4}\upshape\selectfont}

%----- Ensembles : entiers, reels, complexes -----
\newcommand{\Nn}{\mathbb{N}} \newcommand{\N}{\mathbb{N}}
\newcommand{\Zz}{\mathbb{Z}} \newcommand{\Z}{\mathbb{Z}}
\newcommand{\Qq}{\mathbb{Q}} \newcommand{\Q}{\mathbb{Q}}
\newcommand{\Rr}{\mathbb{R}} \newcommand{\R}{\mathbb{R}}
\newcommand{\Cc}{\mathbb{C}} 
\newcommand{\Kk}{\mathbb{K}} \newcommand{\K}{\mathbb{K}}

%----- Modifications de symboles -----
\renewcommand{\epsilon}{\varepsilon}
\renewcommand{\Re}{\mathop{\text{Re}}\nolimits}
\renewcommand{\Im}{\mathop{\text{Im}}\nolimits}
%\newcommand{\llbracket}{\left[\kern-0.15em\left[}
%\newcommand{\rrbracket}{\right]\kern-0.15em\right]}

\renewcommand{\ge}{\geqslant}
\renewcommand{\geq}{\geqslant}
\renewcommand{\le}{\leqslant}
\renewcommand{\leq}{\leqslant}
\renewcommand{\epsilon}{\varepsilon}

%----- Fonctions usuelles -----
\newcommand{\ch}{\mathop{\text{ch}}\nolimits}
\newcommand{\sh}{\mathop{\text{sh}}\nolimits}
\renewcommand{\tanh}{\mathop{\text{th}}\nolimits}
\newcommand{\cotan}{\mathop{\text{cotan}}\nolimits}
\newcommand{\Arcsin}{\mathop{\text{arcsin}}\nolimits}
\newcommand{\Arccos}{\mathop{\text{arccos}}\nolimits}
\newcommand{\Arctan}{\mathop{\text{arctan}}\nolimits}
\newcommand{\Argsh}{\mathop{\text{argsh}}\nolimits}
\newcommand{\Argch}{\mathop{\text{argch}}\nolimits}
\newcommand{\Argth}{\mathop{\text{argth}}\nolimits}
\newcommand{\pgcd}{\mathop{\text{pgcd}}\nolimits} 


%----- Commandes divers ------
\newcommand{\ii}{\mathrm{i}}
\newcommand{\dd}{\text{d}}
\newcommand{\id}{\mathop{\text{id}}\nolimits}
\newcommand{\Ker}{\mathop{\text{Ker}}\nolimits}
\newcommand{\Card}{\mathop{\text{Card}}\nolimits}
\newcommand{\Vect}{\mathop{\text{Vect}}\nolimits}
\newcommand{\Mat}{\mathop{\text{Mat}}\nolimits}
\newcommand{\rg}{\mathop{\text{rg}}\nolimits}
\newcommand{\tr}{\mathop{\text{tr}}\nolimits}


%----- Structure des exercices ------

\newtheoremstyle{styleexo}% name
{2ex}% Space above
{3ex}% Space below
{}% Body font
{}% Indent amount 1
{\bfseries} % Theorem head font
{}% Punctuation after theorem head
{\newline}% Space after theorem head 2
{}% Theorem head spec (can be left empty, meaning ‘normal’)

%\theoremstyle{styleexo}
\newtheorem{exo}{Exercice}
\newtheorem{ind}{Indications}
\newtheorem{cor}{Correction}


\newcommand{\exercice}[1]{} \newcommand{\finexercice}{}
%\newcommand{\exercice}[1]{{\tiny\texttt{#1}}\vspace{-2ex}} % pour afficher le numero absolu, l'auteur...
\newcommand{\enonce}{\begin{exo}} \newcommand{\finenonce}{\end{exo}}
\newcommand{\indication}{\begin{ind}} \newcommand{\finindication}{\end{ind}}
\newcommand{\correction}{\begin{cor}} \newcommand{\fincorrection}{\end{cor}}

\newcommand{\noindication}{\stepcounter{ind}}
\newcommand{\nocorrection}{\stepcounter{cor}}

\newcommand{\fiche}[1]{} \newcommand{\finfiche}{}
\newcommand{\titre}[1]{\centerline{\large \bf #1}}
\newcommand{\addcommand}[1]{}
\newcommand{\video}[1]{}

% Marge
\newcommand{\mymargin}[1]{\marginpar{{\small #1}}}

\def\noqed{\renewcommand{\qedsymbol}{}}


%----- Presentation ------
\setlength{\parindent}{0cm}

%\newcommand{\ExoSept}{\href{http://exo7.emath.fr}{\textbf{\textsf{Exo7}}}}

\definecolor{myred}{rgb}{0.93,0.26,0}
\definecolor{myorange}{rgb}{0.97,0.58,0}
\definecolor{myyellow}{rgb}{1,0.86,0}

\newcommand{\LogoExoSept}[1]{  % input : echelle
{\usefont{U}{cmss}{bx}{n}
\begin{tikzpicture}[scale=0.1*#1,transform shape]
  \fill[color=myorange] (0,0)--(4,0)--(4,-4)--(0,-4)--cycle;
  \fill[color=myred] (0,0)--(0,3)--(-3,3)--(-3,0)--cycle;
  \fill[color=myyellow] (4,0)--(7,4)--(3,7)--(0,3)--cycle;
  \node[scale=5] at (3.5,3.5) {Exo7};
\end{tikzpicture}}
}


\newcommand{\debutmontitre}{
  \author{} \date{} 
  \thispagestyle{empty}
  \hspace*{-10ex}
  \begin{minipage}{\textwidth}
    \titlepage  
  \vspace*{-2.5cm}
  \begin{center}
    \LogoExoSept{2.5}
  \end{center}
  \end{minipage}

  \vspace*{-0cm}
  
  % Astuce pour que le background ne soit pas discrétisé lors de la conversion pdf -> png
\begin{tikzpicture}
        \fill[opacity=0,green!60!black] (0,0)--++(0,0)--++(0,0)--++(0,0)--cycle; 
\end{tikzpicture}

% toc S'affiche trop tot :
% \tableofcontents[hideallsubsections, pausesections]
}

\newcommand{\finmontitre}{
  \end{frame}
  \setcounter{framenumber}{0}
} % ne marche pas pour une raison obscure

%----- Commandes supplementaires ------

% \usepackage[landscape]{geometry}
% \geometry{top=1cm, bottom=3cm, left=2cm, right=10cm, marginparsep=1cm
% }
% \usepackage[a4paper]{geometry}
% \geometry{top=2cm, bottom=2cm, left=2cm, right=2cm, marginparsep=1cm
% }

%\usepackage{standalone}


% New command Arnaud -- november 2011
\setbeamersize{text margin left=24ex}
% si vous modifier cette valeur il faut aussi
% modifier le decalage du titre pour compenser
% (ex : ici =+10ex, titre =-5ex

\theoremstyle{definition}
%\newtheorem{proposition}{Proposition}
%\newtheorem{exemple}{Exemple}
%\newtheorem{theoreme}{Théorème}
%\newtheorem{lemme}{Lemme}
%\newtheorem{corollaire}{Corollaire}
%\newtheorem*{remarque*}{Remarque}
%\newtheorem*{miniexercice}{Mini-exercices}
%\newtheorem{definition}{Définition}

% Commande tikz
\usetikzlibrary{calc}
\usetikzlibrary{patterns,arrows}
\usetikzlibrary{matrix}
\usetikzlibrary{fadings} 

%definition d'un terme
\newcommand{\defi}[1]{{\color{myorange}\textbf{\emph{#1}}}}
\newcommand{\evidence}[1]{{\color{blue}\textbf{\emph{#1}}}}
\newcommand{\assertion}[1]{\emph{\og#1\fg}}  % pour chapitre logique
%\renewcommand{\contentsname}{Sommaire}
\renewcommand{\contentsname}{}
\setcounter{tocdepth}{2}



%------ Figures ------

\def\myscale{1} % par défaut 
\newcommand{\myfigure}[2]{  % entrée : echelle, fichier figure
\def\myscale{#1}
\begin{center}
\footnotesize
{#2}
\end{center}}


%------ Encadrement ------

\usepackage{fancybox}


\newcommand{\mybox}[1]{
\setlength{\fboxsep}{7pt}
\begin{center}
\shadowbox{#1}
\end{center}}

\newcommand{\myboxinline}[1]{
\setlength{\fboxsep}{5pt}
\raisebox{-10pt}{
\shadowbox{#1}
}
}

%--------------- Commande beamer---------------
\newcommand{\beameronly}[1]{#1} % permet de mettre des pause dans beamer pas dans poly


\setbeamertemplate{navigation symbols}{}
\setbeamertemplate{footline}  % tiré du fichier beamerouterinfolines.sty
{
  \leavevmode%
  \hbox{%
  \begin{beamercolorbox}[wd=.333333\paperwidth,ht=2.25ex,dp=1ex,center]{author in head/foot}%
    % \usebeamerfont{author in head/foot}\insertshortauthor%~~(\insertshortinstitute)
    \usebeamerfont{section in head/foot}{\bf\insertshorttitle}
  \end{beamercolorbox}%
  \begin{beamercolorbox}[wd=.333333\paperwidth,ht=2.25ex,dp=1ex,center]{title in head/foot}%
    \usebeamerfont{section in head/foot}{\bf\insertsectionhead}
  \end{beamercolorbox}%
  \begin{beamercolorbox}[wd=.333333\paperwidth,ht=2.25ex,dp=1ex,right]{date in head/foot}%
    % \usebeamerfont{date in head/foot}\insertshortdate{}\hspace*{2em}
    \insertframenumber{} / \inserttotalframenumber\hspace*{2ex} 
  \end{beamercolorbox}}%
  \vskip0pt%
}


\definecolor{mygrey}{rgb}{0.5,0.5,0.5}
\setlength{\parindent}{0cm}
%\DeclareTextFontCommand{\helvetica}{\fontfamily{phv}\selectfont}

% background beamer
\definecolor{couleurhaut}{rgb}{0.85,0.9,1}  % creme
\definecolor{couleurmilieu}{rgb}{1,1,1}  % vert pale
\definecolor{couleurbas}{rgb}{0.85,0.9,1}  % blanc
\setbeamertemplate{background canvas}[vertical shading]%
[top=couleurhaut,middle=couleurmilieu,midpoint=0.4,bottom=couleurbas] 
%[top=fondtitre!05,bottom=fondtitre!60]



\makeatletter
\setbeamertemplate{theorem begin}
{%
  \begin{\inserttheoremblockenv}
  {%
    \inserttheoremheadfont
    \inserttheoremname
    \inserttheoremnumber
    \ifx\inserttheoremaddition\@empty\else\ (\inserttheoremaddition)\fi%
    \inserttheorempunctuation
  }%
}
\setbeamertemplate{theorem end}{\end{\inserttheoremblockenv}}

\newenvironment{theoreme}[1][]{%
   \setbeamercolor{block title}{fg=structure,bg=structure!40}
   \setbeamercolor{block body}{fg=black,bg=structure!10}
   \begin{block}{{\bf Th\'eor\`eme }#1}
}{%
   \end{block}%
}


\newenvironment{proposition}[1][]{%
   \setbeamercolor{block title}{fg=structure,bg=structure!40}
   \setbeamercolor{block body}{fg=black,bg=structure!10}
   \begin{block}{{\bf Proposition }#1}
}{%
   \end{block}%
}

\newenvironment{corollaire}[1][]{%
   \setbeamercolor{block title}{fg=structure,bg=structure!40}
   \setbeamercolor{block body}{fg=black,bg=structure!10}
   \begin{block}{{\bf Corollaire }#1}
}{%
   \end{block}%
}

\newenvironment{mydefinition}[1][]{%
   \setbeamercolor{block title}{fg=structure,bg=structure!40}
   \setbeamercolor{block body}{fg=black,bg=structure!10}
   \begin{block}{{\bf Définition} #1}
}{%
   \end{block}%
}

\newenvironment{lemme}[0]{%
   \setbeamercolor{block title}{fg=structure,bg=structure!40}
   \setbeamercolor{block body}{fg=black,bg=structure!10}
   \begin{block}{\bf Lemme}
}{%
   \end{block}%
}

\newenvironment{remarque}[1][]{%
   \setbeamercolor{block title}{fg=black,bg=structure!20}
   \setbeamercolor{block body}{fg=black,bg=structure!5}
   \begin{block}{Remarque #1}
}{%
   \end{block}%
}


\newenvironment{exemple}[1][]{%
   \setbeamercolor{block title}{fg=black,bg=structure!20}
   \setbeamercolor{block body}{fg=black,bg=structure!5}
   \begin{block}{{\bf Exemple }#1}
}{%
   \end{block}%
}


\newenvironment{miniexercice}[0]{%
   \setbeamercolor{block title}{fg=structure,bg=structure!20}
   \setbeamercolor{block body}{fg=black,bg=structure!5}
   \begin{block}{Mini-exercices}
}{%
   \end{block}%
}


\newenvironment{tp}[0]{%
   \setbeamercolor{block title}{fg=structure,bg=structure!40}
   \setbeamercolor{block body}{fg=black,bg=structure!10}
   \begin{block}{\bf Travaux pratiques}
}{%
   \end{block}%
}
\newenvironment{exercicecours}[1][]{%
   \setbeamercolor{block title}{fg=structure,bg=structure!40}
   \setbeamercolor{block body}{fg=black,bg=structure!10}
   \begin{block}{{\bf Exercice }#1}
}{%
   \end{block}%
}
\newenvironment{algo}[1][]{%
   \setbeamercolor{block title}{fg=structure,bg=structure!40}
   \setbeamercolor{block body}{fg=black,bg=structure!10}
   \begin{block}{{\bf Algorithme}\hfill{\color{gray}\texttt{#1}}}
}{%
   \end{block}%
}


\setbeamertemplate{proof begin}{
   \setbeamercolor{block title}{fg=black,bg=structure!20}
   \setbeamercolor{block body}{fg=black,bg=structure!5}
   \begin{block}{{\footnotesize Démonstration}}
   \footnotesize
   \smallskip}
\setbeamertemplate{proof end}{%
   \end{block}}
\setbeamertemplate{qed symbol}{\openbox}


\makeatother
\usecolortheme[RGB={192,41,0}]{structure}

% Commande spécifique à ce chapitre

\newcommand{\Python}{\texttt{Python}}
\renewcommand{\evidence}[1]{{\color{blue}\textbf{#1}}}

\usepackage{textcomp}

\usepackage{listings}
\lstset{
  upquote=true,
  columns=flexible,
  keepspaces=true,
  basicstyle=\ttfamily,
  commentstyle=\color{gray},
  language=Python,
  showstringspaces=false,
  aboveskip=0em,  
  belowskip=0em,
  escapeinside=||
}

\lstset{
  literate={é}{{\'e}}1
           {è}{{\`e}}1
           {à}{{\`a}}1
}


\newcommand{\codeinline}[1]{\lstinline!#1!}

%%%%%%%%%%%%%%%%%%%%%%%%%%%%%%%%%%%%%%%%%%%%%%%%%%%%%%%%%%%%%
%%%%%%%%%%%%%%%%%%%%%%%%%%%%%%%%%%%%%%%%%%%%%%%%%%%%%%%%%%%%%


\begin{document}


\title{{\bf Algorithmes et mathématiques}}
\subtitle{Premiers pas avec \Python}

\begin{frame}
  
  \debutmontitre

  \pause

{\footnotesize
\hfill
\setbeamercovered{transparent=50}
\begin{minipage}{0.6\textwidth}
  \begin{itemize}
    \item<3-> Hello world!
    \item<4-> Somme des cubes
    \item<5-> Calcul de $\pi$ au hasard
  \end{itemize}
\end{minipage}
}

\end{frame}

\setcounter{framenumber}{0}


%%%%%%%%%%%%%%%%%%%%%%%%%%%%%%%%%%%%%%%%%%%%%%%%%%%%%%%%%%%%%%%%
\section{Hello world!}

\begin{frame}[fragile]


\begin{tp}
\begin{enumerate}
  \item Définir deux variables prenant les valeurs $3$ et $6$.
  \item Calculer leur somme et leur produit.
\end{enumerate}  
\end{tp}

\pause

\begin{algo}[hello-world.py]
\begin{lstlisting}
>>> a=3  |\pause|
>>> b=6  |\pause|
>>> somme = a+b  |\pause|
>>> print(somme)  |\pause|
9  |\pause|
>>> print("La somme est", somme)  |\pause|
La somme est 9  |\pause|
>>> produit = a*b  |\pause|
>>> print("Le produit est", produit)  |\pause|
Le produit est 18
\end{lstlisting}  
\end{algo}

\end{frame}


%%%%%%%%%%%%%%%%%%%%%%%%%%%%%%%%%%%%%%%%%%%%%%%%%%%%%%%%%%%%%%%%
\section{Somme des cubes}

\begin{frame}

\begin{tp}
\begin{enumerate}
  \item Pour un entier $n$ fixé, programmer le calcul de la somme $S_n = 1^3+2^3+3^3+ \cdots + n^3$. 
  \item Définir une fonction qui pour une valeur $n$ renvoie la somme $\Sigma_n = 1+2+3+\cdots+n$.  
  \item Définir une fonction qui pour une valeur $n$ renvoie $S_n$.
  \item Vérifier, pour les premiers entiers, que $S_n = (\Sigma_n)^2$.
\end{enumerate}  
\end{tp}

\end{frame}


\begin{frame}[fragile]

\begin{algo}[somme-cubes.py (1)]
\begin{lstlisting}
n = 10
somme = 0
for i in range(1,n+1):
    somme = somme + i*i*i
print(somme)
\end{lstlisting}  
\end{algo}

\pause

$$\Sigma_n = 1+2+3+\cdots+n \pause =\frac{n(n+1)}{2}$$

\pause

\begin{algo}[somme-cubes.py (2)]
\begin{lstlisting}
def somme_entiers(n):
    return n*(n+1)/2
\end{lstlisting}  
\end{algo}

\end{frame}


\begin{frame}[fragile]

\begin{itemize}
  \item Le signe égal \codeinline{=} pour affecter une valeur à une variable
\pause   
  \item \codeinline{print()} affiche un message

\pause   
  \item Fonction 
  \begin{itemize}
    \item Commence par \codeinline{def ma_fonction(variable):} 
 \pause
    \item Finie par \codeinline{return resultat}
  \end{itemize}     
\pause
  \item
   \begin{itemize}
    \item \codeinline{range(1,n+1)} est $\{1,2,\ldots,n\}$
\pause
    \item \codeinline{range(n)} est $\{0,1,2,\ldots,n-1\}$
  \end{itemize} 
\pause
  \item L'indentation détermine les blocs d'instructions
\end{itemize}
\end{frame}  
 

\begin{frame}[fragile]
\begin{algo}[somme-cubes.py (3)]
\begin{lstlisting}
def somme_cubes(n):
    somme = 0
    for i in range(1,n+1):
        somme = somme + i**3
    return somme 
\end{lstlisting}  
\end{algo}

\pause

\begin{algo}[somme-cubes.py (4)]
\begin{lstlisting}
n = 12
if somme_cubes(n) == (somme_entiers(n)**2):
    print("Pour n=", n, "l'assertion est vraie.")
else:   
    print("L'assertion est fausse !")
\end{lstlisting}  
\end{algo}
\end{frame}




 
 
\begin{frame}

\begin{itemize} 

  \item \codeinline{**} pour les puissances : $5^2$ s'écrit \codeinline{5*5} ou \codeinline{5**2},
    $5^3$ s'écrit \codeinline{5**3},...  
 
\pause   
  \item \evidence{Boucle pour}
avec l'instruction \evidence{\codeinline{for}} qui fait varier un indice
\pause

  \item \evidence{\codeinline{if condition: ... else: ...}} 
  exécute le premier 
  bloc d'instructions si la condition est vraie ; 
   si la condition est fausse cela exécute l'autre bloc

\pause
   
  \item
  \begin{itemize}
     \item \codeinline{a < b} : $a<b$
     \item \codeinline{a <= b} : $a \le b$
     \item \codeinline{a == b} : $a=b$
     \item \codeinline{a != b} : $a \neq b$
  \end{itemize} 

\pause

  \item
  \begin{itemize}
     \item \evidence{\codeinline{a==b}} vaut soit vraie ou faux
\pause

     \item \evidence{\codeinline{a=b}} on affecte dans $a$ la valeur de $b$
  \end{itemize}
\end{itemize}
\end{frame}


%%%%%%%%%%%%%%%%%%%%%%%%%%%%%%%%%%%%%%%%%%%%%%%%%%%%%%%%%%%%%%%%
\section{Calcul de $\pi$ au hasard}

\begin{frame}

\begin{tp}
\begin{enumerate}
  \item Calculer l'aire du carré et de la portion de disque.
  \item Pour un point $(x,y)$ tiré au hasard dans le carré, quelle est la probabilité que le point soit en fait dans la portion de disque ?
  \item Tirer un grand nombre de points au hasard, compter ceux qui sont dans la portion de disque.
  \item En déduire les premières décimales de $\pi$.
\end{enumerate} 
\end{tp}

\myfigure{1}{
\begin{tikzpicture}[scale=3]
  \fill[myred] (0,0) -- (1,0) arc (0:90:1) -- cycle;
  \draw[thick] (0,0)--(0,1)--(1,1)--(1,0)--cycle;
  \draw[thick] (1,0) arc (0:90:1);
  \draw (0,0) node[below] {$(0,0)$};
  \draw (1,0) node[below] {$(1,0)$};
  \draw (0,1) node[left] {$(0,1)$};
\end{tikzpicture}
 
}
\end{frame}


\begin{frame}[fragile]

\begin{algo}[pi-hasard.py]
\begin{lstlisting}
import random            # Module nombres aléatoires 
|\pause|
Tir = 0                  # Numéro du tir
NbTirDansLeDisque = 0    # Nombre de tirs dans le disque
|\pause|
while (Tir < 1000):
    Tir = Tir + 1
    # On tire au hasard un point (x,y) dans [0,1] x [0,1]
    x = random.random()   
    y = random.random()
    if (x*x+y*y <= 1):    # On est dans le disque
        NbTirDansLeDisque = NbTirDansLeDisque + 1
|\pause|        
MonPi = 4*NbTirDansLeDisque / Tir
print("Valeur expérimentale de Pi : %0.3f" %MonPi)
\end{lstlisting}  
\end{algo}
\end{frame}


\begin{frame}[fragile]

\begin{itemize}


  \item \evidence{Boucle tant que}  \evidence{\codeinline{while condition: ...}} 
Tant que la condition est vérifiée les instructions de la boucle sont exécutées

   
\pause   

  \item \codeinline{random()} renvoie un nombre réel de l'intervalle $[0,1[$ au hasard 
  
\pause

  \item
  \begin{itemize}
    \item \evidence{module} : \codeinline{import random}
    \item appel par \codeinline{module.fonction()} donc ici \codeinline{random.random()}
  \end{itemize}
   
\pause 

  \item Commentaires : un dièse \codeinline{#}


\end{itemize}
\end{frame}





%%%%%%%%%%%%%%%%%%%%%%%%%%%%%%%%%%%%%%%%%%%%%%%%%%%%%%%%%%%%%%%%
\section{Mini-exercices}

\begin{frame}

\small
\begin{miniexercice}
\begin{enumerate}
  \item Soit le produit $P_n = (1-\frac12)\times (1-\frac13) \times (1-\frac14)\times \cdots \times(1-\frac1n)$.
  Calculer une valeur approchée de $P_n$ pour les premiers entiers $n$.
  
  \item Que vaut la somme des entiers $i$ qui apparaissent dans l'instruction \codeinline{for i in range(1,10)}.
  Idem pour \codeinline{for i in range(11)}. Idem pour \codeinline{for i in range(1,10,2)}. Idem pour \codeinline{for i in range(0,10,2)}.
  Idem pour \codeinline{for i in range(10,0,-1)}.
  
  \item On considère le cube $[0,1] \times [0,1] \times [0,1]$ et la portion de boule de rayon $1$ centrée à l'origine incluse dans ce cube.
  Faire les calculs de probabilité pour un point tiré au hasard dans le cube d'être en fait dans la portion de boule.
  Faire une fonction pour le vérifier expérimentalement.
  
  \item On lance deux dés. Expérimenter quelle est la probabilité que la somme soit $7$, puis $6$, puis $3$ ?
  Quelle est la probabilité que l'un des deux dés soit un $6$ ? d'avoir un double ?
  La fonction \codeinline{randint(a, b)} du module \codeinline{random} retourne un entier $k$ 
  au hasard, vérifiant $a \le k \le b$.
  
  \item On lance un dé jusqu'à ce que l'on obtienne un $6$. En moyenne au bout de combien de lancer s'arrête-t-on ? 
\end{enumerate}
\end{miniexercice}

\end{frame}

\end{document}