
%%%%%%%%%%%%%%%%%% PREAMBULE %%%%%%%%%%%%%%%%%%


\documentclass[12pt]{article}

\usepackage{amsfonts,amsmath,amssymb,amsthm}
\usepackage[utf8]{inputenc}
\usepackage[T1]{fontenc}
\usepackage[francais]{babel}


% packages
\usepackage{amsfonts,amsmath,amssymb,amsthm}
\usepackage[utf8]{inputenc}
\usepackage[T1]{fontenc}
%\usepackage{lmodern}

\usepackage[francais]{babel}
\usepackage{fancybox}
\usepackage{graphicx}

\usepackage{float}

%\usepackage[usenames, x11names]{xcolor}
\usepackage{tikz}
\usepackage{datetime}

\usepackage{mathptmx}
%\usepackage{fouriernc}
%\usepackage{newcent}
\usepackage[mathcal,mathbf]{euler}

%\usepackage{palatino}
%\usepackage{newcent}


% Commande spéciale prompteur

%\usepackage{mathptmx}
%\usepackage[mathcal,mathbf]{euler}
%\usepackage{mathpple,multido}

\usepackage[a4paper]{geometry}
\geometry{top=2cm, bottom=2cm, left=1cm, right=1cm, marginparsep=1cm}

\newcommand{\change}{{\color{red}\rule{\textwidth}{1mm}\\}}

\newcounter{mydiapo}

\newcommand{\diapo}{\newpage
\hfill {\normalsize  Diapo \themydiapo \quad \texttt{[\jobname]}} \\
\stepcounter{mydiapo}}


%%%%%%% COULEURS %%%%%%%%%%

% Pour blanc sur noir :
%\pagecolor[rgb]{0.5,0.5,0.5}
% \pagecolor[rgb]{0,0,0}
% \color[rgb]{1,1,1}



%\DeclareFixedFont{\myfont}{U}{cmss}{bx}{n}{18pt}
\newcommand{\debuttexte}{
%%%%%%%%%%%%% FONTES %%%%%%%%%%%%%
\renewcommand{\baselinestretch}{1.5}
\usefont{U}{cmss}{bx}{n}
\bfseries

% Taille normale : commenter le reste !
%Taille Arnaud
%\fontsize{19}{19}\selectfont

% Taille Barbara
%\fontsize{21}{22}\selectfont

%Taille François
%\fontsize{25}{30}\selectfont

%Taille Pascal
%\fontsize{25}{30}\selectfont

%Taille Laura
%\fontsize{30}{35}\selectfont


%\myfont
%\usefont{U}{cmss}{bx}{n}

%\Huge
%\addtolength{\parskip}{\baselineskip}
}


% \usepackage{hyperref}
% \hypersetup{colorlinks=true, linkcolor=blue, urlcolor=blue,
% pdftitle={Exo7 - Exercices de mathématiques}, pdfauthor={Exo7}}


%section
% \usepackage{sectsty}
% \allsectionsfont{\bf}
%\sectionfont{\color{Tomato3}\upshape\selectfont}
%\subsectionfont{\color{Tomato4}\upshape\selectfont}

%----- Ensembles : entiers, reels, complexes -----
\newcommand{\Nn}{\mathbb{N}} \newcommand{\N}{\mathbb{N}}
\newcommand{\Zz}{\mathbb{Z}} \newcommand{\Z}{\mathbb{Z}}
\newcommand{\Qq}{\mathbb{Q}} \newcommand{\Q}{\mathbb{Q}}
\newcommand{\Rr}{\mathbb{R}} \newcommand{\R}{\mathbb{R}}
\newcommand{\Cc}{\mathbb{C}} 
\newcommand{\Kk}{\mathbb{K}} \newcommand{\K}{\mathbb{K}}

%----- Modifications de symboles -----
\renewcommand{\epsilon}{\varepsilon}
\renewcommand{\Re}{\mathop{\text{Re}}\nolimits}
\renewcommand{\Im}{\mathop{\text{Im}}\nolimits}
%\newcommand{\llbracket}{\left[\kern-0.15em\left[}
%\newcommand{\rrbracket}{\right]\kern-0.15em\right]}

\renewcommand{\ge}{\geqslant}
\renewcommand{\geq}{\geqslant}
\renewcommand{\le}{\leqslant}
\renewcommand{\leq}{\leqslant}

%----- Fonctions usuelles -----
\newcommand{\ch}{\mathop{\mathrm{ch}}\nolimits}
\newcommand{\sh}{\mathop{\mathrm{sh}}\nolimits}
\renewcommand{\tanh}{\mathop{\mathrm{th}}\nolimits}
\newcommand{\cotan}{\mathop{\mathrm{cotan}}\nolimits}
\newcommand{\Arcsin}{\mathop{\mathrm{Arcsin}}\nolimits}
\newcommand{\Arccos}{\mathop{\mathrm{Arccos}}\nolimits}
\newcommand{\Arctan}{\mathop{\mathrm{Arctan}}\nolimits}
\newcommand{\Argsh}{\mathop{\mathrm{Argsh}}\nolimits}
\newcommand{\Argch}{\mathop{\mathrm{Argch}}\nolimits}
\newcommand{\Argth}{\mathop{\mathrm{Argth}}\nolimits}
\newcommand{\pgcd}{\mathop{\mathrm{pgcd}}\nolimits} 

\newcommand{\Card}{\mathop{\text{Card}}\nolimits}
\newcommand{\Ker}{\mathop{\text{Ker}}\nolimits}
\newcommand{\id}{\mathop{\text{id}}\nolimits}
\newcommand{\ii}{\mathrm{i}}
\newcommand{\dd}{\mathrm{d}}
\newcommand{\Vect}{\mathop{\text{Vect}}\nolimits}
\newcommand{\Mat}{\mathop{\mathrm{Mat}}\nolimits}
\newcommand{\rg}{\mathop{\text{rg}}\nolimits}
\newcommand{\tr}{\mathop{\text{tr}}\nolimits}
\newcommand{\ppcm}{\mathop{\text{ppcm}}\nolimits}

%----- Structure des exercices ------

\newtheoremstyle{styleexo}% name
{2ex}% Space above
{3ex}% Space below
{}% Body font
{}% Indent amount 1
{\bfseries} % Theorem head font
{}% Punctuation after theorem head
{\newline}% Space after theorem head 2
{}% Theorem head spec (can be left empty, meaning ‘normal’)

%\theoremstyle{styleexo}
\newtheorem{exo}{Exercice}
\newtheorem{ind}{Indications}
\newtheorem{cor}{Correction}


\newcommand{\exercice}[1]{} \newcommand{\finexercice}{}
%\newcommand{\exercice}[1]{{\tiny\texttt{#1}}\vspace{-2ex}} % pour afficher le numero absolu, l'auteur...
\newcommand{\enonce}{\begin{exo}} \newcommand{\finenonce}{\end{exo}}
\newcommand{\indication}{\begin{ind}} \newcommand{\finindication}{\end{ind}}
\newcommand{\correction}{\begin{cor}} \newcommand{\fincorrection}{\end{cor}}

\newcommand{\noindication}{\stepcounter{ind}}
\newcommand{\nocorrection}{\stepcounter{cor}}

\newcommand{\fiche}[1]{} \newcommand{\finfiche}{}
\newcommand{\titre}[1]{\centerline{\large \bf #1}}
\newcommand{\addcommand}[1]{}
\newcommand{\video}[1]{}

% Marge
\newcommand{\mymargin}[1]{\marginpar{{\small #1}}}



%----- Presentation ------
\setlength{\parindent}{0cm}

%\newcommand{\ExoSept}{\href{http://exo7.emath.fr}{\textbf{\textsf{Exo7}}}}

\definecolor{myred}{rgb}{0.93,0.26,0}
\definecolor{myorange}{rgb}{0.97,0.58,0}
\definecolor{myyellow}{rgb}{1,0.86,0}

\newcommand{\LogoExoSept}[1]{  % input : echelle
{\usefont{U}{cmss}{bx}{n}
\begin{tikzpicture}[scale=0.1*#1,transform shape]
  \fill[color=myorange] (0,0)--(4,0)--(4,-4)--(0,-4)--cycle;
  \fill[color=myred] (0,0)--(0,3)--(-3,3)--(-3,0)--cycle;
  \fill[color=myyellow] (4,0)--(7,4)--(3,7)--(0,3)--cycle;
  \node[scale=5] at (3.5,3.5) {Exo7};
\end{tikzpicture}}
}



\theoremstyle{definition}
%\newtheorem{proposition}{Proposition}
%\newtheorem{exemple}{Exemple}
%\newtheorem{theoreme}{Théorème}
\newtheorem{lemme}{Lemme}
\newtheorem{corollaire}{Corollaire}
%\newtheorem*{remarque*}{Remarque}
%\newtheorem*{miniexercice}{Mini-exercices}
%\newtheorem{definition}{Définition}




%definition d'un terme
\newcommand{\defi}[1]{{\color{myorange}\textbf{\emph{#1}}}}
\newcommand{\evidence}[1]{{\color{blue}\textbf{\emph{#1}}}}



 %----- Commandes divers ------

\newcommand{\codeinline}[1]{\texttt{#1}}

\newcommand{\codeinline}[1]{\texttt{#1}}


%%%%%%%%%%%%%%%%%%%%%%%%%%%%%%%%%%%%%%%%%%%%%%%%%%%%%%%%%%%%%
%%%%%%%%%%%%%%%%%%%%%%%%%%%%%%%%%%%%%%%%%%%%%%%%%%%%%%%%%%%%%


\begin{document}

\debuttexte

%%%%%%%%%%%%%%%%%%%%%%%%%%%%%%%%%%%%%%%%%%%%%%%%%%%%%%%%%%%
\diapo

\change

\change


Dans cette partie nous allons manipuler les nombres réels à travers le calcul de la constante $\gamma$ d'Euler.


C'est un nombre assez mystérieux car personne ne sait si $\gamma$ est un nombre rationnel ou irrationnel.

\change

Notre objectif est d'avoir le plus de décimales possibles après la virgule en un minimum d'étapes.

\change

Nous verrons ensuite comment les ordinateurs stockent les réels et les problèmes que cela engendre.


%%%%%%%%%%%%%%%%%%%%%%%%%%%%%%%%%%%%%%%%%%%%%%%%%%%%%%%%%%%
\diapo

Considérons la \defi{suite harmonique} :
$$H_n = \frac{1}{1}+\frac{1}{2}+\frac{1}{3}+\cdots + \frac{1}{n}$$

\change

et définissons
$$u_n = H_n - \ln n.$$

\change

Cette suite $(u_n)$ admet une limite finie lorsque $n\to+\infty$ : c'est la \defi{constante $\gamma$ d'Euler}.

\change
Voici votre travail :

nous avons une suite qui tend vers $\gamma$ cela permet d'en trouver une approximation.

Ensuite on considère une suite qui tend vers $\gamma$ mais qui s'en rapproche plus rapidemment,

cela permet d'obtenir plus facilement des chiffres exacts après la virgule.


%%%%%%%%%%%%%%%%%%%%%%%%%%%%%%%%%%%%%%%%%%%%%%%%%%%%%%%%%%%
\diapo

La première suite qui tend vers $\gamma$ ne pose aucun problème,

on calcule la suite harmonique, c'est la somme des $1/i$.

Pour cela on fait une boucle.

Et à la fin on retire le logarithme de $n$.


Vous remarquez que la somme est calculée à partir de la fin. 

 
$1/n + 1/(n-1) +$   jusqu'à $1/2 + 1/1$

 Nous expliquerons pourquoi en fin de leçon.
 
En effet range(n,0,-1) signifie que l'on part de $n$ et le $-1$ en troisième variable
signifie que l'on décrémente les indices.


\change


Evidemment aucun pb pour la seconde suite,

seul ce que l'on retire à la fin change.
 
Testez ces deux fonctions et comparer le nombre de décimales exactes obtenues en fonction de $n$.


%%%%%%%%%%%%%%%%%%%%%%%%%%%%%%%%%%%%%%%%%%%%%%%%%%%%%%%%%%%
\diapo

Il y a deux techniques pour obtenir plus de décimales : (i) pousser plus loin les itérations,
mais pour avoir $1000$ décimales de $\gamma$ les méthodes précédentes sont insuffisantes ; 
(ii) trouver une méthode encore plus efficace. C'est ce que nous allons voir avec
la méthode de Bessel modifiée.
Soit 
$$w_n = \frac{A_n}{B_n} - \ln n 
\quad \text{ avec } \quad
A_n =\sum_{k=1}^{E(\alpha n)} \left( \frac{n^k}{k!} \right)^2 H_k$$

les $H_k$ sont les termes de la suite harmonique.

$$
 \quad \text{ et } \quad 
B_n =\sum_{k=0}^{E(\alpha n)} \left( \frac{n^k}{k!} \right)^2 
$$


\change
où $\alpha = 3.59...$ est la solution de l'équation $\alpha(\ln \alpha - 1)=1$ et
$E(x)$ désigne la partie entière.


\change

Alors 
$$|w_n - \gamma| \le \frac{C}{e^{4n}}$$
où $C$ est une constante (non connue).

Cela signifie que $w_n$ tend très très vite vers $\gamma$.

Car $\frac{C}{e^{4n}}$ tend de façon exponentiellement vite vers $0$.

\change

A vous de programmer cette nouvelle méthode

et en utilisant le module "decimal" qui gère les réels avec une précision arbitraire
calculez les $1000$ premières décimales de $\gamma$.


%%%%%%%%%%%%%%%%%%%%%%%%%%%%%%%%%%%%%%%%%%%%%%%%%%%%%%%%%%%
\diapo

Pour obtenir $N$ décimales il suffit de résoudre l'inéquation $\frac{C}{e^{4n}} \le \frac{1}{10^{N}}$.

\change


On <<passe au log>> pour obtenir $n \ge \frac{N \ln(10)+\ln(C)}{4}$. On ne connaît pas $C$ mais 
ce n'est pas si important.

\change

Moralement pour une itération de plus on obtient (à peu près) une décimale 
de plus (c'est-à-dire un facteur $10$ sur la précision !).


\change

Voici le code qui ne présente pas de difficulté particulière,

je vous laisse l'étudier !


Pour obtenir plus de décimales que la précision standard de Python qui ne gère que $16$ chiffres après la virgule,

il vous faut utiliser le module "decimal" et adapter légèrement l'écriture de 
cet algorithme pour obtenir autant de décimales que vous souhaitez.



%%%%%%%%%%%%%%%%%%%%%%%%%%%%%%%%%%%%%%%%%%%%%%%%%%%%%%%%%%%
\diapo

Pour $n=800$ itérations dans l'algorithme précédent on obtient $1000$ décimales exactes de la 
constante d'Euler :

$\gamma = 0,577... $

Euler en 1781 avait calculer $16$ décimales ; à la main bien sûr !


  
%%%%%%%%%%%%%%%%%%%%%%%%%%%%%%%%%%%%%%%%%%%%%%%%%%%%%%%%%%%
\diapo

En mathématique un réel est un élément de $\Rr$ et son écriture décimale est souvent infinie après la virgule :
par exemple $\pi = 3,1415\ldots$
Mais bien sûr un ordinateur ne peut pas coder une infinité d'informations. Ce qui se rapproche d'un réel est 
un \defi{nombre flottant} dont l'écriture est la suivante :

\change

pour représenter le nombre 

$\pm 1,234\ldots \times 10^{\pm 123}$

\change

Un \defi{nombre flottant} est
composé d'une mantisse qui est un nombre décimal (positif ou négatif) appartenant à $[1,10[$.

Il n'y a bien sûr qu'un nombre fini de décimales.
En Python la mantisse à une précision de 
$16$ chiffres après la virgule. 

\change

L'exposant est un entier (lui aussi positif ou négatif).

\change

Cette réalité informatique fait que des erreurs de calculs peuvent 
apparaître même avec des opérations simples.


Pour voir un exemple de problème faites cette petite manip :


%%%%%%%%%%%%%%%%%%%%%%%%%%%%%%%%%%%%%%%%%%%%%%%%%%%%%%%%%%%
\diapo

Comme Python est très précis nous allons faire une routine qui permet 
de limiter drastiquement le nombre de chiffres et mettre en évidence les erreurs de calculs.

Tout d'abord à partir d'un réel retrouver l'exposant et sa mantisse.


Ensuite on souhaite limiter la précision à seulement $6$ chiffres,

vous devez écrire une fonction qui par exemple pour le réel $123,456789$ ne retient que $6$ chiffre en tout,

donc seulement $123,456$.


\change

Comme on l'a déjà vu dans une section précédente l'exposant $n$ se récupère 
à l'aide de la partie entière du logarithme en base $10$.

\change

Une fois que l'on a l'exposant, on décale du bon nombre de chiffres
afin d'avoir $6$ chiffres avant la virgule.
et on prend la partie entière $m$ qui est l'écriture de la mantisse (sans la virgule). 

\change

Pour obtenir notre nombre tronqué on redécale $m$ vers la droite.

\change

Voyons un exemple, $x=123,456789$, sa notation est $1,23...$ fois $10^2$.
L'exposant est $2$ que l'on trouve comme la partie entière de log de $x$.

Comme on ne veut conserver que $6$ chiffres alors on décale vers la gauche, ici de $3$ chiffres,
afin d'obtenir $6$ chiffres avant la virgule. 

La partie entière permet de faire disparaître les chiffres après la virgule.

On obtient donc un entier $m$ avec $6$ chiffres qui correspond à la mantisse $1,23456$.

On replace la virgule au bon endroit en décalant vers 
la droite pour obtenir notre nombre tronqué $123,456$.



%%%%%%%%%%%%%%%%%%%%%%%%%%%%%%%%%%%%%%%%%%%%%%%%%%%%%%%%%%%
\diapo

Nous allons étudier trois types d'erreurs qui apparaissent lors des calculs avec les nombres flottants.

Tester le premier type d'erreur qui s'appelle l'absorption.


[[pause]]

\change

Chacun des nombres $1234,56$ et $0,007$ est bien un nombre s'écrivant avec 
moins de $6$ décimales

\change

mais leur somme $1234,567$ a besoin d'une décimale de plus,

\change

l'ordinateur ne retient pas la $7$-ème décimale
et ainsi le résultat obtenu est $1234,56$. 

\change

Le $0,007$ n'apparaît pas dans le résultat : c'est le phénomène d'\defi{absorption}.


%%%%%%%%%%%%%%%%%%%%%%%%%%%%%%%%%%%%%%%%%%%%%%%%%%%%%%%%%%%
\diapo

Dans ce nouveau tp vous allez rencontrer un deuxième type de problème : 
l'élimination.

Utiliser la fonction "tronquer" que vous avez programmée auparavant.

Méfiez vous ! Réfléchissez bien à comment l'ordinateur 
stocke et fait les calculs !!


\change

Comme $x-y = 22,6555$ qui n'a que $6$ chiffres alors 
on peut penser que l'ordinateur va obtenir ce résultat.

\change

Il n'en est rien, l'ordinateur ne stocke pas $x$ 
mais \codeinline{tronquer(x)}, de même il ne stocke pas $y$ 
mais \codeinline{tronquer(y)}.

\change

Donc l'ordinateur commence par tronquer $x$ et $y$ avant de faire la différence.

Le calcul est donc 
\codeinline{tronquer(tronquer(x)-tronquer(y))}, 

\change

Ce qui correspond à $1234,87-1212,22=22,6500$.

\change

Quel est le problème ? C'est qu'ensuite l'utilisateur considère --à tort-- que le résultat est 
calculé avec une précision de $6$ chiffres. Donc on peut penser que le résultat est $22,6500$
mais les $2$ derniers chiffres sont une pure invention.

\change



C'est un phénomène d'\defi{élimination}. Lorsque l'on calcule la différence de deux nombres proches, le résultat
a en fait une précision moindre. Plus les nombres sont proches plus le phénomène peut être 
dramatique.

\change

J'en profite pour signaler une erreur d'interprétation fréquente : 

ne confondez pas la \evidence{précision} d'affichage
(par exemple : on calcule souvent $10$ chiffres après la virgule) 
avec l'\evidence{exactitude} du résultat 
(savez vous vraiment combien de décimales sont exactes ?).



%%%%%%%%%%%%%%%%%%%%%%%%%%%%%%%%%%%%%%%%%%%%%%%%%%%%%%%%%%%
\diapo

Enfin voici le dernier type de problème, certainement le plus 
troublant la première fois qu'on le rencontre.

Voici une liste d'instructions à suivre qui devraient vous laissez sceptique !


\change


L'explication de tout cela est simple mais néanmoins troublant.

Commençons par une analogie ; il est impossible de stocker dans un ordinateur l'écriture décimale
de $1/3$  car il y a une infinité de chiffres après la virgule.


Pour $0,1$ ou pour $0,2$ on peut croire que cela ne va pas poser de problème.

\change

Mais Python ne stocke pas les nombres en écriture décimale mais en écriture binaire.

Malheureusement comme c'était le cas pour $1/3$ en écriture décimale 
il ne peut pas stocker
en binaire la valeur exacte de $0,1$ ou $0,2$.

\change

Il stocke donc un nombre en écriture binaire qui se 
rapproche le plus de $0,1$ ou $0.,2$.

\change

Et lorsque on lui demande d'afficher
le nombre stocké, il retourne l'écriture décimale qui se rapproche 
le plus du nombre stocké, mais ce n'est plus
$0,2$, mais un nombre très très proche.




%%%%%%%%%%%%%%%%%%%%%%%%%%%%%%%%%%%%%%%%%%%%%%%%%%%%%%%%%%%
\diapo

Voyons une situation concrète où ces problèmes apparaissent.

Il s'agit simplement de calculer la somme de $1/i^2$

mais dans les questions 2 et 3 on se limite à une écriture à $6$
chiffres.




%%%%%%%%%%%%%%%%%%%%%%%%%%%%%%%%%%%%%%%%%%%%%%%%%%%%%%%%%%%
\diapo

[repartir de la diapo 11]

Je ne parle pas de la question 1 qui
ne pose aucun problème et utilise toute la précision de Python.

\change

Pour la deuxième question on doit maintenant, à chaque calcul, 
limiter notre précision à $6$ chiffres 
(ici $1$ avant la virgule et $5$ après).

Il faut donc tronquer pour la calcul de l'inverse de $i^2$,

puis tronquer pour la somme avec les termes précédents.


\change

Pour la troisième question on part des plus petits termes
et on termine par $1/1^2$


\change

En fait avec une précision maximale et $n$ très grand 
on doit obtenir une bonne approximation de $\frac{\pi^2}{6}$ 
qui est la limite de $S_n$.

\change

Pour  $n = 100\, 000$ ce premier algorithme [montrer] avec écriture tronquée,
et la somme dans l'ordre, retourne $1,64038$. Donc seulement deux chiffres exacts 
après la virgule.

Notez que faire grandir $n$ n'y changera rien, il bloque à $2$ décimales exactes 
après la virgule.

\change

Alors qu'avec ce second l'algorithme où la somme est toujours tronquer
mais effectuer en sens inverse on obtient un bien meilleur résultat 
$1,64490$. 

La raison est un phénomène d’absorption pour ce premier algorithme : 
on rajoute des termes très petits devant une somme qui vaut plus de $1$.


Alors que si l'on part des termes petits, on ajoute des termes petits 
à une somme petite, on garde donc un maximum 
de décimales valides avant de terminer par les plus hautes valeurs.


%%%%%%%%%%%%%%%%%%%%%%%%%%%%%%%%%%%%%%%%%%%%%%%%%%%%%%%%%%%
\diapo

Il vous reste encore du travail avec ces exercices !



\end{document}