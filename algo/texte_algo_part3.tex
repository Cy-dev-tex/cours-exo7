
%%%%%%%%%%%%%%%%%% PREAMBULE %%%%%%%%%%%%%%%%%%


\documentclass[12pt]{article}

\usepackage{amsfonts,amsmath,amssymb,amsthm}
\usepackage[utf8]{inputenc}
\usepackage[T1]{fontenc}
\usepackage[francais]{babel}


% packages
\usepackage{amsfonts,amsmath,amssymb,amsthm}
\usepackage[utf8]{inputenc}
\usepackage[T1]{fontenc}
%\usepackage{lmodern}

\usepackage[francais]{babel}
\usepackage{fancybox}
\usepackage{graphicx}

\usepackage{float}

%\usepackage[usenames, x11names]{xcolor}
\usepackage{tikz}
\usepackage{datetime}

\usepackage{mathptmx}
%\usepackage{fouriernc}
%\usepackage{newcent}
\usepackage[mathcal,mathbf]{euler}

%\usepackage{palatino}
%\usepackage{newcent}


% Commande spéciale prompteur

%\usepackage{mathptmx}
%\usepackage[mathcal,mathbf]{euler}
%\usepackage{mathpple,multido}

\usepackage[a4paper]{geometry}
\geometry{top=2cm, bottom=2cm, left=1cm, right=1cm, marginparsep=1cm}

\newcommand{\change}{{\color{red}\rule{\textwidth}{1mm}\\}}

\newcounter{mydiapo}

\newcommand{\diapo}{\newpage
\hfill {\normalsize  Diapo \themydiapo \quad \texttt{[\jobname]}} \\
\stepcounter{mydiapo}}


%%%%%%% COULEURS %%%%%%%%%%

% Pour blanc sur noir :
%\pagecolor[rgb]{0.5,0.5,0.5}
% \pagecolor[rgb]{0,0,0}
% \color[rgb]{1,1,1}



%\DeclareFixedFont{\myfont}{U}{cmss}{bx}{n}{18pt}
\newcommand{\debuttexte}{
%%%%%%%%%%%%% FONTES %%%%%%%%%%%%%
\renewcommand{\baselinestretch}{1.5}
\usefont{U}{cmss}{bx}{n}
\bfseries

% Taille normale : commenter le reste !
%Taille Arnaud
%\fontsize{19}{19}\selectfont

% Taille Barbara
%\fontsize{21}{22}\selectfont

%Taille François
%\fontsize{25}{30}\selectfont

%Taille Pascal
%\fontsize{25}{30}\selectfont

%Taille Laura
%\fontsize{30}{35}\selectfont


%\myfont
%\usefont{U}{cmss}{bx}{n}

%\Huge
%\addtolength{\parskip}{\baselineskip}
}


% \usepackage{hyperref}
% \hypersetup{colorlinks=true, linkcolor=blue, urlcolor=blue,
% pdftitle={Exo7 - Exercices de mathématiques}, pdfauthor={Exo7}}


%section
% \usepackage{sectsty}
% \allsectionsfont{\bf}
%\sectionfont{\color{Tomato3}\upshape\selectfont}
%\subsectionfont{\color{Tomato4}\upshape\selectfont}

%----- Ensembles : entiers, reels, complexes -----
\newcommand{\Nn}{\mathbb{N}} \newcommand{\N}{\mathbb{N}}
\newcommand{\Zz}{\mathbb{Z}} \newcommand{\Z}{\mathbb{Z}}
\newcommand{\Qq}{\mathbb{Q}} \newcommand{\Q}{\mathbb{Q}}
\newcommand{\Rr}{\mathbb{R}} \newcommand{\R}{\mathbb{R}}
\newcommand{\Cc}{\mathbb{C}} 
\newcommand{\Kk}{\mathbb{K}} \newcommand{\K}{\mathbb{K}}

%----- Modifications de symboles -----
\renewcommand{\epsilon}{\varepsilon}
\renewcommand{\Re}{\mathop{\text{Re}}\nolimits}
\renewcommand{\Im}{\mathop{\text{Im}}\nolimits}
%\newcommand{\llbracket}{\left[\kern-0.15em\left[}
%\newcommand{\rrbracket}{\right]\kern-0.15em\right]}

\renewcommand{\ge}{\geqslant}
\renewcommand{\geq}{\geqslant}
\renewcommand{\le}{\leqslant}
\renewcommand{\leq}{\leqslant}

%----- Fonctions usuelles -----
\newcommand{\ch}{\mathop{\mathrm{ch}}\nolimits}
\newcommand{\sh}{\mathop{\mathrm{sh}}\nolimits}
\renewcommand{\tanh}{\mathop{\mathrm{th}}\nolimits}
\newcommand{\cotan}{\mathop{\mathrm{cotan}}\nolimits}
\newcommand{\Arcsin}{\mathop{\mathrm{Arcsin}}\nolimits}
\newcommand{\Arccos}{\mathop{\mathrm{Arccos}}\nolimits}
\newcommand{\Arctan}{\mathop{\mathrm{Arctan}}\nolimits}
\newcommand{\Argsh}{\mathop{\mathrm{Argsh}}\nolimits}
\newcommand{\Argch}{\mathop{\mathrm{Argch}}\nolimits}
\newcommand{\Argth}{\mathop{\mathrm{Argth}}\nolimits}
\newcommand{\pgcd}{\mathop{\mathrm{pgcd}}\nolimits} 

\newcommand{\Card}{\mathop{\text{Card}}\nolimits}
\newcommand{\Ker}{\mathop{\text{Ker}}\nolimits}
\newcommand{\id}{\mathop{\text{id}}\nolimits}
\newcommand{\ii}{\mathrm{i}}
\newcommand{\dd}{\mathrm{d}}
\newcommand{\Vect}{\mathop{\text{Vect}}\nolimits}
\newcommand{\Mat}{\mathop{\mathrm{Mat}}\nolimits}
\newcommand{\rg}{\mathop{\text{rg}}\nolimits}
\newcommand{\tr}{\mathop{\text{tr}}\nolimits}
\newcommand{\ppcm}{\mathop{\text{ppcm}}\nolimits}

%----- Structure des exercices ------

\newtheoremstyle{styleexo}% name
{2ex}% Space above
{3ex}% Space below
{}% Body font
{}% Indent amount 1
{\bfseries} % Theorem head font
{}% Punctuation after theorem head
{\newline}% Space after theorem head 2
{}% Theorem head spec (can be left empty, meaning ‘normal’)

%\theoremstyle{styleexo}
\newtheorem{exo}{Exercice}
\newtheorem{ind}{Indications}
\newtheorem{cor}{Correction}


\newcommand{\exercice}[1]{} \newcommand{\finexercice}{}
%\newcommand{\exercice}[1]{{\tiny\texttt{#1}}\vspace{-2ex}} % pour afficher le numero absolu, l'auteur...
\newcommand{\enonce}{\begin{exo}} \newcommand{\finenonce}{\end{exo}}
\newcommand{\indication}{\begin{ind}} \newcommand{\finindication}{\end{ind}}
\newcommand{\correction}{\begin{cor}} \newcommand{\fincorrection}{\end{cor}}

\newcommand{\noindication}{\stepcounter{ind}}
\newcommand{\nocorrection}{\stepcounter{cor}}

\newcommand{\fiche}[1]{} \newcommand{\finfiche}{}
\newcommand{\titre}[1]{\centerline{\large \bf #1}}
\newcommand{\addcommand}[1]{}
\newcommand{\video}[1]{}

% Marge
\newcommand{\mymargin}[1]{\marginpar{{\small #1}}}



%----- Presentation ------
\setlength{\parindent}{0cm}

%\newcommand{\ExoSept}{\href{http://exo7.emath.fr}{\textbf{\textsf{Exo7}}}}

\definecolor{myred}{rgb}{0.93,0.26,0}
\definecolor{myorange}{rgb}{0.97,0.58,0}
\definecolor{myyellow}{rgb}{1,0.86,0}

\newcommand{\LogoExoSept}[1]{  % input : echelle
{\usefont{U}{cmss}{bx}{n}
\begin{tikzpicture}[scale=0.1*#1,transform shape]
  \fill[color=myorange] (0,0)--(4,0)--(4,-4)--(0,-4)--cycle;
  \fill[color=myred] (0,0)--(0,3)--(-3,3)--(-3,0)--cycle;
  \fill[color=myyellow] (4,0)--(7,4)--(3,7)--(0,3)--cycle;
  \node[scale=5] at (3.5,3.5) {Exo7};
\end{tikzpicture}}
}



\theoremstyle{definition}
%\newtheorem{proposition}{Proposition}
%\newtheorem{exemple}{Exemple}
%\newtheorem{theoreme}{Théorème}
\newtheorem{lemme}{Lemme}
\newtheorem{corollaire}{Corollaire}
%\newtheorem*{remarque*}{Remarque}
%\newtheorem*{miniexercice}{Mini-exercices}
%\newtheorem{definition}{Définition}




%definition d'un terme
\newcommand{\defi}[1]{{\color{myorange}\textbf{\emph{#1}}}}
\newcommand{\evidence}[1]{{\color{blue}\textbf{\emph{#1}}}}



 %----- Commandes divers ------

\newcommand{\codeinline}[1]{\texttt{#1}}

%%%%%%%%%%%%%%%%%%%%%%%%%%%%%%%%%%%%%%%%%%%%%%%%%%%%%%%%%%%%%
%%%%%%%%%%%%%%%%%%%%%%%%%%%%%%%%%%%%%%%%%%%%%%%%%%%%%%%%%%%%%


\begin{document}

\debuttexte

%%%%%%%%%%%%%%%%%%%%%%%%%%%%%%%%%%%%%%%%%%%%%%%%%%%%%%%%%%%
\diapo

\change

Le but de cette section est le calcul des sinus, cosinus, et tangente d'un angle par nous même,
avec une précision de $8$ chiffres après la virgule.


\change

Comme travail préparatoire nous aurons besoin de calculer quelques valeurs de Arctan

\change

Avant de passer au calcul de la tangente de n'importe quel angle

\change

On en déduira les sinus et cosinus


%%%%%%%%%%%%%%%%%%%%%%%%%%%%%%%%%%%%%%%%%%%%%%%%%%%%%%%%%%%
\diapo


Nous aurons besoin de calculer une fois pour toute 
$\Arctan (1)$, $\Arctan (1/10)$, $\Arctan (1/100)$,...

\change

Je vous rappelle que $\Arctan$ est la bijection réciproque de tangente 
sur l'intervalle $]-\frac\pi2,\frac\pi2[$

Donc l'angle $\theta_i = \Arctan (10^{-i})$ c'est l'angle qui vérifie $\tan \theta_i = 10^{-i}$.

\change


Nous allons utiliser la formule suivante
$\Arctan x = \sum_{k=0}^{+\infty} (-1)^k\frac{x^{2k+1}}{2k+1}$

Autrement dit $\Arctan x= x - \frac{x^3}{3}+ \frac{x^5}{5}-\frac{x^7}{7}+\cdots$


\change

Voici votre travail.

Tout d'abord que peut bien valoir $\Arctan 1$ ?

Puis avec les premiers termes de la formule calculez $\Arctan (10^{-i})$

Et en fait vous vous apercevrez qu'assez peu de termes suffisent.

%%%%%%%%%%%%%%%%%%%%%%%%%%%%%%%%%%%%%%%%%%%%%%%%%%%%%%%%%%%
\diapo


Voici comment programmer le calcul de l'arctangente.

\change

C'est la mise en oeuvre directe de cette formule.

Et le paramètre $n$ signifie que l'on ne va pas jusqu'à l'infini mais seulement jusqu'à $k=n$.

On part d'une somme valant zero puis par recurrence 
on ajoute ou on soustrait le terme $\frac{x^{2k+1}}{2k+1}$.

On teste si $k$ est pair par le reste modulo $2$.

Si $k$ est pair (par exemple ici $k=0$, ici $k=2$) alors on ajoute le terme $\frac{x^{2k+1}}{2k+1}$ .

Si $k$ est impair on le soustrait.

\change

Il faut maintenant justifier que s'arrêter à un certain rang suffit pour notre approximation.

Comme $\tan pi/4 = 1$ alors $\Arctan 1 = \pi/4$.

\change

Donc il nous reste à calculer des arctan avec $x \le 1/10$.

Mais si $x \le 1/10$ alors les termes $x^{2k+1}$ deviennent vite très petits.

\change

Et dès que $k\ge 4$ chaque terme ne contribue pas au $8$ 
première decimale de notre angle $\arctan x$. 

Il faudrait en fait montrer que la somme de tous ces termes ne contribue pas, 
je vous laisse l'écrire.

\change

On garde donc seulement les termes jusqu'à $k=3$.


Pour $x=1/10$, $1/100$ on obtient alors une bonne approximation de $\arctan x$.

\change

En constate que pour $10^{-4}$ et des valeurs encore plus petites
alors l'approximation $\arctan x \simeq x$ suffit à donner $8$ chiffres 
exacts après la virgule.

\change

A partir de cet algorithme on construit une liste "theta" contenant les 
$\theta_0$, $\theta_1$, $\theta_2$,...


%%%%%%%%%%%%%%%%%%%%%%%%%%%%%%%%%%%%%%%%%%%%%%%%%%%%%%%%%%%
\diapo


Le principe pour le calcul de la tangente est le suivant : on connaît un certain nombre d'angles avec leur tangente : les angles $\theta_i$ 

que l'on vient de calculer  avec par définition $\tan \theta_i = 10^{-i}$. 


Fixons un angle $a \in [0,\frac\pi2]$.


Partant du point $M_0 = (1,0)$, nous allons construire des points $M_1,M_2,\ldots,M_n$
jusqu'à ce que $M_n$ soit (à peu près) sur la demi-droite correspondant à l'angle $a$.

L'angle pour passer d'un point $M_k$ à $M_{k+1}$ est l'un des angles $\theta_i$.

Si $M_n$ a pour coordonnées $(x_n,y_n)$ alors ce que l'on cherche c'est $\tan a$ qui vaut presque
$\frac{y_n}{x_n}$.




%%%%%%%%%%%%%%%%%%%%%%%%%%%%%%%%%%%%%%%%%%%%%%%%%%%%%%%%%%%
\diapo

Détaillons une étape.

\change

On part d'un point $M$ de coordonnées $(x,y)$ 

\change

et on considère la rotation centrée à l'origine et d'angle $\theta$

Cette rotation envoie le point $M(x,y)$ sur le point $N(x',y')$ 

\change

Voici comme on calcule les coordonnées de $N$ en fonction de celle de $M$.


$$\begin{pmatrix} x' \\ y' \end{pmatrix}
= \begin{pmatrix} \cos \theta & - \sin \theta \\ \sin \theta & \cos \theta \end{pmatrix}
\begin{pmatrix} x \\y \end{pmatrix}$$


\change

Ce qui signifie exactement 

$$
\left\{ \begin{array}{rcl} 
        x' = x \cos \theta - y \sin \theta \\
        y' = x \sin \theta + y \cos \theta \\
        \end{array}
\right. $$


\change

Enfin le point qui sera important pour  nous est le point $M'$.

C'est le point de la demi-droite $[ON)$ tel que les droites $(OM)$ 
et $(MM')$ soient perpendiculaires en $M$.



%%%%%%%%%%%%%%%%%%%%%%%%%%%%%%%%%%%%%%%%%%%%%%%%%%%%%%%%%%%
\diapo

Voici votre travail :


tout d'abord une partie mathématiques :
il s'agit de trouver les coordonnées de $M'$ uniquement en fonction 
des coordonnées $(x,y)$ de $M$.

Tout va reposer sur le fait que la formule fait uniquement intervenir $x,y$ et 
$\tan \theta$.

La question suivante sert de travail préparatoire :
on a la liste d'angle fixés les $\theta_i$. Il s'agit d’écrire un petit algorithme
qui décompose un angle $a$ comme une somme des angles $\theta_i$. Bien sûr on souhaite faire ceci avec un minimum d'angles.


Enfin après tous ces préparatifs vous pourrez programmer la fonction tangente.
La décomposition en somme d'angle donne une suite de points $M_k$ dont on peut calculer les coordonnées.

Ce qui nous interesse c'est le point $M_n$ qui est situé sur la droite 
qui fait un angle (presque) $a$ avec l'horizontale.

La tangente de $a$ c'est coté opposé sur coté adjacent donc c'est presque $y_n$ divisé par $x_n$.


%%%%%%%%%%%%%%%%%%%%%%%%%%%%%%%%%%%%%%%%%%%%%%%%%%%%%%%%%%%
\diapo

On commence par la deuxième question !

Voici comment décomposer l'angle $a$ en une somme d'angle $\theta_i$.

Il y a donc deux boucles la première boucle tant que l'on n'a pas atteint la précision souhaitée.
Pour nous c'est obtenir $8$ chiffres exactes après la virgule donc on prend comme précision $10^{-9}$ par exemple.

On commence par retirer l'angle le plus grand possible, c'est l'objet de cette deuxième boucle.
On cherche l'angle 
$\theta_i$ le plus grand possible mais qui est quand même plus petit que $a$.

Une fois trouvé on retire $\theta_i$ à l'angle $a$, et on recommence avec notre nouvel angle $a$.



%%%%%%%%%%%%%%%%%%%%%%%%%%%%%%%%%%%%%%%%%%%%%%%%%%%%%%%%%%%
\diapo

Revenons à la première question et aux mathématiques avec le calcul des coordonnées de $M'$.

\change

Dans le triangle rectangle $OMM'$ on a $\cos \theta = \frac{OM}{OM'}$ 

\change

donc $OM' = \frac{OM}{\cos \theta}$.
 
\change 
 
 D'autre part comme la rotation d'angle $\theta$ conserve les distances alors $OM=ON$. 
 
\change
 
Notons $(x'',y'')$ les coordonnées de $M'$.

\change

$N$ a pour coordonnées $(x',y')$. 
L'égalité  $OM' = \frac{OM}{\cos \theta}$ nous dit que 
$x''= \frac{1}{\cos \theta} x'$ et 
    $y''= \frac{1}{\cos \theta} y'$. 


Mais $N$ est l'image de $M$ par la rotation d'angle $\theta$.

\change

On a vu comment comment calculer les coordonnées $x'$ 

\change
et de $y'$ en fonction de $x$ et $y$.


Mais on a en plus ici le facteur $1/\cos \theta$,

\change

ce qui fait que $x'' = x - y \tan \theta$

\change

Et $y'' = x \tan \theta + y$.

\change

Et termes de matrices cela signifie que le vecteurs 
$\begin{pmatrix} x'' \\ y'' \end{pmatrix}$

est le produit du vecteur $\begin{pmatrix} x \\y \end{pmatrix}$

par la matrice $\begin{pmatrix} 1 & - \tan \theta \\ \tan \theta & 1 \end{pmatrix}$


Tout l’intérêt de cette écriture et de cette transformation de $M$ à $M'$ c'est que cela ne fait
intervenir que la tangente de l'angle $\theta$. Pas de sinus, pas de cosinus.

%%%%%%%%%%%%%%%%%%%%%%%%%%%%%%%%%%%%%%%%%%%%%%%%%%%%%%%%%%%
\diapo

\change

Posons $x_0 = 1$, $y_0=0$ et $M_0=\begin{pmatrix}x_0\\y_0\end{pmatrix}$.

\change

Alors on définit par récurrence 
 $M_{k+1} = P(\theta_i) \cdot M_k$

\change

où $P(\theta)$ est la matrice $\begin{pmatrix} 1 & - \tan \theta \\ \tan \theta & 1 \end{pmatrix}$.


Partant de $M_0$ on itère donc notre transformation.

$M_0$ s'envoie sur $M_1$, puis $M_1$ s'envoie sur $M_2$, etc.


\change

Les $\theta_i$ que l'on utilise sont ceux apparaissant 
dans la décomposition de l'angle en somme de $\theta_i$,


que l'on a choisit de sorte que $\tan \theta_i = 10^{-i}$. 


\change

Ainsi si l'on passe d'un point $M_k$ à $M_{k+1}$ 
par un angle $\theta_i$ 

alors les coordonnées du point $M_{k+1}$ s'obtiennent simplement comme :


  $$ \left\{ \begin{array}{l} 
        x_{k+1} = x_k - y_k \cdot 10^{-i} \\
        y_{k+1} = x_k \cdot 10^{-i}+ y_k \\
        \end{array}
\right. $$   

où les $\tan \theta_i$ sont remplacés par $10^{-i}$.

\change

La valeur $\frac{y_n}{x_n}$ est la tangente de la somme des angles 
$\theta_i$, 

\change

donc une approximation de $\tan a$.
  
  
  Notez que chacune des itérations ne nécessite quasiment que des additions
  et peu se faire sans probleme à la main car une multiplication par $10^{-i}$ 
 c'est juste un décalage de la virgule vers la gauche de $i$ chiffres.
 
 C'est ce qui fait la facilité avec laquelle s'éxecute cet algorithme
 que l'on trouve dans les calculatrices.

%%%%%%%%%%%%%%%%%%%%%%%%%%%%%%%%%%%%%%%%%%%%%%%%%%%%%%%%%%%
\diapo

Voici le code pour le calcul de tangente :

$a$ est le paramètre de notre fonction dont on veut calculer la tangente.

On initialise la precision,
et on part du $M_0$ de coordonnées $(1,0)$.

\change

Ensuite on refait notre boucle qui décompose l'angle $a$
en somme des $\theta_i$.

\change

Désormais à chaque fois que l'on trouve qu'il faut ajouter un angle $\theta_i$ 

on calcule aussi les coordonnées du nouveau point (disons $M_{k+1}$) en fonction
des coordonnées du point précédents $M_k$.

\change

Maintenant on repart du point $M_{k+1}$ et on itère.

Notre boucle s'arrête lorsque l'on trouve un point $M_n$ 
qui forme un angle presque $a$ avec l'horizontale.

Ici $a$ est une variable qui change : à chaque fois on retire à $a$ les $\theta_i$, 
on s'arrête lorsque l'angle $a$ est inférieure à la précision.

\change

A la fin on retourne $y/x$ qui correspond à l'approximation de la tangente 
de notre angle $a$ initial.


%%%%%%%%%%%%%%%%%%%%%%%%%%%%%%%%%%%%%%%%%%%%%%%%%%%%%%%%%%%
\diapo


Une fois que l'on sait calculer la tangente d'un angle il n'est pas dur d'en déduire
le sinus et le cosinus.

A vous de jouer !

[pause]


\change

Voici les formules $\cos x = \frac{1}{\sqrt{1+\tan^2 x}}$

\change

$\sin x = \frac{\tan x}{\sqrt{1+\tan^2 x}}$

\change 

ces formules sont valables seulement pour $x$ est compris entre $0$ et $\frac \pi2$

\change


La preuve est simple. On sait que $\cos^2+\sin^2 x = 1$

\change

donc en divisant par $\cos^2 x$ on trouve 
$1+\tan^2 x = \frac{1}{\cos^2 x}$

\change

Comme $x$ est compris entre $0$ et $\frac \pi2$ alors
$\cos x$ est positif donc

on trouve $\cos x = \frac{1}{\sqrt{1+\tan^2 x}}$.

Donc une fois que l'on a calculé $\tan x$ on en déduit $\sin x$ et $\cos x$ par un calcul de racine carrée.



%%%%%%%%%%%%%%%%%%%%%%%%%%%%%%%%%%%%%%%%%%%%%%%%%%%%%%%%%%%
\diapo

Pour poursuivre votre travail voici quelques exercices.


\end{document}