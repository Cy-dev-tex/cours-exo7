
%%%%%%%%%%%%%%%%%% PREAMBULE %%%%%%%%%%%%%%%%%%


\documentclass[12pt]{article}

\usepackage{amsfonts,amsmath,amssymb,amsthm}
\usepackage[utf8]{inputenc}
\usepackage[T1]{fontenc}
\usepackage[francais]{babel}


% packages
\usepackage{amsfonts,amsmath,amssymb,amsthm}
\usepackage[utf8]{inputenc}
\usepackage[T1]{fontenc}
%\usepackage{lmodern}

\usepackage[francais]{babel}
\usepackage{fancybox}
\usepackage{graphicx}

\usepackage{float}

%\usepackage[usenames, x11names]{xcolor}
\usepackage{tikz}
\usepackage{datetime}

\usepackage{mathptmx}
%\usepackage{fouriernc}
%\usepackage{newcent}
\usepackage[mathcal,mathbf]{euler}

%\usepackage{palatino}
%\usepackage{newcent}


% Commande spéciale prompteur

%\usepackage{mathptmx}
%\usepackage[mathcal,mathbf]{euler}
%\usepackage{mathpple,multido}

\usepackage[a4paper]{geometry}
\geometry{top=2cm, bottom=2cm, left=1cm, right=1cm, marginparsep=1cm}

\newcommand{\change}{{\color{red}\rule{\textwidth}{1mm}\\}}

\newcounter{mydiapo}

\newcommand{\diapo}{\newpage
\hfill {\normalsize  Diapo \themydiapo \quad \texttt{[\jobname]}} \\
\stepcounter{mydiapo}}


%%%%%%% COULEURS %%%%%%%%%%

% Pour blanc sur noir :
%\pagecolor[rgb]{0.5,0.5,0.5}
% \pagecolor[rgb]{0,0,0}
% \color[rgb]{1,1,1}



%\DeclareFixedFont{\myfont}{U}{cmss}{bx}{n}{18pt}
\newcommand{\debuttexte}{
%%%%%%%%%%%%% FONTES %%%%%%%%%%%%%
\renewcommand{\baselinestretch}{1.5}
\usefont{U}{cmss}{bx}{n}
\bfseries

% Taille normale : commenter le reste !
%Taille Arnaud
%\fontsize{19}{19}\selectfont

% Taille Barbara
%\fontsize{21}{22}\selectfont

%Taille François
%\fontsize{25}{30}\selectfont

%Taille Pascal
%\fontsize{25}{30}\selectfont

%Taille Laura
%\fontsize{30}{35}\selectfont


%\myfont
%\usefont{U}{cmss}{bx}{n}

%\Huge
%\addtolength{\parskip}{\baselineskip}
}


% \usepackage{hyperref}
% \hypersetup{colorlinks=true, linkcolor=blue, urlcolor=blue,
% pdftitle={Exo7 - Exercices de mathématiques}, pdfauthor={Exo7}}


%section
% \usepackage{sectsty}
% \allsectionsfont{\bf}
%\sectionfont{\color{Tomato3}\upshape\selectfont}
%\subsectionfont{\color{Tomato4}\upshape\selectfont}

%----- Ensembles : entiers, reels, complexes -----
\newcommand{\Nn}{\mathbb{N}} \newcommand{\N}{\mathbb{N}}
\newcommand{\Zz}{\mathbb{Z}} \newcommand{\Z}{\mathbb{Z}}
\newcommand{\Qq}{\mathbb{Q}} \newcommand{\Q}{\mathbb{Q}}
\newcommand{\Rr}{\mathbb{R}} \newcommand{\R}{\mathbb{R}}
\newcommand{\Cc}{\mathbb{C}} 
\newcommand{\Kk}{\mathbb{K}} \newcommand{\K}{\mathbb{K}}

%----- Modifications de symboles -----
\renewcommand{\epsilon}{\varepsilon}
\renewcommand{\Re}{\mathop{\text{Re}}\nolimits}
\renewcommand{\Im}{\mathop{\text{Im}}\nolimits}
%\newcommand{\llbracket}{\left[\kern-0.15em\left[}
%\newcommand{\rrbracket}{\right]\kern-0.15em\right]}

\renewcommand{\ge}{\geqslant}
\renewcommand{\geq}{\geqslant}
\renewcommand{\le}{\leqslant}
\renewcommand{\leq}{\leqslant}

%----- Fonctions usuelles -----
\newcommand{\ch}{\mathop{\mathrm{ch}}\nolimits}
\newcommand{\sh}{\mathop{\mathrm{sh}}\nolimits}
\renewcommand{\tanh}{\mathop{\mathrm{th}}\nolimits}
\newcommand{\cotan}{\mathop{\mathrm{cotan}}\nolimits}
\newcommand{\Arcsin}{\mathop{\mathrm{Arcsin}}\nolimits}
\newcommand{\Arccos}{\mathop{\mathrm{Arccos}}\nolimits}
\newcommand{\Arctan}{\mathop{\mathrm{Arctan}}\nolimits}
\newcommand{\Argsh}{\mathop{\mathrm{Argsh}}\nolimits}
\newcommand{\Argch}{\mathop{\mathrm{Argch}}\nolimits}
\newcommand{\Argth}{\mathop{\mathrm{Argth}}\nolimits}
\newcommand{\pgcd}{\mathop{\mathrm{pgcd}}\nolimits} 

\newcommand{\Card}{\mathop{\text{Card}}\nolimits}
\newcommand{\Ker}{\mathop{\text{Ker}}\nolimits}
\newcommand{\id}{\mathop{\text{id}}\nolimits}
\newcommand{\ii}{\mathrm{i}}
\newcommand{\dd}{\mathrm{d}}
\newcommand{\Vect}{\mathop{\text{Vect}}\nolimits}
\newcommand{\Mat}{\mathop{\mathrm{Mat}}\nolimits}
\newcommand{\rg}{\mathop{\text{rg}}\nolimits}
\newcommand{\tr}{\mathop{\text{tr}}\nolimits}
\newcommand{\ppcm}{\mathop{\text{ppcm}}\nolimits}

%----- Structure des exercices ------

\newtheoremstyle{styleexo}% name
{2ex}% Space above
{3ex}% Space below
{}% Body font
{}% Indent amount 1
{\bfseries} % Theorem head font
{}% Punctuation after theorem head
{\newline}% Space after theorem head 2
{}% Theorem head spec (can be left empty, meaning ‘normal’)

%\theoremstyle{styleexo}
\newtheorem{exo}{Exercice}
\newtheorem{ind}{Indications}
\newtheorem{cor}{Correction}


\newcommand{\exercice}[1]{} \newcommand{\finexercice}{}
%\newcommand{\exercice}[1]{{\tiny\texttt{#1}}\vspace{-2ex}} % pour afficher le numero absolu, l'auteur...
\newcommand{\enonce}{\begin{exo}} \newcommand{\finenonce}{\end{exo}}
\newcommand{\indication}{\begin{ind}} \newcommand{\finindication}{\end{ind}}
\newcommand{\correction}{\begin{cor}} \newcommand{\fincorrection}{\end{cor}}

\newcommand{\noindication}{\stepcounter{ind}}
\newcommand{\nocorrection}{\stepcounter{cor}}

\newcommand{\fiche}[1]{} \newcommand{\finfiche}{}
\newcommand{\titre}[1]{\centerline{\large \bf #1}}
\newcommand{\addcommand}[1]{}
\newcommand{\video}[1]{}

% Marge
\newcommand{\mymargin}[1]{\marginpar{{\small #1}}}



%----- Presentation ------
\setlength{\parindent}{0cm}

%\newcommand{\ExoSept}{\href{http://exo7.emath.fr}{\textbf{\textsf{Exo7}}}}

\definecolor{myred}{rgb}{0.93,0.26,0}
\definecolor{myorange}{rgb}{0.97,0.58,0}
\definecolor{myyellow}{rgb}{1,0.86,0}

\newcommand{\LogoExoSept}[1]{  % input : echelle
{\usefont{U}{cmss}{bx}{n}
\begin{tikzpicture}[scale=0.1*#1,transform shape]
  \fill[color=myorange] (0,0)--(4,0)--(4,-4)--(0,-4)--cycle;
  \fill[color=myred] (0,0)--(0,3)--(-3,3)--(-3,0)--cycle;
  \fill[color=myyellow] (4,0)--(7,4)--(3,7)--(0,3)--cycle;
  \node[scale=5] at (3.5,3.5) {Exo7};
\end{tikzpicture}}
}



\theoremstyle{definition}
%\newtheorem{proposition}{Proposition}
%\newtheorem{exemple}{Exemple}
%\newtheorem{theoreme}{Théorème}
\newtheorem{lemme}{Lemme}
\newtheorem{corollaire}{Corollaire}
%\newtheorem*{remarque*}{Remarque}
%\newtheorem*{miniexercice}{Mini-exercices}
%\newtheorem{definition}{Définition}




%definition d'un terme
\newcommand{\defi}[1]{{\color{myorange}\textbf{\emph{#1}}}}
\newcommand{\evidence}[1]{{\color{blue}\textbf{\emph{#1}}}}



 %----- Commandes divers ------

\newcommand{\codeinline}[1]{\texttt{#1}}

\newcommand{\codeinline}[1]{\texttt{#1}}

%%%%%%%%%%%%%%%%%%%%%%%%%%%%%%%%%%%%%%%%%%%%%%%%%%%%%%%%%%%%%
%%%%%%%%%%%%%%%%%%%%%%%%%%%%%%%%%%%%%%%%%%%%%%%%%%%%%%%%%%%%%


\begin{document}

\debuttexte

%%%%%%%%%%%%%%%%%%%%%%%%%%%%%%%%%%%%%%%%%%%%%%%%%%%%%%%%%%%
\diapo

\change
Notre but dans cette série de leçons est double :

le but principal est de présenter des algorithmes en lien avec les mathématiques.

Le second but est d'en profiter pour apprendre les bases du langage Python,

c'est un langage moderne, de prise en main facile et pourtant très puissant.


\change

Dans cette première leçon on commence par vérifier si tout fonctionne.

\change

Puis on fait quelques calculs

\change

On introduira les boucles (\codeinline{pour} et \codeinline{tant que}),
le test \codeinline{si ... sinon ...} et les fonctions.


%%%%%%%%%%%%%%%%%%%%%%%%%%%%%%%%%%%%%%%%%%%%%%%%%%%%%%%%%%%
\diapo

Pour commencer testons si tout fonctionne!

Démarrez Python et faites les opération élémentaires suivantes.

Faites maintenant une pause sur cette vidéo et continuer lorsque 
vous avez fini ce tp.

[[temps de pause]]

\change

Lorsque l'on démarre Python,
on obtient une invite de commande symbolisée par
trois signes ``superieur''.

Pour définir une variable et y affecter une valeur on 
tape juste "a=3" puis "Entrée".

\change

Même chose avec "b=6".


\change

Et on définit la variable somme qui est a+b.

\change

Pour voir si Python sait calculer il suffit d'afficher la valeur de "somme".

\change

Cela affiche bien $9$ !


\change

Il est plus agréable d'afficher le résultat avec une petit phrase.

\change

\change


Et on fait la même chose pour le produit.


\change

\change


Dans la suite on omettra les symboles \codeinline{>>>}. 


%%%%%%%%%%%%%%%%%%%%%%%%%%%%%%%%%%%%%%%%%%%%%%%%%%%%%%%%%%%
\diapo

Maintenant que tout est en place allons y pour un deuxième exercice.

Dans ce  tp on s’intéresse à la somme des cubes $1^3+2^3+3^3+ \cdots + n^3$.

Il faut calculer la somme des cubes.

Mettre ce calcul sous la forme d'une fonction.

Et comparer le resultat avec le carré de la somme des entiers.

Programmez ces questions avant de passer à la suite.



%%%%%%%%%%%%%%%%%%%%%%%%%%%%%%%%%%%%%%%%%%%%%%%%%%%%%%%%%%%
\diapo


Voici ce que l'on fait pour calculer $S_n$ avec par exemple $n = 10$. 

On affecte d'abord la valeur $0$ à une variable \codeinline{somme}

cela correspond à l'initialisation $S_0=0$

Puis nous avons défini une boucle
avec l'instruction "pour" qui fait varier $i$ entre $1$ et $n$.

Nous calculons successivement $S_1$, $S_2$,\ldots 
en utilisant la formule de récurrence
$S_{i}= S_{i-1}+i^3$. 

Comme nous n'avons pas besoin de conserver toutes les valeurs des $S_i$ alors on garde
le même nom pour toutes les sommes, à chaque étape on affecte à 
\codeinline{somme} l'ancienne valeur de la somme plus $i^3$ :

\codeinline{somme = somme + i*i*i}.

Lorsque la boucle est finie, c'est-à-dire une fois que l'on a fait l'étape avec $i=n$,

alors on affiche le résultat.

\change

On nous demande aussi de définir une fonction qui calcule la somme des
$n$ premiers entiers.

\change

On peut faire une boucle comme précédemment mais on connait aussi une formule directe
pour cette somme.

\change

C'est très simple d'écrire une fonction avec Python.

Ici nous définissons une fonction que l'on appelle "somme-entiers",

elle prend en entrée la variable n.

Et retourne comme resultat $n(n+1)/2$.


Donc si on exécute "somme-entiers(10)" cela renvoie $10*11/2$ donc $55$.

Avec Python les espaces en début de ligne sont importants, ce sont eux qui déterminent le début 
et la fin d'un bloc d'instructions ou d'une fonction.

%%%%%%%%%%%%%%%%%%%%%%%%%%%%%%%%%%%%%%%%%%%%%%%%%%%%%%%%%%%
\diapo


Revenons sur ce que nous avons appris jusque là.

On affecte une valeur à une variable à l'aide du symbole "égal".


\change

On affiche un message avec la fonction \codeinline{print()}.
    
\change

On regroupe nos bouts de code dans des fonctions ce qui permet de les ré-utiliser.

Une \evidence{fonction} en informatique est similaire à une fonction mathématique, 
c'est un objet qui prend en entrée des variables 
(dites variables formelles ou variables muettes, pour nous c'était l'entier $n$)

\change

et retourne une valeur 
(un entier, une liste, une chaîne de caractères,... ici c'était 
$\frac{n(n+1)}{2}$).

\change

Pour parcourir tous les entiers de $1$ à $n$ il faut utiliser la commande 

\codeinline{range(1,n+1)}
C'est bien les entiers \textbf{strictement inférieurs à} $n+1$.

\change

La raison est que l'appel le plus fréquent est la commande 
\codeinline{range(n)} qui elle 
désigne les entiers de $0$ à $n-1$ mais qui contient bien $n$ éléments.
 
\change

 Enfin en Python  (contrairement aux autres langages) 
 c'est l'indentation (les espaces en début de chaque ligne)
 qui détermine les blocs d'instructions. 
  
  
%%%%%%%%%%%%%%%%%%%%%%%%%%%%%%%%%%%%%%%%%%%%%%%%%%%%%%%%%%%
\diapo

Nous reformulons notre calcul de la somme des cubes et l'insérons dans une fonction.

En entrée on a la variable n.

Nous effectuons la même boucle qu'auparavant.

Et on renvoie comme résultat la somme obtenue.

Notez le premier décalage pour signifier que toutes 
ses instructions sont des instructions de la fonction.

Et pour les sous-instructions de la boucle on redécale 
la ou les lignes correspondantes.

Il faut donc être très soigneux dans la présentation du code.

Notez que pour le $i$ au cube on utilise ici l'instruction $i * * 3$.


\change

Maintenant on souhaite comparer la somme des cubes avec le carré de la somme des entiers.

Par exemple choisissons $n=12$.

On effectue le test suivant à l'aide de l'instruction "si ... alors", "sinon alors".

Ici on teste si la somme des cubes est égale la somme des entiers au carré.

Si c'est vrai  alors on affiche le message "oui c'est vrai",

si on n'a pas égalité alors on affiche que "c'est faux".

Je vous laisse tester différentes valeurs de $n$ et vérifier ce qui s'affiche !



%%%%%%%%%%%%%%%%%%%%%%%%%%%%%%%%%%%%%%%%%%%%%%%%%%%%%%%%%%%
\diapo

Encore un petit résumé :

Les puissances se calculent aussi avec \codeinline{**} : $5^2$ s'écrit \codeinline{5*5} ou \codeinline{5**2},
  $5^3$ s'écrit \codeinline{5*5*5} ou \codeinline{5**3},...

\change

Nous avons défini une \evidence{boucle}
avec l'instruction \evidence{\codeinline{for}} qui fait varier un indice 

ici $i$ variait de  $1$ et $n$.

\change

Le test "si alors", "sinon alors" exécute le premier 
  bloc d'instructions si la condition est vraie ; 
   si la condition est fausse cela exécute l'autre bloc.
  
\change

Voic des exemples de conditions mathématiques :
  \begin{itemize}
     \item $a<b$,
     \item $a \le b$,
     \item le test d'égalité : $a=b$,
     \item le test de différence : $a \neq b$.
  \end{itemize}   
  
  
\change

Attention ! Il est important de comprendre que \evidence{\codeinline{a==b}} 
vaut soit vraie ou faux (on compare $a$ et $b$)

\change

  alors qu'avec \evidence{\codeinline{a=b}} on affecte dans $a$ la valeur de $b$. 
  
  

%%%%%%%%%%%%%%%%%%%%%%%%%%%%%%%%%%%%%%%%%%%%%%%%%%%%%%%%%%%
\diapo


Nous allons voir qu'il est possible de calculer les premières 
décimales de $\pi$ par la méthode de Monte-Carlo, 
c'est à dire avec l'aide du hasard.
On considère le carré de coté $1$, le cercle de rayon $1$ 
centré à l'origine, d'équation $x^2+y^2=1$,
et la portion de disque dans le carré.

A l'aide de ces questions vous allez calculer les premières 
décimales de $\pi$.

Je vous laisse lire et travailler !

%%%%%%%%%%%%%%%%%%%%%%%%%%%%%%%%%%%%%%%%%%%%%%%%%%%%%%%%%%%
\diapo

[[on repart du tp]]
Un petit calcul prouve que l'aire de la portion de disque est 
$\frac{\pi}{4}$, l'aire du carré est $1$. 
  Donc la probabilité de tomber dans le disque est $\frac{\pi}{4}$.

[[on passe au code]]
  
\change

Découvrons le code :

on importe d'abord le module qui gère les tirages de nombres au hasard.

\change

On définit deux variables, l'une qui compte le nb de tir.

Et la deuxième qui compte seulement le nb de tir au but, qui arrivent dans la portion de disque.

Au depart elles valent toutes deux $0$.

\change

Puis voici la boucle, 

cette boucle "tant que" est répétée tant que la condition est vraie,

c-a-d ici tant que "Tir < 1000".


On commence par incrémenter le numéro du tir.

A chaque itération on tire une abscisse $x$ au hasard. 

Et une ordonnée $y$.

On utilise une fonction \codeinline{random()} du module "random" 
qui renvoie au hasard un nombre réel de l'intervalle $[0,1[$.


On a donc obtenu un point de coordonnées $(x,y)$ dans le carré.

On teste si on est ou pas dans la portion de disque 
à l'aide de l'inégalité $x^2+y^2 \le 1$.

Je vous rappelle que $x^2+y^2=1$ est l'équation de notre cercle.
 
Si on est dans le cercle alors on incrémente 
le compteur qui compte le nb de tir dans le disque.


\change

On continue 
  jusqu'à ce que l'on atteigne $999$. Pour \codeinline{Tir}$=1000$ la condition
  n'est plus vraie et le bloc d'instructions du \codeinline{while} n'est pas exécuté. 
  
  On passe aux instructions suivantes
  pour afficher le résultat.
  
  La probabilité d'atteindre le disque est approché par le nombre de tir dans le disque divisé par le nombre de tir total.
  
  Comme la probabilité est $\pi/4$ on affiche $4*$ la probabilité pour obtenir une estimation de $\pi$.

Cette méthode n'est pas très efficace, 
il faut énormément de tirs pour obtenir les deux premières décimales de $\pi$.
  



%%%%%%%%%%%%%%%%%%%%%%%%%%%%%%%%%%%%%%%%%%%%%%%%%%%%%%%%%%%
\diapo
  
 Voyons ce que nous avons appris de nouveau.
 
 Nous avons découvert un nouveau type de boucle : 
 la boucle est \evidence{\codeinline{while condition: ...}} Tant que la condition est vérifiée 
  les instructions de la boucle sont exécutées.
 
 
Une fois que la condition n'est plus vérifiée on passe aux instructions d'après.

  \change
  
 La fonction \codeinline{random()} renvoie un nombre réel au hasard 
 de l'intervalle $[0,1[$.   
 
 Bien sûr à chaque appel de la fonction \codeinline{random()} 
 le nombre obtenu est différent !

  \change
  
Cette fonction n'est pas connue par défaut de Python, 
il faut lui indiquer le nom du \evidence{module} où elle se trouve.


  En début de fichier on ajoute \codeinline{import random} pour le module 
  qui gère les tirages au hasard. 
  
  Et pour indiquer 
  qu'une fonction vient d'un module il faut l’appeler par \codeinline{module.fonction()} donc ici \codeinline{random.random()}
  (module et fonction portent ici le même nom !).
  

  \change
  
Lorsque qu'une ligne contient un dièse, 
tout ce qui suit est ignoré. Cela permet
  d'insérer des commentaires, ce qui est essentiel pour relire le code.
  

%%%%%%%%%%%%%%%%%%%%%%%%%%%%%%%%%%%%%%%%%%%%%%%%%%%%%%%%%%%
\diapo

J’espère que vous êtes maintenant motivé pour vous lancer dans la programmation.

La pratique est évidemment indispensable, il faut surmonter les difficultés techniques,

les erreurs de syntaxe,...

Vous trouverez de l'aide et des tutoriels partout sur internet.

Vous pourrez ainsi faire des petits programmes pour répondre à ces questions.


Ce travail en vaut la peine car une fois que vous connaissez vraiment 
un langage il n'est pas très difficile de programmer dans un autre.


\end{document}