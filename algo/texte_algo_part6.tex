
%%%%%%%%%%%%%%%%%% PREAMBULE %%%%%%%%%%%%%%%%%%


\documentclass[12pt]{article}

\usepackage{amsfonts,amsmath,amssymb,amsthm}
\usepackage[utf8]{inputenc}
\usepackage[T1]{fontenc}
\usepackage[francais]{babel}


% packages
\usepackage{amsfonts,amsmath,amssymb,amsthm}
\usepackage[utf8]{inputenc}
\usepackage[T1]{fontenc}
%\usepackage{lmodern}

\usepackage[francais]{babel}
\usepackage{fancybox}
\usepackage{graphicx}

\usepackage{float}

%\usepackage[usenames, x11names]{xcolor}
\usepackage{tikz}
\usepackage{datetime}

\usepackage{mathptmx}
%\usepackage{fouriernc}
%\usepackage{newcent}
\usepackage[mathcal,mathbf]{euler}

%\usepackage{palatino}
%\usepackage{newcent}


% Commande spéciale prompteur

%\usepackage{mathptmx}
%\usepackage[mathcal,mathbf]{euler}
%\usepackage{mathpple,multido}

\usepackage[a4paper]{geometry}
\geometry{top=2cm, bottom=2cm, left=1cm, right=1cm, marginparsep=1cm}

\newcommand{\change}{{\color{red}\rule{\textwidth}{1mm}\\}}

\newcounter{mydiapo}

\newcommand{\diapo}{\newpage
\hfill {\normalsize  Diapo \themydiapo \quad \texttt{[\jobname]}} \\
\stepcounter{mydiapo}}


%%%%%%% COULEURS %%%%%%%%%%

% Pour blanc sur noir :
%\pagecolor[rgb]{0.5,0.5,0.5}
% \pagecolor[rgb]{0,0,0}
% \color[rgb]{1,1,1}



%\DeclareFixedFont{\myfont}{U}{cmss}{bx}{n}{18pt}
\newcommand{\debuttexte}{
%%%%%%%%%%%%% FONTES %%%%%%%%%%%%%
\renewcommand{\baselinestretch}{1.5}
\usefont{U}{cmss}{bx}{n}
\bfseries

% Taille normale : commenter le reste !
%Taille Arnaud
%\fontsize{19}{19}\selectfont

% Taille Barbara
%\fontsize{21}{22}\selectfont

%Taille François
%\fontsize{25}{30}\selectfont

%Taille Pascal
%\fontsize{25}{30}\selectfont

%Taille Laura
%\fontsize{30}{35}\selectfont


%\myfont
%\usefont{U}{cmss}{bx}{n}

%\Huge
%\addtolength{\parskip}{\baselineskip}
}


% \usepackage{hyperref}
% \hypersetup{colorlinks=true, linkcolor=blue, urlcolor=blue,
% pdftitle={Exo7 - Exercices de mathématiques}, pdfauthor={Exo7}}


%section
% \usepackage{sectsty}
% \allsectionsfont{\bf}
%\sectionfont{\color{Tomato3}\upshape\selectfont}
%\subsectionfont{\color{Tomato4}\upshape\selectfont}

%----- Ensembles : entiers, reels, complexes -----
\newcommand{\Nn}{\mathbb{N}} \newcommand{\N}{\mathbb{N}}
\newcommand{\Zz}{\mathbb{Z}} \newcommand{\Z}{\mathbb{Z}}
\newcommand{\Qq}{\mathbb{Q}} \newcommand{\Q}{\mathbb{Q}}
\newcommand{\Rr}{\mathbb{R}} \newcommand{\R}{\mathbb{R}}
\newcommand{\Cc}{\mathbb{C}} 
\newcommand{\Kk}{\mathbb{K}} \newcommand{\K}{\mathbb{K}}

%----- Modifications de symboles -----
\renewcommand{\epsilon}{\varepsilon}
\renewcommand{\Re}{\mathop{\text{Re}}\nolimits}
\renewcommand{\Im}{\mathop{\text{Im}}\nolimits}
%\newcommand{\llbracket}{\left[\kern-0.15em\left[}
%\newcommand{\rrbracket}{\right]\kern-0.15em\right]}

\renewcommand{\ge}{\geqslant}
\renewcommand{\geq}{\geqslant}
\renewcommand{\le}{\leqslant}
\renewcommand{\leq}{\leqslant}

%----- Fonctions usuelles -----
\newcommand{\ch}{\mathop{\mathrm{ch}}\nolimits}
\newcommand{\sh}{\mathop{\mathrm{sh}}\nolimits}
\renewcommand{\tanh}{\mathop{\mathrm{th}}\nolimits}
\newcommand{\cotan}{\mathop{\mathrm{cotan}}\nolimits}
\newcommand{\Arcsin}{\mathop{\mathrm{Arcsin}}\nolimits}
\newcommand{\Arccos}{\mathop{\mathrm{Arccos}}\nolimits}
\newcommand{\Arctan}{\mathop{\mathrm{Arctan}}\nolimits}
\newcommand{\Argsh}{\mathop{\mathrm{Argsh}}\nolimits}
\newcommand{\Argch}{\mathop{\mathrm{Argch}}\nolimits}
\newcommand{\Argth}{\mathop{\mathrm{Argth}}\nolimits}
\newcommand{\pgcd}{\mathop{\mathrm{pgcd}}\nolimits} 

\newcommand{\Card}{\mathop{\text{Card}}\nolimits}
\newcommand{\Ker}{\mathop{\text{Ker}}\nolimits}
\newcommand{\id}{\mathop{\text{id}}\nolimits}
\newcommand{\ii}{\mathrm{i}}
\newcommand{\dd}{\mathrm{d}}
\newcommand{\Vect}{\mathop{\text{Vect}}\nolimits}
\newcommand{\Mat}{\mathop{\mathrm{Mat}}\nolimits}
\newcommand{\rg}{\mathop{\text{rg}}\nolimits}
\newcommand{\tr}{\mathop{\text{tr}}\nolimits}
\newcommand{\ppcm}{\mathop{\text{ppcm}}\nolimits}

%----- Structure des exercices ------

\newtheoremstyle{styleexo}% name
{2ex}% Space above
{3ex}% Space below
{}% Body font
{}% Indent amount 1
{\bfseries} % Theorem head font
{}% Punctuation after theorem head
{\newline}% Space after theorem head 2
{}% Theorem head spec (can be left empty, meaning ‘normal’)

%\theoremstyle{styleexo}
\newtheorem{exo}{Exercice}
\newtheorem{ind}{Indications}
\newtheorem{cor}{Correction}


\newcommand{\exercice}[1]{} \newcommand{\finexercice}{}
%\newcommand{\exercice}[1]{{\tiny\texttt{#1}}\vspace{-2ex}} % pour afficher le numero absolu, l'auteur...
\newcommand{\enonce}{\begin{exo}} \newcommand{\finenonce}{\end{exo}}
\newcommand{\indication}{\begin{ind}} \newcommand{\finindication}{\end{ind}}
\newcommand{\correction}{\begin{cor}} \newcommand{\fincorrection}{\end{cor}}

\newcommand{\noindication}{\stepcounter{ind}}
\newcommand{\nocorrection}{\stepcounter{cor}}

\newcommand{\fiche}[1]{} \newcommand{\finfiche}{}
\newcommand{\titre}[1]{\centerline{\large \bf #1}}
\newcommand{\addcommand}[1]{}
\newcommand{\video}[1]{}

% Marge
\newcommand{\mymargin}[1]{\marginpar{{\small #1}}}



%----- Presentation ------
\setlength{\parindent}{0cm}

%\newcommand{\ExoSept}{\href{http://exo7.emath.fr}{\textbf{\textsf{Exo7}}}}

\definecolor{myred}{rgb}{0.93,0.26,0}
\definecolor{myorange}{rgb}{0.97,0.58,0}
\definecolor{myyellow}{rgb}{1,0.86,0}

\newcommand{\LogoExoSept}[1]{  % input : echelle
{\usefont{U}{cmss}{bx}{n}
\begin{tikzpicture}[scale=0.1*#1,transform shape]
  \fill[color=myorange] (0,0)--(4,0)--(4,-4)--(0,-4)--cycle;
  \fill[color=myred] (0,0)--(0,3)--(-3,3)--(-3,0)--cycle;
  \fill[color=myyellow] (4,0)--(7,4)--(3,7)--(0,3)--cycle;
  \node[scale=5] at (3.5,3.5) {Exo7};
\end{tikzpicture}}
}



\theoremstyle{definition}
%\newtheorem{proposition}{Proposition}
%\newtheorem{exemple}{Exemple}
%\newtheorem{theoreme}{Théorème}
\newtheorem{lemme}{Lemme}
\newtheorem{corollaire}{Corollaire}
%\newtheorem*{remarque*}{Remarque}
%\newtheorem*{miniexercice}{Mini-exercices}
%\newtheorem{definition}{Définition}




%definition d'un terme
\newcommand{\defi}[1]{{\color{myorange}\textbf{\emph{#1}}}}
\newcommand{\evidence}[1]{{\color{blue}\textbf{\emph{#1}}}}



 %----- Commandes divers ------

\newcommand{\codeinline}[1]{\texttt{#1}}

%%%%%%%%%%%%%%%%%%%%%%%%%%%%%%%%%%%%%%%%%%%%%%%%%%%%%%%%%%%%%
%%%%%%%%%%%%%%%%%%%%%%%%%%%%%%%%%%%%%%%%%%%%%%%%%%%%%%%%%%%%%


\begin{document}

\debuttexte

%%%%%%%%%%%%%%%%%%%%%%%%%%%%%%%%%%%%%%%%%%%%%%%%%%%%%%%%%%%
\diapo

\change

Dans cette leçon nous allons discuter de la notion de complexité des algorithmes.

\change


On va tout d'abord se poser la question : Qu'est-ce qu'un algorithme ?


\change


Nous allons ensuite étudier la complexité des algorithmes à travers l'exemple 
de l'addition et de la multiplication  des polynômes.


\change

Nous terminons avec un algorithme remarquable pour multiplier rapidement deux polynômes.

%%%%%%%%%%%%%%%%%%%%%%%%%%%%%%%%%%%%%%%%%%%%%%%%%%%%%%%%%%%
\diapo


Mais commençons par une question toute simple : Qu'est ce qu'un algorithme ?


\change

La réponse n'est pas si aisée.

Un algorithme est une succession d'instructions qui renvoie un résultat. 

\change

Pour être vraiment un algorithme on doit 
justifier que le résultat retourné est \evidence{exact} 
(le programme fait bien ce que l'on souhaite)

\change



et ceci en un \evidence{nombre fini d'étapes} (cela renvoie le résultat en temps fini).

\change


Maintenant certains algorithmes peuvent être plus rapides que d'autres. 


\change


C'est souvent le temps de calcul qui est le
principal critère

\change



Pour certains algorithmes la vitesse d’exécution n'est pas le seul paramètre mais aussi la taille 
de la mémoire occupée.


\change

Le problème avec le temps de calculs c'est qu'il va dépendre 
du langage et de la machine utilisée. 

\change

Il existe une manière plus mathématique de faire : 
la \defi{complexité} d'un algorithme


\change

c'est le nombre d'opérations élémentaires à effectuer.

\change

Ces opérations peuvent être le nombre d'opérations au niveau du processeur, 

mais pour nous ce sera le nombre d'additions ou le nombre de multiplications
à effectuer.



%%%%%%%%%%%%%%%%%%%%%%%%%%%%%%%%%%%%%%%%%%%%%%%%%%%%%%%%%%%
\diapo


Voici votre premier travail sur les polynômes,

tout d'abord on représente le polynômes $a_0+a_1X+\cdots + a_n X^n$ 
sous la forme d'une liste $[a_0,a_1,\ldots,a_n]$.


Votre travail consiste ensuite à écrire une fonction qui calcule la somme 
de deux polynômes.

Ensuite vous devez calculer la complexité de cet algorithme, c'est à dire combien d'additions
de coefficients sont nécessaires pour additionner deux polynômes de degré $\le n$.

Vous devez faire la même chose pour la multiplication : écrire un algorithme et calculer la complexité :
cette fois il s'agit surtout de compter combien de multiplications  on effectue sur les coefficients.

Idem pour la division.



%%%%%%%%%%%%%%%%%%%%%%%%%%%%%%%%%%%%%%%%%%%%%%%%%%%%%%%%%%%
\diapo

Pour la somme la seule difficulté est de gérer les indices, en particulier on ne peut
  appeler un élément d'une liste en dehors des indices où elle est définie.
  
  Voici le code dans le cas plus simple où $\deg A = \deg B$ :  

  il s'agit juste du calcul des coefficients $a_i+b_i$.
  
  \change 
 
 
  Une bonne idée consiste à commencer par définir une fonction "degre(poly)"
  qui renvoie le degré du polynôme (attention aux $0$ non significatifs).


\change

 Calculons la complexité de l'algorithme de la somme,
 
 \change
 
 on suppose que $\deg A$ et $\deg B$ sont $ \le n$ 
 
\change

 il faut faire l'addition des coefficients $a_i+b_i$,
 
\change

 ceci  pour $i$ variant de $0$ à $n$ 
 
 
\change

On a donc à effectuer $n+1$ additions (sur les entiers ou sur les réels)

et donc la complexité est de $n+1$ additions.

La complexité est ici une fonction qui dépend linéairement du degré des polynômes à additionner.


  

%%%%%%%%%%%%%%%%%%%%%%%%%%%%%%%%%%%%%%%%%%%%%%%%%%%%%%%%%%%
\diapo

Pour le produit il faut se rappeler que si on note 

$A(X) = \sum a_i X^i$,

et $B(X) = \sum b_j X^j$ 

\change

Alors le produit $C = A \times B$ s'écrit $= \sum_{k=0}^{m+n} c_k X^k$

\change

 le $k$-ème coefficient de $C$ est $c_k = \sum a_i \times b_j$ pour tous les indices $i$ et $j$ 
 vérifiant $i+j=k$.
 
\change

Voici donc l'algorithme de la multiplication de deux polynômes.


Il y a deux boucles :

une première boucle pour faire varier l'indice $k$ du coefficients $c_k$.

Et une deuxième boucle pour mettre en oeuvre la formule du calcul de chaque $c_k$.


Dans la pratique on fait attention de ne pas accéder à des coefficients qui n'ont pas été définis.



%%%%%%%%%%%%%%%%%%%%%%%%%%%%%%%%%%%%%%%%%%%%%%%%%%%%%%%%%%%
\diapo

\change

 Pour la complexité on commence par compter le nombre de multiplications (dans $\Zz$ ou $\Rr$).
 
\change

 Notons $m = \deg A$ et $n = \deg B$.
 
\change

Chaque coefficient de $A$ va être multiplier une et une seule fois par chaque coefficient de $B$.

 Ainsi il faut multiplier les $m+1$ coefficients de $A$ par les $n+1$ coefficients de $B$ :
 
\change

 il y a donc $(m+1)(n+1)$ multiplications.
 
\change

  
  Comptons maintenant les additions : 
  
 
\change

  les coefficients de $A\times B$ sont : 
  $c_0 = a_0 b_0$, $c_1 = a_0b_1+a_1b_0$, $c_2=a_2b_0+a_1b_1+a_2b_0$,...
   
   
\change


  Nous utilisons l'astuce suivante : nous savons que le produit $A\times B$ est de degré $m+n$
  donc a au plus $m+n+1$ coefficients.
  
\change

  chaque addition regroupe deux termes, 
  
\change
  
Et partant de $(m+1)(n+1)$ produits, 
  
nous devons arriver à $m+n+1$ coefficients.

 \change 
 
  Il y a donc $(m+1)(n+1)-(m+n+1) = mn$ additions.
  
La complexité de l'algorithme de la multiplication de deux polynômes de degré $m$ et $n$
est donc de $(m+1) \times (n+1)$ multiplication de coefficients

et de $m \times n$ additions de coefficients.

\change

Par exemple si $A$ et $B$ sont tous les deux de degré $n$ alors le nombre de multiplications sur les coefficients
vaut $(n+1)^2$ : cela dépend cette fois ci du *carré* des degrés des polynômes.

%%%%%%%%%%%%%%%%%%%%%%%%%%%%%%%%%%%%%%%%%%%%%%%%%%%%%%%%%%%
\diapo

Je vous rappelle d'abord sur un exemple comment on calcule la division euclidienne de deux polynômes.

On souhaite diviser $A$ par $B$.

\change

On pose une division de polynôme comme une division d'entier.

Ici on va obtenir le quotient et ici le reste.

\change

On les obtient ainsi : 

on cherche quel monôme $P_1$ fait diminuer le degré de $A-P_1B$, 

c'est $2X^2$ 

  (le coefficient $2$ est le coefficient dominant de $A$
  
  et le degré $X^2$ correspond à degré de $A$ moins degré de $B$.)
  
  
  On calcule $2X^2$ fois $B$
  
\change


Et on le soustrait à $A$.

On obtient un reste intermédiaire $R_1=X^3-4X^2+3X-1$.

et un quotient intermédiaire $Q_1 = 2X^2$.

\change

 on recommence avec $R_1$ diviser par $B$,
 
 Il faut multiplier $B$ par $X$ pour faire baisser le degré de $R_1$.
 

  
 
\change

  On pose ensuite  $R_2 = R_1-P_2B$ avec $P_2 = X$, $Q_2= Q_1+P_2$,... 
  
  
  Le monôme qui convient est $-3$
  (le coefficient $-3$ est le coefficient dominant de $R_2$
  
  et le degré $X$ est ici $0$ et correspond à degré de $R_2$ moins degré de $B$.) 
  
  
  Et on itérerait ainsi les calculs.
  
\change

Quand est-ce que l'on s'arrête ?

C'est terminé lorsque le degré du reste est strictement plus petit que le degré de $B$.

Comme le degré des restes intermédiaires diminue à chaque étape, nous sommes sûr 
qu'au bout d'un nombre fini d'itération  notre procédure se termine.


%%%%%%%%%%%%%%%%%%%%%%%%%%%%%%%%%%%%%%%%%%%%%%%%%%%%%%%%%%%
\diapo

Voici l'algorithme qui traduit notre calcul de la division euclidienne.

En entrée nous avons le polynôme $A$ que l'on souhaite diviser par le polynôme $B$.

On initialise notre quotient intermédiaire au polynôme nul.

Et notre reste intermédiaire est $A$ au départ.
  
\change

Nous allons itérer chaque étape tant que le degré du reste est supérieur ou égal au degré de $B$.

  
\change

A chaque étape nous calculons le monôme qui apparait dans le quotient :

si ce monôme est $c_k X^k$ alors le coefficient $c_k$ est le  coefficient dominant du reste intermédiaire, 

et le degré $k$ correspond au degré du reste intermédiaire moins le degré de $B$.

Ici "monome" est une fonction toute simple qui à partir de $c_k$ et du degré $k$ renvoie le monôme
$c_k X^k$.


\change

On calcule alors le nouveau reste intermédiaire c'est l'ancien reste auquel on soustrait $B$ fois notre monome $P$.

C'est une version un peu simplifiée du code : car il faudrait remplacer  $-P$  par la liste correspondante.
  
\change

Le nouveau quotient intermédiaire est la somme de l'ancien quotient avec notre monôme $P$.

\change

Et c'est fini !

Lorsque le degré du reste intermédiaire est strictement inférieur au degré de $B$ alors la condition
de notre boucle n'est plus vérifiée. Et on renvoie à la fois le quotient et le reste..

%   
% Pour simplifier l'écriture de cette algorithme nous avons supposer que 
% 
% $B$ est un polynôme unitaire (c-a-d son coefficient de plus haut degré est $1$)
% 
% Par exemple si $A$ et $B$ ont des coefficients entiers et que $B$ est unitaire
% alors cela entraine que $Q$ et $R$ ont aussi des coefficients entiers.
% 
% Nous l'avons implicitement prouver par cet algorithme car nous n'avons jamais fait de division : 
% seulement des addition et multiplications.

Je vous laisse calculer la complexité la division euclidienne, qui se calcule comme successions de
sommes et de produits de polynômes.


%%%%%%%%%%%%%%%%%%%%%%%%%%%%%%%%%%%%%%%%%%%%%%%%%%%%%%%%%%%
\diapo

Nous avons vu qu'avec la façon classique de multiplier deux polynômes la complexité 
est de l'ordre du carré des degrés.

\change

Pour diminuer la complexité de la multiplication de polynômes, 
on va utiliser un paradigme très classique de programmation :
\og diviser pour régner \fg. 

\change

Que l'on illustre ici avec l'algorithme de Karatsuba.

\change

Pour cela, on suppose que les deux polynômes  $P$ et $Q$  à multiplier
sont de degrés strictement inférieurs à $2n$.

\change

On va décomposer chacun d'eux en deux polynômes de degré strictement inférieur à $n$ ainsi.

$P = P_1 + P_2 \cdot X^n$

et 

$Q = Q_1 + Q_2 \cdot X^n$

\change

avec les degrés de $P_1$, $P_2$, $Q_1$ et $Q_2$ strictement inférieurs à $n$.

\change

Voici votre travail :

Première question : réduire la multiplication de $P$ et $Q$ à plusieurs 
multiplications de polynômes mais qui sont tous strictement inférieurs à $n$.

Vous vous aiderez bien sûr de cette décomposition.

Deuxième question que je vous laisse traiter tout seul : mettre en oeuvre cette idée 
dans un algorithme récursif.

On peut calculer la complexité de cet algorithme mais, pas de chance, il est aussi
d'ordre le carré des degrés comme la multiplication classique.

L'idée lumineuse de Karatsuba est que les quatre multiplications 
que l'on a trouvé dans la formule de la question $1$ se ramène à seulement $3$,

par la remarque suivante 

$P_1 \cdot Q_2 + P_2 \cdot Q_1 = (P_1 + P_2 ) \cdot (Q_1 + Q_2 ) - P_1 Q_1 - P_2 Q_2$


Comme on a déjà eu besoin de calculer $P_1Q_1$ et $P_2Q_2$,

on échange deux multiplications et une addition

contre une multiplication et quatre additions.

A vous d'écrire une fonction récursive qui met en oeuvre cette nouvelle façon de multiplier.

Pour conclure essayer de calculer la complexité de l'algorithme de Karatsuba.

%%%%%%%%%%%%%%%%%%%%%%%%%%%%%%%%%%%%%%%%%%%%%%%%%%%%%%%%%%%
\diapo


On part d'un polynôme $P$ de degré strictement inférieurs à $2n$,

il existe une unique écriture de $P$ sous la forme $P = P_1 + P_2 X^n $

avec $P_1$ et $P_2$ deux polynômes de degré strictement inférieurs à $n$.

On fait la même chose pour $Q$.

\change

Nous souhaitons calculer le produit $P\times Q$. Nous utilisons cette décomposition  
et nous développons.

\change

On obtient $P\times Q= P_1 Q_1 + X^n \cdot (P_1 Q_2 + P_2 Q_1 ) + X^{2n} \cdot P_2 Q_2$

\change

On se ramène ainsi aux quatre multiplications $P_1 Q_1$ , $P_1 Q_2$ , $P_2 Q_1$ et $P_2 Q_2$ 
mais entre polynômes de degrés strictement inférieurs à $n$,


\change



il y a aussi deux multiplications par $X^n$ et $X^{2n}$ qui ne  compte pas comme des vraies multiplications.

Ce ne sont que des décalage dans l'écriture d'un polynôme par des ajouts de zéros en tête de liste.

\change

On pourrait programmer un algorithme récursif à l'aide de cette formule :

multiplier deux polynômes d'un certain degré revient à faire $4$ multiplications avec des degrés moitiés.

La complexité vérifie donc $C(n) = 4C(n/2)$ + des termes négligeables. La solution de cette formule de récurrence
serait  $C(n) = O(n^2)$ c'est-à-dire du même ordre que la multiplication classique, une fonction en 
le carré du degré.
 
 

\change

Reprenons alors l'astuce de Karatsuba :

\change

le terme intermédiaire dans le produit développé $P \cdot Q$ se réécrit 
$$P_1 \cdot Q_2 + P_2 \cdot Q_1 = (P_1 + P_2 ) \cdot (Q_1 + Q_2 ) - P_1 Q_1 - P_2 Q_2$$

\change

On calcule donc d'abord $P_1Q_1$ et aussi $P_2Q_2$ ce qui nous fait deux multiplications.

Au lieu de calculer $P_1 \cdot Q_2 + P_2 \cdot Q_1$ ce qui ferait deux autres multiplications,

On fait une troisième multiplication qui est $(P_1 + P_2 ) \cdot (Q_1 + Q_2 )$

et on soustrait les deux produits déjà calculés.



On est passé des $4$ multiplications à seulement $3$ multiplication entre polynômes de degré $<n$, 
ce qui est un gain très appréciable.

\change

Pour écrire l'algorithme nous aurons besoin d'écrire une fonction de découpage
qui pour un entier $n$ et un polynôme $P$ de degré strictement inférieur à $2n$ 
renvoie $P_1$ et $P_2$ de sorte que $P = P_1 + X^n P_2 $

\change
  
   On a aussi besoin d'une fonction "produit-par-un-monome(P,n)" qui renvoie
 le polynôme $X^n \cdot P$ par un décalage dans la liste représentant $P$.


%%%%%%%%%%%%%%%%%%%%%%%%%%%%%%%%%%%%%%%%%%%%%%%%%%%%%%%%%%%
\diapo

Voici l'algorithme de Karatsuba.

Nous allons écrire une fonction "produit-rapide"
qui en entrée prend les deux polynômes $P$ et $Q$ et 
en sortie renvoie leur produit.

On va noter petit $p$ le degré du polynôme  $P$ et petit $q$ celui de $Q$.

\change

Ce sera un algorithme récursif,

et voici la condition d'initialisation (qui en en fait est celle qui termine l'algorithme).

Si le degré de $P$ est zéro, c'est-à-dire $P$ est un polynôme cst
alors le produit $P$ fois $Q$ est juste la constante $P$ fois le polynôme $Q$.

Même chose si c'est $Q$ qui est le polynôme cst.


\change

On découpe maintenant nos deux polynômes $P$ et $Q$. On choisit d'abord $n$ de sorte 
$P$ et $Q$ soient de degré strictement inférieur à $2n$.

Puis on utilise une fonction "decoupe" que l'on programmé auparavant. 
On obtient $P_1, P_2$ et aussi $Q_1,Q_2$ de degré strictement inférieur à $n$.

\change

Nous avons maintenant des produits à calculer : 


tout d'abord le produit de $P_1 \times Q_1$.

Pour cela nous effectuons un appel récursif à notre propre fonction sachant que les degrés
des polynômes à multiplier sont maintenant plus petits.

Idem pour $P_2 \times Q_2$.

\change


Il nous reste le troisième et dernier produit qui est $(P_1 + P_2 ) \cdot (Q_1 + Q_2 )$


\change

Le gros du travail est fini il ne reste plus qu'à 
appliquer la formule qui donne le produit $P \times Q$.

On note $R_1$ le polynôme $(P_1 + P_2 ) \cdot (Q_1 + Q_2 ) - P_1 Q_1 - P_2 Q_2$

(ici j'ai encore triché car on n'a pas le droit d'écrire directement l'opposé d'un polynôme.)

\change

On multiplie $R_1$ par $X^n$ (c'est juste un décalage par ajout de $0$ dans la liste).

\change

On multiplie aussi $P_2Q_2$ par $X^{2n}$ (encore un décalage).

\change

Le produit $P\times Q$ est alors la somme de $P_1\times Q_1 + R_1 + R_2$.

Et c'est terminé.

Il y a donc trois appels récursifs qui correspondent aux trois multiplications de Karatsuba.

%%%%%%%%%%%%%%%%%%%%%%%%%%%%%%%%%%%%%%%%%%%%%%%%%%%%%%%%%%%
\diapo

Nous allons calculer la complexité de l'algorithme de Karatsuba et 
voir qu'elle est meilleure que la multiplication habituelle.



Notons $C(n)$ la complexité de la multiplication par l'algorithme de Karatsuba de 
deux polynômes de degrés strictement inférieurs à $n$. 

\change

 Il y a tout d'abord trois appels récursifs qui font intervenir des polynômes en degré moitié.
 
\change

Il y a aussi des opérations rapides : qui ont une complexité qui varie linéairement avec le degré :

c'est le cas du calcul du degré d'un polynôme, de l'opération de découpage,

et des différentes additions.

\change

Pour la complexité cela signifie que la complexité $C(n)$ vaut $3$ fois la complexité
en degré moitié + des termes petits. 

On pourrait omettre les termes petits mais ici on va les converser : $\gamma$; 
est une constante qui vaut en fait ici
$15/2$ et englobe toutes ces opérations.


Nous avons un sorte de formule de récurrence entre $C(n)$ et $C(n/2)$.


\change


Pour en déduire la valeur de $C(n)$ nous allons introduire une suite intermédiaire $\alpha_\ell$ en posant

$\alpha_\ell= \frac{C(2^\ell)}{3^\ell}$

\change

et par définition $\alpha_0 = C(1)=1$.

\change

Que donne notre relation de récurrence pour $\alpha_\ell$ :

$\alpha_\ell$ c'est  $\frac{C(2^\ell)}{3^\ell}$ et donc par la relation de récurrence c'est 
$= \frac{3 C(2^{\ell}/2)}{3^\ell} + \frac{\gamma 2^\ell}{3^\ell}$

\change

Ce qui donne exactement $ \alpha_{\ell-1} + \gamma \left( \frac23  \right)^\ell$

\change

Il est maintenant très facile de calculer $\alpha_\ell$ :

 $\alpha_\ell = \gamma \sum_{k=1}^\ell \left( \frac23  \right)^{k}+\alpha_0 $
 
 \change
 
 et comme $\alpha_0=1$ on reconnait la somme d'une suite géométrique de raison $2/3$ qui vaut  : 
 $3\gamma\left( 1 -  \left( \frac23  \right)^{\ell+1} \right) +1-\gamma$
 
\change

Revenons à $C(n)$,

\change

en posant $n=2^\ell$

\change

$C(n)=C(2^\ell)$ qui vaut $3^\ell \alpha_\ell$

\change

et par ce que nous venons de calculer cela donne :

$C(2^\ell) = \gamma (3^{\ell+1}-2^{\ell+1}) +(1-\gamma)3^\ell$


\change

Et donc $C(n)$ est de l'ordre de $3^\ell$, 

$O(3^\ell)$ représente une suite qui est plus petit qu'une constante fois $3^\ell$.

\change


Mais $3^\ell$ c'est $\exp (\ell \ln 3)$ que l'on récrit $\exp (\ell \ln 2 \times \ln 3 / \ln 2)$

qui est donc $2^{\ell \frac{\ln 3}{\ln 2}}$ 

\change

mais comme $2^\ell=n$ 

la complexité $C(n)$ est un $O(n^\frac{\ln 3}{\ln 2})$.

\change

En conclusion la complexité de la multiplication de Karatsuba est 
$O(n^\frac{\ln 3}{\ln 2})$ 

c'est à dire que pour multiplier deux polynômes de degré $n$

la complexité est de l'ordre de $n^{1.585}$ ce qui est beaucoup mieux que $n^2$ !


%%%%%%%%%%%%%%%%%%%%%%%%%%%%%%%%%%%%%%%%%%%%%%%%%%%%%%%%%%%
\diapo

Voici des exercices pour continuer à progresser !


\end{document}