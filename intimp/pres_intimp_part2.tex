
%%%%%%%%%%%%%%%%%% PREAMBULE %%%%%%%%%%%%%%%%%%

\documentclass[aspectratio=169,utf8]{beamer}
%\documentclass[aspectratio=169,handout]{beamer}

\usetheme{Boadilla}
%\usecolortheme{seahorse}
\usecolortheme[RGB={245,66,24}]{structure}
\useoutertheme{infolines}

% packages
\usepackage{amsfonts,amsmath,amssymb,amsthm}
\usepackage[utf8]{inputenc}
\usepackage[T1]{fontenc}
\usepackage{lmodern}

\usepackage[francais]{babel}
\usepackage{fancybox}
\usepackage{graphicx}

\usepackage{float}
\usepackage{xfrac}

%\usepackage[usenames, x11names]{xcolor}
\usepackage{tikz}
\usepackage{pgfplots}
\usepackage{datetime}



%-----  Package unités -----
\usepackage{siunitx}
\sisetup{locale = FR,detect-all,per-mode = symbol}

%\usepackage{mathptmx}
%\usepackage{fouriernc}
%\usepackage{newcent}
%\usepackage[mathcal,mathbf]{euler}

%\usepackage{palatino}
%\usepackage{newcent}
% \usepackage[mathcal,mathbf]{euler}



% \usepackage{hyperref}
% \hypersetup{colorlinks=true, linkcolor=blue, urlcolor=blue,
% pdftitle={Exo7 - Exercices de mathématiques}, pdfauthor={Exo7}}


%section
% \usepackage{sectsty}
% \allsectionsfont{\bf}
%\sectionfont{\color{Tomato3}\upshape\selectfont}
%\subsectionfont{\color{Tomato4}\upshape\selectfont}

%----- Ensembles : entiers, reels, complexes -----
\newcommand{\Nn}{\mathbb{N}} \newcommand{\N}{\mathbb{N}}
\newcommand{\Zz}{\mathbb{Z}} \newcommand{\Z}{\mathbb{Z}}
\newcommand{\Qq}{\mathbb{Q}} \newcommand{\Q}{\mathbb{Q}}
\newcommand{\Rr}{\mathbb{R}} \newcommand{\R}{\mathbb{R}}
\newcommand{\Cc}{\mathbb{C}} 
\newcommand{\Kk}{\mathbb{K}} \newcommand{\K}{\mathbb{K}}

%----- Modifications de symboles -----
\renewcommand{\epsilon}{\varepsilon}
\renewcommand{\Re}{\mathop{\text{Re}}\nolimits}
\renewcommand{\Im}{\mathop{\text{Im}}\nolimits}
%\newcommand{\llbracket}{\left[\kern-0.15em\left[}
%\newcommand{\rrbracket}{\right]\kern-0.15em\right]}

\renewcommand{\ge}{\geqslant}
\renewcommand{\geq}{\geqslant}
\renewcommand{\le}{\leqslant}
\renewcommand{\leq}{\leqslant}
\renewcommand{\epsilon}{\varepsilon}

%----- Fonctions usuelles -----
\newcommand{\ch}{\mathop{\text{ch}}\nolimits}
\newcommand{\sh}{\mathop{\text{sh}}\nolimits}
\renewcommand{\tanh}{\mathop{\text{th}}\nolimits}
\newcommand{\cotan}{\mathop{\text{cotan}}\nolimits}
\newcommand{\Arcsin}{\mathop{\text{arcsin}}\nolimits}
\newcommand{\Arccos}{\mathop{\text{arccos}}\nolimits}
\newcommand{\Arctan}{\mathop{\text{arctan}}\nolimits}
\newcommand{\Argsh}{\mathop{\text{argsh}}\nolimits}
\newcommand{\Argch}{\mathop{\text{argch}}\nolimits}
\newcommand{\Argth}{\mathop{\text{argth}}\nolimits}
\newcommand{\pgcd}{\mathop{\text{pgcd}}\nolimits} 


%----- Commandes divers ------
\newcommand{\ii}{\mathrm{i}}
\newcommand{\dd}{\text{d}}
\newcommand{\id}{\mathop{\text{id}}\nolimits}
\newcommand{\Ker}{\mathop{\text{Ker}}\nolimits}
\newcommand{\Card}{\mathop{\text{Card}}\nolimits}
\newcommand{\Vect}{\mathop{\text{Vect}}\nolimits}
\newcommand{\Mat}{\mathop{\text{Mat}}\nolimits}
\newcommand{\rg}{\mathop{\text{rg}}\nolimits}
\newcommand{\tr}{\mathop{\text{tr}}\nolimits}


%----- Structure des exercices ------

\newtheoremstyle{styleexo}% name
{2ex}% Space above
{3ex}% Space below
{}% Body font
{}% Indent amount 1
{\bfseries} % Theorem head font
{}% Punctuation after theorem head
{\newline}% Space after theorem head 2
{}% Theorem head spec (can be left empty, meaning ‘normal’)

%\theoremstyle{styleexo}
\newtheorem{exo}{Exercice}
\newtheorem{ind}{Indications}
\newtheorem{cor}{Correction}


\newcommand{\exercice}[1]{} \newcommand{\finexercice}{}
%\newcommand{\exercice}[1]{{\tiny\texttt{#1}}\vspace{-2ex}} % pour afficher le numero absolu, l'auteur...
\newcommand{\enonce}{\begin{exo}} \newcommand{\finenonce}{\end{exo}}
\newcommand{\indication}{\begin{ind}} \newcommand{\finindication}{\end{ind}}
\newcommand{\correction}{\begin{cor}} \newcommand{\fincorrection}{\end{cor}}

\newcommand{\noindication}{\stepcounter{ind}}
\newcommand{\nocorrection}{\stepcounter{cor}}

\newcommand{\fiche}[1]{} \newcommand{\finfiche}{}
\newcommand{\titre}[1]{\centerline{\large \bf #1}}
\newcommand{\addcommand}[1]{}
\newcommand{\video}[1]{}

% Marge
\newcommand{\mymargin}[1]{\marginpar{{\small #1}}}

\def\noqed{\renewcommand{\qedsymbol}{}}


%----- Presentation ------
\setlength{\parindent}{0cm}

%\newcommand{\ExoSept}{\href{http://exo7.emath.fr}{\textbf{\textsf{Exo7}}}}

\definecolor{myred}{rgb}{0.93,0.26,0}
\definecolor{myorange}{rgb}{0.97,0.58,0}
\definecolor{myyellow}{rgb}{1,0.86,0}

\newcommand{\LogoExoSept}[1]{  % input : echelle
{\usefont{U}{cmss}{bx}{n}
\begin{tikzpicture}[scale=0.1*#1,transform shape]
  \fill[color=myorange] (0,0)--(4,0)--(4,-4)--(0,-4)--cycle;
  \fill[color=myred] (0,0)--(0,3)--(-3,3)--(-3,0)--cycle;
  \fill[color=myyellow] (4,0)--(7,4)--(3,7)--(0,3)--cycle;
  \node[scale=5] at (3.5,3.5) {Exo7};
\end{tikzpicture}}
}


\newcommand{\debutmontitre}{
  \author{} \date{} 
  \thispagestyle{empty}
  \hspace*{-10ex}
  \begin{minipage}{\textwidth}
    \titlepage  
  \vspace*{-2.5cm}
  \begin{center}
    \LogoExoSept{2.5}
  \end{center}
  \end{minipage}

  \vspace*{-0cm}
  
  % Astuce pour que le background ne soit pas discrétisé lors de la conversion pdf -> png
\begin{tikzpicture}
        \fill[opacity=0,green!60!black] (0,0)--++(0,0)--++(0,0)--++(0,0)--cycle; 
\end{tikzpicture}

% toc S'affiche trop tot :
% \tableofcontents[hideallsubsections, pausesections]
}

\newcommand{\finmontitre}{
  \end{frame}
  \setcounter{framenumber}{0}
} % ne marche pas pour une raison obscure

%----- Commandes supplementaires ------

% \usepackage[landscape]{geometry}
% \geometry{top=1cm, bottom=3cm, left=2cm, right=10cm, marginparsep=1cm
% }
% \usepackage[a4paper]{geometry}
% \geometry{top=2cm, bottom=2cm, left=2cm, right=2cm, marginparsep=1cm
% }

%\usepackage{standalone}


% New command Arnaud -- november 2011
\setbeamersize{text margin left=24ex}
% si vous modifier cette valeur il faut aussi
% modifier le decalage du titre pour compenser
% (ex : ici =+10ex, titre =-5ex

\theoremstyle{definition}
%\newtheorem{proposition}{Proposition}
%\newtheorem{exemple}{Exemple}
%\newtheorem{theoreme}{Théorème}
%\newtheorem{lemme}{Lemme}
%\newtheorem{corollaire}{Corollaire}
%\newtheorem*{remarque*}{Remarque}
%\newtheorem*{miniexercice}{Mini-exercices}
%\newtheorem{definition}{Définition}

% Commande tikz
\usetikzlibrary{calc}
\usetikzlibrary{patterns,arrows}
\usetikzlibrary{matrix}
\usetikzlibrary{fadings} 

%definition d'un terme
\newcommand{\defi}[1]{{\color{myorange}\textbf{\emph{#1}}}}
\newcommand{\evidence}[1]{{\color{blue}\textbf{\emph{#1}}}}
\newcommand{\assertion}[1]{\emph{\og#1\fg}}  % pour chapitre logique
%\renewcommand{\contentsname}{Sommaire}
\renewcommand{\contentsname}{}
\setcounter{tocdepth}{2}



%------ Figures ------

\def\myscale{1} % par défaut 
\newcommand{\myfigure}[2]{  % entrée : echelle, fichier figure
\def\myscale{#1}
\begin{center}
\footnotesize
{#2}
\end{center}}


%------ Encadrement ------

\usepackage{fancybox}


\newcommand{\mybox}[1]{
\setlength{\fboxsep}{7pt}
\begin{center}
\shadowbox{#1}
\end{center}}

\newcommand{\myboxinline}[1]{
\setlength{\fboxsep}{5pt}
\raisebox{-10pt}{
\shadowbox{#1}
}
}

%--------------- Commande beamer---------------
\newcommand{\beameronly}[1]{#1} % permet de mettre des pause dans beamer pas dans poly


\setbeamertemplate{navigation symbols}{}
\setbeamertemplate{footline}  % tiré du fichier beamerouterinfolines.sty
{
  \leavevmode%
  \hbox{%
  \begin{beamercolorbox}[wd=.333333\paperwidth,ht=2.25ex,dp=1ex,center]{author in head/foot}%
    % \usebeamerfont{author in head/foot}\insertshortauthor%~~(\insertshortinstitute)
    \usebeamerfont{section in head/foot}{\bf\insertshorttitle}
  \end{beamercolorbox}%
  \begin{beamercolorbox}[wd=.333333\paperwidth,ht=2.25ex,dp=1ex,center]{title in head/foot}%
    \usebeamerfont{section in head/foot}{\bf\insertsectionhead}
  \end{beamercolorbox}%
  \begin{beamercolorbox}[wd=.333333\paperwidth,ht=2.25ex,dp=1ex,right]{date in head/foot}%
    % \usebeamerfont{date in head/foot}\insertshortdate{}\hspace*{2em}
    \insertframenumber{} / \inserttotalframenumber\hspace*{2ex} 
  \end{beamercolorbox}}%
  \vskip0pt%
}


\definecolor{mygrey}{rgb}{0.5,0.5,0.5}
\setlength{\parindent}{0cm}
%\DeclareTextFontCommand{\helvetica}{\fontfamily{phv}\selectfont}

% background beamer
\definecolor{couleurhaut}{rgb}{0.85,0.9,1}  % creme
\definecolor{couleurmilieu}{rgb}{1,1,1}  % vert pale
\definecolor{couleurbas}{rgb}{0.85,0.9,1}  % blanc
\setbeamertemplate{background canvas}[vertical shading]%
[top=couleurhaut,middle=couleurmilieu,midpoint=0.4,bottom=couleurbas] 
%[top=fondtitre!05,bottom=fondtitre!60]



\makeatletter
\setbeamertemplate{theorem begin}
{%
  \begin{\inserttheoremblockenv}
  {%
    \inserttheoremheadfont
    \inserttheoremname
    \inserttheoremnumber
    \ifx\inserttheoremaddition\@empty\else\ (\inserttheoremaddition)\fi%
    \inserttheorempunctuation
  }%
}
\setbeamertemplate{theorem end}{\end{\inserttheoremblockenv}}

\newenvironment{theoreme}[1][]{%
   \setbeamercolor{block title}{fg=structure,bg=structure!40}
   \setbeamercolor{block body}{fg=black,bg=structure!10}
   \begin{block}{{\bf Th\'eor\`eme }#1}
}{%
   \end{block}%
}


\newenvironment{proposition}[1][]{%
   \setbeamercolor{block title}{fg=structure,bg=structure!40}
   \setbeamercolor{block body}{fg=black,bg=structure!10}
   \begin{block}{{\bf Proposition }#1}
}{%
   \end{block}%
}

\newenvironment{corollaire}[1][]{%
   \setbeamercolor{block title}{fg=structure,bg=structure!40}
   \setbeamercolor{block body}{fg=black,bg=structure!10}
   \begin{block}{{\bf Corollaire }#1}
}{%
   \end{block}%
}

\newenvironment{mydefinition}[1][]{%
   \setbeamercolor{block title}{fg=structure,bg=structure!40}
   \setbeamercolor{block body}{fg=black,bg=structure!10}
   \begin{block}{{\bf Définition} #1}
}{%
   \end{block}%
}

\newenvironment{lemme}[0]{%
   \setbeamercolor{block title}{fg=structure,bg=structure!40}
   \setbeamercolor{block body}{fg=black,bg=structure!10}
   \begin{block}{\bf Lemme}
}{%
   \end{block}%
}

\newenvironment{remarque}[1][]{%
   \setbeamercolor{block title}{fg=black,bg=structure!20}
   \setbeamercolor{block body}{fg=black,bg=structure!5}
   \begin{block}{Remarque #1}
}{%
   \end{block}%
}


\newenvironment{exemple}[1][]{%
   \setbeamercolor{block title}{fg=black,bg=structure!20}
   \setbeamercolor{block body}{fg=black,bg=structure!5}
   \begin{block}{{\bf Exemple }#1}
}{%
   \end{block}%
}


\newenvironment{miniexercice}[0]{%
   \setbeamercolor{block title}{fg=structure,bg=structure!20}
   \setbeamercolor{block body}{fg=black,bg=structure!5}
   \begin{block}{Mini-exercices}
}{%
   \end{block}%
}


\newenvironment{tp}[0]{%
   \setbeamercolor{block title}{fg=structure,bg=structure!40}
   \setbeamercolor{block body}{fg=black,bg=structure!10}
   \begin{block}{\bf Travaux pratiques}
}{%
   \end{block}%
}
\newenvironment{exercicecours}[1][]{%
   \setbeamercolor{block title}{fg=structure,bg=structure!40}
   \setbeamercolor{block body}{fg=black,bg=structure!10}
   \begin{block}{{\bf Exercice }#1}
}{%
   \end{block}%
}
\newenvironment{algo}[1][]{%
   \setbeamercolor{block title}{fg=structure,bg=structure!40}
   \setbeamercolor{block body}{fg=black,bg=structure!10}
   \begin{block}{{\bf Algorithme}\hfill{\color{gray}\texttt{#1}}}
}{%
   \end{block}%
}


\setbeamertemplate{proof begin}{
   \setbeamercolor{block title}{fg=black,bg=structure!20}
   \setbeamercolor{block body}{fg=black,bg=structure!5}
   \begin{block}{{\footnotesize Démonstration}}
   \footnotesize
   \smallskip}
\setbeamertemplate{proof end}{%
   \end{block}}
\setbeamertemplate{qed symbol}{\openbox}


\makeatother
\usecolortheme[RGB={102,51,102}]{structure}

%%%%%%%%%%%%%%%%%%%%%%%%%%%%%%%%%%%%%%%%%%%%%%%%%%%%%%%%%%%%%
%%%%%%%%%%%%%%%%%%%%%%%%%%%%%%%%%%%%%%%%%%%%%%%%%%%%%%%%%%%%%


\begin{document}


\title{{\bf Intégrales impropres}}
\subtitle{Fonctions positives}

\begin{frame}
  
  \debutmontitre

  \pause

{\footnotesize
\hfill
\setbeamercovered{transparent=50}
\begin{minipage}{0.6\textwidth}
  \begin{itemize}
    \item<3-> Théorème de comparaison
    \item<4-> Théorème des équivalents
    \item<5-> Intégrales de Riemann
    \item<6-> Intégrales de Bertrand    
      \end{itemize}
\end{minipage}
}

\end{frame}

\setcounter{framenumber}{0}

%%%%%%%%%%%%%%%%%%%%%%%%%%%%%%%%%%%%%%%%%%%%%%%%%%%%%%%%%%%%%%%%
\section*{Théorème de comparaison}

\begin{frame}

\begin{theoreme}[de comparaison]
\label{th:comparaisonintegrales1}
\pause
Soient $f$ et $g$ deux fonctions continues et \evidence{positives} sur
$[a,+\infty[$. Supposons que $f$ soit majorée par $g$ au voisinage de
$+\infty$ : c'est-à-dire \\
\centerline{$\exists A\ge a \quad \forall t>A \qquad f(t)\le g(t)$}
\pause
\begin{enumerate}
  \item Si \  $\int_a^{+\infty} g(t)\;\dd t$ \ converge alors \  $\int_a^{+\infty} f(t)\;\dd t$ \  converge
\pause
  \item Si \  $\int_a^{+\infty} f(t)\;\dd t$ \  diverge alors \  $\int_a^{+\infty} g(t)\;\dd t$ \  diverge
\end{enumerate}
\end{theoreme}

\pause
\medskip

\begin{proof}
\begin{itemize}
  \item Par positivité de l'intégrale, pour tout $x \ge A$,
$\int_A^x f(t)\;\dd t \le \int_A^x g(t)\;\dd t$
\pause
  \item Si $\int_A^{+\infty} g(t)\;\dd t$ converge, alors $\int_A^x f(t)\;\dd t$ est
une fonction croissante et majorée par $\int_A^{+\infty} g(t)\;\dd t$,
donc convergente
\pause
  \item Inversement, si $\int_A^{x} f(t)\;\dd t$ tend vers
$+\infty$, alors $\int_A^{x} g(t)\;\dd t$ tend vers $+\infty$
\qedhere
\end{itemize}
\end{proof}

\end{frame}


\begin{frame}
\begin{exemple}

$\int_1^{+\infty} t^\alpha e^{-t}\;\dd t$ converge quel que soit le réel $\alpha$ 


\begin{itemize}
  \item \pause $t^\alpha e^{-t} = t^\alpha e^{-t/2}\,e^{-t/2}$
  \item \pause $\lim_{t\rightarrow+\infty} t^\alpha e^{-t/2} =0$
  \item \pause En particulier, il existe un réel $A>0$ tel que 
$$\forall t>A\qquad t^\alpha e^{-t/2}\le 1$$  
  \item \pause $\forall t>A\qquad t^\alpha e^{-t}\le e^{-t/2}$ 
  \item \pause $\displaystyle 
\int_1^x e^{-t/2}\;\dd t 
\pause = \left[-2 e^{-t/2}\right]_1^x 
\pause = 2e^{-1/2} -2e^{-x/2}
\pause \xrightarrow{x\rightarrow+\infty}2e^{-1/2}$ 
  \item \pause $\int_1^{+\infty} e^{-t/2}\;\dd t$ converge
  \item \pause Théorème de comparaison : $\int_1^{+\infty} t^\alpha e^{-t}\;\dd t$ converge
\end{itemize}
\end{exemple}
\end{frame}


%%%%%%%%%%%%%%%%%%%%%%%%%%%%%%%%%%%%%%%%%%%%%%%%%%%%%%%%%%%%%%%%
\section*{Théorème des équivalents}

\begin{frame}

\begin{theoreme}[des équivalents]
%\label{th:equivalentintegrales1}
Soient $f$ et $g$ deux fonctions continues et \evidence{strictement positives} sur
$[a,+\infty[$. Supposons qu'elles soient équivalentes au voisinage de $+\infty$, c'est-à-dire :
$\lim_{t\rightarrow+\infty}\frac{f(t)}{g(t)} = 1\;.$
\pause
Alors l'intégrale $\int_a^{+\infty} f(t)\;\dd t$ converge si et seulement si 
$\int_a^{+\infty} g(t)\;\dd t$ converge
\end{theoreme}
\pause
\begin{proof}
\vspace*{-2ex}
\begin{itemize}
  \item $\frac{f(t)}{g(t)} \to 1$ 
\pause  
  \item Donc $\forall \epsilon>0 \quad \exists A>a \quad \forall t>A \qquad (1-\epsilon)g(t)<f(t)<(1+\epsilon)g(t)$
\pause
  \item Théorème de comparaison avec $\epsilon<1$
  \qedhere

\end{itemize}

\end{proof}
\pause
\begin{exemple}
\begin{itemize}
  \item $\frac{t^5+3t+1}{t^3+4}e^{-t} \pause\underset{+\infty}{\sim}\ t^2e^{-t}$
  \pause
  \item $\int_1^{+\infty} t^2e^{-t}\;\dd t$ converge
  \pause
  \item Donc $\int_1^{+\infty} \frac{t^5+3t+1}{t^3+4}e^{-t}\;\dd t$ converge
\end{itemize}

\end{exemple}


\end{frame}

%%%%%%%%%%%%%%%%%%%%%%%%%%%%%%%%%%%%%%%%%%%%%%%%%%%%%%%%%%%%%%%%
\section*{Intégrales de Riemann}

\begin{frame}

\smallskip
 
\defi{Intégrale de Riemann} \qquad $\displaystyle
\int_1^{+\infty} \frac{1}{t^{\alpha}}\;\dd t\qquad \alpha>0$

\pause
\mybox{

\begin{minipage}{0.7\textwidth}\vspace*{-2ex}
$$\text{Si } \quad \alpha > 1\quad \text{ alors }\quad 
\int_1^{+\infty} \frac{1}{t^{\alpha}}\;\dd t \quad\text{ converge}$$
\pause
$$\text{Si } \quad \alpha\le 1\quad \text{ alors }\quad 
\int_1^{+\infty} \frac{1}{t^{\alpha}}\;\dd t \quad\text{ diverge}$$
\end{minipage}
\pause
}

\pause
Preuve :
$$
\int_1^{+\infty} \frac{1}{t^{\alpha}}\;\dd t 
\pause = 
\left\{
\begin{array}{lcl}
\displaystyle{\lim_{x\rightarrow+\infty} 
\left[\frac{1}{-\alpha+1}\frac{1}{t^{\alpha-1}}\right]_1^x}
&\quad\text{si}&\alpha\neq 1\\[2ex]
\pause
\displaystyle{\lim_{x\rightarrow+\infty} 
\Big[\ln t\Big]_1^x}
&\quad\text{si}&\alpha= 1
\end{array}
\right.
$$



\end{frame}


%%%%%%%%%%%%%%%%%%%%%%%%%%%%%%%%%%%%%%%%%%%%%%%%%%%%%%%%%%%%%%%%
\section*{Intégrales de Bertrand}

\begin{frame}

\smallskip

\defi{Intégrale de Bertrand}\qquad
$\displaystyle \int_2^{+\infty} \frac{1}{t\;(\ln t)^{\beta}}\;\dd t \qquad \beta \in \Rr$

\pause
\mybox{
\begin{minipage}{0.7\textwidth}\vspace*{-2ex}
$$\text{Si } \beta > 1\quad \text{ alors }\quad
\int_2^{+\infty}\frac{1}{t\;(\ln t)^{\beta}}\;\dd t
\quad\text{ converge}
$$
\pause
$$
\text{Si } \beta\le 1\quad \text{ alors }\quad
\int_2^{+\infty}\frac{1}{t\;(\ln t)^{\beta}}\;\dd t \quad\text{ diverge}
$$

\end{minipage}
\pause
}
\pause
Preuve :
$$
\int_2^{+\infty} \frac{1}{t\;(\ln t)^{\beta}}\;\dd t 
\pause= 
\left\{
\begin{array}{lcl}
\displaystyle{\lim_{x\rightarrow+\infty} 
\Big[\tfrac{1}{-\beta+1}(\ln t)^{-\beta+1}\Big]_2^x}
&\quad\text{si}&\beta\neq 1\\[2ex]
\pause
\displaystyle{\lim_{x\rightarrow+\infty} 
\Big[\ln(\ln t)\Big]_2^x}
&\quad\text{si}&\beta= 1
\end{array}
\right.
$$

\end{frame}



\begin{frame}

\begin{exemple}
$$
\int_2^{+\infty} 
\sqrt{t^2+3t}\,\ln\left(\cos\frac{1}{t}\right)
\,\sin^2\left(\frac{1}{\ln t}\right)\;\dd t\qquad\text{ converge-t-elle \  ?}
$$
\pause
\begin{itemize}
  \item $\sqrt{t^2+3t} \pause= t\sqrt{1+\frac{3}{t}} \pause\quad  \underset{+\infty}{\sim}\quad   t$
 \pause  
  \item $\ln\left(\cos\frac{1}{t}\right) \pause= 
\ln\left(1-\frac{1}{2t^2}+o\left(\frac{1}{t^2}\right)\right)
\pause\quad \underset{+\infty}{\sim}\quad  -\frac{1}{2t^2}$
 \pause  
  \item $\sin^2\left(\frac{1}{\ln t}\right) 
\pause\quad\underset{+\infty}{\sim}\quad \left(\frac{1}{\ln t}\right)^2$
 \pause
  \item $\text{D'où }\sqrt{t^2+3t}\,\ln\left(\cos\frac{1}{t}\right)\,
\,\sin^2\left(\frac{1}{\ln t}\right)
\quad \underset{+\infty}{\sim}\quad  -\frac{1}{2t\;(\ln t)^2}$
 \pause
  \item Intégrale de Bertrand avec $\beta=2$, donc converge
\end{itemize}
\end{exemple}
\end{frame}

%%%%%%%%%%%%%%%%%%%%%%%%%%%%%%%%%%%%%%%%%%%%%%%%%%%%%%%%%%%%%%%
 \section*{Mini-exercices}

\begin{frame}
\begin{miniexercice}
\begin{enumerate}
  \item \'Etudier la convergence des intégrales suivantes :  
  $\int_{1}^{+\infty} \frac{1}{t}\sin\left(\frac{1}{t}\right)\dd t \ ,\ 
  \int_{\frac{3}{\pi}}^{+\infty}\ln\left(\cos\frac{1}{t}\right)\dd t \ ,\ 
  \int_{-\infty}^{+\infty} te^{-|t|}\dd t \ ,\ 
  \int_{-\infty}^{\ln 2} \frac{e^{-t}}{t^2+1}\dd t$
  
  \item Montrer que $\int_1^{+\infty} (\ln t)^\alpha e^{-t}\;\dd t$ converge,
quel que soit le réel $\alpha$. 

  \item \'Etudier la convergence des intégrales suivantes, en fonction du paramètre $\alpha>0$ :
  $\int_{1}^{+\infty} \left(1-\sqrt[3]{1+\frac{1}{t^\alpha}}\right)\dd t , \quad
  \int_{\frac{1}{\pi}}^{+\infty} \left(\frac{1}{t^\alpha}-\sin\frac{1}{t^\alpha}\right)\dd t , \quad
  \int_{-\infty}^{\pi} \alpha^t \dd t$
  
  \item Discuter selon $\alpha>0$ et $\beta \in \Rr$ la nature de l'intégrale 
  de Bertrand généralisée
  $$\int_2^{+\infty} \frac{1}{t^\alpha(\ln t)^{\beta}}\;\dd t.$$
  
  \item Si $f : [a,+\infty[ \to \Rr$ est une fonction continue et positive telle que
  $\int_a^{+\infty} f(t)\;\dd t = 0$, montrer alors que $f$ est identiquement nulle.
\end{enumerate}
\end{miniexercice}
\end{frame}




\end{document}

