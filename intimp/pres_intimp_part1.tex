
%%%%%%%%%%%%%%%%%% PREAMBULE %%%%%%%%%%%%%%%%%%

\documentclass[aspectratio=169,utf8]{beamer}
%\documentclass[aspectratio=169,handout]{beamer}

\usetheme{Boadilla}
%\usecolortheme{seahorse}
\usecolortheme[RGB={245,66,24}]{structure}
\useoutertheme{infolines}

% packages
\usepackage{amsfonts,amsmath,amssymb,amsthm}
\usepackage[utf8]{inputenc}
\usepackage[T1]{fontenc}
\usepackage{lmodern}

\usepackage[francais]{babel}
\usepackage{fancybox}
\usepackage{graphicx}

\usepackage{float}
\usepackage{xfrac}

%\usepackage[usenames, x11names]{xcolor}
\usepackage{tikz}
\usepackage{pgfplots}
\usepackage{datetime}



%-----  Package unités -----
\usepackage{siunitx}
\sisetup{locale = FR,detect-all,per-mode = symbol}

%\usepackage{mathptmx}
%\usepackage{fouriernc}
%\usepackage{newcent}
%\usepackage[mathcal,mathbf]{euler}

%\usepackage{palatino}
%\usepackage{newcent}
% \usepackage[mathcal,mathbf]{euler}



% \usepackage{hyperref}
% \hypersetup{colorlinks=true, linkcolor=blue, urlcolor=blue,
% pdftitle={Exo7 - Exercices de mathématiques}, pdfauthor={Exo7}}


%section
% \usepackage{sectsty}
% \allsectionsfont{\bf}
%\sectionfont{\color{Tomato3}\upshape\selectfont}
%\subsectionfont{\color{Tomato4}\upshape\selectfont}

%----- Ensembles : entiers, reels, complexes -----
\newcommand{\Nn}{\mathbb{N}} \newcommand{\N}{\mathbb{N}}
\newcommand{\Zz}{\mathbb{Z}} \newcommand{\Z}{\mathbb{Z}}
\newcommand{\Qq}{\mathbb{Q}} \newcommand{\Q}{\mathbb{Q}}
\newcommand{\Rr}{\mathbb{R}} \newcommand{\R}{\mathbb{R}}
\newcommand{\Cc}{\mathbb{C}} 
\newcommand{\Kk}{\mathbb{K}} \newcommand{\K}{\mathbb{K}}

%----- Modifications de symboles -----
\renewcommand{\epsilon}{\varepsilon}
\renewcommand{\Re}{\mathop{\text{Re}}\nolimits}
\renewcommand{\Im}{\mathop{\text{Im}}\nolimits}
%\newcommand{\llbracket}{\left[\kern-0.15em\left[}
%\newcommand{\rrbracket}{\right]\kern-0.15em\right]}

\renewcommand{\ge}{\geqslant}
\renewcommand{\geq}{\geqslant}
\renewcommand{\le}{\leqslant}
\renewcommand{\leq}{\leqslant}
\renewcommand{\epsilon}{\varepsilon}

%----- Fonctions usuelles -----
\newcommand{\ch}{\mathop{\text{ch}}\nolimits}
\newcommand{\sh}{\mathop{\text{sh}}\nolimits}
\renewcommand{\tanh}{\mathop{\text{th}}\nolimits}
\newcommand{\cotan}{\mathop{\text{cotan}}\nolimits}
\newcommand{\Arcsin}{\mathop{\text{arcsin}}\nolimits}
\newcommand{\Arccos}{\mathop{\text{arccos}}\nolimits}
\newcommand{\Arctan}{\mathop{\text{arctan}}\nolimits}
\newcommand{\Argsh}{\mathop{\text{argsh}}\nolimits}
\newcommand{\Argch}{\mathop{\text{argch}}\nolimits}
\newcommand{\Argth}{\mathop{\text{argth}}\nolimits}
\newcommand{\pgcd}{\mathop{\text{pgcd}}\nolimits} 


%----- Commandes divers ------
\newcommand{\ii}{\mathrm{i}}
\newcommand{\dd}{\text{d}}
\newcommand{\id}{\mathop{\text{id}}\nolimits}
\newcommand{\Ker}{\mathop{\text{Ker}}\nolimits}
\newcommand{\Card}{\mathop{\text{Card}}\nolimits}
\newcommand{\Vect}{\mathop{\text{Vect}}\nolimits}
\newcommand{\Mat}{\mathop{\text{Mat}}\nolimits}
\newcommand{\rg}{\mathop{\text{rg}}\nolimits}
\newcommand{\tr}{\mathop{\text{tr}}\nolimits}


%----- Structure des exercices ------

\newtheoremstyle{styleexo}% name
{2ex}% Space above
{3ex}% Space below
{}% Body font
{}% Indent amount 1
{\bfseries} % Theorem head font
{}% Punctuation after theorem head
{\newline}% Space after theorem head 2
{}% Theorem head spec (can be left empty, meaning ‘normal’)

%\theoremstyle{styleexo}
\newtheorem{exo}{Exercice}
\newtheorem{ind}{Indications}
\newtheorem{cor}{Correction}


\newcommand{\exercice}[1]{} \newcommand{\finexercice}{}
%\newcommand{\exercice}[1]{{\tiny\texttt{#1}}\vspace{-2ex}} % pour afficher le numero absolu, l'auteur...
\newcommand{\enonce}{\begin{exo}} \newcommand{\finenonce}{\end{exo}}
\newcommand{\indication}{\begin{ind}} \newcommand{\finindication}{\end{ind}}
\newcommand{\correction}{\begin{cor}} \newcommand{\fincorrection}{\end{cor}}

\newcommand{\noindication}{\stepcounter{ind}}
\newcommand{\nocorrection}{\stepcounter{cor}}

\newcommand{\fiche}[1]{} \newcommand{\finfiche}{}
\newcommand{\titre}[1]{\centerline{\large \bf #1}}
\newcommand{\addcommand}[1]{}
\newcommand{\video}[1]{}

% Marge
\newcommand{\mymargin}[1]{\marginpar{{\small #1}}}

\def\noqed{\renewcommand{\qedsymbol}{}}


%----- Presentation ------
\setlength{\parindent}{0cm}

%\newcommand{\ExoSept}{\href{http://exo7.emath.fr}{\textbf{\textsf{Exo7}}}}

\definecolor{myred}{rgb}{0.93,0.26,0}
\definecolor{myorange}{rgb}{0.97,0.58,0}
\definecolor{myyellow}{rgb}{1,0.86,0}

\newcommand{\LogoExoSept}[1]{  % input : echelle
{\usefont{U}{cmss}{bx}{n}
\begin{tikzpicture}[scale=0.1*#1,transform shape]
  \fill[color=myorange] (0,0)--(4,0)--(4,-4)--(0,-4)--cycle;
  \fill[color=myred] (0,0)--(0,3)--(-3,3)--(-3,0)--cycle;
  \fill[color=myyellow] (4,0)--(7,4)--(3,7)--(0,3)--cycle;
  \node[scale=5] at (3.5,3.5) {Exo7};
\end{tikzpicture}}
}


\newcommand{\debutmontitre}{
  \author{} \date{} 
  \thispagestyle{empty}
  \hspace*{-10ex}
  \begin{minipage}{\textwidth}
    \titlepage  
  \vspace*{-2.5cm}
  \begin{center}
    \LogoExoSept{2.5}
  \end{center}
  \end{minipage}

  \vspace*{-0cm}
  
  % Astuce pour que le background ne soit pas discrétisé lors de la conversion pdf -> png
\begin{tikzpicture}
        \fill[opacity=0,green!60!black] (0,0)--++(0,0)--++(0,0)--++(0,0)--cycle; 
\end{tikzpicture}

% toc S'affiche trop tot :
% \tableofcontents[hideallsubsections, pausesections]
}

\newcommand{\finmontitre}{
  \end{frame}
  \setcounter{framenumber}{0}
} % ne marche pas pour une raison obscure

%----- Commandes supplementaires ------

% \usepackage[landscape]{geometry}
% \geometry{top=1cm, bottom=3cm, left=2cm, right=10cm, marginparsep=1cm
% }
% \usepackage[a4paper]{geometry}
% \geometry{top=2cm, bottom=2cm, left=2cm, right=2cm, marginparsep=1cm
% }

%\usepackage{standalone}


% New command Arnaud -- november 2011
\setbeamersize{text margin left=24ex}
% si vous modifier cette valeur il faut aussi
% modifier le decalage du titre pour compenser
% (ex : ici =+10ex, titre =-5ex

\theoremstyle{definition}
%\newtheorem{proposition}{Proposition}
%\newtheorem{exemple}{Exemple}
%\newtheorem{theoreme}{Théorème}
%\newtheorem{lemme}{Lemme}
%\newtheorem{corollaire}{Corollaire}
%\newtheorem*{remarque*}{Remarque}
%\newtheorem*{miniexercice}{Mini-exercices}
%\newtheorem{definition}{Définition}

% Commande tikz
\usetikzlibrary{calc}
\usetikzlibrary{patterns,arrows}
\usetikzlibrary{matrix}
\usetikzlibrary{fadings} 

%definition d'un terme
\newcommand{\defi}[1]{{\color{myorange}\textbf{\emph{#1}}}}
\newcommand{\evidence}[1]{{\color{blue}\textbf{\emph{#1}}}}
\newcommand{\assertion}[1]{\emph{\og#1\fg}}  % pour chapitre logique
%\renewcommand{\contentsname}{Sommaire}
\renewcommand{\contentsname}{}
\setcounter{tocdepth}{2}



%------ Figures ------

\def\myscale{1} % par défaut 
\newcommand{\myfigure}[2]{  % entrée : echelle, fichier figure
\def\myscale{#1}
\begin{center}
\footnotesize
{#2}
\end{center}}


%------ Encadrement ------

\usepackage{fancybox}


\newcommand{\mybox}[1]{
\setlength{\fboxsep}{7pt}
\begin{center}
\shadowbox{#1}
\end{center}}

\newcommand{\myboxinline}[1]{
\setlength{\fboxsep}{5pt}
\raisebox{-10pt}{
\shadowbox{#1}
}
}

%--------------- Commande beamer---------------
\newcommand{\beameronly}[1]{#1} % permet de mettre des pause dans beamer pas dans poly


\setbeamertemplate{navigation symbols}{}
\setbeamertemplate{footline}  % tiré du fichier beamerouterinfolines.sty
{
  \leavevmode%
  \hbox{%
  \begin{beamercolorbox}[wd=.333333\paperwidth,ht=2.25ex,dp=1ex,center]{author in head/foot}%
    % \usebeamerfont{author in head/foot}\insertshortauthor%~~(\insertshortinstitute)
    \usebeamerfont{section in head/foot}{\bf\insertshorttitle}
  \end{beamercolorbox}%
  \begin{beamercolorbox}[wd=.333333\paperwidth,ht=2.25ex,dp=1ex,center]{title in head/foot}%
    \usebeamerfont{section in head/foot}{\bf\insertsectionhead}
  \end{beamercolorbox}%
  \begin{beamercolorbox}[wd=.333333\paperwidth,ht=2.25ex,dp=1ex,right]{date in head/foot}%
    % \usebeamerfont{date in head/foot}\insertshortdate{}\hspace*{2em}
    \insertframenumber{} / \inserttotalframenumber\hspace*{2ex} 
  \end{beamercolorbox}}%
  \vskip0pt%
}


\definecolor{mygrey}{rgb}{0.5,0.5,0.5}
\setlength{\parindent}{0cm}
%\DeclareTextFontCommand{\helvetica}{\fontfamily{phv}\selectfont}

% background beamer
\definecolor{couleurhaut}{rgb}{0.85,0.9,1}  % creme
\definecolor{couleurmilieu}{rgb}{1,1,1}  % vert pale
\definecolor{couleurbas}{rgb}{0.85,0.9,1}  % blanc
\setbeamertemplate{background canvas}[vertical shading]%
[top=couleurhaut,middle=couleurmilieu,midpoint=0.4,bottom=couleurbas] 
%[top=fondtitre!05,bottom=fondtitre!60]



\makeatletter
\setbeamertemplate{theorem begin}
{%
  \begin{\inserttheoremblockenv}
  {%
    \inserttheoremheadfont
    \inserttheoremname
    \inserttheoremnumber
    \ifx\inserttheoremaddition\@empty\else\ (\inserttheoremaddition)\fi%
    \inserttheorempunctuation
  }%
}
\setbeamertemplate{theorem end}{\end{\inserttheoremblockenv}}

\newenvironment{theoreme}[1][]{%
   \setbeamercolor{block title}{fg=structure,bg=structure!40}
   \setbeamercolor{block body}{fg=black,bg=structure!10}
   \begin{block}{{\bf Th\'eor\`eme }#1}
}{%
   \end{block}%
}


\newenvironment{proposition}[1][]{%
   \setbeamercolor{block title}{fg=structure,bg=structure!40}
   \setbeamercolor{block body}{fg=black,bg=structure!10}
   \begin{block}{{\bf Proposition }#1}
}{%
   \end{block}%
}

\newenvironment{corollaire}[1][]{%
   \setbeamercolor{block title}{fg=structure,bg=structure!40}
   \setbeamercolor{block body}{fg=black,bg=structure!10}
   \begin{block}{{\bf Corollaire }#1}
}{%
   \end{block}%
}

\newenvironment{mydefinition}[1][]{%
   \setbeamercolor{block title}{fg=structure,bg=structure!40}
   \setbeamercolor{block body}{fg=black,bg=structure!10}
   \begin{block}{{\bf Définition} #1}
}{%
   \end{block}%
}

\newenvironment{lemme}[0]{%
   \setbeamercolor{block title}{fg=structure,bg=structure!40}
   \setbeamercolor{block body}{fg=black,bg=structure!10}
   \begin{block}{\bf Lemme}
}{%
   \end{block}%
}

\newenvironment{remarque}[1][]{%
   \setbeamercolor{block title}{fg=black,bg=structure!20}
   \setbeamercolor{block body}{fg=black,bg=structure!5}
   \begin{block}{Remarque #1}
}{%
   \end{block}%
}


\newenvironment{exemple}[1][]{%
   \setbeamercolor{block title}{fg=black,bg=structure!20}
   \setbeamercolor{block body}{fg=black,bg=structure!5}
   \begin{block}{{\bf Exemple }#1}
}{%
   \end{block}%
}


\newenvironment{miniexercice}[0]{%
   \setbeamercolor{block title}{fg=structure,bg=structure!20}
   \setbeamercolor{block body}{fg=black,bg=structure!5}
   \begin{block}{Mini-exercices}
}{%
   \end{block}%
}


\newenvironment{tp}[0]{%
   \setbeamercolor{block title}{fg=structure,bg=structure!40}
   \setbeamercolor{block body}{fg=black,bg=structure!10}
   \begin{block}{\bf Travaux pratiques}
}{%
   \end{block}%
}
\newenvironment{exercicecours}[1][]{%
   \setbeamercolor{block title}{fg=structure,bg=structure!40}
   \setbeamercolor{block body}{fg=black,bg=structure!10}
   \begin{block}{{\bf Exercice }#1}
}{%
   \end{block}%
}
\newenvironment{algo}[1][]{%
   \setbeamercolor{block title}{fg=structure,bg=structure!40}
   \setbeamercolor{block body}{fg=black,bg=structure!10}
   \begin{block}{{\bf Algorithme}\hfill{\color{gray}\texttt{#1}}}
}{%
   \end{block}%
}


\setbeamertemplate{proof begin}{
   \setbeamercolor{block title}{fg=black,bg=structure!20}
   \setbeamercolor{block body}{fg=black,bg=structure!5}
   \begin{block}{{\footnotesize Démonstration}}
   \footnotesize
   \smallskip}
\setbeamertemplate{proof end}{%
   \end{block}}
\setbeamertemplate{qed symbol}{\openbox}


\makeatother
\usecolortheme[RGB={102,51,102}]{structure}

%%%%%%%%%%%%%%%%%%%%%%%%%%%%%%%%%%%%%%%%%%%%%%%%%%%%%%%%%%%%%
%%%%%%%%%%%%%%%%%%%%%%%%%%%%%%%%%%%%%%%%%%%%%%%%%%%%%%%%%%%%%


\begin{document}


\title{{\bf Intégrales impropres}}
\subtitle{Définitions et premières propriétés}

\begin{frame}
  
  \debutmontitre

  \pause

{\footnotesize
\hfill
\setbeamercovered{transparent=50}
\begin{minipage}{0.6\textwidth}
  \begin{itemize}
    \item<3-> Points incertains
    \item<4-> Convergence/divergence
    \item<5-> Relation de Chasles et linéarité
    \item<6-> Positivité
    \item<7-> Critère de Cauchy
  \end{itemize}
\end{minipage}
}

\end{frame}

\setcounter{framenumber}{0}

%%%%%%%%%%%%%%%%%%%%%%%%%%%%%%%%%%%%%%%%%%%%%%%%%%%%%%%%%%%%%%%%
\section*{Points incertains}

\begin{frame}

\begin{itemize}
\item $t\mapsto f(t) = \frac{\sin |t|}{|t|^{3/2}}$ \quad sur  \quad
$]-\infty,0[ \ \cup\ ]0,+\infty[$

\myfigure{0.6}{
\tikzinput{fig_intimp01} 
} 

\pause

\item \evidence{Points incertains} : ici $-\infty$, $+\infty$ et $0$  

\end{itemize}

\pause
$$
\int_{-\infty}^{+\infty} f(t)\;\dd t
=
\int_{-\infty}^{-1} f(t)\;\dd t+
\int_{-1}^{0} f(t)\;\dd t+
\int_{0}^{1} f(t)\;\dd t+
\int_{1}^{+\infty} f(t)\;\dd t\;
$$


\end{frame}

\section*{Convergence/divergence}
\begin{frame}


Quatre types d'intervalle : $]-\infty,a]$, $]a,b]$, $[a,b[$ ou $[a,+\infty[$

\pause

\begin{mydefinition}
%\label{def:intcv}
\begin{itemize}
\item 
\begin{itemize}
  \item $f$ une fonction continue sur $[a,+\infty[$
\pause  
  \item $\int_a^{+\infty} f(t)\;\dd t$ \defi{converge} si $\lim_{x\rightarrow+\infty} 
\int_a^x f(t)\;\dd t$ existe et est finie

 \pause Si c'est le cas, on pose :
\mybox{$\displaystyle
%\label{intcv1}
\int_a^{+\infty} f(t)\;\dd t = \lim_{x\rightarrow+\infty} 
\int_a^x f(t)\;\dd t$}
\end{itemize}

\pause

\item 
\begin{itemize}
  \item $f$ une fonction continue sur $]a,b]$
\pause  
  \item $\int_a^b f(t)\;\dd t$ \defi{converge} si
 $\lim_{x\rightarrow a^+}  \int_x^b f(t)\;\dd t\;$ existe et est finie
 
 \pause
 
 \item Si c'est le cas, on pose :
\mybox{$\displaystyle
%\label{intcv2}
\int_a^{b} f(t)\;\dd t = \lim_{x\rightarrow a^+} 
\int_x^b f(t)\;\dd t$}
  \end{itemize}
\pause

\item Lorsque l'intégrale ne converge pas on dit qu'elle \defi{diverge}
\end{itemize}
\end{mydefinition}

\end{frame}



\section*{Exemples}


\begin{frame}
\begin{exemple}
$$\int_0^{+\infty} \frac{1}{1+t^2}\;\dd t \quad \text{ converge}$$
\pause
\begin{itemize}
  \item $\displaystyle \int_0^{x} \frac{1}{1+t^2}\;\dd t 
  \pause=\Big[\arctan t\Big]_0^x
  \pause=\arctan x$
  \pause

  \item $\displaystyle \lim_{x\rightarrow+\infty}\arctan x 
    = \frac{\pi}{2}$
  \pause
  
  \item $\displaystyle \int_0^{+\infty} \frac{1}{1+t^2}\;\dd t = \frac{\pi}{2}$
\end{itemize}
\pause
\myfigure{0.8}{
\tikzinput{fig_intimp07} 
}
\end{exemple}
\end{frame}


\begin{frame}

\begin{exemple}
\pause
$$
\int_0^1 \frac{1}{t}\;\dd t\qquad \text{ diverge}
$$
\pause
$$
\int_x^1 \frac{1}{t}\;\dd t 
\pause
= \Big[\ln t\Big]_x^1 
\pause
= -\ln x
\pause
\qquad \text{ et } \qquad 
\lim_{x\rightarrow 0^+} -\ln x = +\infty
$$  
\end{exemple}
\pause
\bigskip

\begin{exemple}
$$
\int_0^1 \ln t\;\dd t\qquad \text{ converge}
$$ 
\pause
$$
\int_x^1 \ln t\;\dd t 
\pause
= \Big[t\ln t-t\Big]_x^1 
\pause
= x-x\ln x-1
\pause
 \text{ et } 
\lim_{x\rightarrow 0^+} (x-x\ln x-1) = -1
$$
\end{exemple}
\end{frame}


\section*{Relation de Chasles}

\begin{frame}
\begin{proposition}[Relation de Chasles] 
\pause
Soit $f : [a,+\infty[ \to \Rr$ une fonction continue et soit
  $a' \in [a,+\infty[$. 
  Alors les intégrales impropres $\int_a^{+\infty} f(t) \;\dd t$ 
  et $\int_{a'}^{+\infty} f(t) \;\dd t $ sont de même nature. 
  \pause
  Si elles convergent, alors 
  \mybox{$\displaystyle \int_a^{+\infty} f(t) \;\dd t = \int_{a}^{a'} f(t) \;\dd t+ \int_{a'}^{+\infty} f(t) \;\dd t$}
\end{proposition}
\end{frame}


\section*{Linéarité}

\begin{frame}
\begin{proposition}[Linéarité de l'intégrale]
\label{prop:lineariteintegralescv}
Soient $f$ et $g$ deux fonctions continues sur
$[a,+\infty[$, et $\lambda,\mu$ deux réels. Si
les intégrales $\int_a^{+\infty} f(t)\;\dd t$ et 
$\int_a^{+\infty} g(t)\;\dd t$ convergent, alors 
$\int_a^{+\infty} \big(\lambda f(t)+\mu g(t)\big)\;\dd t$ converge et
\mybox{$\displaystyle\int_a^{+\infty} \big(\lambda f(t) + \mu g(t)\big)\;\dd t = \lambda \int_a^{+\infty}
f(t)\;\dd t +\mu \int_a^{+\infty} g(t)\;\dd t$}
\end{proposition}
\end{frame}


\section*{Positivité}

\begin{frame}
\begin{proposition}[Positivité de l'intégrale]
Soient $f,g : [a,+\infty[ \to \Rr$ des fonctions continues, ayant une intégrale convergente.
\mybox{$\displaystyle \text{ Si }\quad f \le g \quad \text{ alors }\quad
\int_a^{+\infty} f(t)\;\dd t  \le \int_a^{+\infty} g(t)\;\dd t$}
\end{proposition} \pause

\bigskip

En particulier, l'intégrale d'une fonction positive est positive :
\mybox{Si \quad $f\ge 0$ \quad alors \quad $\displaystyle \int_a^{+\infty} f(t)\;\dd t \ge 0$}

\end{frame}

\section*{Critère de Cauchy}

\begin{frame}

\evidence{Rappel} (\emph{Critère de Cauchy pour les limites de fonctions}) \\
Soit $f : [a,+\infty[ \to \Rr$. 
Alors $\lim_{x\to +\infty} f(x)$ existe et est finie  si et seulement si 
$$\forall \epsilon>0\quad\exists M \ge a \qquad 
\left( u,v \ge M \implies \big|f(u)-f(v)\big|<\epsilon\right)$$

\pause

 
\begin{theoreme}[Critère de Cauchy]
\label{th:intimpcauchy}
Soit $f : [a,+\infty[ \to \Rr$ une fonction continue.
L'intégrale impropre $\int_a^{+\infty} f(t) \; \dd t$ converge
si et seulement si
$$\forall \epsilon>0  \quad \exists  M \ge a \qquad
\left( u,v \ge M \implies \Bigl|\int_u^v f(t) \;\dd t\Bigr|<\epsilon\right)$$
\end{theoreme}

\pause

\begin{proof}
\begin{itemize}
  \item Critère de Cauchy pour $F(x)=\int_a^x f(t) \; \dd t$ 
\pause  
  \item $\displaystyle \big|F(u)-F(v)\big|=\big|\int_u^vf(t)\; \dd t\big|$ \qedhere
\end{itemize}
\end{proof}
\end{frame}


\section*{Cas de deux points incertains}

\begin{frame}

Soient $a,b\in \overline{\Rr}=\Rr\cup\{-\infty,+\infty\}$ avec $a<b$
et $f : \; ]a,b[ \to \Rr$ une fonction continue

\pause


\begin{mydefinition}
\begin{itemize}
  \item L'intégrale $\int_a^b f(t)\;\dd t$ \defi{converge} s'il existe
$c \in ]a,b[$ tel que les \evidence{deux} intégrales
impropres $\int_a^c f(t)\;\dd t$ et $\int_c^b f(t)\;\dd t$ convergent
\pause  
  \item La valeur de cette intégrale doublement impropre est alors
$$\int_a^b f(t)\;\dd t = \int_a^c f(t)\;\dd t+\int_c^b f(t)\;\dd t$$
\end{itemize}
\end{mydefinition} 
\end{frame}


\begin{frame}
\begin{exemple}
\begin{itemize}
\item $ \int_{-\infty}^{+\infty} \frac{1}{1+t^2}\;\dd t$ converge
\pause
\begin{itemize}
  \item En effet on a vu que $ \int_0^{+\infty} \frac{1}{1+t^2}\;\dd t$ converge
  et vaut $\frac{\pi}{2}$
 \pause 
  \item Par parité, il en est de même pour $\int_{-\infty}^0 \frac{1}{1+t^2}\;\dd t$
  \pause
  \item $ \int_{-\infty}^{+\infty} \frac{1}{1+t^2}\;\dd t = \pi$
\end{itemize}
\bigskip
\pause
\item $\int_{-\infty}^{+\infty}t\;\dd t$ diverge
\pause
\begin{itemize}
  \item $\int_{0}^{x} t\;\dd t = \frac{x^2}{2}$ tend vers $+\infty$ 
lorsque $x$ tend vers  $+\infty$
  \pause
  \item Donc $\int_0^{+\infty}t\;\dd t$ diverge
  \pause
  \item D'après la définition $\int_{-\infty}^{+\infty}t\;\dd t$ diverge
  \pause
  \item Bien que, pour tout $x$, on ait $\int_{-x}^{+x} t\;\dd t=0$
\end{itemize}
\end{itemize}
\end{exemple}
\end{frame}


%%%%%%%%%%%%%%%%%%%%%%%%%%%%%%%%%%%%%%%%%%%%%%%%%%%%%%%%%%%%%%%
 \section*{Mini-exercices}

\begin{frame}
\begin{miniexercice}
\begin{enumerate}
  \item Pour chacune des intégrales suivantes, déterminer le point incertain, dire
  si l'intégrale converge, et si c'est le cas, calculer la valeur de l'intégrale :
  $$
  \int_0^1 \frac{1}{\sqrt {1-t}} \;\dd t \quad
  \int_0^{+\infty}\cos t \;\dd t \quad
  \int_0^1 \frac{1}{1-t} \;\dd t \quad
  \int_{-\infty}^{\ln 2} e^t \;\dd t
  $$
  
  \item Même exercice pour ces intégrales ayant deux points incertains :
  $$
  \int_{-\infty}^{+\infty} \frac{\dd t}{1+t^2} \quad
  \int_{-\infty}^{1} \frac{\dd t}{(t-1)^2} \quad
  \int_{-\infty}^{+\infty} e^{-|t|}\;\dd t \quad
  \int_0^{+\infty} \frac1t \;\dd t$$
  
   \item \'Ecrire la preuve de la linéarité des intégrales impropres. Même chose 
   pour la positivité.  
\end{enumerate}
\end{miniexercice}
\end{frame}




\end{document}
