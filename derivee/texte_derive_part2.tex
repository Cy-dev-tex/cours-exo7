
%%%%%%%%%%%%%%%%%% PREAMBULE %%%%%%%%%%%%%%%%%%


\documentclass[12pt]{article}

\usepackage{amsfonts,amsmath,amssymb,amsthm}
\usepackage[utf8]{inputenc}
\usepackage[T1]{fontenc}
\usepackage[francais]{babel}


% packages
\usepackage{amsfonts,amsmath,amssymb,amsthm}
\usepackage[utf8]{inputenc}
\usepackage[T1]{fontenc}
%\usepackage{lmodern}

\usepackage[francais]{babel}
\usepackage{fancybox}
\usepackage{graphicx}

\usepackage{float}

%\usepackage[usenames, x11names]{xcolor}
\usepackage{tikz}
\usepackage{datetime}

\usepackage{mathptmx}
%\usepackage{fouriernc}
%\usepackage{newcent}
\usepackage[mathcal,mathbf]{euler}

%\usepackage{palatino}
%\usepackage{newcent}


% Commande spéciale prompteur

%\usepackage{mathptmx}
%\usepackage[mathcal,mathbf]{euler}
%\usepackage{mathpple,multido}

\usepackage[a4paper]{geometry}
\geometry{top=2cm, bottom=2cm, left=1cm, right=1cm, marginparsep=1cm}

\newcommand{\change}{{\color{red}\rule{\textwidth}{1mm}\\}}

\newcounter{mydiapo}

\newcommand{\diapo}{\newpage
\hfill {\normalsize  Diapo \themydiapo \quad \texttt{[\jobname]}} \\
\stepcounter{mydiapo}}


%%%%%%% COULEURS %%%%%%%%%%

% Pour blanc sur noir :
%\pagecolor[rgb]{0.5,0.5,0.5}
% \pagecolor[rgb]{0,0,0}
% \color[rgb]{1,1,1}



%\DeclareFixedFont{\myfont}{U}{cmss}{bx}{n}{18pt}
\newcommand{\debuttexte}{
%%%%%%%%%%%%% FONTES %%%%%%%%%%%%%
\renewcommand{\baselinestretch}{1.5}
\usefont{U}{cmss}{bx}{n}
\bfseries

% Taille normale : commenter le reste !
%Taille Arnaud
%\fontsize{19}{19}\selectfont

% Taille Barbara
%\fontsize{21}{22}\selectfont

%Taille François
%\fontsize{25}{30}\selectfont

%Taille Pascal
%\fontsize{25}{30}\selectfont

%Taille Laura
%\fontsize{30}{35}\selectfont


%\myfont
%\usefont{U}{cmss}{bx}{n}

%\Huge
%\addtolength{\parskip}{\baselineskip}
}


% \usepackage{hyperref}
% \hypersetup{colorlinks=true, linkcolor=blue, urlcolor=blue,
% pdftitle={Exo7 - Exercices de mathématiques}, pdfauthor={Exo7}}


%section
% \usepackage{sectsty}
% \allsectionsfont{\bf}
%\sectionfont{\color{Tomato3}\upshape\selectfont}
%\subsectionfont{\color{Tomato4}\upshape\selectfont}

%----- Ensembles : entiers, reels, complexes -----
\newcommand{\Nn}{\mathbb{N}} \newcommand{\N}{\mathbb{N}}
\newcommand{\Zz}{\mathbb{Z}} \newcommand{\Z}{\mathbb{Z}}
\newcommand{\Qq}{\mathbb{Q}} \newcommand{\Q}{\mathbb{Q}}
\newcommand{\Rr}{\mathbb{R}} \newcommand{\R}{\mathbb{R}}
\newcommand{\Cc}{\mathbb{C}} 
\newcommand{\Kk}{\mathbb{K}} \newcommand{\K}{\mathbb{K}}

%----- Modifications de symboles -----
\renewcommand{\epsilon}{\varepsilon}
\renewcommand{\Re}{\mathop{\text{Re}}\nolimits}
\renewcommand{\Im}{\mathop{\text{Im}}\nolimits}
%\newcommand{\llbracket}{\left[\kern-0.15em\left[}
%\newcommand{\rrbracket}{\right]\kern-0.15em\right]}

\renewcommand{\ge}{\geqslant}
\renewcommand{\geq}{\geqslant}
\renewcommand{\le}{\leqslant}
\renewcommand{\leq}{\leqslant}

%----- Fonctions usuelles -----
\newcommand{\ch}{\mathop{\mathrm{ch}}\nolimits}
\newcommand{\sh}{\mathop{\mathrm{sh}}\nolimits}
\renewcommand{\tanh}{\mathop{\mathrm{th}}\nolimits}
\newcommand{\cotan}{\mathop{\mathrm{cotan}}\nolimits}
\newcommand{\Arcsin}{\mathop{\mathrm{Arcsin}}\nolimits}
\newcommand{\Arccos}{\mathop{\mathrm{Arccos}}\nolimits}
\newcommand{\Arctan}{\mathop{\mathrm{Arctan}}\nolimits}
\newcommand{\Argsh}{\mathop{\mathrm{Argsh}}\nolimits}
\newcommand{\Argch}{\mathop{\mathrm{Argch}}\nolimits}
\newcommand{\Argth}{\mathop{\mathrm{Argth}}\nolimits}
\newcommand{\pgcd}{\mathop{\mathrm{pgcd}}\nolimits} 

\newcommand{\Card}{\mathop{\text{Card}}\nolimits}
\newcommand{\Ker}{\mathop{\text{Ker}}\nolimits}
\newcommand{\id}{\mathop{\text{id}}\nolimits}
\newcommand{\ii}{\mathrm{i}}
\newcommand{\dd}{\mathrm{d}}
\newcommand{\Vect}{\mathop{\text{Vect}}\nolimits}
\newcommand{\Mat}{\mathop{\mathrm{Mat}}\nolimits}
\newcommand{\rg}{\mathop{\text{rg}}\nolimits}
\newcommand{\tr}{\mathop{\text{tr}}\nolimits}
\newcommand{\ppcm}{\mathop{\text{ppcm}}\nolimits}

%----- Structure des exercices ------

\newtheoremstyle{styleexo}% name
{2ex}% Space above
{3ex}% Space below
{}% Body font
{}% Indent amount 1
{\bfseries} % Theorem head font
{}% Punctuation after theorem head
{\newline}% Space after theorem head 2
{}% Theorem head spec (can be left empty, meaning ‘normal’)

%\theoremstyle{styleexo}
\newtheorem{exo}{Exercice}
\newtheorem{ind}{Indications}
\newtheorem{cor}{Correction}


\newcommand{\exercice}[1]{} \newcommand{\finexercice}{}
%\newcommand{\exercice}[1]{{\tiny\texttt{#1}}\vspace{-2ex}} % pour afficher le numero absolu, l'auteur...
\newcommand{\enonce}{\begin{exo}} \newcommand{\finenonce}{\end{exo}}
\newcommand{\indication}{\begin{ind}} \newcommand{\finindication}{\end{ind}}
\newcommand{\correction}{\begin{cor}} \newcommand{\fincorrection}{\end{cor}}

\newcommand{\noindication}{\stepcounter{ind}}
\newcommand{\nocorrection}{\stepcounter{cor}}

\newcommand{\fiche}[1]{} \newcommand{\finfiche}{}
\newcommand{\titre}[1]{\centerline{\large \bf #1}}
\newcommand{\addcommand}[1]{}
\newcommand{\video}[1]{}

% Marge
\newcommand{\mymargin}[1]{\marginpar{{\small #1}}}



%----- Presentation ------
\setlength{\parindent}{0cm}

%\newcommand{\ExoSept}{\href{http://exo7.emath.fr}{\textbf{\textsf{Exo7}}}}

\definecolor{myred}{rgb}{0.93,0.26,0}
\definecolor{myorange}{rgb}{0.97,0.58,0}
\definecolor{myyellow}{rgb}{1,0.86,0}

\newcommand{\LogoExoSept}[1]{  % input : echelle
{\usefont{U}{cmss}{bx}{n}
\begin{tikzpicture}[scale=0.1*#1,transform shape]
  \fill[color=myorange] (0,0)--(4,0)--(4,-4)--(0,-4)--cycle;
  \fill[color=myred] (0,0)--(0,3)--(-3,3)--(-3,0)--cycle;
  \fill[color=myyellow] (4,0)--(7,4)--(3,7)--(0,3)--cycle;
  \node[scale=5] at (3.5,3.5) {Exo7};
\end{tikzpicture}}
}



\theoremstyle{definition}
%\newtheorem{proposition}{Proposition}
%\newtheorem{exemple}{Exemple}
%\newtheorem{theoreme}{Théorème}
\newtheorem{lemme}{Lemme}
\newtheorem{corollaire}{Corollaire}
%\newtheorem*{remarque*}{Remarque}
%\newtheorem*{miniexercice}{Mini-exercices}
%\newtheorem{definition}{Définition}




%definition d'un terme
\newcommand{\defi}[1]{{\color{myorange}\textbf{\emph{#1}}}}
\newcommand{\evidence}[1]{{\color{blue}\textbf{\emph{#1}}}}



 %----- Commandes divers ------

\newcommand{\codeinline}[1]{\texttt{#1}}

%%%%%%%%%%%%%%%%%%%%%%%%%%%%%%%%%%%%%%%%%%%%%%%%%%%%%%%%%%%%%
%%%%%%%%%%%%%%%%%%%%%%%%%%%%%%%%%%%%%%%%%%%%%%%%%%%%%%%%%%%%%



\begin{document}

\debuttexte

%%%%%%%%%%%%%%%%%%%%%%%%%%%%%%%%%%%%%%%%%%%%%%%%%%%%%%%%%%%
\diapo

\change

Voici une partie beaucoup plus pratique : nous allons calculer
des dérivées.  

\change

On commence par les formules pour la somme et le produit

\change

On continue avec une liste de dérivées des fonctions les plus fréquentes.

\change

On passe ensuite à la formule importante pour la dérivée de la composition de deux fonctions.

\change

On termine avec les dérivées secondes, dérivées troisièmes,... et la formule de Leibniz.


%%%%%%%%%%%%%%%%%%%%%%%%%%%%%%%%%%%%%%%%%%%%%%%%%%%%%%%%%%%
\diapo
Commençons avec deux fonctions dérivables $f$ et $g$ définies sur un intervalle $I$.

\change
La fonction $f+g$ est dérivable

et pour tout $x \in I$ :
$(f+g)'(x) = f'(x)+g'(x)$,

C'est à dire que la dérivée d'une somme est la somme des dérivées.

\change

Pour $\lambda$ un réel fixé  $(\lambda f)'(x) = \lambda f'(x)$

\change

Voici la formule pour la dérivée d'un produit :
$(f \times g)'(x) = f'(x)g(x)+f(x)g'(x)$,


\change

Si $f$ ne s'annule pas  alors $1/f$ est dérivable et
  $\left(\frac{1}{f}\right)'(x)=-\frac{f'(x)}{f(x)^2}$

\change 

Et pour le quotient la formule est 
$\left(\frac{f}{g}\right)'(x)=\frac{f'(x)g(x)-f(x)g'(x)}{g(x)^2}$


%%%%%%%%%%%%%%%%%%%%%%%%%%%%%%%%%%%%%%%%%%%%%%%%%%%%%%%%%%%
\diapo

Il est plus facile de mémoriser ces égalités comme égalités de fonctions :

$$(f+g)'=f'+g',\quad  (\lambda f)' = \lambda f', \quad (f \times g)' = f'g+fg', $$

\change

et aussi 
$\left(\frac{1}{f}\right)'=-\frac{f'}{f^2}$

\change

et enfin

$\left(\frac{f}{g}\right)'=\frac{f'g-fg'}{g^2}.$

\change

Prouvons par exemple la formule de la dérivée d'un produit : $(f \times g)' = f'g+fg'$.

\change

Fixons $x_0 \in I$.

\change

Partons du taux taux d'accroissement de $f(x)\times g(x)$ en $x_0$ :

$\frac{f(x)g(x)-f(x_0)g(x_0)}{x-x_0}$


\change

que l'on écrit sous la forme 

$\frac{f(x)-f(x_0)}{x-x_0} g(x)+\frac{g(x)-g(x_0)}{x-x_0}f(x_0) $

\change

pour le premier terme on reconnaît le taux d'accroissement de $f$ en $x_0$ donc ce premier terme tend vers $f'(x_0)g(x_0)$ 

de même le second terme tend vers $g'(x_0)f(x_0).$

\change 

Ceci est vrai pour tout $x_0$ la fonction $f\times g$ est dérivable sur $I$ de dérivée $f'g+fg'$.


%%%%%%%%%%%%%%%%%%%%%%%%%%%%%%%%%%%%%%%%%%%%%%%%%%%%%%%%%%%
\diapo


Voici un tableau qui résume les principales formules à connaître.


\setlength{\arrayrulewidth}{0.05mm}
%\begin{tabular}{|l|l|l|} \hline
\begin{tabular}[t]{|c|c@{\vrule depth 1.2ex height 3ex width 0mm \ }|} 
\hline
\textbf{Fonction}         & \textbf{Dérivée} \\ \hline
   $x^n$         & $nx^{n-1}$  \quad ($n \in \Zz$)   \\ \hline
   $\frac 1x$    & $-\frac{1}{x^2}$              \\ \hline
   $\sqrt{x}$    & $\frac12 \frac1{\sqrt{x}}$   \\ \hline
   $x^\alpha$   & $\alpha x^{\alpha-1}$  \quad ($\alpha\in\Rr$)  \\ \hline
   $e^x$         & $e^x$                        \\ \hline
   $\ln x$       & $\frac 1x$                   \\ \hline
   $\cos x$      & $-\sin x$                    \\ \hline
   $\sin x$      & $\cos x$                     \\ \hline
   $\tan x$      & $1+\tan^2 x = \frac{1}{\cos^2 x}$        \\ \hline
\end{tabular} 


%%%%%%%%%%%%%%%%%%%%%%%%%%%%%%%%%%%%%%%%%%%%%%%%%%%%%%%%%%%
\diapo

Passons à une formule importante qui mérite toute votre attention : la dérivée d'une composition.

Soit $f$ est dérivable en $x$ et $g$ une fonction dérivable en $f(x)$ 
alors tout d'abord  $g\circ f$ est
dérivable en $x$

\change
 et sa dérivée est

$\big( g \circ f \big)'(x) = g'\big( f(x) \big) \cdot f'(x)$


\change

Appliquons cette formule pour calculer la dérivée de $\ln(1+x^2)$.

\change

Ici la fonction $f$ est  $f(x)=1+x^2$ avec  $f'(x) = 2x$.

\change

Et $g(x)=\ln(x)$ donc  $g'(x) = \frac 1x$ ; 

\change

On a bien $\ln(1+x^2)=g\circ f(x)$

\change

Et par la formule sa dérivée est

$\big( g \circ f \big)'(x) = g'\big( f(x) \big) \cdot f'(x)$

\change

On remplace $f$ par $1+x^2$ et $f'$ par $2x$

Maintenant  $g'(u) = \frac 1u$ [montrer au-dessus]

\change

on l'applique en $u=1+x^2$ et on obtient que 
la dérivée de $\ln(1+x^2)$ est $\frac{2x}{1+x^2}.$

%%%%%%%%%%%%%%%%%%%%%%%%%%%%%%%%%%%%%%%%%%%%%%%%%%%%%%%%%%%
\diapo

Ce tableau reprend les dérivées des fonctions classiques composées avec une autre fonction $u$ (qui dépend d'une variable $x$)

Ces formules découlent de la formule de la dérivée d'une composition.


\begin{tabular}[t]{cc@{\vrule depth 1.2ex height 3ex width 0mm \ }} 
\textbf{Fonction}         & \textbf{Dérivée} \\ \hline
   $u^n$         & $nu'u^{n-1}$  \quad  ($n \in \Zz$)   \\ \hline
   $\frac 1u$    & $-\frac{u'}{u^2}$              \\ \hline
   $\sqrt{u}$    & $\frac12 \frac{u'}{\sqrt{u}}$   \\ \hline
   $u^\alpha$   & $\alpha u' u^{\alpha-1}$ \quad ($\alpha\in\Rr$)  \\ \hline
   $e^u$         & $u'e^u$                        \\ \hline
   $\ln u$       & $\frac {u'}{u}$                   \\ \hline
   $\cos u$      & $-u'\sin u$                    \\ \hline
   $\sin u$      & $u'\cos u$                     \\ \hline
   $\tan u$      & $u'(1+\tan^2 u) = \frac{u'}{\cos^2 u}$        \\ \hline
\end{tabular} 




%%%%%%%%%%%%%%%%%%%%%%%%%%%%%%%%%%%%%%%%%%%%%%%%%%%%%%%%%%%
\diapo

Quelques remarques sur toutes ces formules.

Tout d'abord les formules pour $x^n$, $\frac 1x$ $\sqrt x$ et $x^\alpha$ 
sont aussi des conséquences de la dérivée de l'exponentielle.

\change

Par exemple calculons la dérivée de $x^\alpha$

\change

on passe à la forme exponentielle $x^\alpha = e^{\alpha \ln x}$ 

\change

et donc la dérivée de $x^\alpha$

\change

égale la dérivée de $e^{\alpha \ln x}$

\change

On applique la formule de la dérivée d'une composition :
la dérivée de $e^u$ est $u'e^u$.

Donc la dérivée de $x^\alpha$
est $\alpha \frac{1}{x} e^{\alpha \ln x} $

\change

ce qui vaut 
$=\alpha \frac 1x x^{\alpha}$

\change

et ainsi on obtient que la dérivée de $x^\alpha$ est $\alpha x^{\alpha-1}.$

(pause)


\change


Il y a un piège à éviter , 

dès qu'une fonction a un exposant qui varie  il faut absolument 
repasser à la forme exponentielle.

\change


Par exemple pour calculer la dérivée de $2^x$,

ici l'exposant n'est pas une constante c'est $x$ .

\change

on réécrit d'abord $2^x=e^{x\ln 2}$ 

\change

C'est encore une fois une fonction du type $e^u$ de dérivée $u'e^u$.

Et donc $f'(x)=\ln 2 \times e^{x\ln 2}$


\change

$ = \ln 2 \times 2^x $.

Ce serait une grave erreur en partant de $2^x$ de vouloir 
appliquer la formule de la dérivée de $x^\alpha$.

%%%%%%%%%%%%%%%%%%%%%%%%%%%%%%%%%%%%%%%%%%%%%%%%%%%%%%%%%%%
\diapo


Soit $I$ un intervalle ouvert. Soit $f : I \to J$ une fonction dérivable et bijective 

dont on note $f^{-1} : J \to I$ la bijection réciproque. 

\change

Si $f'$ ne s'annule pas sur $I$ alors d'une part 
$f^{-1}$ est dérivable 

\change

et d'autre part pour tout $x \in J$ :

$\big(f^{-1}\big)'(x)= \frac{1}{f'\big( f^{-1}(x) \big)} $

\change


Il peut être plus simple de retrouver la formule à chaque fois

Notons $g=f^{-1}$ la bijection réciproque de $f$.

\change

Que $g$ soit la bijection réciproque de $f$ se traduit par l'égalité :

$f\big( g(x) \big)  = x$

\change

On dérive cette égalité pour obtenir ceci 

En effet à droite la dérivée de $x$ est $1$ ;
à gauche la dérivée de $f\big( g(x) \big) = f \circ g(x)$ est $f'\big(g(x)\big) \cdot g'(x)$.

\change

On se souvient que $g=f^{-1}$ donc on obtient la formule voulue.


%%%%%%%%%%%%%%%%%%%%%%%%%%%%%%%%%%%%%%%%%%%%%%%%%%%%%%%%%%%
\diapo

\change

\'Etudions en détail la fonction $f$ définie par $f(x)=x+\exp(x)$.

\change

Tout d'abord $f$ est dérivable 
car $f$ est la somme de deux fonctions dérivables. En particulier $f$ est continue.

\change

$f$ est strictement croissante car $f$ est la somme de deux fonctions strictement croissante.

\change

Comme $\lim_{x\to-\infty} f(x)=-\infty$ et $\lim_{x\to+\infty} f(x)=+\infty$ 
et que $f$ est continue strictement croissante alors $f$ est une bijection de $\Rr$ dans $\Rr$.

\change

Enfin $f'(x)$ vaut $1 + \exp(x)$ et l'exponentielle est toujours positive alors $f'(x)$ ne s'annule jamais.

\change



On a vu que $f$ était bijective on note $g = f^{-1}$ la bijection réciproque de $f$. 

\change

Même si on ne sait pas calculer $g$, on peut malgré tout connaître des informations sur 
cette fonction : par le corollaire de la page précédente $g$ est dérivable



On calcule $g'$ en dérivant l'égalité $f\big( g(x) \big)  = x$. 

\change

Ce qui donne 
$f'\big(g(x)\big) \cdot g'(x) = 1$

\change

et donc ici 
$g'(x) = \frac{1}{f'\big( g(x) \big)} = \frac{1}{1+\exp\big( g(x) \big)}.$

\change

Pour cette fonction $f$ particulière on peut préciser davantage :
comme  $f\big( g(x)\big) = x$ alors $g(x)+\exp\big( g(x) \big)=x$ donc 
$\exp\big( g(x) \big)=x-g(x)$. 

\change

Cela conduit à :
$g'(x) =  \frac{1}{1+x -g(x)}.$

\change

Par exemple  $f(0)=1$ donc $g(1)=0$ et donc $g'(1)=\frac12$.

\change

Autrement dit l'équation de la tangente 
au graphe de $f^{-1}$ au point d'abscisse $x_0=1$ est donc $y=\frac12 (x-1)$.


%%%%%%%%%%%%%%%%%%%%%%%%%%%%%%%%%%%%%%%%%%%%%%%%%%%%%%%%%%%
\diapo

Voici le graphe de $f$. C'est bien une fonction strictement croissante de$-\infty$ à $+\infty$.

\change

Le graphe de la fonction réciproque s'obtient --dans un repère orthonormé-- par la symétrie 
par rapport à la droite d'équation $(y=x)$.

\change

Et nous avons calculer l'équation de la tangente en $1$. 



%%%%%%%%%%%%%%%%%%%%%%%%%%%%%%%%%%%%%%%%%%%%%%%%%%%%%%%%%%%
\diapo


Soit $f : I \to \Rr$ une fonction dérivable et soit $f'$ sa dérivée.

\change

Si la fonction $f' : I \to \Rr$ est elle même une fonction dérivable alors on note
$f''$ la dérivée de la dérivée que l'on nomme  la \defi{dérivée seconde} de $f$. 

\change


Plus généralement on note :

$f^{(0)} = f, \quad f^{(1)} = f', \quad f^{(2)} = f''$

et par récurrence si  la dérivée d'ordre $n$ existe et est dérivable alors
on appelle sa dérivée la dérivée d'ordre $n+1$ de $f$.

\change


Lorsque la \defi{dérivée $n$-ième} $f^{(n)}$ existe 
 on dira que $f$ est \defi{$n$ fois dérivable}.

\change


Il existe une formule intéressante pour la dérivée $n$-ième d'un produit :
c'est la formule de Leibniz :

Si l'on a $f$ et $g$ deux fonctions $n$ fois dérivables alors 


$ \big( f \cdot g \big)^{(n)} =  f^{(n)} \cdot g + \binom{n}{1}\ f^{(n-1)}\cdot g^{(1)}
+ \cdots + \binom{n}{k} \ f^{(n-k)} \cdot g^{(k)}+\cdots + f \cdot g^{(n)}$

\change

Autrement dit à l'aide des sommes formelles :

$\big( f \cdot g \big)^{(n)} = \sum_{k=0}^n \binom{n}{k} \ f^{(n-k)} \cdot g^{(k)}.$

La démonstration est similaire à celle de la formule du binôme de Newton
et les coefficients que l'on obtient sont les coefficients du binôme de Newton.

\change

Pour $n=1$ on retrouve la formule de la dérivée d'un produit $(f\cdot g)'= f' g + f g'$.

\change

Pour $n=2$, on obtient $(f\cdot g)''= f''g + 2f' g' + fg''$.


%%%%%%%%%%%%%%%%%%%%%%%%%%%%%%%%%%%%%%%%%%%%%%%%%%%%%%%%%%%
\diapo

Mettons en pratique la formule de Leibniz pour calculer les dérivées de $\exp(x) \cdot (x^2+1)$

\change

On commence par le premier facteur : $f(x)=\exp x$ 

c'est facile, la dérivée est l'exponentielle elle-même !
 $f'(x)=\exp x$ 

Donc aussi :  $f''(x)=\exp x$ et plus généralement $f^{(k)}(x)=\exp x$.

\change

Passons au deuxième facteur $g(x)=x^2+1$ sa dérivée est  $g'(x)=2x$,


sa dérivée seconde $g''(x)=2$ 

et toute dérivée d'ordre supérieure à $2$ sera nulle.

\change

Ecrivons la formule de Leibniz :

[grause pause]


\change 


On remplace $f$ et toutes ses dérivées par $\exp x$

Et on remplace $g$, $g'$ et $g''$. La somme s'arrête là car après les dérivées sont nuls.



On se retrouve donc avec une somme qui n'a plus que trois termes.

\change

On peut factoriser par l'exponentielle pour obtenir une expression plus condensée :

$\big( f \cdot g \big)^{(n)}(x) =  \exp(x) \cdot \Big(x^2 + 2nx + \frac{n(n-1)}{2}+1  \Big)$


%%%%%%%%%%%%%%%%%%%%%%%%%%%%%%%%%%%%%%%%%%%%%%%%%%%%%%%%%%%
\diapo

Je vous donne la recette pour réussir en maths :
Entraînez-vous, relisez le cours, et travaillez encore et encore !


\end{document}