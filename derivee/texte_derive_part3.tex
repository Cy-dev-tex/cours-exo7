
%%%%%%%%%%%%%%%%%% PREAMBULE %%%%%%%%%%%%%%%%%%


\documentclass[12pt]{article}

\usepackage{amsfonts,amsmath,amssymb,amsthm}
\usepackage[utf8]{inputenc}
\usepackage[T1]{fontenc}
\usepackage[francais]{babel}


% packages
\usepackage{amsfonts,amsmath,amssymb,amsthm}
\usepackage[utf8]{inputenc}
\usepackage[T1]{fontenc}
%\usepackage{lmodern}

\usepackage[francais]{babel}
\usepackage{fancybox}
\usepackage{graphicx}

\usepackage{float}

%\usepackage[usenames, x11names]{xcolor}
\usepackage{tikz}
\usepackage{datetime}

\usepackage{mathptmx}
%\usepackage{fouriernc}
%\usepackage{newcent}
\usepackage[mathcal,mathbf]{euler}

%\usepackage{palatino}
%\usepackage{newcent}


% Commande spéciale prompteur

%\usepackage{mathptmx}
%\usepackage[mathcal,mathbf]{euler}
%\usepackage{mathpple,multido}

\usepackage[a4paper]{geometry}
\geometry{top=2cm, bottom=2cm, left=1cm, right=1cm, marginparsep=1cm}

\newcommand{\change}{{\color{red}\rule{\textwidth}{1mm}\\}}

\newcounter{mydiapo}

\newcommand{\diapo}{\newpage
\hfill {\normalsize  Diapo \themydiapo \quad \texttt{[\jobname]}} \\
\stepcounter{mydiapo}}


%%%%%%% COULEURS %%%%%%%%%%

% Pour blanc sur noir :
%\pagecolor[rgb]{0.5,0.5,0.5}
% \pagecolor[rgb]{0,0,0}
% \color[rgb]{1,1,1}



%\DeclareFixedFont{\myfont}{U}{cmss}{bx}{n}{18pt}
\newcommand{\debuttexte}{
%%%%%%%%%%%%% FONTES %%%%%%%%%%%%%
\renewcommand{\baselinestretch}{1.5}
\usefont{U}{cmss}{bx}{n}
\bfseries

% Taille normale : commenter le reste !
%Taille Arnaud
%\fontsize{19}{19}\selectfont

% Taille Barbara
%\fontsize{21}{22}\selectfont

%Taille François
%\fontsize{25}{30}\selectfont

%Taille Pascal
%\fontsize{25}{30}\selectfont

%Taille Laura
%\fontsize{30}{35}\selectfont


%\myfont
%\usefont{U}{cmss}{bx}{n}

%\Huge
%\addtolength{\parskip}{\baselineskip}
}


% \usepackage{hyperref}
% \hypersetup{colorlinks=true, linkcolor=blue, urlcolor=blue,
% pdftitle={Exo7 - Exercices de mathématiques}, pdfauthor={Exo7}}


%section
% \usepackage{sectsty}
% \allsectionsfont{\bf}
%\sectionfont{\color{Tomato3}\upshape\selectfont}
%\subsectionfont{\color{Tomato4}\upshape\selectfont}

%----- Ensembles : entiers, reels, complexes -----
\newcommand{\Nn}{\mathbb{N}} \newcommand{\N}{\mathbb{N}}
\newcommand{\Zz}{\mathbb{Z}} \newcommand{\Z}{\mathbb{Z}}
\newcommand{\Qq}{\mathbb{Q}} \newcommand{\Q}{\mathbb{Q}}
\newcommand{\Rr}{\mathbb{R}} \newcommand{\R}{\mathbb{R}}
\newcommand{\Cc}{\mathbb{C}} 
\newcommand{\Kk}{\mathbb{K}} \newcommand{\K}{\mathbb{K}}

%----- Modifications de symboles -----
\renewcommand{\epsilon}{\varepsilon}
\renewcommand{\Re}{\mathop{\text{Re}}\nolimits}
\renewcommand{\Im}{\mathop{\text{Im}}\nolimits}
%\newcommand{\llbracket}{\left[\kern-0.15em\left[}
%\newcommand{\rrbracket}{\right]\kern-0.15em\right]}

\renewcommand{\ge}{\geqslant}
\renewcommand{\geq}{\geqslant}
\renewcommand{\le}{\leqslant}
\renewcommand{\leq}{\leqslant}

%----- Fonctions usuelles -----
\newcommand{\ch}{\mathop{\mathrm{ch}}\nolimits}
\newcommand{\sh}{\mathop{\mathrm{sh}}\nolimits}
\renewcommand{\tanh}{\mathop{\mathrm{th}}\nolimits}
\newcommand{\cotan}{\mathop{\mathrm{cotan}}\nolimits}
\newcommand{\Arcsin}{\mathop{\mathrm{Arcsin}}\nolimits}
\newcommand{\Arccos}{\mathop{\mathrm{Arccos}}\nolimits}
\newcommand{\Arctan}{\mathop{\mathrm{Arctan}}\nolimits}
\newcommand{\Argsh}{\mathop{\mathrm{Argsh}}\nolimits}
\newcommand{\Argch}{\mathop{\mathrm{Argch}}\nolimits}
\newcommand{\Argth}{\mathop{\mathrm{Argth}}\nolimits}
\newcommand{\pgcd}{\mathop{\mathrm{pgcd}}\nolimits} 

\newcommand{\Card}{\mathop{\text{Card}}\nolimits}
\newcommand{\Ker}{\mathop{\text{Ker}}\nolimits}
\newcommand{\id}{\mathop{\text{id}}\nolimits}
\newcommand{\ii}{\mathrm{i}}
\newcommand{\dd}{\mathrm{d}}
\newcommand{\Vect}{\mathop{\text{Vect}}\nolimits}
\newcommand{\Mat}{\mathop{\mathrm{Mat}}\nolimits}
\newcommand{\rg}{\mathop{\text{rg}}\nolimits}
\newcommand{\tr}{\mathop{\text{tr}}\nolimits}
\newcommand{\ppcm}{\mathop{\text{ppcm}}\nolimits}

%----- Structure des exercices ------

\newtheoremstyle{styleexo}% name
{2ex}% Space above
{3ex}% Space below
{}% Body font
{}% Indent amount 1
{\bfseries} % Theorem head font
{}% Punctuation after theorem head
{\newline}% Space after theorem head 2
{}% Theorem head spec (can be left empty, meaning ‘normal’)

%\theoremstyle{styleexo}
\newtheorem{exo}{Exercice}
\newtheorem{ind}{Indications}
\newtheorem{cor}{Correction}


\newcommand{\exercice}[1]{} \newcommand{\finexercice}{}
%\newcommand{\exercice}[1]{{\tiny\texttt{#1}}\vspace{-2ex}} % pour afficher le numero absolu, l'auteur...
\newcommand{\enonce}{\begin{exo}} \newcommand{\finenonce}{\end{exo}}
\newcommand{\indication}{\begin{ind}} \newcommand{\finindication}{\end{ind}}
\newcommand{\correction}{\begin{cor}} \newcommand{\fincorrection}{\end{cor}}

\newcommand{\noindication}{\stepcounter{ind}}
\newcommand{\nocorrection}{\stepcounter{cor}}

\newcommand{\fiche}[1]{} \newcommand{\finfiche}{}
\newcommand{\titre}[1]{\centerline{\large \bf #1}}
\newcommand{\addcommand}[1]{}
\newcommand{\video}[1]{}

% Marge
\newcommand{\mymargin}[1]{\marginpar{{\small #1}}}



%----- Presentation ------
\setlength{\parindent}{0cm}

%\newcommand{\ExoSept}{\href{http://exo7.emath.fr}{\textbf{\textsf{Exo7}}}}

\definecolor{myred}{rgb}{0.93,0.26,0}
\definecolor{myorange}{rgb}{0.97,0.58,0}
\definecolor{myyellow}{rgb}{1,0.86,0}

\newcommand{\LogoExoSept}[1]{  % input : echelle
{\usefont{U}{cmss}{bx}{n}
\begin{tikzpicture}[scale=0.1*#1,transform shape]
  \fill[color=myorange] (0,0)--(4,0)--(4,-4)--(0,-4)--cycle;
  \fill[color=myred] (0,0)--(0,3)--(-3,3)--(-3,0)--cycle;
  \fill[color=myyellow] (4,0)--(7,4)--(3,7)--(0,3)--cycle;
  \node[scale=5] at (3.5,3.5) {Exo7};
\end{tikzpicture}}
}



\theoremstyle{definition}
%\newtheorem{proposition}{Proposition}
%\newtheorem{exemple}{Exemple}
%\newtheorem{theoreme}{Théorème}
\newtheorem{lemme}{Lemme}
\newtheorem{corollaire}{Corollaire}
%\newtheorem*{remarque*}{Remarque}
%\newtheorem*{miniexercice}{Mini-exercices}
%\newtheorem{definition}{Définition}




%definition d'un terme
\newcommand{\defi}[1]{{\color{myorange}\textbf{\emph{#1}}}}
\newcommand{\evidence}[1]{{\color{blue}\textbf{\emph{#1}}}}



 %----- Commandes divers ------

\newcommand{\codeinline}[1]{\texttt{#1}}

%%%%%%%%%%%%%%%%%%%%%%%%%%%%%%%%%%%%%%%%%%%%%%%%%%%%%%%%%%%%%
%%%%%%%%%%%%%%%%%%%%%%%%%%%%%%%%%%%%%%%%%%%%%%%%%%%%%%%%%%%%%



\begin{document}

\debuttexte

%%%%%%%%%%%%%%%%%%%%%%%%%%%%%%%%%%%%%%%%%%%%%%%%%%%%%%%%%%%
\diapo


\change

Nous allons aborder deux notions dans cette leçon :

\change

tout d'abord la notion de maximum ou de minimum local,
puis nous ferons le lien avec les points critiques (un point où la dérivée s'annule).

\change

Nous appliquons ceci pour prouver un résultat important : le théorème de Rolle
qui sous certaines conditions garantie l'existence d'un point critique.


%%%%%%%%%%%%%%%%%%%%%%%%%%%%%%%%%%%%%%%%%%%%%%%%%%%%%%%%%%%
\diapo

Commençons par des définitions.

Soit $f : I \to \Rr$ une fonction définie sur un intervalle $I$.

Tout d'abord on dit que $x_0$ est un \defi{point critique} de $f$ si $f'(x_0)=0$.

Géométriquement ce sont les points où la tangente est horizontale.

\change

Ensuite on dira que $f$ admet un \defi{maximum local en $x_0$}  
s'il existe un intervalle ouvert $J$ contenant $x_0$  tel que 
$$\text{pour tout } x\in I \cap J \quad f(x) \le f(x_0)$$

\change

Dire que $f$ a un maximum local en $x_0$ signifie que $f(x_0)$ 
est la plus grande des valeurs $f(x)$ pour les $x$ proches de $x_0$.

\change

Bien sûr en changeant le sens de l'inégalité on définit ce qu'est un minimum local.

\change

Les voici, notez que si une des extrémités de l'intervalle est fermé alors un minimum ou un maximum local
peut être au bord.

\change

On regroupe ces deux notions en une seule : un \defi{extremum local} est un maximum 
local ou bien un minimum local.

\change

Il ne faut pas confondre un maximum  local avec un maximum global.

Par définition on a un \defi{maximum global} en $x_0$ 
si pour toutes les autres valeurs $f(x)$, $x\in I$
on a $f(x_0)$ est plus grand que $f(x)$

\change

Bien sûr un maximum global est aussi un maximum local, mais la réciproque est fausse.

%%%%%%%%%%%%%%%%%%%%%%%%%%%%%%%%%%%%%%%%%%%%%%%%%%%%%%%%%%%
\diapo


Il y a un lien entre un point critique (un point où la dérivée s'annule)
et un point où $f$ admet un extremum local.

Soit $f$ une fonction défini sur un intervalle *ouvert*.



Voici le résultat : 
 Si $f$ admet un maximum local (ou un minimum local)
en $x_0$ alors $f'(x_0)=0$. 

En d'autres termes, un extremum local  est toujours un point critique. 

\change


Cela se voit sur le dessin : pour ces deux maximums locaux la tangente est bien horizontale,
même chose pour ce minimum local.




%%%%%%%%%%%%%%%%%%%%%%%%%%%%%%%%%%%%%%%%%%%%%%%%%%%%%%%%%%%
\diapo

\'Etudions les extremums de la fonction $f_\lambda$ définie par
$f_\lambda(x)= x^3+\lambda x$ en fonction du paramètre $\lambda \in \Rr$.

\change


La dérivée est $f_\lambda'(x) = 3x^2+\lambda$.

\change

On sait que si $x_0$ est un extremum local
alors $f'_\lambda(x_0)=0$.

\change

Premier cas : $\lambda>0$.

Alors la dérivée est partout strictement positive et ne s'annule donc jamais 

\change

en conséquence il n'y a pas de points critiques
donc pas non plus d'extremums. 

\change

En anticipant sur la suite : $f_\lambda$ est strictement croissante sur $\Rr$.
Et on voit bien sur le graphe qu'il n'y a effectivement ni maximum ni minimum.

\change

Deuxième cas $\lambda = 0$ 

alors la dérivée vaut $3x^2$. 

\change 

Le seul point critique est $x_0=0$.

\change


Mais attention ce n'est ni un maximum local, ni un minimum local. 

\change

En effet si $x<0$, la fonction est plus petite que $0$ et si 
 $x>0$ la fonction est plus grande que $0$

\change

Cela se voit sur le graphe : en $0$ il y a bien une 
tangente horizontale mais ce n'est ni un maximum ni un minimum.



%%%%%%%%%%%%%%%%%%%%%%%%%%%%%%%%%%%%%%%%%%%%%%%%%%%%%%%%%%%
\diapo

On continue l'étude des extremums de la fonction $f_\lambda(x)= x^3+\lambda x$

\change

Il nous reste le cas où $\lambda <0$ 

\change

Alors la dérivée est $f'_\lambda$ vaut $3x^2+\lambda$ que j'écris $3x^2-|\lambda|$ car $\lambda$ négatif.

\change

On peut factoriser sous la forme $3\big(x-\sqrt{\frac{\lambda}{3}}\big)\big(x+\sqrt{\frac{\lambda}{3}}\big)$.

\change

Il y a deux racines donc deux points critiques $x_1$ et et $x_2$

\change

 $f_\lambda'(x) > 0$ sur $]-\infty,x_1[$ et $]x_2,+\infty[$
et $f_\lambda'(x) < 0$ sur $]x_1,x_2[$. 

\change

En anticipant sur la suite on en déduit que $f_\lambda$ est croissante sur $]-\infty,x_1[$, puis décroissante sur $]x_1,x_2[$,

\change

donc $x_1$ est un maximum local. (car autour de $x_1$ toutes les valeurs sont plus petites).

\change

De même  $f_\lambda$ est décroissante avant $x_2$ 
puis croissante après $x_2$
\change 

donc $x_2$ est un minimum local.

\change

Voici le graphe de $f_\lambda$ si $\lambda$ est négatif,
on retrouve bien notre maximum local $x_1$ et notre minimum local $x_2$.

%%%%%%%%%%%%%%%%%%%%%%%%%%%%%%%%%%%%%%%%%%%%%%%%%%%%%%%%%%%
\diapo

Montrons le théorème que nous avons vu : un extremum local  est toujours un point critique. 

\change

Supposons que $x_0$ soit un maximum local de $f$, 

\change

Par définition il existe un intervalle $J$ autour de $x_0$
tel que pour tout $x\in I \cap J$ on a $f(x) \le f(x_0)$.

\change

Prenons d'abord  $x \in I\cap J$ tel que $x < x_0$ 


\change

on a $f(x)-f(x_0) \le 0$ et $x-x_0<0$ 

\change

donc le taux d'accroissement $\frac{f(x)-f(x_0)}{x-x_0} \ge 0$

\change

ainsi à la limite on a encore $\lim_{x \to x_0^-} \frac{f(x)-f(x_0)}{x-x_0} \ge 0$.

\change

Maintenant on repart de $x \in I\cap J$ mais cette fois avec $x > x_0$ 

\change

on a $f(x)-f(x_0) \le 0$ et $x-x_0>0$ 

\change

donc $\frac{f(x)-f(x_0)}{x-x_0} \le 0$

\change

et donc à la limite $\lim_{x \to x_0^+} \frac{f(x)-f(x_0)}{x-x_0} \le 0$.

\change

Or $f$ est dérivable en $x_0$ donc 
$\lim_{x \to x_0^-} \frac{f(x)-f(x_0)}{x-x_0} =f'(x_0).$ qui est donc positif ou nul.

\change

Mais $\lim_{x \to x_0^+} \frac{f(x)-f(x_0)}{x-x_0}$ égale aussi $f'(x_0).$
qui est donc négatif ou nul.

\change

La dérivée est à la fois positive et négative, la seule possibilité est que $f'(x_0)=0$.


%%%%%%%%%%%%%%%%%%%%%%%%%%%%%%%%%%%%%%%%%%%%%%%%%%%%%%%%%%%
\diapo

Faisons quelques remarques sur le théorème : 
si on a un extremum local alors on a un point critique.

Mais la réciproque  est fausse en général.

\change

Par exemple la fonction $f : \Rr \to \Rr$, définie par $f(x)= x^3$
vérifie $f'(0)=0$ donc $0$ est un point critique

\change

mais $x_0=0$ n'est ni maximum local ni un minimum local.

\change

Deuxième remarque nous avons énoncé le théorème dans le cas d'un intervalle ouvert.
Pour le cas d'un intervalle fermé, il faut faire attention
aux extrémités.

\change

 Par exemple si $f : [a,b] \to \Rr$ est une fonction dérivable qui admet un extremum en $x_0$,
alors on est dans l'une des situations suivantes :
\begin{itemize}
  \item $x_0= a$, 
  \item $x_0 =b$,
  \item $x_0 \in ]a,b[$ et dans ce dernier cas on a bien un point critique par le théorème.
\end{itemize}

\change

Ce qui nous intéresse souvent c'est de trouver le maximum ou le minimum d'une fonction sur un intervalle
donné. Dans le cas d'une fonction dérivable sur un intervalle fermé $[a,b]$,
il faut comparer
les valeurs de $f$ aux différents points critiques mais aussi aux extrémités $a$ et $b$.

\change

Par exemple sur ce dessin, le maximum est atteint en un point intérieur de l'intervalle,
il correspond bien à un point critique (la tangente est horizontale).

\change

Ici le maximum est atteint en $a$

\change

Alors qu'ici il est atteint en $b$, 

notez bien qu'aux extrémités on peut avoir un maximum sans que la dérivée s'annule.

%%%%%%%%%%%%%%%%%%%%%%%%%%%%%%%%%%%%%%%%%%%%%%%%%%%%%%%%%%%
\diapo

L'énoncé du théorème de Rolle est simple si l'on fait bien attention aux hypothèses.

On part d'une fonction $f:[a,b] \to \Rr$ telle que 


  $f$ est continue sur l'intervalle fermé $[a,b]$


et dérivable  sur l'intervalle ouvert $]a,b[$


Si l'on a  $f(a)=f(b)$


Alors il existe $c \in ]a,b[$  tel que $f'(c)=0$

\change

L'interprétation géométrique est simple :

si une fonction prend les mêmes valeurs aux extrémités,
alors il existe un point où la tangente est horizontale.

Malheureusement le théorème ne nous dit pas du tout comment calculer $c$,
c'est juste un théorème d'existence. En plus rien ne dit que le $c$ est unique,
il peut y en avoir plusieurs.


%%%%%%%%%%%%%%%%%%%%%%%%%%%%%%%%%%%%%%%%%%%%%%%%%%%%%%%%%%%
\diapo

Soit $P(X)$ le polynôme suivant ayant toutes ces racines réelles.
Les racines sont les $\alpha_1,\alpha_2$ jusqu'à $\alpha_n$
elles sont distinctes et ordonnées.

\change

Dans un premier temps nous allons montrer que le polynôme dérivé
$P'$ a $n-1$ racines réelles distinctes.

\change

On considère $P$ comme une fonction polynomiale,

$P$ est une fonction continue et dérivable sur $\Rr$.

\change

Comme $P(\alpha_1)=0=P(\alpha_2)$ alors par le théorème de Rolle 

il existe $c_1 \in ]\alpha_1,\alpha_2[$ tel que $P'(c_1)=0$. 

Nous avons trouvé une première racine de $P'$.

\change

Plus généralement, pour $1 \le k \le n-1$, comme $P(\alpha_k)=0=P(\alpha_{k+1})$ alors
le théorème de Rolle implique l'existence de $c_k \in ]\alpha_k,\alpha_{k+1}[$ tel que $P'(c_k)=0$.

\change

Nous avons bien trouvé $n-1$ racines de $P'$ : $c_1, c_2 , \cdots , c_{n-1}$.
Comme $P'$ est un polynôme de degré $n-1$, toutes ses racines sont réelles 
et elles sont distinctes car elles appartiennent à des intervalles disjoints.

[split ?]


\change

Montrons maintenant que le polynôme $P+P'$ a $n-1$ racines réelles distinctes.

\change

L'astuce consiste à considérer la fonction auxiliaire $f(x)=P(x)\exp x$.

$f$ est une fonction continue et dérivable sur $\Rr$.

\change

$f$ s'annule comme $P$ en $\alpha_1,\ldots,\alpha_n$.

\change

La dérivée de $f$ est $f'(x)=\big(P(x)+P'(x)\big) \exp x$.

\change

Donc par le théorème de Rolle, comme $f(\alpha_k)=0=f(\alpha_{k+1})$ alors
il existe $\gamma_k \in ]\alpha_k,\alpha_{k+1}[$ tel que $f'(\gamma_k)=0$ 

\change

Mais comme la fonction exponentielle ne s'annule jamais alors
$(P+P')(\gamma_k)=0$. 


Nous avons bien trouvé $n-1$ racines distinctes de $P+P'$ : 
$\gamma_1 < \gamma_2 < \cdots < \gamma_{n-1}$.

Il n'est pas très dur d'en déduire que la dernière racine de $P+P'$ 
est aussi une racine réelle.



%%%%%%%%%%%%%%%%%%%%%%%%%%%%%%%%%%%%%%%%%%%%%%%%%%%%%%%%%%%
\diapo

Passons à la preuve du théorème de Rolle.
Sous ces hypothèses il s'agit de montrer qu'il existe une valeur $c$
strictement comprise entre $a$ et $b$ tel que la dérivée s'annule en $c$.

\change


Tout d'abord, si $f$ est constante sur $[a,b]$ alors n'importe quel $c$ convient
car la dérivée est nulle partout.

\change

Si la fonction n'est pas constante alors il existe un $x_0$ tel que $f(x_0) \neq f(a)$.

\change

On peut supposer par exemple $f(x_0) > f(a)$. (le cas inférieur est similaire).

\change

Alors $f$ est continue sur l'intervalle fermé et borné 
$[a,b]$, elle est donc bornée et atteint ses bornes. 
En particulier elle admet un maximum en un certain 
point $c\in[a,b]$.

\change

Mais $f(c) \ge f(x_0) > f(a)$ donc $c \neq a$. 

\change

Mais par hypothèse $f(a)=f(b)$ donc aussi $c\neq b$.

\change

Ainsi $c$ est vraiment compris strictement entre $a$ et $b$.

\change

Mais $f$ est dérivable sur l'intervalle *ouvert* $]a,b[$, donc elle dérivable en $c$, 

et comme elle admet un maximum en $c$alors $f'(c)=0$.


%%%%%%%%%%%%%%%%%%%%%%%%%%%%%%%%%%%%%%%%%%%%%%%%%%%%%%%%%%%
\diapo

Voici des premiers exercices sur les extremums et le théorème de Rolle,
n'oubliez pas ensuite de passer à des exercices plus difficiles.



\end{document}