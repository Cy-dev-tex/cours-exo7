
%%%%%%%%%%%%%%%%%% PREAMBULE %%%%%%%%%%%%%%%%%%

\documentclass[aspectratio=169,utf8]{beamer}
%\documentclass[aspectratio=169,handout]{beamer}

\usetheme{Boadilla}
%\usecolortheme{seahorse}
\usecolortheme[RGB={245,66,24}]{structure}
\useoutertheme{infolines}

% packages
\usepackage{amsfonts,amsmath,amssymb,amsthm}
\usepackage[utf8]{inputenc}
\usepackage[T1]{fontenc}
\usepackage{lmodern}

\usepackage[francais]{babel}
\usepackage{fancybox}
\usepackage{graphicx}

\usepackage{float}
\usepackage{xfrac}

%\usepackage[usenames, x11names]{xcolor}
\usepackage{tikz}
\usepackage{pgfplots}
\usepackage{datetime}



%-----  Package unités -----
\usepackage{siunitx}
\sisetup{locale = FR,detect-all,per-mode = symbol}

%\usepackage{mathptmx}
%\usepackage{fouriernc}
%\usepackage{newcent}
%\usepackage[mathcal,mathbf]{euler}

%\usepackage{palatino}
%\usepackage{newcent}
% \usepackage[mathcal,mathbf]{euler}



% \usepackage{hyperref}
% \hypersetup{colorlinks=true, linkcolor=blue, urlcolor=blue,
% pdftitle={Exo7 - Exercices de mathématiques}, pdfauthor={Exo7}}


%section
% \usepackage{sectsty}
% \allsectionsfont{\bf}
%\sectionfont{\color{Tomato3}\upshape\selectfont}
%\subsectionfont{\color{Tomato4}\upshape\selectfont}

%----- Ensembles : entiers, reels, complexes -----
\newcommand{\Nn}{\mathbb{N}} \newcommand{\N}{\mathbb{N}}
\newcommand{\Zz}{\mathbb{Z}} \newcommand{\Z}{\mathbb{Z}}
\newcommand{\Qq}{\mathbb{Q}} \newcommand{\Q}{\mathbb{Q}}
\newcommand{\Rr}{\mathbb{R}} \newcommand{\R}{\mathbb{R}}
\newcommand{\Cc}{\mathbb{C}} 
\newcommand{\Kk}{\mathbb{K}} \newcommand{\K}{\mathbb{K}}

%----- Modifications de symboles -----
\renewcommand{\epsilon}{\varepsilon}
\renewcommand{\Re}{\mathop{\text{Re}}\nolimits}
\renewcommand{\Im}{\mathop{\text{Im}}\nolimits}
%\newcommand{\llbracket}{\left[\kern-0.15em\left[}
%\newcommand{\rrbracket}{\right]\kern-0.15em\right]}

\renewcommand{\ge}{\geqslant}
\renewcommand{\geq}{\geqslant}
\renewcommand{\le}{\leqslant}
\renewcommand{\leq}{\leqslant}
\renewcommand{\epsilon}{\varepsilon}

%----- Fonctions usuelles -----
\newcommand{\ch}{\mathop{\text{ch}}\nolimits}
\newcommand{\sh}{\mathop{\text{sh}}\nolimits}
\renewcommand{\tanh}{\mathop{\text{th}}\nolimits}
\newcommand{\cotan}{\mathop{\text{cotan}}\nolimits}
\newcommand{\Arcsin}{\mathop{\text{arcsin}}\nolimits}
\newcommand{\Arccos}{\mathop{\text{arccos}}\nolimits}
\newcommand{\Arctan}{\mathop{\text{arctan}}\nolimits}
\newcommand{\Argsh}{\mathop{\text{argsh}}\nolimits}
\newcommand{\Argch}{\mathop{\text{argch}}\nolimits}
\newcommand{\Argth}{\mathop{\text{argth}}\nolimits}
\newcommand{\pgcd}{\mathop{\text{pgcd}}\nolimits} 


%----- Commandes divers ------
\newcommand{\ii}{\mathrm{i}}
\newcommand{\dd}{\text{d}}
\newcommand{\id}{\mathop{\text{id}}\nolimits}
\newcommand{\Ker}{\mathop{\text{Ker}}\nolimits}
\newcommand{\Card}{\mathop{\text{Card}}\nolimits}
\newcommand{\Vect}{\mathop{\text{Vect}}\nolimits}
\newcommand{\Mat}{\mathop{\text{Mat}}\nolimits}
\newcommand{\rg}{\mathop{\text{rg}}\nolimits}
\newcommand{\tr}{\mathop{\text{tr}}\nolimits}


%----- Structure des exercices ------

\newtheoremstyle{styleexo}% name
{2ex}% Space above
{3ex}% Space below
{}% Body font
{}% Indent amount 1
{\bfseries} % Theorem head font
{}% Punctuation after theorem head
{\newline}% Space after theorem head 2
{}% Theorem head spec (can be left empty, meaning ‘normal’)

%\theoremstyle{styleexo}
\newtheorem{exo}{Exercice}
\newtheorem{ind}{Indications}
\newtheorem{cor}{Correction}


\newcommand{\exercice}[1]{} \newcommand{\finexercice}{}
%\newcommand{\exercice}[1]{{\tiny\texttt{#1}}\vspace{-2ex}} % pour afficher le numero absolu, l'auteur...
\newcommand{\enonce}{\begin{exo}} \newcommand{\finenonce}{\end{exo}}
\newcommand{\indication}{\begin{ind}} \newcommand{\finindication}{\end{ind}}
\newcommand{\correction}{\begin{cor}} \newcommand{\fincorrection}{\end{cor}}

\newcommand{\noindication}{\stepcounter{ind}}
\newcommand{\nocorrection}{\stepcounter{cor}}

\newcommand{\fiche}[1]{} \newcommand{\finfiche}{}
\newcommand{\titre}[1]{\centerline{\large \bf #1}}
\newcommand{\addcommand}[1]{}
\newcommand{\video}[1]{}

% Marge
\newcommand{\mymargin}[1]{\marginpar{{\small #1}}}

\def\noqed{\renewcommand{\qedsymbol}{}}


%----- Presentation ------
\setlength{\parindent}{0cm}

%\newcommand{\ExoSept}{\href{http://exo7.emath.fr}{\textbf{\textsf{Exo7}}}}

\definecolor{myred}{rgb}{0.93,0.26,0}
\definecolor{myorange}{rgb}{0.97,0.58,0}
\definecolor{myyellow}{rgb}{1,0.86,0}

\newcommand{\LogoExoSept}[1]{  % input : echelle
{\usefont{U}{cmss}{bx}{n}
\begin{tikzpicture}[scale=0.1*#1,transform shape]
  \fill[color=myorange] (0,0)--(4,0)--(4,-4)--(0,-4)--cycle;
  \fill[color=myred] (0,0)--(0,3)--(-3,3)--(-3,0)--cycle;
  \fill[color=myyellow] (4,0)--(7,4)--(3,7)--(0,3)--cycle;
  \node[scale=5] at (3.5,3.5) {Exo7};
\end{tikzpicture}}
}


\newcommand{\debutmontitre}{
  \author{} \date{} 
  \thispagestyle{empty}
  \hspace*{-10ex}
  \begin{minipage}{\textwidth}
    \titlepage  
  \vspace*{-2.5cm}
  \begin{center}
    \LogoExoSept{2.5}
  \end{center}
  \end{minipage}

  \vspace*{-0cm}
  
  % Astuce pour que le background ne soit pas discrétisé lors de la conversion pdf -> png
\begin{tikzpicture}
        \fill[opacity=0,green!60!black] (0,0)--++(0,0)--++(0,0)--++(0,0)--cycle; 
\end{tikzpicture}

% toc S'affiche trop tot :
% \tableofcontents[hideallsubsections, pausesections]
}

\newcommand{\finmontitre}{
  \end{frame}
  \setcounter{framenumber}{0}
} % ne marche pas pour une raison obscure

%----- Commandes supplementaires ------

% \usepackage[landscape]{geometry}
% \geometry{top=1cm, bottom=3cm, left=2cm, right=10cm, marginparsep=1cm
% }
% \usepackage[a4paper]{geometry}
% \geometry{top=2cm, bottom=2cm, left=2cm, right=2cm, marginparsep=1cm
% }

%\usepackage{standalone}


% New command Arnaud -- november 2011
\setbeamersize{text margin left=24ex}
% si vous modifier cette valeur il faut aussi
% modifier le decalage du titre pour compenser
% (ex : ici =+10ex, titre =-5ex

\theoremstyle{definition}
%\newtheorem{proposition}{Proposition}
%\newtheorem{exemple}{Exemple}
%\newtheorem{theoreme}{Théorème}
%\newtheorem{lemme}{Lemme}
%\newtheorem{corollaire}{Corollaire}
%\newtheorem*{remarque*}{Remarque}
%\newtheorem*{miniexercice}{Mini-exercices}
%\newtheorem{definition}{Définition}

% Commande tikz
\usetikzlibrary{calc}
\usetikzlibrary{patterns,arrows}
\usetikzlibrary{matrix}
\usetikzlibrary{fadings} 

%definition d'un terme
\newcommand{\defi}[1]{{\color{myorange}\textbf{\emph{#1}}}}
\newcommand{\evidence}[1]{{\color{blue}\textbf{\emph{#1}}}}
\newcommand{\assertion}[1]{\emph{\og#1\fg}}  % pour chapitre logique
%\renewcommand{\contentsname}{Sommaire}
\renewcommand{\contentsname}{}
\setcounter{tocdepth}{2}



%------ Figures ------

\def\myscale{1} % par défaut 
\newcommand{\myfigure}[2]{  % entrée : echelle, fichier figure
\def\myscale{#1}
\begin{center}
\footnotesize
{#2}
\end{center}}


%------ Encadrement ------

\usepackage{fancybox}


\newcommand{\mybox}[1]{
\setlength{\fboxsep}{7pt}
\begin{center}
\shadowbox{#1}
\end{center}}

\newcommand{\myboxinline}[1]{
\setlength{\fboxsep}{5pt}
\raisebox{-10pt}{
\shadowbox{#1}
}
}

%--------------- Commande beamer---------------
\newcommand{\beameronly}[1]{#1} % permet de mettre des pause dans beamer pas dans poly


\setbeamertemplate{navigation symbols}{}
\setbeamertemplate{footline}  % tiré du fichier beamerouterinfolines.sty
{
  \leavevmode%
  \hbox{%
  \begin{beamercolorbox}[wd=.333333\paperwidth,ht=2.25ex,dp=1ex,center]{author in head/foot}%
    % \usebeamerfont{author in head/foot}\insertshortauthor%~~(\insertshortinstitute)
    \usebeamerfont{section in head/foot}{\bf\insertshorttitle}
  \end{beamercolorbox}%
  \begin{beamercolorbox}[wd=.333333\paperwidth,ht=2.25ex,dp=1ex,center]{title in head/foot}%
    \usebeamerfont{section in head/foot}{\bf\insertsectionhead}
  \end{beamercolorbox}%
  \begin{beamercolorbox}[wd=.333333\paperwidth,ht=2.25ex,dp=1ex,right]{date in head/foot}%
    % \usebeamerfont{date in head/foot}\insertshortdate{}\hspace*{2em}
    \insertframenumber{} / \inserttotalframenumber\hspace*{2ex} 
  \end{beamercolorbox}}%
  \vskip0pt%
}


\definecolor{mygrey}{rgb}{0.5,0.5,0.5}
\setlength{\parindent}{0cm}
%\DeclareTextFontCommand{\helvetica}{\fontfamily{phv}\selectfont}

% background beamer
\definecolor{couleurhaut}{rgb}{0.85,0.9,1}  % creme
\definecolor{couleurmilieu}{rgb}{1,1,1}  % vert pale
\definecolor{couleurbas}{rgb}{0.85,0.9,1}  % blanc
\setbeamertemplate{background canvas}[vertical shading]%
[top=couleurhaut,middle=couleurmilieu,midpoint=0.4,bottom=couleurbas] 
%[top=fondtitre!05,bottom=fondtitre!60]



\makeatletter
\setbeamertemplate{theorem begin}
{%
  \begin{\inserttheoremblockenv}
  {%
    \inserttheoremheadfont
    \inserttheoremname
    \inserttheoremnumber
    \ifx\inserttheoremaddition\@empty\else\ (\inserttheoremaddition)\fi%
    \inserttheorempunctuation
  }%
}
\setbeamertemplate{theorem end}{\end{\inserttheoremblockenv}}

\newenvironment{theoreme}[1][]{%
   \setbeamercolor{block title}{fg=structure,bg=structure!40}
   \setbeamercolor{block body}{fg=black,bg=structure!10}
   \begin{block}{{\bf Th\'eor\`eme }#1}
}{%
   \end{block}%
}


\newenvironment{proposition}[1][]{%
   \setbeamercolor{block title}{fg=structure,bg=structure!40}
   \setbeamercolor{block body}{fg=black,bg=structure!10}
   \begin{block}{{\bf Proposition }#1}
}{%
   \end{block}%
}

\newenvironment{corollaire}[1][]{%
   \setbeamercolor{block title}{fg=structure,bg=structure!40}
   \setbeamercolor{block body}{fg=black,bg=structure!10}
   \begin{block}{{\bf Corollaire }#1}
}{%
   \end{block}%
}

\newenvironment{mydefinition}[1][]{%
   \setbeamercolor{block title}{fg=structure,bg=structure!40}
   \setbeamercolor{block body}{fg=black,bg=structure!10}
   \begin{block}{{\bf Définition} #1}
}{%
   \end{block}%
}

\newenvironment{lemme}[0]{%
   \setbeamercolor{block title}{fg=structure,bg=structure!40}
   \setbeamercolor{block body}{fg=black,bg=structure!10}
   \begin{block}{\bf Lemme}
}{%
   \end{block}%
}

\newenvironment{remarque}[1][]{%
   \setbeamercolor{block title}{fg=black,bg=structure!20}
   \setbeamercolor{block body}{fg=black,bg=structure!5}
   \begin{block}{Remarque #1}
}{%
   \end{block}%
}


\newenvironment{exemple}[1][]{%
   \setbeamercolor{block title}{fg=black,bg=structure!20}
   \setbeamercolor{block body}{fg=black,bg=structure!5}
   \begin{block}{{\bf Exemple }#1}
}{%
   \end{block}%
}


\newenvironment{miniexercice}[0]{%
   \setbeamercolor{block title}{fg=structure,bg=structure!20}
   \setbeamercolor{block body}{fg=black,bg=structure!5}
   \begin{block}{Mini-exercices}
}{%
   \end{block}%
}


\newenvironment{tp}[0]{%
   \setbeamercolor{block title}{fg=structure,bg=structure!40}
   \setbeamercolor{block body}{fg=black,bg=structure!10}
   \begin{block}{\bf Travaux pratiques}
}{%
   \end{block}%
}
\newenvironment{exercicecours}[1][]{%
   \setbeamercolor{block title}{fg=structure,bg=structure!40}
   \setbeamercolor{block body}{fg=black,bg=structure!10}
   \begin{block}{{\bf Exercice }#1}
}{%
   \end{block}%
}
\newenvironment{algo}[1][]{%
   \setbeamercolor{block title}{fg=structure,bg=structure!40}
   \setbeamercolor{block body}{fg=black,bg=structure!10}
   \begin{block}{{\bf Algorithme}\hfill{\color{gray}\texttt{#1}}}
}{%
   \end{block}%
}


\setbeamertemplate{proof begin}{
   \setbeamercolor{block title}{fg=black,bg=structure!20}
   \setbeamercolor{block body}{fg=black,bg=structure!5}
   \begin{block}{{\footnotesize Démonstration}}
   \footnotesize
   \smallskip}
\setbeamertemplate{proof end}{%
   \end{block}}
\setbeamertemplate{qed symbol}{\openbox}


\makeatother
\usecolortheme[RGB={0,199,174}]{structure}

%%%%%%%%%%%%%%%%%%%%%%%%%%%%%%%%%%%%%%%%%%%%%%%%%%%%%%%%%%%%%
%%%%%%%%%%%%%%%%%%%%%%%%%%%%%%%%%%%%%%%%%%%%%%%%%%%%%%%%%%%%%



\begin{document}



\title{{\bf Dérivée d'une fonction}}
\subtitle{Théorème des accroissements finis}

\begin{frame}
  
  \debutmontitre

  \pause

{\footnotesize
\hfill
\setbeamercovered{transparent=50}
\begin{minipage}{0.6\textwidth}
  \begin{itemize}
    \item<3-> Théorème
    \item<4-> Fonction croissante et dérivée
    \item<5-> Inégalité des accroissements finis
    \item<6-> Règle de l'Hospital
  \end{itemize}
\end{minipage}
}

\end{frame}

\setcounter{framenumber}{0}


%%%%%%%%%%%%%%%%%%%%%%%%%%%%%%%%%%%%%%%%%%%%%%%%%%%%%%%%%%%%%%%%


\section*{Théorème des accroissements finis}


\begin{frame}

\begin{theoreme}[des accroissements finis]
Soit $f:[a,b] \to \Rr$ une fonction continue sur $[a,b]$ et dérivable sur $]a,b[$

Il existe $c\in]a,b[$ tel que 
\mybox{$f(b)-f(a)= f'(c) \; (b-a)$}
\end{theoreme}


\pause

\myfigure{1.5}{
\tikzinput{fig_derive10} 
}  
\end{frame}

\begin{frame}

\begin{theoreme}[des accroissements finis]
Soit $f:[a,b] \to \Rr$ une fonction continue sur $[a,b]$ et dérivable sur $]a,b[$

Il existe $c\in]a,b[$ tel que 
\mybox{$f(b)-f(a)= f'(c) \; (b-a)$}
\end{theoreme}


\begin{proof}
\pause
\begin{itemize}
  \item Posons $\ell = \frac{f(b)-f(a)}{b-a}$ et $g(x) = f(x) - \ell \cdot (x-a)$
\pause
  \item Alors $g(a)=f(a)$, \quad \pause $g(b)=f(b)- \frac{f(b)-f(a)}{b-a} \cdot (b-a) = f(a)$
\pause
  \item Par le théorème de Rolle, il existe $c \in ]a,b[$ tel que $g'(c) =0$
\pause
  \item Or $g'(x) = f'(x) - \ell$ \pause donc $f'(c)= \frac{f(b)-f(a)}{b-a}$
\end{itemize}

\end{proof}
\end{frame}

%%%%%%%%%%%%%%%%%%%%%%%%%%%%%%%%%%%%%%%%%%%%%%%%%%%%%%%%%%%%%%%%


\section*{Fonction croissante et dérivée}


\begin{frame}

\begin{corollaire}
Soit $f:\![a,b] \to \Rr$ une fonction continue sur $[a,b]$ et dérivable sur $]a,b[$
\pause
\begin{enumerate}[<+->]
\setlength{\itemsep}{5pt} 
  \item $\forall x \in ]a,b[ \quad f'(x) \ge 0 \quad \iff \quad$ $f$ est croissante 
  \item $\forall x \in ]a,b[ \quad f'(x) \le 0 \quad \iff \quad$ $f$ est décroissante 
  \item $\forall x \in ]a,b[ \quad f'(x) = 0 \quad \iff \quad$ $f$ est constante 
  \item $\forall x \in ]a,b[ \quad f'(x) > 0 \quad \implies \quad$ $f$ est strictement croissante 
  \item $\forall x \in ]a,b[ \quad f'(x) < 0 \quad \implies \quad$ $f$ est strictement décroissante
\end{enumerate}
\end{corollaire}
\end{frame}



\begin{frame}
\begin{proof}
\centerline{$f'(x) \ge 0 \quad \iff \quad$ $f$ est croissante}

\pause

\begin{itemize}
  \item[$\Longrightarrow$]
  \begin{itemize}
     \item Supposons la dérivée positive. Soient $x,y \in ]a,b[$ avec $x\le y$
\pause
     \item Par le théorème des accroissements finis, il existe $c\in]x,y[$ tel que
$f(x)-f(y) = f'(c) (x-y)$
\pause
     \item Mais $f'(c) \ge 0$ et $x-y \le 0$ donc $f(x)-f(y) \le 0$
\pause
     \item Cela implique que $f(x) \le f(y)$ donc $f$ est croissante
  \end{itemize}
\pause
  \item[$\Longleftarrow$]
  \begin{itemize}
     \item Supposons que $f$ est croissante. Fixons $x\in]a,b[$
\pause
     \item Pour tout $y > x$ nous avons $y-x>0$ et $f(y)-f(x)\ge 0$
\pause
     \item Le taux d'accroissement vérifie $\frac{f(y)-f(x)}{y-x}\ge 0$
\pause
     \item \`A la limite, quand $y \to x$ : $f'(x) \ge 0$
  \end{itemize}
\end{itemize}
\end{proof}
\end{frame}






%%%%%%%%%%%%%%%%%%%%%%%%%%%%%%%%%%%%%%%%%%%%%%%%%%%%%%%%%%%%%%%%


\section*{Inégalité des accroissements finis}


\begin{frame}

\begin{corollaire}[Inégalité des accroissements finis]
Soit $f : I \to \Rr$ une fonction dérivable sur un intervalle $I$ ouvert

S'il existe une constante $M$ tel que pour tout $x \in I$, $\big|f'(x)\big| \le M$ 
alors
\mybox{$\forall x,y \in I \qquad \big| f(x)-f(y) \big| \le M |x-y|$}
\end{corollaire}

\pause

\begin{proof}
\begin{itemize}
  \item Fixons $x,y \in I$
\pause
  \item il existe alors $c\in]x,y[$ ou $]y,x[$ tel que $f(x)-f(y)=f'(c)(x-y)$
\pause
  \item comme $|f'(c)| \le M$ alors $\big| f(x)-f(y) \big| \le M |x-y|$
\end{itemize}
\end{proof}

\end{frame}



\begin{frame}
\begin{exemple}
Soit $f(x)=\sin(x)$
\pause
\begin{itemize}
  \item Comme $f'(x)=\cos x$ alors $|f'(x)| \le 1$
\pause
  \item L'inégalité des accroissements finis :
\pause
$\text{pour tout } x,y \in \Rr \qquad |\sin x - \sin y | \le |x-y|$
\pause
  \item si l'on fixe $y=0$ alors \myboxinline{$|\sin x| \le |x|$}
\end{itemize}

\pause
\vspace*{-2ex}
\myfigure{0.6}{
\tikzinput{fig_derive11} 
}  
\end{exemple}
\end{frame}

%%%%%%%%%%%%%%%%%%%%%%%%%%%%%%%%%%%%%%%%%%%%%%%%%%%%%%%%%%%%%%%%


\section*{Règle de l'Hospital}


\begin{frame}

\begin{corollaire}[Règle de l'Hospital]
Soient $f,g : I \to \Rr$ deux fonctions dérivables et soit $x_0\in I$ 
\uncover<3->{avec 
\begin{itemize}
  \item $f(x_0)=g(x_0)=0$
  \item $\forall x \in I\setminus\{x_0\} \quad g'(x)\neq0$
\end{itemize}
}
\uncover<2->{
\mybox{Si \quad $\displaystyle \lim_{x\to x_0} \frac{f'(x)}{g'(x)} = \ell \quad (\in \Rr)$ 
\quad alors \quad $\displaystyle \lim_{x\to x_0} \frac{f(x)}{g(x)} = \ell$}
}
\end{corollaire}

\end{frame}


\begin{frame}
\begin{exemple}
Limite en $1$ de $\frac{\ln(x^2+x-1)}{\ln(x)}$

\pause

\begin{itemize}
  \item $f(x)=\ln(x^2+x-1)$, $f(1)=0$, $f'(x)=\frac{2x+1}{x^2+x-1}$
\pause
  \item $g(x)=\ln(x)$, $g(1)=0$, $g'(x)=\frac 1x$
\pause
  \item Prenons $I=]0,1]$, $x_0=1$, alors $g'$ ne s'annule pas sur $I\setminus\{x_0\}$
\end{itemize}
\pause
$$\frac{f'(x)}{g'(x)} \pause = \frac{2x+1}{x^2+x-1} \times x \pause = \frac{2x^2+x}{x^2+x-1} \pause \xrightarrow[x\to 1]{} 3$$
\pause
Donc 
$$\frac{f(x)}{g(x)} \xrightarrow[x\to 1]{} 3$$
\end{exemple}
\end{frame}



%%%%%%%%%%%%%%%%%%%%%%%%%%%%%%%%%%%%%%%%%%%%%%%%%%%%%%%%%%%%%%%%
\section*{Mini-exercices}


\begin{frame}
\begin{miniexercice}
\begin{enumerate}
  \item Soit $f(x) = \frac{x^3}{3}+\frac{x^2}{2}-2x+2$. \'Etudier la fonction $f$. Tracer son graphe.
Montrer que $f$ admet un minimum local et un maximum local.

   \item Soit $f(x)=\sqrt{x}$. Appliquer le théorème des accroissements finis sur l'intervalle $[100,101]$.
En déduire l'encadrement $10+\frac{1}{22} \le \sqrt{101} \le 10 + \frac{1}{20}$.

  \item Appliquer le théorème des accroissements finis pour montrer que $\ln (1+x)-\ln(x) < \frac 1x$ (pour tout $x>0$).

  \item Soit $f(x) = e^x$. Que donne l'inégalité des accroissements finis sur $[0,x]$ ?
%En déduire que pour tout $x \ge 0$, $e^x-1 \le xe^x$.

  \item Appliquer la règle de l'Hospital pour calculer les limites suivantes (quand $x\to 0$):
$\displaystyle \frac{x}{(1+x)^n-1}$ ; $\displaystyle \frac{\ln(x+1)}{\sqrt x}$ ; 
$\displaystyle \frac{1-\cos x}{\tan x}$ ; $\displaystyle \frac{x-\sin x}{x^3}$.
\end{enumerate}
\end{miniexercice}
\end{frame}


\end{document}