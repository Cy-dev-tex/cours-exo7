
%%%%%%%%%%%%%%%%%% PREAMBULE %%%%%%%%%%%%%%%%%%

\documentclass[aspectratio=169,utf8]{beamer}
%\documentclass[aspectratio=169,handout]{beamer}

\usetheme{Boadilla}
%\usecolortheme{seahorse}
\usecolortheme[RGB={245,66,24}]{structure}
\useoutertheme{infolines}

% packages
\usepackage{amsfonts,amsmath,amssymb,amsthm}
\usepackage[utf8]{inputenc}
\usepackage[T1]{fontenc}
\usepackage{lmodern}

\usepackage[francais]{babel}
\usepackage{fancybox}
\usepackage{graphicx}

\usepackage{float}
\usepackage{xfrac}

%\usepackage[usenames, x11names]{xcolor}
\usepackage{tikz}
\usepackage{pgfplots}
\usepackage{datetime}



%-----  Package unités -----
\usepackage{siunitx}
\sisetup{locale = FR,detect-all,per-mode = symbol}

%\usepackage{mathptmx}
%\usepackage{fouriernc}
%\usepackage{newcent}
%\usepackage[mathcal,mathbf]{euler}

%\usepackage{palatino}
%\usepackage{newcent}
% \usepackage[mathcal,mathbf]{euler}



% \usepackage{hyperref}
% \hypersetup{colorlinks=true, linkcolor=blue, urlcolor=blue,
% pdftitle={Exo7 - Exercices de mathématiques}, pdfauthor={Exo7}}


%section
% \usepackage{sectsty}
% \allsectionsfont{\bf}
%\sectionfont{\color{Tomato3}\upshape\selectfont}
%\subsectionfont{\color{Tomato4}\upshape\selectfont}

%----- Ensembles : entiers, reels, complexes -----
\newcommand{\Nn}{\mathbb{N}} \newcommand{\N}{\mathbb{N}}
\newcommand{\Zz}{\mathbb{Z}} \newcommand{\Z}{\mathbb{Z}}
\newcommand{\Qq}{\mathbb{Q}} \newcommand{\Q}{\mathbb{Q}}
\newcommand{\Rr}{\mathbb{R}} \newcommand{\R}{\mathbb{R}}
\newcommand{\Cc}{\mathbb{C}} 
\newcommand{\Kk}{\mathbb{K}} \newcommand{\K}{\mathbb{K}}

%----- Modifications de symboles -----
\renewcommand{\epsilon}{\varepsilon}
\renewcommand{\Re}{\mathop{\text{Re}}\nolimits}
\renewcommand{\Im}{\mathop{\text{Im}}\nolimits}
%\newcommand{\llbracket}{\left[\kern-0.15em\left[}
%\newcommand{\rrbracket}{\right]\kern-0.15em\right]}

\renewcommand{\ge}{\geqslant}
\renewcommand{\geq}{\geqslant}
\renewcommand{\le}{\leqslant}
\renewcommand{\leq}{\leqslant}
\renewcommand{\epsilon}{\varepsilon}

%----- Fonctions usuelles -----
\newcommand{\ch}{\mathop{\text{ch}}\nolimits}
\newcommand{\sh}{\mathop{\text{sh}}\nolimits}
\renewcommand{\tanh}{\mathop{\text{th}}\nolimits}
\newcommand{\cotan}{\mathop{\text{cotan}}\nolimits}
\newcommand{\Arcsin}{\mathop{\text{arcsin}}\nolimits}
\newcommand{\Arccos}{\mathop{\text{arccos}}\nolimits}
\newcommand{\Arctan}{\mathop{\text{arctan}}\nolimits}
\newcommand{\Argsh}{\mathop{\text{argsh}}\nolimits}
\newcommand{\Argch}{\mathop{\text{argch}}\nolimits}
\newcommand{\Argth}{\mathop{\text{argth}}\nolimits}
\newcommand{\pgcd}{\mathop{\text{pgcd}}\nolimits} 


%----- Commandes divers ------
\newcommand{\ii}{\mathrm{i}}
\newcommand{\dd}{\text{d}}
\newcommand{\id}{\mathop{\text{id}}\nolimits}
\newcommand{\Ker}{\mathop{\text{Ker}}\nolimits}
\newcommand{\Card}{\mathop{\text{Card}}\nolimits}
\newcommand{\Vect}{\mathop{\text{Vect}}\nolimits}
\newcommand{\Mat}{\mathop{\text{Mat}}\nolimits}
\newcommand{\rg}{\mathop{\text{rg}}\nolimits}
\newcommand{\tr}{\mathop{\text{tr}}\nolimits}


%----- Structure des exercices ------

\newtheoremstyle{styleexo}% name
{2ex}% Space above
{3ex}% Space below
{}% Body font
{}% Indent amount 1
{\bfseries} % Theorem head font
{}% Punctuation after theorem head
{\newline}% Space after theorem head 2
{}% Theorem head spec (can be left empty, meaning ‘normal’)

%\theoremstyle{styleexo}
\newtheorem{exo}{Exercice}
\newtheorem{ind}{Indications}
\newtheorem{cor}{Correction}


\newcommand{\exercice}[1]{} \newcommand{\finexercice}{}
%\newcommand{\exercice}[1]{{\tiny\texttt{#1}}\vspace{-2ex}} % pour afficher le numero absolu, l'auteur...
\newcommand{\enonce}{\begin{exo}} \newcommand{\finenonce}{\end{exo}}
\newcommand{\indication}{\begin{ind}} \newcommand{\finindication}{\end{ind}}
\newcommand{\correction}{\begin{cor}} \newcommand{\fincorrection}{\end{cor}}

\newcommand{\noindication}{\stepcounter{ind}}
\newcommand{\nocorrection}{\stepcounter{cor}}

\newcommand{\fiche}[1]{} \newcommand{\finfiche}{}
\newcommand{\titre}[1]{\centerline{\large \bf #1}}
\newcommand{\addcommand}[1]{}
\newcommand{\video}[1]{}

% Marge
\newcommand{\mymargin}[1]{\marginpar{{\small #1}}}

\def\noqed{\renewcommand{\qedsymbol}{}}


%----- Presentation ------
\setlength{\parindent}{0cm}

%\newcommand{\ExoSept}{\href{http://exo7.emath.fr}{\textbf{\textsf{Exo7}}}}

\definecolor{myred}{rgb}{0.93,0.26,0}
\definecolor{myorange}{rgb}{0.97,0.58,0}
\definecolor{myyellow}{rgb}{1,0.86,0}

\newcommand{\LogoExoSept}[1]{  % input : echelle
{\usefont{U}{cmss}{bx}{n}
\begin{tikzpicture}[scale=0.1*#1,transform shape]
  \fill[color=myorange] (0,0)--(4,0)--(4,-4)--(0,-4)--cycle;
  \fill[color=myred] (0,0)--(0,3)--(-3,3)--(-3,0)--cycle;
  \fill[color=myyellow] (4,0)--(7,4)--(3,7)--(0,3)--cycle;
  \node[scale=5] at (3.5,3.5) {Exo7};
\end{tikzpicture}}
}


\newcommand{\debutmontitre}{
  \author{} \date{} 
  \thispagestyle{empty}
  \hspace*{-10ex}
  \begin{minipage}{\textwidth}
    \titlepage  
  \vspace*{-2.5cm}
  \begin{center}
    \LogoExoSept{2.5}
  \end{center}
  \end{minipage}

  \vspace*{-0cm}
  
  % Astuce pour que le background ne soit pas discrétisé lors de la conversion pdf -> png
\begin{tikzpicture}
        \fill[opacity=0,green!60!black] (0,0)--++(0,0)--++(0,0)--++(0,0)--cycle; 
\end{tikzpicture}

% toc S'affiche trop tot :
% \tableofcontents[hideallsubsections, pausesections]
}

\newcommand{\finmontitre}{
  \end{frame}
  \setcounter{framenumber}{0}
} % ne marche pas pour une raison obscure

%----- Commandes supplementaires ------

% \usepackage[landscape]{geometry}
% \geometry{top=1cm, bottom=3cm, left=2cm, right=10cm, marginparsep=1cm
% }
% \usepackage[a4paper]{geometry}
% \geometry{top=2cm, bottom=2cm, left=2cm, right=2cm, marginparsep=1cm
% }

%\usepackage{standalone}


% New command Arnaud -- november 2011
\setbeamersize{text margin left=24ex}
% si vous modifier cette valeur il faut aussi
% modifier le decalage du titre pour compenser
% (ex : ici =+10ex, titre =-5ex

\theoremstyle{definition}
%\newtheorem{proposition}{Proposition}
%\newtheorem{exemple}{Exemple}
%\newtheorem{theoreme}{Théorème}
%\newtheorem{lemme}{Lemme}
%\newtheorem{corollaire}{Corollaire}
%\newtheorem*{remarque*}{Remarque}
%\newtheorem*{miniexercice}{Mini-exercices}
%\newtheorem{definition}{Définition}

% Commande tikz
\usetikzlibrary{calc}
\usetikzlibrary{patterns,arrows}
\usetikzlibrary{matrix}
\usetikzlibrary{fadings} 

%definition d'un terme
\newcommand{\defi}[1]{{\color{myorange}\textbf{\emph{#1}}}}
\newcommand{\evidence}[1]{{\color{blue}\textbf{\emph{#1}}}}
\newcommand{\assertion}[1]{\emph{\og#1\fg}}  % pour chapitre logique
%\renewcommand{\contentsname}{Sommaire}
\renewcommand{\contentsname}{}
\setcounter{tocdepth}{2}



%------ Figures ------

\def\myscale{1} % par défaut 
\newcommand{\myfigure}[2]{  % entrée : echelle, fichier figure
\def\myscale{#1}
\begin{center}
\footnotesize
{#2}
\end{center}}


%------ Encadrement ------

\usepackage{fancybox}


\newcommand{\mybox}[1]{
\setlength{\fboxsep}{7pt}
\begin{center}
\shadowbox{#1}
\end{center}}

\newcommand{\myboxinline}[1]{
\setlength{\fboxsep}{5pt}
\raisebox{-10pt}{
\shadowbox{#1}
}
}

%--------------- Commande beamer---------------
\newcommand{\beameronly}[1]{#1} % permet de mettre des pause dans beamer pas dans poly


\setbeamertemplate{navigation symbols}{}
\setbeamertemplate{footline}  % tiré du fichier beamerouterinfolines.sty
{
  \leavevmode%
  \hbox{%
  \begin{beamercolorbox}[wd=.333333\paperwidth,ht=2.25ex,dp=1ex,center]{author in head/foot}%
    % \usebeamerfont{author in head/foot}\insertshortauthor%~~(\insertshortinstitute)
    \usebeamerfont{section in head/foot}{\bf\insertshorttitle}
  \end{beamercolorbox}%
  \begin{beamercolorbox}[wd=.333333\paperwidth,ht=2.25ex,dp=1ex,center]{title in head/foot}%
    \usebeamerfont{section in head/foot}{\bf\insertsectionhead}
  \end{beamercolorbox}%
  \begin{beamercolorbox}[wd=.333333\paperwidth,ht=2.25ex,dp=1ex,right]{date in head/foot}%
    % \usebeamerfont{date in head/foot}\insertshortdate{}\hspace*{2em}
    \insertframenumber{} / \inserttotalframenumber\hspace*{2ex} 
  \end{beamercolorbox}}%
  \vskip0pt%
}


\definecolor{mygrey}{rgb}{0.5,0.5,0.5}
\setlength{\parindent}{0cm}
%\DeclareTextFontCommand{\helvetica}{\fontfamily{phv}\selectfont}

% background beamer
\definecolor{couleurhaut}{rgb}{0.85,0.9,1}  % creme
\definecolor{couleurmilieu}{rgb}{1,1,1}  % vert pale
\definecolor{couleurbas}{rgb}{0.85,0.9,1}  % blanc
\setbeamertemplate{background canvas}[vertical shading]%
[top=couleurhaut,middle=couleurmilieu,midpoint=0.4,bottom=couleurbas] 
%[top=fondtitre!05,bottom=fondtitre!60]



\makeatletter
\setbeamertemplate{theorem begin}
{%
  \begin{\inserttheoremblockenv}
  {%
    \inserttheoremheadfont
    \inserttheoremname
    \inserttheoremnumber
    \ifx\inserttheoremaddition\@empty\else\ (\inserttheoremaddition)\fi%
    \inserttheorempunctuation
  }%
}
\setbeamertemplate{theorem end}{\end{\inserttheoremblockenv}}

\newenvironment{theoreme}[1][]{%
   \setbeamercolor{block title}{fg=structure,bg=structure!40}
   \setbeamercolor{block body}{fg=black,bg=structure!10}
   \begin{block}{{\bf Th\'eor\`eme }#1}
}{%
   \end{block}%
}


\newenvironment{proposition}[1][]{%
   \setbeamercolor{block title}{fg=structure,bg=structure!40}
   \setbeamercolor{block body}{fg=black,bg=structure!10}
   \begin{block}{{\bf Proposition }#1}
}{%
   \end{block}%
}

\newenvironment{corollaire}[1][]{%
   \setbeamercolor{block title}{fg=structure,bg=structure!40}
   \setbeamercolor{block body}{fg=black,bg=structure!10}
   \begin{block}{{\bf Corollaire }#1}
}{%
   \end{block}%
}

\newenvironment{mydefinition}[1][]{%
   \setbeamercolor{block title}{fg=structure,bg=structure!40}
   \setbeamercolor{block body}{fg=black,bg=structure!10}
   \begin{block}{{\bf Définition} #1}
}{%
   \end{block}%
}

\newenvironment{lemme}[0]{%
   \setbeamercolor{block title}{fg=structure,bg=structure!40}
   \setbeamercolor{block body}{fg=black,bg=structure!10}
   \begin{block}{\bf Lemme}
}{%
   \end{block}%
}

\newenvironment{remarque}[1][]{%
   \setbeamercolor{block title}{fg=black,bg=structure!20}
   \setbeamercolor{block body}{fg=black,bg=structure!5}
   \begin{block}{Remarque #1}
}{%
   \end{block}%
}


\newenvironment{exemple}[1][]{%
   \setbeamercolor{block title}{fg=black,bg=structure!20}
   \setbeamercolor{block body}{fg=black,bg=structure!5}
   \begin{block}{{\bf Exemple }#1}
}{%
   \end{block}%
}


\newenvironment{miniexercice}[0]{%
   \setbeamercolor{block title}{fg=structure,bg=structure!20}
   \setbeamercolor{block body}{fg=black,bg=structure!5}
   \begin{block}{Mini-exercices}
}{%
   \end{block}%
}


\newenvironment{tp}[0]{%
   \setbeamercolor{block title}{fg=structure,bg=structure!40}
   \setbeamercolor{block body}{fg=black,bg=structure!10}
   \begin{block}{\bf Travaux pratiques}
}{%
   \end{block}%
}
\newenvironment{exercicecours}[1][]{%
   \setbeamercolor{block title}{fg=structure,bg=structure!40}
   \setbeamercolor{block body}{fg=black,bg=structure!10}
   \begin{block}{{\bf Exercice }#1}
}{%
   \end{block}%
}
\newenvironment{algo}[1][]{%
   \setbeamercolor{block title}{fg=structure,bg=structure!40}
   \setbeamercolor{block body}{fg=black,bg=structure!10}
   \begin{block}{{\bf Algorithme}\hfill{\color{gray}\texttt{#1}}}
}{%
   \end{block}%
}


\setbeamertemplate{proof begin}{
   \setbeamercolor{block title}{fg=black,bg=structure!20}
   \setbeamercolor{block body}{fg=black,bg=structure!5}
   \begin{block}{{\footnotesize Démonstration}}
   \footnotesize
   \smallskip}
\setbeamertemplate{proof end}{%
   \end{block}}
\setbeamertemplate{qed symbol}{\openbox}


\makeatother
\usecolortheme[RGB={153,0,0}]{structure}


\begin{document}

%%%%%%%%%%%%%%%%%%%%%%%%%%%%%%%%%%%%%%%%%%%%%%%%%%%%%%%%%%%%%


\title{{\bf Ensembles et applications}}
\subtitle{Ensembles}

\begin{frame}
  
  \debutmontitre

  \pause

{\footnotesize
\hfill
\setbeamercovered{transparent=50}
\begin{minipage}{0.6\textwidth}
  \begin{itemize}
    \item<3-> Définir des ensembles
    \item<4-> Inclusion, union, intersection, complémentaire
    \item<5-> Règles de calcul
    \item<6-> Produit cartésien
  \end{itemize}
\end{minipage}
}

\end{frame}

\setcounter{framenumber}{0}

%%%%%%%%%%%%%%%%%%%%%%%%%%%%%%%%%%%%%%%%%%%%%%%%%%%%%%%%%%%%%%%%

\section{Motivation}

\begin{frame}

Le \evidence{paradoxe de Russell} \pause 

\begin{itemize}
  \item L'ensemble de tous les ensembles n'existe pas

\pause

  \item \emph{\og Dans une ville, le barbier rase tous ceux qui ne se rasent pas eux-mêmes.
Qui rase le barbier ? \fg\ }
\end{itemize}


\bigskip
\pause

\begin{itemize}
  \item l'ensemble des entiers naturels $\Nn =\{0,1,2,3,\ldots\}$
\pause
  \item l'ensemble des entiers relatifs $\Zz = \{\ldots, -2,-1,0,1,2,\ldots\}$
\pause
  \item l'ensemble des rationnels $\Qq = \big\{ \frac{p}{q} \mid p \in \Zz, q \in \Nn\setminus \{ 0\} \big\}$
\pause
  \item l'ensemble des réels $\Rr$, par exemple $1, \sqrt 2$, $\pi$, $\ln(2)$,\ldots
\pause
  \item l'ensemble des nombres complexes $\Cc$
\end{itemize}

\end{frame}


%%%%%%%%%%%%%%%%%%%%%%%%%%%%%%%%%%%%%%%%%%%%%%%%%%%%%%%%%%%%%%%%
\section{Ensembles}

%---------------------------------------------------------------
\section{Définir des ensembles}

\begin{frame}

\begin{itemize}
  \item Un \defi{ensemble} est une collection d'éléments

\pause

  \item Exemples : $\{ 0, 1 \}$ \; ; \; $\{ \text{rouge}, \text{noir} \}$ 
  \; ; \; $\{0, 1, 2, 3,\ldots\} = \Nn$

\pause

  \item L'\defi{ensemble vide} $\varnothing$ est l'ensemble ne contenant aucun élément

\pause

  \item \myboxinline{$x \in E$} si $x$ est un élément de $E$; \; la négation est $x \notin E$

\pause

  \item Autre façon : une collection d'éléments qui vérifient une propriété

 \pause
  
  \item Exemples
  \begin{itemize}
     \item $\big\{ x \in \Rr \mid  |x-2| < 1 \big\}$
     \item $\big\{ z \in \Cc \mid z^5=1 \big\}$
     \item $\big\{ x \in \Rr \mid 0 \le x \le 1 \big\}=[0,1]$
  \end{itemize}
   
\end{itemize}
\end{frame}

%---------------------------------------------------------------
\section{Inclusion, intersection,...}

\begin{frame}
\begin{itemize}
  \item<1-> \defi{Inclusion} $E \subset F$ si tout élément de $E$ est aussi élément de $F$

Autrement dit: \; $\forall x \in E \; (x \in F)$

\pause

On dit aussi

\assertion{$E$ est un \defi{sous-ensemble} de $F$} ou \assertion{$E$ est une \defi{partie} de $F$}

\bigskip
\pause

  \item \defi{Égalité}  $E = F$ si et seulement si $E \subset F$ et $F \subset E$

\bigskip
\pause

  \item $\mathcal{P}(E)$ : \defi{ensemble des parties} de $E$

\bigskip
\pause

Exemple si $E= \{1,2,3\}$ 
$$\mathcal{P}(\{1,2,3\}) = 
\big\{ \varnothing, \{1\}, \{2\}, \{3\}, \{1,2\}, \{1,3\}, \{2,3\}, \{1,2,3\} \big\}$$
\end{itemize}
\end{frame}

%---------------------------------------------------------------

\begin{frame}
\begin{itemize}
  \item \defi{Complémentaire} si $A \subset E$
\myboxinline{$\complement_E A = \big\{ x \in E \mid x \notin A \big\}$}

\smallskip
\pause

\begin{minipage}{0.5\textwidth}
\myfigure{1}{
\tikzinput{fig_ensembles01} 
}  
\end{minipage}
\pause
\begin{minipage}{0.4\textwidth}
Noté aussi $E \setminus A$ \ \  ou \ \ $\complement A$ 
\end{minipage}


\pause

  \item \defi{Union}
  \myboxinline{$A \cup B = \big\{ x \in E \mid x \in A \ \text{ ou } \ x \in B \big\}$}

\pause

\myfigure{1}{
\tikzinput{fig_ensembles02} 
\qquad
\uncover<7->{
\tikzinput{fig_ensembles03}}
}

  \pause
  \item \defi{Intersection} 
\myboxinline{$A \cap B = \big\{ x \in E \mid x \in A \ \text{ et } \  x \in B \big\}$}

\end{itemize}

\end{frame}

%---------------------------------------------------------------
\section{Règles de calculs}

\begin{frame}

\begin{itemize}
  \item $A \cap (B \cap C) = (A \cap B) \cap C$ 
\pause 
  \item $A \cup (B \cup C) = (A \cup B) \cup C$  
\end{itemize}

\pause
\medskip

\begin{itemize}
  \item $A \cap (B \cup C) = (A \cap B) \cup (A \cap C)$
\pause
  \item $A \cup (B \cap C) = (A \cup B) \cap (A \cup C)$
\end{itemize}

\pause
\medskip

\begin{itemize}
  \item $\complement \left( \complement A \right) = A$
\pause
  \item $A \subset B \quad \Longleftrightarrow \quad \complement B \subset \complement A$
\end{itemize}

\end{frame}

%---------------------------------------------------------------------------------

\begin{frame}

$$\complement \left( A \cap B \right) = \complement A \cup \complement B 
\qquad \qquad
\pause 
\complement \left( A \cup B \right) = \complement A \cap \complement B$$

\pause
\vspace*{-6mm}

\myfigure{1}{
\tikzinput{fig_ensembles04a} 
\quad\pause
\tikzinput{fig_ensembles04b} 
}
\pause
\myfigure{1}{
\tikzinput{fig_ensembles04c} 
\quad\pause
\tikzinput{fig_ensembles04d} 
}

\end{frame}

%---------------------------------------------------------------------------------

\begin{frame}
\[
A \cap (B \cup C) = (A \cap B) \cup (A \cap C)
\]
\pause
\begin{proof}
$\hphantom{\iff\ } x \in A \cap (B \cup C)$ \pause

$\iff x \in A \text{ et } x \in (B \cup C)$ \pause

$\iff  x \in A \text{ et } (x \in B \text{ ou } x \in C)$ \pause

$\iff (x \in A \text{ et } x\in B) \text{ ou } (x \in A \text{ et } x \in C)$ \pause

$\iff (x \in A \cap B) \text{ ou } (x \in A \cap C)$ \pause

$\iff x \in (A\cap B) \cup (A\cap C)$
\end{proof}

\end{frame}

%---------------------------------------------------------------------------------

\begin{frame}
$$\complement \left( A \cap B \right) = \complement A \cup \complement B$$  \pause
\begin{proof}
$\hphantom{\iff\ }x \in \complement \left( A \cap B \right)$ \pause

$\iff x \notin \left( A \cap B \right)$ \pause

$\iff \text{non} \big(x \in A \cap B\big)$ \pause

$\iff \text{non} \big(x \in A \text{ et } x \in B\big)$ \pause

$\iff \text{non} (x \in A) \text{ ou } \text{non} (x \in B)$ \pause

$\iff x \notin A \text{ ou } x\notin B$ \pause

$\iff x \in \complement A \cup \complement B$
\end{proof}

\end{frame}

%---------------------------------------------------------------
\section{Produit cartésien}

\begin{frame}

\begin{minipage}{0.7\textwidth}
\defi{Produit cartésien} $E \times F$ 
c'est l'ensemble des \\ couples $(x,y)$ où $x \in E$ et $y \in F$

\pause
\bigskip

\begin{exemple}
{\small
\begin{enumerate}
  \item $\Rr^2 = \Rr \times \Rr= \big\{ (x,y) \mid x,y \in \Rr \big\}$

\pause

  \item $[0,1] \times \Rr = \big\{ (x,y) \mid 0 \le x \le 1, y \in \Rr \big\}$ 


\uncover<5->{
  \item $[0,1] \times [0,1] \times [0,1] = \big\{ (x,y,z) \mid 0 \le x,y,z \le 1 \big\}$
}

\end{enumerate}
}
\end{exemple}  
\end{minipage}
\begin{minipage}{0.29\textwidth}
\uncover<4->{\myfigure{1}{
\tikzinput{fig_ensembles05} 
}}
\pause
\uncover<6->{\myfigure{1}{
\tikzinput{fig_ensembles06} 
}}  
\end{minipage}

\end{frame}

%---------------------------------------------------------------
\section{Mini-exercices}

\begin{frame}
\begin{miniexercice}
\begin{enumerate}

  \item En utilisant les définitions, montrer : $A \neq B$ si et seulement s'il existe $a \in A \setminus B$
  ou $b \in B \setminus A$.

  \item Énumérer $\mathcal{P}(\{1,2,3,4\})$.

  \item Montrer $A \cup (B \cap C) = (A \cup B) \cap (A \cup C)$ et
$\complement \left( A \cup B \right) = \complement A \cap \complement B$.

  \item Énumérer $\{1,2,3\} \times \{1,2,3,4\}$.

  \item Représenter les sous-ensembles de $\Rr^2$ suivants :
$\big(]0,1[ \cup [2,3[\big) \times [-1,1]$,
$\big( \Rr \setminus (]0,1[ \cup [2,3[\big) \times \big( (\Rr \setminus[-1,1]) \cap [0,2]  \big)$.
\end{enumerate}
\end{miniexercice}
\end{frame}


\end{document}