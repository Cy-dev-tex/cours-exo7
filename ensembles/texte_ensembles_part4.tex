
%%%%%%%%%%%%%%%%%% PREAMBULE %%%%%%%%%%%%%%%%%%


\documentclass[12pt]{article}

\usepackage{amsfonts,amsmath,amssymb,amsthm}
\usepackage[utf8]{inputenc}
\usepackage[T1]{fontenc}
\usepackage[francais]{babel}


% packages
\usepackage{amsfonts,amsmath,amssymb,amsthm}
\usepackage[utf8]{inputenc}
\usepackage[T1]{fontenc}
%\usepackage{lmodern}

\usepackage[francais]{babel}
\usepackage{fancybox}
\usepackage{graphicx}

\usepackage{float}

%\usepackage[usenames, x11names]{xcolor}
\usepackage{tikz}
\usepackage{datetime}

\usepackage{mathptmx}
%\usepackage{fouriernc}
%\usepackage{newcent}
\usepackage[mathcal,mathbf]{euler}

%\usepackage{palatino}
%\usepackage{newcent}


% Commande spéciale prompteur

%\usepackage{mathptmx}
%\usepackage[mathcal,mathbf]{euler}
%\usepackage{mathpple,multido}

\usepackage[a4paper]{geometry}
\geometry{top=2cm, bottom=2cm, left=1cm, right=1cm, marginparsep=1cm}

\newcommand{\change}{{\color{red}\rule{\textwidth}{1mm}\\}}

\newcounter{mydiapo}

\newcommand{\diapo}{\newpage
\hfill {\normalsize  Diapo \themydiapo \quad \texttt{[\jobname]}} \\
\stepcounter{mydiapo}}


%%%%%%% COULEURS %%%%%%%%%%

% Pour blanc sur noir :
%\pagecolor[rgb]{0.5,0.5,0.5}
% \pagecolor[rgb]{0,0,0}
% \color[rgb]{1,1,1}



%\DeclareFixedFont{\myfont}{U}{cmss}{bx}{n}{18pt}
\newcommand{\debuttexte}{
%%%%%%%%%%%%% FONTES %%%%%%%%%%%%%
\renewcommand{\baselinestretch}{1.5}
\usefont{U}{cmss}{bx}{n}
\bfseries

% Taille normale : commenter le reste !
%Taille Arnaud
%\fontsize{19}{19}\selectfont

% Taille Barbara
%\fontsize{21}{22}\selectfont

%Taille François
%\fontsize{25}{30}\selectfont

%Taille Pascal
%\fontsize{25}{30}\selectfont

%Taille Laura
%\fontsize{30}{35}\selectfont


%\myfont
%\usefont{U}{cmss}{bx}{n}

%\Huge
%\addtolength{\parskip}{\baselineskip}
}


% \usepackage{hyperref}
% \hypersetup{colorlinks=true, linkcolor=blue, urlcolor=blue,
% pdftitle={Exo7 - Exercices de mathématiques}, pdfauthor={Exo7}}


%section
% \usepackage{sectsty}
% \allsectionsfont{\bf}
%\sectionfont{\color{Tomato3}\upshape\selectfont}
%\subsectionfont{\color{Tomato4}\upshape\selectfont}

%----- Ensembles : entiers, reels, complexes -----
\newcommand{\Nn}{\mathbb{N}} \newcommand{\N}{\mathbb{N}}
\newcommand{\Zz}{\mathbb{Z}} \newcommand{\Z}{\mathbb{Z}}
\newcommand{\Qq}{\mathbb{Q}} \newcommand{\Q}{\mathbb{Q}}
\newcommand{\Rr}{\mathbb{R}} \newcommand{\R}{\mathbb{R}}
\newcommand{\Cc}{\mathbb{C}} 
\newcommand{\Kk}{\mathbb{K}} \newcommand{\K}{\mathbb{K}}

%----- Modifications de symboles -----
\renewcommand{\epsilon}{\varepsilon}
\renewcommand{\Re}{\mathop{\text{Re}}\nolimits}
\renewcommand{\Im}{\mathop{\text{Im}}\nolimits}
%\newcommand{\llbracket}{\left[\kern-0.15em\left[}
%\newcommand{\rrbracket}{\right]\kern-0.15em\right]}

\renewcommand{\ge}{\geqslant}
\renewcommand{\geq}{\geqslant}
\renewcommand{\le}{\leqslant}
\renewcommand{\leq}{\leqslant}

%----- Fonctions usuelles -----
\newcommand{\ch}{\mathop{\mathrm{ch}}\nolimits}
\newcommand{\sh}{\mathop{\mathrm{sh}}\nolimits}
\renewcommand{\tanh}{\mathop{\mathrm{th}}\nolimits}
\newcommand{\cotan}{\mathop{\mathrm{cotan}}\nolimits}
\newcommand{\Arcsin}{\mathop{\mathrm{Arcsin}}\nolimits}
\newcommand{\Arccos}{\mathop{\mathrm{Arccos}}\nolimits}
\newcommand{\Arctan}{\mathop{\mathrm{Arctan}}\nolimits}
\newcommand{\Argsh}{\mathop{\mathrm{Argsh}}\nolimits}
\newcommand{\Argch}{\mathop{\mathrm{Argch}}\nolimits}
\newcommand{\Argth}{\mathop{\mathrm{Argth}}\nolimits}
\newcommand{\pgcd}{\mathop{\mathrm{pgcd}}\nolimits} 

\newcommand{\Card}{\mathop{\text{Card}}\nolimits}
\newcommand{\Ker}{\mathop{\text{Ker}}\nolimits}
\newcommand{\id}{\mathop{\text{id}}\nolimits}
\newcommand{\ii}{\mathrm{i}}
\newcommand{\dd}{\mathrm{d}}
\newcommand{\Vect}{\mathop{\text{Vect}}\nolimits}
\newcommand{\Mat}{\mathop{\mathrm{Mat}}\nolimits}
\newcommand{\rg}{\mathop{\text{rg}}\nolimits}
\newcommand{\tr}{\mathop{\text{tr}}\nolimits}
\newcommand{\ppcm}{\mathop{\text{ppcm}}\nolimits}

%----- Structure des exercices ------

\newtheoremstyle{styleexo}% name
{2ex}% Space above
{3ex}% Space below
{}% Body font
{}% Indent amount 1
{\bfseries} % Theorem head font
{}% Punctuation after theorem head
{\newline}% Space after theorem head 2
{}% Theorem head spec (can be left empty, meaning ‘normal’)

%\theoremstyle{styleexo}
\newtheorem{exo}{Exercice}
\newtheorem{ind}{Indications}
\newtheorem{cor}{Correction}


\newcommand{\exercice}[1]{} \newcommand{\finexercice}{}
%\newcommand{\exercice}[1]{{\tiny\texttt{#1}}\vspace{-2ex}} % pour afficher le numero absolu, l'auteur...
\newcommand{\enonce}{\begin{exo}} \newcommand{\finenonce}{\end{exo}}
\newcommand{\indication}{\begin{ind}} \newcommand{\finindication}{\end{ind}}
\newcommand{\correction}{\begin{cor}} \newcommand{\fincorrection}{\end{cor}}

\newcommand{\noindication}{\stepcounter{ind}}
\newcommand{\nocorrection}{\stepcounter{cor}}

\newcommand{\fiche}[1]{} \newcommand{\finfiche}{}
\newcommand{\titre}[1]{\centerline{\large \bf #1}}
\newcommand{\addcommand}[1]{}
\newcommand{\video}[1]{}

% Marge
\newcommand{\mymargin}[1]{\marginpar{{\small #1}}}



%----- Presentation ------
\setlength{\parindent}{0cm}

%\newcommand{\ExoSept}{\href{http://exo7.emath.fr}{\textbf{\textsf{Exo7}}}}

\definecolor{myred}{rgb}{0.93,0.26,0}
\definecolor{myorange}{rgb}{0.97,0.58,0}
\definecolor{myyellow}{rgb}{1,0.86,0}

\newcommand{\LogoExoSept}[1]{  % input : echelle
{\usefont{U}{cmss}{bx}{n}
\begin{tikzpicture}[scale=0.1*#1,transform shape]
  \fill[color=myorange] (0,0)--(4,0)--(4,-4)--(0,-4)--cycle;
  \fill[color=myred] (0,0)--(0,3)--(-3,3)--(-3,0)--cycle;
  \fill[color=myyellow] (4,0)--(7,4)--(3,7)--(0,3)--cycle;
  \node[scale=5] at (3.5,3.5) {Exo7};
\end{tikzpicture}}
}



\theoremstyle{definition}
%\newtheorem{proposition}{Proposition}
%\newtheorem{exemple}{Exemple}
%\newtheorem{theoreme}{Théorème}
\newtheorem{lemme}{Lemme}
\newtheorem{corollaire}{Corollaire}
%\newtheorem*{remarque*}{Remarque}
%\newtheorem*{miniexercice}{Mini-exercices}
%\newtheorem{definition}{Définition}




%definition d'un terme
\newcommand{\defi}[1]{{\color{myorange}\textbf{\emph{#1}}}}
\newcommand{\evidence}[1]{{\color{blue}\textbf{\emph{#1}}}}



 %----- Commandes divers ------

\newcommand{\codeinline}[1]{\texttt{#1}}

%%%%%%%%%%%%%%%%%%%%%%%%%%%%%%%%%%%%%%%%%%%%%%%%%%%%%%%%%%%%%
%%%%%%%%%%%%%%%%%%%%%%%%%%%%%%%%%%%%%%%%%%%%%%%%%%%%%%%%%%%%%

\begin{document}

\debuttexte

%%%%%%%%%%%%%%%%%%%%%%%%%%%%%%%%%%%%%%%%%%%%%%%%%%%%%%%%%%%
\diapo

\change

Cette leçon est consacrée aux ensembles finis.

\change

Nous parlerons de 

Cardinal

\change

d'injections, surjections, bijections quand les ensembles sont finis

\change

Nous compterons les nombres d'applications

\change

puis les nombres de sous-ensembles

\change

Nous terminons 
par les coefficients du binôme de Newton

\change

afin d'énoncer la formule du binôme.


%%%%%%%%%%%%%%%%%%%%%%%%%%%%%%%%%%%%%%%%%%%%%%%%%%%%%%%%%%%
\diapo

Un ensemble $E$ est dit \defi{fini} s'il existe un entier $n$ et
une bijection de $E$ vers l'ensemble $\{1,2,\ldots,n\}$

\change

S'il existe ce $n$ est unique, c'est le \defi{cardinal} de $E$ (ou le \defi{nombre d'éléments} de $E$)

on le note $\Card E$

\change

Par exemple

l'ensemble $\{\text{rouge},\text{noir}\}$ est en bijection avec $\{1,2\}$ et est donc de cardinal $=2$

\change  

Bien sûr tous les ensembles ne sont pas des ensembles fini, par exemple
 $\Nn$ contient une infinité d'éléments.

\change

Enfin nous définirons le cardinal de l'ensemble vide comme égal à $0$.


%%%%%%%%%%%%%%%%%%%%%%%%%%%%%%%%%%%%%%%%%%%%%%%%%%%%%%%%%%%
\diapo


Commençons par des propriétés élémentaires.


Soit $A$ est un ensemble fini et $B$ une partie de $A$


Alors d'une part $B$ est aussi un ensemble fini, son cardinal étant plus petit que celui de $A$

\change

D'autre part $\Card (A \setminus B) = \Card A - \Card B$

\change

Autre situations on part de $A, B$ deux ensembles finis 

D'abord si $A$ et $B$ ne s'intersectent pas $\Card (A \cup B) = \Card A + \Card B$

Dans la situation générale 
de deux ensembles $A,B$ finis quelconques

on a la formule des $4$ cardinaux :

$\Card (A \cup B) = \Card A + \Card B - \Card (A\cap B)$

\change

Le dessin illustre cette dernière formule.

Pour compter tous les éléments : 

on compte les élément de $A$, on compte les éléments de $B$,

mais on a compté deux fois les éléments de $A\cap B$,

donc on retire une fois les éléments de $A \cap B$.

%%%%%%%%%%%%%%%%%%%%%%%%%%%%%%%%%%%%%%%%%%%%%%%%%%%%%%%%%%%
\diapo

Regardons maintenant les applications entre deux ensembles finis.


Soit $E,F$ deux ensembles finis et $f$ une application de  $E$ vers $F$.

Etre injectif, surjectif ou bijectif impose des conditions sur les cardinaux.

En détails :

Si $f$ est injective alors $\Card E \le \Card F$

\change

Si $f$ est surjective alors $\Card E \ge \Card F$

\change

Si $f$ est bijective alors $\Card E = \Card F$


%%%%%%%%%%%%%%%%%%%%%%%%%%%%%%%%%%%%%%%%%%%%%%%%%%%%%%%%%%%
\diapo

Nous allons énoncer une sorte de réciproque.

Prenons encore $E,F$ deux ensembles finis et $f$ une application de $E$ vers $F$.

Cette fois nous faisons l'hypothèse 
que l'ensemble de départ et celui d'arrivée ont le même cardinal.

Dans ce cas les trois assertions que je vais énoncer sont équivalentes :

\change

1. $f$ est injective

2. $f$ est surjective

3. $f$ est bijective


Je répète *si* Cardinal de $E$ égale Cardinal de $F$ alors 
$f$ injective équivaut à $f$ surjective équivaut à $f$ bijective.

\change

Nous n'allons pas faire la preuve ici mais il est intéressant de noter que pour la démonstration
on prouve d'abord l'implication $ (i) \implies (ii)$
puis l'implication $(ii)  \implies (iii)$ puis l'implication $(iii) \implies (i) $.

La boucle est bouclée et cela prouve les équivalences.

\change

A l'aide d'une application entre deux ensembles finis, montrez le principe des tiroirs :

Si l'on range dans $k$ tiroirs, $n$ paires de chaussettes (avec $n$ strictement plus grand que $k$)
alors il existe un tiroir contenant au moins deux paires
de chaussettes


Malgré sa formulation amusante, c'est une proposition souvent utile !

%%%%%%%%%%%%%%%%%%%%%%%%%%%%%%%%%%%%%%%%%%%%%%%%%%%%%%%%%%%
\diapo

Reprenons deux ensembles finis $E$ et $F$.
$E$ est de cardinal $n$ et $F$ de cardinal $p$


Combien peut-il il y avoir d'applications différentes qui vont de $E$ vers $F$.

\change

Le nombre d'applications de $E$ dans $F$ est $p^n$


\change

Autrement dit c'est $(\Card F)^{\Card E}$


\change

On en déduit le nombre d'applications de $E$ dans lui même : c'est $n^n$.

\change

Par exemple si $E$ est de cardinal $5$ alors il y $5^5$ applications
différentes de $E$ dans lui même





%%%%%%%%%%%%%%%%%%%%%%%%%%%%%%%%%%%%%%%%%%%%%%%%%%%%%%%%%%%
\diapo

Si l'on ne souhaite plus compter toutes les applications 
mais seulement les bijections alors le résultat est le suivant :

\change


Le nombre de bijections d'un ensemble de cardinal $n$ dans lui-même
est $n!$


Je vous rappelle que $n!=1\times 2 \times 3 \times \cdots \times n$

\change

Par exemple parmi les $5^5$ applications d'un ensemble à $5$ éléments dans lui-même

il y en a $5! = 120$ qui sont bijectives


\change

Une preuve rapide de la démonstration est la suivante :

Si l'ensemble $E$ est $\{1,2,,\ldots,n\}$
et si nous souhaitons construire une bijection $f$


Alors l'image $f(1)$ peut prendre n'importe quelle valeur  : il y a donc $n$ possibilités


\change

Pour l'image $f(2)$ il ne faut pas choisir la même valeur que $f(1)$, tous les autres choix sont possibles
il y a $n-1$ possibilités.

\change

ainsi de suite

\change

Jusqu'à l'image $f(n)$ où il ne reste qu'une seule possibilité.

Conclusion il a $n$ choix fois $n-1$ choix fois $n-2$ choix fois  etc 
il y a bien $n!$ bijections possibles.

%%%%%%%%%%%%%%%%%%%%%%%%%%%%%%%%%%%%%%%%%%%%%%%%%%%%%%%%%%%
\diapo

Nous allons maintenons compter le nombre de parties inclues dans un ensemble $E$ à $n$ éléments.


Il y a $2^{\Card E}$ sous-ensembles de $E$ 

\change

Une autre façon de le dire c'est d'écrire

$\Card \mathcal{P}(E) = 2^n$

le cardinal de l'ensemble des parties de $E$ égal $2^n$.

\change


Si $E$ est l'ensemble $\{1,2,3,4,5\}$ alors il y a $2^5 = 32$ sous-ensembles

\change

C'est un bon exercice de les énumérer :

\begin{itemize}
  \item tout d'abord l'ensemble vide : $\varnothing$

\change

  \item ensuite il y a $5$ singletons : $\{1\}, \{2\},\{3\},\ldots$

\change

  \item puis $10$ paires : $\{1,2\}, \{1,3\}, \ldots, \{2,3\},\{2,4\}, \ldots$

\change

 
  \item et aussi $10$ triplets : $\{1,2,3\},\{1,2,4\},\ldots$

\change

  \item ensuite $5$ ensembles possédant $4$ éléments : $\{1,2,3,4\},\ldots$

\change

  \item enfin n'oubliez pas $E$ tout entier.
\end{itemize}

Nous avons bien trouvé les $32$ parties de $E$.


%%%%%%%%%%%%%%%%%%%%%%%%%%%%%%%%%%%%%%%%%%%%%%%%%%%%%%%%%%%
\diapo

Nous allons noter le nombre de parties à $k$ éléments d'un ensemble à $n$ éléments 

par $\binom{n}{k}$ [$k$ parmi $n$] ou aussi $C_n^k$

Nous appelons ces nombres les coefficients du binôme de Newton.

\change

Par exemple si l'on cherche les parties ayant $k=2$ éléments dans un ensemble ayant $n=3$ éléments : 
on trouve les paires $\{1,2\}$, $\{1,3\}$, $\{2,3\}$

Il y a donc bien $3$ parties à $2$ éléments et donc $\binom{3}{2} = 3$

\change

Autres exemples nous avons déjà énuméré les parties d'un ensemble ayant $n=5$ éléments

Nous avons $\binom{5}{0} = 1$
car la seule partie ayant $0$ éléments est l'ensemble vide.

$\binom{5}{1} = 5$ car nous avons trouvé $5$ singletons;

$\binom{5}{2} = 10$ car nous avons trouvé $10$ paires.

etc $\binom{5}{3} = 10$

$\binom{5}{4} = 5$

$\binom{5}{5} = 1$ car il n'y a que l'ensemble tout entier qui ait $5$ éléments.


\change

Voici quelques propriétés des coefficients du binômes


 $\binom{n}{0}=1$ \quad  $\binom{n}{1}=n$ \quad  $\binom{n}{n}=1$

Par exemple : $\binom{n}{1}=n$ car il y a $n$ singletons.

\change

Ensuite le coefficient $\binom{n}{n-k} = \binom{n}{k}$

En effet compter le nombre de parties $A \subset E$ ayant $k$ éléments revient aussi à compter
le nombre de parties de la forme $\complement A$ (qui ont $n-k$ éléments).


\change

$\binom{n}{0}+\binom{n}{1}+\cdots+\binom{n}{n} = 2^n$

Cette formule exprime que 
faire la somme du nombre de parties à $k$ éléments, pour $k=0,\ldots,n$, 
revient à compter toutes les parties de $E$.



%%%%%%%%%%%%%%%%%%%%%%%%%%%%%%%%%%%%%%%%%%%%%%%%%%%%%%%%%%%
\diapo

Voici une formule qui va nous permettre de calculer les coefficients $k$ parmi $n$.

$\binom n k = \binom{n-1}{k} + \binom{n-1}{k-1}$

\change

La preuve est la suivante, considérons un ensemble $E$ ayant $n$ éléments

et fixons un élément $a$ de $E$.

Alors la partie $E'= E \setminus \{a\}$ contient $n-1$ éléments.

\change


Considérons toutes les parties $A$ inclues dans $E$ ayant $k$ élément.

Il y en a de deux sortes :

\change

celles qui ne contiennent pas l'élément $a$: 

\change 

$A$ est alors une partie à $k$ éléments dans l'ensemble $E'$ qui a $n-1$ éléments. 

Il y a en a donc $\binom{n-1}{k}$ parties possibles.
  
\change

la deuxième sorte sont les parties $A$ qui contiennent $a$ : 

\change

on peut donc les décomposer sous la forme $A = \{a\} \cup A'$ 

avec $A'$ une partie à $k-1$ éléments dans l'ensemble $E'$ qui a $n-1$ éléments. 

Il y a $\binom{n-1}{k-1}$ telle partie $A'$ et donc telle parties $A$.

\change

Bilan le nombre de parties à $k$ élément dans un ensemble à $n$ éléments
est donc $\binom{n-1}{k}$ (pour le premier cas) plus $\binom{n-1}{k-1}$ pour
le second cas.

On trouve bien la formule annoncée.



%%%%%%%%%%%%%%%%%%%%%%%%%%%%%%%%%%%%%%%%%%%%%%%%%%%%%%%%%%%
\diapo

Nous allons calculer les coefficients $k$ parmi $n$ à l'aide du triangle de Pascal.

Nous allons remplir le tableau des coefficients du binôme

\change

Tout d'abord remarquons que ces coefficients comptent le nombre
 d'ensembles à $k$ éléments parmi un ensemble à $n$ éléments donc 

$k$ doit être plus petit que $n$.

Ainsi on doit remplir cette partie du tableau, le coefficient $k$ parmi $n$
étant à la ligne $n$ et la colonne $k$.

\change

Nous avons déjà vu que 
$\binom{n}{0}=1$, $\binom{n}{n}=1$

donc la première colonne et la diagonale sont des $1$.


\change

Pour calculer les autres coefficients nous utilisons la formule de récurrence 

$\binom n k = \binom{n-1}{k-1} + \binom{n-1}{k}$

Disposé de cette façon cela permet de calculer un coefficient de la ligne $n$
en fonction de deux coefficients de la ligne du dessus.

\change

Comment calculer le coefficient $1$ parmi $2$.
Par la formule c'est la somme $1-1$ parmi $2-1$ plus $1$ parmi $2-1$.  

\change

Donc $1$ parmi $2$ égal $0$ parmi $1$ plus $1$ parmi $1$.

Cela tombe bien on connaît ces coefficients.

Regardons sur le tableau, 

$1$ parmi $2$ égale $0$ parmi $1$ plus $1$ parmi $1$
donc $1$ parmi $2$ égale $1+1$ donc vaut $2$.

\change

On continue avec le coefficient suivant $1$ parmi $3$
qui s'écrit en fonctions de deux coefficients juste au-dessus.

\change

On trouve $1$ parmi $3$ égale $0$ parmi $2$ plus $1$ parmi $2$
donc $1$ plus $2$.

$1$ parmi $3$ égale $3$.


\change

Même chose avec $2$ parmi $3$ 

\change

qui est $2+1$ donc $3$ également.

\change

On remplit ainsi le tableau coefficient par coefficient, ligne par ligne

\change

en effectuant à chaque fois la somme du coefficient juste au-dessus et 
de celui au-dessus à gauche.


On trouve ainsi $1, 4, 6, 4, 1$ pour la ligne $n=4$.

\change

$1, 5, 10, 10, 5, 1$ pour la ligne $n=5$.

\change

Et on peut continuer indéfiniment.




%%%%%%%%%%%%%%%%%%%%%%%%%%%%%%%%%%%%%%%%%%%%%%%%%%%%%%%%%%%
\diapo

Une autre façon de calculer les coefficients du binôme de Newton repose sur la formule suivante :

le coefficient $k$ parmi $n$ vaut

$ \binom n k = \frac{n!}{k!(n-k)!}$

\change

La preuve se fait par récurrence sur $n$.

C'est clair pour $n=1$ car nous savons que $1$ parmi $n$ vaut $n$.

\change


Si la formule est vraie au rang $n-1$ 

\change

alors écrivons $\binom n k = \binom{n-1}{k-1} + \binom{n-1}{k}$

\change

et utilisons l'hypothèse de récurrence pour les deux coefficients.

cela donne ceci 


\change

on met en facteur une bonne partie des fractions, il ne reste plus qu'à réduire au même
dénominateur pour trouver $\frac{n!}{k!(n-k)!}$.


%%%%%%%%%%%%%%%%%%%%%%%%%%%%%%%%%%%%%%%%%%%%%%%%%%%%%%%%%%%
\diapo

Nous avons tous les ingrédients pour énoncer la formule du binôme de Newton.

Etant donné deux nombres réels $a$ et $b$ et un entier $n$.
Nous allons développer $(a+b)^n$.

\change

$(a+b)^n$ vaut la somme $\sum_{k=0}^n \binom{n}{k} \ a^{n-k} \cdot b^{k}$

\change

Autrement dit l'écriture extensive est 
$(a+b)^n = \binom{n}{0}\ a^n\cdot b^0 + \binom{n}{1}\ a^{n-1}\cdot b^{1}
+ \cdots + \binom{n}{k} \ a^{n-k} \cdot b^{k}+\cdots + \binom{n}{n}\ a^0\cdot b^n$

\change

Pour $n=2$ on retrouve la formule archi-connue : $(a+b)^2= a^2 + 2ab + b^2$.

\change

Il est aussi bon de connaître $(a+b)^3 = a^3 + 3a^2b + 3ab^2 + b^3$.

\change

Si $a=1$ et $b=1$ on retrouve la formule : $\sum_{k=0}^n \binom{n}{k} = 2^n$.


%%%%%%%%%%%%%%%%%%%%%%%%%%%%%%%%%%%%%%%%%%%%%%%%%%%%%%%%%%%
\diapo

Il vous reste encore pas mal de travail avec ces mini-exercice !



\end{document}