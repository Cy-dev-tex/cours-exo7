
%%%%%%%%%%%%%%%%%% PREAMBULE %%%%%%%%%%%%%%%%%%

\documentclass[aspectratio=169,utf8]{beamer}
%\documentclass[aspectratio=169,handout]{beamer}

\usetheme{Boadilla}
%\usecolortheme{seahorse}
\usecolortheme[RGB={245,66,24}]{structure}
\useoutertheme{infolines}

% packages
\usepackage{amsfonts,amsmath,amssymb,amsthm}
\usepackage[utf8]{inputenc}
\usepackage[T1]{fontenc}
\usepackage{lmodern}

\usepackage[francais]{babel}
\usepackage{fancybox}
\usepackage{graphicx}

\usepackage{float}
\usepackage{xfrac}

%\usepackage[usenames, x11names]{xcolor}
\usepackage{tikz}
\usepackage{pgfplots}
\usepackage{datetime}



%-----  Package unités -----
\usepackage{siunitx}
\sisetup{locale = FR,detect-all,per-mode = symbol}

%\usepackage{mathptmx}
%\usepackage{fouriernc}
%\usepackage{newcent}
%\usepackage[mathcal,mathbf]{euler}

%\usepackage{palatino}
%\usepackage{newcent}
% \usepackage[mathcal,mathbf]{euler}



% \usepackage{hyperref}
% \hypersetup{colorlinks=true, linkcolor=blue, urlcolor=blue,
% pdftitle={Exo7 - Exercices de mathématiques}, pdfauthor={Exo7}}


%section
% \usepackage{sectsty}
% \allsectionsfont{\bf}
%\sectionfont{\color{Tomato3}\upshape\selectfont}
%\subsectionfont{\color{Tomato4}\upshape\selectfont}

%----- Ensembles : entiers, reels, complexes -----
\newcommand{\Nn}{\mathbb{N}} \newcommand{\N}{\mathbb{N}}
\newcommand{\Zz}{\mathbb{Z}} \newcommand{\Z}{\mathbb{Z}}
\newcommand{\Qq}{\mathbb{Q}} \newcommand{\Q}{\mathbb{Q}}
\newcommand{\Rr}{\mathbb{R}} \newcommand{\R}{\mathbb{R}}
\newcommand{\Cc}{\mathbb{C}} 
\newcommand{\Kk}{\mathbb{K}} \newcommand{\K}{\mathbb{K}}

%----- Modifications de symboles -----
\renewcommand{\epsilon}{\varepsilon}
\renewcommand{\Re}{\mathop{\text{Re}}\nolimits}
\renewcommand{\Im}{\mathop{\text{Im}}\nolimits}
%\newcommand{\llbracket}{\left[\kern-0.15em\left[}
%\newcommand{\rrbracket}{\right]\kern-0.15em\right]}

\renewcommand{\ge}{\geqslant}
\renewcommand{\geq}{\geqslant}
\renewcommand{\le}{\leqslant}
\renewcommand{\leq}{\leqslant}
\renewcommand{\epsilon}{\varepsilon}

%----- Fonctions usuelles -----
\newcommand{\ch}{\mathop{\text{ch}}\nolimits}
\newcommand{\sh}{\mathop{\text{sh}}\nolimits}
\renewcommand{\tanh}{\mathop{\text{th}}\nolimits}
\newcommand{\cotan}{\mathop{\text{cotan}}\nolimits}
\newcommand{\Arcsin}{\mathop{\text{arcsin}}\nolimits}
\newcommand{\Arccos}{\mathop{\text{arccos}}\nolimits}
\newcommand{\Arctan}{\mathop{\text{arctan}}\nolimits}
\newcommand{\Argsh}{\mathop{\text{argsh}}\nolimits}
\newcommand{\Argch}{\mathop{\text{argch}}\nolimits}
\newcommand{\Argth}{\mathop{\text{argth}}\nolimits}
\newcommand{\pgcd}{\mathop{\text{pgcd}}\nolimits} 


%----- Commandes divers ------
\newcommand{\ii}{\mathrm{i}}
\newcommand{\dd}{\text{d}}
\newcommand{\id}{\mathop{\text{id}}\nolimits}
\newcommand{\Ker}{\mathop{\text{Ker}}\nolimits}
\newcommand{\Card}{\mathop{\text{Card}}\nolimits}
\newcommand{\Vect}{\mathop{\text{Vect}}\nolimits}
\newcommand{\Mat}{\mathop{\text{Mat}}\nolimits}
\newcommand{\rg}{\mathop{\text{rg}}\nolimits}
\newcommand{\tr}{\mathop{\text{tr}}\nolimits}


%----- Structure des exercices ------

\newtheoremstyle{styleexo}% name
{2ex}% Space above
{3ex}% Space below
{}% Body font
{}% Indent amount 1
{\bfseries} % Theorem head font
{}% Punctuation after theorem head
{\newline}% Space after theorem head 2
{}% Theorem head spec (can be left empty, meaning ‘normal’)

%\theoremstyle{styleexo}
\newtheorem{exo}{Exercice}
\newtheorem{ind}{Indications}
\newtheorem{cor}{Correction}


\newcommand{\exercice}[1]{} \newcommand{\finexercice}{}
%\newcommand{\exercice}[1]{{\tiny\texttt{#1}}\vspace{-2ex}} % pour afficher le numero absolu, l'auteur...
\newcommand{\enonce}{\begin{exo}} \newcommand{\finenonce}{\end{exo}}
\newcommand{\indication}{\begin{ind}} \newcommand{\finindication}{\end{ind}}
\newcommand{\correction}{\begin{cor}} \newcommand{\fincorrection}{\end{cor}}

\newcommand{\noindication}{\stepcounter{ind}}
\newcommand{\nocorrection}{\stepcounter{cor}}

\newcommand{\fiche}[1]{} \newcommand{\finfiche}{}
\newcommand{\titre}[1]{\centerline{\large \bf #1}}
\newcommand{\addcommand}[1]{}
\newcommand{\video}[1]{}

% Marge
\newcommand{\mymargin}[1]{\marginpar{{\small #1}}}

\def\noqed{\renewcommand{\qedsymbol}{}}


%----- Presentation ------
\setlength{\parindent}{0cm}

%\newcommand{\ExoSept}{\href{http://exo7.emath.fr}{\textbf{\textsf{Exo7}}}}

\definecolor{myred}{rgb}{0.93,0.26,0}
\definecolor{myorange}{rgb}{0.97,0.58,0}
\definecolor{myyellow}{rgb}{1,0.86,0}

\newcommand{\LogoExoSept}[1]{  % input : echelle
{\usefont{U}{cmss}{bx}{n}
\begin{tikzpicture}[scale=0.1*#1,transform shape]
  \fill[color=myorange] (0,0)--(4,0)--(4,-4)--(0,-4)--cycle;
  \fill[color=myred] (0,0)--(0,3)--(-3,3)--(-3,0)--cycle;
  \fill[color=myyellow] (4,0)--(7,4)--(3,7)--(0,3)--cycle;
  \node[scale=5] at (3.5,3.5) {Exo7};
\end{tikzpicture}}
}


\newcommand{\debutmontitre}{
  \author{} \date{} 
  \thispagestyle{empty}
  \hspace*{-10ex}
  \begin{minipage}{\textwidth}
    \titlepage  
  \vspace*{-2.5cm}
  \begin{center}
    \LogoExoSept{2.5}
  \end{center}
  \end{minipage}

  \vspace*{-0cm}
  
  % Astuce pour que le background ne soit pas discrétisé lors de la conversion pdf -> png
\begin{tikzpicture}
        \fill[opacity=0,green!60!black] (0,0)--++(0,0)--++(0,0)--++(0,0)--cycle; 
\end{tikzpicture}

% toc S'affiche trop tot :
% \tableofcontents[hideallsubsections, pausesections]
}

\newcommand{\finmontitre}{
  \end{frame}
  \setcounter{framenumber}{0}
} % ne marche pas pour une raison obscure

%----- Commandes supplementaires ------

% \usepackage[landscape]{geometry}
% \geometry{top=1cm, bottom=3cm, left=2cm, right=10cm, marginparsep=1cm
% }
% \usepackage[a4paper]{geometry}
% \geometry{top=2cm, bottom=2cm, left=2cm, right=2cm, marginparsep=1cm
% }

%\usepackage{standalone}


% New command Arnaud -- november 2011
\setbeamersize{text margin left=24ex}
% si vous modifier cette valeur il faut aussi
% modifier le decalage du titre pour compenser
% (ex : ici =+10ex, titre =-5ex

\theoremstyle{definition}
%\newtheorem{proposition}{Proposition}
%\newtheorem{exemple}{Exemple}
%\newtheorem{theoreme}{Théorème}
%\newtheorem{lemme}{Lemme}
%\newtheorem{corollaire}{Corollaire}
%\newtheorem*{remarque*}{Remarque}
%\newtheorem*{miniexercice}{Mini-exercices}
%\newtheorem{definition}{Définition}

% Commande tikz
\usetikzlibrary{calc}
\usetikzlibrary{patterns,arrows}
\usetikzlibrary{matrix}
\usetikzlibrary{fadings} 

%definition d'un terme
\newcommand{\defi}[1]{{\color{myorange}\textbf{\emph{#1}}}}
\newcommand{\evidence}[1]{{\color{blue}\textbf{\emph{#1}}}}
\newcommand{\assertion}[1]{\emph{\og#1\fg}}  % pour chapitre logique
%\renewcommand{\contentsname}{Sommaire}
\renewcommand{\contentsname}{}
\setcounter{tocdepth}{2}



%------ Figures ------

\def\myscale{1} % par défaut 
\newcommand{\myfigure}[2]{  % entrée : echelle, fichier figure
\def\myscale{#1}
\begin{center}
\footnotesize
{#2}
\end{center}}


%------ Encadrement ------

\usepackage{fancybox}


\newcommand{\mybox}[1]{
\setlength{\fboxsep}{7pt}
\begin{center}
\shadowbox{#1}
\end{center}}

\newcommand{\myboxinline}[1]{
\setlength{\fboxsep}{5pt}
\raisebox{-10pt}{
\shadowbox{#1}
}
}

%--------------- Commande beamer---------------
\newcommand{\beameronly}[1]{#1} % permet de mettre des pause dans beamer pas dans poly


\setbeamertemplate{navigation symbols}{}
\setbeamertemplate{footline}  % tiré du fichier beamerouterinfolines.sty
{
  \leavevmode%
  \hbox{%
  \begin{beamercolorbox}[wd=.333333\paperwidth,ht=2.25ex,dp=1ex,center]{author in head/foot}%
    % \usebeamerfont{author in head/foot}\insertshortauthor%~~(\insertshortinstitute)
    \usebeamerfont{section in head/foot}{\bf\insertshorttitle}
  \end{beamercolorbox}%
  \begin{beamercolorbox}[wd=.333333\paperwidth,ht=2.25ex,dp=1ex,center]{title in head/foot}%
    \usebeamerfont{section in head/foot}{\bf\insertsectionhead}
  \end{beamercolorbox}%
  \begin{beamercolorbox}[wd=.333333\paperwidth,ht=2.25ex,dp=1ex,right]{date in head/foot}%
    % \usebeamerfont{date in head/foot}\insertshortdate{}\hspace*{2em}
    \insertframenumber{} / \inserttotalframenumber\hspace*{2ex} 
  \end{beamercolorbox}}%
  \vskip0pt%
}


\definecolor{mygrey}{rgb}{0.5,0.5,0.5}
\setlength{\parindent}{0cm}
%\DeclareTextFontCommand{\helvetica}{\fontfamily{phv}\selectfont}

% background beamer
\definecolor{couleurhaut}{rgb}{0.85,0.9,1}  % creme
\definecolor{couleurmilieu}{rgb}{1,1,1}  % vert pale
\definecolor{couleurbas}{rgb}{0.85,0.9,1}  % blanc
\setbeamertemplate{background canvas}[vertical shading]%
[top=couleurhaut,middle=couleurmilieu,midpoint=0.4,bottom=couleurbas] 
%[top=fondtitre!05,bottom=fondtitre!60]



\makeatletter
\setbeamertemplate{theorem begin}
{%
  \begin{\inserttheoremblockenv}
  {%
    \inserttheoremheadfont
    \inserttheoremname
    \inserttheoremnumber
    \ifx\inserttheoremaddition\@empty\else\ (\inserttheoremaddition)\fi%
    \inserttheorempunctuation
  }%
}
\setbeamertemplate{theorem end}{\end{\inserttheoremblockenv}}

\newenvironment{theoreme}[1][]{%
   \setbeamercolor{block title}{fg=structure,bg=structure!40}
   \setbeamercolor{block body}{fg=black,bg=structure!10}
   \begin{block}{{\bf Th\'eor\`eme }#1}
}{%
   \end{block}%
}


\newenvironment{proposition}[1][]{%
   \setbeamercolor{block title}{fg=structure,bg=structure!40}
   \setbeamercolor{block body}{fg=black,bg=structure!10}
   \begin{block}{{\bf Proposition }#1}
}{%
   \end{block}%
}

\newenvironment{corollaire}[1][]{%
   \setbeamercolor{block title}{fg=structure,bg=structure!40}
   \setbeamercolor{block body}{fg=black,bg=structure!10}
   \begin{block}{{\bf Corollaire }#1}
}{%
   \end{block}%
}

\newenvironment{mydefinition}[1][]{%
   \setbeamercolor{block title}{fg=structure,bg=structure!40}
   \setbeamercolor{block body}{fg=black,bg=structure!10}
   \begin{block}{{\bf Définition} #1}
}{%
   \end{block}%
}

\newenvironment{lemme}[0]{%
   \setbeamercolor{block title}{fg=structure,bg=structure!40}
   \setbeamercolor{block body}{fg=black,bg=structure!10}
   \begin{block}{\bf Lemme}
}{%
   \end{block}%
}

\newenvironment{remarque}[1][]{%
   \setbeamercolor{block title}{fg=black,bg=structure!20}
   \setbeamercolor{block body}{fg=black,bg=structure!5}
   \begin{block}{Remarque #1}
}{%
   \end{block}%
}


\newenvironment{exemple}[1][]{%
   \setbeamercolor{block title}{fg=black,bg=structure!20}
   \setbeamercolor{block body}{fg=black,bg=structure!5}
   \begin{block}{{\bf Exemple }#1}
}{%
   \end{block}%
}


\newenvironment{miniexercice}[0]{%
   \setbeamercolor{block title}{fg=structure,bg=structure!20}
   \setbeamercolor{block body}{fg=black,bg=structure!5}
   \begin{block}{Mini-exercices}
}{%
   \end{block}%
}


\newenvironment{tp}[0]{%
   \setbeamercolor{block title}{fg=structure,bg=structure!40}
   \setbeamercolor{block body}{fg=black,bg=structure!10}
   \begin{block}{\bf Travaux pratiques}
}{%
   \end{block}%
}
\newenvironment{exercicecours}[1][]{%
   \setbeamercolor{block title}{fg=structure,bg=structure!40}
   \setbeamercolor{block body}{fg=black,bg=structure!10}
   \begin{block}{{\bf Exercice }#1}
}{%
   \end{block}%
}
\newenvironment{algo}[1][]{%
   \setbeamercolor{block title}{fg=structure,bg=structure!40}
   \setbeamercolor{block body}{fg=black,bg=structure!10}
   \begin{block}{{\bf Algorithme}\hfill{\color{gray}\texttt{#1}}}
}{%
   \end{block}%
}


\setbeamertemplate{proof begin}{
   \setbeamercolor{block title}{fg=black,bg=structure!20}
   \setbeamercolor{block body}{fg=black,bg=structure!5}
   \begin{block}{{\footnotesize Démonstration}}
   \footnotesize
   \smallskip}
\setbeamertemplate{proof end}{%
   \end{block}}
\setbeamertemplate{qed symbol}{\openbox}


\makeatother
\usecolortheme[RGB={153,0,0}]{structure}

\begin{document}


%%%%%%%%%%%%%%%%%%%%%%%%%%%%%%%%%%%%%%%%%%%%%%%%%%%%%%%%%%%%%
%%%%%%%%%%%%%%%%%%%%%%%%%%%%%%%%%%%%%%%%%%%%%%%%%%%%%%%%%%%%%

\title{{\bf Ensembles et applications}}
\subtitle{Injection, surjection, bijection}


\begin{frame}
  
  \debutmontitre

  \pause

{\footnotesize
\hfill
\setbeamercovered{transparent=50}
\begin{minipage}{0.6\textwidth}
  \begin{itemize}
    \item<3-> Injection, surjection
    \item<4-> Bijection
  \end{itemize}
\end{minipage}
}

\end{frame}

\setcounter{framenumber}{0}

%---------------------------------------------------------------

\section{Injection, surjection}


\begin{frame}

Soit $f : E \to F$
\begin{mydefinition}
$f$ est \defi{injective} si pour tout $x,x' \in E$ avec $f(x)=f(x')$ alors $x=x'$
\end{mydefinition}

\pause

Autrement dit:
\myboxinline{$\forall x, x' \in E \quad \big( f(x)=f(x') \implies x=x'\big)$}

\bigskip
\pause

\myfigure{1}{
\tikzinput{fig_ensembles11a} 
\qquad
\pause
\tikzinput{fig_ensembles11b} 
}

\end{frame} 


%---------------------------------------------------------------

\begin{frame}
Soit $f : E \to F$ 
\begin{mydefinition}
$f$ est \defi{surjective} si pour tout $y \in F$ il existe $x \in E$ tel que $y=f(x)$
\end{mydefinition}

\medskip
\pause

Autrement dit :
\myboxinline{$\forall y \in F \quad \exists x \in E \quad \big( y = f(x) \big)$}

\medskip
\pause

 Une autre formulation : $f$ est surjective si et seulement si $f(E)=F$

\pause

\myfigure{1}{
\tikzinput{fig_ensembles11c} 
\qquad \pause
\tikzinput{fig_ensembles11d} 
}
\end{frame}


\begin{frame}

\begin{remarque}

\myfigure{1}{
\tikzinput{fig_ensembles12a} 
\quad
\tikzinput{fig_ensembles12b} 
}

%Reformulation :
\begin{itemize}



\pause

  \item $f$ est \emph{injective} si et seulement si tout élément $y$ de $F$ a \emph{au plus} $1$ antécédent (éventuellement aucun)

\pause

  \item $f$ est \emph{surjective} si et seulement si tout élément $y$ de $F$ a \emph{au moins} $1$ antécédent
\end{itemize}
\end{remarque}
\end{frame}

\begin{frame}

\myfigure{1}{
\tikzinput{fig_ensembles12c} 
\qquad 
\uncover<2->{\tikzinput{fig_ensembles12d}}
}

\centerline{Fonctions non injectives}

\end{frame} 


\begin{frame}
\myfigure{1}{
\tikzinput{fig_ensembles11e} 
 \qquad \qquad
\uncover<2->{\tikzinput{fig_ensembles11f}} 
}

\centerline{Fonctions non surjectives}
\end{frame}

%---------------------------------------------------------------


\begin{frame}
\begin{exemple}
Soit $f_1 : \Nn \to \Qq$, \quad $f_1(x)= \frac{1}{1+x}$ \pause
  \begin{itemize}
     \item $f_1$ est injective: \pause
     soit $x,x'\in \Nn$ tels que $f_1(x)=f_1(x')$. \pause Alors $\frac{1}{1+x}=\frac{1}{1+x'}$\pause,
     donc $1+x=1+x'$ \pause et donc $x=x'$ \pause
     \item $f_1$ n'est pas surjective. \pause Comme on a toujours  $f_1(x) \le 1$ alors par exemple $y=2$ n'a pas d'antécédent 
  \end{itemize}
\end{exemple}

\pause

\begin{exemple}
 Soit $f_2 : \Zz \to \Nn$, \quad $f_2(x)=x^2$ 
 \pause
  \begin{itemize}
     \item $f_2$ n'est pas injective. \pause
En effet on peut trouver deux éléments $x,x' \in \Zz$ différents tels que $f_2(x)=f_2(x')$.
\pause
Par exemple $x=2$, $x'=-2$ 
\pause
     \item $f_2$ n'est pas non plus surjective, car il existe des éléments $y \in \Nn$ sans aucun antécédent. 
\pause
Par exemple $y=3$ : un antécédent $x$ de $y=3$ satisfait
%si $y=3$ avait un antécédent $x$ par $f_2$, nous aurions $f_2(x)=y$, c'est-à-dire
$f_2(x)=x^2=3$, d'où $x = \pm \sqrt 3$. Mais alors $x \notin \Zz$ 
  \end{itemize}
\end{exemple}
\end{frame}

%---------------------------------------------------------------

\section{Bijection}

\begin{frame}
\begin{mydefinition}
$f$ est \defi{bijective} si elle injective \emph{et} surjective

\pause

Cela équivaut à :

pour tout $y \in F$ il existe un unique $x \in E$ tel que $y=f(x)$
\end{mydefinition}

\pause

Autres formulations équivalentes :
\begin{itemize}
  \item \myboxinline{$\forall y \in F \quad \exists! x \in E \quad \big( y = f(x) \big)$}
\pause
  \item tout élément de $F$ a un unique antécédent par $f$
\end{itemize}

\pause

\myfigure{1}{
\tikzinput{fig_ensembles13a} 
\qquad \pause
\tikzinput{fig_ensembles13b} 
}

\end{frame}

%---------------------------------------------------------------

\begin{frame}
\begin{proposition}
\label{prop:bij1}
\begin{enumerate}
  \item L'application $f : E \to F$ est bijective si et seulement si il existe $g : F \to E$
telle que $f \circ g = \id_F$ et $g \circ f = \id_E$

\pause

  \item Si $f$ est bijective alors $g$ est unique et est bijective,
c'est la \defi{bijection réciproque} de $f$ et est notée $f^{-1}$

\pause

De plus $\left( f^{-1} \right)^{-1} = f$
\end{enumerate}
\end{proposition}
% Preuve repoussée à la fin 

\pause
\bigskip

\begin{remarque}
\begin{itemize}
  \item $f \circ g = \id_F$ \qquad signifie \qquad  $\forall y \in F  \quad f\big(g(y)\big) = y$ \pause
  \item $g \circ f = \id_E$ \qquad signifie \qquad $\forall x \in E \quad g\big(f(x)\big) = x$ \pause
\end{itemize}

\end{remarque}
\end{frame}

\begin{frame}

\myfigure{0.8}{
\tikzinput{fig_ensembles13c} 
\qquad \pause
\tikzinput{fig_ensembles13d} 
}

\pause

\vspace*{-5mm}
\begin{exemple}
\vspace*{-0mm}
\begin{minipage}{0.65\textwidth}
\vspace*{-20mm}
\center
$f : \Rr \to ]0,+\infty[$, $f(x)=\exp(x)$ est bijective

\pause

sa bijection réciproque est 

$g : ]0,+\infty[ \to \Rr$, $g(y)=\ln(y)$

\pause
\bigskip

$\exp\big(\ln(y) \big) = y$ \quad et \quad   $\ln\big(\exp(x)\big) = x$
  
\end{minipage}
\pause
\begin{minipage}{0.29\textwidth}
\myfigure{0.8}{\hspace*{-5mm}
\tikzinput{fig_ensembles13e} 
}  
\end{minipage}


\end{exemple}




\end{frame}


%---------------------------------------------------------------

\begin{frame}

\begin{proposition}
\label{prop:bij2}
Soient $f : E \to F$ et $g : F \to G$ des applications bijectives

L'application $g \circ f$ est bijective 
\pause
et
\mybox{$(g\circ f)^{-1} = f^{-1} \circ g^{-1}$}
\end{proposition}

\end{frame}
%---------------------------------------------------------------

\begin{frame}

\begin{proposition}
\begin{enumerate}
  \item L'application $f : E \to F$ est bijective si et seulement si il existe $g : F \to E$
telle que $f \circ g = \id_F$ et $g \circ f = \id_E$
\end{enumerate}
\end{proposition}

\medskip
\pause

\begin{proof} 
Preuve du sens $\Rightarrow$ 

\pause
\medskip

Supposons $f$ bijective et construisons $g : F \to E$ 

\pause

$f$ est surjective: pour $y \in F$, il existe un $x \in E$ tel que $y=f(x)$

\pause

On pose $g(y)=x$ et on a $f\big( g(y) \big) = f(x) =y$

\pause

En répétant pour tous les $y\in F$, on obtient $f \circ g = \id_F$

\pause

On en déduit aussi que $f \circ g \circ f = \id_F \circ f$,
\pause
c'est-à-dire $f\big( g\circ f(x) \big) = f(x)$

\pause

Or $f$ est injective donc $g\circ f(x)=x$

\pause

C'est vrai pour tout $x\in E$, donc $g\circ f =\id_E$

\end{proof}
\end{frame}




\begin{frame}
\begin{proposition}
\begin{enumerate}
  \item L'application $f : E \to F$ est bijective si et seulement si il existe $g : F \to E$
telle que $f \circ g = \id_F$ et $g \circ f = \id_E$
\end{enumerate}
\end{proposition}

\medskip


\begin{proof}
Preuve du sens $\Leftarrow$ 

\pause
\medskip

Supposons qu'un tel $g$ existe et montrons $f$ bijective

\pause

\begin{itemize}
  \item surjectivité: soit $y \in F$, notons $x = g(y) \in E$

\pause

$f(x) = f\big( g(y) \big) = f \circ g(y) = \id_F(y)=y$

\pause

On peut trouver un antécédent à tout $y\in F$, donc $f$ est surjective

\pause

  \item injectivité: soient $x,x' \in E$ tels que $f(x)=f(x')$

\pause

Alors $g\circ f(x)=g\circ f(x')$ donc $\id_E(x)=\id_E(x')$


\pause
Par conséquent $x=x'$ et $f$ est bien injective
    \end{itemize}
\end{proof}
\end{frame}



%---------------------------------------------------------------

\section{Mini-exercices}


\begin{frame}
\begin{miniexercice}
\begin{enumerate}
  \item Les fonctions suivantes sont-elles injectives, surjectives, bijectives ? 
\begin{itemize}
  \item $f_1 : \Rr \to [0,+\infty[$, $x \mapsto x^2$.
  \item $f_2 : [0,+\infty[ \to [0,+\infty[$, $x \mapsto x^2$.
  \item $f_3 : \Nn \to \Nn$, $x \mapsto x^2$.
  \item $f_4 : \Zz \to \Zz$, $x \mapsto x-7$.
  \item $f_5 : \Rr \to [0, +\infty[$, $x \mapsto |x|$.
\end{itemize}

  \item Montrer que la fonction $f : \  ]1,+\infty[ \to \ ]0,+\infty[$ définie par $f(x)=\frac{1}{x-1}$ 
  est bijective. Calculer sa bijection réciproque.
\end{enumerate}
\end{miniexercice}
\end{frame}




\end{document}