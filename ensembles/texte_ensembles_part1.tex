
%%%%%%%%%%%%%%%%%% PREAMBULE %%%%%%%%%%%%%%%%%%


\documentclass[12pt]{article}

\usepackage{amsfonts,amsmath,amssymb,amsthm}
\usepackage[utf8]{inputenc}
\usepackage[T1]{fontenc}
\usepackage[francais]{babel}


% packages
\usepackage{amsfonts,amsmath,amssymb,amsthm}
\usepackage[utf8]{inputenc}
\usepackage[T1]{fontenc}
%\usepackage{lmodern}

\usepackage[francais]{babel}
\usepackage{fancybox}
\usepackage{graphicx}

\usepackage{float}

%\usepackage[usenames, x11names]{xcolor}
\usepackage{tikz}
\usepackage{datetime}

\usepackage{mathptmx}
%\usepackage{fouriernc}
%\usepackage{newcent}
\usepackage[mathcal,mathbf]{euler}

%\usepackage{palatino}
%\usepackage{newcent}


% Commande spéciale prompteur

%\usepackage{mathptmx}
%\usepackage[mathcal,mathbf]{euler}
%\usepackage{mathpple,multido}

\usepackage[a4paper]{geometry}
\geometry{top=2cm, bottom=2cm, left=1cm, right=1cm, marginparsep=1cm}

\newcommand{\change}{{\color{red}\rule{\textwidth}{1mm}\\}}

\newcounter{mydiapo}

\newcommand{\diapo}{\newpage
\hfill {\normalsize  Diapo \themydiapo \quad \texttt{[\jobname]}} \\
\stepcounter{mydiapo}}


%%%%%%% COULEURS %%%%%%%%%%

% Pour blanc sur noir :
%\pagecolor[rgb]{0.5,0.5,0.5}
% \pagecolor[rgb]{0,0,0}
% \color[rgb]{1,1,1}



%\DeclareFixedFont{\myfont}{U}{cmss}{bx}{n}{18pt}
\newcommand{\debuttexte}{
%%%%%%%%%%%%% FONTES %%%%%%%%%%%%%
\renewcommand{\baselinestretch}{1.5}
\usefont{U}{cmss}{bx}{n}
\bfseries

% Taille normale : commenter le reste !
%Taille Arnaud
%\fontsize{19}{19}\selectfont

% Taille Barbara
%\fontsize{21}{22}\selectfont

%Taille François
%\fontsize{25}{30}\selectfont

%Taille Pascal
%\fontsize{25}{30}\selectfont

%Taille Laura
%\fontsize{30}{35}\selectfont


%\myfont
%\usefont{U}{cmss}{bx}{n}

%\Huge
%\addtolength{\parskip}{\baselineskip}
}


% \usepackage{hyperref}
% \hypersetup{colorlinks=true, linkcolor=blue, urlcolor=blue,
% pdftitle={Exo7 - Exercices de mathématiques}, pdfauthor={Exo7}}


%section
% \usepackage{sectsty}
% \allsectionsfont{\bf}
%\sectionfont{\color{Tomato3}\upshape\selectfont}
%\subsectionfont{\color{Tomato4}\upshape\selectfont}

%----- Ensembles : entiers, reels, complexes -----
\newcommand{\Nn}{\mathbb{N}} \newcommand{\N}{\mathbb{N}}
\newcommand{\Zz}{\mathbb{Z}} \newcommand{\Z}{\mathbb{Z}}
\newcommand{\Qq}{\mathbb{Q}} \newcommand{\Q}{\mathbb{Q}}
\newcommand{\Rr}{\mathbb{R}} \newcommand{\R}{\mathbb{R}}
\newcommand{\Cc}{\mathbb{C}} 
\newcommand{\Kk}{\mathbb{K}} \newcommand{\K}{\mathbb{K}}

%----- Modifications de symboles -----
\renewcommand{\epsilon}{\varepsilon}
\renewcommand{\Re}{\mathop{\text{Re}}\nolimits}
\renewcommand{\Im}{\mathop{\text{Im}}\nolimits}
%\newcommand{\llbracket}{\left[\kern-0.15em\left[}
%\newcommand{\rrbracket}{\right]\kern-0.15em\right]}

\renewcommand{\ge}{\geqslant}
\renewcommand{\geq}{\geqslant}
\renewcommand{\le}{\leqslant}
\renewcommand{\leq}{\leqslant}

%----- Fonctions usuelles -----
\newcommand{\ch}{\mathop{\mathrm{ch}}\nolimits}
\newcommand{\sh}{\mathop{\mathrm{sh}}\nolimits}
\renewcommand{\tanh}{\mathop{\mathrm{th}}\nolimits}
\newcommand{\cotan}{\mathop{\mathrm{cotan}}\nolimits}
\newcommand{\Arcsin}{\mathop{\mathrm{Arcsin}}\nolimits}
\newcommand{\Arccos}{\mathop{\mathrm{Arccos}}\nolimits}
\newcommand{\Arctan}{\mathop{\mathrm{Arctan}}\nolimits}
\newcommand{\Argsh}{\mathop{\mathrm{Argsh}}\nolimits}
\newcommand{\Argch}{\mathop{\mathrm{Argch}}\nolimits}
\newcommand{\Argth}{\mathop{\mathrm{Argth}}\nolimits}
\newcommand{\pgcd}{\mathop{\mathrm{pgcd}}\nolimits} 

\newcommand{\Card}{\mathop{\text{Card}}\nolimits}
\newcommand{\Ker}{\mathop{\text{Ker}}\nolimits}
\newcommand{\id}{\mathop{\text{id}}\nolimits}
\newcommand{\ii}{\mathrm{i}}
\newcommand{\dd}{\mathrm{d}}
\newcommand{\Vect}{\mathop{\text{Vect}}\nolimits}
\newcommand{\Mat}{\mathop{\mathrm{Mat}}\nolimits}
\newcommand{\rg}{\mathop{\text{rg}}\nolimits}
\newcommand{\tr}{\mathop{\text{tr}}\nolimits}
\newcommand{\ppcm}{\mathop{\text{ppcm}}\nolimits}

%----- Structure des exercices ------

\newtheoremstyle{styleexo}% name
{2ex}% Space above
{3ex}% Space below
{}% Body font
{}% Indent amount 1
{\bfseries} % Theorem head font
{}% Punctuation after theorem head
{\newline}% Space after theorem head 2
{}% Theorem head spec (can be left empty, meaning ‘normal’)

%\theoremstyle{styleexo}
\newtheorem{exo}{Exercice}
\newtheorem{ind}{Indications}
\newtheorem{cor}{Correction}


\newcommand{\exercice}[1]{} \newcommand{\finexercice}{}
%\newcommand{\exercice}[1]{{\tiny\texttt{#1}}\vspace{-2ex}} % pour afficher le numero absolu, l'auteur...
\newcommand{\enonce}{\begin{exo}} \newcommand{\finenonce}{\end{exo}}
\newcommand{\indication}{\begin{ind}} \newcommand{\finindication}{\end{ind}}
\newcommand{\correction}{\begin{cor}} \newcommand{\fincorrection}{\end{cor}}

\newcommand{\noindication}{\stepcounter{ind}}
\newcommand{\nocorrection}{\stepcounter{cor}}

\newcommand{\fiche}[1]{} \newcommand{\finfiche}{}
\newcommand{\titre}[1]{\centerline{\large \bf #1}}
\newcommand{\addcommand}[1]{}
\newcommand{\video}[1]{}

% Marge
\newcommand{\mymargin}[1]{\marginpar{{\small #1}}}



%----- Presentation ------
\setlength{\parindent}{0cm}

%\newcommand{\ExoSept}{\href{http://exo7.emath.fr}{\textbf{\textsf{Exo7}}}}

\definecolor{myred}{rgb}{0.93,0.26,0}
\definecolor{myorange}{rgb}{0.97,0.58,0}
\definecolor{myyellow}{rgb}{1,0.86,0}

\newcommand{\LogoExoSept}[1]{  % input : echelle
{\usefont{U}{cmss}{bx}{n}
\begin{tikzpicture}[scale=0.1*#1,transform shape]
  \fill[color=myorange] (0,0)--(4,0)--(4,-4)--(0,-4)--cycle;
  \fill[color=myred] (0,0)--(0,3)--(-3,3)--(-3,0)--cycle;
  \fill[color=myyellow] (4,0)--(7,4)--(3,7)--(0,3)--cycle;
  \node[scale=5] at (3.5,3.5) {Exo7};
\end{tikzpicture}}
}



\theoremstyle{definition}
%\newtheorem{proposition}{Proposition}
%\newtheorem{exemple}{Exemple}
%\newtheorem{theoreme}{Théorème}
\newtheorem{lemme}{Lemme}
\newtheorem{corollaire}{Corollaire}
%\newtheorem*{remarque*}{Remarque}
%\newtheorem*{miniexercice}{Mini-exercices}
%\newtheorem{definition}{Définition}




%definition d'un terme
\newcommand{\defi}[1]{{\color{myorange}\textbf{\emph{#1}}}}
\newcommand{\evidence}[1]{{\color{blue}\textbf{\emph{#1}}}}



 %----- Commandes divers ------

\newcommand{\codeinline}[1]{\texttt{#1}}

%%%%%%%%%%%%%%%%%%%%%%%%%%%%%%%%%%%%%%%%%%%%%%%%%%%%%%%%%%%%%
%%%%%%%%%%%%%%%%%%%%%%%%%%%%%%%%%%%%%%%%%%%%%%%%%%%%%%%%%%%%%

\begin{document}

\debuttexte

%%%%%%%%%%%%%%%%%%%%%%%%%%%%%%%%%%%%%%%%%%%%%%%%%%%%%%%%%%%
\diapo

\change

\change

Nous commençons ce chapitre  
en discutant comment définir des ensembles.

\change

Nous verrons ensuite le vocabulaire de base concernant 
les ensembles comme l'union et l'intersection.

\change

Puis nous étudierons les règles de calculs s'y afférent.

\change

Nous terminons avec le produit cartésien.


%%%%%%%%%%%%%%%%%%%%%%%%%%%%%%%%%%%%%%%%%%%%%%%%%%%%%%%%%%%
\diapo

Au début du \textsc{\romannumeral 20}\textsuperscript{e} siècle un éminent professeur
peaufinait la rédaction du second tome d'un ouvrage qui souhaitait refonder les mathématiques sur des bases logiques.
Il reçut une lettre d'un tout jeune mathématicien :

\change

\emph{\og J'ai bien lu votre premier livre. Malheureusement vous supposez qu'il existe un ensemble
qui contient tous les ensembles. Un tel ensemble ne peut exister. \fg\ }
S'ensuit une démonstration de deux lignes. Tout le travail du professeur s'écroulait.

\change

Ce paradoxe a été popularisé par l’énigme suivante :
\emph{\og Dans une ville, le barbier rase tous ceux qui ne se rasent pas eux-mêmes.
Qui rase le barbier ? \fg\ }
La seule réponse valable est qu'une telle situation ne peut exister.



Rassurez-vous, Russell et d'autres ont fondé la logique et les ensembles sur des bases
solides. Cependant il n'est pas possible dans ce cours de tout redéfinir.



Heureusement, vous connaissez déjà quelques ensembles :


\change


\begin{itemize}
  \item l'ensemble des entiers naturels $\Nn =\{0,1,2,3,\ldots\}$. 

\change
 
  \item l'ensemble des entiers relatifs $\Zz = \{\ldots, -2,-1,0,1,2,\ldots\}$.

\change

  \item l'ensemble des rationnels $\Qq = \big\{ \frac{p}{q} \text{ avec $p$ et $q$ entiers } \big\}.$

\change

  \item l'ensemble des réels $\Rr$, par exemple $1, \sqrt 2$, $\pi$,\ldots

\change

  \item l'ensemble des nombres complexes $\Cc$.
\end{itemize}



%%%%%%%%%%%%%%%%%%%%%%%%%%%%%%%%%%%%%%%%%%%%%%%%%%%%%%%%%%%
\diapo

Nous allons dire qu'un ensemble est une collection d'éléments.

\change

Par exemple $\{ 0, 1 \}$ 


$\{ \text{rouge}, \text{noir} \}$ 


$\{0, 1, 2, 3,\ldots\} = \Nn$

\change

Un ensemble particulier est l'ensemble vide qui est l'ensemble ne contenant aucun élément

\change

On note $x \in E$ si $x$ est un élément de l'ensemble $E$,

si $x$ n'est pas un élément de $E$ on note $x\notin E$.

\change

Une deuxième façon de définir des ensembles et de demander 
aux éléments de vérifier une certaines propriétés.

\change

Par exemple  

\begin{itemize}
     \item on définit ainsi l'ensemble des $x$ réels qui vérifie $|x-2| < 1$.
     \item de même pour l'ensemble des nombres complexes $z$ tel que $z^5=1$.
     \item Enfin l'ensemble des réels compris entre $0$ et $1$ s'écrit aussi comme l'intervalle $[0,1]$.
 \end{itemize}


%%%%%%%%%%%%%%%%%%%%%%%%%%%%%%%%%%%%%%%%%%%%%%%%%%%%%%%%%%%
\diapo

Reprenons à la base le vocabulaire sur les ensembles.

Tout d'abord un ensemble $E$ est inclus dans un autre ensemble $F$
si tout élément de $E$ est aussi un élément de $F$

\change

On dit aussi que $E$ est un sous-ensemble ou une partie de $F$.

\change

Deux ensembles $E$ et $F$ sont égaux, si l'on a les deux inclusions
$E \subset F$ et $F \subset E$.

En termes d'éléments : tout élément de $E$ est aussi un élément de $F$
et réciproquement tout élément de $F$ est aussi un élément de $E$.

\change

Nous noterons $\mathcal{P}(E)$ l'ensemble des parties $E$.

\change

Par exemple si $E$ est constitué des éléments $1,2,3$

alors il y a $6$ parties inclues dans $E$.

les singletons :  $\{1\}, \{2\}, \{3\}$

les paires $\{1,2\}, \{1,3\}, \{2,3\}$

et on n'oublie l'ensemble vide qui est inclus dans tout ensemble.

Ni l'ensemble $E$ tout entier qui est inclus dans lui-même.




%%%%%%%%%%%%%%%%%%%%%%%%%%%%%%%%%%%%%%%%%%%%%%%%%%%%%%%%%%%
\diapo

Continuons avec le complémentaire. Si $A$ est une partie d'un ensemble $E$,

le complémentaire de $A$ dans $E$ est l'ensemble des éléments $x$ de $E$ qui n'appartiennent
pas à $A$.

\change

Illustrons ceci par un dessin : si $E$ est représenté par le grand disque 
et $A$ par le petit disque alors le complémentaire de $A$ dans $E$ est la partie rose.

\change

On note aussi très souvent ce complémentaire $E$ privé de $A$, ou aussi juste complémentaire
de $A$.

\change

Si l'on a deux parties $A, B$ d'un ensemble $E$ alors 
$A$ union $B$ est l'ensemble des $x$ appartenant à $A$ ou bien à $B$.

\change

L'union est ici toute la partie rose, en particulier un élément de l'union peut
appartenir aux deux parties $A$ et $B$. 

\change

$A$ inter $B$ est l'ensemble des $x$ appartenant à $A$ *et* $B$.

\change

Que l'on illustre de la sorte.

%%%%%%%%%%%%%%%%%%%%%%%%%%%%%%%%%%%%%%%%%%%%%%%%%%%%%%%%%%%
\diapo

Quelques règles de calculs avec les ensembles.

$A \cap (B \cap C) = (A \cap B) \cap C$ 

on peut donc écrire $A\cap B \cap C$ sans parenthèse

\change

Même chose avec l'union

$A \cup (B \cup C) = (A \cup B) \cup C$  


\change

Plus subtile :

On a 
 $A \cap (B \cup C) = (A \cap B) \cup (B \cap C)$

Les parenthèses sont indispensables lorsque l'on mélange l'union et l'intersection.

\change

Et aussi

 $A \cup (B \cap C) = (A \cup B) \cap (A \cup C)$

\change

Enfin  $\complement \left( \complement A \right) = A$

\change

Le passage au complément renverse les inclusions :

$A \subset B$ ssi $\complement B \subset \complement A$



%%%%%%%%%%%%%%%%%%%%%%%%%%%%%%%%%%%%%%%%%%%%%%%%%%%%%%%%%%%
\diapo


Le complément échange aussi union et intersection.

En effet :

$\complement \left( A \cap B \right) = \complement A \cup \complement B$ 

\change

Alors que 
$\complement \left( A \cup B \right) = \complement A \cap \complement B$

\change

Illustrons ceci par des dessins, on représente $A$ et son complément,

\change

on représente $B$ et son complément

\change

le complément de $A \cap B$ est la partie rose

et c'est bien l'union des deux compléments en rose au dessus.

\change


le complément de $A \cup B$ est la partie rose

et c'est bien l'intersection des deux compléments en rose au dessus.


%%%%%%%%%%%%%%%%%%%%%%%%%%%%%%%%%%%%%%%%%%%%%%%%%%%%%%%%%%%
\diapo

Voyons comment s'effectue les preuves.

Nous allons montrer $A \cap (B \cup C) = (A \cap B) \cup (B \cap C)$

\change

$x \in A \cap (B \cup C)$ 

\change

$\iff x \in A \text{ et } x \in (B \cup C)$ 

\change

$\iff  x \in A \text{ et } (x \in B \text{ ou } x \in C)$ 

\change

$\iff (x \in A \text{ et } x\in B) \text{ ou } (x \in A \text{ et } x \in C)$ 

\change

$\iff (x \in A \cap B) \text{ ou } (x \in A \cap C)$

\change

$\iff x \in (A\cap B) \cup (A\cap C)$


Donc tout élément de $A \cap (B \cup C)$ est un élément de $(A\cap B) \cup (A\cap C)$
et comme nous avons raisonner par équivalence tout élément de $(A\cap B) \cup (A\cap C)$
est aussi élément de $A \cap (B \cup C)$.

Les deux ensembles sont donc égaux.


%%%%%%%%%%%%%%%%%%%%%%%%%%%%%%%%%%%%%%%%%%%%%%%%%%%%%%%%%%%
\diapo

Montrons également la propriété : 

$\complement \left( A \cap B \right) = \complement A \cup \complement B$

\change

C'est parti :

$x \in \complement \left( A \cap B \right)$ 

\change

$\iff x \notin \left( A \cap B \right)$ 

\change

$\iff \text{non} \big(x \in A \cap B\big)$ 

\change

$\iff \text{non} \big(x \in A \text{ et } x \in B\big)$ 

\change

$\iff \text{non} (x \in A) \text{ ou } \text{non} (x \in B)$ 

\change

$\iff x \notin A \text{ ou } x\notin B$ 

\change

$\iff x \in \complement A \cup \complement B$


Remarquez pour ces démonstrations nous sommes repassés aux éléments de l'ensemble 
et qu'en fait nous avons reformuler 
les propriétés des opérateurs logiques, <<et>>, <<ou>>, <<non>>.


%%%%%%%%%%%%%%%%%%%%%%%%%%%%%%%%%%%%%%%%%%%%%%%%%%%%%%%%%%%
\diapo


Il nous reste à définir le produit cartésien de deux ensembles $E$ et $F$.

$E \times F$ est l'ensemble des couples $(x,y)$ où $x \in E$ et $y \in F$

\change

Vous connaissez déjà $\Rr^2$ qui est 
$\Rr \times \Rr$ c'est-à-dire $\big\{ (x,y) \mid x,y \in \Rr \big\}$

\change


Autre exemple :

$[0,1] \times \Rr$ c'est $\big\{ (x,y) \mid 0 \le x \le 1, y \in \Rr \big\}$ 

\change

Qui correspond à une bande verticale infinie.

\change

On peut faire plusieurs produit,

par exemple 

$[0,1] \times [0,1] \times [0,1]$ c'est l'ensemble de triplets $(x,y,z)$

avec $x,y,z$ compris entre $0$ et $1$.

\change

Cela correspond à un cube dans $\Rr^3$.



%%%%%%%%%%%%%%%%%%%%%%%%%%%%%%%%%%%%%%%%%%%%%%%%%%%%%%%%%%%
\diapo


Travailler les bases en profondeur avant de poursuivre !



\end{document}