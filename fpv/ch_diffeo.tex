
\documentclass[12pt, class=report,crop=false]{standalone}
\usepackage[screen]{../exo7book}


\begin{document}

%====================================================================
\chapitre{Difféomorphismes}
%====================================================================

% Commandes à virer
\newcommand{\ou}{\mathscr{O}}
\newcommand{\f}{\mathscr{F}}
\newcommand{\mat}{\mathscr{M}}
\newcommand{\co}{\mathscr{C}}
\newcommand{\ja}{\mathrm{J}}
\newcommand{\jac}{|\mathrm{J}|}
\newcommand{\rot}{\mathrm{rot}}

%%%%%%%%%%%%%%%%%%%%%%%%%%%%%%%%%%%%%%%%%%%%%%%%%%%%%
\section{Difféomorphismes}

%----------------------------------------------------
\subsection{Définition}

\noindent Soient $U$ et $V$ des OUVERTS ( non vides) de $\R^p$.
\begin{definition}[DIFFEOMORPHISME]
 \textcolor[rgb]{0.98,0.00,0.00}{ 
  On dit qu'une application $f: U \rightarrow V$ est un diff\'eomorphisme de $U$ sur $V$
si et seulement si
\begin{itemize}
  \item[1.] $f$ est une bijection,
  \item[2.] $f$ est de classe $\co^1$, c'est \`a dire contin\^ument diff\'erentiable sur $U$,
  \item[3.] $f^{-1}$ est de classe $\co^1$ sur $V$.
\end{itemize}}
\end{definition}

%----------------------------------------------------
\subsection{Difféomorphisme et jacobienne}

[[Tout écrire en terme de Jacobienne, aps de différentielle quand à valeur vectorielle]]

\begin{proposition}[DIFFEOMORPHISME ET RECIPROQUE]
 \textcolor[rgb]{0.44,0.00,0.87}{
  Si $f: U \rightarrow V$ est un diff\'eomorphisme alors sa diff\'erentielle est en tout point de
$U$ un isomorphisme (de $\R^p$ dans lui-même) et la diff\'erentielle de sa fonction r\'eciproque $f^{-1}$
est li\'ee \`a celle de $f$ par la formule
\begin{equation*}
  d(f^{-1})_y=(df_{f^{-1}(y)})^{-1}, \;\;\mathrm{\; pour\; tout \;} y \in V.
\end{equation*}}
\end{proposition}

\noindent {\textbf{Preuve.}} Faite en cours.


\begin{proposition}[DIFFEOMORPHISME ET JACOBIENNE]
 \textcolor[rgb]{0.44,0.00,0.87}{
  Si $f: U \rightarrow V$ est un diff\'eomorphisme alors sa diff\'erentielle est en tout point de
$U$ un isomorphisme (de $\R^p$ dans lui-même) et la diff\'erentielle de sa fonction r\'eciproque $f^{-1}$
est li\'ee \`a celle de $f$ par la formule
\begin{equation*}
  J(f^{-1})_y=(J(f)_{f^{-1}(y)})^{-1}, \;\;\mathrm{\; pour\; tout \;} y \in V.
\end{equation*}
où $ J(f^{-1})_y$ et $(J(f)_{f^{-1}(y)})^{-1}$ sont respectivement la jacobienne de $f^{-1}$ en $y$ et la jacobienne de 
$f$ en $f^{-1}$ en $y$.}
\end{proposition}

\noindent {\textbf{Preuve.}} Faite en cours.


\subsection{Hanani}

Les applications de $\Rr^n$ dans $\Rr^n$ qui sont bijectives et de classe $\mathscr{C}^1$ ainsi que leur réciproque, sont utilisées comme changements de variables. On les appelle des difféomorphismes.

\vskip6mm

\begin{definition}Soient $U$ et $V$ deux ouverts de $\Rr^n$ et $\Phi:U\to V$. On dit que $f$ est un ${\mathscr C}^1$-difféomorphisme si
\begin{enumerate}
\item $\Phi$ est une bijection de $U$ sur $V$.
\item $\Phi$ est de classe ${\mathscr C}^1$ sur $U$.
\item $\Phi^{-1}$ est de classe ${\mathscr C}^1$ sur $V$.
\end{enumerate}
\end{definition}

\vskip4mm

\noindent Du théorème de composition découle que les différentielles de $\Phi$ et $\Phi^{-1}$ sont elles aussi réciproques l'une de l'autre. Et donc les matrices jacobiennes, qui sont des matrices carrées $n\times n$, sont inverses l'une de l'autre.

\vskip6mm

\begin{proposition}Soit $\Phi:U\to V$ un ${\mathscr C}^1$-difféomorphisme, $A\in U$ et $B\in V$. Alors
$$\left[\ja _{\Phi}(A)\right]^{-1}=\ja _{\Phi ^{-1}}(\Phi(A))\quad \mbox{et}\quad \ja _{\Phi ^{-1}}(B)=\left[\ja _{\Phi}\left(\Phi^{-1}(B)\right)\right]^{-1}.$$
\end{proposition}

\vskip4mm

\noindent Pour un difféomorphisme, le déterminant de la matrice jacobienne joue un r\^ole particulier.

\vskip6mm

\begin{definition}Soient $U$ et $V$ deux ouverts de $\Rr^n$ et $\Phi:U\to V$ une application de classe ${\mathscr C}^1$. On appelle jacobien de $\Phi$ au point $A\in U$ le déterminant de la matrice jacobienne de $\Phi$ au point $A$ :
$$\jac _{\Phi}(A)=\det \left(\ja _{\Phi}(A)\right).$$
\end{definition}

\vskip4mm

\noindent Il est clair que le jacobien d'un difféomorphisme ne s'annule pas, puisque la matrice jacobienne est inversible. La réciproque est donnée par le théorème d'inversion.

\vskip6mm

\begin{theoreme}[d'inversion]Soient $U$ et $V$ deux ouverts de $\Rr^n$ et $\Phi:U\to V$ une application de classe ${\mathscr C}^1$. Si $\Phi$ est bijective et si le jacobien de $\Phi$ ne s'annule pas sur $U$, alors $\Phi$ est un ${\mathscr C}^1$-difféomorphisme de $U$ sur $V$.
\end{theoreme}

\vskip4mm

\noindent{\bf Exemple. }Les passages en coordonnées polaires, cylindriques ou sphériques, sont très souvent utilisés. Détaillons le premier qui consiste à remplacer les coordonnées cartésiennes $(x,y)$ d'un point du plan, par le module $r$ et l'argument $\theta$ du point dans le plan complexe.
$$\begin{array}{ccccl}\Phi &:&U=\Rr^2\setminus \left(\Rr^+\times\{0\}\right)&\to&V=]0,+\infty [\times ]0,2\pi [\\&&(x,y)&\mapsto &\displaystyle (r,\theta).\end{array}$$

\vskip4mm

\noindent Dans la pratique, on travaille avec la réciproque
$$\begin{array}{ccccl}\Psi &:&V&\to&U\\&&(r,\theta)&\mapsto &\displaystyle (x,y)\end{array} \qquad \mbox{ o\`u }\left\{\begin{array}{l} x=r \cos \theta \\ y=r \sin \theta .\end{array}\right.$$
On doit avoir $r=\sqrt{x^2+y^2}$ et le point $(x/r,y/r)$ est dans le cercle unité privé du point $(1,0)$. Donc il existe un unique $\theta \in ]0,2\pi [$ tel que
$$x=r\cos \theta \quad \mbox{et}\quad y=r\sin \theta .$$
Ainsi $\Psi$ est bijective et il est évident qu'elle est de classe ${\mathscr C}^1$. Sa matrice jacobienne est 
$$\ja _{\Psi }(r,\theta )=\left(\begin{array}{cc}\cos \theta &-r\sin \theta \\\displaystyle \sin \theta &r\cos \theta \end{array}\right)$$
et son jacobien, qui vaut $r$, ne s'annule pas sur $V$. Donc $\Psi $ est un ${\mathscr C}^1$-difféomorphisme de $V$ sur $U$. Pour calculer les dérivées partielles de $r$ et $\theta$, on utilise l'inversion matricielle de la jacobienne. En effet, puisque $\Phi =\Psi ^{-1}$,
$$\ja _{\Phi }(x,y)=\left(\begin{array}{cc}\displaystyle \frac{\partial r}{\partial x}&\displaystyle \frac{\partial r}{\partial y}\\ \displaystyle \frac{\partial \theta}{\partial x}&\displaystyle \frac{\partial \theta}{\partial y}\end{array}\right)=\left[\ja _{\Psi }(r,\theta ))\right]^{-1}=\frac{1}{r}\left(\begin{array}{cc}r\cos \theta &r\sin \theta \\ \\ -\sin \theta &\cos \theta\end{array}\right)$$
Ce qui nous donne
$$\left(\begin{array}{cc}\displaystyle \frac{\partial r}{\partial x}&\displaystyle \frac{\partial r}{\partial y}\\ \\ \displaystyle \frac{\partial \theta}{\partial x}&\displaystyle \frac{\partial \theta}{\partial y}\end{array}\right)=\left(\begin{array}{cc}\cos \theta &\sin \theta \\ \\ \displaystyle -\frac{\sin \theta }{r}&\displaystyle \frac{\cos \theta }{r}\end{array}\right)=\left(\begin{array}{cc}\displaystyle \frac{x}{\sqrt{x^2+y^2}} &\displaystyle \frac{y}{\sqrt{x^2+y^2}} \\ \\ \displaystyle -\frac{y}{x^2+y^2}&\displaystyle \frac{x}{x^2+y^2}\end{array}\right).\leqno{(*)}$$

\vskip4mm

\noindent Considérons maintenant une application $f:(x, y)\mapsto f(x,y)$ de $U$ dans $\Rr$. Pour utiliser passer en coordonnées, on doit remplacer les anciennes coordonnées $(x,y)$ par les nouvelles coordonnées $(r,\theta)$, et donc considérer la fonction $g$ de $V$ dans $\Rr$ qui à $(r,\theta)$ associe :
$$g(r,\theta)=f\left(\Phi ^{-1}(r,\theta)\right)=f\left(x(r,\theta),y(r,\theta)\right).$$
La formule de dérivation des fonctions composées donne
$$\left\{\begin{array}{ccl}\displaystyle \frac{\partial f}{\partial x}(x,y)&=&\displaystyle \frac{\partial g}{\partial r}(r,\theta )\frac{\partial r}{\partial x}(x,y)+\frac{\partial g}{\partial \theta}(r,\theta )\frac{\partial \theta }{\partial x}(x,y)\\ \\ \displaystyle \frac{\partial f}{\partial y}(x,y)&=&\displaystyle \frac{\partial g}{\partial r}(r,\theta )\frac{\partial r}{\partial y}(x,y)+\frac{\partial g}{\partial \theta}(r,\theta )\frac{\partial \theta }{\partial x}(x,y).\end{array}\right.$$
Donc, d'après $(*)$, on aura :
$$\left\{\begin{array}{ccl}\displaystyle \frac{\partial f}{\partial x}(x,y)&=&\displaystyle \cos \theta .\frac{\partial g}{\partial r}(r,\theta )-\frac{\sin \theta }{r}.\frac{\partial g}{\partial \theta}(r,\theta )\\ \\ \displaystyle \frac{\partial f}{\partial y}(x,y)&=&\displaystyle \sin \theta .\frac{\partial g}{\partial r}(r,\theta )+\frac{\cos \theta }{r}.\frac{\partial g}{\partial \theta}(r,\theta ).\end{array}\right. \leqno{(\star \star )}$$



%----------------------------------------------------
\subsection{Équations aux dérivées partielles}
 

%----------------------------------------------------
\subsection{}

\subsection{Champs de vecteurs}

\vskip4mm

\noindent Les équations aux dérivées partielles sont omniprésentes en physique. Elles relient entre elles les dérivées partielles d'ordre $1$ et $2$, et font intervenir des combinaisons de dérivées partielles comme le gradient, la divergence ou le rotationnel.

\vskip6mm

\noindent On rappelle que le gradient d'une fonction de deux variables $f$ est le champ de vecteurs de $\Rr^2$ défini par
$$\nabla f=\left(\frac{\partial f}{\partial x},\frac{\partial f}{\partial y}\right).$$
On dispose donc d'un opérateur, noté formellement, $\displaystyle \nabla :=\left(\frac{\partial }{\partial x},\frac{\partial }{\partial y}\right)$ sur les fonctions. De m\^eme, le gradient d'une fonction de trois variables $f$ est le champ de vecteurs de $\Rr^3$ défini par
$$\nabla f=\left(\frac{\partial f}{\partial x},\frac{\partial f}{\partial y},\frac{\partial f}{\partial z}\right).$$
On dispose à nouveau d'un opérateur, noté formellement, $\displaystyle \nabla :=\left(\frac{\partial }{\partial x},\frac{\partial }{\partial y},\frac{\partial }{\partial z}\right)$.

\vskip6mm

\begin{definition}Soit $U$ un ouvert de $\Rr^2$. Soit $F:(x,y)\mapsto \left(P(x,y),Q(x,y)\right)$ une application de classe $\mathscr{C}^1$ de $U$ dans $\Rr^2$. Une telle application est aussi appelée un champ de vecteurs de $\Rr^2$ défini sur $U$. On définit formellement le rotationnel du champ de vecteurs $F$ comme étant le champ de vecteurs de $\Rr$ défini sur $U$ par
$$\mathrm{rot}(F)(x,y)=\det (\nabla ,F)=\left|\begin{array}{cc}\frac{\partial}{\partial x}&P\\ \\ \frac{\partial}{\partial y}&Q
\end{array}\right|(x,y)=\frac{\partial Q}{\partial x}(x,y)-\frac{\partial P}{\partial y}(x,y).$$
\end{definition}

\vskip4mm

\noindent Un champ de vecteurs sera noté indifféremment $F$ ou $\overrightarrow{F}$. On vérifiera à partir de cette définiton et le théorème de Schwarz que, $\mbox{rot}(\nabla f)=0$.

\vskip6mm

\begin{definition}Soit $U$ un ouvert de $\Rr^3$ et $F:(x,y,z)\mapsto \left(P(x,y,z),Q(x,y,z),R(x,y,z)\right)$ une application de classe $\mathscr{C}^1$ de $U$ dans $\Rr^3$, appelée aussi champ de vecteurs de $\Rr^3$ défini sur $U$.
\begin{enumerate}
\item Le rotationnel de $F$ est le champ de vecteurs de $\Rr^3$ donné par
$$\mathrm{rot}(F)=\nabla \wedge F=\left(\frac{\partial R}{\partial y}-\frac{\partial Q}{\partial z},\frac{\partial P}{\partial z}-\frac{\partial R}{\partial x},\frac{\partial Q}{\partial x}-\frac{\partial P}{\partial y}\right).$$
\item La divergence de $F$ est la fonction $\displaystyle \mathrm{div}(F)=\langle \nabla ,F\rangle =\frac{\partial P}{\partial x}+\frac{\partial Q}{\partial y}+\frac{\partial R}{\partial z}$.
\end{enumerate}
\end{definition}

\vskip4mm

\noindent On vérifiera à partir de ces définitons et le théorème de Schwarz que, $\mbox{rot}(\nabla f)=0$ et que, pour un champ de vecteurs $F$ de $\Rr^3$, $\mathrm{div}(\mathrm{rot}(F))=0$.

\vskip6mm

\begin{definition}Soit $F$ un champ de vecteurs défini sur $U$. On dit que $F$ dérivé d'un potentiel sur $U$ s'il existe une fonction $f:U\to \Rr$ telle que $F= \nabla f$ sur $U$. Dans ce cas, on dira que $f$ est un potentiel de $F$.
\end{definition}

\vskip4mm

\begin{theoreme}[\bf Poincaré]Soit $U$ un ouvert simplement connexe de $\Rr^2$ (resp. $\Rr^3$) et $F$ un champ de vecteurs de $\Rr^2$ (resp. $\Rr^3$) de classe ${\mathscr C}^1$ sur $U$. Alors $F$ dérive d'un potentiel sur $U$ si, et seulement si, $\rot F=0$.
\end{theoreme}

\vskip6mm

\noindent{\bf Méthode. }Lorsqu'un champ de vecteurs $\overrightarrow{F}$ dérive d'un potentiel $f$, on écrit $\displaystyle \nabla f=\overrightarrow{F}$. En identifiant les coordonnées, on obtient un système d'équations dont la seule inconnue est $f$. Il faut donc intégrer ce système pour déterminer $f$.

\vskip6mm

\noindent{\bf Exemple. }{\it Montrer que le champ de vecteurs $\overrightarrow{F}(x,y)=y^2\vec{i}+(2xy-1)\vec{j}$ dérive d'un potentiel sur $\Rr^2$ et déterminer les potentiels dont il dérive.}

\vskip4mm

\noindent \underline{\it Solution}. \rm Ici $P(x,y)=y^2$, $Q(x,y)=2xy-1$ et $\displaystyle \frac{\partial P}{\partial y}=2y=\frac{\partial Q}{\partial x}$. Donc $\rot \overrightarrow{F}=0$ et, comme $\Rr^2$ est simplement connexe, $\overrightarrow{F}$ dérive d'un potentiel $f$ sur $\Rr^2$. On aura :
$$\frac{\partial f}{\partial x}(x,y)=P(x,y)=y^2\rightarrow f(x,y)=xy^2+K(y)$$
et 
$$\frac{\partial f}{\partial y}(x,y)=Q(x,y)=2xy-1\rightarrow K'(y)=-1\rightarrow K(y)=-y+C,\quad C\in \Rr.$$
Les potentiels de $\overrightarrow{F}$ sur $\Rr^2$ sont les fonctions $f$ définies par $f(x,y)=xy^2-y+C$.

\vskip8mm

\subsection{Exemples d'équations aux dérivées partielles}

\vskip4mm

\noindent Soit $U$ un ouvert non vide de $\Rr^2$. On note $(x_0,y_0)$ un point de $U$ et $U_1$ (resp. $U_2$) la projection de $U$ sur l'axe $y=0$ (resp. $x=0$).

\vskip6mm

\begin{proposition}Soit $h$ une fonction de classe $\mathscr{C}^0$ sur $U$. On note $H$ la primitive de $h_1:x\mapsto h(x,y)$ sur $U_1$ qui s'annule en $x_0$. Une fonction $f$ de classe $\mathscr{C}^1$ sur $U$ est une solution de 
$$(E_1)\; :\; \frac{\partial f}{\partial x}(x,y)=h(x,y)$$
si, et seulement si, il existe une fonction $k$ de classe $\mathscr{C}^1$ sur $U_2$ telle que
$$\forall (x,y)\in U,\; \; f(x,y)=H(x,y)+k(y).$$
\end{proposition}

\vskip4mm

\noindent{\it Démonstration. }Si $f$ est une solution de $(E_1)$ la fonction $\varphi :x\mapsto f(x,y)-H(x,y)$ est dérivable et de dérivée nulle. Elle est donc constante :
$$\forall x\in U_1,\; \varphi (x)=\varphi (x_0)\rightarrow f(x,y)=H(x,y)+f(x_0,y)$$
et $k:y\mapsto f(x_0,y)$ est bien une fonction de classe $\mathscr{C}^1$ sur $U_2$. Réciproquement, on vérifie qu'une fonction de cette forme est solution de $(E_1)$.

\vskip6mm

\begin{proposition}Soit $h$ une fonction de classe $\mathscr{C}^0$ sur $U_1$ et $H$ une primitive de $h$ sur $U_1$. Une fonction $f$ de classe $\mathscr{C}^2$ sur $U$ est une solution de 
$$(E_2)\; :\; \frac{\partial ^2f}{\partial x\partial y}(x,y)=h(x)$$
si, et seulement si, il existe une fonction $K$ de classe $\mathscr{C}^2$ sur $U_2$ telle que
$$\forall (x,y)\in U,\; \; f(x,y)=yH(x)+K(y).$$
\end{proposition}

\vskip4mm

\noindent{\it Démonstration. }Si $f$ est une solution de $(E_2)$ la fonction $\displaystyle \frac{\partial f}{\partial y}$ est solution d'une équation du type $(E_1)$. Donc
$$\forall (x,y)\in U,\; \frac{\partial f}{\partial y}(x,y)=H(x)+k(y)$$
o\`u $k$ est une fonction de classe $\mathscr{C}^1$ sur $U_2$. Ainsi $f$ est une solution d'une équation du type $(E_1)$. Donc de la forme ci-dessus.  Réciproquement, on vérifie qu'une fonction de cette forme est solution de $(E_2)$.

\vskip6mm

\begin{proposition}Une fonction $f$ de classe $\mathscr{C}^2$ sur $U$ est une solution de 
$$(E_3)\; :\; \frac{\partial ^2f}{\partial x^2}(x,y)=0$$
si, et seulement si, il existe deux fonctions $K$ et $H$ de classe $\mathscr{C}^2$ sur $U_2$ telles que
$$\forall (x,y)\in U,\; \; f(x,y)=xH(y)+K(y).$$
\end{proposition}

\vskip4mm

\noindent{\it Démonstration. }Si $f$ est une solution de $(E_3)$ la fonction $\displaystyle \frac{\partial f}{\partial x}$ est solution d'une équation du type $(E_1)$. Donc
$$\forall (x,y)\in U,\; \frac{\partial f}{\partial x}(x,y)=k(y)$$
o\`u $k$ est une fonction de classe $\mathscr{C}^1$ sur $U_2$. Ainsi $f$ est une solution d'une équation du type $(E_1)$. Donc de la forme ci-dessus. Réciproquement, on vérifie qu'une fonction de cette forme est solution de $(E_3)$.

\vskip6mm

\noindent{\bf Résolution à l'aide d'un difféomorphisme. }Pour intégrer une EDP, $(E)$ donnée, on utilise un changement de variables pour se ramener à une EDP plus simple. Soit
$$\begin{array}{ccccl}\Phi &:&U&\to&V\\&&(x,y)&\mapsto &\displaystyle (u,v).\end{array}$$
un $\mathscr{C}^1$-difféomorphisme. Pour une fonction $f$ solution de $(E)$, on pose $g=f\circ \Phi ^{-1}$. C'est à dire $f=g\circ \Phi$.
\begin{enumerate}
\item On utilise la formule de dérivation des fonctions composées pour exprimer les dérivées partielles de $f$ en fonction de $g$, $u$ et $u$.
\item On remplace dans l'équation $(E)$ ce qui donne l'EDP $(E')$ satisfaite par $g$.
\item On intègre $(E')$ et on en déduit les solutions $f$ de $(E)$.
\end{enumerate}

\vskip6mm

\noindent{\bf Exemple. }Intégrons dans $U=\{(x,y)\in \Rr^2|\ x>0\}$ l'EDP suivante :
$$(E)\;\; :\; \; x\frac{\partial f}{\partial x}+y\frac{\partial f}{\partial y}=\sqrt{x^2+y^2}.$$
\rm On pose $\displaystyle V=]0,+\infty [\times \left]-\frac{\pi}{2},\frac{\pi}{2}\right[$, et on considère l'application $\Phi : V\to U$ définie par
$$\Phi (r,\theta )=(r\cos \theta,r\sin \theta )$$
\begin{enumerate}
\item L'application $\Phi$ est un ${\mathscr C}^1$-difféomorphisme de $V$ sur $U$, et
$$\forall (x,y)\in U,\; \;\Phi ^{-1}(x,y)=\left( \sqrt{x^2+y^2},\arctan\frac{y}{x}\right).$$
\item Soit $f$ une fonction de classe ${\mathscr C}^1$ solution de $(E)$ sur $U$. On considère la fonction $g$ définie sur $V$ par
$$g(r,\theta )=f(x,y)\mbox{ avec }(x,y)=(r\cos \theta,r\sin \theta ).$$
\begin{enumerate}
\item On exprime les dérivées partielles premières de $f$ en fonction de $g$, $r$ et $\theta$ (cf. les relations $(\star \star )$ ci-dessus).
\item On reporte dans l'équation $(E)$ ce qui donne :
$$r\frac{\partial g}{\partial r}(r,\theta )=r\Leftrightarrow \frac{\partial g}{\partial r}(r,\theta )=1.$$
\item On voit que $g$ est une solution d'une équation du type $(E_1)$, donc $g(r,\theta )=r+k(\theta )$ o\`u $k$ est une fonction de classe ${\mathscr C}^1$ sur $\displaystyle \left]-\frac{\pi}{2},\frac{\pi}{2}\right[$. On en déduit que toute solution $f$ de $(E)$ est de la forme :
$$\displaystyle f(x,y)=\sqrt{x^2+y^2}+k\left(\arctan \frac{y}{x}\right).$$
\end{enumerate}                 
\end{enumerate}


%----------------------------------------------------
\begin{miniexercices}
\sauteligne
\begin{enumerate}
  \item 
\end{enumerate}
\end{miniexercices}



\subsection{Introduction}



%%%%%%%%%%%%%%%%%%%%%%%%%%%%%%%%%%%%%%%%%%%%%%%%%%%%%
\section{Théorème d'inversion locale}

%----------------------------------------------------
\subsection{}


%----------------------------------------------------
\subsection{}


%----------------------------------------------------
\subsection{}


%----------------------------------------------------
\subsection{}
 
 
%----------------------------------------------------
\begin{miniexercices}
\sauteligne
\begin{enumerate}
  \item 
\end{enumerate}
\end{miniexercices}


\subsection{Pujo : Th\'eor\`eme d'inversion locale}
\begin{theoreme}[THEOREME D'INVERSION LOCALE]
\textcolor[rgb]{0.44,0.00,0.87}{
Si
\begin{itemize}
  \item[1.]$f: U \rightarrow V$ est de classe $\co^1$,
  \item[2.]$a \in U$ est tel que $df_a$ soit un isomorphisme (de $\R^p$ dans lui-même),
  \end{itemize}
alors il existe un voisinage ouvert $U_a$ de $a$ dans $U$ et un voisinage ouvert
$V_b$ de $b=f(a)$ dans $V$ tel que la restriction de $f$ \`a $U_a$ soit un diff\'eomorphisme de $U_a$
sur $V_b$.}
\end{theoreme}

\noindent {\textbf{Preuve.}} Pas faite en cours.

\begin{corollaire}[THEOREME D'INVERSION GLOBALE]
 \textcolor[rgb]{0.44,0.00,0.87}{
  Soit $f: U \rightarrow \R^p$ une application de classe $\co^1$ avec $U$ un ouvert
non vide. C'est un diff\'eomorphisme de $U$ sur $f(U)$ si et seulement si
\begin{itemize}
  \item[1.] elle est injective, et
  \item[2.] sa diff\'erentielle est en tout point de $U$ un isomorphisme  (de $\R^p$ dans lui-même).
\end{itemize}}
\end{corollaire}

\noindent {\textbf{Preuve.}} Pas faite en cours.

\begin{corollaire} [FORMULATION AVEC JACOBIENNE]\textcolor[rgb]{0.44,0.00,0.87}{ \textbf{Dimension finie.}
  Soit $U$ un ouvert de $\R^p$ et $f: U \rightarrow \R^p$ injective et de classe $\co^1$.
Alors $f$ est un diff\'eomorphisme si et seulement si le d\'eterminant de sa matrice jacobienne
(que l'on appelle jacobien de $f$) ne s'annule pas sur $U$.}
\end{corollaire}

\noindent {\textbf{Preuve.}} Pas faite en cours.


\subsection{Fichou}

\noindent{Soient $U$ un ouvert de $\mathbb{R}^n$, $F$ une application de $U$ dans $\mathbb{R}^n$ et $V = F(U) \subset \mathbb{R}^n$.

\begin{definition}
$F$ est {\bf inversible} sur $U$ s'il existe une application $G$ de $V$ dans $\mathbb{R}^n$ telle que $G \circ F = {\bf 1}_{U}$ et $F \circ G = {\bf 1}_{V}$\,.
\end{definition}

\begin{definition}
$F$ de $\mathbb{R}^n$ dans $\mathbb{R}^n$ est {\bf localement inversible} en $X_0 \in \mathbb{R}^n$ s'il existe des ouverts $U$ et $V$ avec $X_0 \in U$ et $F(X_0) \in V$ et $F(U)=V$ tel que $F$ est inversible sur $U$. 
\end{definition}


\noindent{{\bf Exemples} 
\begin{enumerate}
\item[(1)] $f : \mathbb{R} \rightarrow \mathbb{R}$ \;\;\; avec $f(x) = x^3$
\item[(2)] $f : \mathbb{R} \rightarrow \mathbb{R}$ \;\;\; avec $f(x) = x^2$
\item[(3)] Si $A \in \mathbb{R}^n$, soit $F$ de $\mathbb{R}^n$ dans $\mathbb{R}^n$ avec $F(X) = X + A$.
\item[(4)] $U = \lbrace (r\,,\,\theta) \; / \; r > 0 \,,\, 0 < \theta < \pi \rbrace$\\
$F(r\,,\,\theta) = (r\, \cos \theta \,,\, r\, \sin \theta)$
\end{enumerate}



\begin{theoreme}(d'inversion locale)\\
Soient $F$ définie sur un domaine $D$ de $\mathbb{R}^n$ \`a valeurs dans $\mathbb{R}^n$ de classe $\mathcal{C}^1$ et $X_0$ un point intérieur \`a $D$. Alors si $dF(X_0)$ est inversible (en tant qu'application linéaire) $F$ est localement inversible en $F_0$. Si $G$ désigne son inverse locale, $G$ est aussi de classe $C^1$ et en $Y=F(X)$, pour $X$ proche de $X_0$, on a $dG(y)=dF(X)^{-1}$ (l'exposant désigne ici l'opération d'inversion d'une matrice).
\end{theoreme}
Une démonstration de ce théor\`eme est donnée en annexe.

%%%%%%%%%%%%%%%%%%%%%%%%%%%%%%%%%%%%%%%%%%%%%%%%%%%%%
\section{Théorème des fonctions implicites}

%----------------------------------------------------
\subsection{}


%----------------------------------------------------
\subsection{}


%----------------------------------------------------
\subsection{}


%----------------------------------------------------
\subsection{}
 
 
%----------------------------------------------------
\begin{miniexercices}
\sauteligne
\begin{enumerate}
  \item 
\end{enumerate}
\end{miniexercices}



\subsection{Pujo : Th\'eor\`eme des fonctions implicites}

\noindent Le th\'eor\`eme des fonctions implicites concerne la r\'esolution d'\'equations non-lin\'eaires de la forme
\begin{equation*}
  f(x,y)=0,
\end{equation*}
et doit son nom au fait que, sous les hypoth\`eses que l'on va pr\'eciser, on peut en tirer $y$ comme
fonction de $x$: on dit alors que $f(x,y)=0$ d\'efinit implicitement $y$, ou encore $y$ comme
fonction implicite de $x$.\\
Donnons d'abord une formulation générale (qui peut être utilisée sans passer par les matrices jacobiennes),
puis un cas particulier de fonctions de $\R^2$ à valeurs dans $\R$ pour finalement énoncé le résulat avec les matrices jacobiennes.




\begin{theoreme}[THEOREME DES FONCTIONS IMPLICITES]\textcolor[rgb]{0.44,0.00,0.87}{
 Soient $E$, $F$ et $G$, trois espaces de dimension finie.  Soit $U$ un ouvert de $E \times F$ et $f: U \rightarrow G$ une fonction de classe
$\co^1$. On suppose qu'il existe $(a,b)  \in U$ tel que $f(a,b)=0_G$ et la diff\'erentielle partielle de $f$ par rapport \`a $y$,
$d_2f$ est telle que $d_2f_{(a,b)}$ soit un isomorphisme de $F$ sur $G$. Alors
il existe un voisinage ouvert $U_{(a,b)}$ de $(a,b)$ dans $U$, un voisinage ouvert $W_a$ de $a$ dans $E$ et une fonction de classe $\co^1(W_a,F)$
\begin{equation*}
  \varphi: W_a \rightarrow F
\end{equation*}
telle que
\begin{equation*}
  ((x,y) \in U_{(a,b)} \;\mathrm{et}\; f(x,y)=0_G)\Leftrightarrow y=\varphi(x).
\end{equation*} }
\end{theoreme}

\noindent {\textbf{Preuve.}} Faite en cours.

\begin{proposition}[DIFFERENTIELLES FONCTION IMPLICITE] \textcolor[rgb]{0.44,0.00,0.87}{
Sous les hypoth\`eses du th\'eor\`eme des fonctions implicites, et quitte \`a r\'eduire $W_a$ on a
\begin{equation*}
  d\varphi_x(h)=-(d_2f_{(x,\varphi(x))})^{-1}d_1f_{(x,\varphi(x))}(h)
\end{equation*}
pour tout $x \in W_a$ et pour tout $h \in E$.}
\end{proposition}

\noindent {\textbf{Preuve.}} Faite en cours.


%\noindent Voici un r\'esultat qui permet de simplifier la v\'erification des hypoth\`eses des th\'eor\`emes
%d'inversion locale ou des fonctions implicites.
%
%\begin{theoreme} \textcolor[rgb]{0.44,0.00,0.87}{
%  Si $E$ et $F$ sont des espaces de Banach, si $u$ est un application lin\'eaire continue et bijective de $E$ sur $F$,
%alors sa r\'eciproque est continue.}
%\end{theoreme}
%\noindent N.B.:
%\begin{itemize}
%  \item[1.] on rappelle qu'en dimension finie ce r\'esultat n'a pas d'int\'er\^et puisque toutes les applications lin\'eaires sont continues.
%  \item[2.] Ainsi, pour v\'erifier que $u$ est un isomorphisme de $E$ sur $F$, il suffit de v\'erifier que $u$ est lin\'eaire, continue et bijective.
%\end{itemize}

\begin{proposition}[FONCTIONS DE $E \subset \R^2 \mapsto \R$] 
\textcolor[rgb]{0.44,0.00,0.87}{ 
Soient $U \subset \R^2$, $U$ ouvert et $f:U \rightarrow \R$ une application
de classe $\co^1$ sur $U$. On suppose qu'il existe $(a,b) \in U$ tel que 
$f(a,b)=0$ et que $\dfrac{\partial f}{\partial y}(a,b) \neq 0$.
Alors il existe un voisinage $U_{(a,b)}$ de $(a,b)$ dans $U$, un voisinage
ouvert $W_a$ de $a$ dans $U$ et une fonction de classe $\co^1 (W_a, \R)$
\begin{equation*}
\varphi: W_a \rightarrow \R,
\end{equation*}
telle que
\begin{equation*}
((x,y) \in U_{(a,b)} \;\mathrm{et\;} f(x,y)=0)\Leftrightarrow y=\varphi(x),
\end{equation*}
et quitte à réduire $W_a$ on a 
\begin{equation*}
\dfrac{\partial f}{\partial y}(x,\varphi(x)) \neq 0, \; \mathrm{et \;}\varphi'(x)=-\dfrac{\dfrac{\partial f}{\partial x}(x,\varphi(x))}{\dfrac{\partial f}{\partial y}(x,\varphi(x))}
\end{equation*}
 }
\end{proposition}

\noindent {\textbf{Preuve.}} Pas faite en cours.

\begin{proposition} [FONCTIONS DE $U \subset \R^{p+q} \mapsto \R^q$] 
\textcolor[rgb]{0.44,0.00,0.87}{ 
Soient $U \subset \R^p \times R^q$, $E$ ouvert et $f:U \rightarrow \R^q$ une application
de classe $\co^1$ sur $U$. On note $f_i$, $i=1,...,q$ les composantes de $f$ chacune définie de $U$ à valeurs dans $\R$. On suppose qu'il existe $(a,b) \in U$ tel que  
$f(a,b)=0$ et que la matrice définie par les coefficients
$\lbrace(\dfrac{\partial f_i}{\partial x_{p+j}})(a,b)\rbrace_{1 \leq i,j \leq q} $ est inversible (autrement dit le déterminant de cette matrice
est non nul).
Alors il existe un voisinage $U_{(a,b)}$ de $(a,b)$ dans $U$, un voisinage
ouvert $W_a$ de $a$ dans $\R^p$ et une fonction de classe $\co^1 (W_a, \R^q)$
\begin{equation*}
\varphi: W_a \rightarrow \R^q,
\end{equation*}
telle que
\begin{equation*}
((x,y) \in U_{(a,b)} \;\mathrm{et\;} f(x,y)=0)\Leftrightarrow y=\varphi(x),
\end{equation*}
et quitte à réduire $W_a$ on a la jacobienne de $\varphi$ en $(x_1,...,x_p)$\\
$J_{\varphi}(x_1,...,x_p)= $
\begin{equation*}
-\left(
\begin{array}{lll}
\dfrac{\partial f_1}{\partial x_{p+1}}(x,\varphi(x))&...&\dfrac{\partial f_1}{\partial x_{p+q}}(x,\varphi(x))\\
\vdots & &\vdots\\
\dfrac{\partial f_q}{\partial x_{p+1}}(x,\varphi(x))&...&\dfrac{\partial f_q}{\partial x_{p+q}}(x,\varphi(x))
 \end{array}
\right)^{-1}
\left(
\begin{array}{lll}
\dfrac{\partial f_1}{\partial x_{1}}(x,\varphi(x))&...&\dfrac{\partial f_1}{\partial x_{p}}(x,\varphi(x))\\
\vdots & &\vdots\\
\dfrac{\partial f_q}{\partial x_{1}}(x,\varphi(x))&...&\dfrac{\partial f_q}{\partial x_{p}}(x,\varphi(x))
 \end{array}
\right).
\end{equation*}
 }
\end{proposition}

\noindent {\textbf{Preuve.}} Pas faite en cours.

\subsection{Fichou : Fonctions implicites : cas $f(x\,,\,y) = 0$}
 

\noindent{Soit $f : \mathbb{R}^2 \rightarrow \mathbb{R}$. On consid\`ere la courbe de niveau $\lbrace f(x\,,\,y) = 0 \rbrace = N_{0}$\,.

\begin{definition}
On dit que la fonction $y = \varphi(x)$ est {\bf définie implicitement par} $f(x\,,\,y) = 0$ si $f(x\,,\, \varphi(x)) = 0$, c'est-\`a-dire si $(x\,,\, \varphi(x)) \in N_{0}$\,.\\
Alors on dit que $y = \varphi(x)$ est une {\bf fonction implicite} de $f(x\,,\,y) = 0$.
\end{definition}

\noindent{{\bf Exemple}\\
$f(x\,,\,y) = \ln(xy) - \sin x$ \,\, avec $xy > 0$ \\
$f(x\,,\,y) = x^2 + y^2 - 1$. Faire un dessin!



\begin{theoreme} (des fonctions implicites)\\
Soient $f : \mathbb{R}^2 \rightarrow \mathbb{R}$ une fonction de classe $\mathcal{C}^1$ et $(x_{0}\,,\,y_{0})$ un point tel que $f(x_{0}\,,\,y_{0}) = 0$.\\
Si $\displaystyle \frac{\partial f}{\partial y}(x_{0}\,,\,y_{0}) \neq 0$ alors :
\begin{enumerate}
\item[(i)] Il existe une fonction implicite $y = \varphi(x)$ de classe $\mathcal{C}^1$, définie sur l'intervalle ouvert $B(x_{0}\,,\,\varepsilon)$, tel que pour tout $x\in B(x_0,\epsilon)$ on ait $y_{0} = \varphi(x_{0})$ et $f(x\,,\,\varphi(x)) = 0$.
\item[(ii)] De plus, la dérivée de $\varphi$ est donnée par $\displaystyle \varphi'(x) = \frac{-\frac{\partial f}{\partial x}(x\,,\, \varphi(x))}{\frac{\partial f}{\partial y}(x\,,\, \varphi(x))}$ en tout point de $B(x_0,\epsilon)$ o\`u $\displaystyle \frac{\partial f}{\partial y}(x\,,\,\varphi(x)) \neq 0$.
\end{enumerate}
\end{theoreme}
\begin{proof} ~~
C'est une conséquence du théor\`eme d'inversion locale. Soit $f$ une fonction $C^1$ de deux variables et $(x_0,y_0)$ tel que $f(x_0,y_0)=0$ et ${{\partial f}\over{\partial y}}(x_0,y_0)\neq 0$. Considérons la fonction $F$ définie par
$$
F(x,y)=(x,f(x,y)).
$$
La matrice jacobienne de $F$ est
$$
\begin{pmatrix}  1&0\\ {{\partial f}\over{\partial x}} & {{\partial f}\over{\partial y}}  \end{pmatrix}.
$$
Par hypoth\`ese ${{\partial f}\over{\partial y}}$ ne s'annule pas en $(x_0,y_0)$. La matrice $dF(x_0,y_0)$ est donc inversible et d'apr\`es le théor\`eme d'inversion locale, $F$ est localement inversible en $(x_0,y_0)$ : il existe $r>0$ tel que $F$ soit une bijection de la boule $B=B((x_0,y_0),r)$ sur son image et l'application inverse, appelons la $G$ est $C^1$ sur l'ouvert $F(B)$. \'Ecrivons $G(s,t)=(g_1(s,t),g_2(s,t))$ les coordonnées de $G$. Comme $G$ est l'inverse de $F$ on a, pour tout $(s,t)$ dans $F(B)$ (en utilisant la définition de $F$) :
$$
(s,t)=F(g_1(s,t),g_2(s,t))=(g_1(s,t),f(g_1(s,t),g_2(s,t))).
$$
On a donc les égalités : $g_1(s,t)=s$ et $f(s,g_2(s,t))=t$.
Les points $(x,y)$ de $B$ pour lesquels $f(x,y)=0$ sont les points dont l'image par $F$ est de la forme $(x,0)$. Ce sont donc les points $G(x,0)$ pour $(x,0)$ dans $F(B)$, soit encore, d'apr\`es la forme de l'application $G$, les points $(x,g_2(x,0))$ pour $(x,0)$ dans $F(B)$. Or $F(B)$ est un ouvert contenant $(x_0,0)$. Il existe donc $\alpha>0$ tel que, pour $x\in]x_0-\alpha,x_0+\alpha[$, $(x,y)\in B$, l'équation $f(x,y)=0$ équivaut \`a $y=g_2(x,0)$. Il suffit d'écrire $\phi(x)=g_2(x,0)$ pour voir qu'on a bien établi le résultat souhaité.
\end{proof}

\noindent{{\bf Exemple}\\
Le cas du cercle.\\
 \'Etude au point $(lambda,0)$ de $f(x\,,\,y) = x(x^2+y^2)-\lambda(x^2-y^2)$. 


\noindent{{\bf Remarque} : On retrouve ainsi une équation de la tangente aux courbes de niveau.

\subsection{Fonctions implicites : cas $f(x_{1} \, \dots \,   x_{n}) = 0$}

L'étude est similaire pour les hypersurfaces de niveau en plusieurs variables, o\`u on va pouvoir exprimer une variable en fonction des autres si la dérivée partielle correspondante n'est pas nulle. 

\begin{theoreme}
Si $f\colon\thinspace \mathbb{R}^n\to\mathbb{R}$ est de classe $\mathcal{C}^1$ et si $\displaystyle \frac{\partial f}{\partial x_{n}} \, (X_{0}) \neq 0$ alors : 
\begin{enumerate}
\item[(i)]  La fonction implicite $x_{n} = \varphi (x_{1}  \, \dots \, x_{n-1})$ existe sur une boule ouverte $B((x_{1,0}  \, \dots \,  x_{n-1 , 0})\,,\, \varepsilon)$ et on a : $f(x_{1} \, \dots \, x_{n-1} \,,\, \varphi(x_{1} \, \dots \, x_{n-1})) = 0$.
\item[(ii)] $\displaystyle \frac{\partial \varphi}{\partial x_{i}} \, = \, \frac{-\; \frac{\partial f}{\partial x_{i}} (x_{1} \, \dots \, x_{n-1} \,,\, \varphi(x_{1} \, \dots \, x_{n-1}))}{\frac{\partial f}{\partial x_{n}} (x_{1} \, \dots \, x_{n-1} \,,\, \varphi(x_{1} \, \dots \, x_{n-1}))}$
\end{enumerate}
\end{theoreme}

\auteurs{
\\
D'après un cours de ...

Revu et augmenté par Arnaud Bodin.

Relu par Stéphanie Bodin et Vianney Combet.
}


\finchapitre 
\end{document}


