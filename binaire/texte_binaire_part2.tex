%%%%%%%%%%%%%%%%%% PREAMBULE %%%%%%%%%%%%%%%%%%


\documentclass[12pt]{article}

\usepackage{amsfonts,amsmath,amssymb,amsthm}
\usepackage[utf8]{inputenc}
\usepackage[T1]{fontenc}
\usepackage[francais]{babel}


% packages
\usepackage{amsfonts,amsmath,amssymb,amsthm}
\usepackage[utf8]{inputenc}
\usepackage[T1]{fontenc}
%\usepackage{lmodern}

\usepackage[francais]{babel}
\usepackage{fancybox}
\usepackage{graphicx}

\usepackage{float}

%\usepackage[usenames, x11names]{xcolor}
\usepackage{tikz}
\usepackage{datetime}

\usepackage{mathptmx}
%\usepackage{fouriernc}
%\usepackage{newcent}
\usepackage[mathcal,mathbf]{euler}

%\usepackage{palatino}
%\usepackage{newcent}


% Commande spéciale prompteur

%\usepackage{mathptmx}
%\usepackage[mathcal,mathbf]{euler}
%\usepackage{mathpple,multido}

\usepackage[a4paper]{geometry}
\geometry{top=2cm, bottom=2cm, left=1cm, right=1cm, marginparsep=1cm}

\newcommand{\change}{{\color{red}\rule{\textwidth}{1mm}\\}}

\newcounter{mydiapo}

\newcommand{\diapo}{\newpage
\hfill {\normalsize  Diapo \themydiapo \quad \texttt{[\jobname]}} \\
\stepcounter{mydiapo}}


%%%%%%% COULEURS %%%%%%%%%%

% Pour blanc sur noir :
%\pagecolor[rgb]{0.5,0.5,0.5}
% \pagecolor[rgb]{0,0,0}
% \color[rgb]{1,1,1}



%\DeclareFixedFont{\myfont}{U}{cmss}{bx}{n}{18pt}
\newcommand{\debuttexte}{
%%%%%%%%%%%%% FONTES %%%%%%%%%%%%%
\renewcommand{\baselinestretch}{1.5}
\usefont{U}{cmss}{bx}{n}
\bfseries

% Taille normale : commenter le reste !
%Taille Arnaud
%\fontsize{19}{19}\selectfont

% Taille Barbara
%\fontsize{21}{22}\selectfont

%Taille François
%\fontsize{25}{30}\selectfont

%Taille Pascal
%\fontsize{25}{30}\selectfont

%Taille Laura
%\fontsize{30}{35}\selectfont


%\myfont
%\usefont{U}{cmss}{bx}{n}

%\Huge
%\addtolength{\parskip}{\baselineskip}
}


% \usepackage{hyperref}
% \hypersetup{colorlinks=true, linkcolor=blue, urlcolor=blue,
% pdftitle={Exo7 - Exercices de mathématiques}, pdfauthor={Exo7}}


%section
% \usepackage{sectsty}
% \allsectionsfont{\bf}
%\sectionfont{\color{Tomato3}\upshape\selectfont}
%\subsectionfont{\color{Tomato4}\upshape\selectfont}

%----- Ensembles : entiers, reels, complexes -----
\newcommand{\Nn}{\mathbb{N}} \newcommand{\N}{\mathbb{N}}
\newcommand{\Zz}{\mathbb{Z}} \newcommand{\Z}{\mathbb{Z}}
\newcommand{\Qq}{\mathbb{Q}} \newcommand{\Q}{\mathbb{Q}}
\newcommand{\Rr}{\mathbb{R}} \newcommand{\R}{\mathbb{R}}
\newcommand{\Cc}{\mathbb{C}} 
\newcommand{\Kk}{\mathbb{K}} \newcommand{\K}{\mathbb{K}}

%----- Modifications de symboles -----
\renewcommand{\epsilon}{\varepsilon}
\renewcommand{\Re}{\mathop{\text{Re}}\nolimits}
\renewcommand{\Im}{\mathop{\text{Im}}\nolimits}
%\newcommand{\llbracket}{\left[\kern-0.15em\left[}
%\newcommand{\rrbracket}{\right]\kern-0.15em\right]}

\renewcommand{\ge}{\geqslant}
\renewcommand{\geq}{\geqslant}
\renewcommand{\le}{\leqslant}
\renewcommand{\leq}{\leqslant}

%----- Fonctions usuelles -----
\newcommand{\ch}{\mathop{\mathrm{ch}}\nolimits}
\newcommand{\sh}{\mathop{\mathrm{sh}}\nolimits}
\renewcommand{\tanh}{\mathop{\mathrm{th}}\nolimits}
\newcommand{\cotan}{\mathop{\mathrm{cotan}}\nolimits}
\newcommand{\Arcsin}{\mathop{\mathrm{Arcsin}}\nolimits}
\newcommand{\Arccos}{\mathop{\mathrm{Arccos}}\nolimits}
\newcommand{\Arctan}{\mathop{\mathrm{Arctan}}\nolimits}
\newcommand{\Argsh}{\mathop{\mathrm{Argsh}}\nolimits}
\newcommand{\Argch}{\mathop{\mathrm{Argch}}\nolimits}
\newcommand{\Argth}{\mathop{\mathrm{Argth}}\nolimits}
\newcommand{\pgcd}{\mathop{\mathrm{pgcd}}\nolimits} 

\newcommand{\Card}{\mathop{\text{Card}}\nolimits}
\newcommand{\Ker}{\mathop{\text{Ker}}\nolimits}
\newcommand{\id}{\mathop{\text{id}}\nolimits}
\newcommand{\ii}{\mathrm{i}}
\newcommand{\dd}{\mathrm{d}}
\newcommand{\Vect}{\mathop{\text{Vect}}\nolimits}
\newcommand{\Mat}{\mathop{\mathrm{Mat}}\nolimits}
\newcommand{\rg}{\mathop{\text{rg}}\nolimits}
\newcommand{\tr}{\mathop{\text{tr}}\nolimits}
\newcommand{\ppcm}{\mathop{\text{ppcm}}\nolimits}

%----- Structure des exercices ------

\newtheoremstyle{styleexo}% name
{2ex}% Space above
{3ex}% Space below
{}% Body font
{}% Indent amount 1
{\bfseries} % Theorem head font
{}% Punctuation after theorem head
{\newline}% Space after theorem head 2
{}% Theorem head spec (can be left empty, meaning ‘normal’)

%\theoremstyle{styleexo}
\newtheorem{exo}{Exercice}
\newtheorem{ind}{Indications}
\newtheorem{cor}{Correction}


\newcommand{\exercice}[1]{} \newcommand{\finexercice}{}
%\newcommand{\exercice}[1]{{\tiny\texttt{#1}}\vspace{-2ex}} % pour afficher le numero absolu, l'auteur...
\newcommand{\enonce}{\begin{exo}} \newcommand{\finenonce}{\end{exo}}
\newcommand{\indication}{\begin{ind}} \newcommand{\finindication}{\end{ind}}
\newcommand{\correction}{\begin{cor}} \newcommand{\fincorrection}{\end{cor}}

\newcommand{\noindication}{\stepcounter{ind}}
\newcommand{\nocorrection}{\stepcounter{cor}}

\newcommand{\fiche}[1]{} \newcommand{\finfiche}{}
\newcommand{\titre}[1]{\centerline{\large \bf #1}}
\newcommand{\addcommand}[1]{}
\newcommand{\video}[1]{}

% Marge
\newcommand{\mymargin}[1]{\marginpar{{\small #1}}}



%----- Presentation ------
\setlength{\parindent}{0cm}

%\newcommand{\ExoSept}{\href{http://exo7.emath.fr}{\textbf{\textsf{Exo7}}}}

\definecolor{myred}{rgb}{0.93,0.26,0}
\definecolor{myorange}{rgb}{0.97,0.58,0}
\definecolor{myyellow}{rgb}{1,0.86,0}

\newcommand{\LogoExoSept}[1]{  % input : echelle
{\usefont{U}{cmss}{bx}{n}
\begin{tikzpicture}[scale=0.1*#1,transform shape]
  \fill[color=myorange] (0,0)--(4,0)--(4,-4)--(0,-4)--cycle;
  \fill[color=myred] (0,0)--(0,3)--(-3,3)--(-3,0)--cycle;
  \fill[color=myyellow] (4,0)--(7,4)--(3,7)--(0,3)--cycle;
  \node[scale=5] at (3.5,3.5) {Exo7};
\end{tikzpicture}}
}



\theoremstyle{definition}
%\newtheorem{proposition}{Proposition}
%\newtheorem{exemple}{Exemple}
%\newtheorem{theoreme}{Théorème}
\newtheorem{lemme}{Lemme}
\newtheorem{corollaire}{Corollaire}
%\newtheorem*{remarque*}{Remarque}
%\newtheorem*{miniexercice}{Mini-exercices}
%\newtheorem{definition}{Définition}




%definition d'un terme
\newcommand{\defi}[1]{{\color{myorange}\textbf{\emph{#1}}}}
\newcommand{\evidence}[1]{{\color{blue}\textbf{\emph{#1}}}}



 %----- Commandes divers ------

\newcommand{\codeinline}[1]{\texttt{#1}}

%%%%%%%%%%%%%%%%%%%%%%%%%%%%%%%%%%%%%%%%%%%%%%%%%%%%%%%%%%%%%
%%%%%%%%%%%%%%%%%%%%%%%%%%%%%%%%%%%%%%%%%%%%%%%%%%%%%%%%%%%%%

\begin{document}

\debuttexte

%%%%%%%%%%%%%%%%%%%%%%%%%%%%%%%%%%%%%%%%%%%%%%%%%%%%%%%%%%%
\diapo %titre

Dans cette deuxième partie, nous allons voir comment passer de la représentation décimale d'un entier à sa représentation binaire et vice-versa.

\change
Nous aborderons ce sujet en trois points :

\change

- tout d'abord découvrir certaines fonctionnalités du langage Python relatives au binaire ;

\change

- ensuite nous verrons quel algorithme utiliser pour déterminer l'écriture bianire d'un entier ;

\change

enfin, nous verrons deux méthodes pour effectuer l'opération inverse, l'une s'avérant bien meilleure que l'autre.

%%%%%%%%%%%%%%%%%%%%%%%%%%%%%%%%%%%%%%%%%%%%%%%%%%%%%%%%%%%
\diapo % binaire en Python 

Beaucoup de langages de programmation permettent de travailler avec l'écriture binaire des nombres, et Python est l'un de ceux-là.

\change

La fonction prédéfinie \codeinline{bin} donne l'écriture binaire du nombre entier qu'on lui passe

\change

Par exemple, pour obtenir l'écriture binaire de 47, on écrit l'expression \codeinline{bin (47)}

\change
et l'interpréteur du langage Python nous donne la valeur de cette expression, c'est-à-dire ici l'écriture binaire de 47

\change
Un autre exemple  nous donne l'écriture binaire de l'entier 3010.

\change

La présence d'apostrophes autour des réponses fournies par l'interpréteur signale que les valeurs renvoyées par la fonction \codeinline{bin} sont des chaînes de caractères et non des nombres. Il s'agit bien d'une représentation écrite de nombres. À noter que ces chaînes de caractères sont préfixées par \codeinline{'0b'}. 

\change
Bien entendu la  fonction \codeinline{bin} s'applique aussi aux nombres négatifs
. Voici l'écriture binaire du nombre -47 qui est ici celle du nombre 47 précédée du signe \codeinline{-}. C'est une représentation usuelle qu'on nomme représentation signe + valeur absolue.

Il faut savoir qu'il existe d'autres représentations binaires des entiers négatifs que nous n'aborderons pas ici.

\change

En règle générale, lorsque dans un programme informatique nous avons à écrire un nombre, les langages nous permettent de l'écrire naturellement en décimal.

Mais si nous en avons besoin pour une raison ou une autre, il est tout à fait possible d'écrire un nombre directement en binaire. En Python, il suffit pour cela de faire précéder l'écriture binaire du nombre par \codeinline{0b}.

Nous voyons ici que le nombre écrit \codeinline{0b101} est reconnu comme étant le nombre 5.

À noter qu'autour du nombre écrit en binaire, il ne faut pas mettre d'apostrophes.

\change

Enfin, dans une expression comprenant plusieurs nombres il est tout à fait possible de mêler représentations décimales et binaires.

Cet exemple montre la somme de deux nombres dont le premier est écrit en binaire et le second en décimal.

%%%%%%%%%%%%%%%%%%%%%%%%%%%%%%%%%%%%%%%%%%%%%%%%%%%%%%%%%%%
\diapo % Calcul de l'écriture binaire

Intéressons nous maintenant à une méthode algorithmique simple pour déterminer l'écriture binaire d'un nombre

\change
Prenons l'exemple de notre nombre 47 :

\change
Effectuons une division euclidienne de notre nombre par 2.

Nous obtenons un quotient $q_1=23$ et un reste $r_0=1$.

\change
Divisons maintenant notre quotient $q_1$ par 2.

Nous obtenons un nouveau quotient $q_2=11$ et un nouveau reste $r_1=1$.

\change
Recommençons avec $q_2$ pour obtenir un troisième quotient $q_3=5$ et un reste $r_2=1$.

\change
Encore une nouvelle division qui donne un quotient $q_4=2$ et un reste $r_3=1$.

\change
Une divsion encore pour obtenir un quotient $q_5=1$ et un reste $r_4=0$.

\change
Et enfin un quotient $q_6=0$ et un reste $r_5=1$.

Cette division est la dernière puisque le quotient obtenu est nul.
  
\change
En regroupant les restes successifs obtenus de droite à gauche, on reconnaît l'écriture binaire de 2.


En résumé, on peut construire  l'écriture binaire d'un nombre entier
\change


- par des divisions euclidiennes successives par 2 ;

\change
- chacune de ces divisions donnant un reste qui bien entendu ne peut prendre que l'une ou l'autre des deux valeurs 0 ou 1 ;

\change
- on arrête les divisions dès qu'on obtient un reste nul.

\change

L'écriture binaire du nombre s'obtient en alignant de droite à gauche les différenrs restes obtenus au cours des divisions.

Pas besoin de connaître les puissances de 2 pour établir cette écriture.

%%%%%%%%%%%%%%%%%%%%%%%%%%%%%%%%%%%%%%%%%%%%%%%%%%%%%%%%%%%
\diapo % Calcul de l'écriture binaire algo formalisé


Voici donc une version formalisée en pseudo-code de l'algorithme que nous venons de voir sur un exemple.

\hrule\medskip

Cet algorithme prend en entrée un entier naturel $n$ non nul,

et produit en sortie l'écriture binaire de cet entier.

\hrule\medskip

Nous l'avons vu l'algorithme consiste à faire une succession de divisions euclidiennes par 2. Ces divisions successives produisent un certain nombre de quotients et de restes. Nous utilisons un indice $i$ pour ces quotients et restes, indice initialisé à 0 comme le montre la ligne 1.

\hrule\medskip

À la ligne 2, nous attribuons la valeur $n$, l'entier dont on veut la représentation binaire à un premier quotient fictif $q_0$.

\hrule\medskip

Les lignes 3 à 7 forment une boucle tant que exprimant la répétition des divisions euclidiennes. Cette répétition est conditionnée par le fait que le dernier quotient $q_i$ n'est pas nul.

Lorsque c'est le cas, on calcul un nouveau quotient $q_{i+1}$ (ligne 4), et un nouveau reste $r_i$ (ligne 5).
Et on incrémente l'indice $i$ (ligne 6) avant d'entamer une éventuelle nouvelle division.

\hrule\medskip

En sortie de boucle, il ne reste plus qu'à construire la représentation binaire de $n$ en regroupant les bits $r_i$ obtenus de droite à gauche.

%%%%%%%%%%%%%%%%%%%%%%%%%%%%%%%%%%%%%%%%%%%%%%%%%%%%%%%%%%%
\diapo % Calcul d'un entier représenté en binaire 1

Envisageons maintenant l'opération inverse, à savoir le calcul du nombre entier dont on connaît une représentation binaire.

Nous allons envisager deux méthodes pour le faire.

\change

La première de ces méthodes est celle que nous avons déjà implictement utilisée : on calcule la somme des puissances de 2 correspondant au bits à 1 dans l'écriture binaire.

\change
Voyons à nouveau cela avec $n=\overline{101111}_2$

\change
Par définition de l'écriture binaire, cela signifie que 
$$ n = 1\times 2^5 + 0\times 2^4 + 1\times 2^3 + 1\times 2^2 + 1\times 2^1 + 1\times 2^0,$$

\change
donc égal à
$$ 32 + 8 + 4 + 2 + 1,$$

\change
et donc égal à $47$.




%%%%%%%%%%%%%%%%%%%%%%%%%%%%%%%%%%%%%%%%%%%%%%%%%%%%%%%%%%%
\diapo % Calcul d'un entier représenté en binaire 2 Algo de Horner

La seconde méthode, bien que moins immédiate, est nettement plus performante.

\change
C'est un cas particulier de l'algorithme de Horner.

\change
Prenons le même exemple

\change 
et réécrivons l'interprétation de cette représentation en termes de somme coefficientée de puissances de 2,

\change
mais au lieu de calculer les puissances de 2, mettons 2 en facteur de tous les termes correpondant à des puissances de 2 d'exposant au moins égal à 1.

On obtient une nouvelle expression dans laquelle la partie parenthésée est une somme coefficientée de puissances de 2, mais avec un exposant limité à 4 cette fois.

\change
On effectue la même opération de factorisation pour l'expression entre parenthèse, et on obtient une nouvelle somme de puissances d'exposant limité à trois.

\change
On recommence
\change
et encore.

Que remarque-t-on dans cette dernière formulation de $n$ ?

Il n'y a plus de puissances de 2, mais à la place seules des multiplications par 2 et des additions. Et outre les facteurs 2, ne figurent que les bits de $n$.

Si on effectue toutes ces opérations dans l'ordre imposé par le parenthésage, nous obtenons (sans surprise) le nombre
\change 
47.

\change
Donc dans cette méthode, n'apparaissent que
\change
- les bits de l'écriture binaire

\change
- des additions et des multiplications par 2.

%%%%%%%%%%%%%%%%%%%%%%%%%%%%%%%%%%%%%%%%%%%%%%%%%%%%%%%%%%%
\diapo % Calcul d'un entier représenté en binaire 3 Algo de Horner formalisé

Terminons cette partie par une présentation formalisée de l'algorithme de Horner que nous venons de voir sur un exemple.

\hrule\medskip

Cet algorithme prend en entrée une représentation binaire d'un entier naturel $n$ les bits étant indexés de droite à gauche de $0$ à $t-1$. Autrement dit l'entier $n$ est écrit avec $t$ bits.

Et cet algorithme renvoie la valeur de $n$.

\hrule\medskip

On commence par initialiser une variable $n$ en lui attribuant pour valeur le bit le plus à gauche. (ligne 1)

\hrule\medskip
Les lignes 2 à 4 forment une boucle pour marquant la succession de multiplication par 2 et d'ajout d'un nouveau bit.
Les indices de parcours des bits vont de $t-2$ à $0$ en décroissant.

\hrule\medskip

Une fois cette itération terminée, la valeur de la variable $n$ est l'entier qu'on souhaite.




\end{document}
