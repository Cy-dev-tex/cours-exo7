\documentclass[class=report,crop=false]{standalone}
\usepackage[screen]{../exo7book}

% Commande ponctuelle
\newcommand{\alenvers}[1]{\rotatebox[origin=c]{180}{#1}}
\newcommand{\vect}{\overrightarrow}

\begin{document}

%====================================================================
\chapitre{Calcul formel}
%====================================================================

%\insertvideo{CzGMmG7klkM}{partie 3. Suites récurrentes et preuves formelles}

%%%%%%%%%%%%%%%%%%%%%%%%%%%%%%%%%%%%%%%%%%%%%%%%%%%%%%%%%%%%%%%%
\setcounter{section}{2}
\section{Suites récurrentes et preuves formelles}

Commençons par un exemple. Après avoir déclaré les deux variables $a$ et $b$
par \codeinline{var('a,b')}, la commande

\centerline{\codeinline{expand((a+b)^2-a^2-2*a*b-b^2)}}

renvoie $0$.
Ce simple calcul doit être considéré comme une preuve par ordinateur
de l'identité $(a+b)^2 = a^2+2ab+b^2$,
sans qu'il y ait obligation de vérification.
Cette << preuve >> est fondée sur la confiance que nous avons en \Sage\ 
pour manipuler les expressions algébriques.
Nous allons utiliser la puissance du calcul formel, non seulement pour faire des 
expérimentations, mais aussi pour effectuer des preuves (presque) à notre place.

\bigskip

Rappelons la démarche que l'on adopte :
\begin{enumerate}
  \item Je me pose des questions.
  
  \item J'écris un algorithme pour tenter d'y répondre.
  
  \item Je trouve une conjecture sur la base des résultats expérimentaux observés.
  
  \item Je tente de prouver la conjecture.
\end{enumerate}



%--------------------------------------------------------
\subsection{La suite de Fibonacci}


La suite de Fibonacci est définie par les relations suivantes :
$$F_0 = 0, \qquad F_1 = 1 \qquad \mbox{et} \qquad F_{n} = F_{n-1} + F_{n-2} \quad \text{ pour } n\ge 2.$$


\begin{tp}
\sauteligne
\begin{enumerate}
  \item Calculer les $10$ premiers termes de la suite de Fibonacci.
  \item \textbf{Recherche d'une conjecture.} On cherche s'il peut exister 
  des constantes $\phi,\psi,c$ réelles telles que
  $$F_n = c \big(\phi^n-\psi^n\big).$$
  Nous allons calculer ces constantes.
   \begin{enumerate}
    \item On pose $G_n = \phi^n$. Quelle équation doit satisfaire $\phi$
    afin que $G_{n} = G_{n-1}+G_{n-2}$ ?
    \item Résoudre cette équation.
    \item Calculer $c,\phi,\psi$.
  \end{enumerate}
  
  \item \textbf{Preuve.} On note $H_n = \frac{1}{\sqrt{5}}  \big(\phi^n-\psi^n\big)$
  où $\phi = \frac{1+\sqrt5}{2}$ et $\psi=\frac{1-\sqrt5}{2}$.
  \begin{enumerate}
    \item Vérifier que $F_n=H_n$ pour les premières valeurs de $n$.
    \item Montrer à l'aide du calcul formel que $H_{n} = H_{n-1}+H_{n-2}$. Conclure.
  \end{enumerate}  
\end{enumerate}
\end{tp}

\begin{enumerate}
  \item 


Voici une fonction qui calcule le terme de rang $n$ de la suite de Fibonacci
en appliquant la formule de récurrence.

\insertcode{algos/fibonacci-tex1.sage}{fibonacci.sage (1)}

On affiche les valeurs de $F_0$ à $F_9$ :
\insertcode{algos/fibonacci-tex2.sage}{fibonacci.sage (2)}


Ce qui donne :
$$0 \qquad 1 \qquad 1 \qquad 2 \qquad 3 \qquad 5 \qquad 8 \qquad 13 \qquad 21 
\qquad 34 $$

  \item 
  \begin{enumerate}
    \item On pose $G_n = \phi^n$ et on suppose que $G_{n} = G_{n-1}+G_{n-2}$.
  C'est-à-dire que l'on veut résoudre $\phi^n = \phi^{n-1} + \phi^{n-2}$.
  Notons $(E)$ cette équation $X^n = X^{n-1} + X^{n-2}$,
  ce qui équivaut à l'équation $X^{n-2}\big(X^2-X-1 \big) = 0$.
  On écarte la solution triviale $X=0$ pour obtenir l'équation 
  $X^2-X-1 = 0$.  
  
  
    \item On profite de \Sage\ pour calculer les deux racines de cette dernière
  équation.
  \insertcode{algos/fibonacci-tex3.sage}{fibonacci.sage (3)}
  
  Les solutions sont fournies sous la forme d'une liste que l'on appelle
  \codeinline{solutions}. On récupère chaque solution
  avec \codeinline{solutions[0]} et \codeinline{solutions[1]}.
  Par exemple, \codeinline{solutions[1]} renvoie \codeinline{x == 1/2*sqrt(5) + 1/2}.
  Pour récupérer la partie droite de cette solution et la nommer $\phi$, 
  on utilise la fonction \codeinline{rhs()} (pour \emph{right hand side})
  qui renvoie la partie droite d'une équation (de même \codeinline{lhs()}, 
  pour \emph{left hand side} renverrait la partie gauche d'une équation).
  
  On obtient donc les deux racines
  $$\phi = \frac{1+\sqrt5}{2} \qquad \mbox{et} \qquad \psi  = \frac{1-\sqrt5}{2}.$$
  
  $\phi^n$ et $\psi^n$ vérifient la même relation de récurrence que la suite de Fibonacci
  mais bien sûr les valeurs prises ne sont pas entières.
  
    \item Comme $F_1=1$ alors $c(\phi-\psi)=1$, ce qui donne $c = \frac{1}{\sqrt 5}$.
  \end{enumerate}  
  La conséquence de notre analyse est que \emph{si} une formule 
  $F_n = c(\phi^n-\psi^n)$ existe alors les constantes sont
  nécessairement $c= \frac{1}{\sqrt 5}$, $\phi = \frac{1+\sqrt5}{2}$, 
  $\psi  = \frac{1-\sqrt5}{2}$.
  

  
  \item  
  Dans cette question, nous allons maintenant montrer avec \Sage\ qu'avec ces constantes on a effectivement 
  $$F_n = \frac{1}{\sqrt5}\left(\left(\frac{1+\sqrt5}{2}\right)^n- 
  \left(\frac{1-\sqrt5}{2}\right)^n\right).$$

%  $$\displaystyle F_n = \frac{1}{\sqrt5}\left(\left(\tfrac{1+\sqrt5}{2}\right)^n- 
%    \left(\tfrac{1-\sqrt5}{2}\right)^n\right).$$
  \begin{enumerate}
    \item On définit une nouvelle fonction \codeinline{conjec} qui correspond à l'expression de $H_n$,
    la conjecture que l'on souhaite montrer.
    On vérifie que l'on a l'égalité souhaitée pour les premières valeurs de $n$.
    
    \insertcode{algos/fibonacci-tex4.sage}{fibonacci.sage (4)}
    
    \item Nous allons montrer que $F_n=H_n$ pour tout $n\ge0$ en nous aidant du calcul formel.
    \begin{itemize}
      \item \textbf{Initialisation.}
      Il est clair que $F_0=H_0=0$ et $F_1=H_1=1$.
      On peut même demander à l'ordinateur de le faire, en effet
      \codeinline{fibonacci(0)-conjec(0)} et \codeinline{fibonacci(1)-conjec(1)}
      renvoient tous les deux $0$.
      
      \item \textbf{Relation de récurrence.}
      Nous allons montrer que $H_n$ vérifie exactement la même relation de récurrence que les nombres de Fibonacci.
      Encore une fois, on peut le faire à la main ou bien demander à l'ordinateur de le faire pour nous :
      \insertcode{algos/fibonacci-tex5.sage}{fibonacci.sage (5)}
      

      
      Le résultat est $0$. Comme le calcul est valable pour la variable formelle $n$,
      il est vrai quel que soit $n\ge0$. Ainsi $H_n=H_{n-1}+H_{n-2}$.    
      Il a tout de même fallu préciser à \Sage\ de simplifier les racines carrées
      avec la commande \codeinline{canonicalize_radical()} (la commande \codeinline{simplify_radical} ne fait plus partie des nouvelles version de \Sage\ ).   
      
      \item \textbf{Conclusion.} 
      Les suites $(F_n)$ et $(H_n)$ ont les mêmes termes initiaux et 
      vérifient la même relation de récurrence. Elles ont donc les mêmes termes pour chaque rang :
      pour tout $n\ge0$, $F_n=H_n$.
      
      
    \end{itemize}

  \end{enumerate}
  
\end{enumerate}

%--------------------------------------------------------
\subsection{L'identité de Cassini}


A votre tour maintenant d'expérimenter, conjecturer et prouver 
\emph{l'identité de Cassini}.
\begin{tp}
Calculer, pour différentes valeurs de $n$, la quantité
$$F_{n+1}F_{n-1}-F_n^2.$$

Que constatez-vous ? 
Proposez une formule générale.
Prouvez cette formule qui s'appelle l'identité de Cassini.   
\end{tp}


Notons pour $n \ge 1$ :
$$C_n = F_{n+1}F_{n-1}-F_n^2.$$
Après quelques tests, on conjecture la formule très simple $C_n = (-1)^n$.
  
Voici deux preuves : une preuve utilisant le calcul formel
et une autre à la main.

\textbf{Première preuve -- \`A l'aide du calcul formel.}


On utilise la formule   
$$F_n = \frac{1}{\sqrt5}\left(\left(\frac{1+\sqrt5}{2}\right)^n- 
  \left(\frac{1-\sqrt5}{2}\right)^n\right)$$
et on demande simplement à l'ordinateur de vérifier l'identité !

\insertcode{algos/cassini-tex1.sage}{cassini.sage (1)}


\insertcode{algos/cassini-tex2.sage}{cassini.sage (2)}

L'ordinateur effectue (presque) tous les calculs pour nous.
Le seul problème est que \Sage\ échoue de peu à simplifier l'expression 
contenant des racines carrées (peut-être les versions futures feront mieux ?).
En effet, après simplification on obtient :
$$C_n = (-1) ^n \frac{(\sqrt 5 -1)^n (\sqrt 5 + 1)^n}{2^{2n}}.$$

Il ne reste qu'à conclure à la main. 
En utilisant l'identité remarquable $(a-b)(a+b)=a^2-b^2$,
on obtient facilement $(\sqrt 5 -1)(\sqrt 5 + 1) = (\sqrt 5)^2 - 1 = 4$
et donc $C_n = (-1) ^n \frac{4^n}{2^{2n}} = (-1)^n$.

\bigskip

\textbf{Seconde preuve -- Par récurrence.}

On souhaite montrer que l'assertion
$$(\mathcal{H}_n) \qquad F_n^2 = F_{n+1}F_{n-1} - (-1)^n$$
est vrai pour tout $n \ge 1$.

\begin{itemize}
  \item Pour $n=1$, $F_1^2 = 1^2 = 1\times 0 - (-1)^1 = F_2\times F_0 - (-1)^1$.
  $\mathcal{H}_1$ est donc vraie.
  \item Fixons $n \ge 1$ et supposons $\mathcal{H}_n$ vraie.
  Montrons que $\mathcal{H}_{n+1}$ est vraie. On a : 
  $$F_{n+1}^2 = F_{n+1}\times (F_n+F_{n-1})
  = F_{n+1}F_n+F_{n+1}F_{n-1}.$$
  Par l'hypothèse $\mathcal{H}_n$, $F_{n+1}F_{n-1}=F_n^2+(-1)^n$.
  Donc
  $$F_{n+1}^2 =F_{n+1}F_n+ F_n^2+(-1)^n = (F_{n+1}+F_n)F_n - (-1)^{n+1} 
  = F_{n+2}F_n - (-1)^{n+1}.$$
  Donc $\mathcal{H}_{n+1}$ est vraie.
  \item Conclusion, par le principe de récurrence, pour tout $n\ge1$, 
  l'identité de Cassini est vérifiée.
\end{itemize}




%--------------------------------------------------------
\subsection{Une autre suite récurrente double}

\begin{tp}
Soit $(u_n)_{n \in \Nn}$ la suite définie par 
$$u_0 = \frac32, \qquad u_1 = \frac53 \qquad \mbox{et} \qquad u_n = 1 + 4\frac{u_{n-2}-1}{u_{n-1}u_{n-2}}
\quad \text{ pour } n\ge 2.$$
Calculer les premiers termes de cette suite. \'Emettre une conjecture. 
La prouver par récurrence.

\emph{Indication.} Les puissances de $2$ seront utiles...
\end{tp}

Commençons par la partie expérimentale, le calcul des premières valeurs de la suite $(u_n)$ :
\insertcode{algos/suite-recurrente-tex1.sage}{suite-recurrente.sage (1)}

Notez l'utilisation de la double affectation du type \codeinline{x, y = newx, newy}.
C'est très pratique comme par exemple dans \codeinline{a, b = a+b, a-b}
et évite l'introduction d'une variable auxiliaire.

Cette fonction \codeinline{masuite} permet de calculer les premiers termes de la suite $(u_n)$ :
$$\frac32 \qquad \frac 53 \qquad \frac95 \qquad \frac{17}{9} \qquad \frac{33}{17} \qquad \frac{65}{33} \qquad \cdots$$


Au vu des premiers termes, on conjecture que notre suite $(u_n)$ coïncide 
avec la suite $(v_n)$ définie par :
$$v_n = \frac{2^{n+1} + 1}{2^n + 1}.$$

On définit donc une fonction qui renvoie la valeur de $v_n$ en fonction de $n$.
\insertcode{algos/suite-recurrente-tex2.sage}{suite-recurrente.sage (2)} 


Pour renforcer notre conjecture, on vérifie que $(u_n)$ et $(v_n)$ sont égales pour les $20$ premiers termes :

\insertcode{algos/suite-recurrente-tex3.sage}{suite-recurrente.sage (3)} 

%La commande \codeinline{var('n')} permet ici de déclarer la variable $n$ afin de l'utiliser
%dans la boucle. \`A la fin de la boucle $n$ vaut $19$.
Il est à noter que l'on peut calculer $u_n$ pour des valeurs de $n$ aussi grandes que l'on veut mais que l'on n'a pas une formule
directe pour calculer $u_n$ en fonction de $n$. L'intérêt de $(v_n)$ est de fournir une 
telle formule.

\bigskip

Voici maintenant une preuve assistée par l'ordinateur.
Notre conjecture à démontrer est : pour tout $n \ge 0$, $v_n=u_n$.
Pour cela, nous allons va prouver que $v_n$ et $u_n$ coïncident sur les deux premiers termes et vérifient
la même relation de récurrence. Ce seront donc deux suites égales.

\begin{itemize}
  \item \textbf{Initialisation.} Vérifions que $v_0=u_0$ et $v_1=u_1$.
  Il suffit de vérifier que \codeinline{conjec(0)-3/2}
  et \codeinline{conjec(1)-5/3} renvoient tous les deux $0$.
  
  \item \textbf{Relation de récurrence.}
  Par définition, $(u_n)$ vérifie la relation de récurrence 
  $u_n = 1 + 4\frac{u_{n-2}-1}{u_{n-1}u_{n-2}}$.
  Définissons une fonction correspondant à cette relation : 
  à partir d'une valeur $x$ (correspond à $u_{n-1}$) et $y$ (correspondant à $u_{n-2}$)
  on renvoie $1 + 4\frac{y-1}{xy}$.
  
  \insertcode{algos/suite-recurrente-tex4.sage}{suite-recurrente.sage (4)} 
  
  On demande maintenant à l'ordinateur de vérifier que la suite $(v_n)$ vérifie aussi 
  cette relation de récurrence :
  \insertcode{algos/suite-recurrente-tex5.sage}{suite-recurrente.sage (5)}
  
  Le résultat obtenu est $0$. Donc $v_n$ vérifie pour tout $n$ 
  la relation $v_n = 1 + 4\frac{v_{n-2}-1}{v_{n-1}v_{n-2}}$.
  
  \item \textbf{Conclusion.} Les suites $(v_n)$ et $(u_n)$ sont égales aux rangs $0$ et $1$ 
  et vérifient la même relation de récurrence pour $n\ge2$. Ainsi $u_n=v_n$ pour tout $n\in \Nn$.
  Les suites $(u_n)$ et $(v_n)$ sont donc égales.
  
\end{itemize}
Notez qu'il a fallu guider \Sage\ pour simplifier notre résultat à l'aide de la fonction
\codeinline{simplify_rational} qui simplifie les fractions rationnelles.

\begin{remarque*}
Quelques commentaires sur les dernières lignes de code.
\begin{itemize}
  \item La commande \codeinline{var('n')} est essentielle. Elle permet de réinitialiser 
  la variable $n$ en une variable formelle. Souvenez-vous qu'auparavant 
  $n$ valait $19$. Après l'instruction \codeinline{var('n')} la variable $n$ redevient 
  une indéterminée \codeinline{'n'} sans valeur spécifique.
   
  \item Pour bien comprendre ce point, supposons que vous souhaitiez vérifier que
  $(n+2)^3 = n^3 + 6n^2+12n+8$.
  Vous pouvez-bien sûr le vérifier par exemple pour les vingts premières valeurs de $n$ : $0,1,2,\ldots,19$.
  Mais en définissant la variable formelle \codeinline{var('n')} et à l'aide
  de la commande \codeinline{expand((n+2)^3-n^3-6*n^2-12*n-8))} le logiciel de calcul formel
  << prouve >> directement que l'égalité est vérifiée pour tout $n$.
  Si vous oubliez de réinitialiser la variable $n$ en une variable formelle, 
  vous ne l'aurez vérifiée que pour les vingt premiers entiers !
 
\end{itemize}
  
  
\end{remarque*}


% %--------------------------------------------------------
% %\subsection{Mini-exercices}
% 
% \begin{miniexercices}
% \begin{enumerate}
%   \item Trouver une expression simple pour $1^2+3^2+5^2+\cdots +(2n+1)^2$.
%   Puis prouver la formule.
%   
%   \item   
%   
%   \item Nous nous intéressons à une propriété très intéressante du Triangle de Pascal : 
%   les sommes des diagonales forment la suite de Fibonacci : 
% \end{enumerate}
% \end{miniexercices}

\finchapitre
\end{document}

