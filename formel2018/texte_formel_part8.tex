
%%%%%%%%%%%%%%%%%% PREAMBULE %%%%%%%%%%%%%%%%%%


\documentclass[12pt]{article}

\usepackage{amsfonts,amsmath,amssymb,amsthm}
\usepackage[utf8]{inputenc}
\usepackage[T1]{fontenc}
\usepackage[francais]{babel}


% packages
\usepackage{amsfonts,amsmath,amssymb,amsthm}
\usepackage[utf8]{inputenc}
\usepackage[T1]{fontenc}
%\usepackage{lmodern}

\usepackage[francais]{babel}
\usepackage{fancybox}
\usepackage{graphicx}

\usepackage{float}

%\usepackage[usenames, x11names]{xcolor}
\usepackage{tikz}
\usepackage{datetime}

\usepackage{mathptmx}
%\usepackage{fouriernc}
%\usepackage{newcent}
\usepackage[mathcal,mathbf]{euler}

%\usepackage{palatino}
%\usepackage{newcent}


% Commande spéciale prompteur

%\usepackage{mathptmx}
%\usepackage[mathcal,mathbf]{euler}
%\usepackage{mathpple,multido}

\usepackage[a4paper]{geometry}
\geometry{top=2cm, bottom=2cm, left=1cm, right=1cm, marginparsep=1cm}

\newcommand{\change}{{\color{red}\rule{\textwidth}{1mm}\\}}

\newcounter{mydiapo}

\newcommand{\diapo}{\newpage
\hfill {\normalsize  Diapo \themydiapo \quad \texttt{[\jobname]}} \\
\stepcounter{mydiapo}}


%%%%%%% COULEURS %%%%%%%%%%

% Pour blanc sur noir :
%\pagecolor[rgb]{0.5,0.5,0.5}
% \pagecolor[rgb]{0,0,0}
% \color[rgb]{1,1,1}



%\DeclareFixedFont{\myfont}{U}{cmss}{bx}{n}{18pt}
\newcommand{\debuttexte}{
%%%%%%%%%%%%% FONTES %%%%%%%%%%%%%
\renewcommand{\baselinestretch}{1.5}
\usefont{U}{cmss}{bx}{n}
\bfseries

% Taille normale : commenter le reste !
%Taille Arnaud
%\fontsize{19}{19}\selectfont

% Taille Barbara
%\fontsize{21}{22}\selectfont

%Taille François
%\fontsize{25}{30}\selectfont

%Taille Pascal
%\fontsize{25}{30}\selectfont

%Taille Laura
%\fontsize{30}{35}\selectfont


%\myfont
%\usefont{U}{cmss}{bx}{n}

%\Huge
%\addtolength{\parskip}{\baselineskip}
}


% \usepackage{hyperref}
% \hypersetup{colorlinks=true, linkcolor=blue, urlcolor=blue,
% pdftitle={Exo7 - Exercices de mathématiques}, pdfauthor={Exo7}}


%section
% \usepackage{sectsty}
% \allsectionsfont{\bf}
%\sectionfont{\color{Tomato3}\upshape\selectfont}
%\subsectionfont{\color{Tomato4}\upshape\selectfont}

%----- Ensembles : entiers, reels, complexes -----
\newcommand{\Nn}{\mathbb{N}} \newcommand{\N}{\mathbb{N}}
\newcommand{\Zz}{\mathbb{Z}} \newcommand{\Z}{\mathbb{Z}}
\newcommand{\Qq}{\mathbb{Q}} \newcommand{\Q}{\mathbb{Q}}
\newcommand{\Rr}{\mathbb{R}} \newcommand{\R}{\mathbb{R}}
\newcommand{\Cc}{\mathbb{C}} 
\newcommand{\Kk}{\mathbb{K}} \newcommand{\K}{\mathbb{K}}

%----- Modifications de symboles -----
\renewcommand{\epsilon}{\varepsilon}
\renewcommand{\Re}{\mathop{\text{Re}}\nolimits}
\renewcommand{\Im}{\mathop{\text{Im}}\nolimits}
%\newcommand{\llbracket}{\left[\kern-0.15em\left[}
%\newcommand{\rrbracket}{\right]\kern-0.15em\right]}

\renewcommand{\ge}{\geqslant}
\renewcommand{\geq}{\geqslant}
\renewcommand{\le}{\leqslant}
\renewcommand{\leq}{\leqslant}

%----- Fonctions usuelles -----
\newcommand{\ch}{\mathop{\mathrm{ch}}\nolimits}
\newcommand{\sh}{\mathop{\mathrm{sh}}\nolimits}
\renewcommand{\tanh}{\mathop{\mathrm{th}}\nolimits}
\newcommand{\cotan}{\mathop{\mathrm{cotan}}\nolimits}
\newcommand{\Arcsin}{\mathop{\mathrm{Arcsin}}\nolimits}
\newcommand{\Arccos}{\mathop{\mathrm{Arccos}}\nolimits}
\newcommand{\Arctan}{\mathop{\mathrm{Arctan}}\nolimits}
\newcommand{\Argsh}{\mathop{\mathrm{Argsh}}\nolimits}
\newcommand{\Argch}{\mathop{\mathrm{Argch}}\nolimits}
\newcommand{\Argth}{\mathop{\mathrm{Argth}}\nolimits}
\newcommand{\pgcd}{\mathop{\mathrm{pgcd}}\nolimits} 

\newcommand{\Card}{\mathop{\text{Card}}\nolimits}
\newcommand{\Ker}{\mathop{\text{Ker}}\nolimits}
\newcommand{\id}{\mathop{\text{id}}\nolimits}
\newcommand{\ii}{\mathrm{i}}
\newcommand{\dd}{\mathrm{d}}
\newcommand{\Vect}{\mathop{\text{Vect}}\nolimits}
\newcommand{\Mat}{\mathop{\mathrm{Mat}}\nolimits}
\newcommand{\rg}{\mathop{\text{rg}}\nolimits}
\newcommand{\tr}{\mathop{\text{tr}}\nolimits}
\newcommand{\ppcm}{\mathop{\text{ppcm}}\nolimits}

%----- Structure des exercices ------

\newtheoremstyle{styleexo}% name
{2ex}% Space above
{3ex}% Space below
{}% Body font
{}% Indent amount 1
{\bfseries} % Theorem head font
{}% Punctuation after theorem head
{\newline}% Space after theorem head 2
{}% Theorem head spec (can be left empty, meaning ‘normal’)

%\theoremstyle{styleexo}
\newtheorem{exo}{Exercice}
\newtheorem{ind}{Indications}
\newtheorem{cor}{Correction}


\newcommand{\exercice}[1]{} \newcommand{\finexercice}{}
%\newcommand{\exercice}[1]{{\tiny\texttt{#1}}\vspace{-2ex}} % pour afficher le numero absolu, l'auteur...
\newcommand{\enonce}{\begin{exo}} \newcommand{\finenonce}{\end{exo}}
\newcommand{\indication}{\begin{ind}} \newcommand{\finindication}{\end{ind}}
\newcommand{\correction}{\begin{cor}} \newcommand{\fincorrection}{\end{cor}}

\newcommand{\noindication}{\stepcounter{ind}}
\newcommand{\nocorrection}{\stepcounter{cor}}

\newcommand{\fiche}[1]{} \newcommand{\finfiche}{}
\newcommand{\titre}[1]{\centerline{\large \bf #1}}
\newcommand{\addcommand}[1]{}
\newcommand{\video}[1]{}

% Marge
\newcommand{\mymargin}[1]{\marginpar{{\small #1}}}



%----- Presentation ------
\setlength{\parindent}{0cm}

%\newcommand{\ExoSept}{\href{http://exo7.emath.fr}{\textbf{\textsf{Exo7}}}}

\definecolor{myred}{rgb}{0.93,0.26,0}
\definecolor{myorange}{rgb}{0.97,0.58,0}
\definecolor{myyellow}{rgb}{1,0.86,0}

\newcommand{\LogoExoSept}[1]{  % input : echelle
{\usefont{U}{cmss}{bx}{n}
\begin{tikzpicture}[scale=0.1*#1,transform shape]
  \fill[color=myorange] (0,0)--(4,0)--(4,-4)--(0,-4)--cycle;
  \fill[color=myred] (0,0)--(0,3)--(-3,3)--(-3,0)--cycle;
  \fill[color=myyellow] (4,0)--(7,4)--(3,7)--(0,3)--cycle;
  \node[scale=5] at (3.5,3.5) {Exo7};
\end{tikzpicture}}
}



\theoremstyle{definition}
%\newtheorem{proposition}{Proposition}
%\newtheorem{exemple}{Exemple}
%\newtheorem{theoreme}{Théorème}
\newtheorem{lemme}{Lemme}
\newtheorem{corollaire}{Corollaire}
%\newtheorem*{remarque*}{Remarque}
%\newtheorem*{miniexercice}{Mini-exercices}
%\newtheorem{definition}{Définition}




%definition d'un terme
\newcommand{\defi}[1]{{\color{myorange}\textbf{\emph{#1}}}}
\newcommand{\evidence}[1]{{\color{blue}\textbf{\emph{#1}}}}



 %----- Commandes divers ------

\newcommand{\codeinline}[1]{\texttt{#1}}

%%%%%%%%%%%%%%%%%%%%%%%%%%%%%%%%%%%%%%%%%%%%%%%%%%%%%%%%%%%%%
%%%%%%%%%%%%%%%%%%%%%%%%%%%%%%%%%%%%%%%%%%%%%%%%%%%%%%%%%%%%%


\begin{document}

\debuttexte


%%%%%%%%%%%%%%%%%%%%%%%%%%%%%%%%%%%%%%%%%%%%%%%%%%%%%%%%%%%
\diapo

Dans cette partie, nous allons apprendre à manipuler des polynômes.

\change
Voici le plan que nous allons suivre :

\change
Nous commencerons par introduire quelques commandes de base,

\change
puis nous étudierons deux problèmes concrets : l'algorithme de Horner 

\change 
puis l'interpolation de Lagrange.



%%%%%%%%%%%%%%%%%%%%%%%%%%%%%%%%%%%%%%%%%%%%%%%%%%%%%%%%%%%
\diapo

Voici un premier tp pour se familiariser avec les commandes élémentaires.

On se donne deux polynômes $P$ et $Q$ explicites à coefficients rationnels et nous allons répondre aux questions suivantes.

\change
Comparer le degré du produit et la somme des degrés,

\change
comparer le degré de la différence  et le maximum des degrés,

\change 
développer $Q(X)$ et déterminer le coefficient de son terme de degré $3$,

\change 
effectuer la division euclidienne du polynôme $P(X)$ par $(X+1)^2$,

\change 
et enfin, trouver les racines de $Q$ et aussi celles de $P$.

%%%%%%%%%%%%%%%%%%%%%%%%%%%%%%%%%%%%%%%%%%%%%%%%%%%%%%%%%%%
\diapo

\change  
On commence par déclarer l'anneau $R$ des polynômes à coefficients dans $\Qq$ (les rationnels) 
où [grand] $X$ est la variable. 

~

Nous définissons ensuite les polynômes $P$ et $Q$.

\change

\change
La commande \codeinline{degree()} permet d'obtenir le degré d'un polynôme.


On peut ainsi tester les égalités sur les degrés proposées par l'énoncé.
  
\change
Cette première égalité est vraie dans le cas particulier de notre exemple, mais est également vraie en général :
le degré du produit est bien la somme des degrés.

\change
Par contre, la seconde est fausse (il n'y a pas toujours égalité des degrés).

L'énoncé vrai en toute généralité est que :
  $\deg(P - Q)$ est inférieur au maximum des degrés de $P$ et $Q$.

\change
On développe un polynôme parla commande \codeinline{expand}.

\change
On récupère les coefficients en considérant le polynôme comme une liste :
  
\codeinline{Q[k]} renvoie le coefficient de $X^k$.
  
\change
 On obtient le quotient et le reste de la division euclidienne de polynômes comme avec les entiers (avec la double barre oblique et le signe de pourcentage).
 
\change	  
 Le quotient \codeinline{$P // (X+1)^2$} vaut $X^2 - 2X$. 

\change	
  Le reste \codeinline{$P \% (X+1)^2$} vaut $2X - 1$.
  
\change
La dernière question sur les racines est ambigüe : il faut en effet préciser dans quel ensemble on cherche les racines.
  
Les cherche-t-on dans le corps des rationnels, des réels ou des complexes ?
% $\Qq$ $\Rr$ ou $\Cc$ ? 
  
De plus, souhaitons-nous une racine exacte ou approchée ?
  
  
\change	  
 La commande \codeinline{roots} sans argument renvoie les racines qui sont dans le corps de base (ici $\Qq$). 
  Sur cet exemple, elle retourne la liste vide car $P$ n'a pas de racines rationnelles.
  
\change	
\codeinline{Q.roots} renvoie une liste composée d'un unique tuple contenant deux éléments $-1$ et $4$. 

Cela signifie que 
%\codeinline{[(-1, 4)]} car 
$-1$ est une racine de multiplicité $4$.
  
\change	
En précisant \codeinline{QQbar} dans cette commande, nous obtenons les racines exactes de $P$ dans $\Cc$ (pour un polynôme à coefficients dans $\Qq$), à savoir ici deux racines réelles et deux racines imaginaires pures.
    % \codeinline{-1.817354021023971?},
    % \codeinline{1.817354021023971?}, 
    % \codeinline{-0.5502505227003375?*I},
    % \codeinline{0.5502505227003375?*I}. 
    % Le point d'interrogation en fin d'écriture signifie que Sage\ calcule avec les valeurs exactes, 
    % mais n'affiche que les premières décimales.

  
\change
En précisant maintenant \codeinline{RR}, nous obntenons les racines réelles \emph{approchées}.
  %  \codeinline{-1.81735402102397}, \codeinline{1.81735402102397}. 
  On quitte donc la résolution formelle
    pour une résolution numérique approchée.
    
  
\change    
De même ici pour les racines complexes \emph{approchées}.        

~

Vous êtes maintenant en possession de quelques outils permettant de manipuler les polynômes.

% \change

% \change



%%%%%%%%%%%%%%%%%%%%%%%%%%%%%%%%%%%%%%%%%%%%%%%%%%%%%%%%%%%
\diapo

Dans ce tp, nous allons tracer dans le plan complexe l'image d'un cercle centré à l'origine et de rayon $r$, 
par un polynôme $P$ à coefficients complexes.

\change
Le but est, en faisant varier le paramètre $r$, d'obtenir une illustration graphique du théorème de d'Alembert-Gauss.

\change
Nous mènerons les calculs et les visualisations pour un polynôme particulier.

%%%%%%%%%%%%%%%%%%%%%%%%%%%%%%%%%%%%%%%%%%%%%%%%%%%%%%%%%%%
\diapo

Voici, pour quelques valeurs de $r$, les images extraites de l'animation.

%des valeurs de $r$ 
%valant
%successivement : $r_0 = 0.5$, $r_1 = 0.6176\ldots$,
%$r_2 = 0.9534\ldots$, $r_3 = 1.2082\ldots$, $r_4 = 1.4055\ldots$ et $r_5 = 1.5$.

\change

Les valeurs intéressantes sont lorsque la courbe passe par l'origine, comme ici.

En effet, la courbe passe par l'origine
lorsqu'il existe un nombre complexe $re^{\ii t}$ tel que $P(re^{\ii t}) = 0$, autrement dit lorsque 
$P$ admet une racine de module $r$.

~

 D'après le théorème de d'Alembert-Gauss, un polynôme de degré $n$ admet au plus
$n$ racines. 
%Ainsi, il y a au plus $n$ valeurs $r_1,\ldots,r_n$ pour lesquelles $\mathcal{C}_{r_i}$ passe par l'origine.


Pour notre polynôme exemple, de degré $4$, 

\change
Voici l'illustration de l'existence d'une deuxième racine dont le module est différent de celui de la première..

\change
Puis une troisème.

\change
Et enfin d'une quatrième.

Nous obtenons donc $4$ racines, ici de modules distincts.

\change
Ensuite la courbe ne semble plus repasser par l'origine.




%%%%%%%%%%%%%%%%%%%%%%%%%%%%%%%%%%%%%%%%%%%%%%%%%%%%%%%%%%%
\diapo

%\change
Voici maintenant comment procéder :


On commence par définir l'anneau $\Cc[X]$ des polynômes à coefficients complexes non formels.
	 %par \codeinline{R.<X> = CC[]}.
  Les calculs seront donc des calculs approchés.% avec des nombres complexes.
  %Notre polynôme $P$ est défini par \codeinline{$P = X^4-X^3+X^2-I*X+1$}.
  
  Puis on définit le polynôme $P$ pour l'illustration.
  
  
\change
  Le cercle de rayon $r$ est l'ensemble des complexes de la forme $re^{\ii t}$ pour $t\in [0,2\pi]$.

  La courbe cherchée, c'est-à-dire l'image du cercle centré en l'origine et de rayon $r$ est donc l'ensemble  des $P(re^{\ii t})$ pour $ t \in  [0,2\pi]$.
  
  Nous pouvons décrire ces points images sous forme partie réelle et partie imaginaire et obtenir ainsi une courbe paramétré.

Une animation est simplement une juxtaposition d'images. 

Chaque image est obtenue par notre fonction. Il suffit de faire varier le rayon $r$ : ici de $0.5$ à $1.5$ par pas de $0.05$.

~

Il n'y a plus qu'à l'afficher.


%%%%%%%%%%%%%%%%%%%%%%%%%%%%%%%%%%%%%%%%%%%%%%%%%%%%%%%%%%%
\diapo

L'algorithme de Horner permet d'évaluer rapidement un polynôme $P$ en une valeur $\alpha$.

\change
Voyons comment il fonctionne à la main avec l'exemple de $P(X) = X^5 + X^4 - 5X^3 - 3X - 2$
et $\alpha = 2$.


\change
Sur la première ligne du tableau, on écrit les coefficients de $P$ : $a_n,a_{n-1},\ldots,a_1,a_0$. Ici, $a_5=1$, $a_4=1$...

\change
On reporte en troisième ligne / première colonne, $a_n$, le coefficient dominant de $P$.

\change
On multiplie ce nombre par $\alpha$ et on écrit ce produit en deuxième ligne / deuxième colonne, c'est-à-dire juste au dessus à droite 
% (sur la deuxième ligne, décalé d'un cran à droite),

\change
puis on ajoute $a_{n-1}$ et on écrit cette somme en troisième ligne / deuxième colonne, c'est-à-dire juste au dessous.
% (sur la troisième ligne juste en dessous). 

\change
Et on recommence. À chaque étape, on multiplie le dernier nombre obtenu par $\alpha$

\change
 et on lui ajoute le coefficient de $P$ correspondant. 
 
\change

\change

\change

\change

\change

\change
Le dernier nombre obtenu est $P(\alpha)$.


Ici le dernier nombre est $0$, donc $P(2)=0$. 

\change
Conclusion : $2$ est racine de $P$.

\change
Avec le même polynôme et en choisissant maintenant $\alpha=-1$, le tableau se complète ainsi :

\change
ce qui implique $P(-1)=6$.



%%%%%%%%%%%%%%%%%%%%%%%%%%%%%%%%%%%%%%%%%%%%%%%%%%%%%%%%%%%
\diapo

Dans ce tp, nous allons voir que cet algorithme de Horner permet d'effectuer des calculs efficaces avec les polynômes.


Dans la première question, nous comparerons le nombre d'opérations nécessaires pour évaluer un polynôme en une valeur donnée 

par le calcul direct, 

puis par le schéma de Horner qui consiste en fait à évaluer $P(\alpha)$ en l'écrivant :

$$P(\alpha) =  \Big(\big((a_n \alpha+a_{n-1})\alpha +a_{n-2} \big) \alpha + \cdots +a_1\Big) \alpha +a_0.$$

Ensuite, nous écrirons une fonction qui, utilisant l'algorithme de Horner, permettra d'évaluer un polynôme en une valeur donnée.

%%%%%%%%%%%%%%%%%%%%%%%%%%%%%%%%%%%%%%%%%%%%%%%%%%%%%%%%%%%
\diapo

Tout d'abord, le calcul d'un monôme $a_k\cdot \alpha^k$ de degré $k$ nécessite naïvement $k$ multiplications.

Donc pour calculer
$P(\alpha)$ par le calcul direct, il faut $n+(n-1)+\cdots+k+\cdots+1+0 = \frac{n(n+1)}{2}$ multiplications (et $n$ additions).

\change
La méthode de Horner 
       ne nécessite seulement que $n$ multiplications (et $n$ additions). 

~

On passe donc d'un coût d'ordre $\frac12n^2$ à un coût d'ordre $n$ ; le gain est conséquent.%énorme. 

\change
    Formalisons le calcul précédent en définissant la suite $b$ par : $b_{n} = a_n$ puis
 $b_{k} = \alpha b_{k+1}  + a_k$.
 
\change
Cette formalisation nous permet d'obtenir l'implémentation suivante : 

  n est le degré de $P$,
  
  on récupère d'abord le coefficient de plus haut degré,
  
  on applique la formule de récurrence pour les indices valant successivement $n-1$, $n-2$,... jusqu'à $0$, ce qui se code par une boucle "pour" de pas -1.

$P(\alpha)$ est alors la valeur $b_0$, que l'on renvoie.




%%%%%%%%%%%%%%%%%%%%%%%%%%%%%%%%%%%%%%%%%%%%%%%%%%%%%%%%%%%
\diapo

Nous passons maintenant à l'interpolation de Lagrange et à la manière dont on peut la programmer.

~

On commence par rappeler l'énoncé mathématique : étant données $n+1$ valeurs $x_i$, et $n+1$ valeurs $y_i$, il existe un unique polynôme de degré au plus $n$ prenant la valeur $y_i$ en $x_i$.

\change
Ce théorème permet en particulier d'approximer une fonction par des polynômes.

Pour cela, on fixe $n+1$ abscisses $x_i$, ce qui nous donne, par $f$, $n+1$ ordonnées $f(x_i)$.

Il existe donc un polynôme de degré inférieur à $n$ qui prend les mêmes valeurs que la fonction $f$ en les $n+1$ valeurs $x_i$. 

Ce polynôme est de plus unique.



%%%%%%%%%%%%%%%%%%%%%%%%%%%%%%%%%%%%%%%%%%%%%%%%%%%%%%%%%%%
\diapo

Dans ce tp consacré à l'interpolation de Lagrange, nous montrerons d'abord l'unicité du polynôme d'interpolation.

\change
Puis nous définirons les polynômes de Lagrange :

le $i$-ème polynôme de Lagrange est le produit des quotients de  $X-x_j$ divisé par $x_i-x_j$, dans lequel l'indice $j$ prend les valeurs de $0$ à $n$ sauf la valeur $i$.

~

Nous montrerons que le polynôme défini par $\displaystyle \sum_{i=0}^{n} y_i L_i(X)$ répond à notre problème d'interpolation.

\change
Bien sûr, nous écrirons ensuite une fonction qui calcule le polynôme d'interpolation et l'utiliserons sur un exemple.
%et appliquez la pour approximer cette fonction.

%%%%%%%%%%%%%%%%%%%%%%%%%%%%%%%%%%%%%%%%%%%%%%%%%%%%%%%%%%%
\diapo


Commençons par l'unicité.

Supposons que $P$ et $Q$ soient deux polynômes de degré inférieur à $n$ vérifiant
tous les deux $P(x_i)=Q(x_i)=y_i$ pour tout $i$ de $0$ à $n$.
  
\change  
Le polynôme $P-Q$ vérifie alors $(P-Q)(x_i)=0$, pour  $i$ variant de $0$ à $n$. 
 
\change 
Ainsi $P-Q$ est un polynôme de degré inférieur à $n$ ayant $n+1$ racines distinctes. 

\change
Ce polynôme est donc nécessairement le polynôme nul. Ainsi $P-Q=0$ et par suite les polynômes $P$ et $Q$ sont égaux.

On a donc montrer que si un tel polynôme d'interpolation existe, il est unique.

Il s'agit maintenant de le trouver !

\change
Nous allons le construire à partir des polynômes de Lagrange $L_i$.

\change
Les polynômes $L_i$ sont de degré exactement $n$ 

\change
et sont définis de sorte que $L_i(x_i) = 1$ et $L_i(x_j)=0$ pour tout  $j$ différent de $i$.
  
\change  
  Il est alors clair que le polynôme $P$ défini par $P(X) = \displaystyle \sum_{j=0}^{n} y_j L_j(X)$ est aussi de degré inférieur à $n$ 

\change   
et surtout que, par construction, il vérifie $P(x_i) = y_i$ pour tout $i$ variant de $0$ à $n$.
  
\change
Pour l'implémentation, on commence par définir l'anneau de polynômes (les coefficients sont ici des réels approchés),

\change
la fonction à interpoler,

\change
et la liste des points d'interpolation de la fonction $f$, ici le nombre de points est fixé en fonction de $n$ et les abscisses sont équidistantes.


%%%%%%%%%%%%%%%%%%%%%%%%%%%%%%%%%%%%%%%%%%%%%%%%%%%%%%%%%%%
\diapo

La fonction suivante renvoie le polynôme interpolateur souhaité pour une liste de points donnée en paramètre.

$n$ est le degré (le nombre de points moins un),

on extrait la liste des abscisses,

et celle des ordonnées.

\change

On construit ensuite la liste des polynômes de Lagrange.

Pour cela, on calcule, au préalable, le numérateur $A$

et le dénominateur $B$ pour chaque polynôme de Lagrange.

\change
Il ne reste plus alors qu'à calculer le polynôme interpolateur par 
la formule $\displaystyle \sum_{i=0}^{n} y_i L_i$ donnée par le théorème démontré précédemment.

~

Cette fonction pourra maintenant ête utilisée pour différentes listes de points d'interpolation.

%%%%%%%%%%%%%%%%%%%%%%%%%%%%%%%%%%%%%%%%%%%%%%%%%%%%%%%%%%%
\diapo
Nous allons voir dans ce qui suit quelques tracés d'interpolation de la fonction donnée dans l'énoncé : $f(x) = \sin(2\pi x)e^{-x}$.

La courbe représentation de $f$ est en rouge, celle du polynôme d'interpolation en bleu.

Nous avons ici le cas où $n=3$.
  
  Remaquez que la fonction et le polynôme coincident bien en $n+1$ soit $4$ points.
  
  \change
  Pour $n=4$, le polynôme est le polynôme nul et passent bien par ces 5 points.
  
\change
  
$n=5$

\change

$n=6$

\change

$n=7$

\change

  À partir de $n=8$, il devient difficile de distinguer le
  graphe de la fonction de celui du polynôme d'interpolation.




\end{document}
