

%%%%%%%%%%%%%%%%%% PREAMBULE %%%%%%%%%%%%%%%%%%

\documentclass[aspectratio=169,utf8]{beamer}
%\documentclass[aspectratio=169,handout]{beamer}

\usetheme{Boadilla}
%\usecolortheme{seahorse}
\usecolortheme[RGB={245,66,24}]{structure}
\useoutertheme{infolines}

% packages
\usepackage{amsfonts,amsmath,amssymb,amsthm}
\usepackage[utf8]{inputenc}
\usepackage[T1]{fontenc}
\usepackage{lmodern}

\usepackage[francais]{babel}
\usepackage{fancybox}
\usepackage{graphicx}

\usepackage{float}
\usepackage{xfrac}

%\usepackage[usenames, x11names]{xcolor}
\usepackage{tikz}
\usepackage{pgfplots}
\usepackage{datetime}



%-----  Package unités -----
\usepackage{siunitx}
\sisetup{locale = FR,detect-all,per-mode = symbol}

%\usepackage{mathptmx}
%\usepackage{fouriernc}
%\usepackage{newcent}
%\usepackage[mathcal,mathbf]{euler}

%\usepackage{palatino}
%\usepackage{newcent}
% \usepackage[mathcal,mathbf]{euler}



% \usepackage{hyperref}
% \hypersetup{colorlinks=true, linkcolor=blue, urlcolor=blue,
% pdftitle={Exo7 - Exercices de mathématiques}, pdfauthor={Exo7}}


%section
% \usepackage{sectsty}
% \allsectionsfont{\bf}
%\sectionfont{\color{Tomato3}\upshape\selectfont}
%\subsectionfont{\color{Tomato4}\upshape\selectfont}

%----- Ensembles : entiers, reels, complexes -----
\newcommand{\Nn}{\mathbb{N}} \newcommand{\N}{\mathbb{N}}
\newcommand{\Zz}{\mathbb{Z}} \newcommand{\Z}{\mathbb{Z}}
\newcommand{\Qq}{\mathbb{Q}} \newcommand{\Q}{\mathbb{Q}}
\newcommand{\Rr}{\mathbb{R}} \newcommand{\R}{\mathbb{R}}
\newcommand{\Cc}{\mathbb{C}} 
\newcommand{\Kk}{\mathbb{K}} \newcommand{\K}{\mathbb{K}}

%----- Modifications de symboles -----
\renewcommand{\epsilon}{\varepsilon}
\renewcommand{\Re}{\mathop{\text{Re}}\nolimits}
\renewcommand{\Im}{\mathop{\text{Im}}\nolimits}
%\newcommand{\llbracket}{\left[\kern-0.15em\left[}
%\newcommand{\rrbracket}{\right]\kern-0.15em\right]}

\renewcommand{\ge}{\geqslant}
\renewcommand{\geq}{\geqslant}
\renewcommand{\le}{\leqslant}
\renewcommand{\leq}{\leqslant}
\renewcommand{\epsilon}{\varepsilon}

%----- Fonctions usuelles -----
\newcommand{\ch}{\mathop{\text{ch}}\nolimits}
\newcommand{\sh}{\mathop{\text{sh}}\nolimits}
\renewcommand{\tanh}{\mathop{\text{th}}\nolimits}
\newcommand{\cotan}{\mathop{\text{cotan}}\nolimits}
\newcommand{\Arcsin}{\mathop{\text{arcsin}}\nolimits}
\newcommand{\Arccos}{\mathop{\text{arccos}}\nolimits}
\newcommand{\Arctan}{\mathop{\text{arctan}}\nolimits}
\newcommand{\Argsh}{\mathop{\text{argsh}}\nolimits}
\newcommand{\Argch}{\mathop{\text{argch}}\nolimits}
\newcommand{\Argth}{\mathop{\text{argth}}\nolimits}
\newcommand{\pgcd}{\mathop{\text{pgcd}}\nolimits} 


%----- Commandes divers ------
\newcommand{\ii}{\mathrm{i}}
\newcommand{\dd}{\text{d}}
\newcommand{\id}{\mathop{\text{id}}\nolimits}
\newcommand{\Ker}{\mathop{\text{Ker}}\nolimits}
\newcommand{\Card}{\mathop{\text{Card}}\nolimits}
\newcommand{\Vect}{\mathop{\text{Vect}}\nolimits}
\newcommand{\Mat}{\mathop{\text{Mat}}\nolimits}
\newcommand{\rg}{\mathop{\text{rg}}\nolimits}
\newcommand{\tr}{\mathop{\text{tr}}\nolimits}


%----- Structure des exercices ------

\newtheoremstyle{styleexo}% name
{2ex}% Space above
{3ex}% Space below
{}% Body font
{}% Indent amount 1
{\bfseries} % Theorem head font
{}% Punctuation after theorem head
{\newline}% Space after theorem head 2
{}% Theorem head spec (can be left empty, meaning ‘normal’)

%\theoremstyle{styleexo}
\newtheorem{exo}{Exercice}
\newtheorem{ind}{Indications}
\newtheorem{cor}{Correction}


\newcommand{\exercice}[1]{} \newcommand{\finexercice}{}
%\newcommand{\exercice}[1]{{\tiny\texttt{#1}}\vspace{-2ex}} % pour afficher le numero absolu, l'auteur...
\newcommand{\enonce}{\begin{exo}} \newcommand{\finenonce}{\end{exo}}
\newcommand{\indication}{\begin{ind}} \newcommand{\finindication}{\end{ind}}
\newcommand{\correction}{\begin{cor}} \newcommand{\fincorrection}{\end{cor}}

\newcommand{\noindication}{\stepcounter{ind}}
\newcommand{\nocorrection}{\stepcounter{cor}}

\newcommand{\fiche}[1]{} \newcommand{\finfiche}{}
\newcommand{\titre}[1]{\centerline{\large \bf #1}}
\newcommand{\addcommand}[1]{}
\newcommand{\video}[1]{}

% Marge
\newcommand{\mymargin}[1]{\marginpar{{\small #1}}}

\def\noqed{\renewcommand{\qedsymbol}{}}


%----- Presentation ------
\setlength{\parindent}{0cm}

%\newcommand{\ExoSept}{\href{http://exo7.emath.fr}{\textbf{\textsf{Exo7}}}}

\definecolor{myred}{rgb}{0.93,0.26,0}
\definecolor{myorange}{rgb}{0.97,0.58,0}
\definecolor{myyellow}{rgb}{1,0.86,0}

\newcommand{\LogoExoSept}[1]{  % input : echelle
{\usefont{U}{cmss}{bx}{n}
\begin{tikzpicture}[scale=0.1*#1,transform shape]
  \fill[color=myorange] (0,0)--(4,0)--(4,-4)--(0,-4)--cycle;
  \fill[color=myred] (0,0)--(0,3)--(-3,3)--(-3,0)--cycle;
  \fill[color=myyellow] (4,0)--(7,4)--(3,7)--(0,3)--cycle;
  \node[scale=5] at (3.5,3.5) {Exo7};
\end{tikzpicture}}
}


\newcommand{\debutmontitre}{
  \author{} \date{} 
  \thispagestyle{empty}
  \hspace*{-10ex}
  \begin{minipage}{\textwidth}
    \titlepage  
  \vspace*{-2.5cm}
  \begin{center}
    \LogoExoSept{2.5}
  \end{center}
  \end{minipage}

  \vspace*{-0cm}
  
  % Astuce pour que le background ne soit pas discrétisé lors de la conversion pdf -> png
\begin{tikzpicture}
        \fill[opacity=0,green!60!black] (0,0)--++(0,0)--++(0,0)--++(0,0)--cycle; 
\end{tikzpicture}

% toc S'affiche trop tot :
% \tableofcontents[hideallsubsections, pausesections]
}

\newcommand{\finmontitre}{
  \end{frame}
  \setcounter{framenumber}{0}
} % ne marche pas pour une raison obscure

%----- Commandes supplementaires ------

% \usepackage[landscape]{geometry}
% \geometry{top=1cm, bottom=3cm, left=2cm, right=10cm, marginparsep=1cm
% }
% \usepackage[a4paper]{geometry}
% \geometry{top=2cm, bottom=2cm, left=2cm, right=2cm, marginparsep=1cm
% }

%\usepackage{standalone}


% New command Arnaud -- november 2011
\setbeamersize{text margin left=24ex}
% si vous modifier cette valeur il faut aussi
% modifier le decalage du titre pour compenser
% (ex : ici =+10ex, titre =-5ex

\theoremstyle{definition}
%\newtheorem{proposition}{Proposition}
%\newtheorem{exemple}{Exemple}
%\newtheorem{theoreme}{Théorème}
%\newtheorem{lemme}{Lemme}
%\newtheorem{corollaire}{Corollaire}
%\newtheorem*{remarque*}{Remarque}
%\newtheorem*{miniexercice}{Mini-exercices}
%\newtheorem{definition}{Définition}

% Commande tikz
\usetikzlibrary{calc}
\usetikzlibrary{patterns,arrows}
\usetikzlibrary{matrix}
\usetikzlibrary{fadings} 

%definition d'un terme
\newcommand{\defi}[1]{{\color{myorange}\textbf{\emph{#1}}}}
\newcommand{\evidence}[1]{{\color{blue}\textbf{\emph{#1}}}}
\newcommand{\assertion}[1]{\emph{\og#1\fg}}  % pour chapitre logique
%\renewcommand{\contentsname}{Sommaire}
\renewcommand{\contentsname}{}
\setcounter{tocdepth}{2}



%------ Figures ------

\def\myscale{1} % par défaut 
\newcommand{\myfigure}[2]{  % entrée : echelle, fichier figure
\def\myscale{#1}
\begin{center}
\footnotesize
{#2}
\end{center}}


%------ Encadrement ------

\usepackage{fancybox}


\newcommand{\mybox}[1]{
\setlength{\fboxsep}{7pt}
\begin{center}
\shadowbox{#1}
\end{center}}

\newcommand{\myboxinline}[1]{
\setlength{\fboxsep}{5pt}
\raisebox{-10pt}{
\shadowbox{#1}
}
}

%--------------- Commande beamer---------------
\newcommand{\beameronly}[1]{#1} % permet de mettre des pause dans beamer pas dans poly


\setbeamertemplate{navigation symbols}{}
\setbeamertemplate{footline}  % tiré du fichier beamerouterinfolines.sty
{
  \leavevmode%
  \hbox{%
  \begin{beamercolorbox}[wd=.333333\paperwidth,ht=2.25ex,dp=1ex,center]{author in head/foot}%
    % \usebeamerfont{author in head/foot}\insertshortauthor%~~(\insertshortinstitute)
    \usebeamerfont{section in head/foot}{\bf\insertshorttitle}
  \end{beamercolorbox}%
  \begin{beamercolorbox}[wd=.333333\paperwidth,ht=2.25ex,dp=1ex,center]{title in head/foot}%
    \usebeamerfont{section in head/foot}{\bf\insertsectionhead}
  \end{beamercolorbox}%
  \begin{beamercolorbox}[wd=.333333\paperwidth,ht=2.25ex,dp=1ex,right]{date in head/foot}%
    % \usebeamerfont{date in head/foot}\insertshortdate{}\hspace*{2em}
    \insertframenumber{} / \inserttotalframenumber\hspace*{2ex} 
  \end{beamercolorbox}}%
  \vskip0pt%
}


\definecolor{mygrey}{rgb}{0.5,0.5,0.5}
\setlength{\parindent}{0cm}
%\DeclareTextFontCommand{\helvetica}{\fontfamily{phv}\selectfont}

% background beamer
\definecolor{couleurhaut}{rgb}{0.85,0.9,1}  % creme
\definecolor{couleurmilieu}{rgb}{1,1,1}  % vert pale
\definecolor{couleurbas}{rgb}{0.85,0.9,1}  % blanc
\setbeamertemplate{background canvas}[vertical shading]%
[top=couleurhaut,middle=couleurmilieu,midpoint=0.4,bottom=couleurbas] 
%[top=fondtitre!05,bottom=fondtitre!60]



\makeatletter
\setbeamertemplate{theorem begin}
{%
  \begin{\inserttheoremblockenv}
  {%
    \inserttheoremheadfont
    \inserttheoremname
    \inserttheoremnumber
    \ifx\inserttheoremaddition\@empty\else\ (\inserttheoremaddition)\fi%
    \inserttheorempunctuation
  }%
}
\setbeamertemplate{theorem end}{\end{\inserttheoremblockenv}}

\newenvironment{theoreme}[1][]{%
   \setbeamercolor{block title}{fg=structure,bg=structure!40}
   \setbeamercolor{block body}{fg=black,bg=structure!10}
   \begin{block}{{\bf Th\'eor\`eme }#1}
}{%
   \end{block}%
}


\newenvironment{proposition}[1][]{%
   \setbeamercolor{block title}{fg=structure,bg=structure!40}
   \setbeamercolor{block body}{fg=black,bg=structure!10}
   \begin{block}{{\bf Proposition }#1}
}{%
   \end{block}%
}

\newenvironment{corollaire}[1][]{%
   \setbeamercolor{block title}{fg=structure,bg=structure!40}
   \setbeamercolor{block body}{fg=black,bg=structure!10}
   \begin{block}{{\bf Corollaire }#1}
}{%
   \end{block}%
}

\newenvironment{mydefinition}[1][]{%
   \setbeamercolor{block title}{fg=structure,bg=structure!40}
   \setbeamercolor{block body}{fg=black,bg=structure!10}
   \begin{block}{{\bf Définition} #1}
}{%
   \end{block}%
}

\newenvironment{lemme}[0]{%
   \setbeamercolor{block title}{fg=structure,bg=structure!40}
   \setbeamercolor{block body}{fg=black,bg=structure!10}
   \begin{block}{\bf Lemme}
}{%
   \end{block}%
}

\newenvironment{remarque}[1][]{%
   \setbeamercolor{block title}{fg=black,bg=structure!20}
   \setbeamercolor{block body}{fg=black,bg=structure!5}
   \begin{block}{Remarque #1}
}{%
   \end{block}%
}


\newenvironment{exemple}[1][]{%
   \setbeamercolor{block title}{fg=black,bg=structure!20}
   \setbeamercolor{block body}{fg=black,bg=structure!5}
   \begin{block}{{\bf Exemple }#1}
}{%
   \end{block}%
}


\newenvironment{miniexercice}[0]{%
   \setbeamercolor{block title}{fg=structure,bg=structure!20}
   \setbeamercolor{block body}{fg=black,bg=structure!5}
   \begin{block}{Mini-exercices}
}{%
   \end{block}%
}


\newenvironment{tp}[0]{%
   \setbeamercolor{block title}{fg=structure,bg=structure!40}
   \setbeamercolor{block body}{fg=black,bg=structure!10}
   \begin{block}{\bf Travaux pratiques}
}{%
   \end{block}%
}
\newenvironment{exercicecours}[1][]{%
   \setbeamercolor{block title}{fg=structure,bg=structure!40}
   \setbeamercolor{block body}{fg=black,bg=structure!10}
   \begin{block}{{\bf Exercice }#1}
}{%
   \end{block}%
}
\newenvironment{algo}[1][]{%
   \setbeamercolor{block title}{fg=structure,bg=structure!40}
   \setbeamercolor{block body}{fg=black,bg=structure!10}
   \begin{block}{{\bf Algorithme}\hfill{\color{gray}\texttt{#1}}}
}{%
   \end{block}%
}


\setbeamertemplate{proof begin}{
   \setbeamercolor{block title}{fg=black,bg=structure!20}
   \setbeamercolor{block body}{fg=black,bg=structure!5}
   \begin{block}{{\footnotesize Démonstration}}
   \footnotesize
   \smallskip}
\setbeamertemplate{proof end}{%
   \end{block}}
\setbeamertemplate{qed symbol}{\openbox}


\makeatother
\usecolortheme[RGB={102,102,0}]{structure}

%%%%%%%%%%%%%%%%%%%%%%%%%%%%%%%%%%%%%%%%%%%%%%%%%%%%%%%%%%%%%
%%%%%%%%%%%%%%%%%%%%%%%%%%%%%%%%%%%%%%%%%%%%%%%%%%%%%%%%%%%%%

\begin{document}




\title{{\bf Nombres complexes}}
\subtitle{Racines carr\'ees, \'equations du second degr\'e}


\begin{frame}
  
  \debutmontitre

  \pause

{\footnotesize
\hfill
\setbeamercovered{transparent=50}
\begin{minipage}{0.6\textwidth}
  \begin{itemize}
    \item<3-> Racines carrées 
    \item<4-> Équation du second degré
  \end{itemize}
\end{minipage}
}
\vspace*{1cm}
\end{frame}

\setcounter{framenumber}{0}



%%%%%%%%%%%%%%%%%%%%%%%%%%%%%%%%%%%%%%%%%%%%%%%%%%%%%%%%%%%%%%%%
\section{Racines carr\'ees}

\begin{frame}

Une \defi{racine carr\'ee} de $z\in \Cc$ est un nombre complexe $\omega$ tel que
\mybox{$\omega^2=z$}

\pause

Un complexe $z$ admet deux racines carrées : $\omega$ et $-\omega$

\pause
\bigskip
\begin{exemple}
\begin{itemize}
\item Les racines carrées d'un réel positif $x$ sont : $\sqrt{x}$ et $-\sqrt{x}$
\pause
\item Les racines carrées de $-1$ sont : $\ii$ et $-\ii$
\pause
\item Les racines carrées de $\ii$ sont : $\frac{\sqrt{2}}{2}(1+\ii)$ et $-\frac{\sqrt{2}}{2}(1+\ii)$
\end{itemize}
\end{exemple}


\end{frame}


\begin{frame}

\begin{center}
Calcul des racines carrées de $z=a + \ii b$ : \hspace{2ex}
{\color{blue}$\omega= x + \ii y$} et {\color{blue}$-\omega$}
\end{center}

\pause
\begin{proof}
\[
\begin{aligned}
&  \uncover<5->{\left\{ \begin{array}{l}}
  \ \omega^2 \ = \ z \\
   \uncover<5->{\left| \omega \right|^2 = \left| z \right| }
   \uncover<5->{\end{array} \right.}
\pause
\quad\Leftrightarrow\quad
   \uncover<6->{\left\{ \begin{array}{l}}
   \left( x + \ii y \right)^2 = a + \ii b\\
   \uncover<6->{ \left| \omega \right|^2 = \left| z \right| }
   \uncover<6->{\end{array} \right.}
\pause
\quad\Leftrightarrow\qquad
  \left\{ \begin{array}{l}
     {\hspace{-3.5em}\color{blue}(Re)} \hspace{1.5em} x^2 - y^2 = a\\
     {\hspace{-3.5em}\color{blue}(Im)} \hspace{1.5em} 2 xy = b\\
   \uncover<6->{ x^2+y^2 = \sqrt{a^2+b^2}}
  \end{array} \right.\\
\pause\pause\pause
& \quad\Leftrightarrow\quad
  \left\{ \begin{array}{l}
     2 x^2 = \sqrt{a^2 + b^2} + a\\
     2 y^2 = \sqrt{a^2 + b^2} - a\\
     2 xy = b
  \end{array} \right.
\pause
\quad\Leftrightarrow\quad
  \left\{ \begin{array}{l}
     x = \pm \frac{1}{\sqrt{2}}  \sqrt{\sqrt{a^2 + b^2} + a}\\
     y = \pm \frac{1}{\sqrt{2}}  \sqrt{\sqrt{a^2 + b^2} - a}\\
     \alert<9->{2 xy = b}
  \end{array} \right.
\end{aligned}
\]

\pause\pause

Si {\color{red}$b \geqslant 0$}, $x$ et $y$ sont \alert<9->{de
m\^eme signe ou nuls} donc
\[ \omega = x+\ii y =\pm \frac{1}{\sqrt{2}}  \left( \sqrt{\sqrt{a^2 + b^2} + a} 
{\color{red}\,+\,} \ii \sqrt{\sqrt{a^2 + b^2} - a} \right) \]

\pause

et si {\color{red}$b \leqslant 0$}
\[ \omega = \pm \frac{1}{\sqrt{2}}  \left( \sqrt{\sqrt{a^2 + b^2} + a} 
{\color{red}\,-\,} \ii \sqrt{\sqrt{a^2 + b^2} - a} \right) \]
\end{proof}


\end{frame}


\begin{frame}

\begin{exemple}
Les racines carrées de $\ii$ 
\uncover<8->{sont $\frac{\sqrt{2}}{2}(1+\ii)$ et $-\frac{\sqrt{2}}{2}(1+\ii)$}
\end{exemple}

\pause
\begin{proof}
\[
\begin{aligned}
&  \uncover<4->{\left\{ \begin{array}{l}}
  \ \omega^2 \ = \ \ii \\
   \uncover<4->{\left| \omega \right|^2 = \left| \ii \right| }
   \uncover<4->{\end{array} \right.}
\pause
\quad\Leftrightarrow\quad
   \uncover<4->{\left\{ \begin{array}{l}}
   \left( x + \ii y \right)^2 = \ii \\
   \uncover<4->{ \left| \omega \right|^2 = 1 }
   \uncover<4->{\end{array} \right.}
\quad\Leftrightarrow\qquad
  \left\{ \begin{array}{l}
     {\hspace{-3.5em}\color{blue}(Re)} \hspace{1.5em} x^2 - y^2 = 0\\
     {\hspace{-3.5em}\color{blue}(Im)} \hspace{1.5em} 2 xy = 1\\
   \uncover<4->{ x^2+y^2 = 1}
  \end{array} \right.\\
\pause  \pause
& \quad\Leftrightarrow\quad
  \left\{ \begin{array}{l}
     2 x^2 = 1\\
     2 y^2 = 1\\
     2 xy = 1
  \end{array} \right.
\pause
\quad\Leftrightarrow\quad
  \left\{ \begin{array}{l}
     x = \pm \frac{1}{\sqrt{2}} \\
     y = \pm \frac{1}{\sqrt{2}} \\
     2 xy = 1
  \end{array} \right. \\
\pause
& \quad\Leftrightarrow\quad
  \left\{ \begin{array}{ll}
     & \hspace*{-3ex} x + \ii y = \frac{1}{\sqrt{2}}+\ii\frac{1}{\sqrt{2}} \\
     \hspace*{-1.5ex}\text{ou} & \\
     & \hspace*{-3ex} x + \ii y = -\frac{1}{\sqrt{2}}-\ii\frac{1}{\sqrt{2}} \\
  \end{array} \right. \\
\end{aligned}
\]


\end{proof}
  
\end{frame}



%%%%%%%%%%%%%%%%%%%%%%%%%%%%%%%%%%%%%%%%%%%%%%%%%%%%%%%%%%%%%%%%
\section{\'Equation du second degr\'e}

\begin{frame}

\begin{proposition}
  L'\'equation \[az^2 + bz + c = 0\]
avec $a, b, c \in \Cc$ et $a \neq 0$, possède deux solutions $z_1, z_2 \in \Cc$\\

 \mybox{
  $ z_1 = \dfrac{- b + \delta}{2 a} \quad \text{ et } \quad z_2 = \dfrac{- b - \delta}{2a}$}

\pause

o\`u $\delta\in \Cc$ est une racine carr\'ee de $\Delta = b^2 - 4 ac$
\end{proposition}

\medskip

\pause

\begin{itemize}
  \item $z^2 + z + \frac{1 - \ii }{4}=0$
\pause\quad 
 $\Delta = \ii $ \quad\pause  $\delta = \frac{\sqrt2}{2}(1 +  \ii)$ 

\pause
\vspace*{-1ex}
\hfill les solutions sont $z = \dfrac{- 1 \pm \frac{\sqrt2}{2}(1 +  \ii)}{2}$

\pause
  \item Si $a,b,c \in \Rr$ on retrouve les formules déjà connues

\pause
  \item Si $\Delta=0$ alors les solutions $z_1 = z_2 = - \frac{b}{2 a}$ sont confondues

\end{itemize}

\end{frame}


\begin{frame}
 \mybox{
  $ z_1 = \dfrac{- b + \delta}{2 a} \quad \text{ et } \quad z_2 
= \dfrac{- b - \delta}{2a}\quad$ avec  $\quad\delta^2=\Delta = b^2 - 4 ac$}

\pause

\begin{proof}
  On écrit la factorisation
  \begin{eqnarray*}
    az^2 + bz + c & = & a \left( z^2 + \frac{b}{a} z + \frac{c}{a} \right) = a
    \left( \left( z + \frac{b}{2 a} \right)^2 - \frac{b^2}{4 a^2} +
    \frac{c}{a} \right) \\
\pause
    & = & a \left( \left( z + \frac{b}{2 a} \right)^2 -
    \frac{\Delta}{4 a^2} \right)
    = a \left( \left( z + \frac{b}{2 a} \right)^2 - \frac{\delta^2}{4
    a^2} \right) \\
\pause
    & = & a \left( \left( z + \frac{b}{2 a} \right) - \frac{\delta}{2
    a^{}} \right)  \left( \left( z + \frac{b}{2 a} \right) + \frac{\delta}{2
    a^{}} \right)\\
\pause
    & = & a \left( z - \frac{- b + \delta}{2 a} \right)  \left( z - \frac{- b
    - \delta}{2 a} \right) = a \left( z - z_1 \right)  \left( z - z_2 \right)
  \end{eqnarray*}
\end{proof}

\end{frame}
%%%%%%%%%%%%%%%%%%%%%%%%%%%%%%%%%%%%%%%%%%%%%%%%%%%%%%%%%%%%%%%%

\begin{frame}

Soit
$$P(z)=a_n z^n+a_{n-1}z^{n-1}+\cdots + a_2 z^2 + a_1z+a_0$$
un polynôme de degré $n$, avec $a_i \in \Cc$

\bigskip

\begin{theoreme}[(d'Alembert--Gauss)] 
L'équation 
$$P(z)=0$$
possède $n$ solutions $z_1, z_2,\ldots,z_n$ dans $\Cc$
\end{theoreme}

\pause
\bigskip

Autrement dit :
$$P(z) = a_n(z-z_1)(z-z_2)\cdots(z-z_n)$$
\end{frame}


%%%%%%%%%%%%%%%%%%%%%%%%%%%%%%%%%%%%%%%%%%%%%%%%%%%%%%%%%%%%%%%%
\section{Mini-exercices}

\begin{frame}
\begin{miniexercice}
\begin{enumerate}
  \item Calculer les racines carrées de $-\ii$, $3-4\ii$. 
  \item Résoudre les équations : $z^2+z-1=0$, $2z^2 + (-10-10\ii)z+24-10\ii = 0$.
  \item Résoudre l'équation $z^2+(\ii-\sqrt 2)z-\ii\sqrt 2$, puis l'équation $Z^4+(\ii-\sqrt 2)Z^2-\ii\sqrt 2$.
  \item Montrer que si $P(z)=z^2+bz+c$ possède pour racines $z_1, z_2 \in \Cc$ alors 
$z_1+z_2=-b$ et $z_1 \cdot z_2 = c$.
  \item Trouver les paires de nombres dont la somme vaut $\ii$ et le produit $1$.
  \item Soit $P (z) = a_n z^n + a_{n - 1} z^{n - 1} + \cdots  + a_0$ avec $a_i \in \Rr$ pour tout $i$.
Montrer que si $z$ est racine de $P$ alors $\bar z$ aussi.
\end{enumerate}
\end{miniexercice}
\end{frame}



\end{document}