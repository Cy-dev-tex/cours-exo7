
%%%%%%%%%%%%%%%%%% PREAMBULE %%%%%%%%%%%%%%%%%%


\documentclass[12pt]{article}

\usepackage{amsfonts,amsmath,amssymb,amsthm}
\usepackage[utf8]{inputenc}
\usepackage[T1]{fontenc}
\usepackage[francais]{babel}


% packages
\usepackage{amsfonts,amsmath,amssymb,amsthm}
\usepackage[utf8]{inputenc}
\usepackage[T1]{fontenc}
%\usepackage{lmodern}

\usepackage[francais]{babel}
\usepackage{fancybox}
\usepackage{graphicx}

\usepackage{float}

%\usepackage[usenames, x11names]{xcolor}
\usepackage{tikz}
\usepackage{datetime}

\usepackage{mathptmx}
%\usepackage{fouriernc}
%\usepackage{newcent}
\usepackage[mathcal,mathbf]{euler}

%\usepackage{palatino}
%\usepackage{newcent}


% Commande spéciale prompteur

%\usepackage{mathptmx}
%\usepackage[mathcal,mathbf]{euler}
%\usepackage{mathpple,multido}

\usepackage[a4paper]{geometry}
\geometry{top=2cm, bottom=2cm, left=1cm, right=1cm, marginparsep=1cm}

\newcommand{\change}{{\color{red}\rule{\textwidth}{1mm}\\}}

\newcounter{mydiapo}

\newcommand{\diapo}{\newpage
\hfill {\normalsize  Diapo \themydiapo \quad \texttt{[\jobname]}} \\
\stepcounter{mydiapo}}


%%%%%%% COULEURS %%%%%%%%%%

% Pour blanc sur noir :
%\pagecolor[rgb]{0.5,0.5,0.5}
% \pagecolor[rgb]{0,0,0}
% \color[rgb]{1,1,1}



%\DeclareFixedFont{\myfont}{U}{cmss}{bx}{n}{18pt}
\newcommand{\debuttexte}{
%%%%%%%%%%%%% FONTES %%%%%%%%%%%%%
\renewcommand{\baselinestretch}{1.5}
\usefont{U}{cmss}{bx}{n}
\bfseries

% Taille normale : commenter le reste !
%Taille Arnaud
%\fontsize{19}{19}\selectfont

% Taille Barbara
%\fontsize{21}{22}\selectfont

%Taille François
%\fontsize{25}{30}\selectfont

%Taille Pascal
%\fontsize{25}{30}\selectfont

%Taille Laura
%\fontsize{30}{35}\selectfont


%\myfont
%\usefont{U}{cmss}{bx}{n}

%\Huge
%\addtolength{\parskip}{\baselineskip}
}


% \usepackage{hyperref}
% \hypersetup{colorlinks=true, linkcolor=blue, urlcolor=blue,
% pdftitle={Exo7 - Exercices de mathématiques}, pdfauthor={Exo7}}


%section
% \usepackage{sectsty}
% \allsectionsfont{\bf}
%\sectionfont{\color{Tomato3}\upshape\selectfont}
%\subsectionfont{\color{Tomato4}\upshape\selectfont}

%----- Ensembles : entiers, reels, complexes -----
\newcommand{\Nn}{\mathbb{N}} \newcommand{\N}{\mathbb{N}}
\newcommand{\Zz}{\mathbb{Z}} \newcommand{\Z}{\mathbb{Z}}
\newcommand{\Qq}{\mathbb{Q}} \newcommand{\Q}{\mathbb{Q}}
\newcommand{\Rr}{\mathbb{R}} \newcommand{\R}{\mathbb{R}}
\newcommand{\Cc}{\mathbb{C}} 
\newcommand{\Kk}{\mathbb{K}} \newcommand{\K}{\mathbb{K}}

%----- Modifications de symboles -----
\renewcommand{\epsilon}{\varepsilon}
\renewcommand{\Re}{\mathop{\text{Re}}\nolimits}
\renewcommand{\Im}{\mathop{\text{Im}}\nolimits}
%\newcommand{\llbracket}{\left[\kern-0.15em\left[}
%\newcommand{\rrbracket}{\right]\kern-0.15em\right]}

\renewcommand{\ge}{\geqslant}
\renewcommand{\geq}{\geqslant}
\renewcommand{\le}{\leqslant}
\renewcommand{\leq}{\leqslant}

%----- Fonctions usuelles -----
\newcommand{\ch}{\mathop{\mathrm{ch}}\nolimits}
\newcommand{\sh}{\mathop{\mathrm{sh}}\nolimits}
\renewcommand{\tanh}{\mathop{\mathrm{th}}\nolimits}
\newcommand{\cotan}{\mathop{\mathrm{cotan}}\nolimits}
\newcommand{\Arcsin}{\mathop{\mathrm{Arcsin}}\nolimits}
\newcommand{\Arccos}{\mathop{\mathrm{Arccos}}\nolimits}
\newcommand{\Arctan}{\mathop{\mathrm{Arctan}}\nolimits}
\newcommand{\Argsh}{\mathop{\mathrm{Argsh}}\nolimits}
\newcommand{\Argch}{\mathop{\mathrm{Argch}}\nolimits}
\newcommand{\Argth}{\mathop{\mathrm{Argth}}\nolimits}
\newcommand{\pgcd}{\mathop{\mathrm{pgcd}}\nolimits} 

\newcommand{\Card}{\mathop{\text{Card}}\nolimits}
\newcommand{\Ker}{\mathop{\text{Ker}}\nolimits}
\newcommand{\id}{\mathop{\text{id}}\nolimits}
\newcommand{\ii}{\mathrm{i}}
\newcommand{\dd}{\mathrm{d}}
\newcommand{\Vect}{\mathop{\text{Vect}}\nolimits}
\newcommand{\Mat}{\mathop{\mathrm{Mat}}\nolimits}
\newcommand{\rg}{\mathop{\text{rg}}\nolimits}
\newcommand{\tr}{\mathop{\text{tr}}\nolimits}
\newcommand{\ppcm}{\mathop{\text{ppcm}}\nolimits}

%----- Structure des exercices ------

\newtheoremstyle{styleexo}% name
{2ex}% Space above
{3ex}% Space below
{}% Body font
{}% Indent amount 1
{\bfseries} % Theorem head font
{}% Punctuation after theorem head
{\newline}% Space after theorem head 2
{}% Theorem head spec (can be left empty, meaning ‘normal’)

%\theoremstyle{styleexo}
\newtheorem{exo}{Exercice}
\newtheorem{ind}{Indications}
\newtheorem{cor}{Correction}


\newcommand{\exercice}[1]{} \newcommand{\finexercice}{}
%\newcommand{\exercice}[1]{{\tiny\texttt{#1}}\vspace{-2ex}} % pour afficher le numero absolu, l'auteur...
\newcommand{\enonce}{\begin{exo}} \newcommand{\finenonce}{\end{exo}}
\newcommand{\indication}{\begin{ind}} \newcommand{\finindication}{\end{ind}}
\newcommand{\correction}{\begin{cor}} \newcommand{\fincorrection}{\end{cor}}

\newcommand{\noindication}{\stepcounter{ind}}
\newcommand{\nocorrection}{\stepcounter{cor}}

\newcommand{\fiche}[1]{} \newcommand{\finfiche}{}
\newcommand{\titre}[1]{\centerline{\large \bf #1}}
\newcommand{\addcommand}[1]{}
\newcommand{\video}[1]{}

% Marge
\newcommand{\mymargin}[1]{\marginpar{{\small #1}}}



%----- Presentation ------
\setlength{\parindent}{0cm}

%\newcommand{\ExoSept}{\href{http://exo7.emath.fr}{\textbf{\textsf{Exo7}}}}

\definecolor{myred}{rgb}{0.93,0.26,0}
\definecolor{myorange}{rgb}{0.97,0.58,0}
\definecolor{myyellow}{rgb}{1,0.86,0}

\newcommand{\LogoExoSept}[1]{  % input : echelle
{\usefont{U}{cmss}{bx}{n}
\begin{tikzpicture}[scale=0.1*#1,transform shape]
  \fill[color=myorange] (0,0)--(4,0)--(4,-4)--(0,-4)--cycle;
  \fill[color=myred] (0,0)--(0,3)--(-3,3)--(-3,0)--cycle;
  \fill[color=myyellow] (4,0)--(7,4)--(3,7)--(0,3)--cycle;
  \node[scale=5] at (3.5,3.5) {Exo7};
\end{tikzpicture}}
}



\theoremstyle{definition}
%\newtheorem{proposition}{Proposition}
%\newtheorem{exemple}{Exemple}
%\newtheorem{theoreme}{Théorème}
\newtheorem{lemme}{Lemme}
\newtheorem{corollaire}{Corollaire}
%\newtheorem*{remarque*}{Remarque}
%\newtheorem*{miniexercice}{Mini-exercices}
%\newtheorem{definition}{Définition}




%definition d'un terme
\newcommand{\defi}[1]{{\color{myorange}\textbf{\emph{#1}}}}
\newcommand{\evidence}[1]{{\color{blue}\textbf{\emph{#1}}}}



 %----- Commandes divers ------

\newcommand{\codeinline}[1]{\texttt{#1}}

%%%%%%%%%%%%%%%%%%%%%%%%%%%%%%%%%%%%%%%%%%%%%%%%%%%%%%%%%%%%%
%%%%%%%%%%%%%%%%%%%%%%%%%%%%%%%%%%%%%%%%%%%%%%%%%%%%%%%%%%%%%



\begin{document}

\debuttexte


%%%%%%%%%%%%%%%%%%%%%%%%%%%%%%%%%%%%%%%%%%%%%%%%%%%%%%%%%%%
\diapo

\change

Nous allons utiliser les nombres complexes pour des problèmes géométriques :

\change

Nous allons d'abord calculer l'équation complexe d'une droite

\change

puis l'équation complexe d'un cercle

\change

cela nous permettra de trouver les lignes de niveau
du type $MA/MB=constante$


%%%%%%%%%%%%%%%%%%%%%%%%%%%%%%%%%%%%%%%%%%%%%%%%%%%%%%%%%%%
\diapo

Nous identifions le plan $\Rr^2$ à l'ensemble des nombres complexes

A un point de coordonnées $(x,y)$ on associe le complexe $z=x+\ii y$

Et on dit que $z$ est l'affixe du point de coordonnées $(x,y)$

\change

\begin{itemize}
\item L'origine $O$ a pour affixe $0$

\change

\item Le point de coordonnées $(1,0)$ a pour affixe $1$

\change

\item Le point de coordonnées $(0,1)$ a pour affixe $\ii$

\change

\item Le point de coordonnées $\big(\rho\cos\theta,\rho\sin\theta\big )$ a pour affixe $\rho e^{\ii \theta}$
\end{itemize}


%%%%%%%%%%%%%%%%%%%%%%%%%%%%%%%%%%%%%%%%%%%%%%%%%%%%%%%%%%%
\diapo

Nous allons calculer l'équation complexe d'une droite

Soit $D$ la droite d'équation réelle $ax+by=c$

\change

Alors l'équation complexe de $D$ est 

$\bar \omega z + \omega \bar z = k$

d'inconnue $z\in \Cc$

avec $\omega$ le complexe non nul $a+\ii b$ et le réel $k=2c$.

C'est-à-dire que l'ensemble des points d'affixe $z$ qui vérifient cette
équation est la droite $D$

\change

Pour la démonstration on écrit $z=x+\ii y$

\change


et donc $x = \frac{z+\bar z}{2}$ et $y = \frac{z - \bar z}{2 \ii }$

\change

$ax+by=c$ équivaut à $a(z+\bar z) -\ii  b(z-\bar z)=2c$

\change


qui équivaut à $(a-\ii  b)z+(a+\ii  b)\bar z = 2c$

comme nous l'avions annoncé.

%%%%%%%%%%%%%%%%%%%%%%%%%%%%%%%%%%%%%%%%%%%%%%%%%%%%%%%%%%%
\diapo


Nous allons aussi calculer l'équation complexe d'un cercle

Soit $C$ le cercle de centre grand $\Omega$ et de rayon $r$

Notons petit $\omega$ l'affixe du point grand $\Omega$

\change

Le point $M$ appartient au cercle $C$
si et seulement si la distance de $\Omega$ à $M$ égale $r$

\change

Si l'on note $z$ l'affixe de $M$ alors cela équivaut à

$|z-\omega|=r$

on élève au carré pour obtenir $|z-\omega|^2=r^2$

\change

on remplace $|z-\omega|^2$ par $(z-\omega)\overline{(z+\omega)}$

\change

et on développe pour obtenir l'équation du cercle

\change

En conclusion l'équation du cercle centré en un point d'affixe
petit $\omega$ et de rayon $r$ est 

$z\bar z - \bar \omega z - \omega \bar z = r^2-|\omega|^2$

%%%%%%%%%%%%%%%%%%%%%%%%%%%%%%%%%%%%%%%%%%%%%%%%%%%%%%%%%%%
\diapo


Fixons deux point $A$ et $B$ du plan

Nous allons déterminer tous les points $M$ tels que la distance
$MA$ et la distance $MB$ vérifient $MA/MB=k$
où $k$ est un réel positif fixé

\change

Nous allons noter petit $a$ l'affixe de $A$, petit $b$ l'affixe de $B$
et petit $z$ l'affixe de $M$

$\dfrac{MA}{MB}=k$

\change

équivaut à 

$\dfrac{|z-a|}{|z-b|}=k$

\change

ou encore $|z-a|^2 = k^2 |z-b|^2$

\change 

ce qui équivaut à 

$(z-a)\overline{(z-a)} = k^2 (z-b)\overline{(z-b)}$

\change

En développant on obtient cette équation.


\change

Si $k=1$, le terme en $z\bar z$ disparaît et si on pose $\omega = a-b$

l'équation devient 
$z\bar \omega  + \bar z \omega = |a|^2-|b|^2$ 


\change

C'est l'équation d'une droite, qui est en fait la médiatrice de $[AB]$

\change

Si $k\neq 1$ on pose $\omega = \frac{a-k^2b}{1-k^2}$ 

\change

et en divisant par $1-k^2$ on obtient l'équation
d'un cercle de centre $\omega$ 

\change

dont on peut calculer le rayon.

\change

Il faut donc retenir :

si $k=1$, c'est une droite, la médiatrice de $[AB]$, 

\change

sinon c'est un cercle et on refait les calculs de la démonstration pour trouver son équation.

%%%%%%%%%%%%%%%%%%%%%%%%%%%%%%%%%%%%%%%%%%%%%%%%%%%%%%%%%%%
\diapo

Voici ce que cela donne avec l'exemple de $A$ point d'affixe $+1$
et $B$ point d'affixe $-1$

\change

Pour $k=3$ l'ensemble des points $MA/MB=k$ est un cercle

\change

Pour $k=2$ on trouve également un cercle

\change

par exemple pour le point $M$ représenté ici, la distance $MA$
est bien le double de la distance $MB$

\change

Si on diminue encore la valeur de $k$ alors le cercle s'agrandit

\change

Jusqu'à obtenir pour $k=1$ un droite :

ce sont les points qui vérifie $MA=MB$

c'est donc la médiatrice de $[AB]$

\change

Pour les valeurs de $k$ inférieures à $1$ on retrouve des cercles

\change

de plus en plus petits 

\change

\medskip

\change

En résumé voici sur un même dessin ces quelques lignes de niveau





%%%%%%%%%%%%%%%%%%%%%%%%%%%%%%%%%%%%%%%%%%%%%%%%%%%%%%%%%%%
\diapo


Pour finir, exercez-vous avec ces quelques questions.


\end{document}