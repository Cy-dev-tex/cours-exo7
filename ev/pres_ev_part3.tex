
%%%%%%%%%%%%%%%%%% PREAMBULE %%%%%%%%%%%%%%%%%%

\documentclass[aspectratio=169,utf8]{beamer}
%\documentclass[aspectratio=169,handout]{beamer}

\usetheme{Boadilla}
%\usecolortheme{seahorse}
\usecolortheme[RGB={245,66,24}]{structure}
\useoutertheme{infolines}

% packages
\usepackage{amsfonts,amsmath,amssymb,amsthm}
\usepackage[utf8]{inputenc}
\usepackage[T1]{fontenc}
\usepackage{lmodern}

\usepackage[francais]{babel}
\usepackage{fancybox}
\usepackage{graphicx}

\usepackage{float}
\usepackage{xfrac}

%\usepackage[usenames, x11names]{xcolor}
\usepackage{tikz}
\usepackage{pgfplots}
\usepackage{datetime}



%-----  Package unités -----
\usepackage{siunitx}
\sisetup{locale = FR,detect-all,per-mode = symbol}

%\usepackage{mathptmx}
%\usepackage{fouriernc}
%\usepackage{newcent}
%\usepackage[mathcal,mathbf]{euler}

%\usepackage{palatino}
%\usepackage{newcent}
% \usepackage[mathcal,mathbf]{euler}



% \usepackage{hyperref}
% \hypersetup{colorlinks=true, linkcolor=blue, urlcolor=blue,
% pdftitle={Exo7 - Exercices de mathématiques}, pdfauthor={Exo7}}


%section
% \usepackage{sectsty}
% \allsectionsfont{\bf}
%\sectionfont{\color{Tomato3}\upshape\selectfont}
%\subsectionfont{\color{Tomato4}\upshape\selectfont}

%----- Ensembles : entiers, reels, complexes -----
\newcommand{\Nn}{\mathbb{N}} \newcommand{\N}{\mathbb{N}}
\newcommand{\Zz}{\mathbb{Z}} \newcommand{\Z}{\mathbb{Z}}
\newcommand{\Qq}{\mathbb{Q}} \newcommand{\Q}{\mathbb{Q}}
\newcommand{\Rr}{\mathbb{R}} \newcommand{\R}{\mathbb{R}}
\newcommand{\Cc}{\mathbb{C}} 
\newcommand{\Kk}{\mathbb{K}} \newcommand{\K}{\mathbb{K}}

%----- Modifications de symboles -----
\renewcommand{\epsilon}{\varepsilon}
\renewcommand{\Re}{\mathop{\text{Re}}\nolimits}
\renewcommand{\Im}{\mathop{\text{Im}}\nolimits}
%\newcommand{\llbracket}{\left[\kern-0.15em\left[}
%\newcommand{\rrbracket}{\right]\kern-0.15em\right]}

\renewcommand{\ge}{\geqslant}
\renewcommand{\geq}{\geqslant}
\renewcommand{\le}{\leqslant}
\renewcommand{\leq}{\leqslant}
\renewcommand{\epsilon}{\varepsilon}

%----- Fonctions usuelles -----
\newcommand{\ch}{\mathop{\text{ch}}\nolimits}
\newcommand{\sh}{\mathop{\text{sh}}\nolimits}
\renewcommand{\tanh}{\mathop{\text{th}}\nolimits}
\newcommand{\cotan}{\mathop{\text{cotan}}\nolimits}
\newcommand{\Arcsin}{\mathop{\text{arcsin}}\nolimits}
\newcommand{\Arccos}{\mathop{\text{arccos}}\nolimits}
\newcommand{\Arctan}{\mathop{\text{arctan}}\nolimits}
\newcommand{\Argsh}{\mathop{\text{argsh}}\nolimits}
\newcommand{\Argch}{\mathop{\text{argch}}\nolimits}
\newcommand{\Argth}{\mathop{\text{argth}}\nolimits}
\newcommand{\pgcd}{\mathop{\text{pgcd}}\nolimits} 


%----- Commandes divers ------
\newcommand{\ii}{\mathrm{i}}
\newcommand{\dd}{\text{d}}
\newcommand{\id}{\mathop{\text{id}}\nolimits}
\newcommand{\Ker}{\mathop{\text{Ker}}\nolimits}
\newcommand{\Card}{\mathop{\text{Card}}\nolimits}
\newcommand{\Vect}{\mathop{\text{Vect}}\nolimits}
\newcommand{\Mat}{\mathop{\text{Mat}}\nolimits}
\newcommand{\rg}{\mathop{\text{rg}}\nolimits}
\newcommand{\tr}{\mathop{\text{tr}}\nolimits}


%----- Structure des exercices ------

\newtheoremstyle{styleexo}% name
{2ex}% Space above
{3ex}% Space below
{}% Body font
{}% Indent amount 1
{\bfseries} % Theorem head font
{}% Punctuation after theorem head
{\newline}% Space after theorem head 2
{}% Theorem head spec (can be left empty, meaning ‘normal’)

%\theoremstyle{styleexo}
\newtheorem{exo}{Exercice}
\newtheorem{ind}{Indications}
\newtheorem{cor}{Correction}


\newcommand{\exercice}[1]{} \newcommand{\finexercice}{}
%\newcommand{\exercice}[1]{{\tiny\texttt{#1}}\vspace{-2ex}} % pour afficher le numero absolu, l'auteur...
\newcommand{\enonce}{\begin{exo}} \newcommand{\finenonce}{\end{exo}}
\newcommand{\indication}{\begin{ind}} \newcommand{\finindication}{\end{ind}}
\newcommand{\correction}{\begin{cor}} \newcommand{\fincorrection}{\end{cor}}

\newcommand{\noindication}{\stepcounter{ind}}
\newcommand{\nocorrection}{\stepcounter{cor}}

\newcommand{\fiche}[1]{} \newcommand{\finfiche}{}
\newcommand{\titre}[1]{\centerline{\large \bf #1}}
\newcommand{\addcommand}[1]{}
\newcommand{\video}[1]{}

% Marge
\newcommand{\mymargin}[1]{\marginpar{{\small #1}}}

\def\noqed{\renewcommand{\qedsymbol}{}}


%----- Presentation ------
\setlength{\parindent}{0cm}

%\newcommand{\ExoSept}{\href{http://exo7.emath.fr}{\textbf{\textsf{Exo7}}}}

\definecolor{myred}{rgb}{0.93,0.26,0}
\definecolor{myorange}{rgb}{0.97,0.58,0}
\definecolor{myyellow}{rgb}{1,0.86,0}

\newcommand{\LogoExoSept}[1]{  % input : echelle
{\usefont{U}{cmss}{bx}{n}
\begin{tikzpicture}[scale=0.1*#1,transform shape]
  \fill[color=myorange] (0,0)--(4,0)--(4,-4)--(0,-4)--cycle;
  \fill[color=myred] (0,0)--(0,3)--(-3,3)--(-3,0)--cycle;
  \fill[color=myyellow] (4,0)--(7,4)--(3,7)--(0,3)--cycle;
  \node[scale=5] at (3.5,3.5) {Exo7};
\end{tikzpicture}}
}


\newcommand{\debutmontitre}{
  \author{} \date{} 
  \thispagestyle{empty}
  \hspace*{-10ex}
  \begin{minipage}{\textwidth}
    \titlepage  
  \vspace*{-2.5cm}
  \begin{center}
    \LogoExoSept{2.5}
  \end{center}
  \end{minipage}

  \vspace*{-0cm}
  
  % Astuce pour que le background ne soit pas discrétisé lors de la conversion pdf -> png
\begin{tikzpicture}
        \fill[opacity=0,green!60!black] (0,0)--++(0,0)--++(0,0)--++(0,0)--cycle; 
\end{tikzpicture}

% toc S'affiche trop tot :
% \tableofcontents[hideallsubsections, pausesections]
}

\newcommand{\finmontitre}{
  \end{frame}
  \setcounter{framenumber}{0}
} % ne marche pas pour une raison obscure

%----- Commandes supplementaires ------

% \usepackage[landscape]{geometry}
% \geometry{top=1cm, bottom=3cm, left=2cm, right=10cm, marginparsep=1cm
% }
% \usepackage[a4paper]{geometry}
% \geometry{top=2cm, bottom=2cm, left=2cm, right=2cm, marginparsep=1cm
% }

%\usepackage{standalone}


% New command Arnaud -- november 2011
\setbeamersize{text margin left=24ex}
% si vous modifier cette valeur il faut aussi
% modifier le decalage du titre pour compenser
% (ex : ici =+10ex, titre =-5ex

\theoremstyle{definition}
%\newtheorem{proposition}{Proposition}
%\newtheorem{exemple}{Exemple}
%\newtheorem{theoreme}{Théorème}
%\newtheorem{lemme}{Lemme}
%\newtheorem{corollaire}{Corollaire}
%\newtheorem*{remarque*}{Remarque}
%\newtheorem*{miniexercice}{Mini-exercices}
%\newtheorem{definition}{Définition}

% Commande tikz
\usetikzlibrary{calc}
\usetikzlibrary{patterns,arrows}
\usetikzlibrary{matrix}
\usetikzlibrary{fadings} 

%definition d'un terme
\newcommand{\defi}[1]{{\color{myorange}\textbf{\emph{#1}}}}
\newcommand{\evidence}[1]{{\color{blue}\textbf{\emph{#1}}}}
\newcommand{\assertion}[1]{\emph{\og#1\fg}}  % pour chapitre logique
%\renewcommand{\contentsname}{Sommaire}
\renewcommand{\contentsname}{}
\setcounter{tocdepth}{2}



%------ Figures ------

\def\myscale{1} % par défaut 
\newcommand{\myfigure}[2]{  % entrée : echelle, fichier figure
\def\myscale{#1}
\begin{center}
\footnotesize
{#2}
\end{center}}


%------ Encadrement ------

\usepackage{fancybox}


\newcommand{\mybox}[1]{
\setlength{\fboxsep}{7pt}
\begin{center}
\shadowbox{#1}
\end{center}}

\newcommand{\myboxinline}[1]{
\setlength{\fboxsep}{5pt}
\raisebox{-10pt}{
\shadowbox{#1}
}
}

%--------------- Commande beamer---------------
\newcommand{\beameronly}[1]{#1} % permet de mettre des pause dans beamer pas dans poly


\setbeamertemplate{navigation symbols}{}
\setbeamertemplate{footline}  % tiré du fichier beamerouterinfolines.sty
{
  \leavevmode%
  \hbox{%
  \begin{beamercolorbox}[wd=.333333\paperwidth,ht=2.25ex,dp=1ex,center]{author in head/foot}%
    % \usebeamerfont{author in head/foot}\insertshortauthor%~~(\insertshortinstitute)
    \usebeamerfont{section in head/foot}{\bf\insertshorttitle}
  \end{beamercolorbox}%
  \begin{beamercolorbox}[wd=.333333\paperwidth,ht=2.25ex,dp=1ex,center]{title in head/foot}%
    \usebeamerfont{section in head/foot}{\bf\insertsectionhead}
  \end{beamercolorbox}%
  \begin{beamercolorbox}[wd=.333333\paperwidth,ht=2.25ex,dp=1ex,right]{date in head/foot}%
    % \usebeamerfont{date in head/foot}\insertshortdate{}\hspace*{2em}
    \insertframenumber{} / \inserttotalframenumber\hspace*{2ex} 
  \end{beamercolorbox}}%
  \vskip0pt%
}


\definecolor{mygrey}{rgb}{0.5,0.5,0.5}
\setlength{\parindent}{0cm}
%\DeclareTextFontCommand{\helvetica}{\fontfamily{phv}\selectfont}

% background beamer
\definecolor{couleurhaut}{rgb}{0.85,0.9,1}  % creme
\definecolor{couleurmilieu}{rgb}{1,1,1}  % vert pale
\definecolor{couleurbas}{rgb}{0.85,0.9,1}  % blanc
\setbeamertemplate{background canvas}[vertical shading]%
[top=couleurhaut,middle=couleurmilieu,midpoint=0.4,bottom=couleurbas] 
%[top=fondtitre!05,bottom=fondtitre!60]



\makeatletter
\setbeamertemplate{theorem begin}
{%
  \begin{\inserttheoremblockenv}
  {%
    \inserttheoremheadfont
    \inserttheoremname
    \inserttheoremnumber
    \ifx\inserttheoremaddition\@empty\else\ (\inserttheoremaddition)\fi%
    \inserttheorempunctuation
  }%
}
\setbeamertemplate{theorem end}{\end{\inserttheoremblockenv}}

\newenvironment{theoreme}[1][]{%
   \setbeamercolor{block title}{fg=structure,bg=structure!40}
   \setbeamercolor{block body}{fg=black,bg=structure!10}
   \begin{block}{{\bf Th\'eor\`eme }#1}
}{%
   \end{block}%
}


\newenvironment{proposition}[1][]{%
   \setbeamercolor{block title}{fg=structure,bg=structure!40}
   \setbeamercolor{block body}{fg=black,bg=structure!10}
   \begin{block}{{\bf Proposition }#1}
}{%
   \end{block}%
}

\newenvironment{corollaire}[1][]{%
   \setbeamercolor{block title}{fg=structure,bg=structure!40}
   \setbeamercolor{block body}{fg=black,bg=structure!10}
   \begin{block}{{\bf Corollaire }#1}
}{%
   \end{block}%
}

\newenvironment{mydefinition}[1][]{%
   \setbeamercolor{block title}{fg=structure,bg=structure!40}
   \setbeamercolor{block body}{fg=black,bg=structure!10}
   \begin{block}{{\bf Définition} #1}
}{%
   \end{block}%
}

\newenvironment{lemme}[0]{%
   \setbeamercolor{block title}{fg=structure,bg=structure!40}
   \setbeamercolor{block body}{fg=black,bg=structure!10}
   \begin{block}{\bf Lemme}
}{%
   \end{block}%
}

\newenvironment{remarque}[1][]{%
   \setbeamercolor{block title}{fg=black,bg=structure!20}
   \setbeamercolor{block body}{fg=black,bg=structure!5}
   \begin{block}{Remarque #1}
}{%
   \end{block}%
}


\newenvironment{exemple}[1][]{%
   \setbeamercolor{block title}{fg=black,bg=structure!20}
   \setbeamercolor{block body}{fg=black,bg=structure!5}
   \begin{block}{{\bf Exemple }#1}
}{%
   \end{block}%
}


\newenvironment{miniexercice}[0]{%
   \setbeamercolor{block title}{fg=structure,bg=structure!20}
   \setbeamercolor{block body}{fg=black,bg=structure!5}
   \begin{block}{Mini-exercices}
}{%
   \end{block}%
}


\newenvironment{tp}[0]{%
   \setbeamercolor{block title}{fg=structure,bg=structure!40}
   \setbeamercolor{block body}{fg=black,bg=structure!10}
   \begin{block}{\bf Travaux pratiques}
}{%
   \end{block}%
}
\newenvironment{exercicecours}[1][]{%
   \setbeamercolor{block title}{fg=structure,bg=structure!40}
   \setbeamercolor{block body}{fg=black,bg=structure!10}
   \begin{block}{{\bf Exercice }#1}
}{%
   \end{block}%
}
\newenvironment{algo}[1][]{%
   \setbeamercolor{block title}{fg=structure,bg=structure!40}
   \setbeamercolor{block body}{fg=black,bg=structure!10}
   \begin{block}{{\bf Algorithme}\hfill{\color{gray}\texttt{#1}}}
}{%
   \end{block}%
}


\setbeamertemplate{proof begin}{
   \setbeamercolor{block title}{fg=black,bg=structure!20}
   \setbeamercolor{block body}{fg=black,bg=structure!5}
   \begin{block}{{\footnotesize Démonstration}}
   \footnotesize
   \smallskip}
\setbeamertemplate{proof end}{%
   \end{block}}
\setbeamertemplate{qed symbol}{\openbox}


\makeatother
\usecolortheme[RGB={205,0,0}]{structure}

%%%%%%%%%%%%%%%%%%%%%%%%%%%%%%%%%%%%%%%%%%%%%%%%%%%%%%%%%%%%%
%%%%%%%%%%%%%%%%%%%%%%%%%%%%%%%%%%%%%%%%%%%%%%%%%%%%%%%%%%%%%


\begin{document}


\title{{\bf Espaces vectoriels}}
\subtitle{Sous-espace vectoriel (début)}

\begin{frame}
  
  \debutmontitre

  \pause

{\footnotesize
\hfill
\setbeamercovered{transparent=50}
\begin{minipage}{0.6\textwidth}
  \begin{itemize}
    \item<3-> Définition d'un sous-espace vectoriel
    \item<4-> Un sous-espace vectoriel \\ est un espace vectoriel  
  \end{itemize}
\end{minipage}
}

\end{frame}

\setcounter{framenumber}{0}


%%%%%%%%%%%%%%%%%%%%%%%%%%%%%%%%%%%%%%%%%%%%%%%%%%%%%%%%%%%%%%%%
\section{Définition d'un sous-espace vectoriel}

\begin{frame}
Soit $E$ un $\Kk$-espace vectoriel
\begin{mydefinition} 
Une partie  $F$ de $E$ est appelée un \defi{sous-espace vectoriel} si  
 \begin{itemize}
   \item $0_E \in F$
   
   \item $u+v \in F$ \  pour tous $u,v \in F$
   
   \item $\lambda \cdot u \in F$ pour tout $\lambda \in \Kk$ et tout $u \in F$ 
 \end{itemize}
\end{mydefinition}

\end{frame}

\begin{frame}
\begin{exemple}
$F=\big\{(x,y)\in \Rr^2\mid x+y=0\big\}$  est un sous-espace 
  vectoriel de $\Rr^2$
  
\hfill\hfill\begin{minipage}{0.3\textwidth}
\myfigure{0.7}{
\tikzinput{fig_ev03} 
}   
\end{minipage}
\vspace*{-13ex}
\pause
  \begin{enumerate}
    \item $(0,0) \in F$

    \pause
    
    \item 
    \begin{itemize}
      \item si $u=(x_1,y_1)$ et $v=(x_2,y_2)$ appartiennent à $F$
      \pause
      \item alors $x_1+y_1=0$ et $x_2+y_2=0$
      \pause
      \item donc $(x_1+x_2)+(y_1+y_2)=0$
      \pause
      \item et ainsi $u+v=(x_1+x_2,y_1+y_2)$ appartient à $F$
    \end{itemize}

       
    
    \pause
    
    \item
    \begin{itemize}
      \item si $u=(x,y) \in F$ et $\lambda \in \Rr$
      \item alors $x+y=0$ donc $\lambda x + \lambda y = 0$
      \item d'où $\lambda u \in F$
    \end{itemize}

  \end{enumerate}

  

\end{exemple}
\end{frame}


\begin{frame}
\begin{exemple}
\begin{itemize}
  \item L'ensemble des fonctions continues sur $\Rr$ est un sous-espace vectoriel 
  de l'espace vectoriel des fonctions de $\Rr$ dans $\Rr$
  

  \pause
  
  \begin{itemize}
    \item la fonction nulle est continue
    \pause
    
    \item la somme de deux fonctions continues est continue
    \pause
    
    \item une constante fois une fonction continue est une fonction continue 
  \end{itemize}    

  \pause
  
  \item L'ensemble des suites réelles convergentes est un 
  sous-espace vectoriel de l'espace vectoriel des suites réelles
  
\end{itemize}
\end{exemple}
\end{frame}


\begin{frame}
Voici des sous-ensembles qui \evidence{ne sont pas} des sous-espaces vectoriels

\pause
\begin{enumerate}

\uncover<2->{  \item $F_1=\big\{(x,y)\in \Rr^2\mid x+y=2\big\}$ n'est pas un s.e.v. de $\Rr^2$}

\uncover<3->{  En effet le vecteur nul $(0,0)$ n'appartient pas à $F_1$}
  
\uncover<4->{   \item $F_2=\big\{(x,y)\in \Rr^2\mid x=0 \text{ ou } y=0 \big\}$ n'est pas un s.e.v. de $\Rr^2$}
  
\uncover<5->{  En effet $u=(1,0),v=(0,1) \in F_2$, mais $u+v=(1,1) \notin F_2$} 
  
\uncover<6->{   \item $F_3=\big\{(x,y)\in \Rr^2\mid x \ge 0 \text{ et } y\ge 0\big\}$ n'est pas un s.e.v. de $\Rr^2$}   
  
\uncover<7->{  En effet $u=(1,1) \in F_3$ mais, pour $\lambda = -1$,  $-u = (-1,-1) \notin F_3$}
\end{enumerate}  

\myfigure{0.9}{
\uncover<2->{\tikzinput{fig_ev04-1}} 
\quad
\uncover<4->{\tikzinput{fig_ev04-2}} 
\quad
\uncover<6->{\tikzinput{fig_ev04-3}} 
}
\end{frame}


%%%%%%%%%%%%%%%%%%%%%%%%%%%%%%%%%%%%%%%%%%%%%%%%%%%%%%%%%%%%%%%%
\section{Un sous-espace vectoriel est un espace vectoriel}

\begin{frame}

\begin{theoreme}
\label{th:sevisev}
Soient $E$ un $\Kk$-espace vectoriel et $F$ un sous-espace vectoriel de $E$. 
Alors $F$ est lui-même un $\Kk$-espace vectoriel pour les lois
induites par $E$ 
\end{theoreme}

\pause

\bigskip
\bigskip

\textbf{Méthodologie} 
\hfil \og L'ensemble $F$ est-il un espace vectoriel ? \fg
\begin{itemize}
  \item trouver un espace vectoriel $E$ qui contient $F$
  \item prouver que $F$ est un sous-espace vectoriel de $E$
\end{itemize}

\end{frame}

\begin{frame}
\begin{exemple}
\begin{enumerate}
  \item Est-ce que l'ensemble $\mathcal{P}$ des fonctions paires  forme un espace vectoriel ?

  \pause
  
  \begin{itemize}
    \item $\mathcal{P}$ sous-ensemble de l'espace vectoriel $\mathcal{F}(\Rr,\Rr)$
  \pause
    \item $\mathcal{P}=\big\{ f \in \mathcal{F}(\Rr, \Rr ) \mid \forall x \in \Rr , f(-x)=f(x) \big\}$
  \end{itemize}
\pause
 $\mathcal{P}$ est un sous-espace vectoriel de $\mathcal{F}(\Rr,\Rr)$
    \pause 
  \begin{enumerate}
    \item la fonction nulle est une fonction paire 
    \pause
    \item si $f,g \in \mathcal{P}$ alors $f+g \in\mathcal{P}$
    \pause
    \item si $f\in\mathcal{P}$ et si $\lambda \in \Rr$ alors $\lambda f\in\mathcal{P}$
  \end{enumerate} 
  \pause
  Par le théorème $\mathcal{P}$ est un espace vectoriel
\pause  
  \item L'ensemble des fonctions impaires \\
  \hfil $\mathcal{I}=\big\{ f \in \mathcal{F}(\Rr, \Rr ) \mid \forall x \in \Rr , f(-x)=-f(x) \big\}$ \\
  est un espace vectoriel
\pause  
  \item L'ensemble $\mathcal{S}_n$ des matrices symétriques est un espace vectoriel
  
\end{enumerate}
\end{exemple} 
\end{frame}


\begin{frame}

\begin{itemize}
  \item $A \in M_{n,p}(\Rr)$
  \item $AX = 0$ un système de $n$ équations à $p$ inconnues
  \item $  \left(
\begin{array}{lcl}
a_{11} & \dots & a_{1p}\\
\vdots &&\vdots\\
a_{n1} & \dots & a_{np}
\end{array}\right)
  \left(
\begin{array}{c}
x_1 \\ \vdots\\x_p
\end{array}\right) =   \left(
\begin{array}{c}
0\\ \vdots \\ 0
\end{array}\right)
$
\end{itemize}

\pause

\begin{theoreme}
\label{th:axsev}
L'ensemble des vecteurs $X$ solutions de $AX=0$ est un sous-espace vectoriel de $\Rr^p$
\end{theoreme}

\pause

\begin{proof}
Soit $F$ l'ensemble des vecteurs $X \in \Rr^p$ solutions de l'équation $AX=0$
\begin{itemize}
  \item $0 \in F$
  \item $F$ est stable par addition : si $AX = 0$ et $AX' = 0$ alors $A(X + X') = AX + AX' = 0$
  \item  $F$ est stable par multiplication par un scalaire : si $AX=0$ alors $A (\lambda X) = 0$ \qedhere
\end{itemize}
\end{proof}
 

\end{frame}


\begin{frame}
\begin{exemple}
 $$\left(
\begin{array}{crc}
1 & -2 & 3\\ 2 & -4 & 6\\ 3 & -6 & 9
\end{array}\right) 
\left(\begin{array}{c}
x \\ y\\ z
\end{array}\right) = 
\left(\begin{array}{c}
0\\ 0\\0
\end{array}\right)$$

\pause

\begin{itemize}
  \item Solutions de ce système \\ \hfil $F = \big\{ (x =  2s - 3t, y  =  s, z  =  t) \mid s,t \in \Rr \big\}$
  \pause  
  \item $F$ est un sous-espace vectoriel de $\Rr^3$
  \pause
  \item $F$ est un espace vectoriel
  \pause
  \item Autre vision : $F$ d'équation $(x = 2y - 3z)$, c'est un plan passant par l'origine
\end{itemize}
\end{exemple} 
\end{frame}





%%%%%%%%%%%%%%%%%%%%%%%%%%%%%%%%%%%%%%%%%%%%%%%%%%%%%%%%%%%%%%%%
\section{Mini-exercices}

\begin{frame}

\begin{miniexercice}
Parmi les ensembles suivants, reconnaître ceux qui sont des sous-espaces
vectoriels :
  \begin{enumerate}
    \item $\big\{(x,y,z)\in \Rr^3 \mid  x+y=0\big\}$
    \item $\big\{(x,y,z,t)\in \Rr^4 \mid  x=t \text{ et } y=z\big\}$
    \item $\big\{(x,y,z)\in \Rr^3 \mid z=1\big\}$
    \item $\big\{(x,y)\in \Rr^2 \mid x^2+xy\ge 0\big\}$
    \item $\big\{(x,y)\in \Rr^2 \mid x^2+y^2 \ge 1\big\}$
    \item $\big\{f \in \mathcal{F}(\Rr,\Rr) \mid f(0)=1\big\}$
    \item $\big\{f \in \mathcal{F}(\Rr,\Rr) \mid f(1)=0\big\}$
    \item $\big\{f \in \mathcal{F}(\Rr,\Rr) \mid f \text{ est croissante }\big\}$  
    \item $\big\{ (u_n)_{n\in\Nn} \mid (u_n) \text{ tend vers } 0 \big\}$  
  \end{enumerate}

\end{miniexercice}

\end{frame}

\end{document}