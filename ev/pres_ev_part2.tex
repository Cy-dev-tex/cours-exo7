
%%%%%%%%%%%%%%%%%% PREAMBULE %%%%%%%%%%%%%%%%%%

\documentclass[aspectratio=169,utf8]{beamer}
%\documentclass[aspectratio=169,handout]{beamer}

\usetheme{Boadilla}
%\usecolortheme{seahorse}
\usecolortheme[RGB={245,66,24}]{structure}
\useoutertheme{infolines}

% packages
\usepackage{amsfonts,amsmath,amssymb,amsthm}
\usepackage[utf8]{inputenc}
\usepackage[T1]{fontenc}
\usepackage{lmodern}

\usepackage[francais]{babel}
\usepackage{fancybox}
\usepackage{graphicx}

\usepackage{float}
\usepackage{xfrac}

%\usepackage[usenames, x11names]{xcolor}
\usepackage{tikz}
\usepackage{pgfplots}
\usepackage{datetime}



%-----  Package unités -----
\usepackage{siunitx}
\sisetup{locale = FR,detect-all,per-mode = symbol}

%\usepackage{mathptmx}
%\usepackage{fouriernc}
%\usepackage{newcent}
%\usepackage[mathcal,mathbf]{euler}

%\usepackage{palatino}
%\usepackage{newcent}
% \usepackage[mathcal,mathbf]{euler}



% \usepackage{hyperref}
% \hypersetup{colorlinks=true, linkcolor=blue, urlcolor=blue,
% pdftitle={Exo7 - Exercices de mathématiques}, pdfauthor={Exo7}}


%section
% \usepackage{sectsty}
% \allsectionsfont{\bf}
%\sectionfont{\color{Tomato3}\upshape\selectfont}
%\subsectionfont{\color{Tomato4}\upshape\selectfont}

%----- Ensembles : entiers, reels, complexes -----
\newcommand{\Nn}{\mathbb{N}} \newcommand{\N}{\mathbb{N}}
\newcommand{\Zz}{\mathbb{Z}} \newcommand{\Z}{\mathbb{Z}}
\newcommand{\Qq}{\mathbb{Q}} \newcommand{\Q}{\mathbb{Q}}
\newcommand{\Rr}{\mathbb{R}} \newcommand{\R}{\mathbb{R}}
\newcommand{\Cc}{\mathbb{C}} 
\newcommand{\Kk}{\mathbb{K}} \newcommand{\K}{\mathbb{K}}

%----- Modifications de symboles -----
\renewcommand{\epsilon}{\varepsilon}
\renewcommand{\Re}{\mathop{\text{Re}}\nolimits}
\renewcommand{\Im}{\mathop{\text{Im}}\nolimits}
%\newcommand{\llbracket}{\left[\kern-0.15em\left[}
%\newcommand{\rrbracket}{\right]\kern-0.15em\right]}

\renewcommand{\ge}{\geqslant}
\renewcommand{\geq}{\geqslant}
\renewcommand{\le}{\leqslant}
\renewcommand{\leq}{\leqslant}
\renewcommand{\epsilon}{\varepsilon}

%----- Fonctions usuelles -----
\newcommand{\ch}{\mathop{\text{ch}}\nolimits}
\newcommand{\sh}{\mathop{\text{sh}}\nolimits}
\renewcommand{\tanh}{\mathop{\text{th}}\nolimits}
\newcommand{\cotan}{\mathop{\text{cotan}}\nolimits}
\newcommand{\Arcsin}{\mathop{\text{arcsin}}\nolimits}
\newcommand{\Arccos}{\mathop{\text{arccos}}\nolimits}
\newcommand{\Arctan}{\mathop{\text{arctan}}\nolimits}
\newcommand{\Argsh}{\mathop{\text{argsh}}\nolimits}
\newcommand{\Argch}{\mathop{\text{argch}}\nolimits}
\newcommand{\Argth}{\mathop{\text{argth}}\nolimits}
\newcommand{\pgcd}{\mathop{\text{pgcd}}\nolimits} 


%----- Commandes divers ------
\newcommand{\ii}{\mathrm{i}}
\newcommand{\dd}{\text{d}}
\newcommand{\id}{\mathop{\text{id}}\nolimits}
\newcommand{\Ker}{\mathop{\text{Ker}}\nolimits}
\newcommand{\Card}{\mathop{\text{Card}}\nolimits}
\newcommand{\Vect}{\mathop{\text{Vect}}\nolimits}
\newcommand{\Mat}{\mathop{\text{Mat}}\nolimits}
\newcommand{\rg}{\mathop{\text{rg}}\nolimits}
\newcommand{\tr}{\mathop{\text{tr}}\nolimits}


%----- Structure des exercices ------

\newtheoremstyle{styleexo}% name
{2ex}% Space above
{3ex}% Space below
{}% Body font
{}% Indent amount 1
{\bfseries} % Theorem head font
{}% Punctuation after theorem head
{\newline}% Space after theorem head 2
{}% Theorem head spec (can be left empty, meaning ‘normal’)

%\theoremstyle{styleexo}
\newtheorem{exo}{Exercice}
\newtheorem{ind}{Indications}
\newtheorem{cor}{Correction}


\newcommand{\exercice}[1]{} \newcommand{\finexercice}{}
%\newcommand{\exercice}[1]{{\tiny\texttt{#1}}\vspace{-2ex}} % pour afficher le numero absolu, l'auteur...
\newcommand{\enonce}{\begin{exo}} \newcommand{\finenonce}{\end{exo}}
\newcommand{\indication}{\begin{ind}} \newcommand{\finindication}{\end{ind}}
\newcommand{\correction}{\begin{cor}} \newcommand{\fincorrection}{\end{cor}}

\newcommand{\noindication}{\stepcounter{ind}}
\newcommand{\nocorrection}{\stepcounter{cor}}

\newcommand{\fiche}[1]{} \newcommand{\finfiche}{}
\newcommand{\titre}[1]{\centerline{\large \bf #1}}
\newcommand{\addcommand}[1]{}
\newcommand{\video}[1]{}

% Marge
\newcommand{\mymargin}[1]{\marginpar{{\small #1}}}

\def\noqed{\renewcommand{\qedsymbol}{}}


%----- Presentation ------
\setlength{\parindent}{0cm}

%\newcommand{\ExoSept}{\href{http://exo7.emath.fr}{\textbf{\textsf{Exo7}}}}

\definecolor{myred}{rgb}{0.93,0.26,0}
\definecolor{myorange}{rgb}{0.97,0.58,0}
\definecolor{myyellow}{rgb}{1,0.86,0}

\newcommand{\LogoExoSept}[1]{  % input : echelle
{\usefont{U}{cmss}{bx}{n}
\begin{tikzpicture}[scale=0.1*#1,transform shape]
  \fill[color=myorange] (0,0)--(4,0)--(4,-4)--(0,-4)--cycle;
  \fill[color=myred] (0,0)--(0,3)--(-3,3)--(-3,0)--cycle;
  \fill[color=myyellow] (4,0)--(7,4)--(3,7)--(0,3)--cycle;
  \node[scale=5] at (3.5,3.5) {Exo7};
\end{tikzpicture}}
}


\newcommand{\debutmontitre}{
  \author{} \date{} 
  \thispagestyle{empty}
  \hspace*{-10ex}
  \begin{minipage}{\textwidth}
    \titlepage  
  \vspace*{-2.5cm}
  \begin{center}
    \LogoExoSept{2.5}
  \end{center}
  \end{minipage}

  \vspace*{-0cm}
  
  % Astuce pour que le background ne soit pas discrétisé lors de la conversion pdf -> png
\begin{tikzpicture}
        \fill[opacity=0,green!60!black] (0,0)--++(0,0)--++(0,0)--++(0,0)--cycle; 
\end{tikzpicture}

% toc S'affiche trop tot :
% \tableofcontents[hideallsubsections, pausesections]
}

\newcommand{\finmontitre}{
  \end{frame}
  \setcounter{framenumber}{0}
} % ne marche pas pour une raison obscure

%----- Commandes supplementaires ------

% \usepackage[landscape]{geometry}
% \geometry{top=1cm, bottom=3cm, left=2cm, right=10cm, marginparsep=1cm
% }
% \usepackage[a4paper]{geometry}
% \geometry{top=2cm, bottom=2cm, left=2cm, right=2cm, marginparsep=1cm
% }

%\usepackage{standalone}


% New command Arnaud -- november 2011
\setbeamersize{text margin left=24ex}
% si vous modifier cette valeur il faut aussi
% modifier le decalage du titre pour compenser
% (ex : ici =+10ex, titre =-5ex

\theoremstyle{definition}
%\newtheorem{proposition}{Proposition}
%\newtheorem{exemple}{Exemple}
%\newtheorem{theoreme}{Théorème}
%\newtheorem{lemme}{Lemme}
%\newtheorem{corollaire}{Corollaire}
%\newtheorem*{remarque*}{Remarque}
%\newtheorem*{miniexercice}{Mini-exercices}
%\newtheorem{definition}{Définition}

% Commande tikz
\usetikzlibrary{calc}
\usetikzlibrary{patterns,arrows}
\usetikzlibrary{matrix}
\usetikzlibrary{fadings} 

%definition d'un terme
\newcommand{\defi}[1]{{\color{myorange}\textbf{\emph{#1}}}}
\newcommand{\evidence}[1]{{\color{blue}\textbf{\emph{#1}}}}
\newcommand{\assertion}[1]{\emph{\og#1\fg}}  % pour chapitre logique
%\renewcommand{\contentsname}{Sommaire}
\renewcommand{\contentsname}{}
\setcounter{tocdepth}{2}



%------ Figures ------

\def\myscale{1} % par défaut 
\newcommand{\myfigure}[2]{  % entrée : echelle, fichier figure
\def\myscale{#1}
\begin{center}
\footnotesize
{#2}
\end{center}}


%------ Encadrement ------

\usepackage{fancybox}


\newcommand{\mybox}[1]{
\setlength{\fboxsep}{7pt}
\begin{center}
\shadowbox{#1}
\end{center}}

\newcommand{\myboxinline}[1]{
\setlength{\fboxsep}{5pt}
\raisebox{-10pt}{
\shadowbox{#1}
}
}

%--------------- Commande beamer---------------
\newcommand{\beameronly}[1]{#1} % permet de mettre des pause dans beamer pas dans poly


\setbeamertemplate{navigation symbols}{}
\setbeamertemplate{footline}  % tiré du fichier beamerouterinfolines.sty
{
  \leavevmode%
  \hbox{%
  \begin{beamercolorbox}[wd=.333333\paperwidth,ht=2.25ex,dp=1ex,center]{author in head/foot}%
    % \usebeamerfont{author in head/foot}\insertshortauthor%~~(\insertshortinstitute)
    \usebeamerfont{section in head/foot}{\bf\insertshorttitle}
  \end{beamercolorbox}%
  \begin{beamercolorbox}[wd=.333333\paperwidth,ht=2.25ex,dp=1ex,center]{title in head/foot}%
    \usebeamerfont{section in head/foot}{\bf\insertsectionhead}
  \end{beamercolorbox}%
  \begin{beamercolorbox}[wd=.333333\paperwidth,ht=2.25ex,dp=1ex,right]{date in head/foot}%
    % \usebeamerfont{date in head/foot}\insertshortdate{}\hspace*{2em}
    \insertframenumber{} / \inserttotalframenumber\hspace*{2ex} 
  \end{beamercolorbox}}%
  \vskip0pt%
}


\definecolor{mygrey}{rgb}{0.5,0.5,0.5}
\setlength{\parindent}{0cm}
%\DeclareTextFontCommand{\helvetica}{\fontfamily{phv}\selectfont}

% background beamer
\definecolor{couleurhaut}{rgb}{0.85,0.9,1}  % creme
\definecolor{couleurmilieu}{rgb}{1,1,1}  % vert pale
\definecolor{couleurbas}{rgb}{0.85,0.9,1}  % blanc
\setbeamertemplate{background canvas}[vertical shading]%
[top=couleurhaut,middle=couleurmilieu,midpoint=0.4,bottom=couleurbas] 
%[top=fondtitre!05,bottom=fondtitre!60]



\makeatletter
\setbeamertemplate{theorem begin}
{%
  \begin{\inserttheoremblockenv}
  {%
    \inserttheoremheadfont
    \inserttheoremname
    \inserttheoremnumber
    \ifx\inserttheoremaddition\@empty\else\ (\inserttheoremaddition)\fi%
    \inserttheorempunctuation
  }%
}
\setbeamertemplate{theorem end}{\end{\inserttheoremblockenv}}

\newenvironment{theoreme}[1][]{%
   \setbeamercolor{block title}{fg=structure,bg=structure!40}
   \setbeamercolor{block body}{fg=black,bg=structure!10}
   \begin{block}{{\bf Th\'eor\`eme }#1}
}{%
   \end{block}%
}


\newenvironment{proposition}[1][]{%
   \setbeamercolor{block title}{fg=structure,bg=structure!40}
   \setbeamercolor{block body}{fg=black,bg=structure!10}
   \begin{block}{{\bf Proposition }#1}
}{%
   \end{block}%
}

\newenvironment{corollaire}[1][]{%
   \setbeamercolor{block title}{fg=structure,bg=structure!40}
   \setbeamercolor{block body}{fg=black,bg=structure!10}
   \begin{block}{{\bf Corollaire }#1}
}{%
   \end{block}%
}

\newenvironment{mydefinition}[1][]{%
   \setbeamercolor{block title}{fg=structure,bg=structure!40}
   \setbeamercolor{block body}{fg=black,bg=structure!10}
   \begin{block}{{\bf Définition} #1}
}{%
   \end{block}%
}

\newenvironment{lemme}[0]{%
   \setbeamercolor{block title}{fg=structure,bg=structure!40}
   \setbeamercolor{block body}{fg=black,bg=structure!10}
   \begin{block}{\bf Lemme}
}{%
   \end{block}%
}

\newenvironment{remarque}[1][]{%
   \setbeamercolor{block title}{fg=black,bg=structure!20}
   \setbeamercolor{block body}{fg=black,bg=structure!5}
   \begin{block}{Remarque #1}
}{%
   \end{block}%
}


\newenvironment{exemple}[1][]{%
   \setbeamercolor{block title}{fg=black,bg=structure!20}
   \setbeamercolor{block body}{fg=black,bg=structure!5}
   \begin{block}{{\bf Exemple }#1}
}{%
   \end{block}%
}


\newenvironment{miniexercice}[0]{%
   \setbeamercolor{block title}{fg=structure,bg=structure!20}
   \setbeamercolor{block body}{fg=black,bg=structure!5}
   \begin{block}{Mini-exercices}
}{%
   \end{block}%
}


\newenvironment{tp}[0]{%
   \setbeamercolor{block title}{fg=structure,bg=structure!40}
   \setbeamercolor{block body}{fg=black,bg=structure!10}
   \begin{block}{\bf Travaux pratiques}
}{%
   \end{block}%
}
\newenvironment{exercicecours}[1][]{%
   \setbeamercolor{block title}{fg=structure,bg=structure!40}
   \setbeamercolor{block body}{fg=black,bg=structure!10}
   \begin{block}{{\bf Exercice }#1}
}{%
   \end{block}%
}
\newenvironment{algo}[1][]{%
   \setbeamercolor{block title}{fg=structure,bg=structure!40}
   \setbeamercolor{block body}{fg=black,bg=structure!10}
   \begin{block}{{\bf Algorithme}\hfill{\color{gray}\texttt{#1}}}
}{%
   \end{block}%
}


\setbeamertemplate{proof begin}{
   \setbeamercolor{block title}{fg=black,bg=structure!20}
   \setbeamercolor{block body}{fg=black,bg=structure!5}
   \begin{block}{{\footnotesize Démonstration}}
   \footnotesize
   \smallskip}
\setbeamertemplate{proof end}{%
   \end{block}}
\setbeamertemplate{qed symbol}{\openbox}


\makeatother
\usecolortheme[RGB={205,0,0}]{structure}

%%%%%%%%%%%%%%%%%%%%%%%%%%%%%%%%%%%%%%%%%%%%%%%%%%%%%%%%%%%%%
%%%%%%%%%%%%%%%%%%%%%%%%%%%%%%%%%%%%%%%%%%%%%%%%%%%%%%%%%%%%%


\begin{document}


\title{{\bf Espaces vectoriels}}
\subtitle{Espace vectoriel (fin)}

\begin{frame}
  
  \debutmontitre

  \pause

{\footnotesize
\hfill
\setbeamercovered{transparent=50}
\begin{minipage}{0.6\textwidth}
  \begin{itemize}
    \item<3-> Détail des axiomes de la définition
    \item<4-> Exemples
    \item<5-> Règles de calcul    
  \end{itemize}
\end{minipage}
}

\end{frame}

\setcounter{framenumber}{0}


%%%%%%%%%%%%%%%%%%%%%%%%%%%%%%%%%%%%%%%%%%%%%%%%%%%%%%%%%%%%%%%%
\section{Détail des axiomes de la définition}

\begin{frame}


Soit $E$ un $\Kk$-espace vectoriel
\begin{itemize}
  \item Les éléments de $E$ seront appelés des \defi{vecteurs}
  \item Les éléments du corps $\Kk$ seront appelés des \defi{scalaires}
\end{itemize}

\pause
\bigskip

\defi{Loi interne} \\
La loi de composition interne, c'est une application de $E \times E$ dans $E$ :
$$\begin{array}{rcl}
E \times E & \to & E\\
(u, v) & \mapsto & u+v
\end{array}$$

\pause
\bigskip
\defi{Loi externe} \\
La loi de composition externe, 
 c'est une application de $\Kk \times E$ dans $E$ : 
$$\begin{array}{rcl}
\Kk \times E & \to & E\\
(\lambda, u ) & \mapsto & \lambda \cdot u 
\end{array}$$

\end{frame}


\begin{frame}
\textbf{Axiomes relatifs à la loi interne}

 \begin{enumerate}
 
\pause 
 \item \evidence{Commutativité} 
 
 Pour tous $u,v \in E$, $u + v = v + u$
 
\pause

 \item \evidence{Associativité} 

 Pour tous $u,v,w \in E$, on a $u + (v+w) = (u+v) +w$
 
\pause 

 \item \evidence{Existence d'un élément neutre}


 Il existe un élément de $E$, noté $0_{E}$, vérifiant : 
  
   \hfil pour tout $u \in E$, $u+0_{E}=u$
   
   \pause
   \begin{itemize}
     \item On a aussi $0_E+u=u$
     \pause
     \item Cet élément $0_E$ s'appelle aussi le \defi{vecteur nul}   
   \end{itemize}
   
\pause 
 
 \item \evidence{Existence d'un symétrique}
 
 Pour tout $u$ de $E$ il existe un élément $u'$ de $E$ tel que 
$u+u'=0_E$ 
\pause
   \begin{itemize}
     \item On a aussi $u'+u=0_E$
     \pause
     \item Cet élément $u'$ est noté $-u$
   \end{itemize}


 \end{enumerate} 

\end{frame}


\begin{frame}
\begin{proposition}
\begin{itemize}
  \item Il existe un unique élément neutre $0_{E}$
    \pause  
  \item Pour $u$ un élément de $E$, il existe un unique symétrique $-u$ 
\end{itemize}
\end{proposition}

   \pause 

\begin{proof}
\begin{itemize}
  \item Soient $0_{E}$ et $0'_{E}$ deux éléments neutres. 
  \pause Alors, pour tout $u$ de $E$
$$(\star) \quad u + 0_{E}=0_{E}+u=u \quad \text{ et } \quad (\star\star)\quad u + 0'_{E}=0'_{E}+u=u$$
\vspace*{-4ex}
   \pause
  \begin{itemize}
    \item Alors $(\star)$ avec $u=0'_{E}$ donne $0'_{E}+0_{E}=0_{E}+0'_{E}=0'_{E}$
    \pause  
    \item Et $(\star\star)$ avec $u=0_{E}$ donne $0_{E}+0'_{E}=0'_{E}+0_{E}=0_{E}$
     \pause 
    \item D'où $0_{E}=0'_{E}$
  \end{itemize}
  
   \pause 
  \item Si $u'$ et $u''$ sont deux symétriques du même $u$,
  \pause
  on a
$$u+u'=u'+u=0_{E}  \qquad \text{ et } \qquad u+u''=u''+u=0_{E}$$
\vspace*{-4ex}
\pause
  \begin{itemize}
    \item $u'+(u+u'')= u'+ 0_{E}= u'$
   \pause
    \item $u'+(u+u'')=(u'+u)+u''=0_{E}+u''=u''$
   \pause
    \item On en déduit $u'=u''$
  \end{itemize}
\end{itemize}
\vspace*{-4ex}
\end{proof}
\end{frame}


\begin{frame}
\bigskip
\textbf{Axiomes relatifs à la loi externe}
   
\pause 
 \begin{enumerate}  \setcounter{enumi}{4}
 \item Soit $1$ l'élément neutre de la multiplication de $\Kk$. 
 Pour tout élément $u$ de $E$, on a 
 $$1 \cdot u=u$$
   
\pause  
 \item Pour tous éléments $\lambda$ et $\mu$ de $\Kk$ et pour tout élément $u$ de $E$, on a 
 $$\lambda \cdot (\mu \cdot u) = (\lambda \times \mu )\cdot u$$
 \end{enumerate} 
   
\pause  
\bigskip
\textbf{Axiomes liant les deux lois}
   
\pause 
 \begin{enumerate}  \setcounter{enumi}{6}
 \item \evidence{Distributivité} par rapport à l'addition des vecteurs
 
 Pour tout élément  $\lambda$ de $\Kk$ et pour tous éléments $u$ et $v$ de $E$, on a 
 $$\lambda \cdot (u+v) =\lambda \cdot u + \lambda \cdot v$$
 
    
\pause 
 \item \evidence{Distributivité} par rapport à l'addition des scalaires
 
 Pour tous $\lambda$ et $\mu$ de $\Kk$  et 
pour tout élément $u$ de $E$, on a :
$$(\lambda + \mu ) \cdot u=\lambda \cdot u + \mu \cdot u $$
 
 \end{enumerate} 
\end{frame}

%%%%%%%%%%%%%%%%%%%%%%%%%%%%%%%%%%%%%%%%%%%%%%%%%%%%%%%%%%%%%%%%
\section{Exemples}

\begin{frame}
\centerline{\evidence{L'espace vectoriel des fonctions de $\Rr$ dans $\Rr$}}

\medskip

$\mathcal{F}(\Rr, \Rr)$ : l'ensemble des fonctions $f : \Rr \longrightarrow \Rr$
    
\pause 

\begin{itemize}
  \item \textbf{Loi interne} 
  
Pour $f, g \in \mathcal{F}(\Rr, \Rr)$, $f+g$ est définie par
$$\forall x \in \Rr \quad (f+g)(x)=f(x)+g(x)$$
    
\pause   
  \item \textbf{Loi externe} 
  
Pour $\lambda \in \Rr$ et $f \in \mathcal{F}(\Rr, \Rr)$, la fonction 
$\lambda \cdot f$ est définie par
$$\forall x \in \Rr \quad (\lambda \cdot f) (x)=\lambda \times f (x)$$
 
    
\pause   
  \item \textbf{\'Elément neutre} 
  
  L'élément neutre pour l'addition est la fonction nulle, définie par
$$\forall x \in \Rr \quad f(x)=0$$
     
\pause  
  \item \textbf{Symétrique}  
  
Le symétrique de $f \in \mathcal{F}(\Rr , \Rr)$ est $g$ définie par
$$\forall x \in \Rr \quad g(x)=-f(x)$$
   
\end{itemize}
\end{frame}



\begin{frame}

\centerline{\evidence{Le $\Rr$-espace vectoriel des suites réelles}}

\bigskip

$\mathcal{S}= \mathcal{F}(\Nn, \Rr)$ : l'ensemble des suites réelles $(u_n)_{n\in \Nn}$

\pause 
\begin{itemize}
  \item  \textbf{Loi interne}  
  
Pour $u=(u_n)_{n \in \Nn}$ et $v=(v_n)_{n \in \Nn}$ deux suites, la suite $u+v$ 
est la suite dont le terme général est 
$$u_n+v_n$$ 

\pause
  \item \textbf{Loi externe}   
  
Si $\lambda \in \Rr$  et $u=(u_n)_{n \in \Nn}$ est une suite, 
alors $\lambda \cdot u$ est la suite dont le terme général est
 $$\lambda \times u_n$$
  
\pause 
  \item \textbf{\'Elément neutre}  
  
  L'élément neutre est la suite dont tous les termes sont nuls

\pause  
  \item \textbf{Symétrique}  
  
  Le symétrique de $u=(u_n)_{n \in \Nn}$ est la suite dont le terme général est
 $$-u_n$$

\end{itemize}

\end{frame}

\begin{frame}\centerline{\evidence{Les matrices}}

\bigskip

L'ensemble $M_{n,p}(\Rr)$ des matrices à $n$ lignes et $p$ colonnes à coefficients 
dans $\Rr$ est muni d'une structure de $\Rr$-espace vectoriel

\pause
\bigskip

\begin{itemize}
  \item La loi interne est l'addition de deux matrices
  \item La loi externe est la multiplication d'une matrice par un scalaire
  \item L'élément neutre pour la loi interne est la matrice nulle
  \item Le symétrique de la matrice $A=(a_{i,j})$ est la matrice 
 $(-a_{i,j})$
\end{itemize}

\end{frame}

\begin{frame}
\centerline{\textbf{Autres exemples}}
\bigskip

\pause
\begin{enumerate}
  \item \evidence{L'espace vectoriel $\Rr[X]$ des polynômes}
\pause
    \begin{itemize}
      \item $P(X) = a_nX^n+\cdots+a_2X^2+a_1X+a_0$
      \item L'addition est l'addition de deux polynômes $P(X)+Q(X)$
      \item La multiplication par un scalaire $\lambda \in \Rr$ est $\lambda \cdot P(X)$
      \item L'élément neutre est le polynôme nul
      \item L'opposé de $P(X)$ est $-P(X)$
    \end{itemize}
\pause 
  \item 
  \evidence{L'ensemble des fonctions continues de $\Rr$ dans $\Rr$} 
\pause  

  L'ensemble des fonctions dérivables de $\Rr$ dans $\Rr$,...
\pause 
  \item ...
\end{enumerate}
\end{frame}



%%%%%%%%%%%%%%%%%%%%%%%%%%%%%%%%%%%%%%%%%%%%%%%%%%%%%%%%%%%%%%%%
\section{Règles de calcul}

\begin{frame}

Soit $E$ un espace vectoriel sur un corps $\Kk$

Soient $u \in E$ et $\lambda \in \Kk$
\pause
\begin{proposition}

 \begin{enumerate}\setlength{\itemsep}{6pt}
 \item $0 \cdot u = 0_E$
\pause 
 \item $\lambda \cdot 0_E = 0_E$
\pause 
 \item $(-1)\cdot u = -u$
\pause 
 \item \myboxinline{$\lambda \cdot u = 0_E \iff \lambda = 0$ \ ou \ $u = 0_E$}
 \end{enumerate} 
\end{proposition}

\bigskip
\pause
\begin{itemize}
  \item La \defi{soustraction} $u-v = u+(-v)$
\pause  
  \item $\lambda (u-v)=\lambda u -\lambda v$
 \quad et \quad  $(\lambda -\mu)u=\lambda u-\mu u$
\end{itemize}

\end{frame}





%%%%%%%%%%%%%%%%%%%%%%%%%%%%%%%%%%%%%%%%%%%%%%%%%%%%%%%%%%%%%%%%
\section{Mini-exercices}

\begin{frame}

\begin{miniexercice}
\begin{enumerate}
  \item  Justifier si les objets suivants sont des espaces vectoriels. 
  \begin{enumerate}
    \item L'ensemble des fonctions réelles sur
$\lbrack 0,1 \rbrack$, continues, positives ou nulles, pour
l'addition et le produit par un réel.
    \item L'ensemble des fonctions réelles sur $\Rr$ vérifiant
$\lim_{x \to+\infty} f(x)=0$ pour les mêmes opérations.
    \item L'ensemble des fonctions sur $\Rr$ telles que $f(3)=7$.
    \item L'ensemble $\Rr_+^*$ pour les opérations $x \oplus y=xy$ et 
$\lambda\cdot x=x^{\lambda}$ $(\lambda\in \Rr)$.
    \item L'ensemble des points $(x,y)$ de $\Rr^2$ vérifiant
$\sin(x+y)=0$. 
    \item L'ensemble des vecteurs $(x,y,z)$ de $\Rr^3$ orthogonaux
au vecteur $(-1,3,-2)$.
    \item L'ensemble des fonctions de classe $\mathcal{C}^2$ vérifiant $f''+f=0$. 
    \item L'ensemble des fonctions continues sur $\lbrack0,1 \rbrack$
vérifiant $\int_0^1f(x) \, \sin x \; dx=0$.
    \item L'ensemble des matrices 
    $\left(\begin{smallmatrix} a & b \\ c & d \end {smallmatrix}\right) \in M_{2}(\Rr)$ vérifiant $a+d=0$.   
    \end{enumerate}
  
  \item Prouver les propriétés de la soustraction : 
 $\lambda \cdot (u-v)=\lambda \cdot u -\lambda \cdot  v$  et $(\lambda -\mu) \cdot u=\lambda \cdot u-\mu \cdot u$.
\end{enumerate} 
\end{miniexercice}

\end{frame}

\end{document}