
%%%%%%%%%%%%%%%%%% PREAMBULE %%%%%%%%%%%%%%%%%%

\documentclass[aspectratio=169,utf8]{beamer}
%\documentclass[aspectratio=169,handout]{beamer}

\usetheme{Boadilla}
%\usecolortheme{seahorse}
\usecolortheme[RGB={245,66,24}]{structure}
\useoutertheme{infolines}

% packages
\usepackage{amsfonts,amsmath,amssymb,amsthm}
\usepackage[utf8]{inputenc}
\usepackage[T1]{fontenc}
\usepackage{lmodern}

\usepackage[francais]{babel}
\usepackage{fancybox}
\usepackage{graphicx}

\usepackage{float}
\usepackage{xfrac}

%\usepackage[usenames, x11names]{xcolor}
\usepackage{tikz}
\usepackage{pgfplots}
\usepackage{datetime}



%-----  Package unités -----
\usepackage{siunitx}
\sisetup{locale = FR,detect-all,per-mode = symbol}

%\usepackage{mathptmx}
%\usepackage{fouriernc}
%\usepackage{newcent}
%\usepackage[mathcal,mathbf]{euler}

%\usepackage{palatino}
%\usepackage{newcent}
% \usepackage[mathcal,mathbf]{euler}



% \usepackage{hyperref}
% \hypersetup{colorlinks=true, linkcolor=blue, urlcolor=blue,
% pdftitle={Exo7 - Exercices de mathématiques}, pdfauthor={Exo7}}


%section
% \usepackage{sectsty}
% \allsectionsfont{\bf}
%\sectionfont{\color{Tomato3}\upshape\selectfont}
%\subsectionfont{\color{Tomato4}\upshape\selectfont}

%----- Ensembles : entiers, reels, complexes -----
\newcommand{\Nn}{\mathbb{N}} \newcommand{\N}{\mathbb{N}}
\newcommand{\Zz}{\mathbb{Z}} \newcommand{\Z}{\mathbb{Z}}
\newcommand{\Qq}{\mathbb{Q}} \newcommand{\Q}{\mathbb{Q}}
\newcommand{\Rr}{\mathbb{R}} \newcommand{\R}{\mathbb{R}}
\newcommand{\Cc}{\mathbb{C}} 
\newcommand{\Kk}{\mathbb{K}} \newcommand{\K}{\mathbb{K}}

%----- Modifications de symboles -----
\renewcommand{\epsilon}{\varepsilon}
\renewcommand{\Re}{\mathop{\text{Re}}\nolimits}
\renewcommand{\Im}{\mathop{\text{Im}}\nolimits}
%\newcommand{\llbracket}{\left[\kern-0.15em\left[}
%\newcommand{\rrbracket}{\right]\kern-0.15em\right]}

\renewcommand{\ge}{\geqslant}
\renewcommand{\geq}{\geqslant}
\renewcommand{\le}{\leqslant}
\renewcommand{\leq}{\leqslant}
\renewcommand{\epsilon}{\varepsilon}

%----- Fonctions usuelles -----
\newcommand{\ch}{\mathop{\text{ch}}\nolimits}
\newcommand{\sh}{\mathop{\text{sh}}\nolimits}
\renewcommand{\tanh}{\mathop{\text{th}}\nolimits}
\newcommand{\cotan}{\mathop{\text{cotan}}\nolimits}
\newcommand{\Arcsin}{\mathop{\text{arcsin}}\nolimits}
\newcommand{\Arccos}{\mathop{\text{arccos}}\nolimits}
\newcommand{\Arctan}{\mathop{\text{arctan}}\nolimits}
\newcommand{\Argsh}{\mathop{\text{argsh}}\nolimits}
\newcommand{\Argch}{\mathop{\text{argch}}\nolimits}
\newcommand{\Argth}{\mathop{\text{argth}}\nolimits}
\newcommand{\pgcd}{\mathop{\text{pgcd}}\nolimits} 


%----- Commandes divers ------
\newcommand{\ii}{\mathrm{i}}
\newcommand{\dd}{\text{d}}
\newcommand{\id}{\mathop{\text{id}}\nolimits}
\newcommand{\Ker}{\mathop{\text{Ker}}\nolimits}
\newcommand{\Card}{\mathop{\text{Card}}\nolimits}
\newcommand{\Vect}{\mathop{\text{Vect}}\nolimits}
\newcommand{\Mat}{\mathop{\text{Mat}}\nolimits}
\newcommand{\rg}{\mathop{\text{rg}}\nolimits}
\newcommand{\tr}{\mathop{\text{tr}}\nolimits}


%----- Structure des exercices ------

\newtheoremstyle{styleexo}% name
{2ex}% Space above
{3ex}% Space below
{}% Body font
{}% Indent amount 1
{\bfseries} % Theorem head font
{}% Punctuation after theorem head
{\newline}% Space after theorem head 2
{}% Theorem head spec (can be left empty, meaning ‘normal’)

%\theoremstyle{styleexo}
\newtheorem{exo}{Exercice}
\newtheorem{ind}{Indications}
\newtheorem{cor}{Correction}


\newcommand{\exercice}[1]{} \newcommand{\finexercice}{}
%\newcommand{\exercice}[1]{{\tiny\texttt{#1}}\vspace{-2ex}} % pour afficher le numero absolu, l'auteur...
\newcommand{\enonce}{\begin{exo}} \newcommand{\finenonce}{\end{exo}}
\newcommand{\indication}{\begin{ind}} \newcommand{\finindication}{\end{ind}}
\newcommand{\correction}{\begin{cor}} \newcommand{\fincorrection}{\end{cor}}

\newcommand{\noindication}{\stepcounter{ind}}
\newcommand{\nocorrection}{\stepcounter{cor}}

\newcommand{\fiche}[1]{} \newcommand{\finfiche}{}
\newcommand{\titre}[1]{\centerline{\large \bf #1}}
\newcommand{\addcommand}[1]{}
\newcommand{\video}[1]{}

% Marge
\newcommand{\mymargin}[1]{\marginpar{{\small #1}}}

\def\noqed{\renewcommand{\qedsymbol}{}}


%----- Presentation ------
\setlength{\parindent}{0cm}

%\newcommand{\ExoSept}{\href{http://exo7.emath.fr}{\textbf{\textsf{Exo7}}}}

\definecolor{myred}{rgb}{0.93,0.26,0}
\definecolor{myorange}{rgb}{0.97,0.58,0}
\definecolor{myyellow}{rgb}{1,0.86,0}

\newcommand{\LogoExoSept}[1]{  % input : echelle
{\usefont{U}{cmss}{bx}{n}
\begin{tikzpicture}[scale=0.1*#1,transform shape]
  \fill[color=myorange] (0,0)--(4,0)--(4,-4)--(0,-4)--cycle;
  \fill[color=myred] (0,0)--(0,3)--(-3,3)--(-3,0)--cycle;
  \fill[color=myyellow] (4,0)--(7,4)--(3,7)--(0,3)--cycle;
  \node[scale=5] at (3.5,3.5) {Exo7};
\end{tikzpicture}}
}


\newcommand{\debutmontitre}{
  \author{} \date{} 
  \thispagestyle{empty}
  \hspace*{-10ex}
  \begin{minipage}{\textwidth}
    \titlepage  
  \vspace*{-2.5cm}
  \begin{center}
    \LogoExoSept{2.5}
  \end{center}
  \end{minipage}

  \vspace*{-0cm}
  
  % Astuce pour que le background ne soit pas discrétisé lors de la conversion pdf -> png
\begin{tikzpicture}
        \fill[opacity=0,green!60!black] (0,0)--++(0,0)--++(0,0)--++(0,0)--cycle; 
\end{tikzpicture}

% toc S'affiche trop tot :
% \tableofcontents[hideallsubsections, pausesections]
}

\newcommand{\finmontitre}{
  \end{frame}
  \setcounter{framenumber}{0}
} % ne marche pas pour une raison obscure

%----- Commandes supplementaires ------

% \usepackage[landscape]{geometry}
% \geometry{top=1cm, bottom=3cm, left=2cm, right=10cm, marginparsep=1cm
% }
% \usepackage[a4paper]{geometry}
% \geometry{top=2cm, bottom=2cm, left=2cm, right=2cm, marginparsep=1cm
% }

%\usepackage{standalone}


% New command Arnaud -- november 2011
\setbeamersize{text margin left=24ex}
% si vous modifier cette valeur il faut aussi
% modifier le decalage du titre pour compenser
% (ex : ici =+10ex, titre =-5ex

\theoremstyle{definition}
%\newtheorem{proposition}{Proposition}
%\newtheorem{exemple}{Exemple}
%\newtheorem{theoreme}{Théorème}
%\newtheorem{lemme}{Lemme}
%\newtheorem{corollaire}{Corollaire}
%\newtheorem*{remarque*}{Remarque}
%\newtheorem*{miniexercice}{Mini-exercices}
%\newtheorem{definition}{Définition}

% Commande tikz
\usetikzlibrary{calc}
\usetikzlibrary{patterns,arrows}
\usetikzlibrary{matrix}
\usetikzlibrary{fadings} 

%definition d'un terme
\newcommand{\defi}[1]{{\color{myorange}\textbf{\emph{#1}}}}
\newcommand{\evidence}[1]{{\color{blue}\textbf{\emph{#1}}}}
\newcommand{\assertion}[1]{\emph{\og#1\fg}}  % pour chapitre logique
%\renewcommand{\contentsname}{Sommaire}
\renewcommand{\contentsname}{}
\setcounter{tocdepth}{2}



%------ Figures ------

\def\myscale{1} % par défaut 
\newcommand{\myfigure}[2]{  % entrée : echelle, fichier figure
\def\myscale{#1}
\begin{center}
\footnotesize
{#2}
\end{center}}


%------ Encadrement ------

\usepackage{fancybox}


\newcommand{\mybox}[1]{
\setlength{\fboxsep}{7pt}
\begin{center}
\shadowbox{#1}
\end{center}}

\newcommand{\myboxinline}[1]{
\setlength{\fboxsep}{5pt}
\raisebox{-10pt}{
\shadowbox{#1}
}
}

%--------------- Commande beamer---------------
\newcommand{\beameronly}[1]{#1} % permet de mettre des pause dans beamer pas dans poly


\setbeamertemplate{navigation symbols}{}
\setbeamertemplate{footline}  % tiré du fichier beamerouterinfolines.sty
{
  \leavevmode%
  \hbox{%
  \begin{beamercolorbox}[wd=.333333\paperwidth,ht=2.25ex,dp=1ex,center]{author in head/foot}%
    % \usebeamerfont{author in head/foot}\insertshortauthor%~~(\insertshortinstitute)
    \usebeamerfont{section in head/foot}{\bf\insertshorttitle}
  \end{beamercolorbox}%
  \begin{beamercolorbox}[wd=.333333\paperwidth,ht=2.25ex,dp=1ex,center]{title in head/foot}%
    \usebeamerfont{section in head/foot}{\bf\insertsectionhead}
  \end{beamercolorbox}%
  \begin{beamercolorbox}[wd=.333333\paperwidth,ht=2.25ex,dp=1ex,right]{date in head/foot}%
    % \usebeamerfont{date in head/foot}\insertshortdate{}\hspace*{2em}
    \insertframenumber{} / \inserttotalframenumber\hspace*{2ex} 
  \end{beamercolorbox}}%
  \vskip0pt%
}


\definecolor{mygrey}{rgb}{0.5,0.5,0.5}
\setlength{\parindent}{0cm}
%\DeclareTextFontCommand{\helvetica}{\fontfamily{phv}\selectfont}

% background beamer
\definecolor{couleurhaut}{rgb}{0.85,0.9,1}  % creme
\definecolor{couleurmilieu}{rgb}{1,1,1}  % vert pale
\definecolor{couleurbas}{rgb}{0.85,0.9,1}  % blanc
\setbeamertemplate{background canvas}[vertical shading]%
[top=couleurhaut,middle=couleurmilieu,midpoint=0.4,bottom=couleurbas] 
%[top=fondtitre!05,bottom=fondtitre!60]



\makeatletter
\setbeamertemplate{theorem begin}
{%
  \begin{\inserttheoremblockenv}
  {%
    \inserttheoremheadfont
    \inserttheoremname
    \inserttheoremnumber
    \ifx\inserttheoremaddition\@empty\else\ (\inserttheoremaddition)\fi%
    \inserttheorempunctuation
  }%
}
\setbeamertemplate{theorem end}{\end{\inserttheoremblockenv}}

\newenvironment{theoreme}[1][]{%
   \setbeamercolor{block title}{fg=structure,bg=structure!40}
   \setbeamercolor{block body}{fg=black,bg=structure!10}
   \begin{block}{{\bf Th\'eor\`eme }#1}
}{%
   \end{block}%
}


\newenvironment{proposition}[1][]{%
   \setbeamercolor{block title}{fg=structure,bg=structure!40}
   \setbeamercolor{block body}{fg=black,bg=structure!10}
   \begin{block}{{\bf Proposition }#1}
}{%
   \end{block}%
}

\newenvironment{corollaire}[1][]{%
   \setbeamercolor{block title}{fg=structure,bg=structure!40}
   \setbeamercolor{block body}{fg=black,bg=structure!10}
   \begin{block}{{\bf Corollaire }#1}
}{%
   \end{block}%
}

\newenvironment{mydefinition}[1][]{%
   \setbeamercolor{block title}{fg=structure,bg=structure!40}
   \setbeamercolor{block body}{fg=black,bg=structure!10}
   \begin{block}{{\bf Définition} #1}
}{%
   \end{block}%
}

\newenvironment{lemme}[0]{%
   \setbeamercolor{block title}{fg=structure,bg=structure!40}
   \setbeamercolor{block body}{fg=black,bg=structure!10}
   \begin{block}{\bf Lemme}
}{%
   \end{block}%
}

\newenvironment{remarque}[1][]{%
   \setbeamercolor{block title}{fg=black,bg=structure!20}
   \setbeamercolor{block body}{fg=black,bg=structure!5}
   \begin{block}{Remarque #1}
}{%
   \end{block}%
}


\newenvironment{exemple}[1][]{%
   \setbeamercolor{block title}{fg=black,bg=structure!20}
   \setbeamercolor{block body}{fg=black,bg=structure!5}
   \begin{block}{{\bf Exemple }#1}
}{%
   \end{block}%
}


\newenvironment{miniexercice}[0]{%
   \setbeamercolor{block title}{fg=structure,bg=structure!20}
   \setbeamercolor{block body}{fg=black,bg=structure!5}
   \begin{block}{Mini-exercices}
}{%
   \end{block}%
}


\newenvironment{tp}[0]{%
   \setbeamercolor{block title}{fg=structure,bg=structure!40}
   \setbeamercolor{block body}{fg=black,bg=structure!10}
   \begin{block}{\bf Travaux pratiques}
}{%
   \end{block}%
}
\newenvironment{exercicecours}[1][]{%
   \setbeamercolor{block title}{fg=structure,bg=structure!40}
   \setbeamercolor{block body}{fg=black,bg=structure!10}
   \begin{block}{{\bf Exercice }#1}
}{%
   \end{block}%
}
\newenvironment{algo}[1][]{%
   \setbeamercolor{block title}{fg=structure,bg=structure!40}
   \setbeamercolor{block body}{fg=black,bg=structure!10}
   \begin{block}{{\bf Algorithme}\hfill{\color{gray}\texttt{#1}}}
}{%
   \end{block}%
}


\setbeamertemplate{proof begin}{
   \setbeamercolor{block title}{fg=black,bg=structure!20}
   \setbeamercolor{block body}{fg=black,bg=structure!5}
   \begin{block}{{\footnotesize Démonstration}}
   \footnotesize
   \smallskip}
\setbeamertemplate{proof end}{%
   \end{block}}
\setbeamertemplate{qed symbol}{\openbox}


\makeatother
\usecolortheme[RGB={205,0,0}]{structure}

%%%%%%%%%%%%%%%%%%%%%%%%%%%%%%%%%%%%%%%%%%%%%%%%%%%%%%%%%%%%%
%%%%%%%%%%%%%%%%%%%%%%%%%%%%%%%%%%%%%%%%%%%%%%%%%%%%%%%%%%%%%

\begin{document}


\title{{\bf Espaces vectoriels}}
\subtitle{Sous-espace vectoriel (milieu)}

\begin{frame}
  
  \debutmontitre

  \pause

{\footnotesize
\hfill
\setbeamercovered{transparent=50}
\begin{minipage}{0.6\textwidth}
  \begin{itemize}
    \item<3-> Combinaisons linéaires
    \item<4-> Caractérisation d'un sous-espace vectoriel
    \item<5-> Intersection de deux sous-espaces vectoriels    
  \end{itemize}
\end{minipage}
}

\end{frame}

\setcounter{framenumber}{0}



%%%%%%%%%%%%%%%%%%%%%%%%%%%%%%%%%%%%%%%%%%%%%%%%%%%%%%%%%%%%%%%%
\section{Combinaisons linéaires}



\begin{frame}
Soient  $v_1, v_2, \ldots, v_n$, $n$  vecteurs d'un espace vectoriel $E$
\begin{mydefinition}
  \begin{itemize}
    \item Tout vecteur de la forme  
 $$u=\lambda_1 v_1+\lambda_2v_2+ \cdots + \lambda_nv_n$$
 est appelé \defi{combinaison linéaire} des vecteurs $v_1, v_2, \ldots, v_n$
 
 \pause
 
    \item Les scalaires $\lambda_1, \lambda_2, \ldots , \lambda_n \in \Kk$ sont 
    les \defi{coefficients} de la combinaison linéaire
  \end{itemize}

\end{mydefinition}
 
 \bigskip
 \pause
 
\emph{Remarque :} Si $n=1$, alors $u=\lambda_1 v_1$ et on dit que $u$ est \defi{colinéaire} à $v_1$ 
 
\end{frame}


\begin{frame}
\begin{exemple}
\begin{enumerate}
  \item Dans le $\Rr$-espace vectoriel $\Rr^3$, $(3,3,1)$ est combinaison linéaire des vecteurs 
 $(1,1,0)$ et $(1,1,1)$ car on a 
 $$(3,3,1)=2(1,1,0)+(1,1,1)$$
 \vspace*{-2ex}
 \pause   
  \item Dans le $\Rr$-espace vectoriel $\Rr^2$, le vecteur
$u=(2,1)$ \emph{n'est pas} colinéaire au vecteur $v_1=(1,1)$
 \pause  
 
 \medskip 
  \item Soit $E=\mathcal{F}(\Rr, \Rr)$ et $f_0$, $f_1$, $f_2$ et $f_3$ les fonctions définies par 
 $$f_0(x)=1 \quad f_1(x)=x \quad f_2(x)=x^2 \quad f_3(x)=x^3$$
 \pause
 Alors la fonction $f$ définie par $f(x)=x^3-2x ^2-7x-4$
 est combinaison linéaire des fonctions $f_0, f_1, f_2, f_3$ puisque l'on a 
 $$f=f_{3}-2f_2-7f_1-4f_0$$ 
  
 
\end{enumerate}
\end{exemple}
\end{frame}



\begin{frame}
\begin{exemple}
\begin{itemize}
  \item Soient $u = \left(\begin{smallmatrix}1\\ 2\\ -1\end{smallmatrix}\right)$ et $v =
   \left(\begin{smallmatrix}6\\4\\2\end{smallmatrix}\right)$ deux vecteurs de $\Rr^3$

   \pause
   
  \item  Montrons que $w =
   \left(\begin{smallmatrix}9\\ 2\\ 7\end{smallmatrix}\right)$ est combinaison linéaire de $u$ et $v$
  
  \pause
  
  \item On cherche donc $\lambda$ et $\mu$ tels que $w=\lambda u + \mu v$
  
  \pause
  
  \item $\left(\begin{matrix}9\\2\\7\end{matrix}\right) 
 =  \lambda \left(\begin{matrix}1\cr 2\cr -1\end{matrix}\right) + \mu \left(\begin{matrix}6\\4\\2\end{matrix}\right)
 \pause
 =  \left(\begin{matrix}\lambda\\ 2\lambda\\ -\lambda\end{matrix}\right) 
     + \left(\begin{matrix}6\mu\\ 4\mu\\ 2\mu\end{matrix}\right)
 \pause
 =  \left(\begin{matrix}\lambda + 6\mu\\ 2\lambda + 4\mu\\ -\lambda + 2\mu\end{matrix}\right)$  
  
  \pause
  
  \item \'Equivaut à $\left\{
\begin{array}{rcl}
9 & = & \lambda + 6\mu\\ 2 & = & 2\lambda + 4\mu\\ 7 & = & -\lambda + 2\mu 
\end{array}\right.
$
  
  \pause
  \item Une solution de ce système est $(\lambda = -3, \mu = 2)$,
  \pause
  \item Ce qui implique que $w$ est combinaison linéaire de $u$ et $v$

\medskip
\pause
\hfil $\left(\begin{smallmatrix}9\\2\\7\end{smallmatrix}\right)
=  -3\left(\begin{smallmatrix}1\\2\\-1\end{smallmatrix}\right) 
+ 2\left(\begin{smallmatrix}6\\4\\2\end{smallmatrix}\right)$
\end{itemize}
\end{exemple}
\end{frame}


\begin{frame}
\begin{exemple}
\begin{itemize}
  \item Soient $u = \left(\begin{smallmatrix}1\\ 2\\ -1\end{smallmatrix}\right)$ et $v =
  \left(\begin{smallmatrix}6\\4\\2\end{smallmatrix}\right)$

   \pause
   
  \item Montrons que $ w = \left(\begin{smallmatrix}4\\ -1\\ 8\end{smallmatrix}\right)$
  n'est pas combinaison linéaire de $u$ et $v$

   \pause
   
  \item $\begin{pmatrix}4\\-1\\8\end{pmatrix} 
  = \lambda\begin{pmatrix}1\\2\\-1\end{pmatrix} + \mu\begin{pmatrix}6\\4\\2\end{pmatrix}
   \pause  \quad \iff \quad 
\left\{\begin{array}{rcl}
4 & = &\lambda + 6 \mu\\ -1 & = &2\lambda + 4\mu\\ 8 & = & -\lambda + 2\mu
\end{array}\right.$
 
   \pause
   
  \item Or ce système n'a aucune solution.
  Donc il n'existe pas $\lambda,\mu \in \Rr$ tels que $w=\lambda u + \mu v$
\end{itemize}

\end{exemple}
\end{frame}


%%%%%%%%%%%%%%%%%%%%%%%%%%%%%%%%%%%%%%%%%%%%%%%%%%%%%%%%%%%%%%%%
\section{Caractérisation d'un sous-espace vectoriel}

\begin{frame}
\begin{theoreme}[Caractérisation par la notion de combinaison linéaire]
Soient $E$ un $\Kk$-espace vectoriel et $F$ une partie non vide de $E$

$F$ est un sous-espace vectoriel de $E$ si et seulement si
$$\lambda u + \mu v \in F \qquad \text{pour tous } u,v \in F \quad \text{ et tous } \lambda, \mu \in \Kk$$

\pause

Autrement dit si et seulement si toute combinaison linéaire de deux éléments 
de $F$ appartient à $F$
\end{theoreme}  

\end{frame}

%%%%%%%%%%%%%%%%%%%%%%%%%%%%%%%%%%%%%%%%%%%%%%%%%%%%%%%%%%%%%%%%
\section{Intersection de deux sous-espaces vectoriels}

\begin{frame}
Soient $F,G$ deux sous-espaces vectoriels d'un $\Kk$-espace vectoriel $E$
\begin{proposition}
L'intersection $F \cap G$ est un sous-espace vectoriel de $E$
\end{proposition}

\pause
   
$F_1 \cap F_2 \cap F_3 \cap \cdots \cap F_n$ est un sous-espace vectoriel 

\pause

\begin{proof}
\begin{itemize}
  \item $0_E \in F$, $0_E\in G$ donc $0_E \in F \cap G$
\pause  
  \item 
  \begin{itemize}
    \item Soient $u,v \in F \cap G$
    \item $F$ est un sous-espace vectoriel, alors $u,v \in F$ implique $u+v\in F$
    \item De même $u,v \in G$ implique $u+v \in G$
    \item Donc $u+v \in F \cap G$
  \end{itemize}
\pause   
  \item 
  \begin{itemize}
    \item Soient $u \in F\cap G$ et $\lambda  \in \Kk$
    \item $F$ est un sous-espace vectoriel, alors $u \in F$ implique $\lambda u \in F$
    \item De même $u \in G$ implique $\lambda u \in G$
    \item Donc $\lambda u \in F \cap G$
  \end{itemize}
\pause 
\end{itemize}
Conclusion : $F\cap G$ est un sous-espace vectoriel de $E$
\end{proof}




\end{frame}


\begin{frame}
\begin{exemple}
$$\mathcal{D}= \big\{ (x,y,z) \in \Rr^3\mid x+3y+z =0 \;\; \text{ et } \;\; x-y+2z=0 \big\}$$

\myfigure{0.6}{
\tikzinput{fig_ev05} 
}

\pause
\vspace*{-3ex}
 \begin{itemize}
  \item $F=\big\{ (x,y,z) \in \Rr^3\mid x+3y+z =0 \big\}$
\pause
  \item $G=\big\{ (x,y,z) \in \Rr^3 \mid x-y+2z =0 \big\}$
\pause  
  \item $F$ et $G$ sont des plans vectoriels
\pause  
  \item $\mathcal{D} =F \cap G$ est un sous-espace vectoriel
\end{itemize} 


 
\end{exemple}
\end{frame}


\begin{frame} 


La réunion de deux sous-espaces vectoriels n'est pas en général 
un sous-espace vectoriel

\pause

\begin{itemize}
\uncover<2->{  \item $F=\big\{(x,y) \in \Rr^2 \mid x=0\big\}$ et $G=\big\{(x,y) \in \Rr^2 \mid y=0\big\}$}
\uncover<3->{  \item $(0,1) \in F$, $(1,0) \in G$}
\uncover<4->{  \item $(0,1)+(1,0)=(1,1) \notin F \cup G$}
\uncover<5->{  \item $F\cup G$ \textcolor{myred}{n'est pas} un sous-espace vectoriel}
\end{itemize}
\uncover<2->{
\myfigure{0.7}{
\tikzinput{fig_ev04-2bis-pres} 
}
}

\end{frame}

%%%%%%%%%%%%%%%%%%%%%%%%%%%%%%%%%%%%%%%%%%%%%%%%%%%%%%%%%%%%%%%%
\section{Mini-exercices}

\begin{frame}

\begin{miniexercice}
\begin{enumerate}
  \item Peut-on trouver $t\in \Rr$ tel que les vecteurs 
  $\left(\begin{smallmatrix}-2 \\ \sqrt2 \\ t \end{smallmatrix}\right)$ et
  $\left(\begin{smallmatrix}-4\sqrt2 \\ 4t \\ 2\sqrt2 \end{smallmatrix}\right)$ soient colinéaires ?
  
  \vspace*{-1ex}  
  \item Peut-on trouver $t\in \Rr$ tel que le vecteur  
  $\left(\begin{smallmatrix}1 \\ 3t \\ t \end{smallmatrix}\right)$ soit une combinaison linéaire
  de $\left(\begin{smallmatrix} 1 \\ 3 \\ 2 \end{smallmatrix}\right)$ et 
  $\left(\begin{smallmatrix} -1 \\ 1 \\ -1 \end{smallmatrix}\right)$ ?
    
\end{enumerate}
\end{miniexercice}

\end{frame}

\end{document}