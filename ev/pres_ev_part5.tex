
%%%%%%%%%%%%%%%%%% PREAMBULE %%%%%%%%%%%%%%%%%%

\documentclass[aspectratio=169,utf8]{beamer}
%\documentclass[aspectratio=169,handout]{beamer}

\usetheme{Boadilla}
%\usecolortheme{seahorse}
\usecolortheme[RGB={245,66,24}]{structure}
\useoutertheme{infolines}

% packages
\usepackage{amsfonts,amsmath,amssymb,amsthm}
\usepackage[utf8]{inputenc}
\usepackage[T1]{fontenc}
\usepackage{lmodern}

\usepackage[francais]{babel}
\usepackage{fancybox}
\usepackage{graphicx}

\usepackage{float}
\usepackage{xfrac}

%\usepackage[usenames, x11names]{xcolor}
\usepackage{tikz}
\usepackage{pgfplots}
\usepackage{datetime}



%-----  Package unités -----
\usepackage{siunitx}
\sisetup{locale = FR,detect-all,per-mode = symbol}

%\usepackage{mathptmx}
%\usepackage{fouriernc}
%\usepackage{newcent}
%\usepackage[mathcal,mathbf]{euler}

%\usepackage{palatino}
%\usepackage{newcent}
% \usepackage[mathcal,mathbf]{euler}



% \usepackage{hyperref}
% \hypersetup{colorlinks=true, linkcolor=blue, urlcolor=blue,
% pdftitle={Exo7 - Exercices de mathématiques}, pdfauthor={Exo7}}


%section
% \usepackage{sectsty}
% \allsectionsfont{\bf}
%\sectionfont{\color{Tomato3}\upshape\selectfont}
%\subsectionfont{\color{Tomato4}\upshape\selectfont}

%----- Ensembles : entiers, reels, complexes -----
\newcommand{\Nn}{\mathbb{N}} \newcommand{\N}{\mathbb{N}}
\newcommand{\Zz}{\mathbb{Z}} \newcommand{\Z}{\mathbb{Z}}
\newcommand{\Qq}{\mathbb{Q}} \newcommand{\Q}{\mathbb{Q}}
\newcommand{\Rr}{\mathbb{R}} \newcommand{\R}{\mathbb{R}}
\newcommand{\Cc}{\mathbb{C}} 
\newcommand{\Kk}{\mathbb{K}} \newcommand{\K}{\mathbb{K}}

%----- Modifications de symboles -----
\renewcommand{\epsilon}{\varepsilon}
\renewcommand{\Re}{\mathop{\text{Re}}\nolimits}
\renewcommand{\Im}{\mathop{\text{Im}}\nolimits}
%\newcommand{\llbracket}{\left[\kern-0.15em\left[}
%\newcommand{\rrbracket}{\right]\kern-0.15em\right]}

\renewcommand{\ge}{\geqslant}
\renewcommand{\geq}{\geqslant}
\renewcommand{\le}{\leqslant}
\renewcommand{\leq}{\leqslant}
\renewcommand{\epsilon}{\varepsilon}

%----- Fonctions usuelles -----
\newcommand{\ch}{\mathop{\text{ch}}\nolimits}
\newcommand{\sh}{\mathop{\text{sh}}\nolimits}
\renewcommand{\tanh}{\mathop{\text{th}}\nolimits}
\newcommand{\cotan}{\mathop{\text{cotan}}\nolimits}
\newcommand{\Arcsin}{\mathop{\text{arcsin}}\nolimits}
\newcommand{\Arccos}{\mathop{\text{arccos}}\nolimits}
\newcommand{\Arctan}{\mathop{\text{arctan}}\nolimits}
\newcommand{\Argsh}{\mathop{\text{argsh}}\nolimits}
\newcommand{\Argch}{\mathop{\text{argch}}\nolimits}
\newcommand{\Argth}{\mathop{\text{argth}}\nolimits}
\newcommand{\pgcd}{\mathop{\text{pgcd}}\nolimits} 


%----- Commandes divers ------
\newcommand{\ii}{\mathrm{i}}
\newcommand{\dd}{\text{d}}
\newcommand{\id}{\mathop{\text{id}}\nolimits}
\newcommand{\Ker}{\mathop{\text{Ker}}\nolimits}
\newcommand{\Card}{\mathop{\text{Card}}\nolimits}
\newcommand{\Vect}{\mathop{\text{Vect}}\nolimits}
\newcommand{\Mat}{\mathop{\text{Mat}}\nolimits}
\newcommand{\rg}{\mathop{\text{rg}}\nolimits}
\newcommand{\tr}{\mathop{\text{tr}}\nolimits}


%----- Structure des exercices ------

\newtheoremstyle{styleexo}% name
{2ex}% Space above
{3ex}% Space below
{}% Body font
{}% Indent amount 1
{\bfseries} % Theorem head font
{}% Punctuation after theorem head
{\newline}% Space after theorem head 2
{}% Theorem head spec (can be left empty, meaning ‘normal’)

%\theoremstyle{styleexo}
\newtheorem{exo}{Exercice}
\newtheorem{ind}{Indications}
\newtheorem{cor}{Correction}


\newcommand{\exercice}[1]{} \newcommand{\finexercice}{}
%\newcommand{\exercice}[1]{{\tiny\texttt{#1}}\vspace{-2ex}} % pour afficher le numero absolu, l'auteur...
\newcommand{\enonce}{\begin{exo}} \newcommand{\finenonce}{\end{exo}}
\newcommand{\indication}{\begin{ind}} \newcommand{\finindication}{\end{ind}}
\newcommand{\correction}{\begin{cor}} \newcommand{\fincorrection}{\end{cor}}

\newcommand{\noindication}{\stepcounter{ind}}
\newcommand{\nocorrection}{\stepcounter{cor}}

\newcommand{\fiche}[1]{} \newcommand{\finfiche}{}
\newcommand{\titre}[1]{\centerline{\large \bf #1}}
\newcommand{\addcommand}[1]{}
\newcommand{\video}[1]{}

% Marge
\newcommand{\mymargin}[1]{\marginpar{{\small #1}}}

\def\noqed{\renewcommand{\qedsymbol}{}}


%----- Presentation ------
\setlength{\parindent}{0cm}

%\newcommand{\ExoSept}{\href{http://exo7.emath.fr}{\textbf{\textsf{Exo7}}}}

\definecolor{myred}{rgb}{0.93,0.26,0}
\definecolor{myorange}{rgb}{0.97,0.58,0}
\definecolor{myyellow}{rgb}{1,0.86,0}

\newcommand{\LogoExoSept}[1]{  % input : echelle
{\usefont{U}{cmss}{bx}{n}
\begin{tikzpicture}[scale=0.1*#1,transform shape]
  \fill[color=myorange] (0,0)--(4,0)--(4,-4)--(0,-4)--cycle;
  \fill[color=myred] (0,0)--(0,3)--(-3,3)--(-3,0)--cycle;
  \fill[color=myyellow] (4,0)--(7,4)--(3,7)--(0,3)--cycle;
  \node[scale=5] at (3.5,3.5) {Exo7};
\end{tikzpicture}}
}


\newcommand{\debutmontitre}{
  \author{} \date{} 
  \thispagestyle{empty}
  \hspace*{-10ex}
  \begin{minipage}{\textwidth}
    \titlepage  
  \vspace*{-2.5cm}
  \begin{center}
    \LogoExoSept{2.5}
  \end{center}
  \end{minipage}

  \vspace*{-0cm}
  
  % Astuce pour que le background ne soit pas discrétisé lors de la conversion pdf -> png
\begin{tikzpicture}
        \fill[opacity=0,green!60!black] (0,0)--++(0,0)--++(0,0)--++(0,0)--cycle; 
\end{tikzpicture}

% toc S'affiche trop tot :
% \tableofcontents[hideallsubsections, pausesections]
}

\newcommand{\finmontitre}{
  \end{frame}
  \setcounter{framenumber}{0}
} % ne marche pas pour une raison obscure

%----- Commandes supplementaires ------

% \usepackage[landscape]{geometry}
% \geometry{top=1cm, bottom=3cm, left=2cm, right=10cm, marginparsep=1cm
% }
% \usepackage[a4paper]{geometry}
% \geometry{top=2cm, bottom=2cm, left=2cm, right=2cm, marginparsep=1cm
% }

%\usepackage{standalone}


% New command Arnaud -- november 2011
\setbeamersize{text margin left=24ex}
% si vous modifier cette valeur il faut aussi
% modifier le decalage du titre pour compenser
% (ex : ici =+10ex, titre =-5ex

\theoremstyle{definition}
%\newtheorem{proposition}{Proposition}
%\newtheorem{exemple}{Exemple}
%\newtheorem{theoreme}{Théorème}
%\newtheorem{lemme}{Lemme}
%\newtheorem{corollaire}{Corollaire}
%\newtheorem*{remarque*}{Remarque}
%\newtheorem*{miniexercice}{Mini-exercices}
%\newtheorem{definition}{Définition}

% Commande tikz
\usetikzlibrary{calc}
\usetikzlibrary{patterns,arrows}
\usetikzlibrary{matrix}
\usetikzlibrary{fadings} 

%definition d'un terme
\newcommand{\defi}[1]{{\color{myorange}\textbf{\emph{#1}}}}
\newcommand{\evidence}[1]{{\color{blue}\textbf{\emph{#1}}}}
\newcommand{\assertion}[1]{\emph{\og#1\fg}}  % pour chapitre logique
%\renewcommand{\contentsname}{Sommaire}
\renewcommand{\contentsname}{}
\setcounter{tocdepth}{2}



%------ Figures ------

\def\myscale{1} % par défaut 
\newcommand{\myfigure}[2]{  % entrée : echelle, fichier figure
\def\myscale{#1}
\begin{center}
\footnotesize
{#2}
\end{center}}


%------ Encadrement ------

\usepackage{fancybox}


\newcommand{\mybox}[1]{
\setlength{\fboxsep}{7pt}
\begin{center}
\shadowbox{#1}
\end{center}}

\newcommand{\myboxinline}[1]{
\setlength{\fboxsep}{5pt}
\raisebox{-10pt}{
\shadowbox{#1}
}
}

%--------------- Commande beamer---------------
\newcommand{\beameronly}[1]{#1} % permet de mettre des pause dans beamer pas dans poly


\setbeamertemplate{navigation symbols}{}
\setbeamertemplate{footline}  % tiré du fichier beamerouterinfolines.sty
{
  \leavevmode%
  \hbox{%
  \begin{beamercolorbox}[wd=.333333\paperwidth,ht=2.25ex,dp=1ex,center]{author in head/foot}%
    % \usebeamerfont{author in head/foot}\insertshortauthor%~~(\insertshortinstitute)
    \usebeamerfont{section in head/foot}{\bf\insertshorttitle}
  \end{beamercolorbox}%
  \begin{beamercolorbox}[wd=.333333\paperwidth,ht=2.25ex,dp=1ex,center]{title in head/foot}%
    \usebeamerfont{section in head/foot}{\bf\insertsectionhead}
  \end{beamercolorbox}%
  \begin{beamercolorbox}[wd=.333333\paperwidth,ht=2.25ex,dp=1ex,right]{date in head/foot}%
    % \usebeamerfont{date in head/foot}\insertshortdate{}\hspace*{2em}
    \insertframenumber{} / \inserttotalframenumber\hspace*{2ex} 
  \end{beamercolorbox}}%
  \vskip0pt%
}


\definecolor{mygrey}{rgb}{0.5,0.5,0.5}
\setlength{\parindent}{0cm}
%\DeclareTextFontCommand{\helvetica}{\fontfamily{phv}\selectfont}

% background beamer
\definecolor{couleurhaut}{rgb}{0.85,0.9,1}  % creme
\definecolor{couleurmilieu}{rgb}{1,1,1}  % vert pale
\definecolor{couleurbas}{rgb}{0.85,0.9,1}  % blanc
\setbeamertemplate{background canvas}[vertical shading]%
[top=couleurhaut,middle=couleurmilieu,midpoint=0.4,bottom=couleurbas] 
%[top=fondtitre!05,bottom=fondtitre!60]



\makeatletter
\setbeamertemplate{theorem begin}
{%
  \begin{\inserttheoremblockenv}
  {%
    \inserttheoremheadfont
    \inserttheoremname
    \inserttheoremnumber
    \ifx\inserttheoremaddition\@empty\else\ (\inserttheoremaddition)\fi%
    \inserttheorempunctuation
  }%
}
\setbeamertemplate{theorem end}{\end{\inserttheoremblockenv}}

\newenvironment{theoreme}[1][]{%
   \setbeamercolor{block title}{fg=structure,bg=structure!40}
   \setbeamercolor{block body}{fg=black,bg=structure!10}
   \begin{block}{{\bf Th\'eor\`eme }#1}
}{%
   \end{block}%
}


\newenvironment{proposition}[1][]{%
   \setbeamercolor{block title}{fg=structure,bg=structure!40}
   \setbeamercolor{block body}{fg=black,bg=structure!10}
   \begin{block}{{\bf Proposition }#1}
}{%
   \end{block}%
}

\newenvironment{corollaire}[1][]{%
   \setbeamercolor{block title}{fg=structure,bg=structure!40}
   \setbeamercolor{block body}{fg=black,bg=structure!10}
   \begin{block}{{\bf Corollaire }#1}
}{%
   \end{block}%
}

\newenvironment{mydefinition}[1][]{%
   \setbeamercolor{block title}{fg=structure,bg=structure!40}
   \setbeamercolor{block body}{fg=black,bg=structure!10}
   \begin{block}{{\bf Définition} #1}
}{%
   \end{block}%
}

\newenvironment{lemme}[0]{%
   \setbeamercolor{block title}{fg=structure,bg=structure!40}
   \setbeamercolor{block body}{fg=black,bg=structure!10}
   \begin{block}{\bf Lemme}
}{%
   \end{block}%
}

\newenvironment{remarque}[1][]{%
   \setbeamercolor{block title}{fg=black,bg=structure!20}
   \setbeamercolor{block body}{fg=black,bg=structure!5}
   \begin{block}{Remarque #1}
}{%
   \end{block}%
}


\newenvironment{exemple}[1][]{%
   \setbeamercolor{block title}{fg=black,bg=structure!20}
   \setbeamercolor{block body}{fg=black,bg=structure!5}
   \begin{block}{{\bf Exemple }#1}
}{%
   \end{block}%
}


\newenvironment{miniexercice}[0]{%
   \setbeamercolor{block title}{fg=structure,bg=structure!20}
   \setbeamercolor{block body}{fg=black,bg=structure!5}
   \begin{block}{Mini-exercices}
}{%
   \end{block}%
}


\newenvironment{tp}[0]{%
   \setbeamercolor{block title}{fg=structure,bg=structure!40}
   \setbeamercolor{block body}{fg=black,bg=structure!10}
   \begin{block}{\bf Travaux pratiques}
}{%
   \end{block}%
}
\newenvironment{exercicecours}[1][]{%
   \setbeamercolor{block title}{fg=structure,bg=structure!40}
   \setbeamercolor{block body}{fg=black,bg=structure!10}
   \begin{block}{{\bf Exercice }#1}
}{%
   \end{block}%
}
\newenvironment{algo}[1][]{%
   \setbeamercolor{block title}{fg=structure,bg=structure!40}
   \setbeamercolor{block body}{fg=black,bg=structure!10}
   \begin{block}{{\bf Algorithme}\hfill{\color{gray}\texttt{#1}}}
}{%
   \end{block}%
}


\setbeamertemplate{proof begin}{
   \setbeamercolor{block title}{fg=black,bg=structure!20}
   \setbeamercolor{block body}{fg=black,bg=structure!5}
   \begin{block}{{\footnotesize Démonstration}}
   \footnotesize
   \smallskip}
\setbeamertemplate{proof end}{%
   \end{block}}
\setbeamertemplate{qed symbol}{\openbox}


\makeatother
\usecolortheme[RGB={205,0,0}]{structure}

%%%%%%%%%%%%%%%%%%%%%%%%%%%%%%%%%%%%%%%%%%%%%%%%%%%%%%%%%%%%%
%%%%%%%%%%%%%%%%%%%%%%%%%%%%%%%%%%%%%%%%%%%%%%%%%%%%%%%%%%%%%

\begin{document}


\title{{\bf Espaces vectoriels}}
\subtitle{Sous-espace vectoriel (fin)}

\begin{frame}
  
  \debutmontitre

  \pause

{\footnotesize
\hfill
\setbeamercovered{transparent=50}
\begin{minipage}{0.6\textwidth}
  \begin{itemize}
    \item<3-> Somme de deux sous-espaces vectoriels
    \item<4-> Sous-espaces vectoriels supplémentaires
    \item<5-> Sous-espace engendré  
  \end{itemize}
\end{minipage}
}

\end{frame}

\setcounter{framenumber}{0}



%%%%%%%%%%%%%%%%%%%%%%%%%%%%%%%%%%%%%%%%%%%%%%%%%%%%%%%%%%%%%%%%
\section{Somme de deux sous-espaces vectoriels}

\begin{frame}
Soient $F$ et $G$ deux sous-espaces vectoriels d'un $\Kk$-espace vectoriel $E$
\begin{mydefinition}
L'ensemble de tous les éléments $u+v$, où $u$ est un élément de 
$F$ et $v$ un élément de $G$, est appelé \defi{somme} des sous-espaces vectoriels 
$F$ et $G$
\end{mydefinition}

\pause

%\vspace*{-1ex}

\begin{minipage}{0.6\textwidth}
\mybox{$F+G=\big\{u+v \mid u \in F, v \in G \big\}$}  
\end{minipage}
\pause
\begin{minipage}{0.39\textwidth}
\myfigure{0.7}{
\tikzinput{fig_ev06} 
}
\end{minipage}

\pause

\begin{proposition}
\begin{enumerate}
  \item $F+G$ est un sous-espace vectoriel de $E$
  \pause
  \item $F+G$ est le plus petit sous-espace vectoriel contenant $F$ et $G$
\end{enumerate}
\end{proposition}
\end{frame}


\begin{frame}
 

\begin{exemple}

$$F=\big\{(x,y,z) \in \Rr^3\mid y=z=0\big\} 
\ \ \text{et} \ \  
G=\big\{(x,y,z) \in \Rr^3 \mid x=z=0\big\}$$
\pause
\begin{minipage}{0.39\textwidth}
\myfigure{0.7}{
\tikzinput{fig_ev07} 
} 
\end{minipage}
\pause
\begin{minipage}{0.6\textwidth}
\begin{itemize}
  \item Un élément $w$ de $F+G$ s'écrit $w=u+v$ où $u \in F$ et $v \in G$
  \pause
  \item Comme $u\in F$, $u=(x,0,0)$
  \pause
  \item Comme $v \in G$, $v=(0,y,0)$
  \pause
  \item Donc $w=(x,y,0)$
  \pause
  \item $F+G=\big\{(x,y,z) \in \Rr^3\mid z=0\big\}$
\end{itemize}  
\end{minipage}


\end{exemple}
\end{frame}


\begin{frame}
\begin{exemple}


$$F=\big\{(x,y,z) \in \Rr^3\mid x=0\big\}
\quad \text{ et } \quad 
G=\big\{(x,y,z) \in \Rr^3 \mid y=0\big\}$$


\hfill\hfill\begin{minipage}{0.3\textwidth}
\myfigure{0.7}{
\tikzinput{fig_ev08} 
}              
\end{minipage}
\vspace*{-12ex}

\pause

\begin{itemize}\setlength{\itemsep}{7pt}
  \item Montrons que $F+G=\Rr^3$
\pause  
  \item Soit  $w=(x,y,z) \in \Rr^3$
\pause  
  \item $w=(x,y,z)=(0,y,z)+(x,0,0)$, avec
$(0,y,z) \in F$ et $(x,0,0) \in G$
\pause  
  \item Donc $w \in F+G$
\pause  
  \item Pas unicité : $(1,2,3)=(0,2,3)+ (1,0,0)= (0,2,0)+(1,0,3)$
\end{itemize}
\end{exemple}
\end{frame}



%%%%%%%%%%%%%%%%%%%%%%%%%%%%%%%%%%%%%%%%%%%%%%%%%%%%%%%%%%%%%%%%
\section{Sous-espaces vectoriels supplémentaires}

\begin{frame}
Soient $F$ et $G$ deux sous-espaces vectoriels de $E$
\begin{mydefinition}
$F$ et $G$ sont en \defi{somme directe} dans $E$ si 
\begin{itemize}
  \item $F \cap G = \{ 0_E \}$
  \item $F+G=E$
\end{itemize}
\end{mydefinition}
\pause
On note alors $F \oplus G=E$

\pause
$F$ et $G$ sont des sous-espaces vectoriels \defi{supplémentaires} dans $E$

\pause
\begin{proposition}
\label{prop:directeunique}
$F$ et $G$ sont supplémentaires dans $E$ si et seulement si tout 
élément de $E$ s'écrit d'une manière \evidence{unique} 
comme la somme d'un élément de $F$ et d'un élément de $G$
\end{proposition}

\pause
\medskip

{\small \'Ecriture unique : si $\left\{\begin{array}{l}w=u+v\\w=u'+v' \end{array}\right.$  avec 
$\left\{\begin{array}{l}u\in F, v\in G\\ u'\in F, v'\in G\end{array}\right.$
alors $\left\{\begin{array}{l}u=u'\\v=v'\end{array}\right.$}
\end{frame}


\begin{frame}
\begin{exemple}
\begin{enumerate}
  \item $F = \big\{ (x,0) \in \Rr^2 \mid x \in \Rr \big\}$
et $G = \big\{ (0,y) \in \Rr^2 \mid y \in \Rr \big\}$

\hfill\hfill\begin{minipage}{0.3\textwidth}
\myfigure{0.8}{
\tikzinput{fig_ev09} 
}   
\end{minipage}
\vspace*{-8ex}
\pause
  \begin{itemize}
    \item $F \oplus G = \Rr^2$ ?
\pause    
    \item $F\cap G = \{ (0,0) \}$
\pause    
    \item $F+G = \Rr^2$ car $(x,y)=(x,0)+(0,y)$
\pause    
    \item Conclusion : $F \oplus G = \Rr^2$
 \pause   
    \item Autre méthode : la décomposition $(x,y)=(x,0)+(0,y)$ est unique
  \end{itemize}


 
\pause

  \item Gardons $F$ et notons $G' = \big\{ (x,x) \in \Rr^2 \mid x \in \Rr \big\}$.
Montrons que l'on a aussi $F\oplus G'=\Rr^2$ 
\pause
  \begin{itemize}
    \item $F \cap G' =\{(0,0)\}$   
\pause    
    \item $F+G' = \Rr^2$ : $(x,y) = (x-y,0) + (y,y)$
  \end{itemize} 
  
\pause  
  \item Deux droites distinctes du plan 
  passant par l'origine forment des sous-espaces supplémentaires
\end{enumerate}
\end{exemple}
\end{frame}


\begin{frame}
\begin{exemple}
\begin{minipage}{0.6\textwidth}
$F=\big\{ (x,y,z) \in \Rr^3\mid x-y-z=0\big\}$ et  
$G=\big\{(x,y,z) \in \Rr^3 \mid y=z=0\big\}$  
\end{minipage}
\begin{minipage}{0.39\textwidth}
\myfigure{0.7}{
\tikzinput{fig_ev10} 
}  
\end{minipage}




\pause


\begin{itemize}
  \item $F\cap G=\{0\}$ \pause
  :  si $u=(x,y,z) \in F \cap G$ 
alors  $x-y-z=0$ (car $u \in F$) et  $y=z=0$ (car $u \in G$), donc  $u=(0,0,0)$

\pause
  \item Montrons $F+G=\Rr^3$
\pause  
  \begin{itemize}
    \item Soit $u=(x,y,z) \in \Rr^3$
    \pause
    \item On cherche $v \in F$ et $w \in G$ tels que $u=v+w$
    \pause
    \item $v=(y_1+z_1, y_1, z_1)$ et $w=(x_2,0,0)$
    \pause
    \item Donc $(x,y,z)= (y_1+z_1+x_2,y_1,z_1)$
    \pause
    \item Ainsi $y_1=y$, $z_1=z$, $x_2=x-y-z$
    \pause
    \item $(x,y,z)=(y+z,y,z)+ (x-y-z, 0,0) \in F + G$
  \end{itemize}
\pause
\end{itemize}
Conclusion : $F \oplus G=\Rr^3$
\end{exemple}
\end{frame}


\begin{frame}
  
\begin{exemple}
\label{ex:evsomme} 
$E = \mathcal{F}(\Rr,\Rr)$, $\mathcal{P}$ fonctions paires,  $\mathcal{I}$ fonctions impaires 

Montrons $\mathcal{P}\oplus\mathcal{I}=\mathcal{F}(\Rr,\Rr)$

\pause

\begin{enumerate}
  \item Montrons $\mathcal{P} \cap \mathcal{I} = \{ 0_{\mathcal{F}(\Rr,\Rr)} \}$
  \pause
  \begin{itemize}\setlength{\itemsep}{4pt}
    \item Soit $f \in \mathcal{P} \cap \mathcal{I}$
    \pause
    \item Soit $x \in \Rr$. $f(-x)=f(x)$ et $f(-x)=-f(x)$ alors $f(x)=-f(x)$, ce qui implique $f(x)=0$
    \pause
    \item $f$ est la fonction nulle
  \end{itemize}
 
   \medskip
  \pause 
  \item Montrons $\mathcal{P}+\mathcal{I}=\mathcal{F}(\Rr,\Rr)$
  \pause
  \begin{itemize}\setlength{\itemsep}{4pt}
    \item Soit $f \in \mathcal{F}(\Rr,\Rr)$
    \pause
    \item $g(x)=\frac{f(x)+f(-x)}2$, $g$ est paire
    \pause
    \item $h(x)=\frac{f(x)-f(-x)}2$, $h$ est impaire
    \pause
    \item $f(x)=g(x)+h(x)$ 
  \end{itemize}
 
\end{enumerate}

\end{exemple}
\end{frame}



%%%%%%%%%%%%%%%%%%%%%%%%%%%%%%%%%%%%%%%%%%%%%%%%%%%%%%%%%%%%%%%%
\section{Sous-espace engendré}

\begin{frame}
Soit  $\{v_1, \dots , v_n\}$  des vecteurs d'un 
$\Kk$-espace vectoriel $E$
\begin{theoreme}
\begin{itemize}
  \item L'ensemble des combinaisons linéaires des vecteurs 
  $\{v_1, \dots , v_n\}$ est un sous-espace vectoriel de $E$
  \pause  
  \item C'est le plus petit sous-espace vectoriel de $E$ 
  contenant les vecteurs  $v_1, \ldots , v_n$ 
\end{itemize}
 
\end{theoreme}

\medskip
\pause

C'est le \defi{sous-espace engendré par $v_1, \ldots , v_n$}, noté $\Vect (v_1, \ldots , v_n )$

\pause

 \mybox{$u \in \Vect( v_1, \dots , v_n )\iff
 \exists \ \lambda_1, \dots , \lambda_n \in \Kk \ \ 
 u=\lambda_1v_1+ \dots+\lambda_nv_n$}


\end{frame}


\begin{frame} 

\small
\begin{enumerate}\setlength{\itemsep}{4pt}
  \item \defi{Droite vectorielle} $\Vect (u) =\{ \lambda u \mid \lambda \in \Kk \} = \Kk u$ \quad ($u\neq 0_E$)

\myfigure{0.5}{
\tikzinput{fig_ev11-1}
\pause
\qquad \qquad
\tikzinput{fig_ev11-2} 
}


\vspace*{-4ex} 
  \item $\Vect (u,v) = \big\{ \lambda u + \mu v \mid \lambda, \mu \in \Kk \big\}$
  
  Si $u$ et $v$ ne sont pas colinéaires, c'est un \defi{plan vectoriel}
 
  \pause
 
  \item $u = \left(\begin{smallmatrix}1 \\ 1 \\ 1 \end{smallmatrix}\right)$,
  $v = \left(\begin{smallmatrix}1 \\ 2 \\ 3 \end{smallmatrix}\right) \in \Rr^3$.  
  Déterminons $\mathcal{P} = \Vect (u,v)$ 
  \pause
%\vspace*{-2ex}
$$ \begin{array}{l}
 \left(\begin{smallmatrix}x \\ y \\ z \end{smallmatrix}\right) \in \Vect (u,v)
 \pause
 \iff  \left(\begin{smallmatrix}x \\ y \\ z \end{smallmatrix}\right) 
 = \lambda u + \mu v  \quad \text{pour certains $\lambda,\mu \in \Rr$} \\
  \pause
  \iff  \left(\begin{smallmatrix}x \\ y \\ z \end{smallmatrix}\right) = 
 \lambda\left(\begin{smallmatrix}1 \\ 1 \\ 1 \end{smallmatrix}\right) + \mu \left(\begin{smallmatrix}1 \\ 2 \\ 3 \end{smallmatrix}\right)
  \pause
  \iff  
 \left\{
 \begin{array}{rcl}
   x & = & \lambda + \mu \\
   y & = & \lambda + 2 \mu \\
   z & = & \lambda + 3 \mu \\
 \end{array}\right. \\
 \end{array} $$
 \pause
\'Equation cartésienne : $(x-2y+z=0)$

  \pause
 
\item $E = \mathcal{F}(\Rr,\Rr)$, $f_0(x)=1$, $f_1(x)=x$ et $f_2(x)=x^2$

 \pause
$\Vect (f_0, f_1, f_2) =\big\{ f \mid f(x) = ax^2+bx+c \big\} = \Rr_2[x]$

\end{enumerate}

\end{frame}



%%%%%%%%%%%%%%%%%%%%%%%%%%%%%%%%%%%%%%%%%%%%%%%%%%%%%%%%%%%%%%%%
\section{Mini-exercices}

\begin{frame}

\begin{miniexercice}
\begin{enumerate}
  \item Trouver des sous-espaces vectoriels distincts $F$ et $G$ de $\Rr^3$
  tels que
  \begin{enumerate}
    \item $F+G = \Rr^3$ et $F\cap G \neq \{0\}$ ; 
    \item $F+G \neq \Rr^3$ et $F\cap G = \{0\}$ ;
    \item $F+G = \Rr^3$ et $F\cap G = \{0\}$ ;
    \item $F+G \neq \Rr^3$ et $F\cap G \neq \{0\}$.
  \end{enumerate}

  \item Soient $F = \big\{ (x,y,z) \in \Rr^3 \mid x+y+z = 0\big\}$ et 
  $G = \Vect \big\{ (1,1,1) \big\} \subset \Rr^3$.
  \begin{enumerate}
    \item Montrer que $F$ est un espace vectoriel. Trouver deux vecteurs $u,v$ 
    tels que $F = \Vect(u,v)$.
    \item Calculer $F \cap G$ et montrer que $F+G = \Rr^3$. Que conclure ? 
  \end{enumerate}
  

  \item Soient $A=\left(\begin{smallmatrix}1 & 0 \\ 0 & 0 \end{smallmatrix}\right)$,
  $B=\left(\begin{smallmatrix}0 & 0 \\ 0 & 1\end{smallmatrix}\right)$,
  $C=\left(\begin{smallmatrix}0 & 1 \\ 0 & 0\end{smallmatrix}\right)$,
  $D=\left(\begin{smallmatrix}0 & 0 \\ 1 & 0\end{smallmatrix}\right)$ des matrices de $M_2(\Rr)$.
  \begin{enumerate}
    \item Quel est l'espace vectoriel $F$ engendré par $A$ et $B$ ? Idem avec $G$ engendré par $C$ et $D$.
    \item Calculer $F\cap G$. Montrer que $F+G = M_2(\Rr)$. Conclure.
  \end{enumerate}

\end{enumerate}
\end{miniexercice}

\end{frame}

\end{document}