
%%%%%%%%%%%%%%%%%% PREAMBULE %%%%%%%%%%%%%%%%%%


\documentclass[12pt]{article}

\usepackage{amsfonts,amsmath,amssymb,amsthm}
\usepackage[utf8]{inputenc}
\usepackage[T1]{fontenc}
\usepackage[francais]{babel}


% packages
\usepackage{amsfonts,amsmath,amssymb,amsthm}
\usepackage[utf8]{inputenc}
\usepackage[T1]{fontenc}
%\usepackage{lmodern}

\usepackage[francais]{babel}
\usepackage{fancybox}
\usepackage{graphicx}

\usepackage{float}

%\usepackage[usenames, x11names]{xcolor}
\usepackage{tikz}
\usepackage{datetime}

\usepackage{mathptmx}
%\usepackage{fouriernc}
%\usepackage{newcent}
\usepackage[mathcal,mathbf]{euler}

%\usepackage{palatino}
%\usepackage{newcent}


% Commande spéciale prompteur

%\usepackage{mathptmx}
%\usepackage[mathcal,mathbf]{euler}
%\usepackage{mathpple,multido}

\usepackage[a4paper]{geometry}
\geometry{top=2cm, bottom=2cm, left=1cm, right=1cm, marginparsep=1cm}

\newcommand{\change}{{\color{red}\rule{\textwidth}{1mm}\\}}

\newcounter{mydiapo}

\newcommand{\diapo}{\newpage
\hfill {\normalsize  Diapo \themydiapo \quad \texttt{[\jobname]}} \\
\stepcounter{mydiapo}}


%%%%%%% COULEURS %%%%%%%%%%

% Pour blanc sur noir :
%\pagecolor[rgb]{0.5,0.5,0.5}
% \pagecolor[rgb]{0,0,0}
% \color[rgb]{1,1,1}



%\DeclareFixedFont{\myfont}{U}{cmss}{bx}{n}{18pt}
\newcommand{\debuttexte}{
%%%%%%%%%%%%% FONTES %%%%%%%%%%%%%
\renewcommand{\baselinestretch}{1.5}
\usefont{U}{cmss}{bx}{n}
\bfseries

% Taille normale : commenter le reste !
%Taille Arnaud
%\fontsize{19}{19}\selectfont

% Taille Barbara
%\fontsize{21}{22}\selectfont

%Taille François
%\fontsize{25}{30}\selectfont

%Taille Pascal
%\fontsize{25}{30}\selectfont

%Taille Laura
%\fontsize{30}{35}\selectfont


%\myfont
%\usefont{U}{cmss}{bx}{n}

%\Huge
%\addtolength{\parskip}{\baselineskip}
}


% \usepackage{hyperref}
% \hypersetup{colorlinks=true, linkcolor=blue, urlcolor=blue,
% pdftitle={Exo7 - Exercices de mathématiques}, pdfauthor={Exo7}}


%section
% \usepackage{sectsty}
% \allsectionsfont{\bf}
%\sectionfont{\color{Tomato3}\upshape\selectfont}
%\subsectionfont{\color{Tomato4}\upshape\selectfont}

%----- Ensembles : entiers, reels, complexes -----
\newcommand{\Nn}{\mathbb{N}} \newcommand{\N}{\mathbb{N}}
\newcommand{\Zz}{\mathbb{Z}} \newcommand{\Z}{\mathbb{Z}}
\newcommand{\Qq}{\mathbb{Q}} \newcommand{\Q}{\mathbb{Q}}
\newcommand{\Rr}{\mathbb{R}} \newcommand{\R}{\mathbb{R}}
\newcommand{\Cc}{\mathbb{C}} 
\newcommand{\Kk}{\mathbb{K}} \newcommand{\K}{\mathbb{K}}

%----- Modifications de symboles -----
\renewcommand{\epsilon}{\varepsilon}
\renewcommand{\Re}{\mathop{\text{Re}}\nolimits}
\renewcommand{\Im}{\mathop{\text{Im}}\nolimits}
%\newcommand{\llbracket}{\left[\kern-0.15em\left[}
%\newcommand{\rrbracket}{\right]\kern-0.15em\right]}

\renewcommand{\ge}{\geqslant}
\renewcommand{\geq}{\geqslant}
\renewcommand{\le}{\leqslant}
\renewcommand{\leq}{\leqslant}

%----- Fonctions usuelles -----
\newcommand{\ch}{\mathop{\mathrm{ch}}\nolimits}
\newcommand{\sh}{\mathop{\mathrm{sh}}\nolimits}
\renewcommand{\tanh}{\mathop{\mathrm{th}}\nolimits}
\newcommand{\cotan}{\mathop{\mathrm{cotan}}\nolimits}
\newcommand{\Arcsin}{\mathop{\mathrm{Arcsin}}\nolimits}
\newcommand{\Arccos}{\mathop{\mathrm{Arccos}}\nolimits}
\newcommand{\Arctan}{\mathop{\mathrm{Arctan}}\nolimits}
\newcommand{\Argsh}{\mathop{\mathrm{Argsh}}\nolimits}
\newcommand{\Argch}{\mathop{\mathrm{Argch}}\nolimits}
\newcommand{\Argth}{\mathop{\mathrm{Argth}}\nolimits}
\newcommand{\pgcd}{\mathop{\mathrm{pgcd}}\nolimits} 

\newcommand{\Card}{\mathop{\text{Card}}\nolimits}
\newcommand{\Ker}{\mathop{\text{Ker}}\nolimits}
\newcommand{\id}{\mathop{\text{id}}\nolimits}
\newcommand{\ii}{\mathrm{i}}
\newcommand{\dd}{\mathrm{d}}
\newcommand{\Vect}{\mathop{\text{Vect}}\nolimits}
\newcommand{\Mat}{\mathop{\mathrm{Mat}}\nolimits}
\newcommand{\rg}{\mathop{\text{rg}}\nolimits}
\newcommand{\tr}{\mathop{\text{tr}}\nolimits}
\newcommand{\ppcm}{\mathop{\text{ppcm}}\nolimits}

%----- Structure des exercices ------

\newtheoremstyle{styleexo}% name
{2ex}% Space above
{3ex}% Space below
{}% Body font
{}% Indent amount 1
{\bfseries} % Theorem head font
{}% Punctuation after theorem head
{\newline}% Space after theorem head 2
{}% Theorem head spec (can be left empty, meaning ‘normal’)

%\theoremstyle{styleexo}
\newtheorem{exo}{Exercice}
\newtheorem{ind}{Indications}
\newtheorem{cor}{Correction}


\newcommand{\exercice}[1]{} \newcommand{\finexercice}{}
%\newcommand{\exercice}[1]{{\tiny\texttt{#1}}\vspace{-2ex}} % pour afficher le numero absolu, l'auteur...
\newcommand{\enonce}{\begin{exo}} \newcommand{\finenonce}{\end{exo}}
\newcommand{\indication}{\begin{ind}} \newcommand{\finindication}{\end{ind}}
\newcommand{\correction}{\begin{cor}} \newcommand{\fincorrection}{\end{cor}}

\newcommand{\noindication}{\stepcounter{ind}}
\newcommand{\nocorrection}{\stepcounter{cor}}

\newcommand{\fiche}[1]{} \newcommand{\finfiche}{}
\newcommand{\titre}[1]{\centerline{\large \bf #1}}
\newcommand{\addcommand}[1]{}
\newcommand{\video}[1]{}

% Marge
\newcommand{\mymargin}[1]{\marginpar{{\small #1}}}



%----- Presentation ------
\setlength{\parindent}{0cm}

%\newcommand{\ExoSept}{\href{http://exo7.emath.fr}{\textbf{\textsf{Exo7}}}}

\definecolor{myred}{rgb}{0.93,0.26,0}
\definecolor{myorange}{rgb}{0.97,0.58,0}
\definecolor{myyellow}{rgb}{1,0.86,0}

\newcommand{\LogoExoSept}[1]{  % input : echelle
{\usefont{U}{cmss}{bx}{n}
\begin{tikzpicture}[scale=0.1*#1,transform shape]
  \fill[color=myorange] (0,0)--(4,0)--(4,-4)--(0,-4)--cycle;
  \fill[color=myred] (0,0)--(0,3)--(-3,3)--(-3,0)--cycle;
  \fill[color=myyellow] (4,0)--(7,4)--(3,7)--(0,3)--cycle;
  \node[scale=5] at (3.5,3.5) {Exo7};
\end{tikzpicture}}
}



\theoremstyle{definition}
%\newtheorem{proposition}{Proposition}
%\newtheorem{exemple}{Exemple}
%\newtheorem{theoreme}{Théorème}
\newtheorem{lemme}{Lemme}
\newtheorem{corollaire}{Corollaire}
%\newtheorem*{remarque*}{Remarque}
%\newtheorem*{miniexercice}{Mini-exercices}
%\newtheorem{definition}{Définition}




%definition d'un terme
\newcommand{\defi}[1]{{\color{myorange}\textbf{\emph{#1}}}}
\newcommand{\evidence}[1]{{\color{blue}\textbf{\emph{#1}}}}



 %----- Commandes divers ------

\newcommand{\codeinline}[1]{\texttt{#1}}

%%%%%%%%%%%%%%%%%%%%%%%%%%%%%%%%%%%%%%%%%%%%%%%%%%%%%%%%%%%%%
%%%%%%%%%%%%%%%%%%%%%%%%%%%%%%%%%%%%%%%%%%%%%%%%%%%%%%%%%%%%%



\begin{document}

\debuttexte


%%%%%%%%%%%%%%%%%%%%%%%%%%%%%%%%%%%%%%%%%%%%%%%%%%%%%%%%%%%
\diapo

\change

Nous avons déjà rencontré la notion d'application linéaire dans le cas 
d'une fonction de $\R^p$ dans $\R^n$.
C'est d'ailleurs le bon moment pour voir ou revoir le chapitre 
\og L'espace vectoriel $\Rr^n$.

\change


Nous allons voir que cette notion se généralise à des espaces vectoriels quelconques. 

\change

Nous verrons quelques exemples simples

\change

ainsi que des propriétés élémentaires.

%%%%%%%%%%%%%%%%%%%%%%%%%%%%%%%%%%%%%%%%%%%%%%%%%%%%%%%%%%
\diapo




Soient $E$ et $F$ deux $\Kk$-espaces vectoriels. 
Une application $f$ de $E$ dans $F$ est une \defi{application 
linéaire} si elle satisfait aux deux conditions suivantes : 

\change

(1)  $f(u+v)=f(u)+f(v)$, pour tous $u, v \in  E$ ;

\change

(2) $f(\lambda \cdot u)=\lambda \cdot f(u)$, pour tout $u \in E$ et tout $\lambda \in \Kk$.


Autrement dit : une application est linéaire si elle 
\og respecte \fg\ les deux lois des espaces vectoriels : la loi $+$ et la loi $\cdot$.

Dans la pratique on oublie les "points" et on écrit $f(\lambda u)=\lambda f(u)$


\change

L'ensemble des applications linéaires de $E$ dans $F$ est noté $\mathcal{L}(E,F)$.

%%%%%%%%%%%%%%%%%%%%%%%%%%%%%%%%%%%%%%%%%%%%%%%%%%%%%%%%%%%
\diapo
Voyons un exemple : soit l'application $f$
de $\Rr^3$ vers $\Rr^2$
définie par $f(x,y,z) = (-2x,y+3z)$.

\change

$f$ est une application linéaire. 


\change

Vérifions les deux points :

Tout d'abord, soient  $u=(x,y,z)$ et $v=(x',y',z')$ 
deux éléments de $\Rr^3$ 
et calculons $f(u+v)$.

\change

C'est donc $f(x+x', y+y', z+z')$ par définition de $u+v$

\change

Maintenant par définition de $f$ cela vaut $\big(-2(x+x'), y+y' + 3(z+z')\big)$

\change

que l'on réécrit :
$(-2x, y+3z)+(-2x', y'+3z')$

\change

Ce qui donne exactement $f(u)+f(v)$.

\change

Passons au deuxième point : soient  $u=(x,y,z)$ un vecteur et $\lambda \in \Rr$ un scalaire
et calculons $f(\lambda \cdot u)$

\change

c'est donc $f(\lambda x, \lambda y, \lambda z)$

\change

et par définition de $f$ cela vaut $(-2\lambda x,   \lambda y + 3\lambda z)$

\change

ce qui s'écrit aussi $\lambda \cdot (-2x,y+3z)$

\change

qui est exactement $\lambda \cdot f(u)$


%%%%%%%%%%%%%%%%%%%%%%%%%%%%%%%%%%%%%%%%%%%%%%%%%%%%%%%%%%
\diapo

Toutes les applications ne sont pas des applications linéaires !

Voici un exemple simple l'application $f : \Rr \to \Rr$ définie par $f(x)=x^2$.

\change

On a $f(1)=1$ et $f(2)=4$. Donc $f(2)$ n'est pas égal à $2f(1)$. 

\change

Ce qui fait
que l'on n'a pas l'égalité $f(\lambda x)=\lambda f(x)$ pour un certain choix de $\lambda,x$.

\change

Donc $f$ n'est pas linéaire. 

\change

Notez que l'on n'a pas non plus $f(x+x')=f(x)+f(x')$ dès que $xx'\neq0$.


%%%%%%%%%%%%%%%%%%%%%%%%%%%%%%%%%%%%%%%%%%%%%%%%%%%%%%%%%%%
\diapo

Voici d'autres exemples d'applications linéaires :


Fixons une matrice $A$ à $n$ lignes et $p$ colonnes.

Et définissons l'application $f : \R^p \longrightarrow \R^n$ par 
$f(X) = AX$

Alors $f$ est une application linéaire.

\change

On continue avec deux applications linéaires un peu spéciales.

D'abord l'\defi{application nulle}, notée $0_{\mathcal{L}(E,F)}$ : 
qui est simplement l'application définie par $f(u) = 0_F$ pour tout vecteur $u$.
 
\change

Puis l'\defi{application identité}, notée $\id_E$ :
est une application de $E$ dans $E$ définie par
$f(u) = u$ pour tout vecteur $u$.



%%%%%%%%%%%%%%%%%%%%%%%%%%%%%%%%%%%%%%%%%%%%%%%%%%%%%%%%%%
\diapo


$E$ et $F$ désignent toujours deux $\Kk$-espaces vectoriels et $f$ est une 
application linéaire de $E$ dans $F$.

D'une part $f(0_{E})=0_{F}$, (le vecteur nul est envoyé sur le vecteur nul)

d'autre part $f(-u)=-f(u)$, (l'image du symétrique est le symétrique de l'image)


Pour la preuve il suffit d'appliquer la définition de la linéarité avec ici $\lambda =0$, 

et là avec $\lambda =-1$.


%%%%%%%%%%%%%%%%%%%%%%%%%%%%%%%%%%%%%%%%%%%%%%%%%%%%%%%%%%%
\diapo

Pour démontrer qu'une application est linéaire, 
on peut aussi utiliser une propriété plus \og concentrée \fg,  
donnée par la caractérisation suivante :


Soient $E$ et $F$ deux $\Kk$-espaces vectoriels et 
$f$ une application de $E$ dans $F$.


L'application $f$ est linéaire si et seulement si, 
pour tous vecteurs $u$ et $v$ et pour tous scalaires  
$\lambda$ et $\mu$ :
$f(\lambda u + \mu v)=\lambda f(u)+\mu f(v)$.

\change

Plus généralement, une application linéaire $f$ préserve les combinaisons linéaires :
pour tous scalaires $\lambda_1, \dots , \lambda_n$ et tous vecteurs $v_1, \dots , v_n \in E$, on a 
$$f(\lambda_1 v_1 +\dots + \lambda_n v_n) = \lambda_1 f(v_1) + \dots + \lambda_n
f(v_n).$$



%%%%%%%%%%%%%%%%%%%%%%%%%%%%%%%%%%%%%%%%%%%%%%%%%%%%%%%%%%
\diapo

On termine avec du vocabulaire pour nommer les applications linéaires et leurs ensembles.


Soient $E$ et $F$ deux $\Kk$-espaces vectoriels.

Une application linéaire d'un espace vectoriel $E$ dans un espace vectoriel $F$ est 
  aussi appelée \defi{morphisme} ou même parfois \defi{homomorphisme} d'espaces vectoriels.

  Pour l'ensemble de toutes les applications linéaires de $E$ dans $F$, on le note $\mathcal{L}(E,F)$.
 
  \change
  
  Il existe un vocabulaire spécifique dans le cas où l'espace d'arrivée est le même que l'espace de départ.
   
  Une application linéaire de $E$ dans $E$ s'appelle \defi{endomorphisme} de $E$.
  
  Et pour l'ensemble des applications linéaire de $E$ dans $E$  on note simplement $\mathcal{L}(E)$.

%%%%%%%%%%%%%%%%%%%%%%%%%%%%%%%%%%%%%%%%%%%%%%%%%%%%%%%%%%%
\diapo

Voici quelques exemples simples pour vous faire patienter avant d'aborder des exemples plus compliqués.

\end{document}
