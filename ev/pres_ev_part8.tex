
%%%%%%%%%%%%%%%%%% PREAMBULE %%%%%%%%%%%%%%%%%%

\documentclass[aspectratio=169,utf8]{beamer}
%\documentclass[aspectratio=169,handout]{beamer}

\usetheme{Boadilla}
%\usecolortheme{seahorse}
\usecolortheme[RGB={245,66,24}]{structure}
\useoutertheme{infolines}

% packages
\usepackage{amsfonts,amsmath,amssymb,amsthm}
\usepackage[utf8]{inputenc}
\usepackage[T1]{fontenc}
\usepackage{lmodern}

\usepackage[francais]{babel}
\usepackage{fancybox}
\usepackage{graphicx}

\usepackage{float}
\usepackage{xfrac}

%\usepackage[usenames, x11names]{xcolor}
\usepackage{tikz}
\usepackage{pgfplots}
\usepackage{datetime}



%-----  Package unités -----
\usepackage{siunitx}
\sisetup{locale = FR,detect-all,per-mode = symbol}

%\usepackage{mathptmx}
%\usepackage{fouriernc}
%\usepackage{newcent}
%\usepackage[mathcal,mathbf]{euler}

%\usepackage{palatino}
%\usepackage{newcent}
% \usepackage[mathcal,mathbf]{euler}



% \usepackage{hyperref}
% \hypersetup{colorlinks=true, linkcolor=blue, urlcolor=blue,
% pdftitle={Exo7 - Exercices de mathématiques}, pdfauthor={Exo7}}


%section
% \usepackage{sectsty}
% \allsectionsfont{\bf}
%\sectionfont{\color{Tomato3}\upshape\selectfont}
%\subsectionfont{\color{Tomato4}\upshape\selectfont}

%----- Ensembles : entiers, reels, complexes -----
\newcommand{\Nn}{\mathbb{N}} \newcommand{\N}{\mathbb{N}}
\newcommand{\Zz}{\mathbb{Z}} \newcommand{\Z}{\mathbb{Z}}
\newcommand{\Qq}{\mathbb{Q}} \newcommand{\Q}{\mathbb{Q}}
\newcommand{\Rr}{\mathbb{R}} \newcommand{\R}{\mathbb{R}}
\newcommand{\Cc}{\mathbb{C}} 
\newcommand{\Kk}{\mathbb{K}} \newcommand{\K}{\mathbb{K}}

%----- Modifications de symboles -----
\renewcommand{\epsilon}{\varepsilon}
\renewcommand{\Re}{\mathop{\text{Re}}\nolimits}
\renewcommand{\Im}{\mathop{\text{Im}}\nolimits}
%\newcommand{\llbracket}{\left[\kern-0.15em\left[}
%\newcommand{\rrbracket}{\right]\kern-0.15em\right]}

\renewcommand{\ge}{\geqslant}
\renewcommand{\geq}{\geqslant}
\renewcommand{\le}{\leqslant}
\renewcommand{\leq}{\leqslant}
\renewcommand{\epsilon}{\varepsilon}

%----- Fonctions usuelles -----
\newcommand{\ch}{\mathop{\text{ch}}\nolimits}
\newcommand{\sh}{\mathop{\text{sh}}\nolimits}
\renewcommand{\tanh}{\mathop{\text{th}}\nolimits}
\newcommand{\cotan}{\mathop{\text{cotan}}\nolimits}
\newcommand{\Arcsin}{\mathop{\text{arcsin}}\nolimits}
\newcommand{\Arccos}{\mathop{\text{arccos}}\nolimits}
\newcommand{\Arctan}{\mathop{\text{arctan}}\nolimits}
\newcommand{\Argsh}{\mathop{\text{argsh}}\nolimits}
\newcommand{\Argch}{\mathop{\text{argch}}\nolimits}
\newcommand{\Argth}{\mathop{\text{argth}}\nolimits}
\newcommand{\pgcd}{\mathop{\text{pgcd}}\nolimits} 


%----- Commandes divers ------
\newcommand{\ii}{\mathrm{i}}
\newcommand{\dd}{\text{d}}
\newcommand{\id}{\mathop{\text{id}}\nolimits}
\newcommand{\Ker}{\mathop{\text{Ker}}\nolimits}
\newcommand{\Card}{\mathop{\text{Card}}\nolimits}
\newcommand{\Vect}{\mathop{\text{Vect}}\nolimits}
\newcommand{\Mat}{\mathop{\text{Mat}}\nolimits}
\newcommand{\rg}{\mathop{\text{rg}}\nolimits}
\newcommand{\tr}{\mathop{\text{tr}}\nolimits}


%----- Structure des exercices ------

\newtheoremstyle{styleexo}% name
{2ex}% Space above
{3ex}% Space below
{}% Body font
{}% Indent amount 1
{\bfseries} % Theorem head font
{}% Punctuation after theorem head
{\newline}% Space after theorem head 2
{}% Theorem head spec (can be left empty, meaning ‘normal’)

%\theoremstyle{styleexo}
\newtheorem{exo}{Exercice}
\newtheorem{ind}{Indications}
\newtheorem{cor}{Correction}


\newcommand{\exercice}[1]{} \newcommand{\finexercice}{}
%\newcommand{\exercice}[1]{{\tiny\texttt{#1}}\vspace{-2ex}} % pour afficher le numero absolu, l'auteur...
\newcommand{\enonce}{\begin{exo}} \newcommand{\finenonce}{\end{exo}}
\newcommand{\indication}{\begin{ind}} \newcommand{\finindication}{\end{ind}}
\newcommand{\correction}{\begin{cor}} \newcommand{\fincorrection}{\end{cor}}

\newcommand{\noindication}{\stepcounter{ind}}
\newcommand{\nocorrection}{\stepcounter{cor}}

\newcommand{\fiche}[1]{} \newcommand{\finfiche}{}
\newcommand{\titre}[1]{\centerline{\large \bf #1}}
\newcommand{\addcommand}[1]{}
\newcommand{\video}[1]{}

% Marge
\newcommand{\mymargin}[1]{\marginpar{{\small #1}}}

\def\noqed{\renewcommand{\qedsymbol}{}}


%----- Presentation ------
\setlength{\parindent}{0cm}

%\newcommand{\ExoSept}{\href{http://exo7.emath.fr}{\textbf{\textsf{Exo7}}}}

\definecolor{myred}{rgb}{0.93,0.26,0}
\definecolor{myorange}{rgb}{0.97,0.58,0}
\definecolor{myyellow}{rgb}{1,0.86,0}

\newcommand{\LogoExoSept}[1]{  % input : echelle
{\usefont{U}{cmss}{bx}{n}
\begin{tikzpicture}[scale=0.1*#1,transform shape]
  \fill[color=myorange] (0,0)--(4,0)--(4,-4)--(0,-4)--cycle;
  \fill[color=myred] (0,0)--(0,3)--(-3,3)--(-3,0)--cycle;
  \fill[color=myyellow] (4,0)--(7,4)--(3,7)--(0,3)--cycle;
  \node[scale=5] at (3.5,3.5) {Exo7};
\end{tikzpicture}}
}


\newcommand{\debutmontitre}{
  \author{} \date{} 
  \thispagestyle{empty}
  \hspace*{-10ex}
  \begin{minipage}{\textwidth}
    \titlepage  
  \vspace*{-2.5cm}
  \begin{center}
    \LogoExoSept{2.5}
  \end{center}
  \end{minipage}

  \vspace*{-0cm}
  
  % Astuce pour que le background ne soit pas discrétisé lors de la conversion pdf -> png
\begin{tikzpicture}
        \fill[opacity=0,green!60!black] (0,0)--++(0,0)--++(0,0)--++(0,0)--cycle; 
\end{tikzpicture}

% toc S'affiche trop tot :
% \tableofcontents[hideallsubsections, pausesections]
}

\newcommand{\finmontitre}{
  \end{frame}
  \setcounter{framenumber}{0}
} % ne marche pas pour une raison obscure

%----- Commandes supplementaires ------

% \usepackage[landscape]{geometry}
% \geometry{top=1cm, bottom=3cm, left=2cm, right=10cm, marginparsep=1cm
% }
% \usepackage[a4paper]{geometry}
% \geometry{top=2cm, bottom=2cm, left=2cm, right=2cm, marginparsep=1cm
% }

%\usepackage{standalone}


% New command Arnaud -- november 2011
\setbeamersize{text margin left=24ex}
% si vous modifier cette valeur il faut aussi
% modifier le decalage du titre pour compenser
% (ex : ici =+10ex, titre =-5ex

\theoremstyle{definition}
%\newtheorem{proposition}{Proposition}
%\newtheorem{exemple}{Exemple}
%\newtheorem{theoreme}{Théorème}
%\newtheorem{lemme}{Lemme}
%\newtheorem{corollaire}{Corollaire}
%\newtheorem*{remarque*}{Remarque}
%\newtheorem*{miniexercice}{Mini-exercices}
%\newtheorem{definition}{Définition}

% Commande tikz
\usetikzlibrary{calc}
\usetikzlibrary{patterns,arrows}
\usetikzlibrary{matrix}
\usetikzlibrary{fadings} 

%definition d'un terme
\newcommand{\defi}[1]{{\color{myorange}\textbf{\emph{#1}}}}
\newcommand{\evidence}[1]{{\color{blue}\textbf{\emph{#1}}}}
\newcommand{\assertion}[1]{\emph{\og#1\fg}}  % pour chapitre logique
%\renewcommand{\contentsname}{Sommaire}
\renewcommand{\contentsname}{}
\setcounter{tocdepth}{2}



%------ Figures ------

\def\myscale{1} % par défaut 
\newcommand{\myfigure}[2]{  % entrée : echelle, fichier figure
\def\myscale{#1}
\begin{center}
\footnotesize
{#2}
\end{center}}


%------ Encadrement ------

\usepackage{fancybox}


\newcommand{\mybox}[1]{
\setlength{\fboxsep}{7pt}
\begin{center}
\shadowbox{#1}
\end{center}}

\newcommand{\myboxinline}[1]{
\setlength{\fboxsep}{5pt}
\raisebox{-10pt}{
\shadowbox{#1}
}
}

%--------------- Commande beamer---------------
\newcommand{\beameronly}[1]{#1} % permet de mettre des pause dans beamer pas dans poly


\setbeamertemplate{navigation symbols}{}
\setbeamertemplate{footline}  % tiré du fichier beamerouterinfolines.sty
{
  \leavevmode%
  \hbox{%
  \begin{beamercolorbox}[wd=.333333\paperwidth,ht=2.25ex,dp=1ex,center]{author in head/foot}%
    % \usebeamerfont{author in head/foot}\insertshortauthor%~~(\insertshortinstitute)
    \usebeamerfont{section in head/foot}{\bf\insertshorttitle}
  \end{beamercolorbox}%
  \begin{beamercolorbox}[wd=.333333\paperwidth,ht=2.25ex,dp=1ex,center]{title in head/foot}%
    \usebeamerfont{section in head/foot}{\bf\insertsectionhead}
  \end{beamercolorbox}%
  \begin{beamercolorbox}[wd=.333333\paperwidth,ht=2.25ex,dp=1ex,right]{date in head/foot}%
    % \usebeamerfont{date in head/foot}\insertshortdate{}\hspace*{2em}
    \insertframenumber{} / \inserttotalframenumber\hspace*{2ex} 
  \end{beamercolorbox}}%
  \vskip0pt%
}


\definecolor{mygrey}{rgb}{0.5,0.5,0.5}
\setlength{\parindent}{0cm}
%\DeclareTextFontCommand{\helvetica}{\fontfamily{phv}\selectfont}

% background beamer
\definecolor{couleurhaut}{rgb}{0.85,0.9,1}  % creme
\definecolor{couleurmilieu}{rgb}{1,1,1}  % vert pale
\definecolor{couleurbas}{rgb}{0.85,0.9,1}  % blanc
\setbeamertemplate{background canvas}[vertical shading]%
[top=couleurhaut,middle=couleurmilieu,midpoint=0.4,bottom=couleurbas] 
%[top=fondtitre!05,bottom=fondtitre!60]



\makeatletter
\setbeamertemplate{theorem begin}
{%
  \begin{\inserttheoremblockenv}
  {%
    \inserttheoremheadfont
    \inserttheoremname
    \inserttheoremnumber
    \ifx\inserttheoremaddition\@empty\else\ (\inserttheoremaddition)\fi%
    \inserttheorempunctuation
  }%
}
\setbeamertemplate{theorem end}{\end{\inserttheoremblockenv}}

\newenvironment{theoreme}[1][]{%
   \setbeamercolor{block title}{fg=structure,bg=structure!40}
   \setbeamercolor{block body}{fg=black,bg=structure!10}
   \begin{block}{{\bf Th\'eor\`eme }#1}
}{%
   \end{block}%
}


\newenvironment{proposition}[1][]{%
   \setbeamercolor{block title}{fg=structure,bg=structure!40}
   \setbeamercolor{block body}{fg=black,bg=structure!10}
   \begin{block}{{\bf Proposition }#1}
}{%
   \end{block}%
}

\newenvironment{corollaire}[1][]{%
   \setbeamercolor{block title}{fg=structure,bg=structure!40}
   \setbeamercolor{block body}{fg=black,bg=structure!10}
   \begin{block}{{\bf Corollaire }#1}
}{%
   \end{block}%
}

\newenvironment{mydefinition}[1][]{%
   \setbeamercolor{block title}{fg=structure,bg=structure!40}
   \setbeamercolor{block body}{fg=black,bg=structure!10}
   \begin{block}{{\bf Définition} #1}
}{%
   \end{block}%
}

\newenvironment{lemme}[0]{%
   \setbeamercolor{block title}{fg=structure,bg=structure!40}
   \setbeamercolor{block body}{fg=black,bg=structure!10}
   \begin{block}{\bf Lemme}
}{%
   \end{block}%
}

\newenvironment{remarque}[1][]{%
   \setbeamercolor{block title}{fg=black,bg=structure!20}
   \setbeamercolor{block body}{fg=black,bg=structure!5}
   \begin{block}{Remarque #1}
}{%
   \end{block}%
}


\newenvironment{exemple}[1][]{%
   \setbeamercolor{block title}{fg=black,bg=structure!20}
   \setbeamercolor{block body}{fg=black,bg=structure!5}
   \begin{block}{{\bf Exemple }#1}
}{%
   \end{block}%
}


\newenvironment{miniexercice}[0]{%
   \setbeamercolor{block title}{fg=structure,bg=structure!20}
   \setbeamercolor{block body}{fg=black,bg=structure!5}
   \begin{block}{Mini-exercices}
}{%
   \end{block}%
}


\newenvironment{tp}[0]{%
   \setbeamercolor{block title}{fg=structure,bg=structure!40}
   \setbeamercolor{block body}{fg=black,bg=structure!10}
   \begin{block}{\bf Travaux pratiques}
}{%
   \end{block}%
}
\newenvironment{exercicecours}[1][]{%
   \setbeamercolor{block title}{fg=structure,bg=structure!40}
   \setbeamercolor{block body}{fg=black,bg=structure!10}
   \begin{block}{{\bf Exercice }#1}
}{%
   \end{block}%
}
\newenvironment{algo}[1][]{%
   \setbeamercolor{block title}{fg=structure,bg=structure!40}
   \setbeamercolor{block body}{fg=black,bg=structure!10}
   \begin{block}{{\bf Algorithme}\hfill{\color{gray}\texttt{#1}}}
}{%
   \end{block}%
}


\setbeamertemplate{proof begin}{
   \setbeamercolor{block title}{fg=black,bg=structure!20}
   \setbeamercolor{block body}{fg=black,bg=structure!5}
   \begin{block}{{\footnotesize Démonstration}}
   \footnotesize
   \smallskip}
\setbeamertemplate{proof end}{%
   \end{block}}
\setbeamertemplate{qed symbol}{\openbox}


\makeatother
\usecolortheme[RGB={205,0,0}]{structure}

%%%%%%%%%%%%%%%%%%%%%%%%%%%%%%%%%%%%%%%%%%%%%%%%%%%%%%%%%%%%%
%%%%%%%%%%%%%%%%%%%%%%%%%%%%%%%%%%%%%%%%%%%%%%%%%%%%%%%%%%%%%

\begin{document}


\title{{\bf Espaces vectoriels}}
\subtitle{Application linéaire (fin)}

\begin{frame}
  
  \debutmontitre

  \pause

{\footnotesize
\hfill
\setbeamercovered{transparent=50}
\begin{minipage}{0.6\textwidth}
  \begin{itemize}
    \item<3-> Image d'une application linéaire
    \item<4-> Noyau d'une application linéaire
    \item<5-> L'espace vectoriel $\mathcal{L}(E,F)$
    \item<6-> Composition et inverse d'applications linéaires    
  \end{itemize}
\end{minipage}
}

\end{frame}

\setcounter{framenumber}{0}


%%%%%%%%%%%%%%%%%%%%%%%%%%%%%%%%%%%%%%%%%%%%%%%%%%%%%%%%%%%%%%%%
\section{Image d'une application linéaire}

\begin{frame}


\defi{Image directe} $f : E \longrightarrow F$ et $A \subset E$
$$f(A)=\big\{ f(x)  \mid x\in A \big\}$$

\pause
\bigskip

\begin{itemize}
  \item $E$ et $F$ des $\Kk$-espaces vectoriels
  \item $f : E \to F$ une application linéaire
  \item $f(E)$ est l'\defi{image} de l'application linéaire $f$ et est notée \defi{$\Im f$}
\end{itemize}

\pause

\begin{proposition}
\begin{enumerate}
  \item Si $E'$ est un sous-espace vectoriel de $E$, 
  alors $f(E')$ est un sous-espace vectoriel de $F$

  \pause
  
  \item En particulier, $\Im f$ est un sous-espace vectoriel de $F$
\end{enumerate}
 \end{proposition}

\pause
 
\centerline{$f$ est surjective si et seulement si $\Im f =F$}

\end{frame}


\begin{frame}

\begin{proposition}
\begin{enumerate}
  \item Si $E'$ est un sous-espace vectoriel de $E$, 
  alors $f(E')$ est un sous-espace vectoriel de $F$
  
  \item En particulier, $\Im f$ est un sous-espace vectoriel de $F$
\end{enumerate}
 \end{proposition}
 
\begin{proof}

\pause

\begin{itemize}
  \item Tout d'abord, comme $0_{E} \in E'$ alors $0_{F} = f(0_{E}) \in f(E')$

\pause  
  \item Montrons que pour tous $y_1,y_2 \in f(E')$ et tous $\lambda,\mu \in \Kk$, 
  l'élément $\lambda y_1 + \mu y_2$ appartient à $f(E')$
\pause  
  \begin{itemize}
    \item il existe $x_1\in E'$ tel que $f(x_1)=y_1$
\pause     
    \item il existe $x_2\in E'$ tel que $f(x_2)=y_2$
\pause     
    \item $\lambda x_1 + \mu x_2 \in E'$, car $E'$ est un sous-espace vectoriel
\pause 
    \item comme $f$ est linéaire $f(\lambda x_1+ \mu x_2)=\lambda f(x_1)+ \mu f(x_2) =\lambda y_1 + \mu y_2$
\pause    
    \item donc $\lambda y_1 + \mu y_2 \in f(E')$ \qedhere
  \end{itemize}
\end{itemize}


\end{proof}
\end{frame}


%%%%%%%%%%%%%%%%%%%%%%%%%%%%%%%%%%%%%%%%%%%%%%%%%%%%%%%%%%%%%%%%
\section{Noyau d'une application linéaire}

\begin{frame}
\centerline{$E,F$ espaces vectoriels, $f : E \longrightarrow F$ application linéaire}
\begin{mydefinition}
Le \defi{noyau} de $f$, noté  $\Ker(f)$, est l'ensemble des 
éléments de $E$ dont l'image est $0_{F}$ :
\myboxinline{$\Ker (f)=\big\{x \in E \mid f(x)=0_{F}\big\}$}
\end{mydefinition}
\pause
\centerline{$\Ker(f) = f^{-1} \{0_F\}$}

\pause
\begin{proposition}
Le noyau de $f$ est un sous-espace vectoriel de $E$
\end{proposition}

\pause
\begin{proof} 
\begin{itemize}
  \item $f(0_E)=0_F$ donc $0_{E} \in \Ker (f)$
\pause  
  \item Soient $x_1,x_2 \in \Ker (f)$ 
et $\lambda,\mu \in \Kk$, montrons que 
$\lambda x_1+\mu x_2$ est un élément de $\Ker (f)$ : \\
\pause
\hfil $f(\lambda x_1+\mu x_2)\pause=\lambda f(x_1)+\mu f(x_2)\pause= \lambda 0_{F} + \mu 0_{F}\pause=0_{F}$
\qedhere
\end{itemize}
\end{proof}
\end{frame}


\begin{frame}
\begin{exemple}
$$\begin{array}{rcl}
f : \quad \Rr^3 & \to & \Rr^2\\
(x,y,z) & \mapsto & (-2x,y+3z)
 \end{array}$$
 \pause
\vspace*{-4ex}
\begin{itemize}
  
  \item \evidence{Noyau $\Ker(f)$}
  \vspace*{-2ex}
 \pause  
$$
\begin{array}{l}
(x,y,z) \in \Ker(f)    \pause \iff f(x,y,z)=(0,0) \\
 \pause
\iff (-2x,y+3z)=(0,0)  \pause \iff \left\{\begin{array}{rcl}
                               -2x  & = & 0 \\
                               y+3z & = & 0 \\
                               \end{array}\right. \\
 \pause  \iff  (x,y,z) = (0,-3z,z), \quad z \in \Rr \\   
\end{array}
$$ 
\pause
Donc $\Ker(f) = \big\{ (0,-3z,z) \mid  z \in \Rr \big\} \pause = \Vect\big\{ (0,-3,1) \big\}$ 

 \pause
  \item \evidence{Image de $f$}  \pause \qquad Fixons $(x',y')\in \Rr^2$
\vspace*{-2ex}  
 \pause
$$
(x',y')= f(x,y,z) \pause
  \Leftrightarrow  (-2x,y+3z)=(x',y') 
 \pause  \Leftrightarrow  \left\{\begin{array}{rcl}
             -2x  & \!=\! & x' \\
             y+3z & \!=\! & y' \\
          \end{array}\right.    
$$
\vspace*{-2ex}
 \pause
  \begin{itemize}
    \item On peut prendre par exemple $x = -\frac{x'}{2}$, $y'=y$, $z=0$
 \pause    
    \item Donc pour tout $(x',y') \in \Rr^2$, on a $f(-\frac{x'}{2},y',0) = (x',y')$
  \pause   
    \item Donc $\Im(f)= \Rr^2$, et $f$ est surjective
  \end{itemize}
\end{itemize}
\end{exemple}
\end{frame}


\begin{frame}

\begin{exemple}
\begin{itemize}
  \item Soit $A \in M_{n,p}(\Rr)$
  \pause
  \item Soit $f: \Rr^p \longrightarrow \Rr^n$ l'application linéaire définie par $f(X) = AX$
  \pause
  \item Alors $\Ker(f) = \big\{ X \in \Rr^p \mid AX=0 \big\}$ : c'est l'ensemble des $X \in \Rr^p$
solutions du système linéaire homogène $AX=0$
  \pause
  \item On verra plus tard que $\Im(f)$ est l'espace engendré 
par les colonnes de la matrice $A$
\end{itemize}
\end{exemple}

\bigskip
\pause

\begin{exemple}
\begin{itemize}
  \item Soit $f : \Rr^3 \to \Rr$ l'application linéaire, $f(x,y,z)=ax+by+cz$
  \pause
  \item $\Ker f = \big\{ (x,y,z) \in \Rr^3 \mid ax+by+cz=0 \big\}$
  \pause
  \item $\Ker f = \mathcal{P}$ : le plan d'équation $(ax+by+cz=0)$
  \pause
  \item On retrouve : un plan passant par l'origine est un sous-espace vectoriel
\end{itemize}
\end{exemple}

\end{frame}



\begin{frame}
$f : E \longrightarrow F$ application linéaire entre espaces vectoriels
\begin{theoreme}
\mybox{$f$ injective \quad $\iff \quad \Ker(f) = \big\{0_E\big\}$}
\end{theoreme} 
\pause
Pour montrer que $f$ est injective, il suffit de vérifier que : \\
\centerline{si $f(x)=0_F$ alors $x=0_E$}

\pause

\begin{proof}\pause
\vspace*{-1ex}
\begin{itemize}
  \item Supposons que $f$ soit injective et montrons que $\Ker(f)=\{0_E\}$
  \pause
  \begin{itemize}
    \item Soit $x$ un élément de $\Ker(f)$, on a $f(x)=0_{F}$
    \pause
    \item Comme $f$ est linéaire, on a aussi $f(0_{E})=0_{F}$
    \pause
    \item Comme $f$ est injective et $f(x)=f(0_{E})$, on déduit $x=0_{E}$
%    \item Donc $\Ker (f)=\{0_{E}\}$
  \end{itemize}
\vspace*{-1ex}
\pause
  \item Réciproquement, supposons maintenant que $\Ker (f)=\{0_E\}$
  \pause
\vspace*{-1ex} 
  \begin{itemize}
    \item Soient $x, y \in E$ tels que $f(x)=f(y)$, on a donc $f(x)-f(y)=0_{F}$
    \pause
    \item Comme $f$ est linéaire, on en déduit $f(x-y)=0_{F}$
    \pause
    \item C'est-à-dire $x-y \in \Ker (f)$
    \pause
    \item Donc $x-y=0_{E}$, soit $x=y$   \qedhere
  \end{itemize}
\end{itemize}
\vspace*{-2.5ex} 
\end{proof}
\end{frame}



\begin{frame}
\begin{exemple}
$$\begin{array}{rcl}
f : \Rr_n[X] & \longrightarrow & \Rr_{n+1}[X]\\
P(X)  & \longmapsto & X \cdot P(X)
\end{array}$$
\pause
\begin{itemize}\setlength{\itemsep}{6pt}
  \item $f$ est une application linéaire
  \pause
  \item \evidence{Noyau de $f$}
  \pause
  \begin{itemize}
    \item $P(X) = a_n X^n + \cdots +a_1 X + a_0 \in \Rr_n[X]$ tel que $X \cdot P(X) = 0$
    \pause
    \item Alors $a_n X^{n+1} + \cdots+ a_1 X^2 + a_0 X = 0$
    \pause
    \item Ainsi, $a_i = 0$ pour tout $i \in \{0,\ldots, n\}$ et donc $P(X) = 0$
    \pause
    \item $\Ker(f) = \{0\}$ \pause et $f$ est injective
  \end{itemize}
  \pause  
  \item \evidence{Image de $f$}
  \pause
  \begin{itemize}
    \item {\footnotesize $\Im(f)$ est l'ensemble des polynômes de $\Rr_{n+1}[X]$ sans terme constant}
    \pause
    \item $\Im(f) = \Vect \big\{X, X^2,\dots, X^{n+1}\big\}$
    \pause
    \item $f$ n'est pas surjective
  \end{itemize}
\end{itemize}
\end{exemple}
\end{frame}

%%%%%%%%%%%%%%%%%%%%%%%%%%%%%%%%%%%%%%%%%%%%%%%%%%%%%%%%%%%%%%%%
\section{L'espace vectoriel $\mathcal{L}(E,F)$}

\begin{frame}

\begin{proposition}
L'ensemble des applications linéaires entre deux $\Kk$-espaces vectoriels $E$ et $F$, 
noté $\mathcal{L}(E,F)$, est un $\Kk$-espace vectoriel
\end{proposition}

\pause

   $$(f+g)(u)=f(u)+g(u)\quad \text{ et } \quad (\lambda \cdot f)(u)=\lambda f(u)$$

\end{frame}



%%%%%%%%%%%%%%%%%%%%%%%%%%%%%%%%%%%%%%%%%%%%%%%%%%%%%%%%%%%%%%%%
\section{Composition et inverse d'applications linéaires}

\begin{frame}
Soient $E, F, G$  trois $\Kk$-espaces vectoriels, $f$ une application linéaire de $E$ dans $F$ et 
$g$ une application linéaire de $F$ dans $G$
\begin{proposition}
$g \circ f$ est une application linéaire de $E$ dans $G$
\end{proposition}

\pause

En particulier, le composé de deux endomorphismes de $E$ est un endomorphisme de $E$.
Autrement dit, $\circ$ est une loi de composition interne sur $\mathcal{L}(E)$

\pause
\bigskip

\begin{itemize}
  \item $g \circ (f_1+f_2)=g \circ f_1+ g \circ f_2$
  
  \item $(g_1+g_2) \circ f =g_1 \circ f + g_2 \circ f$
  
  \item $(\lambda g) \circ f =g \circ (\lambda f) =\lambda (g \circ f)$
\end{itemize}
\end{frame}


\begin{frame}
\textbf{Vocabulaire}

Soient $E$ et $F$ deux $\Kk$-espaces vectoriels

\begin{itemize}   
  \item Une application linéaire \evidence{bijective} de $E$ sur $F$ est appelée 
  \defi{isomorphisme} d'espaces vectoriels
  \pause
  \item Proposition : $f^{-1}$ est une application linéaire
  \pause
  \item Les espaces vectoriels $E$ et $F$ 
  sont alors dits \defi{isomorphes}
  
  \bigskip
  \pause
   
  \item Un endomorphisme bijectif de $E$ (c'est-à-dire une application linéaire bijective de $E$ dans $E$)
  est appelé \defi{automorphisme} de $E$
  \pause
  \item L'ensemble des automorphismes de $E$ est noté $GL(E)$
\end{itemize}
\end{frame}

% 
% \begin{frame}
% Soient $E$ et $F$ deux $\Kk$-espaces vectoriels
% \begin{proposition}
% Si $f$ est un isomorphisme de $E$ sur $F$, alors  
% $f^{-1}$ est un isomorphisme de $F$ sur $E$
% \end{proposition}
% \end{frame}


\begin{frame}
\begin{exemple}
\begin{itemize}
  \item Soit $f : \Rr^2 \to \Rr^2$ définie par $f(x,y)=(2x+3y,x+y)$
\pause  
  \item $f$ est linéaire
  \pause
  \item Pour prouver que $f$ est bijective, on pourrait calculer son noyau et son image
  \pause
  \item Calcul direct de l'inverse
  \begin{itemize}
  \pause
    \item Résoudre $f(x,y)=(x',y')$ : système linéaire 
    \pause
    \item On trouve $(x,y) = (-x'+3y',x'-2y')$
    \pause
    \item $f^{-1}(x',y')= (-x'+3y',x'-2y')$
    \pause
    \item $f^{-1}$ est l'inverse de $f$  \pause et  $f^{-1}$ est une application linéaire
  \end{itemize} 
\end{itemize}
\end{exemple}

\pause
\begin{itemize}
  \item $f : \Rr^n \to \Rr^n$ définie par $f(X)=AX$ \quad ($A \in M_n(\Rr)$)
  \pause
  \item $A$ est inversible $\iff$ $f$ bijective
  \pause
  \item $f^{-1}(X)= A^{-1} X$
  \pause
  \item Exemple : $X = \begin{pmatrix} x \\ y \end{pmatrix} \qquad 
A = \begin{pmatrix} 2 & 3 \\ 1 & 1 \end{pmatrix} \qquad 
A^{-1} = \begin{pmatrix} -1 & 3 \\ 1 & -2 \end{pmatrix}$
\end{itemize}

\end{frame}


%%%%%%%%%%%%%%%%%%%%%%%%%%%%%%%%%%%%%%%%%%%%%%%%%%%%%%%%%%%%%%%%
\section{Mini-exercices}

\begin{frame}

\begin{miniexercice}
\begin{enumerate}
  \item Soit $f : \Rr^3 \to \Rr^3$ définie par $f(x,y,z)=(-x,y+z,2z)$.
  Montrer que $f$ est une application linéaire. Calculer $\Ker(f)$ et $\Im(f)$.
  $f$ admet-elle un inverse ? Même question avec $f(x,y,z) = (x-y,x+y,y)$.
   
  \item Soient $E$ un espace vectoriel, et $F,G$ deux sous-espaces tels que $E= F \oplus G$.
  Chaque $u\in E$ se décompose de manière unique $u=v+w$ avec $v \in F$, $w\in G$.
  La \defi{symétrie} par rapport à $F$ parallèlement à $G$ est l'application $s : E \to E$
  définie par $s(u)=v-w$. Faire un dessin. Montrer que $s$ est une application linéaire.
  Montrer que $s^2 = \id_E$. Calculer $\Ker(s)$ et $\Im(s)$. $s$ admet-elle un inverse ?

  \item Soit $f : \Rr_n[X] \to \Rr_n[X]$ définie par $P(X) \mapsto P''(X)$ (où $P''$ désigne la dérivée seconde). 
  Montrer que $f$ est une application linéaire. Calculer $\Ker(f)$ et $\Im(f)$. $f$ admet-elle un inverse ?
\end{enumerate}
\end{miniexercice}

\end{frame}

\end{document}