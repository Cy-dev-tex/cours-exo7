
%%%%%%%%%%%%%%%%%% PREAMBULE %%%%%%%%%%%%%%%%%%


\documentclass[12pt]{article}

\usepackage{amsfonts,amsmath,amssymb,amsthm}
\usepackage[utf8]{inputenc}
\usepackage[T1]{fontenc}
\usepackage[francais]{babel}


% packages
\usepackage{amsfonts,amsmath,amssymb,amsthm}
\usepackage[utf8]{inputenc}
\usepackage[T1]{fontenc}
%\usepackage{lmodern}

\usepackage[francais]{babel}
\usepackage{fancybox}
\usepackage{graphicx}

\usepackage{float}

%\usepackage[usenames, x11names]{xcolor}
\usepackage{tikz}
\usepackage{datetime}

\usepackage{mathptmx}
%\usepackage{fouriernc}
%\usepackage{newcent}
\usepackage[mathcal,mathbf]{euler}

%\usepackage{palatino}
%\usepackage{newcent}


% Commande spéciale prompteur

%\usepackage{mathptmx}
%\usepackage[mathcal,mathbf]{euler}
%\usepackage{mathpple,multido}

\usepackage[a4paper]{geometry}
\geometry{top=2cm, bottom=2cm, left=1cm, right=1cm, marginparsep=1cm}

\newcommand{\change}{{\color{red}\rule{\textwidth}{1mm}\\}}

\newcounter{mydiapo}

\newcommand{\diapo}{\newpage
\hfill {\normalsize  Diapo \themydiapo \quad \texttt{[\jobname]}} \\
\stepcounter{mydiapo}}


%%%%%%% COULEURS %%%%%%%%%%

% Pour blanc sur noir :
%\pagecolor[rgb]{0.5,0.5,0.5}
% \pagecolor[rgb]{0,0,0}
% \color[rgb]{1,1,1}



%\DeclareFixedFont{\myfont}{U}{cmss}{bx}{n}{18pt}
\newcommand{\debuttexte}{
%%%%%%%%%%%%% FONTES %%%%%%%%%%%%%
\renewcommand{\baselinestretch}{1.5}
\usefont{U}{cmss}{bx}{n}
\bfseries

% Taille normale : commenter le reste !
%Taille Arnaud
%\fontsize{19}{19}\selectfont

% Taille Barbara
%\fontsize{21}{22}\selectfont

%Taille François
%\fontsize{25}{30}\selectfont

%Taille Pascal
%\fontsize{25}{30}\selectfont

%Taille Laura
%\fontsize{30}{35}\selectfont


%\myfont
%\usefont{U}{cmss}{bx}{n}

%\Huge
%\addtolength{\parskip}{\baselineskip}
}


% \usepackage{hyperref}
% \hypersetup{colorlinks=true, linkcolor=blue, urlcolor=blue,
% pdftitle={Exo7 - Exercices de mathématiques}, pdfauthor={Exo7}}


%section
% \usepackage{sectsty}
% \allsectionsfont{\bf}
%\sectionfont{\color{Tomato3}\upshape\selectfont}
%\subsectionfont{\color{Tomato4}\upshape\selectfont}

%----- Ensembles : entiers, reels, complexes -----
\newcommand{\Nn}{\mathbb{N}} \newcommand{\N}{\mathbb{N}}
\newcommand{\Zz}{\mathbb{Z}} \newcommand{\Z}{\mathbb{Z}}
\newcommand{\Qq}{\mathbb{Q}} \newcommand{\Q}{\mathbb{Q}}
\newcommand{\Rr}{\mathbb{R}} \newcommand{\R}{\mathbb{R}}
\newcommand{\Cc}{\mathbb{C}} 
\newcommand{\Kk}{\mathbb{K}} \newcommand{\K}{\mathbb{K}}

%----- Modifications de symboles -----
\renewcommand{\epsilon}{\varepsilon}
\renewcommand{\Re}{\mathop{\text{Re}}\nolimits}
\renewcommand{\Im}{\mathop{\text{Im}}\nolimits}
%\newcommand{\llbracket}{\left[\kern-0.15em\left[}
%\newcommand{\rrbracket}{\right]\kern-0.15em\right]}

\renewcommand{\ge}{\geqslant}
\renewcommand{\geq}{\geqslant}
\renewcommand{\le}{\leqslant}
\renewcommand{\leq}{\leqslant}

%----- Fonctions usuelles -----
\newcommand{\ch}{\mathop{\mathrm{ch}}\nolimits}
\newcommand{\sh}{\mathop{\mathrm{sh}}\nolimits}
\renewcommand{\tanh}{\mathop{\mathrm{th}}\nolimits}
\newcommand{\cotan}{\mathop{\mathrm{cotan}}\nolimits}
\newcommand{\Arcsin}{\mathop{\mathrm{Arcsin}}\nolimits}
\newcommand{\Arccos}{\mathop{\mathrm{Arccos}}\nolimits}
\newcommand{\Arctan}{\mathop{\mathrm{Arctan}}\nolimits}
\newcommand{\Argsh}{\mathop{\mathrm{Argsh}}\nolimits}
\newcommand{\Argch}{\mathop{\mathrm{Argch}}\nolimits}
\newcommand{\Argth}{\mathop{\mathrm{Argth}}\nolimits}
\newcommand{\pgcd}{\mathop{\mathrm{pgcd}}\nolimits} 

\newcommand{\Card}{\mathop{\text{Card}}\nolimits}
\newcommand{\Ker}{\mathop{\text{Ker}}\nolimits}
\newcommand{\id}{\mathop{\text{id}}\nolimits}
\newcommand{\ii}{\mathrm{i}}
\newcommand{\dd}{\mathrm{d}}
\newcommand{\Vect}{\mathop{\text{Vect}}\nolimits}
\newcommand{\Mat}{\mathop{\mathrm{Mat}}\nolimits}
\newcommand{\rg}{\mathop{\text{rg}}\nolimits}
\newcommand{\tr}{\mathop{\text{tr}}\nolimits}
\newcommand{\ppcm}{\mathop{\text{ppcm}}\nolimits}

%----- Structure des exercices ------

\newtheoremstyle{styleexo}% name
{2ex}% Space above
{3ex}% Space below
{}% Body font
{}% Indent amount 1
{\bfseries} % Theorem head font
{}% Punctuation after theorem head
{\newline}% Space after theorem head 2
{}% Theorem head spec (can be left empty, meaning ‘normal’)

%\theoremstyle{styleexo}
\newtheorem{exo}{Exercice}
\newtheorem{ind}{Indications}
\newtheorem{cor}{Correction}


\newcommand{\exercice}[1]{} \newcommand{\finexercice}{}
%\newcommand{\exercice}[1]{{\tiny\texttt{#1}}\vspace{-2ex}} % pour afficher le numero absolu, l'auteur...
\newcommand{\enonce}{\begin{exo}} \newcommand{\finenonce}{\end{exo}}
\newcommand{\indication}{\begin{ind}} \newcommand{\finindication}{\end{ind}}
\newcommand{\correction}{\begin{cor}} \newcommand{\fincorrection}{\end{cor}}

\newcommand{\noindication}{\stepcounter{ind}}
\newcommand{\nocorrection}{\stepcounter{cor}}

\newcommand{\fiche}[1]{} \newcommand{\finfiche}{}
\newcommand{\titre}[1]{\centerline{\large \bf #1}}
\newcommand{\addcommand}[1]{}
\newcommand{\video}[1]{}

% Marge
\newcommand{\mymargin}[1]{\marginpar{{\small #1}}}



%----- Presentation ------
\setlength{\parindent}{0cm}

%\newcommand{\ExoSept}{\href{http://exo7.emath.fr}{\textbf{\textsf{Exo7}}}}

\definecolor{myred}{rgb}{0.93,0.26,0}
\definecolor{myorange}{rgb}{0.97,0.58,0}
\definecolor{myyellow}{rgb}{1,0.86,0}

\newcommand{\LogoExoSept}[1]{  % input : echelle
{\usefont{U}{cmss}{bx}{n}
\begin{tikzpicture}[scale=0.1*#1,transform shape]
  \fill[color=myorange] (0,0)--(4,0)--(4,-4)--(0,-4)--cycle;
  \fill[color=myred] (0,0)--(0,3)--(-3,3)--(-3,0)--cycle;
  \fill[color=myyellow] (4,0)--(7,4)--(3,7)--(0,3)--cycle;
  \node[scale=5] at (3.5,3.5) {Exo7};
\end{tikzpicture}}
}



\theoremstyle{definition}
%\newtheorem{proposition}{Proposition}
%\newtheorem{exemple}{Exemple}
%\newtheorem{theoreme}{Théorème}
\newtheorem{lemme}{Lemme}
\newtheorem{corollaire}{Corollaire}
%\newtheorem*{remarque*}{Remarque}
%\newtheorem*{miniexercice}{Mini-exercices}
%\newtheorem{definition}{Définition}




%definition d'un terme
\newcommand{\defi}[1]{{\color{myorange}\textbf{\emph{#1}}}}
\newcommand{\evidence}[1]{{\color{blue}\textbf{\emph{#1}}}}



 %----- Commandes divers ------

\newcommand{\codeinline}[1]{\texttt{#1}}

%%%%%%%%%%%%%%%%%%%%%%%%%%%%%%%%%%%%%%%%%%%%%%%%%%%%%%%%%%%%%
%%%%%%%%%%%%%%%%%%%%%%%%%%%%%%%%%%%%%%%%%%%%%%%%%%%%%%%%%%%%%



\begin{document}

\debuttexte


%%%%%%%%%%%%%%%%%%%%%%%%%%%%%%%%%%%%%%%%%%%%%%%%%%%%%%%%%%%
\diapo

Nous allons maintenant voir un processus de chiffrement radicalement différent

il s'agit de la cryptographie à clé publique.

\change

\change

Nous commencerons par énoncer le principe de Kerckhoffs

\change

Nous évoquerons un problème mathématique difficile : la factorisation d'entiers

\change

Nous aborderons la notion de fonction à sens unique

\change

Nous reviendrons sur le principe de chiffrement à clé privée

\change

Avant d'étudier le principe du chiffrement à clé publique.
  

%%%%%%%%%%%%%%%%%%%%%%%%%%%%%%%%%%%%%%%%%%%%%%%%%%%%%%%%%%%
\diapo

Pour envoyer des messages secrets, les Grecs utilisaient un messager.

Il commençaient par lui raser la tête,

tatouaient le message sur son crâne et attendaient que les cheveux repoussent.

A ce moment là seulement, le messager pouvait se mettre en route et effectuer sa mission !

=

La sécurité d'un tel protocole d'échange d'informations repose uniquement sur le fait de tenir secrète la méthode utilisée.

\change

Ce procédé, rapporté dans des écrits de l'époque, va à l'encontre d'un principe énoncé beaucoup plus tard par
Auguste Kerckhoffs, en 1883, et qui dit que : 

\centerline{<<La sécurité d'un système de chiffrement ne doit reposer que sur le secret la clé.>>}

\change

Plus précisément, cela revient à dire que : 

\centerline{<<L'ennemi peut avoir connaissance du système de chiffrement.>>}

\change
Ce principe est novateur dans la mesure 
où intuitivement il semble opportun de dissimuler le maximum 
de choses possibles : à la fois la clé (bien sûr) mais aussi le système de chiffrement. 

\change
Mais l'objectif visé par Kerckhoffs est plus académique, 
il pense qu'un système (dépendant d'un secret) dont 
le mécanisme est connu de tous 

sera testé, attaqué, étudié, 

et finalement utilisé s'il s'avère intéressant et robuste.

\change
Bien évidemment, même si le système peut être rendu public, 
la sécurité est basée sur le caractère secret de la clé utilisée dans sa mise en oeuvre.


%%%%%%%%%%%%%%%%%%%%%%%%%%%%%%%%%%%%%%%%%%%%%%%%%%%%%%%%%%%
\diapo


Une question se pose alors : quels sont les outils mathématiques capables de répondre au principe énoné par Kerckoffs ?

=

Nous allons voir qu'il en existe un tout simple et connu de tous : la multiplication ! Voyons comment.

\change

Si je vous demande combien font $5\times7$, vous répondez immédiatement $35$.

\change
Si je vous demande de factoriser $35$, c'est-à-dire de le décomposer en produit de facteurs premiers, vous répondez rapidement $5\times 7$.

\change
Cependant ces deux questions ne sont pas du même ordre de difficulté.

=

Si je vous demande maintenant de factoriser $1591$, vous allez sans doute devoir faire plusieurs tentatives,

\change
alors que si je vous avais directement demandé de calculer $37\times 43$ cela ne vous aurait posé aucun problème.

\change
On s'aperçoit rapidement que calculer le produit de deux nombres premiers $p$ et $q$ est plus facile que retrouver 
$p$ et $q$ à partir du produit $p \times q$.

\change
Formalisons ceci avec la notion de complexité.

La \defi{complexité} estime le temps de calcul (ou le nombre d'opérations élémentaires)
nécessaire pour effectuer une opération.

\change
Commençons par la complexité de l'addition :

\change
disons que calculer la somme de deux chiffres (par exemple $6+8$) soit de complexité $1$
(par exemple $1$ seconde pour un humain, $1$ millionième de seconde pour un ordinateur).

\change
Pour calculer la somme de deux entiers à $n$ chiffres, la complexité est de l'ordre de $n$ (en négligeant les retenues)

\change
Par exemple pour la somme $1234+2323$, il faut faire $4$ additions de chiffres, 
donc environ $4$ secondes pour un humain.

\change

\change
La multiplication de deux entiers à $n$ chiffres est de complexité d'ordre $n^2$. 

\change
Par exemple pour multiplier $1234$ par $2323$ il faut faire
$16$ multiplications de chiffres (et quelques additions)

Chaque chiffre de $1234$ est à multiplier par chaque chiffre
de $2323$.

\change
Par contre la meilleure méthode de factorisation connue est de complexité d'ordre
$\exp(4n^\frac13)$ 

C'est moins que $\exp(n)$, mais lorsque $n$ tend vers $+\infty$ c'est beaucoup plus que $n^d$ pour tout $d$.


%%%%%%%%%%%%%%%%%%%%%%%%%%%%%%%%%%%%%%%%%%%%%%%%%%%%%%%%%%%
\diapo

Voici un tableau donnant une idée de la difficulté croissante à multiplier et factoriser des nombres à $n$ chiffres :

=

Ici pour la multiplication, nous avons une croissance en le carré de $n$.

alors que pour la factorisation nous avons une croissance de type exponentiel.

=

Pour des entiers de plusieurs centaines de chiffres, le problème de la factorisation ne peut être résolu 
en un temps raisonnable, même par un ordinateur.

C'est ce problème, qui n'est pas symétrique (on dit aussi asymétrique), qui est à la base de la sécurité du chiffrement RSA que nous détaillerons plus tard : connaître $p$ et $q$ apporte plus d'information utile que le produit $p\times q$. 

Même si en théorie il est possible à partir du produit $p\times q$ de retrouver $p$ et $q$, 
c'est en pratique impossible lorsque la taille des objets grandit.


%%%%%%%%%%%%%%%%%%%%%%%%%%%%%%%%%%%%%%%%%%%%%%%%%%%%%%%%%%%
\diapo

Il existe d'autres situations mathématiques asymétriques :
que l'on appelle les \defi{fonctions à sens unique}.

\change
En d'autres termes, étant donnée une fonction $f$, 

\change
il est possible connaissant $x$ de calculer <<facilement>> $f(x)$ ;

\change
mais inversement, connaissant un élément de l'ensemble image de $f$, il est <<difficile>> voir impossible 
de retrouver son antécédent. 

\change
Dans le cadre de la cryptographie, posséder une fonction à sens 
unique qui joue le rôle de chiffrement n'a que peu de sens. 

En effet, il est absolument indispensable d'avoir un moyen efficace pour déchiffrer les messages 
transmis après chiffrement. 

On parle alors de \defi{fonction à sens unique à trappe}. C'est-à-dire que le calcul de l'image réciproque est difficile en général sauf si on a accès à une information supplémentaire.

\change
Prenons par exemple le cas de la fonction $f$  : $$x \longmapsto x^3 \pmod{100}.$$

et tentons de résoudre le problème suivant : 

trouver $x$ tel que $x^3 \equiv 11 \pmod{100}$.

\change
On peut pour cela faire une recherche exhaustive, 

c'est-à-dire essayer successivement 
  $0$, $1$, $2$, $3$, ...,  $99$, 
  
\change 
on trouve alors : 

$71^3=357\,911 \equiv 11 \pmod{100}.$
    
\change
Pour ce problème, il existe un trappe, une astuce. Il suffit de considérer la fonction :

$$y \longmapsto y^7 \pmod{100}$$

qui permet de répondre directement à la question !

En effet 

$11^7= 19\,487\,171 \equiv 71 \pmod{100}.$

=

Conclusion : le problème posé est difficile à résoudre, sauf pour ceux qui 
connaissent la trappe secrète. 

%%%%%%%%%%%%%%%%%%%%%%%%%%%%%%%%%%%%%%%%%%%%%%%%%%%%%%%%%%%
\diapo

Avant de poursuivre plus avant, effectuons un petit retour en arrière.

=

Les protocoles étudiés dans les chapitres précédents étaient tous 
des chiffrements à clé privée.

Voici un schéma résumant ce type de chiffrement.

Le message initial M est chiffré par Bruno à l'aide de la clé C.

Le message crypté est ensuite transmis à Alice,

qui à l'aide de la clé C déchiffre le message.


||| Les deux interlocuteurs partagent donc la même clé l'un pour chiffrer et l'autre pour déchiffrer.

Cela pose bien sûr un problème majeur déjà évoqué : Alice et Bruno doivent d'abord échanger cette clé.


%%%%%%%%%%%%%%%%%%%%%%%%%%%%%%%%%%%%%%%%%%%%%%%%%%%%%%%%%%%
\diapo

Les protocoles de chiffrement à clé privée peuvent, de façon imagée, se représenter ainsi.

Tout se passe comme si Alice et Bruno possédaient chacun une clé d'un coffre fort.

Dans ce cas, Bruno peut déposer quand il le souhaite un message qu'il destine à Alice.

Et Alice peut venir en prendre connaissance quand bon lui semble.

%%%%%%%%%%%%%%%%%%%%%%%%%%%%%%%%%%%%%%%%%%%%%%%%%%%%%%%%%%%
\diapo

Passons maintenant au protocole de chiffrement à clé publique. 


L'association des mots <<clé>> et <<publique>> peut paraître incongrue, 

mais il signifie que le principe de chiffrement est accessible à tous 

mais que le déchiffrement nécessite une information supplémentaire : une clé (bien sûr secrète).


De façon imagée, si Bruno veut envoyer un message à Alice,

il le dépose dans la boîte aux lettres d'Alice. 

Et puisqu'Alice est la seule à en avoir la clé, elle sera la seule à prendre connaissance du message de Bruno.

=

Ici la clé publique est symbolisée par la boîte aux lettres, tout le monde peut y déposer
un message, 

la clé qui ouvre la boîte aux lettres est la clé privée, clé qui reste secrète.


%%%%%%%%%%%%%%%%%%%%%%%%%%%%%%%%%%%%%%%%%%%%%%%%%%%%%%%%%%%
\diapo

Voyons maintenant le schéma résumant le chiffrement à clé publique.

Tout d'abord, il y a maintenant deux clés : l'une publique et l'autre privée.

Ces deux clés sont préparées par Alice (pensez à la boîte aux lettres et à sa clé).

Elle prépare une clé publique qu'elle diffuse à tout le monde
et une clé privée qu'elle garde secrète.

Si Bruno veut envoyer un message secret à Alice,

il commence par récupérer la clé publique d'Alice,

il crypte son message à l'aide de cette clé publique.

Il transmet le message crypté à Alice.

Alice est la seule à pouvoir décrypter le message en utilisant sa clé secrète.


%%%%%%%%%%%%%%%%%%%%%%%%%%%%%%%%%%%%%%%%%%%%%%%%%%%%%%%%%%%
\diapo

Les paramètres d'un protocole de \defi{chiffrement à clé publique} sont donc : 

\change
les fonctions de chiffrement et de déchiffrement C et D (qui sont publiques),

\change
la clé publique du destinataire qui va permettre de paramétrer la fonction de chiffrement,

\change
la clé privée du destinataire qui va permettre de paramétrer la fonction de déchiffrement.


\change
Donnons tout de suite un exemple en prenant appui sur la fonction à trappe vue précédemment.

Nous nous posions la question (difficile) de trouver $x$ tel que $x^3 \equiv 11 \pmod{100}$.

Rappelez vous : nous avions trouvé comme réponse $71$ et avions mis en évidence une trappe $7$ ou plus exactement la fonction 
$$y\longmapsto y^{7} \pmod{100}$$
pour calculer directement 71.

\change
Dans ce cadre :

les fonctions de chiffrement et de déchiffrement sont du type $x$ à la puissance quelque chose modulo $100$.

\change
la clé publique d'Alice qui vaut ici $3$

permet de définir complètement la fonction de chiffrement $C$ : 

$x\longmapsto x^3 \pmod{100}.$

\change
Enfin  la clé privée d'Alice, ici $7$,
qui permet de définir complètement la fonction de déchiffrement $D$ : 

$x\longmapsto x^7 \pmod{100}.$


\end{document}
