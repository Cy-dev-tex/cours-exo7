
%%%%%%%%%%%%%%%%%% PREAMBULE %%%%%%%%%%%%%%%%%%


\documentclass[12pt]{article}

\usepackage{amsfonts,amsmath,amssymb,amsthm}
\usepackage[utf8]{inputenc}
\usepackage[T1]{fontenc}
\usepackage[francais]{babel}


% packages
\usepackage{amsfonts,amsmath,amssymb,amsthm}
\usepackage[utf8]{inputenc}
\usepackage[T1]{fontenc}
%\usepackage{lmodern}

\usepackage[francais]{babel}
\usepackage{fancybox}
\usepackage{graphicx}

\usepackage{float}

%\usepackage[usenames, x11names]{xcolor}
\usepackage{tikz}
\usepackage{datetime}

\usepackage{mathptmx}
%\usepackage{fouriernc}
%\usepackage{newcent}
\usepackage[mathcal,mathbf]{euler}

%\usepackage{palatino}
%\usepackage{newcent}


% Commande spéciale prompteur

%\usepackage{mathptmx}
%\usepackage[mathcal,mathbf]{euler}
%\usepackage{mathpple,multido}

\usepackage[a4paper]{geometry}
\geometry{top=2cm, bottom=2cm, left=1cm, right=1cm, marginparsep=1cm}

\newcommand{\change}{{\color{red}\rule{\textwidth}{1mm}\\}}

\newcounter{mydiapo}

\newcommand{\diapo}{\newpage
\hfill {\normalsize  Diapo \themydiapo \quad \texttt{[\jobname]}} \\
\stepcounter{mydiapo}}


%%%%%%% COULEURS %%%%%%%%%%

% Pour blanc sur noir :
%\pagecolor[rgb]{0.5,0.5,0.5}
% \pagecolor[rgb]{0,0,0}
% \color[rgb]{1,1,1}



%\DeclareFixedFont{\myfont}{U}{cmss}{bx}{n}{18pt}
\newcommand{\debuttexte}{
%%%%%%%%%%%%% FONTES %%%%%%%%%%%%%
\renewcommand{\baselinestretch}{1.5}
\usefont{U}{cmss}{bx}{n}
\bfseries

% Taille normale : commenter le reste !
%Taille Arnaud
%\fontsize{19}{19}\selectfont

% Taille Barbara
%\fontsize{21}{22}\selectfont

%Taille François
%\fontsize{25}{30}\selectfont

%Taille Pascal
%\fontsize{25}{30}\selectfont

%Taille Laura
%\fontsize{30}{35}\selectfont


%\myfont
%\usefont{U}{cmss}{bx}{n}

%\Huge
%\addtolength{\parskip}{\baselineskip}
}


% \usepackage{hyperref}
% \hypersetup{colorlinks=true, linkcolor=blue, urlcolor=blue,
% pdftitle={Exo7 - Exercices de mathématiques}, pdfauthor={Exo7}}


%section
% \usepackage{sectsty}
% \allsectionsfont{\bf}
%\sectionfont{\color{Tomato3}\upshape\selectfont}
%\subsectionfont{\color{Tomato4}\upshape\selectfont}

%----- Ensembles : entiers, reels, complexes -----
\newcommand{\Nn}{\mathbb{N}} \newcommand{\N}{\mathbb{N}}
\newcommand{\Zz}{\mathbb{Z}} \newcommand{\Z}{\mathbb{Z}}
\newcommand{\Qq}{\mathbb{Q}} \newcommand{\Q}{\mathbb{Q}}
\newcommand{\Rr}{\mathbb{R}} \newcommand{\R}{\mathbb{R}}
\newcommand{\Cc}{\mathbb{C}} 
\newcommand{\Kk}{\mathbb{K}} \newcommand{\K}{\mathbb{K}}

%----- Modifications de symboles -----
\renewcommand{\epsilon}{\varepsilon}
\renewcommand{\Re}{\mathop{\text{Re}}\nolimits}
\renewcommand{\Im}{\mathop{\text{Im}}\nolimits}
%\newcommand{\llbracket}{\left[\kern-0.15em\left[}
%\newcommand{\rrbracket}{\right]\kern-0.15em\right]}

\renewcommand{\ge}{\geqslant}
\renewcommand{\geq}{\geqslant}
\renewcommand{\le}{\leqslant}
\renewcommand{\leq}{\leqslant}

%----- Fonctions usuelles -----
\newcommand{\ch}{\mathop{\mathrm{ch}}\nolimits}
\newcommand{\sh}{\mathop{\mathrm{sh}}\nolimits}
\renewcommand{\tanh}{\mathop{\mathrm{th}}\nolimits}
\newcommand{\cotan}{\mathop{\mathrm{cotan}}\nolimits}
\newcommand{\Arcsin}{\mathop{\mathrm{Arcsin}}\nolimits}
\newcommand{\Arccos}{\mathop{\mathrm{Arccos}}\nolimits}
\newcommand{\Arctan}{\mathop{\mathrm{Arctan}}\nolimits}
\newcommand{\Argsh}{\mathop{\mathrm{Argsh}}\nolimits}
\newcommand{\Argch}{\mathop{\mathrm{Argch}}\nolimits}
\newcommand{\Argth}{\mathop{\mathrm{Argth}}\nolimits}
\newcommand{\pgcd}{\mathop{\mathrm{pgcd}}\nolimits} 

\newcommand{\Card}{\mathop{\text{Card}}\nolimits}
\newcommand{\Ker}{\mathop{\text{Ker}}\nolimits}
\newcommand{\id}{\mathop{\text{id}}\nolimits}
\newcommand{\ii}{\mathrm{i}}
\newcommand{\dd}{\mathrm{d}}
\newcommand{\Vect}{\mathop{\text{Vect}}\nolimits}
\newcommand{\Mat}{\mathop{\mathrm{Mat}}\nolimits}
\newcommand{\rg}{\mathop{\text{rg}}\nolimits}
\newcommand{\tr}{\mathop{\text{tr}}\nolimits}
\newcommand{\ppcm}{\mathop{\text{ppcm}}\nolimits}

%----- Structure des exercices ------

\newtheoremstyle{styleexo}% name
{2ex}% Space above
{3ex}% Space below
{}% Body font
{}% Indent amount 1
{\bfseries} % Theorem head font
{}% Punctuation after theorem head
{\newline}% Space after theorem head 2
{}% Theorem head spec (can be left empty, meaning ‘normal’)

%\theoremstyle{styleexo}
\newtheorem{exo}{Exercice}
\newtheorem{ind}{Indications}
\newtheorem{cor}{Correction}


\newcommand{\exercice}[1]{} \newcommand{\finexercice}{}
%\newcommand{\exercice}[1]{{\tiny\texttt{#1}}\vspace{-2ex}} % pour afficher le numero absolu, l'auteur...
\newcommand{\enonce}{\begin{exo}} \newcommand{\finenonce}{\end{exo}}
\newcommand{\indication}{\begin{ind}} \newcommand{\finindication}{\end{ind}}
\newcommand{\correction}{\begin{cor}} \newcommand{\fincorrection}{\end{cor}}

\newcommand{\noindication}{\stepcounter{ind}}
\newcommand{\nocorrection}{\stepcounter{cor}}

\newcommand{\fiche}[1]{} \newcommand{\finfiche}{}
\newcommand{\titre}[1]{\centerline{\large \bf #1}}
\newcommand{\addcommand}[1]{}
\newcommand{\video}[1]{}

% Marge
\newcommand{\mymargin}[1]{\marginpar{{\small #1}}}



%----- Presentation ------
\setlength{\parindent}{0cm}

%\newcommand{\ExoSept}{\href{http://exo7.emath.fr}{\textbf{\textsf{Exo7}}}}

\definecolor{myred}{rgb}{0.93,0.26,0}
\definecolor{myorange}{rgb}{0.97,0.58,0}
\definecolor{myyellow}{rgb}{1,0.86,0}

\newcommand{\LogoExoSept}[1]{  % input : echelle
{\usefont{U}{cmss}{bx}{n}
\begin{tikzpicture}[scale=0.1*#1,transform shape]
  \fill[color=myorange] (0,0)--(4,0)--(4,-4)--(0,-4)--cycle;
  \fill[color=myred] (0,0)--(0,3)--(-3,3)--(-3,0)--cycle;
  \fill[color=myyellow] (4,0)--(7,4)--(3,7)--(0,3)--cycle;
  \node[scale=5] at (3.5,3.5) {Exo7};
\end{tikzpicture}}
}



\theoremstyle{definition}
%\newtheorem{proposition}{Proposition}
%\newtheorem{exemple}{Exemple}
%\newtheorem{theoreme}{Théorème}
\newtheorem{lemme}{Lemme}
\newtheorem{corollaire}{Corollaire}
%\newtheorem*{remarque*}{Remarque}
%\newtheorem*{miniexercice}{Mini-exercices}
%\newtheorem{definition}{Définition}




%definition d'un terme
\newcommand{\defi}[1]{{\color{myorange}\textbf{\emph{#1}}}}
\newcommand{\evidence}[1]{{\color{blue}\textbf{\emph{#1}}}}



 %----- Commandes divers ------

\newcommand{\codeinline}[1]{\texttt{#1}}

%%%%%%%%%%%%%%%%%%%%%%%%%%%%%%%%%%%%%%%%%%%%%%%%%%%%%%%%%%%%%
%%%%%%%%%%%%%%%%%%%%%%%%%%%%%%%%%%%%%%%%%%%%%%%%%%%%%%%%%%%%%



\begin{document}

\debuttexte


%%%%%%%%%%%%%%%%%%%%%%%%%%%%%%%%%%%%%%%%%%%%%%%%%%%%%%%%%%%
\diapo

\change

Nous allons voir des moyens de plus en plus performants de crypter des messages.

\change

Nous allons même commencer par un système inviolable mais qui 
malheureusement n'est pas utilisable dans la pratique.

\change

La célèbre machine Enigma était une machine électro-mécanique qui tentait d'imiter ce chiffrement parfait, nous en verrons le principe.

\change

Nous terminerons en expliquant le principe du chiffrement DES.
%, qui tentait cette imitation numériquement.

%%%%%%%%%%%%%%%%%%%%%%%%%%%%%%%%%%%%%%%%%%%%%%%%%%%%%%%%%%%
\diapo

L'inconvénient des chiffrements précédemment étudiés est qu'une même lettre est régulièrement chiffrée de la même façon.


En effet, la correspondance d'un alphabet à un ou plusieurs autres est fixée une fois pour toutes,

\change

ce qui fait qu'une attaque statistique est toujours possible.

\change

Nous allons voir qu'en changeant la correspondance à chaque lettre, il est possible de créer un chiffrement 
parfait !

\change

Expliquons d'abord le principe à l'aide d'une simple analogie :

\change

Imaginons que j'ai choisi deux entiers dont la somme est $100$,

et que je vous pose la question : Pouvez-vous me donner l'un d'entre eux ? 

\change

Il vous est bien sûr impossible de répondre à une telle question car elle possède théoriquement plusieurs solutions :

\change

$0+100$, $1+99$, $2+98$,...


\change

Par contre, si je vous donne un indice, l'un des entiers, par exemple $c$ alors vous trouverez l'autre immédiatement en effectuant l'opération $100-c$.


%%%%%%%%%%%%%%%%%%%%%%%%%%%%%%%%%%%%%%%%%%%%%%%%%%%%%%%%%%%
\diapo

Voyons le principe du chiffrement parfait sur un exemple.


Supposons qu'Alice veuille envoyer à Bruno le message secret $M$ suivant : 
$${ATTAQUE \ LE \ CHATEAU}$$


\change

Avant de procéder au chiffrement du message Alice choisit une clé secrète $C$ 

\change

qu'elle transmet à Bruno.

\change

La première contrainte est que cette clé secrète soit de même longueur que le message.

((Il est donc important qu'Alice choisisse cette clé après avoir rédigé son message.))

\change

Cette clé est composée d'entiers de $0$ à $25$. 

Mais ce qui est le plus important c'est que ces entiers apparaissent comme s'ils avaient été tirés au hasard. 

\change

Prenons par exemple pour clé $C$ la suite : 
$${[4, 18, 2, 0, 21, 12, 18, 13, 7, 11, 23, 22, 19, 2, 16, 9]}$$


\change

Voici maintenant comment Alice crypte son message.

En ce qui concerne la première lettre, elle effectue un décalage de type César correspondant au premier entier de la clé $C$.

La lettre A est décalée de $4$ lettres et devient donc E.

\change

La deuxième lettre subit un décalage correspondant au deuxième entier : le premier T devient L.

\change

Le second T du message est quant à lui décalé de $2$ lettres, il devient V. 

\change

Le A suivant est décalé de $0$ lettre, cela reste A...

\change

A la fin Alice obtient le message crypté $X$ qu'elle transmet à Bruno : 
$${ELVALGW \ YL \ NEWMGQD}$$

=

|| Pour le décrypter, Bruno, qui connaît la clé, n'a qu'à appliquer les décalages dans l'autre sens comme nous avons maintenant l'habitude de le faire.

=

Notez bien que deux lettres identiques n'ont aucune raison d'être cryptées de la même façon. 

Par exemple, les lettres T du message initial sont cryptés dans l'ordre par un L, un V et un M.


%%%%%%%%%%%%%%%%%%%%%%%%%%%%%%%%%%%%%%%%%%%%%%%%%%%%%%%%%%%
\diapo

Formalisons un peu cette opération. 

=

On identifie comme depuis le début de ce cours $A$ avec $0$, $B$ avec $1$, ..., $Z$ avec $25$.

\change

Alors le message crypté $X$ est juste une somme du message $M$ avec la clé secrète $C$, 
la somme s'effectuant terme à terme modulo $26$.


Notons cette opération $X=M \oplus C$.

\change

Détaillons les opérations : 

On transforme d'abord ATTAQUE en entiers.

On obtient $0$ $19$ $19$ $0$ $16$ $20$ $4$

\change

On ajoute colonne par colonne la clé $4$ $18$ $2$ $0$ $21$ $12$ $18$ 

ici en fait le début de la clé puisque nous nous intéressons au début du message

\change

On obtient le résultat $4$ $11$ $21$ $0$ $11$ $6$ $22$

qui correspond aux lettres ELVALGW.


\change

Bruno reçoit le message crypté $X$ et il connaît $C$, il effectue donc la soustraction terme à terme modulo $26$ :
$X \ominus C$ et retrouve le message $M$.

=

|| Pourquoi ce système est-il inviolable ? 

Pour chacune des lettres, la personne qui tente de déchiffrer le message est 
face à la question posée précédemment, à savoir, 
trouver deux nombres en connaissant seulement leur somme.

Toutes les possibilités s'offrent à lui.

Il est donc dans l'impossibilité d'atteindre son objectif.

\change

Il y a néanmoins trois principes essentiels à respecter pour que ce système soit inviolable :

\change

La longueur de la clé doit être  égale à la longueur du message.
  
\change

Cette clé doit être choisie au hasard
  
\change

et elle ne doit servir qu'une seule fois.

\change

Ce système est parfait en théorie, mais ne l'est pas du tout en pratique !

\change

Tout d'abord il faut que la clé soit aussi longue que le message. Pour un message 
court cela ne pose pas de problème, mais pour crypter une image par exemple cela devient très lourd.

\change

Ensuite, pour chaque message, Alice doit posséder un moyen sécurisé d'envoyer la clé secrète à Bruno 
avant même de lui faire parvenir le message. 

Cela n'a pas tellement de sens, à moins de s'être mis d'accord au préalable sur une longue liste de clés secrètes.



%%%%%%%%%%%%%%%%%%%%%%%%%%%%%%%%%%%%%%%%%%%%%%%%%%%%%%%%%%%
\diapo

Dans le système décrit précédemment, un des points de sécurité repose sur le fait de générer efficacement de longues clés, comme si elles avaient
été tirées au hasard. 

Nous allons étudier deux systèmes de chiffrement qui s'inspirent de cette idée : 

- une méthode électro-mécanique -- la machine Enigma --

- puis une méthode numérique -- le DES --.

\change

=

La machine Enigma est une machine électro-mécanique qui ressemble à une machine à écrire.

\change

Lorsque qu'une touche est enfoncée, des disques internes sont actionnés, un courant circule et le caractère crypté s'allume.

\change

Cette machine, qui sert aussi au déchiffrement, était utilisée pour les communications de l'armée 
allemande durant la seconde guerre mondiale. 

\change

Ce que les 
Allemands ne savaient pas, c'est que les services secrets polonais et britanniques avaient réussi à en percer les mystères.

\change

Ce long travail d'études et de recherches a nécessité tout le génie 
d'Alan Turing à la fois mathématicien, cryptologue et informaticien 
mais aussi des innovations technologiques avec la construction de calculateurs programmables, 
ancêtres de nos ordinateurs.

%%%%%%%%%%%%%%%%%%%%%%%%%%%%%%%%%%%%%%%%%%%%%%%%%%%%%%%%%%%
\diapo

Nous allons voir une version très simplifiée de la machine Enigma. 

=

Notre machine simplifiée est dans un premier temps composée de deux anneaux, cela doit vous rappeler quelque chose.


\change

Tout d'abord un anneau extérieur contenant l'alphabet "ABCDE..." en clair, 
ce sont les touches du clavier avec lesquelles est tapé le message.

Cet anneau est fixe.
  
\change

Le deuxième anneau représenté ici à l'intérieur du premier contient 
l'alphabet ayant subi une permutation, 
ici "NOJAK..." jusqu'à "GWQRUEY". 

\change

Bien sûr l'ordre de l’alphabet sur cet anneau intérieur doit rester secret.

\change

Cet anneau intérieur est mobile et tourne d'un cran chaque fois qu'une touche a été enfoncée. 

%%%%%%%%%%%%%%%%%%%%%%%%%%%%%%%%%%%%%%%%%%%%%%%%%%%%%%%%%%%
\diapo
Voyons le processus de chiffrement du mot "BAC" avec la clé de chiffrement 
"NOJAK..." dont on fixe le point de départ, par exemple ici la lettre $G$.

\change

C'est-à-dire que l'opérateur va dans un premier temps tourner l'anneau intérieur de sorte que la lettre A de l'anneau extérieur et fixe 
  soit en face de la lettre G de l'anneau intérieur mobile (et donc B en face de W,...).

C'est la phase de mise en place des anneaux.
 
\change

Pour crypter la première lettre, l'opérateur tape la première lettre du message en clair ici le B de "BAC": 
la machine affiche la correspondance, ici W.

\change

L'anneau intérieur tourne alors de $1/26$ème de tour, 

=

maintenant le A extérieur se retrouve en face du W, la lettre B en face de la lettre Q,...

et on est prêt pour crypter la deuxième lettre.

%%%%%%%%%%%%%%%%%%%%%%%%%%%%%%%%%%%%%%%%%%%%%%%%%%%%%%%%%%%
\diapo

Nous avons crypté la première lettre $B$ puis l'anneau intérieur a tourné d'un cran.

\change 

Pour crypter la deuxième lettre A, l'opérateur la tape au clavier :

la machine affiche la correspondance, c'est de nouveau W.

\change

L'anneau intérieur tourne une nouvelle fois de $1/26$ème de tour, 
maintenant la lettre A à l'extérieur est en face de la lettre Q, 
le B en face du R, le C en face du U,...

\change

L'opérateur tape la troisième lettre du message C : 
la machine affiche la correspondance U et effectue sa rotation.
   
\change

Le message crypté est donc "WWU"
   
%Il n'y a pas d'attaque statistique possible, 
%ici les deux W correspondent à des lettres différentes.

|| Ce processus de chiffrement n'est autre qu'un chiffrement par bloc comme nous l'avons vu dans les chapitres précédents.

=

Mais l'intérêt de cette présentation est de permettre immédiatement des généralisations.

\diapo

La machine Enigma ne possédait pas un seul anneau interne mobile, nous avons vu que cela revenait à un chiffrement par bloc de type Vigenère. 

%Elle était beaucoup plus sophistiquée et possédait notamment plusieurs anneaux intérieurs.

Par exemple pour deux anneaux (ici le premier en jaune et le second en vert) 
placés ici dans leur position initiale : 

\change 

B s'envoie sur W par le premier anneau, qui lui-même est envoyé sur A par le second : 
la lettre B est donc cryptée est A.

\change 

Ensuite l'anneau intérieur numéro 1, effectue $1/26$ème de tour et on continue 
à parcourir le message.

Pour un message plus long que notre exemple, lorsque l'anneau intérieur (ici en clair)
a effectué une rotation complète ($26$ lettres ont été tapées) 
l'anneau intérieur (ici en vert) effectue $1/26$ème
de tour.

||  C'est comme sur un compteur kilométrique de voiture :
à chaque kilomètre le chiffre des unités tourne,
tous les $10$ kilomètres le chiffre des dizaines tourne aussi,...

-- Ici, le disque en jaune joue le rôle des unités et le disque en vert celui des dizaines.

=

|| Cette situation se généralise à plusieurs anneaux augmentant 
la complexité et éloignant ainsi les possibilités d'attaque du système.


%Ce système était rendu largement plus compliqué pendant la guerre.
%Le nombre de clés possibles dépassait plusieurs milliards de milliards !

% 
% %%%%%%%%%%%%%%%%%%%%%%%%%%%%%%%%%%%%%%%%%%%%%%%%%%%%%%%%%%%
% \diapo
% 
% Les ordinateurs ne connaissent en fait que deux nombres : $0$ et $1$.
% 
% \change
% l'addition est celle de $\Zz/2\Zz$ :
% 
% \change
% $0+0=0 \qquad 0+1=1 \qquad 1+0=1$
% 
% mais attention $1+1=0$.
% 
% \change
% Si on a deux mots $m$ et $c$ composés de $0$ et de $1$ 
% 
% \change
% alors on peut définir l'addition $m \oplus c$, cette addition 
% se fait \emph{bit} par \emph{bit} (c'est-à-dire colonne par colonne, sans retenue).
% 
% \change
% Exemple $m = 10101010$, $c = 01110011$, notons $x= m \oplus c$ :
% 
% \change
% on dispose l'addition ainsi :
% 
% \change
% et on trouve 
% 
% $x = 11011001$.
% 
% Souvenez-vous, on travail colonne par colonne, sans retenue,
% et $1+1=0$.
% 
% \change
% Pour nous, $m$ est le message en clair, $c$ la clé de cryptage et $x$ le message crypté.
% 
% \change
% Comment déchiffrer le message ? Connaissant $x$ et $c$, c'est très facile. 
% 
% \change
% 
% \change
% Comme $1+1=0$, alors $-1=+1$ donc ajouter ou soustraire est la même opération !
% Pour décrypter le message il suffit donc de calculer $x\oplus c$ et on retrouve $m$.
% 
% \change
% 
% On peut aussi le prouver de la façon suivante : comme $0+0=0$ et $1+1=0$ alors
% quelque soit $c$, $c \oplus c=0$. 
% 
% Donc
% $$x \oplus c = (m \oplus c) \oplus c = m \oplus (c \oplus c) = m \oplus 0 = m.$$
% 
% Vérifiez-le sur l'exemple ci-dessus.
% 


%%%%%%%%%%%%%%%%%%%%%%%%%%%%%%%%%%%%%%%%%%%%%%%%%%%%%%%%%%%
\diapo


La machine Enigma génère mécaniquement un alphabet différent à chaque caractère crypté, 
tentant ainsi de se rapprocher d'un chiffrement parfait.

\change

Nous allons voir une autre méthode, cette fois numérique : le DES. 

Le DES est un protocole de chiffrement par blocs qui a été comme son nom l'indique le standard de chiffrement américain et international de la fin des années 70 jusqu'en 2001.

\change

Commençons par rappeler que l'objectif est de pouvoir utiliser une clé très longue.

\change

Pour ne pas avoir à retenir l'intégralité de cette longue clé, nous allons tenter de la produire de façon presque aléatoire, on dit alors pseudo-aléatoire, à partir d'une clé de petite taille.

\change

Voyons un exemple élémentaire qui utilise la notion de suite récurrentes.

Soit $(u_n)$ la suite définie par la donnée de deux entiers $a$ et $b$, de son premier terme $u_0$ et de la relation de récurrence
$$u_{n+1} \equiv a\times u_n + b \pmod{26}.$$

\change

Par exemple pour $a=2$, $b=5$ et $u_0=6$, 

\change

nous avons
$$u_0=6$$

$$u_1=2\times 6+5=17 \pmod{26}$$

$$u_2=2\times 17+5= 39 = 13 \pmod{26} \qquad \mbox{etc.}$$

$$(6 \quad 17\quad 13)\quad 5\quad 15 \quad9\quad 23\quad 25\quad 3 \quad11\quad 1 \quad7\quad 19\quad 17\quad 13\quad 5$$

\change

Les trois nombres $(a,b,u_0)$ représentent la clé principale et 

\change

les éléments de la suite $(u_n)_{n\in\Nn}$ les clés secondaires ou vu sous un autre angle une très longue clé.

=

|| L'avantage de cette construction est qu'à partir d'une clé principale on a généré une longue voir très longue liste de clés secondaires. 

Mais l'inconvénient est que cette liste n'est pas si aléatoire que cela. 

En effet, elle se répète, elle est periodique à partir d'un certain rang, sa période est de longueur $12$ et constituée de : 
$$17,\quad 13,\quad 5,\quad 15,\quad 9,\quad 23,\quad 25,\quad 3,\quad 11,\quad 1,\quad 7,\quad 19$$


%%%%%%%%%%%%%%%%%%%%%%%%%%%%%%%%%%%%%%%%%%%%%%%%%%%%%%%%%%%
\diapo

%Le système DES est une version sophistiquée de ce processus :
%à partir d'une clé courte et d'opérations élémentaires on crypte un message.
Comme lors de l'étude de la machine Enigma, nous allons présenter une 
version très simplifiée du DES afin d'en expliquer les étapes élémentaires.

\change

Les additions se feront \emph{terme} à \emph{terme}.

\change

Pour changer, nous allons ici travailler modulo $10$.

\change

Par exemple :
$[1 \ 2 \ 3 \  4] \oplus [7\  8\  9\  0] = [8\  0\  2\  4]$ 

car ($1+7 \equiv 8 \pmod{10}$,
$2+8\equiv 0 \pmod{10}$,...)

\change

Notre message est coupé en blocs, pour nos explications 
ce seront des blocs de longueur $8$. 

\change

La clé sera ici de longueur $4$.

\change

Voici enfin le message (pour notre exemple nous prenons un seul bloc): 
$M = [1\  2\  3\  4\  5\  6\  7\  8]$ 

\change

et la clé : $C=[3\  1\  3\  2]$.

%%%%%%%%%%%%%%%%%%%%%%%%%%%%%%%%%%%%%%%%%%%%%%%%%%%%%%%%%%%
\diapo

((Après avoir défini notre cadre de travail, nous allons expliciter les transformations 
subies par le message initial.))

Voyons la première étape ou le premier tour : 

--Nous partons de $M_0$ le message initial que nous avons découpé en deux parties, 
une partie gauche et une partie droite :
$$M_0 = [ G_0 \ \|\  D_0] = [1\  2\  3\  4\  \| \ 5\  6\  7\  8]$$

\change 

Nous souhaitons obtenir le message $M_1$ défini par
$$M_1 = [ D_0 \ \|\  C \oplus \sigma(G_0)]$$
où $\sigma$ est une permutation circulaire.


On effectue pour cela trois opérations pour passer de $M_0$ à $M_1$ :


\change

Tout d'abord on échange la partie droite et la partie gauche de $M_0$ :
  
\change  

$M_0  \longmapsto  [5\  6\  7\  8\  \|\  1\  2\  3\  4]$


\change

Ensuite sur la nouvelle partie droite, on permute circulairement les nombres, c'est la signification de $\sigma$.
  
\change  

$\longmapsto  [5\  6\  7\  8\  \| 2\  3\  4\  1]$

\change 

Enfin on ajoute la clé secrète $C$ à la partie droite (je vous rappelle que 
$C=[3\  1\  3\  2]$).

\change 

pour obtenir le nouveau message $M_1 =  [5\  6\  7\  8\  \| \ 5\  4\  7\  3]$.


%%%%%%%%%%%%%%%%%%%%%%%%%%%%%%%%%%%%%%%%%%%%%%%%%%%%%%%%%%%
\diapo

Pour la suite de notre protocole, nous allons recommencer le même processus en partant du message $M_1$. 

=

Cela revient à appliquer la formule de récurrence,
qui partant de $M_i=[G_i \ \| \ D_i]$ (découpé en deux parties), définit
$$M_{i+1} = [ D_i \ \| \ C \oplus \sigma(G_i)]$$


\change

On part donc de $M_1 = [5\  6\  7\  8\  \| 5\  4\  7\  3]$.

Et on applique le processus de récurrence (toujours en trois opérations).


\change

On échange la partie droite et la partie gauche de $M_1$ :
$$M_1  \longmapsto  [5\  4\  7\  3\  \| \ 5\  6\  7\  8]$$

\change

Sur la nouvelle partie droite, on applique $\sigma$, la permutation circulaire.
$$\longmapsto  [5\  4\  7\  3\  \| \ 6\  7\  8\  5]$$

\change

Puis on ajoute la clé secrète $C$ à droite pour obtenir $M_2$.
$$\longmapsto  [5\  4\  7\  3\  \| \ 9\  8\  1\  7] = M_2$$


\change

On peut décider de s'arrêter après ce tour et renvoyer 
le message crypté $X=M_2 = [5\  4\  7\  3\ 9\  8\  1\  7]$. 

=

Comme chaque opération élémentaire est inversible, 
il est élémentaire de définir le protocole qui permet de déchiffrer un message.

=

Dans la version complète du DES les calculs sont effectués modulo 2 et clé et blocs sont de longueur $64$.

Outre quelques transformations des messages initiaux et finaux, le processus de récurrence est appliqué 16 fois. 

De plus, à chaque tour, on utilise une clé différente.

Il convient donc, à l'image de ce que nous avons vu précédemment, de générer 16 clés (dites secondaires) à partir de la clé principale.



\end{document}
