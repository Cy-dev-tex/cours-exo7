
%%%%%%%%%%%%%%%%%% PREAMBULE %%%%%%%%%%%%%%%%%%


\documentclass[12pt]{article}

\usepackage{amsfonts,amsmath,amssymb,amsthm}
\usepackage[utf8]{inputenc}
\usepackage[T1]{fontenc}
\usepackage[francais]{babel}


% packages
\usepackage{amsfonts,amsmath,amssymb,amsthm}
\usepackage[utf8]{inputenc}
\usepackage[T1]{fontenc}
%\usepackage{lmodern}

\usepackage[francais]{babel}
\usepackage{fancybox}
\usepackage{graphicx}

\usepackage{float}

%\usepackage[usenames, x11names]{xcolor}
\usepackage{tikz}
\usepackage{datetime}

\usepackage{mathptmx}
%\usepackage{fouriernc}
%\usepackage{newcent}
\usepackage[mathcal,mathbf]{euler}

%\usepackage{palatino}
%\usepackage{newcent}


% Commande spéciale prompteur

%\usepackage{mathptmx}
%\usepackage[mathcal,mathbf]{euler}
%\usepackage{mathpple,multido}

\usepackage[a4paper]{geometry}
\geometry{top=2cm, bottom=2cm, left=1cm, right=1cm, marginparsep=1cm}

\newcommand{\change}{{\color{red}\rule{\textwidth}{1mm}\\}}

\newcounter{mydiapo}

\newcommand{\diapo}{\newpage
\hfill {\normalsize  Diapo \themydiapo \quad \texttt{[\jobname]}} \\
\stepcounter{mydiapo}}


%%%%%%% COULEURS %%%%%%%%%%

% Pour blanc sur noir :
%\pagecolor[rgb]{0.5,0.5,0.5}
% \pagecolor[rgb]{0,0,0}
% \color[rgb]{1,1,1}



%\DeclareFixedFont{\myfont}{U}{cmss}{bx}{n}{18pt}
\newcommand{\debuttexte}{
%%%%%%%%%%%%% FONTES %%%%%%%%%%%%%
\renewcommand{\baselinestretch}{1.5}
\usefont{U}{cmss}{bx}{n}
\bfseries

% Taille normale : commenter le reste !
%Taille Arnaud
%\fontsize{19}{19}\selectfont

% Taille Barbara
%\fontsize{21}{22}\selectfont

%Taille François
%\fontsize{25}{30}\selectfont

%Taille Pascal
%\fontsize{25}{30}\selectfont

%Taille Laura
%\fontsize{30}{35}\selectfont


%\myfont
%\usefont{U}{cmss}{bx}{n}

%\Huge
%\addtolength{\parskip}{\baselineskip}
}


% \usepackage{hyperref}
% \hypersetup{colorlinks=true, linkcolor=blue, urlcolor=blue,
% pdftitle={Exo7 - Exercices de mathématiques}, pdfauthor={Exo7}}


%section
% \usepackage{sectsty}
% \allsectionsfont{\bf}
%\sectionfont{\color{Tomato3}\upshape\selectfont}
%\subsectionfont{\color{Tomato4}\upshape\selectfont}

%----- Ensembles : entiers, reels, complexes -----
\newcommand{\Nn}{\mathbb{N}} \newcommand{\N}{\mathbb{N}}
\newcommand{\Zz}{\mathbb{Z}} \newcommand{\Z}{\mathbb{Z}}
\newcommand{\Qq}{\mathbb{Q}} \newcommand{\Q}{\mathbb{Q}}
\newcommand{\Rr}{\mathbb{R}} \newcommand{\R}{\mathbb{R}}
\newcommand{\Cc}{\mathbb{C}} 
\newcommand{\Kk}{\mathbb{K}} \newcommand{\K}{\mathbb{K}}

%----- Modifications de symboles -----
\renewcommand{\epsilon}{\varepsilon}
\renewcommand{\Re}{\mathop{\text{Re}}\nolimits}
\renewcommand{\Im}{\mathop{\text{Im}}\nolimits}
%\newcommand{\llbracket}{\left[\kern-0.15em\left[}
%\newcommand{\rrbracket}{\right]\kern-0.15em\right]}

\renewcommand{\ge}{\geqslant}
\renewcommand{\geq}{\geqslant}
\renewcommand{\le}{\leqslant}
\renewcommand{\leq}{\leqslant}

%----- Fonctions usuelles -----
\newcommand{\ch}{\mathop{\mathrm{ch}}\nolimits}
\newcommand{\sh}{\mathop{\mathrm{sh}}\nolimits}
\renewcommand{\tanh}{\mathop{\mathrm{th}}\nolimits}
\newcommand{\cotan}{\mathop{\mathrm{cotan}}\nolimits}
\newcommand{\Arcsin}{\mathop{\mathrm{Arcsin}}\nolimits}
\newcommand{\Arccos}{\mathop{\mathrm{Arccos}}\nolimits}
\newcommand{\Arctan}{\mathop{\mathrm{Arctan}}\nolimits}
\newcommand{\Argsh}{\mathop{\mathrm{Argsh}}\nolimits}
\newcommand{\Argch}{\mathop{\mathrm{Argch}}\nolimits}
\newcommand{\Argth}{\mathop{\mathrm{Argth}}\nolimits}
\newcommand{\pgcd}{\mathop{\mathrm{pgcd}}\nolimits} 

\newcommand{\Card}{\mathop{\text{Card}}\nolimits}
\newcommand{\Ker}{\mathop{\text{Ker}}\nolimits}
\newcommand{\id}{\mathop{\text{id}}\nolimits}
\newcommand{\ii}{\mathrm{i}}
\newcommand{\dd}{\mathrm{d}}
\newcommand{\Vect}{\mathop{\text{Vect}}\nolimits}
\newcommand{\Mat}{\mathop{\mathrm{Mat}}\nolimits}
\newcommand{\rg}{\mathop{\text{rg}}\nolimits}
\newcommand{\tr}{\mathop{\text{tr}}\nolimits}
\newcommand{\ppcm}{\mathop{\text{ppcm}}\nolimits}

%----- Structure des exercices ------

\newtheoremstyle{styleexo}% name
{2ex}% Space above
{3ex}% Space below
{}% Body font
{}% Indent amount 1
{\bfseries} % Theorem head font
{}% Punctuation after theorem head
{\newline}% Space after theorem head 2
{}% Theorem head spec (can be left empty, meaning ‘normal’)

%\theoremstyle{styleexo}
\newtheorem{exo}{Exercice}
\newtheorem{ind}{Indications}
\newtheorem{cor}{Correction}


\newcommand{\exercice}[1]{} \newcommand{\finexercice}{}
%\newcommand{\exercice}[1]{{\tiny\texttt{#1}}\vspace{-2ex}} % pour afficher le numero absolu, l'auteur...
\newcommand{\enonce}{\begin{exo}} \newcommand{\finenonce}{\end{exo}}
\newcommand{\indication}{\begin{ind}} \newcommand{\finindication}{\end{ind}}
\newcommand{\correction}{\begin{cor}} \newcommand{\fincorrection}{\end{cor}}

\newcommand{\noindication}{\stepcounter{ind}}
\newcommand{\nocorrection}{\stepcounter{cor}}

\newcommand{\fiche}[1]{} \newcommand{\finfiche}{}
\newcommand{\titre}[1]{\centerline{\large \bf #1}}
\newcommand{\addcommand}[1]{}
\newcommand{\video}[1]{}

% Marge
\newcommand{\mymargin}[1]{\marginpar{{\small #1}}}



%----- Presentation ------
\setlength{\parindent}{0cm}

%\newcommand{\ExoSept}{\href{http://exo7.emath.fr}{\textbf{\textsf{Exo7}}}}

\definecolor{myred}{rgb}{0.93,0.26,0}
\definecolor{myorange}{rgb}{0.97,0.58,0}
\definecolor{myyellow}{rgb}{1,0.86,0}

\newcommand{\LogoExoSept}[1]{  % input : echelle
{\usefont{U}{cmss}{bx}{n}
\begin{tikzpicture}[scale=0.1*#1,transform shape]
  \fill[color=myorange] (0,0)--(4,0)--(4,-4)--(0,-4)--cycle;
  \fill[color=myred] (0,0)--(0,3)--(-3,3)--(-3,0)--cycle;
  \fill[color=myyellow] (4,0)--(7,4)--(3,7)--(0,3)--cycle;
  \node[scale=5] at (3.5,3.5) {Exo7};
\end{tikzpicture}}
}



\theoremstyle{definition}
%\newtheorem{proposition}{Proposition}
%\newtheorem{exemple}{Exemple}
%\newtheorem{theoreme}{Théorème}
\newtheorem{lemme}{Lemme}
\newtheorem{corollaire}{Corollaire}
%\newtheorem*{remarque*}{Remarque}
%\newtheorem*{miniexercice}{Mini-exercices}
%\newtheorem{definition}{Définition}




%definition d'un terme
\newcommand{\defi}[1]{{\color{myorange}\textbf{\emph{#1}}}}
\newcommand{\evidence}[1]{{\color{blue}\textbf{\emph{#1}}}}



 %----- Commandes divers ------

\newcommand{\codeinline}[1]{\texttt{#1}}

%%%%%%%%%%%%%%%%%%%%%%%%%%%%%%%%%%%%%%%%%%%%%%%%%%%%%%%%%%%%%
%%%%%%%%%%%%%%%%%%%%%%%%%%%%%%%%%%%%%%%%%%%%%%%%%%%%%%%%%%%%%



\begin{document}

\debuttexte


%%%%%%%%%%%%%%%%%%%%%%%%%%%%%%%%%%%%%%%%%%%%%%%%%%%%%%%%%%%
\diapo

\change

Dans cette partie nous présentons les outils mathématiques
nécessaires pour la mise en place du cryptosystème RSA.

\change

Nous revenons sur le petit théorème de Fermat et en donnons une version 
améliorée.

\change

Même chose avec l'algorithme d'Euclide

\change
qui dans sa version étendue permet le calcul d'inverse modulo $n$

\change
On termine avec une méthode de calcul rapide de puissances modulaires.


%%%%%%%%%%%%%%%%%%%%%%%%%%%%%%%%%%%%%%%%%%%%%%%%%%%%%%%%%%%
\diapo

Nous connaissons le petit théorème de Fermat 


$\equiv \equiv \equiv$
Si $p$ est un nombre premier et $a$ est un entier quelconque alors

$a^p \equiv a \pmod p$

\change

Son corollaire est tout aussi utile :

Maintenant, si le nombre premier $p$ ne divise pas l'entier $a$ alors

on peut diviser l'égalité précédente par $a$ pour obtenir :

$a^{p-1} \equiv 1 \pmod p$


%%%%%%%%%%%%%%%%%%%%%%%%%%%%%%%%%%%%%%%%%%%%%%%%%%%%%%%%%%%
\diapo

Nous allons voir une version améliorée du petit théorème de Fermat 
dans le cas qui nous intéresse :


Soient maintenant $p$ et $q$ deux nombres premiers distincts et soit $n$ le produit $p q$. 

Pour n'importe quel entier $a$ vérifiant $\pgcd(a,n)=1$ nous avons :

$\equiv \equiv \equiv$
$a^{(p-1)(q-1)} \equiv 1 \pmod n$

%\change

%On note $(p-1)(q-1)$ par $\varphi(n)=$,

%$\phi$ s'appelle la fonciton fonction d'Euler.

\change

L'hypothèse $\pgcd(a,n)=1$ équivaut ici à ce que $a$ ne soit divisible ni par $p$, ni par $q$.

\change

Voyons un exemple avec $p=5$, $q=7$, 

leur produit est $n=35$ 

et ainsi $
%\varphi(n)=
(p-1)\times(q-1)=4\cdot 6 = 24$.


Alors pour tout entier $a$ premier avec $35$, 

c'est-à-dire ni divisible par $5$ ni par $7$

alors on a  $a^{25} \equiv 1 \pmod{35}$.



%%%%%%%%%%%%%%%%%%%%%%%%%%%%%%%%%%%%%%%%%%%%%%%%%%%%%%%%%%%
\diapo

Nous allons démnontrer ce résultat.

\change
Pour cela, notons $c = a^{(p-1)(q-1)}$.

\change
Nous allons d'abord calculer $c$ modulo $p$ :

$\equiv \equiv \equiv $
$c \equiv a^{(p-1)(q-1)} \equiv (a^{(p-1)})^{q-1}$

mais par le petit théorème de Fermat : $a^{p-1} \equiv 1 \pmod p$, car $p$ ne divise pas $a$

donc 
$(a^{(p-1)})^{q-1} \equiv 1^{q-1} \equiv 1 \pmod{p}$

\change
Calculons ce même $c$ mais cette fois modulo $q$ :

\change
De façon similaire, $c \equiv 1 \pmod{q}$

où l'on applique cette fois le petit théorème de Fermat pour $a^{q-1} \equiv 1 \pmod q$, car $q$ ne divise pas $a$.


|||
Conclusion partielle : $c \equiv 1 \pmod{p}$ et $c \equiv 1 \pmod q$.


\change
Nous allons en déduire que $c \equiv 1 \pmod{pq}$.

\change
$\equiv \equiv \equiv $
Comme $c \equiv 1 \pmod{p}$ alors il existe $\alpha$ tel que $c = 1 + \alpha p$ ;

\change
de même comme $c \equiv 1 \pmod{q}$ alors il existe $\beta$ tel que $c = 1 + \beta q$.

\change
Donc $c-1$ vaut à la fois $\alpha p$ et $\beta q$.

De cette égalité, on obtient que $p | \beta q$.

\change
Comme $p$ et $q$ sont premiers entre eux (car ce sont des nombres premiers distincts)
le lemme de Gauss permet d'en déduire que $p | \beta$.

\change
Il existe donc $\beta'\in \Zz$ tel que $\beta = \beta' p $.

\change
Ainsi $c = 1 + \beta q = 1 + \beta' pq$.

\change
Ce qui fait que $c \equiv 1 \pmod{pq}$,

||| 
ce qui revient à dire que $a^{(p-1)(q-1)} \equiv 1 \pmod n$. 


%%%%%%%%%%%%%%%%%%%%%%%%%%%%%%%%%%%%%%%%%%%%%%%%%%%%%%%%%%%
\diapo

Nous avons déjà étudié l'algorithme d'Euclide qui repose sur le principe
que 

$\equiv \equiv \equiv $
$\pgcd(a,b)= \pgcd(b, a \pmod{b})$.

|||
$a \pmod{b}$ est  bien le reste de la division euclidienne de $a$ par $b$.

\change


Ce principe se transforme en algorithme de la façon suivante :

$\equiv \equiv \equiv $
On initialise une suite par $a_0=a$ et une autre par $b_0=b$.

Puis à chaque étape 
$a_{i+1}  =  b_i$ (le prochain dividende est l'actuel diviseur)

et 
$b_{i+1}  =  a_i \pmod {b_i}$ (le prochain diviseur est l'actuel reste).

|||
On s'arrête dès que $b_i$ est nul. 

Le pgcd est alors le dernier reste non nul,

c'est-à-dire $b_{i-1}$ qui est également $a_i$.


\change

Voici sa mise en \oe uvre informatique.

$\equiv \equiv \equiv $
On fait une boucle avec $a$ et $b$ qui varie, représentant les suites $a_i$ et $b_i$.

On profite du fait que Python permet les affectations simultanées :

voici les nouveaux $a$ et $b$ ,

calculer à partir des anciens.

La boucle tourne jusqu'à ce que le reste $b$ soit nul,

|||
et on renvoie le dernier reste non nul.




%%%%%%%%%%%%%%%%%%%%%%%%%%%%%%%%%%%%%%%%%%%%%%%%%%%%%%%%%%%
\diapo


Nous avons vu aussi comment \og remonter \fg\ l'algorithme d'Euclide à la main pour
obtenir les coefficients de Bézout $u,v$ tels que $au+bv=\pgcd(a,b)$.


Cependant il nous faut une méthode plus automatique pour obtenir ces coefficients, il s'agit de 
l'\defi{algorithme d'Euclide étendu}.

\change

$\equiv \equiv \equiv $
On définit deux suites $(x_i)$, $(y_i)$ qui nous fourniront les coefficients de Bézout.

L'initialisation est :
$$x_0=1 \qquad  x_1=0 \qquad y_0=0 \qquad y_1 = 1$$

et la formule de récurrence pour $i\geq 1$ :

$$x_{i+1} = x_{i-1} - q_i x_i \qquad y_{i+1} = y_{i-1} - q_i y_i$$

où $q_i$ est le quotient de la division euclidienne de $a_i$ par $b_i$ définis dans l'algorithme d'Euclide.
 
 
|||
Comme précédemment on itère jusqu'à ce que le reste $b_i$ soit nul.

On obtient alors le pgcd mais en plus $x_i$ et $y_i$ sont les coefficients de Bézout.

\change

Voici l'implémentation de l'algorithme
 
$\equiv \equiv \equiv $
Vous reconnaissez la structure de l'algorithme d'Euclide classique,

avec en plus les calculs des éléments des nouvelles suites x et y 

Cet algorithme renvoie d'abord le pgcd, puis les coefficients $u,v$ 

|||
tels que $au+bv = \pgcd(a,b)$.




%%%%%%%%%%%%%%%%%%%%%%%%%%%%%%%%%%%%%%%%%%%%%%%%%%%%%%%%%%%
\diapo


Soit $a$ un entier, 


on dit que l'entier $x$ est un \defi{inverse de $a$ modulo $n$}

$\equiv \equiv \equiv $
si $ax \equiv 1 \pmod{n}$.



Trouver un inverse de $a$ modulo $n$ est donc 
un cas particulier de l'équation $ax \equiv b \pmod n$.

Attention un tel inverse n'existe pas toujours !

\change
La proposition suivante permet justement de savoir s'il existe un inverse et va nous dire comment le calculer.

Tout d'abord 
$a$ admet un inverse modulo $n$ si et seulement si $a$ et $n$ sont premiers entre eux.


\change
ensuite si on est dans ce cas 
et si on trouvé des coefficients de Bézout $u,v$ tels que 
$au+nv=1$ alors $u$ est un inverse de $a$ modulo $n$.



En d'autres termes, trouver un inverse de $a$ modulo $n$ revient à calculer les coefficients de Bézout
associés à la paire $(a,n)$.

\change

Le deuxième point est immédiat.

Le premier point est essentiellement une reformulation du théorème de Bézout :

$\pgcd(a,n)=1$

\change

$\iff  \exists u,v \in \Zz \qquad au+nv=1$

\change
$\iff  \exists u \in \Zz \qquad au \equiv 1 \pmod {n}$


% 
% %%%%%%%%%%%%%%%%%%%%%%%%%%%%%%%%%%%%%%%%%%%%%%%%%%%%%%%%%%%
% \diapo
% 
% Voici le code :
% 
% \change
% 
% \change
% 
% \change
% 



%%%%%%%%%%%%%%%%%%%%%%%%%%%%%%%%%%%%%%%%%%%%%%%%%%%%%%%%%%%
\diapo

Nous aurons par la suite besoin de calculer rapidement des puissances modulo un entier $n$.
Pour cela il existe une méthode beaucoup plus efficace que de calculer d'abord 
$a^k$ puis de le réduire modulo $n$ ou bien encore de faire $k-1$ multiplications modulaires. 
Il faut garder à l'esprit que les entiers
que l'on va mani\-puler ont des dizaines voir des centaines de chiffres.


\change
Voyons la méthode sur l'exemple du calcul de $5^{11} \pmod{14}$.


L'idée est de seulement calculer $5, 5^2, 5^4, 5^8...$ %, 5^{16},...$
modulo $14$.


\change
$\equiv \equiv \equiv $
En effet, on remarque que l'exposant $11 = 8 + 2 + 1$

\change
donc
$$5 ^{11} = 5^8 \times 5^2 \times 5^1.$$

\change
Calculons donc les $5^{2^i} \pmod {14}$ :

\change
$5  \equiv  5 \pmod{14}$

\change
et
$5^2  \equiv  25 \equiv 11 \pmod{14}$

\change
Pour $5^4$ 

\change
on l'écrit sous la forme $5^2 \times 5^2$

et on profite du fait que l'on vient juste de calculer $5^2$.

\change
donc $5^4 \equiv 11 \times 11$

\change
ce qui donne $121$

\change 
qui est congru à $9 \pmod{14}$.

\change
Même principe pour $5^8$

\change
c'est $5^4 \times 5^4$

\change

ce qui est donc congru à $9 \times 9$

\change

soit $81$

\change

donc $\equiv 11 \pmod{14}$.

\change

On rassemble tous nos calculs :

$5^{11} = 5^8 \times 5^2 \times 5^1$


\change

qui est donc congru à $11 \times 11 \times 5$

\change

égale à $11 \times 55$

\change
$\equiv 11 \times 13$

\change
$\equiv 143$

\change
$\equiv 3 \pmod{14}$.


||| 
Nous obtenons donc un calcul de $5^{11} \pmod{14}$ en $5$ opérations au lieu de $10$ 
si on avait fait naïvement $5\times 5 \times 5 \cdots$, et plus l'exposant est 
grand plus le gain l'est aussi.



%%%%%%%%%%%%%%%%%%%%%%%%%%%%%%%%%%%%%%%%%%%%%%%%%%%%%%%%%%%
\diapo

Voici une formulation générale de la méthode d'exponentiation modulaire.

$\equiv \equiv \equiv $
On écrit le développement de l'exposant $k$ en base $2$ :
$(k_\ell, \ldots, k_2, k_1, k_0)$ avec $k_i \in \{0,1\}$ de sorte que 
$$k = \sum_{i=0}^\ell k_i 2^i.$$


\change
On calcule successivement les $x^{2^i}$ modulo $n$ 

par récurrence en utilisant que $x^{2^{i+1}} = x^{2^i} \times x^{2^i}$

et en effectuant tous les calculs modulo notre entier $n$.


\change
On obtient alors 
$$x^k = x^{\sum_{i=0}^\ell k_i 2^i} = \prod_{i=0}^\ell (x^{2^i})^{k_i}.$$
où les $k_i$ valent 0 ou 1.

\change
Revoyons cela sur l'exemple précédent :

L'entier $11$ s'écrit en base $2$ :
$({\color{red}1},{\color{red}0},{\color{red}1},{\color{red}1})$, 
8+2+1

\change
On avait calculé $5$, $5^2$, $5^4$ et $5^8$

\change
Et maintenant
$$5^{11} = (5^{2^3})^{{\color{red}1}} \times (5^{2^2})^{{\color{red}0}}
\times (5^{2^1})^{{\color{red}1}} \times (5^{2^0})^{{\color{red}1}}$$

|||
et on remplace les facteurs par les valeurs obtenues.


%%%%%%%%%%%%%%%%%%%%%%%%%%%%%%%%%%%%%%%%%%%%%%%%%%%%%%%%%%%
\diapo


Voici un autre exemple : calculons $17^{154} \pmod {100}$.

\change
$\equiv \equiv \equiv $
Tout d'abord on décompose l'exposant $k=154$ en base $2$ :
$154 = 128 + 16 + 8 + 2 = 2^7 + 2^4 + 2^3 + 2^1$, 

\change
il s'écrit donc en base $2$ : $(1,0,0,1,1,0,1,0)$.

\change
Ensuite on calcule $17, 17^2, 17^4, 17^8,...,17^{128}$ modulo $100$.

Voici les calculs :

n'ayez pas peur, c'est en fait très simple !

N'oubliez pas d'abord que tous les calculs sont modulo $100$,
donc si on obtient un nombre à 3 chiffres ou plus alors 
on ne retient que les deux derniers chiffres (c'est cela le reste modulo $100$).

Ensuite 
pour chaque nouvelle puissance,
on effectue une élévation au carré, c'est-à-dire une multiplication,
 en prenant soin d'utiliser le résultat de l'étape précédente.
Par exemple 

$17^8=17^4 \times 17^4=21\times 21=441\equiv 41 \pmod{100}$.

%Il suffit donc d'utiliser le calcul de la ligne du dessus.


\change
Il ne reste qu'à rassembler :

$17^{154}$ est égal à  $17^{128} \times 17^{16} \times 17^{8} \times 17^{2}$

on remplace chaque facteur par sa valeur modulo $100$.

|||
et au final on trouve que $17^{154} \equiv 29 \pmod {100}$.

%%%%%%%%%%%%%%%%%%%%%%%%%%%%%%%%%%%%%%%%%%%%%%%%%%%%%%%%%%%
\diapo

On en déduit un algorithme pour le calcul rapide des puissances.

$\equiv \equiv \equiv $
On souhaite calculer $x^k$ modulo $n$.

L'algorithme est une boucle qui fait varier $x$ et l'exposant $k$

On teste s'il faut tenir compte dans le résultat de la valeur de $x$ à la puissance $2$ à la puissance quelque chose.
(ce test revient à vérifier si $k_i$, le $i$-ème chiffre de l'écriture de $k$ en base $2$, vaut $0$ ou $1$).

si oui alors on multiplie (modulo $n$).

Ensuite on passe à la puissance suivante : donc on aura successivement $x^2$, $x^4$,...

et on divise (de manière entière) l'exposant par $2$ c'est-à-dire on va regarder $k_{i+1}$ (le chiffre suivant).


A la fin on obtient $x^k$ modulo $n$. 


\change
En fait Python\ intègre déjà cette exponentiation rapide par la commande \codeinline{pow(x,k,n)} calcule $x^k$ modulo $n$,

\change
il faut en tout cas éviter \codeinline{(x ** k) \% n} qui n'est pas adapté.

\end{document}
