
%%%%%%%%%%%%%%%%%% PREAMBULE %%%%%%%%%%%%%%%%%%


\documentclass[12pt]{article}

\usepackage{amsfonts,amsmath,amssymb,amsthm}
\usepackage[utf8]{inputenc}
\usepackage[T1]{fontenc}
\usepackage[francais]{babel}


% packages
\usepackage{amsfonts,amsmath,amssymb,amsthm}
\usepackage[utf8]{inputenc}
\usepackage[T1]{fontenc}
%\usepackage{lmodern}

\usepackage[francais]{babel}
\usepackage{fancybox}
\usepackage{graphicx}

\usepackage{float}

%\usepackage[usenames, x11names]{xcolor}
\usepackage{tikz}
\usepackage{datetime}

\usepackage{mathptmx}
%\usepackage{fouriernc}
%\usepackage{newcent}
\usepackage[mathcal,mathbf]{euler}

%\usepackage{palatino}
%\usepackage{newcent}


% Commande spéciale prompteur

%\usepackage{mathptmx}
%\usepackage[mathcal,mathbf]{euler}
%\usepackage{mathpple,multido}

\usepackage[a4paper]{geometry}
\geometry{top=2cm, bottom=2cm, left=1cm, right=1cm, marginparsep=1cm}

\newcommand{\change}{{\color{red}\rule{\textwidth}{1mm}\\}}

\newcounter{mydiapo}

\newcommand{\diapo}{\newpage
\hfill {\normalsize  Diapo \themydiapo \quad \texttt{[\jobname]}} \\
\stepcounter{mydiapo}}


%%%%%%% COULEURS %%%%%%%%%%

% Pour blanc sur noir :
%\pagecolor[rgb]{0.5,0.5,0.5}
% \pagecolor[rgb]{0,0,0}
% \color[rgb]{1,1,1}



%\DeclareFixedFont{\myfont}{U}{cmss}{bx}{n}{18pt}
\newcommand{\debuttexte}{
%%%%%%%%%%%%% FONTES %%%%%%%%%%%%%
\renewcommand{\baselinestretch}{1.5}
\usefont{U}{cmss}{bx}{n}
\bfseries

% Taille normale : commenter le reste !
%Taille Arnaud
%\fontsize{19}{19}\selectfont

% Taille Barbara
%\fontsize{21}{22}\selectfont

%Taille François
%\fontsize{25}{30}\selectfont

%Taille Pascal
%\fontsize{25}{30}\selectfont

%Taille Laura
%\fontsize{30}{35}\selectfont


%\myfont
%\usefont{U}{cmss}{bx}{n}

%\Huge
%\addtolength{\parskip}{\baselineskip}
}


% \usepackage{hyperref}
% \hypersetup{colorlinks=true, linkcolor=blue, urlcolor=blue,
% pdftitle={Exo7 - Exercices de mathématiques}, pdfauthor={Exo7}}


%section
% \usepackage{sectsty}
% \allsectionsfont{\bf}
%\sectionfont{\color{Tomato3}\upshape\selectfont}
%\subsectionfont{\color{Tomato4}\upshape\selectfont}

%----- Ensembles : entiers, reels, complexes -----
\newcommand{\Nn}{\mathbb{N}} \newcommand{\N}{\mathbb{N}}
\newcommand{\Zz}{\mathbb{Z}} \newcommand{\Z}{\mathbb{Z}}
\newcommand{\Qq}{\mathbb{Q}} \newcommand{\Q}{\mathbb{Q}}
\newcommand{\Rr}{\mathbb{R}} \newcommand{\R}{\mathbb{R}}
\newcommand{\Cc}{\mathbb{C}} 
\newcommand{\Kk}{\mathbb{K}} \newcommand{\K}{\mathbb{K}}

%----- Modifications de symboles -----
\renewcommand{\epsilon}{\varepsilon}
\renewcommand{\Re}{\mathop{\text{Re}}\nolimits}
\renewcommand{\Im}{\mathop{\text{Im}}\nolimits}
%\newcommand{\llbracket}{\left[\kern-0.15em\left[}
%\newcommand{\rrbracket}{\right]\kern-0.15em\right]}

\renewcommand{\ge}{\geqslant}
\renewcommand{\geq}{\geqslant}
\renewcommand{\le}{\leqslant}
\renewcommand{\leq}{\leqslant}

%----- Fonctions usuelles -----
\newcommand{\ch}{\mathop{\mathrm{ch}}\nolimits}
\newcommand{\sh}{\mathop{\mathrm{sh}}\nolimits}
\renewcommand{\tanh}{\mathop{\mathrm{th}}\nolimits}
\newcommand{\cotan}{\mathop{\mathrm{cotan}}\nolimits}
\newcommand{\Arcsin}{\mathop{\mathrm{Arcsin}}\nolimits}
\newcommand{\Arccos}{\mathop{\mathrm{Arccos}}\nolimits}
\newcommand{\Arctan}{\mathop{\mathrm{Arctan}}\nolimits}
\newcommand{\Argsh}{\mathop{\mathrm{Argsh}}\nolimits}
\newcommand{\Argch}{\mathop{\mathrm{Argch}}\nolimits}
\newcommand{\Argth}{\mathop{\mathrm{Argth}}\nolimits}
\newcommand{\pgcd}{\mathop{\mathrm{pgcd}}\nolimits} 

\newcommand{\Card}{\mathop{\text{Card}}\nolimits}
\newcommand{\Ker}{\mathop{\text{Ker}}\nolimits}
\newcommand{\id}{\mathop{\text{id}}\nolimits}
\newcommand{\ii}{\mathrm{i}}
\newcommand{\dd}{\mathrm{d}}
\newcommand{\Vect}{\mathop{\text{Vect}}\nolimits}
\newcommand{\Mat}{\mathop{\mathrm{Mat}}\nolimits}
\newcommand{\rg}{\mathop{\text{rg}}\nolimits}
\newcommand{\tr}{\mathop{\text{tr}}\nolimits}
\newcommand{\ppcm}{\mathop{\text{ppcm}}\nolimits}

%----- Structure des exercices ------

\newtheoremstyle{styleexo}% name
{2ex}% Space above
{3ex}% Space below
{}% Body font
{}% Indent amount 1
{\bfseries} % Theorem head font
{}% Punctuation after theorem head
{\newline}% Space after theorem head 2
{}% Theorem head spec (can be left empty, meaning ‘normal’)

%\theoremstyle{styleexo}
\newtheorem{exo}{Exercice}
\newtheorem{ind}{Indications}
\newtheorem{cor}{Correction}


\newcommand{\exercice}[1]{} \newcommand{\finexercice}{}
%\newcommand{\exercice}[1]{{\tiny\texttt{#1}}\vspace{-2ex}} % pour afficher le numero absolu, l'auteur...
\newcommand{\enonce}{\begin{exo}} \newcommand{\finenonce}{\end{exo}}
\newcommand{\indication}{\begin{ind}} \newcommand{\finindication}{\end{ind}}
\newcommand{\correction}{\begin{cor}} \newcommand{\fincorrection}{\end{cor}}

\newcommand{\noindication}{\stepcounter{ind}}
\newcommand{\nocorrection}{\stepcounter{cor}}

\newcommand{\fiche}[1]{} \newcommand{\finfiche}{}
\newcommand{\titre}[1]{\centerline{\large \bf #1}}
\newcommand{\addcommand}[1]{}
\newcommand{\video}[1]{}

% Marge
\newcommand{\mymargin}[1]{\marginpar{{\small #1}}}



%----- Presentation ------
\setlength{\parindent}{0cm}

%\newcommand{\ExoSept}{\href{http://exo7.emath.fr}{\textbf{\textsf{Exo7}}}}

\definecolor{myred}{rgb}{0.93,0.26,0}
\definecolor{myorange}{rgb}{0.97,0.58,0}
\definecolor{myyellow}{rgb}{1,0.86,0}

\newcommand{\LogoExoSept}[1]{  % input : echelle
{\usefont{U}{cmss}{bx}{n}
\begin{tikzpicture}[scale=0.1*#1,transform shape]
  \fill[color=myorange] (0,0)--(4,0)--(4,-4)--(0,-4)--cycle;
  \fill[color=myred] (0,0)--(0,3)--(-3,3)--(-3,0)--cycle;
  \fill[color=myyellow] (4,0)--(7,4)--(3,7)--(0,3)--cycle;
  \node[scale=5] at (3.5,3.5) {Exo7};
\end{tikzpicture}}
}



\theoremstyle{definition}
%\newtheorem{proposition}{Proposition}
%\newtheorem{exemple}{Exemple}
%\newtheorem{theoreme}{Théorème}
\newtheorem{lemme}{Lemme}
\newtheorem{corollaire}{Corollaire}
%\newtheorem*{remarque*}{Remarque}
%\newtheorem*{miniexercice}{Mini-exercices}
%\newtheorem{definition}{Définition}




%definition d'un terme
\newcommand{\defi}[1]{{\color{myorange}\textbf{\emph{#1}}}}
\newcommand{\evidence}[1]{{\color{blue}\textbf{\emph{#1}}}}



 %----- Commandes divers ------

\newcommand{\codeinline}[1]{\texttt{#1}}

%%%%%%%%%%%%%%%%%%%%%%%%%%%%%%%%%%%%%%%%%%%%%%%%%%%%%%%%%%%%%
%%%%%%%%%%%%%%%%%%%%%%%%%%%%%%%%%%%%%%%%%%%%%%%%%%%%%%%%%%%%%

\definecolor{coul_prive}{rgb}{0.93,0.26,0}
\definecolor{coul_public}{rgb}{0.06,0.63,0}

\newcommand{\prive}[1]{{\bf\color{coul_prive} #1}}
\newcommand{\public}[1]{{\bf\color{coul_public} #1}}


\begin{document}

\debuttexte


%%%%%%%%%%%%%%%%%%%%%%%%%%%%%%%%%%%%%%%%%%%%%%%%%%%%%%%%%%%
\diapo

Nous avons tout préparer pour arriver au but de ce cours : présenter et comprendre le système de chiffrement à clé publique RSA.

\change

Nous allons détaillé son principe :

\change

comment calculer une clé publique et une clé privée,

\change

comment chiffrer un message,

\change

et bien sûr comment le déchiffrer.

%\change

%Nous dirons quelques mot sur la sécurité du code RSA

\change

Efin, nous présenterons les algorithmes, qui sont particulièrement simples.

%%%%%%%%%%%%%%%%%%%%%%%%%%%%%%%%%%%%%%%%%%%%%%%%%%%%%%%%%%%
\diapo



%Voici le but ultime de ce cours : la chiffrement RSA.
Il est temps de relire l'introduction du 
chapitre \og Arithmétique \fg\ pour s'apercevoir
que nous sommes prêts à découvrir le système de chiffrement RSA !

Ce protocole, sans aucun doute le plus utilisé actuellement, a été présenté pour la première fois en 1977 par Ronald Rivest, Adi Shamir et Leonard Adleman.

=

Voyons tout d'abord les outils que nous avons étudiés et qui interviendront dans la mise en place de ce système de chiffrement à clé publique.

Nous avons besoin :


d'un problème difficile 

=

nous avons abordé celui de la factorisation d'entiers et pris conscience que pour des entiers de plusieurs centaines de chiffres décimaux c'était un problème très difficile.

\change

La clé secrète et la clé publique seront calculées à l'aide de l'\evidence{algorithme d'Euclide} 
et des \evidence{coefficients de Bézout}.

\change

Tous ces calculs se feront \evidence{modulo} un entier

\change

et le fait que le déchiffrement fournisse le message initial sera établi grâce au \evidence{petit théorème de Fermat}.

\change

=

|| Lorsque Bruno souhaite envoyer un message  à Alice, le processus se décompose en trois étapes : 

\change

Première étape  : Alice prépare une clé publique et une clé privée,

\change

Deuxième étape  : Bruno utilise la clé publique d'Alice pour crypter son message,

\change

Enfin, troisième étape : Alice reçoit le message crypté et le déchiffre grâce à sa clé privée.


%%%%%%%%%%%%%%%%%%%%%%%%%%%%%%%%%%%%%%%%%%%%%%%%%%%%%%%%%%%
\diapo

Afin de permettre à ses correspondants de lui envoyer des messages cryptés, 
Alice doit préparer sa clé publique qu'elle diffusera
à tout le monde.

Elle prépare également sa clé privée qu'elle sera seule à connaître et qui lui permettra de déchiffrer les messages qu'on lui enverra.

\change
Pour produire ses clés publiques et privées, Alice effectue, une fois pour toutes, les opérations suivantes :

\begin{itemize}
\item elle choisit deux nombres premiers distincts $p$ et $q$ 
(dans la pratique ce sont de très grands nombres,
  jusqu'à plusieurs centaines de chiffres décimaux),
  

\item puis elle calcule ${\color{coul_public}n} = p \times q$,

\item et enfin elle calcule $\varphi(n) = (p-1) \times (q-1)$.
\end{itemize}

|| Vous noterez que le calcul de $\varphi(n)$ n'est possible que si la décomposition de $n$ sous la forme $p\times q$ est connue.

\change

Voici un exemple simple, que nous allons suivre tout au long de l'exposé, pour lequel nous pourrons effectuer touts     les calculs à la main :

\begin{itemize}
\item On choisit $p = 5$ et $q = 17$ 
  
\item alors ${\color{coul_public}n} = p \times q = 85$
  
\item et $\varphi(n) = (p-1) \times (q-1) = 4\times 16 = 64$
\end{itemize}


%% \change
%% Le deuxième exemple sera plus compliqué et les calculs seront fait à la calculatrice ou à
%% l'ordinateur.

%% On choisit $p = 101$ et $q = 103$ 
  
%% Alors ${\color{coul_public}n} = p \times q = 10\;403$
 
%% et $\varphi(n) = (p-1) \times (q-1) = 10\;200$.


%%%%%%%%%%%%%%%%%%%%%%%%%%%%%%%%%%%%%%%%%%%%%%%%%%%%%%%%%%%
\diapo

Voici la suite de la première étape qui est l'étape de préparation des clés :


Alice poursuit cette préparation en choisissant un entier qui sera utilisé comme exposant.

\change

Précisément : Alice choisit un exposant ${\color{coul_public}e}$ 
tel que $\pgcd(e,\varphi(n))=1$ c'est-à-dire tel que $e$ et $\varphi(n)$ soient premiers entre eux.


Elle calcule ensuite l'inverse ${\color{coul_prive}d}$ de ${\color{coul_public}e}$ module $\varphi(n)$ :

c'est-à-dire l'entier $d$ vérifiant $d \times e \equiv 1 \pmod {\varphi(n)}$. 

Ce calcul se fait bien sûr à l'aide de l'algorithme d'Euclide étendu.

\change

Choisissons pour notre exemple 
${\color{coul_public}e} = 5$ qui est bien premier avec $\varphi(n)$ qui vaut $64$.

\change

Alice applique l'algorithme d'Euclide étendu pour calculer les coefficients de Bézout correspondant à  
l'égalité  $\pgcd(e,\varphi(n))=1$. 

Elle trouve   $5 \times 13 + 64 \times (-1) = 1$ 

Donc $5 \times 13 \equiv 1 \pmod {64}$

et l'inverse de $e$ modulo $\varphi(n)$ est ${\color{coul_prive}d} = 13$.
 
%% \change

%% Pour l'exemple plus compliqué on choisit ${\color{coul_public}e} = 7$ 

%% et on bien  $\pgcd(e,\varphi(n)) = \pgcd(7,10\;200)=1$,
  
%% \change

%% L'algorithme d'Euclide étendu les coefficients de Bézout pour $\pgcd(e,\varphi(n))=1$ donne

%%   $7 \times (-1457) + 10\; 200 \times 1 = 1$. 
  
%%   On cherche un bon représentant et comme 
%%    $-1457 \equiv 8743\pmod{\varphi(n)}$, 
%%   alors pour ${\color{coul_prive}d} = 8743 $
  
%%   on a   $d \times e \equiv 1 \pmod {\varphi(n)}$.


%%%%%%%%%%%%%%%%%%%%%%%%%%%%%%%%%%%%%%%%%%%%%%%%%%%%%%%%%%%
\diapo

La première étape est presque finie :

La \defi{clé publique} d'Alice est constituée des deux nombres :
{${\color{coul_public}n}$ \  et \  ${\color{coul_public}e}$}

Et comme son nom l'indique Alice peut la communiquer à ses correspondants.

\change


Alice garde pour elle sa \defi{clé privée} :
{\ ${\color{coul_prive}d}$\ }

=

|| Alice peut maintenant détruire $p$, $q$ et $\varphi(n)$ qui ne sont plus utiles.


\change

Pour notre exemple nous avons donc 
 ${\color{coul_public}n} = 85$ et ${\color{coul_public}e} = 5$
 qui constituent la clé publique

et ${\color{coul_prive}d} = 13$
qui est la clé privée.

%%  \change
 
%%  et pour l'exemple plus compliqué : ${\color{coul_public}n} = 10\;403$ et ${\color{coul_public}e} = 7$

%% et ${\color{coul_prive}d} = 8743$


%%%%%%%%%%%%%%%%%%%%%%%%%%%%%%%%%%%%%%%%%%%%%%%%%%%%%%%%%%%
\diapo

La deuxième étape présente maintenant l'opération de chiffrement.

\change

Rappelons que Bruno veut envoyer un message secret à Alice.

\change

Il commence par numériser son message, il le transforme en un entier $m$ (le plus souvent en plusieurs)

\change

car il faut qu'il soit strictement inférieur à $n$.

En effet, nous allons travailler modulo $n$.

\change

Choisissons comme message pour notre exemple : $m=10$.

%% et pour l'exemple 2 c'est $m=1234$.

%%%%%%%%%%%%%%%%%%%%%%%%%%%%%%%%%%%%%%%%%%%%%%%%%%%%%%%%%%%
\diapo

Nous avons tous les éléments pour passer au chiffrement du message.

\change

Bruno commence par se procurer la clé publique d'Alice : ${\color{coul_public}n}$ et ${\color{coul_public}e}$

\change
avec laquelle il calcule, à l'aide de l'algorithme d'exponentiation rapide, le message chiffré :

$$x \equiv m^{\color{coul_public}e} \pmod{{\color{coul_public}n}}$$

\change

Bruno transmet ce message $x$ à Alice

\change

Pour l'exemple nous avons  $m = 10$, ${\color{coul_public}n} = 85$ et ${\color{coul_public}e} = 5$

\change
donc le message secret s'obtient comme 
$$x \equiv m^e \pmod n \equiv 10^{5} \pmod {85}$$

\change

On peut ici faire les calculs à la main :
$10^2 = 100 \equiv 15 \pmod {85}$

\change

$10^4 = (10^2)^2 \equiv 15^2 \equiv 225 \equiv 55 \pmod{85}$

\change 

Donc puisque $5=4+1$ nous avons $10^5 = 10^4 \times 10 \equiv 55 \times 10 = 550 \equiv 40 \pmod{85}$

\change


Le message chiffré que Bruno transmet à Alice est donc $x=40$.

%% \change

%% Pour l'autre exemple $m = 1234$, ${\color{coul_public}n} = 10\;403$ et ${\color{coul_public}e} = 7$
%% donc 
%% $x \equiv m^e \pmod n$

%% donc $x \equiv 1234^{7} \pmod {10\;403}$

%% On utilise l'ordinateur pour obtenir que $x= 10 \; 378$.



%%%%%%%%%%%%%%%%%%%%%%%%%%%%%%%%%%%%%%%%%%%%%%%%%%%%%%%%%%%
\diapo

Voici la dernière étape du protocole : le déchiffrement du message.

\change

Alice reçoit donc le message $x$, chiffré par Bruno,

\change

elle va le décrypter à l'aide de sa clé privée ${\color{coul_prive}d}$,


\change
grâce au calcul :
$$m = x^{\color{coul_prive}d} \pmod{\color{coul_public}n}$$

qui utilise encore une fois l'algorithme d'exponentiation rapide.

Nous allons prouver dans quelques instants, que par cette opération Alice 
retrouve bien le message original $m$ de Bruno.


\change


Dans notre exemple, Alice possède les éléments suivants : 

$x = 40$, ${\color{coul_prive}d} = 13$ et ${\color{coul_public}n} = 85$ 

\change

et doit calculer 

$$40^{13} \pmod{85}.$$

\change
ce que nous allons faire maintenant :

$40^2 = 1600 \equiv 70 \pmod{85} $

$40^4 = (40^2)^2 \equiv 70^2 \equiv 4900 \equiv 55 \pmod{85}$

$40^8 = (40^4)^2 \equiv 55^2 \equiv 3025 \equiv 50 \pmod{85}$

\change
On note que l'exposant $13$ se décompose en base $2$ : $13 = 8 + 4 + 1$,
donc
$$40^{13} = 40^8 \times 40^4 \times 40 \equiv 50 \times 55 \times 40 \equiv 10 \pmod{85}$$ 

\change
qui est bien le message $m$ de Bruno.

%% \change

%% \textbf{Exemple 2.} $x = 10 \; 378$, ${\color{coul_prive}d} = 8743$, ${\color{coul_public}n} = 10\;403$.

%% \change
%% On calcule par ordinateur $x^d \equiv (10 \; 378)^{8743} \pmod{10\;403}$
%% qui vaut exactement le message original de Bruno $m=1234$. 


%%%%%%%%%%%%%%%%%%%%%%%%%%%%%%%%%%%%%%%%%%%%%%%%%%%%%%%%%%%
\diapo

Voici une vue synthétique du système de chiffrement RSA.

Tout d'abord, les clés publique et privée d'Alice.

Le chiffrement du message $m$ à l'aide de la clé publique $n$, $e$ : $m^e\pmod{n}$

et le déchiffrement du message reçu $x$ à l'aide de la clé privée $d$ : $x^d\pmod{n}$

qui redonne le message $m$ de départ.

% %%%%%%%%%%%%%%%%%%%%%%%%%%%%%%%%%%%%%%%%%%%%%%%%%%%%%%%%%%%
% \diapo
% 
% Le principe de déchiffrement du système RSA repose sur le petit théorème de Fermat amélioré.
% 
% Nous énonçons le résultat sous la forme d'un lemme : 
% 
% "Soit $d$ l'inverse de $e$ modulo $\varphi(n)$ où $n$ est produit de deux nombres premiers.
% 
% {Si $x \equiv m^e \pmod n$ alors $m \equiv x^d \pmod n$.}"
% 
% 
% Ce lemme assure que le message original $m$ de Bruno, 
% 
% chiffré par clé publique d'Alice $(e,n)$, 
% 
% peut-être retrouvé par Alice à l'aide de sa clé secrète $d$ par le calcul indiqué précédemment.
% 
% \change
% 
% Passons à la preuve de ce résultat.
% 
% Que $d$ soit l'inverse de $e$ modulo $\varphi(n)$ signifie 
% 
% \change
% 
% que $d \cdot e \equiv 1 \pmod {\varphi(n)}$.
% 
% \change
% 
% Autrement dit, il existe
% $k \in \Zz$ tel que $d \cdot e = 1 + k \cdot \varphi(n)$.
% 
% \change
%  On rappelle que par le petit théorème de Fermat amélioré, 
%  
% \change
%   $m^{\varphi(n)}\equiv m^{(p-1)(q-1)} \equiv 1 \pmod n$.
% 
%   \change
%   
% Si $x \equiv m^e \pmod n$ alors
% 
% \change
% 
% $x^d \equiv (m^e)^d$
% 
% \change
% $\equiv m^{e\cdot d}$
% 
% \change
% $\equiv m^{1+k \cdot \varphi(n)}$
% 
% \change
% $\equiv m \cdot m^{k \cdot \varphi(n)}$
% 
% \change
% $\equiv m \cdot (m^{\varphi(n)})^k$
% 
% \change
% $\equiv m \cdot (1)^k$
%        
% \change
% $\equiv m \pmod {n}$     
%       
% ce qui achève la preuve du lemme.

%%%%%%%%%%%%%%%%%%%%%%%%%%%%%%%%%%%%%%%%%%%%%%%%%%%%%%%%%%%
\diapo

%FRFRFRFRFR


Le principe de déchiffrement du système RSA repose sur le petit théorème de Fermat amélioré.

Nous énonçons le résultat sous la forme d'un lemme de déchiffrement : 

"Soit $d$ l'inverse de $e$ modulo $\varphi(n)$ où $n$ est produit de deux nombres premiers.

{Si $x \equiv m^e \pmod n$ alors $m \equiv x^d \pmod n$.}"


Ce lemme assure que le message original $m$ de Bruno, 

chiffré par clé publique d'Alice $(e,n)$, 

peut-être retrouvé par Alice à l'aide de sa clé secrète $d$ par le calcul indiqué précédemment.

\change

Passons à la preuve de ce résultat.

Que $d$ soit l'inverse de $e$ modulo $\varphi(n)$ signifie 

\change

que $d \cdot e \equiv 1 \pmod {\varphi(n)}$.

\change

Autrement dit, il existe
$k \in \Zz$ tel que $d \cdot e = 1 + k \cdot \varphi(n)$.

\change
 On rappelle que par le petit théorème de Fermat amélioré, 
 
\change
  lorsque $m$ est premier à $n$, $m^{\varphi(n)}$  $= m^{(p-1)(q-1)} \equiv 1 \pmod n$.

  \change
  
Si $\pgcd(m,n)=1$, c'est-à-dire que nous pouvons utiliser le théorème de Fermat amélioré, alors modulo $n$ : 

\change

$(m^e)^d$

\change
$\equiv m^{1+k \cdot \varphi(n)}$

\change
$\equiv m \cdot m^{k \cdot \varphi(n)}$

\change
$\equiv m \cdot (m^{\varphi(n)})^k$

\change
$\equiv m \cdot (1)^k$
       
\change
$\equiv m \pmod {n}$     


%%%%%%%%%%%%%%%%%%%%%%%%%%%%%%%%%%%%%%%%%%%%%%%%%%%%%%%%%%%
\diapo

%FRFRFRFRFR

Poursuivons la preuve du lemme en étudiant maintenant le cas où le $\pgcd(m,n)\not=1$. 

Comme $n$ est le produit des deux nombres premiers $p$ et $q$ et que $m$ 
est strictement plus petit que $n$

alors si $m$ et $n$ ne sont pas premiers entre eux cela implique que 

$p$ divise $m$ ou bien $q$ divise $m$ (mais pas les deux en même temps).


Faisons l'hypothèse $\pgcd(m,n)=p$ et  $\pgcd(m,q)=1$, le cas  $\pgcd(m,n)=q$  et $\pgcd(m,p)=1$ se traiterait de la même manière.

\change

Etudions $(m^e)^d$ à la fois modulo $p$ et modulo $q$ 
à l'image de ce que nous avons fait dans la preuve du théorème de Fermat amélioré.

%Modulo $p$ : 
Puisque $p$ divise $m$, $m$ est nul modulo $p$,

\change 

et comme $p$ divise également $(m^e)^d$, 
$(m^e)^d$ est aussi nul modulo $p$.

\change 

Nous avons donc $(m^e)^d \equiv  m \pmod{p}$

\change 

Modulo $q$ maintenant : puisque $q$ ne divise pas $m$, 
il suffit de réécrire comme précédemment :
 
$$(m^e)^d \equiv m \cdot (m^{\varphi(n)})^k \equiv (m^{q-1})^{(p-1)k} \equiv m \pmod {q}$$
      
et donc $(m^e)^d \equiv  m \pmod{q}$

\change

Bilan : $(m^e)^d \equiv  m \pmod{p}$ et $(m^e)^d \equiv  m \pmod{q}$,

comme $p$ et $q$ sont deux nombres premiers distincts, ils sont premiers entre eux 
et on peut écrire comme dans la preuve du petit théorème de Fermat amélioré que  
$$(m^e)^d  \equiv  m \pmod{n}$$

=
||ce qui achève la preuve du lemme.


%%%%%%%%%%%%%%%%%%%%%%%%%%%%%%%%%%%%%%%%%%%%%%%%%%%%%%%%%%%
\diapo

La mise en \oe uvre sur machine est maintenant très simple. 

=

Tout d'abord -étape 1- Alice choisit 
deux nombres premiers $p$ et $q$ et un exposant $e$ premier à $(p-1)(q-1)$.

Le calcul de la clé privée qui est l'inverse de $e$ modulo $(p-1)\times (q-1)$ est réalisé par la fonction cle\_privee dans laquelle : 

on calcule $n=p\times q$, puis $\phi=(p-1)(q-1)$, 

les coefficients de Bezout sont calculés par l'algorithme d'Euclide étendu avec notre fonction euclide\_etendu

et la valeur l'inverse de $e$ modulo $\phi$ est renvoyé par la fonction.

\change

Le chiffrement d'un message $m$, sous forme d'une nombre, est possible par tout le monde, 
connaissant la clé publique $(n,e)$.

C'est juste $m^e$ modulo $n$.


\change

Seule Alice peut déchiffrer ce message crypté $x$, à l'aide de sa clé privée $d$ et bien sûr de $n$

par le calcul de $x^d$ modulo $n$.

et ainsi retrouver le message initial que souhaitait lui transmettre Bruno.

\end{document}
