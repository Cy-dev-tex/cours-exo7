
%%%%%%%%%%%%%%%%%% PREAMBULE %%%%%%%%%%%%%%%%%%

\documentclass[aspectratio=169,utf8]{beamer}
%\documentclass[aspectratio=169,handout]{beamer}

\usetheme{Boadilla}
%\usecolortheme{seahorse}
\usecolortheme[RGB={245,66,24}]{structure}
\useoutertheme{infolines}

% packages
\usepackage{amsfonts,amsmath,amssymb,amsthm}
\usepackage[utf8]{inputenc}
\usepackage[T1]{fontenc}
\usepackage{lmodern}

\usepackage[francais]{babel}
\usepackage{fancybox}
\usepackage{graphicx}

\usepackage{float}
\usepackage{xfrac}

%\usepackage[usenames, x11names]{xcolor}
\usepackage{tikz}
\usepackage{pgfplots}
\usepackage{datetime}



%-----  Package unités -----
\usepackage{siunitx}
\sisetup{locale = FR,detect-all,per-mode = symbol}

%\usepackage{mathptmx}
%\usepackage{fouriernc}
%\usepackage{newcent}
%\usepackage[mathcal,mathbf]{euler}

%\usepackage{palatino}
%\usepackage{newcent}
% \usepackage[mathcal,mathbf]{euler}



% \usepackage{hyperref}
% \hypersetup{colorlinks=true, linkcolor=blue, urlcolor=blue,
% pdftitle={Exo7 - Exercices de mathématiques}, pdfauthor={Exo7}}


%section
% \usepackage{sectsty}
% \allsectionsfont{\bf}
%\sectionfont{\color{Tomato3}\upshape\selectfont}
%\subsectionfont{\color{Tomato4}\upshape\selectfont}

%----- Ensembles : entiers, reels, complexes -----
\newcommand{\Nn}{\mathbb{N}} \newcommand{\N}{\mathbb{N}}
\newcommand{\Zz}{\mathbb{Z}} \newcommand{\Z}{\mathbb{Z}}
\newcommand{\Qq}{\mathbb{Q}} \newcommand{\Q}{\mathbb{Q}}
\newcommand{\Rr}{\mathbb{R}} \newcommand{\R}{\mathbb{R}}
\newcommand{\Cc}{\mathbb{C}} 
\newcommand{\Kk}{\mathbb{K}} \newcommand{\K}{\mathbb{K}}

%----- Modifications de symboles -----
\renewcommand{\epsilon}{\varepsilon}
\renewcommand{\Re}{\mathop{\text{Re}}\nolimits}
\renewcommand{\Im}{\mathop{\text{Im}}\nolimits}
%\newcommand{\llbracket}{\left[\kern-0.15em\left[}
%\newcommand{\rrbracket}{\right]\kern-0.15em\right]}

\renewcommand{\ge}{\geqslant}
\renewcommand{\geq}{\geqslant}
\renewcommand{\le}{\leqslant}
\renewcommand{\leq}{\leqslant}
\renewcommand{\epsilon}{\varepsilon}

%----- Fonctions usuelles -----
\newcommand{\ch}{\mathop{\text{ch}}\nolimits}
\newcommand{\sh}{\mathop{\text{sh}}\nolimits}
\renewcommand{\tanh}{\mathop{\text{th}}\nolimits}
\newcommand{\cotan}{\mathop{\text{cotan}}\nolimits}
\newcommand{\Arcsin}{\mathop{\text{arcsin}}\nolimits}
\newcommand{\Arccos}{\mathop{\text{arccos}}\nolimits}
\newcommand{\Arctan}{\mathop{\text{arctan}}\nolimits}
\newcommand{\Argsh}{\mathop{\text{argsh}}\nolimits}
\newcommand{\Argch}{\mathop{\text{argch}}\nolimits}
\newcommand{\Argth}{\mathop{\text{argth}}\nolimits}
\newcommand{\pgcd}{\mathop{\text{pgcd}}\nolimits} 


%----- Commandes divers ------
\newcommand{\ii}{\mathrm{i}}
\newcommand{\dd}{\text{d}}
\newcommand{\id}{\mathop{\text{id}}\nolimits}
\newcommand{\Ker}{\mathop{\text{Ker}}\nolimits}
\newcommand{\Card}{\mathop{\text{Card}}\nolimits}
\newcommand{\Vect}{\mathop{\text{Vect}}\nolimits}
\newcommand{\Mat}{\mathop{\text{Mat}}\nolimits}
\newcommand{\rg}{\mathop{\text{rg}}\nolimits}
\newcommand{\tr}{\mathop{\text{tr}}\nolimits}


%----- Structure des exercices ------

\newtheoremstyle{styleexo}% name
{2ex}% Space above
{3ex}% Space below
{}% Body font
{}% Indent amount 1
{\bfseries} % Theorem head font
{}% Punctuation after theorem head
{\newline}% Space after theorem head 2
{}% Theorem head spec (can be left empty, meaning ‘normal’)

%\theoremstyle{styleexo}
\newtheorem{exo}{Exercice}
\newtheorem{ind}{Indications}
\newtheorem{cor}{Correction}


\newcommand{\exercice}[1]{} \newcommand{\finexercice}{}
%\newcommand{\exercice}[1]{{\tiny\texttt{#1}}\vspace{-2ex}} % pour afficher le numero absolu, l'auteur...
\newcommand{\enonce}{\begin{exo}} \newcommand{\finenonce}{\end{exo}}
\newcommand{\indication}{\begin{ind}} \newcommand{\finindication}{\end{ind}}
\newcommand{\correction}{\begin{cor}} \newcommand{\fincorrection}{\end{cor}}

\newcommand{\noindication}{\stepcounter{ind}}
\newcommand{\nocorrection}{\stepcounter{cor}}

\newcommand{\fiche}[1]{} \newcommand{\finfiche}{}
\newcommand{\titre}[1]{\centerline{\large \bf #1}}
\newcommand{\addcommand}[1]{}
\newcommand{\video}[1]{}

% Marge
\newcommand{\mymargin}[1]{\marginpar{{\small #1}}}

\def\noqed{\renewcommand{\qedsymbol}{}}


%----- Presentation ------
\setlength{\parindent}{0cm}

%\newcommand{\ExoSept}{\href{http://exo7.emath.fr}{\textbf{\textsf{Exo7}}}}

\definecolor{myred}{rgb}{0.93,0.26,0}
\definecolor{myorange}{rgb}{0.97,0.58,0}
\definecolor{myyellow}{rgb}{1,0.86,0}

\newcommand{\LogoExoSept}[1]{  % input : echelle
{\usefont{U}{cmss}{bx}{n}
\begin{tikzpicture}[scale=0.1*#1,transform shape]
  \fill[color=myorange] (0,0)--(4,0)--(4,-4)--(0,-4)--cycle;
  \fill[color=myred] (0,0)--(0,3)--(-3,3)--(-3,0)--cycle;
  \fill[color=myyellow] (4,0)--(7,4)--(3,7)--(0,3)--cycle;
  \node[scale=5] at (3.5,3.5) {Exo7};
\end{tikzpicture}}
}


\newcommand{\debutmontitre}{
  \author{} \date{} 
  \thispagestyle{empty}
  \hspace*{-10ex}
  \begin{minipage}{\textwidth}
    \titlepage  
  \vspace*{-2.5cm}
  \begin{center}
    \LogoExoSept{2.5}
  \end{center}
  \end{minipage}

  \vspace*{-0cm}
  
  % Astuce pour que le background ne soit pas discrétisé lors de la conversion pdf -> png
\begin{tikzpicture}
        \fill[opacity=0,green!60!black] (0,0)--++(0,0)--++(0,0)--++(0,0)--cycle; 
\end{tikzpicture}

% toc S'affiche trop tot :
% \tableofcontents[hideallsubsections, pausesections]
}

\newcommand{\finmontitre}{
  \end{frame}
  \setcounter{framenumber}{0}
} % ne marche pas pour une raison obscure

%----- Commandes supplementaires ------

% \usepackage[landscape]{geometry}
% \geometry{top=1cm, bottom=3cm, left=2cm, right=10cm, marginparsep=1cm
% }
% \usepackage[a4paper]{geometry}
% \geometry{top=2cm, bottom=2cm, left=2cm, right=2cm, marginparsep=1cm
% }

%\usepackage{standalone}


% New command Arnaud -- november 2011
\setbeamersize{text margin left=24ex}
% si vous modifier cette valeur il faut aussi
% modifier le decalage du titre pour compenser
% (ex : ici =+10ex, titre =-5ex

\theoremstyle{definition}
%\newtheorem{proposition}{Proposition}
%\newtheorem{exemple}{Exemple}
%\newtheorem{theoreme}{Théorème}
%\newtheorem{lemme}{Lemme}
%\newtheorem{corollaire}{Corollaire}
%\newtheorem*{remarque*}{Remarque}
%\newtheorem*{miniexercice}{Mini-exercices}
%\newtheorem{definition}{Définition}

% Commande tikz
\usetikzlibrary{calc}
\usetikzlibrary{patterns,arrows}
\usetikzlibrary{matrix}
\usetikzlibrary{fadings} 

%definition d'un terme
\newcommand{\defi}[1]{{\color{myorange}\textbf{\emph{#1}}}}
\newcommand{\evidence}[1]{{\color{blue}\textbf{\emph{#1}}}}
\newcommand{\assertion}[1]{\emph{\og#1\fg}}  % pour chapitre logique
%\renewcommand{\contentsname}{Sommaire}
\renewcommand{\contentsname}{}
\setcounter{tocdepth}{2}



%------ Figures ------

\def\myscale{1} % par défaut 
\newcommand{\myfigure}[2]{  % entrée : echelle, fichier figure
\def\myscale{#1}
\begin{center}
\footnotesize
{#2}
\end{center}}


%------ Encadrement ------

\usepackage{fancybox}


\newcommand{\mybox}[1]{
\setlength{\fboxsep}{7pt}
\begin{center}
\shadowbox{#1}
\end{center}}

\newcommand{\myboxinline}[1]{
\setlength{\fboxsep}{5pt}
\raisebox{-10pt}{
\shadowbox{#1}
}
}

%--------------- Commande beamer---------------
\newcommand{\beameronly}[1]{#1} % permet de mettre des pause dans beamer pas dans poly


\setbeamertemplate{navigation symbols}{}
\setbeamertemplate{footline}  % tiré du fichier beamerouterinfolines.sty
{
  \leavevmode%
  \hbox{%
  \begin{beamercolorbox}[wd=.333333\paperwidth,ht=2.25ex,dp=1ex,center]{author in head/foot}%
    % \usebeamerfont{author in head/foot}\insertshortauthor%~~(\insertshortinstitute)
    \usebeamerfont{section in head/foot}{\bf\insertshorttitle}
  \end{beamercolorbox}%
  \begin{beamercolorbox}[wd=.333333\paperwidth,ht=2.25ex,dp=1ex,center]{title in head/foot}%
    \usebeamerfont{section in head/foot}{\bf\insertsectionhead}
  \end{beamercolorbox}%
  \begin{beamercolorbox}[wd=.333333\paperwidth,ht=2.25ex,dp=1ex,right]{date in head/foot}%
    % \usebeamerfont{date in head/foot}\insertshortdate{}\hspace*{2em}
    \insertframenumber{} / \inserttotalframenumber\hspace*{2ex} 
  \end{beamercolorbox}}%
  \vskip0pt%
}


\definecolor{mygrey}{rgb}{0.5,0.5,0.5}
\setlength{\parindent}{0cm}
%\DeclareTextFontCommand{\helvetica}{\fontfamily{phv}\selectfont}

% background beamer
\definecolor{couleurhaut}{rgb}{0.85,0.9,1}  % creme
\definecolor{couleurmilieu}{rgb}{1,1,1}  % vert pale
\definecolor{couleurbas}{rgb}{0.85,0.9,1}  % blanc
\setbeamertemplate{background canvas}[vertical shading]%
[top=couleurhaut,middle=couleurmilieu,midpoint=0.4,bottom=couleurbas] 
%[top=fondtitre!05,bottom=fondtitre!60]



\makeatletter
\setbeamertemplate{theorem begin}
{%
  \begin{\inserttheoremblockenv}
  {%
    \inserttheoremheadfont
    \inserttheoremname
    \inserttheoremnumber
    \ifx\inserttheoremaddition\@empty\else\ (\inserttheoremaddition)\fi%
    \inserttheorempunctuation
  }%
}
\setbeamertemplate{theorem end}{\end{\inserttheoremblockenv}}

\newenvironment{theoreme}[1][]{%
   \setbeamercolor{block title}{fg=structure,bg=structure!40}
   \setbeamercolor{block body}{fg=black,bg=structure!10}
   \begin{block}{{\bf Th\'eor\`eme }#1}
}{%
   \end{block}%
}


\newenvironment{proposition}[1][]{%
   \setbeamercolor{block title}{fg=structure,bg=structure!40}
   \setbeamercolor{block body}{fg=black,bg=structure!10}
   \begin{block}{{\bf Proposition }#1}
}{%
   \end{block}%
}

\newenvironment{corollaire}[1][]{%
   \setbeamercolor{block title}{fg=structure,bg=structure!40}
   \setbeamercolor{block body}{fg=black,bg=structure!10}
   \begin{block}{{\bf Corollaire }#1}
}{%
   \end{block}%
}

\newenvironment{mydefinition}[1][]{%
   \setbeamercolor{block title}{fg=structure,bg=structure!40}
   \setbeamercolor{block body}{fg=black,bg=structure!10}
   \begin{block}{{\bf Définition} #1}
}{%
   \end{block}%
}

\newenvironment{lemme}[0]{%
   \setbeamercolor{block title}{fg=structure,bg=structure!40}
   \setbeamercolor{block body}{fg=black,bg=structure!10}
   \begin{block}{\bf Lemme}
}{%
   \end{block}%
}

\newenvironment{remarque}[1][]{%
   \setbeamercolor{block title}{fg=black,bg=structure!20}
   \setbeamercolor{block body}{fg=black,bg=structure!5}
   \begin{block}{Remarque #1}
}{%
   \end{block}%
}


\newenvironment{exemple}[1][]{%
   \setbeamercolor{block title}{fg=black,bg=structure!20}
   \setbeamercolor{block body}{fg=black,bg=structure!5}
   \begin{block}{{\bf Exemple }#1}
}{%
   \end{block}%
}


\newenvironment{miniexercice}[0]{%
   \setbeamercolor{block title}{fg=structure,bg=structure!20}
   \setbeamercolor{block body}{fg=black,bg=structure!5}
   \begin{block}{Mini-exercices}
}{%
   \end{block}%
}


\newenvironment{tp}[0]{%
   \setbeamercolor{block title}{fg=structure,bg=structure!40}
   \setbeamercolor{block body}{fg=black,bg=structure!10}
   \begin{block}{\bf Travaux pratiques}
}{%
   \end{block}%
}
\newenvironment{exercicecours}[1][]{%
   \setbeamercolor{block title}{fg=structure,bg=structure!40}
   \setbeamercolor{block body}{fg=black,bg=structure!10}
   \begin{block}{{\bf Exercice }#1}
}{%
   \end{block}%
}
\newenvironment{algo}[1][]{%
   \setbeamercolor{block title}{fg=structure,bg=structure!40}
   \setbeamercolor{block body}{fg=black,bg=structure!10}
   \begin{block}{{\bf Algorithme}\hfill{\color{gray}\texttt{#1}}}
}{%
   \end{block}%
}


\setbeamertemplate{proof begin}{
   \setbeamercolor{block title}{fg=black,bg=structure!20}
   \setbeamercolor{block body}{fg=black,bg=structure!5}
   \begin{block}{{\footnotesize Démonstration}}
   \footnotesize
   \smallskip}
\setbeamertemplate{proof end}{%
   \end{block}}
\setbeamertemplate{qed symbol}{\openbox}


\makeatother
\usecolortheme[RGB={153, 204, 0}]{structure}

% Commande spécifique à ce chapitre
\newcommand{\entourer}[1]{\ovalbox{\color{myred}#1}}
\newcommand{\barrer}[1]{{\color{lightgray} #1}}
\newcounter{saveenumi}



%%%%%%%%%%%%%%%%%%%%%%%%%%%%%%%%%%%%%%%%%%%%%%%%%%%%%%%%%%%%%
%%%%%%%%%%%%%%%%%%%%%%%%%%%%%%%%%%%%%%%%%%%%%%%%%%%%%%%%%%%%%



\begin{document}



\title{{\bf Arithmétique}}
\subtitle{Nombres premiers}

\begin{frame}
  
  \debutmontitre

  \pause

{\footnotesize
\hfill
\setbeamercovered{transparent=50}
\begin{minipage}{0.6\textwidth}
  \begin{itemize}
    \item<3-> Une infinité de nombres premiers
    \item<4-> Eratosthène et Euclide
    \item<5-> Décomposition en facteurs premiers
  \end{itemize}
\end{minipage}
}

\end{frame}

\setcounter{framenumber}{0}


%%%%%%%%%%%%%%%%%%%%%%%%%%%%%%%%%%%%%%%%%%%%%%%%%%%%%%%%%%%%%%%%


%---------------------------------------------------------------
\section{Nombres premiers}

\begin{frame}

\begin{mydefinition}
Un \defi{nombre premier} $p$ est un entier $\ge 2$ dont les seuls diviseurs  
positifs sont $1$ et $p$
\end{mydefinition}

\pause
\bigskip

$2, 3, 5, 7, 11$ sont premiers

\pause
\medskip

$4 = 2 \times 2$, \quad $6=2 \times 3$, \quad $8= 2 \times 4$ ne sont pas premiers
\end{frame}


\begin{frame}
\begin{lemme}
\label{lem:divprem}
Tout entier $n \ge 2$ admet un diviseur qui est un nombre premier
\end{lemme}

\pause
\medskip

\begin{proof}
$$\mathcal{D} = \big\{ k \ge 2 \mid \  k | n \big\}$$ 

\pause
$\mathcal{D}$ est non vide (car $n \in \mathcal{D}$)

\pause 

Notons $p = \min \mathcal{D}$ 
alors $p$ divise $n$

\pause 
\medskip

Montrons que $p$ est un nombre premier

\pause

\quad Par l'absurde, supposons que $p$ ne soit pas premier

\quad il admet alors un diviseur $q$ tel que $1 < q < p$

\pause

\quad $q$ est aussi un diviseur de $n$ et donc $q \in \mathcal{D}$ avec $q<p$

\pause

\quad Ce qui donne une contradiction
\end{proof}
\end{frame}

\section{Une infinité de nombres premiers}

\begin{frame}
  
\begin{proposition}
Il existe une infinité de nombres premiers
\end{proposition}

\pause
\medskip

\begin{proof}
Par l'absurde, supposons qu'il n'y ait qu'un nombre fini de nombres premiers 

\pause 

Notons les $p_1=2$, $p_2=3$, $p_3$,\ldots, $p_n$

\pause 

Soit $N=p_1\times p_2\times \cdots \times p_n+ 1$

Soit $p$ un diviseur premier de $N$

\pause

Alors d'une part $p$ est l'un des entiers $p_i$
donc $p | p_1\times \cdots \times p_n$

\pause

D'autre part $p|N$ donc $p$ divise la différence $N-p_1\times \cdots \times p_n=1$

\pause

Cela implique que $p=1$

\pause 

Ce qui contredit que $p$ soit un nombre premier
\end{proof}

\end{frame}

%---------------------------------------------------------------
\section{Eratosthène et Euclide}

\begin{frame}
  

Le \evidence{crible d'Eratosthène}. Comment trouver les nombres premiers ? 
{\small
$$2\ \ 3\ \ 4\ \ 5\ \ 6\ \ 7\ \ 8\ \ 9\ \ 10\ \ 11\ \ 12\ \ 13\ \ 14\ \ 15\ \ 
16\ \ 17\ \ 18\ \ 19\ \ 20\ \ 21\ \ 22\ \ 23\ \ 24\ \ 25$$
\pause
$$\entourer{2}\ 3\ \ \barrer{4}\ \ 5\ \ \barrer{6}\ \ 7\ \ \barrer{8}\ \ 9\ \ \barrer{10}\ \ 11\ \ \barrer{12}\ \ 
13\ \ \barrer{14}\ \ 15\ \ 
\barrer{16}\ \ 17\ \ \barrer{18}\ \ 19\ \ \barrer{20}\ \ 21\ \ \barrer{22}\ \ 23\ \ \barrer{24}\ \ 25$$
\pause
$$\entourer{2}\ \entourer{3}\ \barrer{4}\ \ 5\ \ \barrer{6}\ \ 7\ \ \barrer{8}\ \ \barrer{9}\ \ \barrer{10}\ \ 11\ \ \barrer{12}\ \ 
13\ \ \barrer{14}\ \ \barrer{15}\ \ 
\barrer{16}\ \ 17\ \ \barrer{18}\ \ 19\ \ \barrer{20}\ \ \barrer{21}\ \ \barrer{22}\ \ 23\ \ \barrer{24}\ \ 25$$
\pause
$$\entourer{2}\ \entourer{3}\ \barrer{4}\ \entourer{5}\ \barrer{6}\ \ 7\ \ \barrer{8}\ \ \barrer{9}\ \ \barrer{10}\ \ 11\ \ \barrer{12}\ \ 
13\ \ \barrer{14}\ \ \barrer{15}\ \ 
\barrer{16}\ \ 17\ \ \barrer{18}\ \ 19\ \ \barrer{20}\ \ \barrer{21}\ \ \barrer{22}\ \ 23\ \ \barrer{24}\ \ \barrer{25}$$
\pause
$$\entourer{2}\ \entourer{3}\ \barrer{4}\ \entourer{5}\ \barrer{6}\ \entourer{7}\ \barrer{8}\ \ 
\barrer{9}\ \ \barrer{10}\ \entourer{11}\ \barrer{12}\ 
\entourer{13}\ \barrer{14}\ \ \barrer{15}\ \ 
\barrer{16}\ \entourer{17}\ \barrer{18}\ \entourer{19}\ \barrer{20}\ \ \barrer{21}\ \ \barrer{22}\ \entourer{23}
\ \barrer{24}\ \ \barrer{25}$$
}

\pause

Si un nombre $n$ n'est pas premier alors un de ses facteurs est $\le \sqrt{n}$

\pause

Par exemple pour tester si un nombre $\le 100$ est premier
\begin{itemize}
  \item il suffit de tester les diviseurs $\le 10$
\pause
  \item il suffit même de tester la divisibilité par $2, 3, 5$ et $7$
\pause
  \item {\small Exemple : $89$ n'est pas divisible par $2,3,5,7$, c'est un nombre premier}
\end{itemize}

\end{frame}



\begin{frame}
  
\begin{proposition}[Lemme d'Euclide]
Soit $p$ un nombre premier.
Si $p | ab$ alors $p|a$ ou $p | b$
\end{proposition}

\pause

\begin{proof}
Si $p$ ne divise pas $a$ alors $p$ et $a$ sont premiers entre eux 

\pause

Ainsi par le lemme de Gauss $p | b$
\end{proof}

\pause
\medskip

\begin{exemple}
Si $p$ est premier, $\sqrt{p} \notin \Qq$

\pause
\medskip

{\footnotesize
\quad Preuve par l'absurde : écrivons $\sqrt p =\frac ab$
\quad avec $a \in \Zz, b \in \Nn^*$ et $\pgcd(a,b)=1$

\pause

\quad $p b^2 = a^2$ donc $p | a^2$ donc par le lemme d'Euclide $p | a$

\pause

\quad On écrit $a = p a'$ avec $a'$ un entier donc $b^2 = p a'^2$

\pause

\quad Ainsi $p | b^2$ et donc $p|b$

\pause

\quad Maintenant $p|a$ et $p|b$ donc $a$ et $b$ ne sont pas premiers entre eux. Contradiction

\pause

\quad Conclusion : $\sqrt p$ n'est pas rationnel
}
\end{exemple}
\end{frame}



%---------------------------------------------------------------
\section{Décomposition en facteurs premiers}


\begin{frame}
  

\begin{theoreme}
Soit $n \ge 2$ un entier. Il \alt<1-2>{existe\ }{\evidence{existe}} des nombres premiers $p_1 < p_2 < \ldots < p_r$
et des exposants entiers $\alpha_1, \alpha_2, \ldots, \alpha_r \ge 1$ tels que :
$$n = p_1^{\alpha_1} \times p_2^{\alpha_2} \times \cdots \times p_r^{\alpha_r}$$
\pause
De plus les $p_i$ et les $\alpha_i$ ($i=1,\ldots,r$) sont \alt<2>{uniques}{\evidence{uniques}}
\end{theoreme}

\pause
\pause
\medskip

$24 = 2^3 \times 3$ est la décomposition en facteurs premiers

\pause
$36 = 2^2 \times 9$ n'est pas la décomposition en facteurs premiers

\pause
\bigskip

Pourquoi $1$ n'est pas un nombre premier ?

\pause
Il n'y aurait plus unicité de la décomposition 
\pause
\centerline{$24 = 2^3 \times 3 = 1 \times 2^3 \times 3 = 1^2 \times 2^3 \times 3 = \cdots$}
\end{frame}



\begin{frame}
\begin{exemple}
$$504 = 2^{\color{red}{3}} \times 3^{\color{red}{2}} \times 7 
\pause
\qquad 300 = 2^{\color{blue}{2}} \times 3 \times 5^{\color{blue}{2}}$$
\pause
$$504 = 2^{\color{red}{3}} \times 3^{\color{red}{2}} \times 5^{\color{red}{0}} \times 7^{\color{red}{1}} \qquad 
300 = 2^{\color{blue}{2}} \times 3^{\color{blue}{1}} \times 5^{\color{blue}{2}} \times 7^{\color{blue}{0}}$$

\pause
\medskip

Le pgcd s'obtient en prenant le plus petit exposant de chaque facteur premier
\pause
$$\pgcd(504,300) =  2^{\color{blue}{2}} \times 3^{\color{blue}{1}} \times 
5^{\color{red}{0}} \times 7^{\color{blue}{0}} \pause = 12$$

\pause
\medskip

Pour le ppcm on prend le plus grand exposant de chaque facteur premier

\pause
$$\text{ppcm}(504,300) =  2^{\color{red}{3}} \times 3^{\color{red}{2}} 
\times 5^{\color{blue}{2}} \times 7^{\color{red}{1}} = 12\,600$$
\end{exemple}

\end{frame}




%---------------------------------------------------------------
\section*{Mini-exercices}

\begin{frame}
\begin{miniexercice}
\begin{enumerate}
  \item Montrer que $n!+1$ n'est divisible par aucun des entiers $2,3,\ldots,n$. Est-ce toujours un nombre premier ?
  \item Trouver tous les nombres premiers $\le 103$.
  \item Décomposer $a=2\,340$ et $b=15\,288$ en facteurs premiers. Calculer leur pgcd et leur ppcm.
  \item Décomposer $48\,400$ en produit de facteurs premiers. Combien $48\,400$ admet-il de diviseurs ?
  \item Soient $a,b \ge 0$. \`A l'aide de la décomposition en facteurs premiers, 
reprouver la formule
$\pgcd(a,b) \times \text{ppcm}(a,b) = a \times b$.
\end{enumerate}  
\end{miniexercice}
\end{frame}


\end{document}