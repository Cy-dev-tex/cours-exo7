
%%%%%%%%%%%%%%%%%% PREAMBULE %%%%%%%%%%%%%%%%%%

\documentclass[aspectratio=169,utf8]{beamer}
%\documentclass[aspectratio=169,handout]{beamer}

\usetheme{Boadilla}
%\usecolortheme{seahorse}
\usecolortheme[RGB={245,66,24}]{structure}
\useoutertheme{infolines}

% packages
\usepackage{amsfonts,amsmath,amssymb,amsthm}
\usepackage[utf8]{inputenc}
\usepackage[T1]{fontenc}
\usepackage{lmodern}

\usepackage[francais]{babel}
\usepackage{fancybox}
\usepackage{graphicx}

\usepackage{float}
\usepackage{xfrac}

%\usepackage[usenames, x11names]{xcolor}
\usepackage{tikz}
\usepackage{pgfplots}
\usepackage{datetime}



%-----  Package unités -----
\usepackage{siunitx}
\sisetup{locale = FR,detect-all,per-mode = symbol}

%\usepackage{mathptmx}
%\usepackage{fouriernc}
%\usepackage{newcent}
%\usepackage[mathcal,mathbf]{euler}

%\usepackage{palatino}
%\usepackage{newcent}
% \usepackage[mathcal,mathbf]{euler}



% \usepackage{hyperref}
% \hypersetup{colorlinks=true, linkcolor=blue, urlcolor=blue,
% pdftitle={Exo7 - Exercices de mathématiques}, pdfauthor={Exo7}}


%section
% \usepackage{sectsty}
% \allsectionsfont{\bf}
%\sectionfont{\color{Tomato3}\upshape\selectfont}
%\subsectionfont{\color{Tomato4}\upshape\selectfont}

%----- Ensembles : entiers, reels, complexes -----
\newcommand{\Nn}{\mathbb{N}} \newcommand{\N}{\mathbb{N}}
\newcommand{\Zz}{\mathbb{Z}} \newcommand{\Z}{\mathbb{Z}}
\newcommand{\Qq}{\mathbb{Q}} \newcommand{\Q}{\mathbb{Q}}
\newcommand{\Rr}{\mathbb{R}} \newcommand{\R}{\mathbb{R}}
\newcommand{\Cc}{\mathbb{C}} 
\newcommand{\Kk}{\mathbb{K}} \newcommand{\K}{\mathbb{K}}

%----- Modifications de symboles -----
\renewcommand{\epsilon}{\varepsilon}
\renewcommand{\Re}{\mathop{\text{Re}}\nolimits}
\renewcommand{\Im}{\mathop{\text{Im}}\nolimits}
%\newcommand{\llbracket}{\left[\kern-0.15em\left[}
%\newcommand{\rrbracket}{\right]\kern-0.15em\right]}

\renewcommand{\ge}{\geqslant}
\renewcommand{\geq}{\geqslant}
\renewcommand{\le}{\leqslant}
\renewcommand{\leq}{\leqslant}
\renewcommand{\epsilon}{\varepsilon}

%----- Fonctions usuelles -----
\newcommand{\ch}{\mathop{\text{ch}}\nolimits}
\newcommand{\sh}{\mathop{\text{sh}}\nolimits}
\renewcommand{\tanh}{\mathop{\text{th}}\nolimits}
\newcommand{\cotan}{\mathop{\text{cotan}}\nolimits}
\newcommand{\Arcsin}{\mathop{\text{arcsin}}\nolimits}
\newcommand{\Arccos}{\mathop{\text{arccos}}\nolimits}
\newcommand{\Arctan}{\mathop{\text{arctan}}\nolimits}
\newcommand{\Argsh}{\mathop{\text{argsh}}\nolimits}
\newcommand{\Argch}{\mathop{\text{argch}}\nolimits}
\newcommand{\Argth}{\mathop{\text{argth}}\nolimits}
\newcommand{\pgcd}{\mathop{\text{pgcd}}\nolimits} 


%----- Commandes divers ------
\newcommand{\ii}{\mathrm{i}}
\newcommand{\dd}{\text{d}}
\newcommand{\id}{\mathop{\text{id}}\nolimits}
\newcommand{\Ker}{\mathop{\text{Ker}}\nolimits}
\newcommand{\Card}{\mathop{\text{Card}}\nolimits}
\newcommand{\Vect}{\mathop{\text{Vect}}\nolimits}
\newcommand{\Mat}{\mathop{\text{Mat}}\nolimits}
\newcommand{\rg}{\mathop{\text{rg}}\nolimits}
\newcommand{\tr}{\mathop{\text{tr}}\nolimits}


%----- Structure des exercices ------

\newtheoremstyle{styleexo}% name
{2ex}% Space above
{3ex}% Space below
{}% Body font
{}% Indent amount 1
{\bfseries} % Theorem head font
{}% Punctuation after theorem head
{\newline}% Space after theorem head 2
{}% Theorem head spec (can be left empty, meaning ‘normal’)

%\theoremstyle{styleexo}
\newtheorem{exo}{Exercice}
\newtheorem{ind}{Indications}
\newtheorem{cor}{Correction}


\newcommand{\exercice}[1]{} \newcommand{\finexercice}{}
%\newcommand{\exercice}[1]{{\tiny\texttt{#1}}\vspace{-2ex}} % pour afficher le numero absolu, l'auteur...
\newcommand{\enonce}{\begin{exo}} \newcommand{\finenonce}{\end{exo}}
\newcommand{\indication}{\begin{ind}} \newcommand{\finindication}{\end{ind}}
\newcommand{\correction}{\begin{cor}} \newcommand{\fincorrection}{\end{cor}}

\newcommand{\noindication}{\stepcounter{ind}}
\newcommand{\nocorrection}{\stepcounter{cor}}

\newcommand{\fiche}[1]{} \newcommand{\finfiche}{}
\newcommand{\titre}[1]{\centerline{\large \bf #1}}
\newcommand{\addcommand}[1]{}
\newcommand{\video}[1]{}

% Marge
\newcommand{\mymargin}[1]{\marginpar{{\small #1}}}

\def\noqed{\renewcommand{\qedsymbol}{}}


%----- Presentation ------
\setlength{\parindent}{0cm}

%\newcommand{\ExoSept}{\href{http://exo7.emath.fr}{\textbf{\textsf{Exo7}}}}

\definecolor{myred}{rgb}{0.93,0.26,0}
\definecolor{myorange}{rgb}{0.97,0.58,0}
\definecolor{myyellow}{rgb}{1,0.86,0}

\newcommand{\LogoExoSept}[1]{  % input : echelle
{\usefont{U}{cmss}{bx}{n}
\begin{tikzpicture}[scale=0.1*#1,transform shape]
  \fill[color=myorange] (0,0)--(4,0)--(4,-4)--(0,-4)--cycle;
  \fill[color=myred] (0,0)--(0,3)--(-3,3)--(-3,0)--cycle;
  \fill[color=myyellow] (4,0)--(7,4)--(3,7)--(0,3)--cycle;
  \node[scale=5] at (3.5,3.5) {Exo7};
\end{tikzpicture}}
}


\newcommand{\debutmontitre}{
  \author{} \date{} 
  \thispagestyle{empty}
  \hspace*{-10ex}
  \begin{minipage}{\textwidth}
    \titlepage  
  \vspace*{-2.5cm}
  \begin{center}
    \LogoExoSept{2.5}
  \end{center}
  \end{minipage}

  \vspace*{-0cm}
  
  % Astuce pour que le background ne soit pas discrétisé lors de la conversion pdf -> png
\begin{tikzpicture}
        \fill[opacity=0,green!60!black] (0,0)--++(0,0)--++(0,0)--++(0,0)--cycle; 
\end{tikzpicture}

% toc S'affiche trop tot :
% \tableofcontents[hideallsubsections, pausesections]
}

\newcommand{\finmontitre}{
  \end{frame}
  \setcounter{framenumber}{0}
} % ne marche pas pour une raison obscure

%----- Commandes supplementaires ------

% \usepackage[landscape]{geometry}
% \geometry{top=1cm, bottom=3cm, left=2cm, right=10cm, marginparsep=1cm
% }
% \usepackage[a4paper]{geometry}
% \geometry{top=2cm, bottom=2cm, left=2cm, right=2cm, marginparsep=1cm
% }

%\usepackage{standalone}


% New command Arnaud -- november 2011
\setbeamersize{text margin left=24ex}
% si vous modifier cette valeur il faut aussi
% modifier le decalage du titre pour compenser
% (ex : ici =+10ex, titre =-5ex

\theoremstyle{definition}
%\newtheorem{proposition}{Proposition}
%\newtheorem{exemple}{Exemple}
%\newtheorem{theoreme}{Théorème}
%\newtheorem{lemme}{Lemme}
%\newtheorem{corollaire}{Corollaire}
%\newtheorem*{remarque*}{Remarque}
%\newtheorem*{miniexercice}{Mini-exercices}
%\newtheorem{definition}{Définition}

% Commande tikz
\usetikzlibrary{calc}
\usetikzlibrary{patterns,arrows}
\usetikzlibrary{matrix}
\usetikzlibrary{fadings} 

%definition d'un terme
\newcommand{\defi}[1]{{\color{myorange}\textbf{\emph{#1}}}}
\newcommand{\evidence}[1]{{\color{blue}\textbf{\emph{#1}}}}
\newcommand{\assertion}[1]{\emph{\og#1\fg}}  % pour chapitre logique
%\renewcommand{\contentsname}{Sommaire}
\renewcommand{\contentsname}{}
\setcounter{tocdepth}{2}



%------ Figures ------

\def\myscale{1} % par défaut 
\newcommand{\myfigure}[2]{  % entrée : echelle, fichier figure
\def\myscale{#1}
\begin{center}
\footnotesize
{#2}
\end{center}}


%------ Encadrement ------

\usepackage{fancybox}


\newcommand{\mybox}[1]{
\setlength{\fboxsep}{7pt}
\begin{center}
\shadowbox{#1}
\end{center}}

\newcommand{\myboxinline}[1]{
\setlength{\fboxsep}{5pt}
\raisebox{-10pt}{
\shadowbox{#1}
}
}

%--------------- Commande beamer---------------
\newcommand{\beameronly}[1]{#1} % permet de mettre des pause dans beamer pas dans poly


\setbeamertemplate{navigation symbols}{}
\setbeamertemplate{footline}  % tiré du fichier beamerouterinfolines.sty
{
  \leavevmode%
  \hbox{%
  \begin{beamercolorbox}[wd=.333333\paperwidth,ht=2.25ex,dp=1ex,center]{author in head/foot}%
    % \usebeamerfont{author in head/foot}\insertshortauthor%~~(\insertshortinstitute)
    \usebeamerfont{section in head/foot}{\bf\insertshorttitle}
  \end{beamercolorbox}%
  \begin{beamercolorbox}[wd=.333333\paperwidth,ht=2.25ex,dp=1ex,center]{title in head/foot}%
    \usebeamerfont{section in head/foot}{\bf\insertsectionhead}
  \end{beamercolorbox}%
  \begin{beamercolorbox}[wd=.333333\paperwidth,ht=2.25ex,dp=1ex,right]{date in head/foot}%
    % \usebeamerfont{date in head/foot}\insertshortdate{}\hspace*{2em}
    \insertframenumber{} / \inserttotalframenumber\hspace*{2ex} 
  \end{beamercolorbox}}%
  \vskip0pt%
}


\definecolor{mygrey}{rgb}{0.5,0.5,0.5}
\setlength{\parindent}{0cm}
%\DeclareTextFontCommand{\helvetica}{\fontfamily{phv}\selectfont}

% background beamer
\definecolor{couleurhaut}{rgb}{0.85,0.9,1}  % creme
\definecolor{couleurmilieu}{rgb}{1,1,1}  % vert pale
\definecolor{couleurbas}{rgb}{0.85,0.9,1}  % blanc
\setbeamertemplate{background canvas}[vertical shading]%
[top=couleurhaut,middle=couleurmilieu,midpoint=0.4,bottom=couleurbas] 
%[top=fondtitre!05,bottom=fondtitre!60]



\makeatletter
\setbeamertemplate{theorem begin}
{%
  \begin{\inserttheoremblockenv}
  {%
    \inserttheoremheadfont
    \inserttheoremname
    \inserttheoremnumber
    \ifx\inserttheoremaddition\@empty\else\ (\inserttheoremaddition)\fi%
    \inserttheorempunctuation
  }%
}
\setbeamertemplate{theorem end}{\end{\inserttheoremblockenv}}

\newenvironment{theoreme}[1][]{%
   \setbeamercolor{block title}{fg=structure,bg=structure!40}
   \setbeamercolor{block body}{fg=black,bg=structure!10}
   \begin{block}{{\bf Th\'eor\`eme }#1}
}{%
   \end{block}%
}


\newenvironment{proposition}[1][]{%
   \setbeamercolor{block title}{fg=structure,bg=structure!40}
   \setbeamercolor{block body}{fg=black,bg=structure!10}
   \begin{block}{{\bf Proposition }#1}
}{%
   \end{block}%
}

\newenvironment{corollaire}[1][]{%
   \setbeamercolor{block title}{fg=structure,bg=structure!40}
   \setbeamercolor{block body}{fg=black,bg=structure!10}
   \begin{block}{{\bf Corollaire }#1}
}{%
   \end{block}%
}

\newenvironment{mydefinition}[1][]{%
   \setbeamercolor{block title}{fg=structure,bg=structure!40}
   \setbeamercolor{block body}{fg=black,bg=structure!10}
   \begin{block}{{\bf Définition} #1}
}{%
   \end{block}%
}

\newenvironment{lemme}[0]{%
   \setbeamercolor{block title}{fg=structure,bg=structure!40}
   \setbeamercolor{block body}{fg=black,bg=structure!10}
   \begin{block}{\bf Lemme}
}{%
   \end{block}%
}

\newenvironment{remarque}[1][]{%
   \setbeamercolor{block title}{fg=black,bg=structure!20}
   \setbeamercolor{block body}{fg=black,bg=structure!5}
   \begin{block}{Remarque #1}
}{%
   \end{block}%
}


\newenvironment{exemple}[1][]{%
   \setbeamercolor{block title}{fg=black,bg=structure!20}
   \setbeamercolor{block body}{fg=black,bg=structure!5}
   \begin{block}{{\bf Exemple }#1}
}{%
   \end{block}%
}


\newenvironment{miniexercice}[0]{%
   \setbeamercolor{block title}{fg=structure,bg=structure!20}
   \setbeamercolor{block body}{fg=black,bg=structure!5}
   \begin{block}{Mini-exercices}
}{%
   \end{block}%
}


\newenvironment{tp}[0]{%
   \setbeamercolor{block title}{fg=structure,bg=structure!40}
   \setbeamercolor{block body}{fg=black,bg=structure!10}
   \begin{block}{\bf Travaux pratiques}
}{%
   \end{block}%
}
\newenvironment{exercicecours}[1][]{%
   \setbeamercolor{block title}{fg=structure,bg=structure!40}
   \setbeamercolor{block body}{fg=black,bg=structure!10}
   \begin{block}{{\bf Exercice }#1}
}{%
   \end{block}%
}
\newenvironment{algo}[1][]{%
   \setbeamercolor{block title}{fg=structure,bg=structure!40}
   \setbeamercolor{block body}{fg=black,bg=structure!10}
   \begin{block}{{\bf Algorithme}\hfill{\color{gray}\texttt{#1}}}
}{%
   \end{block}%
}


\setbeamertemplate{proof begin}{
   \setbeamercolor{block title}{fg=black,bg=structure!20}
   \setbeamercolor{block body}{fg=black,bg=structure!5}
   \begin{block}{{\footnotesize Démonstration}}
   \footnotesize
   \smallskip}
\setbeamertemplate{proof end}{%
   \end{block}}
\setbeamertemplate{qed symbol}{\openbox}


\makeatother
\usecolortheme[RGB={153, 204, 0}]{structure}

% Commande spécifique à ce chapitre
\newcounter{saveenumi}

%%%%%%%%%%%%%%%%%%%%%%%%%%%%%%%%%%%%%%%%%%%%%%%%%%%%%%%%%%%%%
%%%%%%%%%%%%%%%%%%%%%%%%%%%%%%%%%%%%%%%%%%%%%%%%%%%%%%%%%%%%%



\begin{document}



\title{{\bf Arithmétique}}
\subtitle{Congruences}

\begin{frame}
  
  \debutmontitre

  \pause

{\footnotesize
\hfill
\setbeamercovered{transparent=50}
\begin{minipage}{0.6\textwidth}
  \begin{itemize}
    \item<3-> Définition
    \item<4-> \'Equation de congruence $ax \equiv b \pmod n$
    \item<5-> Petit théorème de Fermat    
  \end{itemize}
\end{minipage}
}

\end{frame}

\setcounter{framenumber}{0}


%%%%%%%%%%%%%%%%%%%%%%%%%%%%%%%%%%%%%%%%%%%%%%%%%%%%%%%%%%%%%%%%

 

%---------------------------------------------------------------
\section{Définition}

\begin{frame}

\begin{mydefinition}
Fixons $n \ge 2$. $a$ est \defi{congru} à $b$ \defi{modulo} $n$ si $n$ divise $b-a$

\pause

On note 
$$a \equiv b \pmod n$$ 
\end{mydefinition}

\pause

\mybox{$a \equiv b \pmod n \quad \iff \quad \exists k \in \Zz \quad a = b +kn$}

\pause
\begin{itemize}
  \item Exemples : $15 \equiv 1 \pmod 7$, \pause $72 \equiv 2 \pmod {7}$, \pause $3 \equiv -11 \pmod{7}$
\pause

  \item $a\equiv 0 \pmod n \iff n \text{ divise } a$
\pause
  \item Autres notations : $a=b \pmod n$ \quad ou \quad  $a \equiv b\  [n]$
\pause  
  \item Si la division euclidienne de $a$ par $n$ est $a = bn + r$ (et $0 \le r < n$) 
alors $a \equiv r \pmod n$ 
\pause
  \item La relation <<congru modulo $n$>> est une relation d'équivalence
\pause
  \begin{itemize}
    \item $a \equiv a \pmod n$
\pause
    \item si $a \equiv b \pmod n$ alors $b \equiv a \pmod n$
\pause
    \item si $a \equiv b \pmod n$ et $b \equiv c \pmod n$ alors $a \equiv c \pmod n$
  \end{itemize}
\end{itemize}

\end{frame}

%---------------------------------------------------------------
\section{Opérations}

\begin{frame}

\begin{proposition}
Soient $a,b,c,d \in \Zz$ tels que  $a \equiv b \pmod n$ et $c \equiv d \pmod n$
\pause
\begin{enumerate}
  \item  $a+c \equiv b+d \pmod n$
\pause
  \item $a\times c \equiv b \times d \pmod n$
\pause
  \item $a^k \equiv b^k \pmod n$ (pour tout $k\ge 0$)
\end{enumerate}
\end{proposition}

\pause

\begin{exemple}
\begin{itemize}
  \item $5x+8 \equiv 3 \pmod 5$ \quad pour tout $x \in \Zz$
\pause
  \item $11^{2345} \equiv 1^{2345} \equiv 1 \pmod {10}$
\end{itemize}
\end{exemple}

\pause

{\footnotesize
Montrons que si $a \equiv b \pmod n$ et $c \equiv d \pmod n$ alors $ac \equiv bd \pmod n$

\pause

\quad $a \equiv b \pmod n$ donc il existe $k \in \Zz$ tel que $a=b + kn$ 

\pause

\quad $c \equiv d \pmod n$ donc il existe $\ell \in \Zz$ tel que $c = d + \ell n$

\pause

\quad Alors $a\times c = (b+kn)\times(d+\ell n) = bd + (b \ell + d k + k\ell n)n = bd + mn$ 

\pause

\quad Ainsi $ac \equiv bd \pmod n$
}

\end{frame}

%---------------------------------------------------------------
\section{Divisibilité par $9$}

\begin{frame}

\begin{exemple}[Critère de divisibilité par $9$]

\pause

\mybox{\!$N$ est divisible par $9$ \!$\iff$\! la somme de ses chiffres est divisible par $9$\!}

\pause
\vspace*{-3ex}
\begin{itemize}
  \item $9 | N$ équivaut à $N \equiv 0 \pmod 9$
\pause
  \item $10 \equiv 1 \pmod 9$, \pause $10^2 \equiv 1 \pmod 9$, \pause $10^3 \equiv 1 \pmod 9$,...
\pause
  \item $N = \underline{a_k \cdots a_2 a_1 a_0} 
\pause
= 10^k a_k + \cdots + 10^2 a_2 + 10^1 a_1 + a_0$
\pause
  \item 
\vspace*{-1ex}
$$\begin{array}{rl}
N &=  10^k a_k + \cdots + 10^2 a_2 + 10^1 a_1 + a_0\\
\pause
  &\equiv  a_k + \cdots + a_2 + a_1 + a_0 \pmod 9 \\
\end{array}$$
\vspace*{-2ex}\pause
  \item $N \equiv 0 \pmod 9$ $\iff$ la somme des chiffres vaut $0$ modulo $9$
\end{itemize}

\pause
\vspace*{-1ex}
$$\begin{array}{rcl}
 N &=& 488\, 889 \\
\pause
   &=& 4 \cdot 10^5 + 8 \cdot 10^4 + 8 \cdot 10^3 + 8 \cdot 10^2 + 8 \cdot 10 + 9   \\
\pause
   &\equiv& 4 + 8 + 8 + 8 + 8 + 9 \pmod{9} \\
\pause
   &\equiv& 45 
\pause \ \equiv \ 9 \ 
\pause \equiv \ 0 \pmod{9}\\
\end{array}$$
\end{exemple}
  
\end{frame}

%---------------------------------------------------------------
\section{Calculs}

\begin{frame}
  
\begin{exemple}
Calcul de $2^{21} \pmod {37}$

\pause

\begin{enumerate}
  \item On calcule $2^{21}$, puis le reste de la division euclidienne de $2^{21}$ par $37$

\pause

  \item Calculs successifs des $2^k$ modulo $37$ 

\pause

\begin{itemize}
  \item $2^1 \equiv 2 \pmod {37}$, \pause $2^2 \equiv 4 \pmod {37}$, \pause $2^3 \equiv 8 \pmod {37}$, \pause $2^4 \equiv 16 \pmod {37}$,
\pause $2^5 \equiv 32 \pmod {37}$
\pause
  \item $2^6 \equiv 64 \pause \equiv 27 \pmod {37}$
\pause
  \item $2^7 \pause \equiv 2 \cdot 2^6 \pause \equiv 2 \cdot 27 \pause \equiv 54 \pause \equiv 17 \pmod {37}$
\pause
  \item $2^{8} \equiv 34 \pause \equiv -3 \pmod {37}$
\pause
  \item $2^9 \equiv 2 \cdot 2^8 \pause \equiv 2 \cdot(-3) \equiv -6 \pause \equiv 31 \pmod{37}$ \quad ...
\end{itemize}

\pause

  \item Méthode efficace
\begin{itemize}
\pause
  \item L'exposant $21$ écrit en base $2$ : $21 = 2^{4} + 2^{2} + 2^{0} = 16 + 4 + 1$
\pause
  \item $2^{21} =  2^{16} \cdot 2^{4} \cdot 2^{1}$
\pause
  \item $2^8 \equiv  (2^4)^2 \equiv 16^2 \equiv 256 \equiv 34 \equiv -3 \pmod {37}$
\pause
  \item $2^{16} \equiv  \left(2^{8}\right)^2 \pause \equiv (-3)^2 \equiv 9 \pmod{37}$
\pause
  \item $2^{21} \equiv  2^{16} \cdot 2^{4} \cdot 2^{1} \pause \equiv 9 \times 16 \times 2 
\pause \equiv 288 \pause \equiv 29 \pmod {37}$
\end{itemize}

\end{enumerate}
\end{exemple}
\end{frame}


%---------------------------------------------------------------
\section{\'Equation de congruence $ax \equiv b \pmod n$}

\begin{frame}
  
\mybox{$ax \equiv b \pmod n$}
\pause
\vspace*{-1ex}
\begin{exemple}
\vspace*{-1ex}
$$9x \equiv 6 \pmod{24}$$
\pause
\vspace*{-4ex}
{\small
\begin{itemize}[<+->]
  \item Il existe $x\in \Zz$ tel que $9x \equiv 6 \pmod{24}$ $\iff$ il existe $x,k \in \Zz$ tels que $9x = 6 + 24k$
  \item L'équation $9x-24k=6$ a des solutions car $\pgcd(9,24)=3$ divise $6$
  \item En divisant par le pgcd on obtient l'équation équivalente $3x-8k=2$
  \item Solution particulière $(x_0=6,k_0=2)$ (à la main ou algorithme d'Euclide)
  \item On sait trouver toutes les solutions : \myboxinline{$x=6+ 8\ell$, $\ell \in \Zz$}
  \item On regroupe les solutions en classes modulo $24$ :
\vspace*{-1ex}
\mybox{$x_1 = 6 + 24 m, \quad x_2 = 14 + 24 m, \quad x_3=22+24m \quad \text{ avec } m\in\Zz$}
\end{itemize}
}

\end{exemple}
\end{frame}


%---------------------------------------------------------------
\section{Petit théorème de Fermat}


\begin{frame}
  


\begin{theoreme}[Petit théorème de Fermat]
Si $p$ est un nombre premier et $a \in \Zz$ alors
\myboxinline{$a^p \equiv a \pmod p$}
\end{theoreme}

\pause
\bigskip
\bigskip

\begin{corollaire}
Si $p$ ne divise pas $a$ alors
\myboxinline{$a^{p-1} \equiv 1 \pmod p$} 
\end{corollaire}
\end{frame}


\begin{frame}
\begin{exemple}

Calcul de $14^{3141} \pmod {17}$

\pause

\begin{itemize}
  \item $14^{16} \equiv 1 \pmod {17}$ car $17$ est un nombre premier
\pause
  \item Division euclidienne de $3141$ par $16$ : $3141 = 16\times 196 + 5$
\pause
$$
\begin{array}{rl}
14^{3141} &\equiv 14^{16 \times 196 + 5} 
\pause
          \equiv 14^{16\times 196}\times 14^5 \\
\pause
          &\equiv \left(\alert<7->{14^{16}}\right)^{196}\times 14^5 
\pause
\equiv \textcolor{red}{1}^{196} \times 14^5 
\pause
\equiv  14^5\pmod{17}
\end{array}$$
\pause
  \item Calcul de $14^5 \pmod{17}$
  \begin{itemize}
\pause
     \item $14 \equiv -3 \pmod {17}$ 
\pause
     \item $14^2\equiv (-3)^2 \equiv 9 \pmod {17}$
\pause
     \item $14^3 \equiv 14^2 \times 14 \equiv 9 \times (-3) \equiv -27 \equiv 7 \pmod{17}$
\pause
     \item $14^5 \equiv 14^2 \times 14^3 \equiv 9 \times 7 \equiv 63 \equiv 12 \pmod {17}$
  \end{itemize}
\pause
  \item Conclusion $14^{3141}  \equiv 14^5 \equiv 12 \pmod {17}$
\end{itemize}

\end{exemple}
\end{frame}



%---------------------------------------------------------------
\section*{Mini-exercices}

\begin{frame}
\begin{miniexercice}
\begin{enumerate}
  \item Calculer les restes modulo $10$ de $122+455$, $122\times 455$, $122^{455}$.
Mêmes calculs modulo $11$, puis modulo $12$.
  \item Prouver qu'un entier est divisible par $3$ si et seulement 
si la somme de ses chiffres est divisible par $3$.
  \item Calculer $3^{10} \pmod {23}$. 
  \item Calculer $3^{100} \pmod {23}$.
  \item Résoudre les équations $3x\equiv 4 \pmod{7}$, $4x \equiv 14 \pmod {30}$.
\end{enumerate} 
\end{miniexercice}
\end{frame}


\end{document}