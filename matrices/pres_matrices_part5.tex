
%%%%%%%%%%%%%%%%%% PREAMBULE %%%%%%%%%%%%%%%%%%

\documentclass[aspectratio=169,utf8]{beamer}
%\documentclass[aspectratio=169,handout]{beamer}

\usetheme{Boadilla}
%\usecolortheme{seahorse}
\usecolortheme[RGB={245,66,24}]{structure}
\useoutertheme{infolines}

% packages
\usepackage{amsfonts,amsmath,amssymb,amsthm}
\usepackage[utf8]{inputenc}
\usepackage[T1]{fontenc}
\usepackage{lmodern}

\usepackage[francais]{babel}
\usepackage{fancybox}
\usepackage{graphicx}

\usepackage{float}
\usepackage{xfrac}

%\usepackage[usenames, x11names]{xcolor}
\usepackage{tikz}
\usepackage{pgfplots}
\usepackage{datetime}



%-----  Package unités -----
\usepackage{siunitx}
\sisetup{locale = FR,detect-all,per-mode = symbol}

%\usepackage{mathptmx}
%\usepackage{fouriernc}
%\usepackage{newcent}
%\usepackage[mathcal,mathbf]{euler}

%\usepackage{palatino}
%\usepackage{newcent}
% \usepackage[mathcal,mathbf]{euler}



% \usepackage{hyperref}
% \hypersetup{colorlinks=true, linkcolor=blue, urlcolor=blue,
% pdftitle={Exo7 - Exercices de mathématiques}, pdfauthor={Exo7}}


%section
% \usepackage{sectsty}
% \allsectionsfont{\bf}
%\sectionfont{\color{Tomato3}\upshape\selectfont}
%\subsectionfont{\color{Tomato4}\upshape\selectfont}

%----- Ensembles : entiers, reels, complexes -----
\newcommand{\Nn}{\mathbb{N}} \newcommand{\N}{\mathbb{N}}
\newcommand{\Zz}{\mathbb{Z}} \newcommand{\Z}{\mathbb{Z}}
\newcommand{\Qq}{\mathbb{Q}} \newcommand{\Q}{\mathbb{Q}}
\newcommand{\Rr}{\mathbb{R}} \newcommand{\R}{\mathbb{R}}
\newcommand{\Cc}{\mathbb{C}} 
\newcommand{\Kk}{\mathbb{K}} \newcommand{\K}{\mathbb{K}}

%----- Modifications de symboles -----
\renewcommand{\epsilon}{\varepsilon}
\renewcommand{\Re}{\mathop{\text{Re}}\nolimits}
\renewcommand{\Im}{\mathop{\text{Im}}\nolimits}
%\newcommand{\llbracket}{\left[\kern-0.15em\left[}
%\newcommand{\rrbracket}{\right]\kern-0.15em\right]}

\renewcommand{\ge}{\geqslant}
\renewcommand{\geq}{\geqslant}
\renewcommand{\le}{\leqslant}
\renewcommand{\leq}{\leqslant}
\renewcommand{\epsilon}{\varepsilon}

%----- Fonctions usuelles -----
\newcommand{\ch}{\mathop{\text{ch}}\nolimits}
\newcommand{\sh}{\mathop{\text{sh}}\nolimits}
\renewcommand{\tanh}{\mathop{\text{th}}\nolimits}
\newcommand{\cotan}{\mathop{\text{cotan}}\nolimits}
\newcommand{\Arcsin}{\mathop{\text{arcsin}}\nolimits}
\newcommand{\Arccos}{\mathop{\text{arccos}}\nolimits}
\newcommand{\Arctan}{\mathop{\text{arctan}}\nolimits}
\newcommand{\Argsh}{\mathop{\text{argsh}}\nolimits}
\newcommand{\Argch}{\mathop{\text{argch}}\nolimits}
\newcommand{\Argth}{\mathop{\text{argth}}\nolimits}
\newcommand{\pgcd}{\mathop{\text{pgcd}}\nolimits} 


%----- Commandes divers ------
\newcommand{\ii}{\mathrm{i}}
\newcommand{\dd}{\text{d}}
\newcommand{\id}{\mathop{\text{id}}\nolimits}
\newcommand{\Ker}{\mathop{\text{Ker}}\nolimits}
\newcommand{\Card}{\mathop{\text{Card}}\nolimits}
\newcommand{\Vect}{\mathop{\text{Vect}}\nolimits}
\newcommand{\Mat}{\mathop{\text{Mat}}\nolimits}
\newcommand{\rg}{\mathop{\text{rg}}\nolimits}
\newcommand{\tr}{\mathop{\text{tr}}\nolimits}


%----- Structure des exercices ------

\newtheoremstyle{styleexo}% name
{2ex}% Space above
{3ex}% Space below
{}% Body font
{}% Indent amount 1
{\bfseries} % Theorem head font
{}% Punctuation after theorem head
{\newline}% Space after theorem head 2
{}% Theorem head spec (can be left empty, meaning ‘normal’)

%\theoremstyle{styleexo}
\newtheorem{exo}{Exercice}
\newtheorem{ind}{Indications}
\newtheorem{cor}{Correction}


\newcommand{\exercice}[1]{} \newcommand{\finexercice}{}
%\newcommand{\exercice}[1]{{\tiny\texttt{#1}}\vspace{-2ex}} % pour afficher le numero absolu, l'auteur...
\newcommand{\enonce}{\begin{exo}} \newcommand{\finenonce}{\end{exo}}
\newcommand{\indication}{\begin{ind}} \newcommand{\finindication}{\end{ind}}
\newcommand{\correction}{\begin{cor}} \newcommand{\fincorrection}{\end{cor}}

\newcommand{\noindication}{\stepcounter{ind}}
\newcommand{\nocorrection}{\stepcounter{cor}}

\newcommand{\fiche}[1]{} \newcommand{\finfiche}{}
\newcommand{\titre}[1]{\centerline{\large \bf #1}}
\newcommand{\addcommand}[1]{}
\newcommand{\video}[1]{}

% Marge
\newcommand{\mymargin}[1]{\marginpar{{\small #1}}}

\def\noqed{\renewcommand{\qedsymbol}{}}


%----- Presentation ------
\setlength{\parindent}{0cm}

%\newcommand{\ExoSept}{\href{http://exo7.emath.fr}{\textbf{\textsf{Exo7}}}}

\definecolor{myred}{rgb}{0.93,0.26,0}
\definecolor{myorange}{rgb}{0.97,0.58,0}
\definecolor{myyellow}{rgb}{1,0.86,0}

\newcommand{\LogoExoSept}[1]{  % input : echelle
{\usefont{U}{cmss}{bx}{n}
\begin{tikzpicture}[scale=0.1*#1,transform shape]
  \fill[color=myorange] (0,0)--(4,0)--(4,-4)--(0,-4)--cycle;
  \fill[color=myred] (0,0)--(0,3)--(-3,3)--(-3,0)--cycle;
  \fill[color=myyellow] (4,0)--(7,4)--(3,7)--(0,3)--cycle;
  \node[scale=5] at (3.5,3.5) {Exo7};
\end{tikzpicture}}
}


\newcommand{\debutmontitre}{
  \author{} \date{} 
  \thispagestyle{empty}
  \hspace*{-10ex}
  \begin{minipage}{\textwidth}
    \titlepage  
  \vspace*{-2.5cm}
  \begin{center}
    \LogoExoSept{2.5}
  \end{center}
  \end{minipage}

  \vspace*{-0cm}
  
  % Astuce pour que le background ne soit pas discrétisé lors de la conversion pdf -> png
\begin{tikzpicture}
        \fill[opacity=0,green!60!black] (0,0)--++(0,0)--++(0,0)--++(0,0)--cycle; 
\end{tikzpicture}

% toc S'affiche trop tot :
% \tableofcontents[hideallsubsections, pausesections]
}

\newcommand{\finmontitre}{
  \end{frame}
  \setcounter{framenumber}{0}
} % ne marche pas pour une raison obscure

%----- Commandes supplementaires ------

% \usepackage[landscape]{geometry}
% \geometry{top=1cm, bottom=3cm, left=2cm, right=10cm, marginparsep=1cm
% }
% \usepackage[a4paper]{geometry}
% \geometry{top=2cm, bottom=2cm, left=2cm, right=2cm, marginparsep=1cm
% }

%\usepackage{standalone}


% New command Arnaud -- november 2011
\setbeamersize{text margin left=24ex}
% si vous modifier cette valeur il faut aussi
% modifier le decalage du titre pour compenser
% (ex : ici =+10ex, titre =-5ex

\theoremstyle{definition}
%\newtheorem{proposition}{Proposition}
%\newtheorem{exemple}{Exemple}
%\newtheorem{theoreme}{Théorème}
%\newtheorem{lemme}{Lemme}
%\newtheorem{corollaire}{Corollaire}
%\newtheorem*{remarque*}{Remarque}
%\newtheorem*{miniexercice}{Mini-exercices}
%\newtheorem{definition}{Définition}

% Commande tikz
\usetikzlibrary{calc}
\usetikzlibrary{patterns,arrows}
\usetikzlibrary{matrix}
\usetikzlibrary{fadings} 

%definition d'un terme
\newcommand{\defi}[1]{{\color{myorange}\textbf{\emph{#1}}}}
\newcommand{\evidence}[1]{{\color{blue}\textbf{\emph{#1}}}}
\newcommand{\assertion}[1]{\emph{\og#1\fg}}  % pour chapitre logique
%\renewcommand{\contentsname}{Sommaire}
\renewcommand{\contentsname}{}
\setcounter{tocdepth}{2}



%------ Figures ------

\def\myscale{1} % par défaut 
\newcommand{\myfigure}[2]{  % entrée : echelle, fichier figure
\def\myscale{#1}
\begin{center}
\footnotesize
{#2}
\end{center}}


%------ Encadrement ------

\usepackage{fancybox}


\newcommand{\mybox}[1]{
\setlength{\fboxsep}{7pt}
\begin{center}
\shadowbox{#1}
\end{center}}

\newcommand{\myboxinline}[1]{
\setlength{\fboxsep}{5pt}
\raisebox{-10pt}{
\shadowbox{#1}
}
}

%--------------- Commande beamer---------------
\newcommand{\beameronly}[1]{#1} % permet de mettre des pause dans beamer pas dans poly


\setbeamertemplate{navigation symbols}{}
\setbeamertemplate{footline}  % tiré du fichier beamerouterinfolines.sty
{
  \leavevmode%
  \hbox{%
  \begin{beamercolorbox}[wd=.333333\paperwidth,ht=2.25ex,dp=1ex,center]{author in head/foot}%
    % \usebeamerfont{author in head/foot}\insertshortauthor%~~(\insertshortinstitute)
    \usebeamerfont{section in head/foot}{\bf\insertshorttitle}
  \end{beamercolorbox}%
  \begin{beamercolorbox}[wd=.333333\paperwidth,ht=2.25ex,dp=1ex,center]{title in head/foot}%
    \usebeamerfont{section in head/foot}{\bf\insertsectionhead}
  \end{beamercolorbox}%
  \begin{beamercolorbox}[wd=.333333\paperwidth,ht=2.25ex,dp=1ex,right]{date in head/foot}%
    % \usebeamerfont{date in head/foot}\insertshortdate{}\hspace*{2em}
    \insertframenumber{} / \inserttotalframenumber\hspace*{2ex} 
  \end{beamercolorbox}}%
  \vskip0pt%
}


\definecolor{mygrey}{rgb}{0.5,0.5,0.5}
\setlength{\parindent}{0cm}
%\DeclareTextFontCommand{\helvetica}{\fontfamily{phv}\selectfont}

% background beamer
\definecolor{couleurhaut}{rgb}{0.85,0.9,1}  % creme
\definecolor{couleurmilieu}{rgb}{1,1,1}  % vert pale
\definecolor{couleurbas}{rgb}{0.85,0.9,1}  % blanc
\setbeamertemplate{background canvas}[vertical shading]%
[top=couleurhaut,middle=couleurmilieu,midpoint=0.4,bottom=couleurbas] 
%[top=fondtitre!05,bottom=fondtitre!60]



\makeatletter
\setbeamertemplate{theorem begin}
{%
  \begin{\inserttheoremblockenv}
  {%
    \inserttheoremheadfont
    \inserttheoremname
    \inserttheoremnumber
    \ifx\inserttheoremaddition\@empty\else\ (\inserttheoremaddition)\fi%
    \inserttheorempunctuation
  }%
}
\setbeamertemplate{theorem end}{\end{\inserttheoremblockenv}}

\newenvironment{theoreme}[1][]{%
   \setbeamercolor{block title}{fg=structure,bg=structure!40}
   \setbeamercolor{block body}{fg=black,bg=structure!10}
   \begin{block}{{\bf Th\'eor\`eme }#1}
}{%
   \end{block}%
}


\newenvironment{proposition}[1][]{%
   \setbeamercolor{block title}{fg=structure,bg=structure!40}
   \setbeamercolor{block body}{fg=black,bg=structure!10}
   \begin{block}{{\bf Proposition }#1}
}{%
   \end{block}%
}

\newenvironment{corollaire}[1][]{%
   \setbeamercolor{block title}{fg=structure,bg=structure!40}
   \setbeamercolor{block body}{fg=black,bg=structure!10}
   \begin{block}{{\bf Corollaire }#1}
}{%
   \end{block}%
}

\newenvironment{mydefinition}[1][]{%
   \setbeamercolor{block title}{fg=structure,bg=structure!40}
   \setbeamercolor{block body}{fg=black,bg=structure!10}
   \begin{block}{{\bf Définition} #1}
}{%
   \end{block}%
}

\newenvironment{lemme}[0]{%
   \setbeamercolor{block title}{fg=structure,bg=structure!40}
   \setbeamercolor{block body}{fg=black,bg=structure!10}
   \begin{block}{\bf Lemme}
}{%
   \end{block}%
}

\newenvironment{remarque}[1][]{%
   \setbeamercolor{block title}{fg=black,bg=structure!20}
   \setbeamercolor{block body}{fg=black,bg=structure!5}
   \begin{block}{Remarque #1}
}{%
   \end{block}%
}


\newenvironment{exemple}[1][]{%
   \setbeamercolor{block title}{fg=black,bg=structure!20}
   \setbeamercolor{block body}{fg=black,bg=structure!5}
   \begin{block}{{\bf Exemple }#1}
}{%
   \end{block}%
}


\newenvironment{miniexercice}[0]{%
   \setbeamercolor{block title}{fg=structure,bg=structure!20}
   \setbeamercolor{block body}{fg=black,bg=structure!5}
   \begin{block}{Mini-exercices}
}{%
   \end{block}%
}


\newenvironment{tp}[0]{%
   \setbeamercolor{block title}{fg=structure,bg=structure!40}
   \setbeamercolor{block body}{fg=black,bg=structure!10}
   \begin{block}{\bf Travaux pratiques}
}{%
   \end{block}%
}
\newenvironment{exercicecours}[1][]{%
   \setbeamercolor{block title}{fg=structure,bg=structure!40}
   \setbeamercolor{block body}{fg=black,bg=structure!10}
   \begin{block}{{\bf Exercice }#1}
}{%
   \end{block}%
}
\newenvironment{algo}[1][]{%
   \setbeamercolor{block title}{fg=structure,bg=structure!40}
   \setbeamercolor{block body}{fg=black,bg=structure!10}
   \begin{block}{{\bf Algorithme}\hfill{\color{gray}\texttt{#1}}}
}{%
   \end{block}%
}


\setbeamertemplate{proof begin}{
   \setbeamercolor{block title}{fg=black,bg=structure!20}
   \setbeamercolor{block body}{fg=black,bg=structure!5}
   \begin{block}{{\footnotesize Démonstration}}
   \footnotesize
   \smallskip}
\setbeamertemplate{proof end}{%
   \end{block}}
\setbeamertemplate{qed symbol}{\openbox}


\makeatother
\usecolortheme[RGB={191,146,10}]{structure}

%%%%%%%%%%%%%%%%%%%%%%%%%%%%%%%%%%%%%%%%%%%%%%%%%%%%%%%%%%%%%
%%%%%%%%%%%%%%%%%%%%%%%%%%%%%%%%%%%%%%%%%%%%%%%%%%%%%%%%%%%%%

\newcommand{\mathinsist}{\onslide}
\begin{document}


\title{{\bf Matrices}}
\subtitle{Inverse d'une matrice : systèmes linéaires et matrices élémentaires}

\begin{frame}
  
  \debutmontitre

  \pause

{\footnotesize
\hfill
\setbeamercovered{transparent=50}
\begin{minipage}{0.6\textwidth}
  \begin{itemize}
    \item<3-> Matrices et systèmes linéaires
    \item<4-> Matrices inversibles et systèmes linéaires
    \item<5-> Les matrices élémentaires
    \item<6-> \'Equivalence à une matrice échelonnée
    \item<7-> Matrices élémentaires et inverse d'une matrice   
  \end{itemize}
\end{minipage}
}

\end{frame}

\setcounter{framenumber}{0}


%%%%%%%%%%%%%%%%%%%%%%%%%%%%%%%%%%%%%%%%%%%%%%%%%%%%%%%%%%%%%%%%
\section{Matrices et systèmes linéaires}

\begin{frame}
Le système linéaire à $n$ lignes et $p$ inconnues
\[ \left\{ 
\begin{array}{ccccccccc}
a_{11} \  x_1 &+& a_{12}\  x_2 &+& \cdots &+& a_{1p}\  x_p & = & b_1\\
a_{21}\  x_1 &+& a_{22}\  x_2 &+& \cdots &+& a_{2p}\  x_p & = & b_2\\
&&\dots  && &&\\
a_{n1}\  x_1 &+& a_{n2}\  x_2 &+& \cdots &+& a_{np}\  x_p & = & b_n
\end{array} \right.
\]
\pause
s'écrit aussi
\begin{equation*}\begin{array}{cccc}
\underbrace{
\left(
\begin{array}{ccc}
a_{11} & \dots & a_{1p}\\
a_{21} & \dots & a_{2p}\\
\vdots &&\vdots\\
a_{n1} &\dots & a_{np}
\end{array}
\right)
}
&
\underbrace{
\left(
\begin{array}{c}
x_1\\
x_2\\
\vdots\\
x_p
\end{array}
\right)
}
& = &
\underbrace{
\left(
\begin{array}{c}
b_1\\
b_2\\
\vdots\\
b_n
\end{array}
\right)
}
\\
A & X & &B
\end{array}\end{equation*}

\pause

\begin{theoreme}
Un système d'équations linéaires n'a soit aucune solution, 
soit une seule solution, soit une infinité de solutions
\end{theoreme}

\end{frame}

%%%%%%%%%%%%%%%%%%%%%%%%%%%%%%%%%%%%%%%%%%%%%%%%%%%%%%%%%%%%%%%%
\section{Matrices inversibles et systèmes linéaires}


\begin{frame}

Cas où le nombre d'équations égale le nombre d'inconnues:

\begin{equation*}\begin{array}{cccc}
\underbrace{
\left(
\begin{array}{ccc}
a_{11} & \dots & a_{1n}\\
a_{21} & \dots & a_{2n}\\
\vdots &&\vdots\\
a_{n1} &\dots & a_{nn}
\end{array}
\right)
}
&
\underbrace{
\left(
\begin{array}{c}
x_1\\
x_2\\
\vdots\\
x_n
\end{array}
\right)
}
& = &
\underbrace{
\left(
\begin{array}{c}
b_1\\
b_2\\
\vdots\\
b_n
\end{array}
\right)
} 
\\
A & X & & B
\end{array}\end{equation*}

\pause

\begin{proposition}
Si la matrice $A$ est inversible, alors
la solution du système $AX=B$ est unique et est :
\mybox{$X = A^{-1}B$}
\end{proposition}

\pause

%Preuve : 
$AX= B \iff A^{-1}AX = A^{-1}B \iff I X = A^{-1}B\iff X = A^{-1}B$

%\iff \big(A^{-1}A\big)X = A^{-1}B  
%$X = A^{-1}B \iff AX = A\big(A^{-1}B\big) = \big(AA^{-1}\big)B = I \cdot B=B$

\end{frame}




%%%%%%%%%%%%%%%%%%%%%%%%%%%%%%%%%%%%%%%%%%%%%%%%%%%%%%%%%%%%%%%%
\section{Les matrices élémentaires}

\begin{frame}

Pour calculer l'inverse d'une matrice $A$, nous avons utilisé trois opérations
 élémentaires sur les lignes
\pause
\begin{enumerate}
  \item $L_i \leftarrow \lambda L_i$ avec $\lambda \neq 0$
   
  \item $L_i \leftarrow L_i+\lambda L_j$ avec $\lambda \in \Kk$ (et $j\neq i$)

  \item $L_i \leftrightarrow L_j$
\end{enumerate}

\bigskip
\pause

\begin{itemize}
  \item Trois matrices élémentaires 
$E_{L_i \leftarrow \lambda L_i}$, $E_{L_i \leftarrow L_i+\lambda L_j}$, 
$E_{L_i \leftrightarrow L_j}$ 

\pause

  \item Le produit $E\times A$ correspondra à l'opération élémentaire sur $A$
  
\pause  

  \item Les opérations élémentaires sur les lignes sont réversibles,
ce qui entraîne que les matrices élémentaires sont inversibles 
%   
%   \item Le résultat de la multiplication d'un matrice élémentaire $E$ par $A$ 
% est la matrice obtenue en effectuant l'opération élémentaire correspondante sur $A$. 
%   
\end{itemize}

\end{frame}


\begin{frame}
\begin{itemize}\setlength{\itemsep}{8pt}
  \item $L_i \leftarrow \lambda L_i$ avec $\lambda \neq 0$ : 
  \evidence{multiplier une ligne par un réel non nul}
\pause
  \item La matrice $E_{L_i \leftarrow \lambda L_i}$ est la matrice obtenue en 
    multipliant par $\lambda$ la $i$-ème ligne de la matrice identité $I$     
\pause
  \item  $ E_{L_2 \leftarrow 5 L_2}=
    \begin{pmatrix}
    1 & 0 & 0 & 0\\
    0 & \mathinsist{5} & 0 & 0\\
    0 & 0 & 1 & 0\\
    0 & 0 & 0 & 1\end{pmatrix}$
\pause    
  \item La matrice $E_{L_i \leftarrow \lambda L_i} \times A$ est la matrice obtenue en 
    multipliant par $\lambda$ la $i$-ème ligne de $A$
\pause  
  \item $E_{L_2 \leftarrow \frac13 L_2}  \times A
  \pause
= \begin{pmatrix}
  1&0&0\\0&\frac13&0\\0&0&1  
  \end{pmatrix}
  \times
  \begin{pmatrix}
  x_1&x_2&x_3\\y_1&y_2&y_3\\z_1&z_2&z_3  
  \end{pmatrix}
  \pause
  =   \begin{pmatrix}
  x_1&x_2&x_3\\ \frac13y_1&\frac13y_2&\frac13y_3\\z_1&z_2&z_3  
  \end{pmatrix}
$
\end{itemize}

\end{frame}



\begin{frame}
\begin{itemize}\setlength{\itemsep}{8pt}
  \item $L_i \leftarrow L_i+\lambda L_j$ avec $\lambda \in \Kk$ (et $j\neq i$) :
  \evidence{ajouter à la ligne $L_i$ un multiple d'une autre ligne $L_j$}
\pause  
  \item  La matrice $E_{L_i \leftarrow L_i+\lambda L_j}$ est la  matrice
obtenue en ajoutant $\lambda$ fois la $j$-ème ligne de $I$ à la $i$-ème ligne de $I$
\pause

  \item $  E_{L_2 \leftarrow L_2 -3 L_1}=
    \begin{pmatrix}
    1 & 0 & 0 & 0\\
    \mathinsist{-3} & 1 & 0 & 0\\
    0 & 0 & 1 & 0\\
    0 & 0 & 0 & 1
    \end{pmatrix}$
    
\pause    
  \item La matrice $E_{L_i \leftarrow L_i+\lambda L_j} \times A$ est la  matrice obtenue 
  en ajoutant $\lambda$ fois la $j$-ème ligne de $A$ à la $i$-ème ligne de $A$
\pause  
  
  \item {\small $E_{L_1 \leftarrow L_1-7 L_3}  \times A
  \pause
\arraycolsep=1.6pt
= \begin{pmatrix}
  1&0&-7\\0&1&0\\0&0&1  
  \end{pmatrix}
  \!\!\times\!\!
  \begin{pmatrix}
  x_1&x_2&x_3\\y_1&y_2&y_3\\z_1&z_2&z_3  
  \end{pmatrix}
\pause
    =   \begin{pmatrix}
  x_1\!-\!7z_1&x_2\!-\!7z_2&x_3\!-\!7z_3\\ y_1&y_2&y_3\\z_1&z_2&z_3  
  \end{pmatrix}
$
%  
%  \pause
%  \hfill\hfill$ \arraycolsep=1.4pt
%  =   \begin{pmatrix}
%  x_1\!-\!7z_1&x_2\!-\!7z_2&x_3\!-\!7z_3\\ y_1&y_2&y_3\\z_1&z_2&z_3  
%  \end{pmatrix}
%$
}
\end{itemize}

 
\end{frame}


\begin{frame}
\begin{itemize}\setlength{\itemsep}{8pt}
  \item $L_i \leftrightarrow L_j$ : \evidence{échanger deux lignes}
\pause 
  \item La matrice  $E_{L_i \leftrightarrow L_j}$ est la matrice obtenue 
   en permutant les $i$-ème et $j$-ème lignes de $I$
\pause
  \item $ E_{L_2 \leftrightarrow L_4} = E_{L_4 \leftrightarrow L_2} = 
   \begin{pmatrix}
    1 & 0 & 0 & 0\\
    0 & 0 & 0 & \mathinsist{1}\\
    0 & 0 & 1 & 0\\
    0 & \mathinsist{1} & 0 & 0
    \end{pmatrix}$  
\pause
  \item La matrice  $E_{L_i \leftrightarrow L_j} \times A$ est la matrice obtenue 
   en permutant les $i$-ème et $j$-ème lignes de $A$
\pause   
   
  \item $E_{L_2 \leftrightarrow L_3}  \times A
  \pause
= \begin{pmatrix}
  1&0&0\\0&0&1\\0&1&0  
  \end{pmatrix}
  \times
  \begin{pmatrix}
  x_1&x_2&x_3\\y_1&y_2&y_3\\z_1&z_2&z_3  
  \end{pmatrix}
  \pause
  =   \begin{pmatrix}
  x_1&x_2&x_3\\z_1&z_2&z_3\\ y_1&y_2&y_3  
  \end{pmatrix}
$ 
\end{itemize}

\end{frame}


%%%%%%%%%%%%%%%%%%%%%%%%%%%%%%%%%%%%%%%%%%%%%%%%%%%%%%%%%%%%%%%%
\section{\'Equivalence à une matrice échelonnée}

\begin{frame}
\vspace*{-1ex}
\begin{mydefinition} 
Deux matrices $A$ et $B$ sont dites \defi{équivalentes par lignes} si l'une 
peut être obtenue à partir de l'autre par une suite d'opérations 
élémentaires sur les lignes. On note $A \sim B$
\end{mydefinition} 
\pause\vspace*{-1ex}
\begin{mydefinition}
Une matrice est \defi{échelonnée} si :
\begin{itemize}
  \item le nombre de zéros commençant une ligne croît strictement ligne par ligne 
  (et s'arrête s'il n'y a plus que des zéros)
\end{itemize}  
\uncover<4->{
Elle est \defi{échelonnée réduite} si en plus : 
\begin{itemize}
\setcounter{enumi}{1} 
  \item le premier coefficient non nul d'une ligne vaut $1$

  \item et c'est le seul élément non nul de sa colonne
\end{itemize}
}
\end{mydefinition}
\vspace*{-2ex}
{\footnotesize
$$\uncover<3->{
\begin{pmatrix}
+ & * & * & * & * & * & * \\
0 & 0 & + & * & * & * & * \\
0 & 0 & 0 & + & * & * & * \\
0 & 0 & 0 & 0 & 0 & 0 & + \\
0 & 0 & 0 & 0 & 0 & 0 & 0 \\
0 & 0 & 0 & 0 & 0 & 0 & 0 \\
\end{pmatrix}
}
\qquad \qquad
\uncover<5->{
\begin{pmatrix}
1 & * & 0 & 0 & * & * & 0 \\
0 & 0 & 1 & 0 & * & * & 0 \\
0 & 0 & 0 & 1 & * & * & 0 \\
0 & 0 & 0 & 0 & 0 & 0 & 1 \\
0 & 0 & 0 & 0 & 0 & 0 & 0 \\
0 & 0 & 0 & 0 & 0 & 0 & 0 \\
\end{pmatrix}
}
$$
}
\end{frame}


\begin{frame}
\begin{theoreme}\label{Gauss} 
\'Etant donnée une matrice $A\in M_{n,p}(\Kk)$, il existe une unique  
matrice échelonnée réduite $U$ obtenue à partir de $A$ 
par des opérations élémentaires sur les lignes 
\end{theoreme}


% L'existence se démontre grâce à 
% l'algorithme de Gauss. L'idée générale consiste à utiliser des substitutions de
% lignes pour placer des zéros là où il faut de façon à 
% créer d'abord une forme échelonnée, puis une forme
% échelonnée réduite. 

\end{frame}


%%%%%%%%%%%%%%%%%%%%%%%%%%%%%%%%%%%%%%%%%%%%%%%%%%%%%%%%%%%%%%%%
\section{Matrices élémentaires et inverse d'une matrice}


\begin{frame}

Soit $A\in M_{n}(\Kk)$
\begin{theoreme} 
\label{th:invequi}
La matrice $A$ est inversible si et seulement si sa forme échelonnée réduite 
est la matrice identité $I$
\end{theoreme}

\pause

\begin{corollaire}
Les assertions suivantes sont équivalentes :
\begin{itemize}
  \item[(i)] La matrice $A$ est inversible
  
  \item[(ii)] Le système linéaire $AX=\left(\begin{smallmatrix} 0 \\ \vdots \\ 0\end{smallmatrix}\right)$ a une unique solution 
  $X=\left(\begin{smallmatrix} 0 \\ \vdots \\ 0\end{smallmatrix}\right)$
             
  \item[(iii)] Pour tout second membre $B$, le système linéaire $AX=B$
  a une unique solution $X$
\end{itemize}
\end{corollaire}

\end{frame}


%%%%%%%%%%%%%%%%%%%%%%%%%%%%%%%%%%%%%%%%%%%%%%%%%%%%%%%%%%%%%%%%
\section{\'Equivalence à une matrice échelonnée}

\begin{frame}
\begin{theoreme}
\'Etant donnée une matrice $A \in M_{n,p}(\Kk)$, il existe une unique  
matrice échelonnée réduite $U$ obtenue à partir de $A$ 
par des opérations élémentaires sur les lignes 
\end{theoreme}
\end{frame}



\begin{frame}

\textbf{Partie A. Passage à une forme échelonnée}

\pause
\textbf{\'Etape A.1.}  \evidence{Choix du pivot}

\pause
\begin{itemize}
  \item Soit la première colonne ne contient que des zéros $\mapsto$ A.3
  \pause
  \item Un terme $a_{i1} \neq 0$ est un \defi{pivot}
%   
%   \item Si c'est le terme $a_{11}$, on passe directement à l'étape A.2
  \pause
  \item On échange les lignes $1$ 
et $i$ ($L_1 \leftrightarrow L_i$) $\mapsto$ A.2
\end{itemize}

\pause
\vspace*{-2ex}
{\small$$
\arraycolsep=1.4mm
%\hspace*{-2em}  
A\sim \begin{pmatrix}
\mathinsist{a'_{11}}&a'_{12}&\cdots&a'_{1j}&\cdots&a'_{1p}\\
a'_{21}&a'_{22}&\cdots&a'_{2j}&\cdots&a'_{2p}\\
\vdots&\vdots&&\vdots&&\vdots\\
a'_{i1}&a'_{i2}&\cdots&a'_{ij}&\cdots&a'_{ip}\\
\vdots&\vdots&&\vdots&&\vdots\\
a'_{n1}&a'_{n2}&\cdots&a'_{nj}&\cdots&a'_{np}\\
\end{pmatrix}
\quad
\uncover<11->{
A\sim\begin{pmatrix}
\mathinsist{a'_{11}}&a'_{12}&\cdots&a'_{1j}&\cdots&a'_{1p}\\
0&a''_{22}&\cdots&a''_{2j}&\cdots&a''_{2p}\\
\vdots&\vdots&&\vdots&&\vdots\\
0&a''_{i2}&\cdots&a''_{ij}&\cdots&a''_{ip}\\
\vdots&\vdots&&\vdots&&\vdots\\
0&a''_{n2}&\cdots&a''_{nj}&\cdots&a''_{np}\\
\end{pmatrix}}
$$}

\pause
\textbf{\'Etape A.2.}  \evidence{\'Elimination}

\pause
\begin{itemize}
  \item On ne touche plus à la ligne $1$
  \pause
  \item On élimine tous les termes $a'_{i1}$ ($i\ge 2$) sous le pivot
  \pause
  \item $L_2 \leftarrow L_2 - \frac{a'_{21}}{a'_{11}}L_1$ \, ,\quad 
$L_3 \leftarrow L_3 - \frac{a'_{31}}{a'_{11}}L_1$ \ \ \ldots 
\end{itemize}

\end{frame}



\begin{frame}

\textbf{\'Etape A.3.}  \evidence{Boucle}

\pause
{\small
$$
%\hspace*{-2em}
\arraycolsep=1.4mm
A \sim \begin{pmatrix}
\mathinsist{a^1_{11}}&a^1_{12}&\cdots&a^1_{1j}&\cdots&a^1_{1p}\\
0&a^1_{22}&\cdots&a^1_{2j}&\cdots&a^1_{2p}\\
\vdots&\vdots&&\vdots&&\vdots\\
0&a^1_{i2}&\cdots&a^1_{ij}&\cdots&a^1_{ip}\\
\vdots&\vdots&&\vdots&&\vdots\\
0&a^1_{n2}&\cdots&a^1_{nj}&\cdots&a^1_{np}\\
\end{pmatrix}
\quad 
\uncover<6->{
A \sim 
\begin{pmatrix}
a^1_{11}&a^1_{12}&\cdots&a^1_{1j}&\cdots&a^1_{1p}\\
0&\mathinsist{a^2_{22}}&\cdots&a^2_{2j}&\cdots&a^2_{2p}\\
\vdots&\vdots&&\vdots&&\vdots\\
0&0&\cdots&a^2_{ij}&\cdots&a^2_{ip}\\
\vdots&\vdots&&\vdots&&\vdots\\
0&0&\cdots&a^2_{nj}&\cdots&a^2_{np}\\
\end{pmatrix}}
$$}

\pause
\begin{itemize}
  \item La première colonne est celle d'une matrice échelonnée
  \pause
  \item Si $a^1_{11}\neq0$ on conserve aussi la première ligne, $\mapsto$ A.1
  appliquée à la sous-matrice $(n-1)\times(p-1)$
  \pause
  \item Si $a^1_{11}=0$, $\mapsto$ A.1 à la sous-matrice 
$n\times(p-1)$
  \pause\pause
  \item Chaque itération de la boucle s'applique à une matrice qui a 
une colonne de moins que la précédente
  \pause 
  \item Au bout d'au 
plus $p-1$ itérations de la boucle, on aura obtenu une 
matrice échelonnée
\end{itemize}
\end{frame}



\begin{frame}
\textbf{Partie B. Passage à une forme échelonnée réduite}  

\pause
\medskip

\textbf{\'Etape B.1.}  \evidence{Homothéties}

\pause
\begin{itemize}
  \item On repère le premier élément non nul de chaque ligne non nulle
  \pause
  \item On multiplie cette ligne par l'inverse de cet élément
  \pause
  \item Exemple : si le premier élément non nul de la ligne $i$ est $\alpha\neq0$, alors
on effectue $L_i \leftarrow \frac1\alpha L_i$
  \pause
  \item Ceci crée une matrice échelonnée avec des $1$ en position de pivot
  
\end{itemize}

\pause
\medskip
\textbf{\'Etape B.2.}  \evidence{\'Elimination}

\pause
\begin{itemize}
  \item On élimine les termes situés au-dessus des positions de pivot 
en procédant à partir du bas à droite de la matrice
  \pause
  \item Ceci crée une matrice échelonnée et réduite
\end{itemize}

\end{frame}

%--------------------------------------------------------------

\begin{frame}
\evidence{Exemple}
\vspace*{-3ex}
$$A=\begin{pmatrix}
\mathinsist<3->{1}&2&3&4\\
0&2&4&6\\
-1&0&1&0
\end{pmatrix}$$

\pause
\textbf{A. Passage à une forme échelonnée}

\pause

\begin{itemize}[<+->]
  \item \'Etape A.1. Choix du pivot $a_{11}^1=1$
  
  \item \'Etape A.2. $L_3 \leftarrow L_3 + L_1$
  \hfil
$A\sim\begin{pmatrix}1&2&3&4\\
0& \mathinsist<5->{2} &4&6\\
0&2&4&4
\end{pmatrix}$

  \item \'Etape A.1. Choix du pivot $a^2_{22}=2$
  
  \item \'Etape A.2. $L_3 \leftarrow L_3 - L_2$
\hfil   $A\sim\begin{pmatrix}1&2&3&4\\
0&2&4&6\\
0&0&0&-2
\end{pmatrix}$

  \item Cette matrice est échelonnée
\end{itemize}

\end{frame}


\begin{frame}
\evidence{Exemple}
\vspace*{-3ex}
$$A\sim\begin{pmatrix}1&2&3&4\\
0&2&4&6\\
0&0&0&-2
\end{pmatrix}$$

\pause

\textbf{B. Passage à une forme échelonnée réduite}

\pause

\begin{itemize}[<+->]
  \item \'Etape B.1. $L_2 \leftarrow \frac12 L_2$ et $L_3 \leftarrow -\frac12 L_3$ 
    \hfil
$A\sim\begin{pmatrix}1&2&3&4\\
0&1&2&3\\
0&0&0&1
\end{pmatrix}$  
  
  \item \'Etape B.2. $L_2 \leftarrow L_2-3L_3$ et $L_1 \leftarrow L_1 - 4L_3$
  \hfill\hfill
$A\sim\begin{pmatrix}1&2&3&0\\
0&1&2&0\\
0&0&0&1
\end{pmatrix}$ 

  \item \'Etape B.2. $L_1 \leftarrow L_1-2L_2$
  \hfil
  $A\sim\begin{pmatrix}1&0&-1&0\\
0&1&2&0\\
0&0&0&1
\end{pmatrix}$

  \item Cette matrice est échelonnée et réduite
\end{itemize}
\end{frame}








%%%%%%%%%%%%%%%%%%%%%%%%%%%%%%%%%%%%%%%%%%%%%%%%%%%%%%%%%%%%%%%%
\section{Mini-exercices}

\begin{frame}

\begin{miniexercice}
\begin{enumerate}
  \item Exprimer les systèmes linéaires suivants sous forme matricielle 
  et les résoudre en inversant la matrice :
  $\left\{\begin{array}{l}2x+4y=7\\-2x+3y=-14\end{array} \right.$,\quad 
  $\left\{\begin{array}{l}x+z=1\\-2y+3z=1\\x+z=1\end{array} \right.$,\quad 
  $\left\{\begin{array}{l}x+t=\alpha\\x-2y=\beta\\x+y+t=2\\y+t=4\end{array} \right.$.
  
  
  \item \'Ecrire les matrices $4\times 4$ correspondant aux opérations élémentaires :
  $L_2 \leftarrow \frac13 L_2$, $L_3 \leftarrow L_3-\frac14 L_2$, $L_1 \leftrightarrow L_4$.
  Sans calculs, écrire leurs inverses. \'Ecrire la matrice $4\times 4$ 
  de l'opération $L_1 \leftarrow L_1-2L_3+3L_4$.
  
  \item \'Ecrire les matrices suivantes sous forme échelonnée, puis échelonnée réduite :
  $\left(\begin{smallmatrix}1&2&3\\1&4&0\\-2&-2&-3\end{smallmatrix}\right)$, 
  $\left(\begin{smallmatrix}1&0&2\\1&-1&1\\2&-2&3\end{smallmatrix}\right)$, 
  $\left(\begin{smallmatrix}2&0&-2&0\\0&-1&1&0\\1&-2&1&4\\-1&2&-1&-2\end{smallmatrix}\right)$.

\end{enumerate}
\end{miniexercice}

\end{frame}

\end{document}