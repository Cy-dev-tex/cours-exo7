
%%%%%%%%%%%%%%%%%% PREAMBULE %%%%%%%%%%%%%%%%%%

\documentclass[aspectratio=169,utf8]{beamer}
%\documentclass[aspectratio=169,handout]{beamer}

\usetheme{Boadilla}
%\usecolortheme{seahorse}
\usecolortheme[RGB={245,66,24}]{structure}
\useoutertheme{infolines}

% packages
\usepackage{amsfonts,amsmath,amssymb,amsthm}
\usepackage[utf8]{inputenc}
\usepackage[T1]{fontenc}
\usepackage{lmodern}

\usepackage[francais]{babel}
\usepackage{fancybox}
\usepackage{graphicx}

\usepackage{float}
\usepackage{xfrac}

%\usepackage[usenames, x11names]{xcolor}
\usepackage{tikz}
\usepackage{pgfplots}
\usepackage{datetime}



%-----  Package unités -----
\usepackage{siunitx}
\sisetup{locale = FR,detect-all,per-mode = symbol}

%\usepackage{mathptmx}
%\usepackage{fouriernc}
%\usepackage{newcent}
%\usepackage[mathcal,mathbf]{euler}

%\usepackage{palatino}
%\usepackage{newcent}
% \usepackage[mathcal,mathbf]{euler}



% \usepackage{hyperref}
% \hypersetup{colorlinks=true, linkcolor=blue, urlcolor=blue,
% pdftitle={Exo7 - Exercices de mathématiques}, pdfauthor={Exo7}}


%section
% \usepackage{sectsty}
% \allsectionsfont{\bf}
%\sectionfont{\color{Tomato3}\upshape\selectfont}
%\subsectionfont{\color{Tomato4}\upshape\selectfont}

%----- Ensembles : entiers, reels, complexes -----
\newcommand{\Nn}{\mathbb{N}} \newcommand{\N}{\mathbb{N}}
\newcommand{\Zz}{\mathbb{Z}} \newcommand{\Z}{\mathbb{Z}}
\newcommand{\Qq}{\mathbb{Q}} \newcommand{\Q}{\mathbb{Q}}
\newcommand{\Rr}{\mathbb{R}} \newcommand{\R}{\mathbb{R}}
\newcommand{\Cc}{\mathbb{C}} 
\newcommand{\Kk}{\mathbb{K}} \newcommand{\K}{\mathbb{K}}

%----- Modifications de symboles -----
\renewcommand{\epsilon}{\varepsilon}
\renewcommand{\Re}{\mathop{\text{Re}}\nolimits}
\renewcommand{\Im}{\mathop{\text{Im}}\nolimits}
%\newcommand{\llbracket}{\left[\kern-0.15em\left[}
%\newcommand{\rrbracket}{\right]\kern-0.15em\right]}

\renewcommand{\ge}{\geqslant}
\renewcommand{\geq}{\geqslant}
\renewcommand{\le}{\leqslant}
\renewcommand{\leq}{\leqslant}
\renewcommand{\epsilon}{\varepsilon}

%----- Fonctions usuelles -----
\newcommand{\ch}{\mathop{\text{ch}}\nolimits}
\newcommand{\sh}{\mathop{\text{sh}}\nolimits}
\renewcommand{\tanh}{\mathop{\text{th}}\nolimits}
\newcommand{\cotan}{\mathop{\text{cotan}}\nolimits}
\newcommand{\Arcsin}{\mathop{\text{arcsin}}\nolimits}
\newcommand{\Arccos}{\mathop{\text{arccos}}\nolimits}
\newcommand{\Arctan}{\mathop{\text{arctan}}\nolimits}
\newcommand{\Argsh}{\mathop{\text{argsh}}\nolimits}
\newcommand{\Argch}{\mathop{\text{argch}}\nolimits}
\newcommand{\Argth}{\mathop{\text{argth}}\nolimits}
\newcommand{\pgcd}{\mathop{\text{pgcd}}\nolimits} 


%----- Commandes divers ------
\newcommand{\ii}{\mathrm{i}}
\newcommand{\dd}{\text{d}}
\newcommand{\id}{\mathop{\text{id}}\nolimits}
\newcommand{\Ker}{\mathop{\text{Ker}}\nolimits}
\newcommand{\Card}{\mathop{\text{Card}}\nolimits}
\newcommand{\Vect}{\mathop{\text{Vect}}\nolimits}
\newcommand{\Mat}{\mathop{\text{Mat}}\nolimits}
\newcommand{\rg}{\mathop{\text{rg}}\nolimits}
\newcommand{\tr}{\mathop{\text{tr}}\nolimits}


%----- Structure des exercices ------

\newtheoremstyle{styleexo}% name
{2ex}% Space above
{3ex}% Space below
{}% Body font
{}% Indent amount 1
{\bfseries} % Theorem head font
{}% Punctuation after theorem head
{\newline}% Space after theorem head 2
{}% Theorem head spec (can be left empty, meaning ‘normal’)

%\theoremstyle{styleexo}
\newtheorem{exo}{Exercice}
\newtheorem{ind}{Indications}
\newtheorem{cor}{Correction}


\newcommand{\exercice}[1]{} \newcommand{\finexercice}{}
%\newcommand{\exercice}[1]{{\tiny\texttt{#1}}\vspace{-2ex}} % pour afficher le numero absolu, l'auteur...
\newcommand{\enonce}{\begin{exo}} \newcommand{\finenonce}{\end{exo}}
\newcommand{\indication}{\begin{ind}} \newcommand{\finindication}{\end{ind}}
\newcommand{\correction}{\begin{cor}} \newcommand{\fincorrection}{\end{cor}}

\newcommand{\noindication}{\stepcounter{ind}}
\newcommand{\nocorrection}{\stepcounter{cor}}

\newcommand{\fiche}[1]{} \newcommand{\finfiche}{}
\newcommand{\titre}[1]{\centerline{\large \bf #1}}
\newcommand{\addcommand}[1]{}
\newcommand{\video}[1]{}

% Marge
\newcommand{\mymargin}[1]{\marginpar{{\small #1}}}

\def\noqed{\renewcommand{\qedsymbol}{}}


%----- Presentation ------
\setlength{\parindent}{0cm}

%\newcommand{\ExoSept}{\href{http://exo7.emath.fr}{\textbf{\textsf{Exo7}}}}

\definecolor{myred}{rgb}{0.93,0.26,0}
\definecolor{myorange}{rgb}{0.97,0.58,0}
\definecolor{myyellow}{rgb}{1,0.86,0}

\newcommand{\LogoExoSept}[1]{  % input : echelle
{\usefont{U}{cmss}{bx}{n}
\begin{tikzpicture}[scale=0.1*#1,transform shape]
  \fill[color=myorange] (0,0)--(4,0)--(4,-4)--(0,-4)--cycle;
  \fill[color=myred] (0,0)--(0,3)--(-3,3)--(-3,0)--cycle;
  \fill[color=myyellow] (4,0)--(7,4)--(3,7)--(0,3)--cycle;
  \node[scale=5] at (3.5,3.5) {Exo7};
\end{tikzpicture}}
}


\newcommand{\debutmontitre}{
  \author{} \date{} 
  \thispagestyle{empty}
  \hspace*{-10ex}
  \begin{minipage}{\textwidth}
    \titlepage  
  \vspace*{-2.5cm}
  \begin{center}
    \LogoExoSept{2.5}
  \end{center}
  \end{minipage}

  \vspace*{-0cm}
  
  % Astuce pour que le background ne soit pas discrétisé lors de la conversion pdf -> png
\begin{tikzpicture}
        \fill[opacity=0,green!60!black] (0,0)--++(0,0)--++(0,0)--++(0,0)--cycle; 
\end{tikzpicture}

% toc S'affiche trop tot :
% \tableofcontents[hideallsubsections, pausesections]
}

\newcommand{\finmontitre}{
  \end{frame}
  \setcounter{framenumber}{0}
} % ne marche pas pour une raison obscure

%----- Commandes supplementaires ------

% \usepackage[landscape]{geometry}
% \geometry{top=1cm, bottom=3cm, left=2cm, right=10cm, marginparsep=1cm
% }
% \usepackage[a4paper]{geometry}
% \geometry{top=2cm, bottom=2cm, left=2cm, right=2cm, marginparsep=1cm
% }

%\usepackage{standalone}


% New command Arnaud -- november 2011
\setbeamersize{text margin left=24ex}
% si vous modifier cette valeur il faut aussi
% modifier le decalage du titre pour compenser
% (ex : ici =+10ex, titre =-5ex

\theoremstyle{definition}
%\newtheorem{proposition}{Proposition}
%\newtheorem{exemple}{Exemple}
%\newtheorem{theoreme}{Théorème}
%\newtheorem{lemme}{Lemme}
%\newtheorem{corollaire}{Corollaire}
%\newtheorem*{remarque*}{Remarque}
%\newtheorem*{miniexercice}{Mini-exercices}
%\newtheorem{definition}{Définition}

% Commande tikz
\usetikzlibrary{calc}
\usetikzlibrary{patterns,arrows}
\usetikzlibrary{matrix}
\usetikzlibrary{fadings} 

%definition d'un terme
\newcommand{\defi}[1]{{\color{myorange}\textbf{\emph{#1}}}}
\newcommand{\evidence}[1]{{\color{blue}\textbf{\emph{#1}}}}
\newcommand{\assertion}[1]{\emph{\og#1\fg}}  % pour chapitre logique
%\renewcommand{\contentsname}{Sommaire}
\renewcommand{\contentsname}{}
\setcounter{tocdepth}{2}



%------ Figures ------

\def\myscale{1} % par défaut 
\newcommand{\myfigure}[2]{  % entrée : echelle, fichier figure
\def\myscale{#1}
\begin{center}
\footnotesize
{#2}
\end{center}}


%------ Encadrement ------

\usepackage{fancybox}


\newcommand{\mybox}[1]{
\setlength{\fboxsep}{7pt}
\begin{center}
\shadowbox{#1}
\end{center}}

\newcommand{\myboxinline}[1]{
\setlength{\fboxsep}{5pt}
\raisebox{-10pt}{
\shadowbox{#1}
}
}

%--------------- Commande beamer---------------
\newcommand{\beameronly}[1]{#1} % permet de mettre des pause dans beamer pas dans poly


\setbeamertemplate{navigation symbols}{}
\setbeamertemplate{footline}  % tiré du fichier beamerouterinfolines.sty
{
  \leavevmode%
  \hbox{%
  \begin{beamercolorbox}[wd=.333333\paperwidth,ht=2.25ex,dp=1ex,center]{author in head/foot}%
    % \usebeamerfont{author in head/foot}\insertshortauthor%~~(\insertshortinstitute)
    \usebeamerfont{section in head/foot}{\bf\insertshorttitle}
  \end{beamercolorbox}%
  \begin{beamercolorbox}[wd=.333333\paperwidth,ht=2.25ex,dp=1ex,center]{title in head/foot}%
    \usebeamerfont{section in head/foot}{\bf\insertsectionhead}
  \end{beamercolorbox}%
  \begin{beamercolorbox}[wd=.333333\paperwidth,ht=2.25ex,dp=1ex,right]{date in head/foot}%
    % \usebeamerfont{date in head/foot}\insertshortdate{}\hspace*{2em}
    \insertframenumber{} / \inserttotalframenumber\hspace*{2ex} 
  \end{beamercolorbox}}%
  \vskip0pt%
}


\definecolor{mygrey}{rgb}{0.5,0.5,0.5}
\setlength{\parindent}{0cm}
%\DeclareTextFontCommand{\helvetica}{\fontfamily{phv}\selectfont}

% background beamer
\definecolor{couleurhaut}{rgb}{0.85,0.9,1}  % creme
\definecolor{couleurmilieu}{rgb}{1,1,1}  % vert pale
\definecolor{couleurbas}{rgb}{0.85,0.9,1}  % blanc
\setbeamertemplate{background canvas}[vertical shading]%
[top=couleurhaut,middle=couleurmilieu,midpoint=0.4,bottom=couleurbas] 
%[top=fondtitre!05,bottom=fondtitre!60]



\makeatletter
\setbeamertemplate{theorem begin}
{%
  \begin{\inserttheoremblockenv}
  {%
    \inserttheoremheadfont
    \inserttheoremname
    \inserttheoremnumber
    \ifx\inserttheoremaddition\@empty\else\ (\inserttheoremaddition)\fi%
    \inserttheorempunctuation
  }%
}
\setbeamertemplate{theorem end}{\end{\inserttheoremblockenv}}

\newenvironment{theoreme}[1][]{%
   \setbeamercolor{block title}{fg=structure,bg=structure!40}
   \setbeamercolor{block body}{fg=black,bg=structure!10}
   \begin{block}{{\bf Th\'eor\`eme }#1}
}{%
   \end{block}%
}


\newenvironment{proposition}[1][]{%
   \setbeamercolor{block title}{fg=structure,bg=structure!40}
   \setbeamercolor{block body}{fg=black,bg=structure!10}
   \begin{block}{{\bf Proposition }#1}
}{%
   \end{block}%
}

\newenvironment{corollaire}[1][]{%
   \setbeamercolor{block title}{fg=structure,bg=structure!40}
   \setbeamercolor{block body}{fg=black,bg=structure!10}
   \begin{block}{{\bf Corollaire }#1}
}{%
   \end{block}%
}

\newenvironment{mydefinition}[1][]{%
   \setbeamercolor{block title}{fg=structure,bg=structure!40}
   \setbeamercolor{block body}{fg=black,bg=structure!10}
   \begin{block}{{\bf Définition} #1}
}{%
   \end{block}%
}

\newenvironment{lemme}[0]{%
   \setbeamercolor{block title}{fg=structure,bg=structure!40}
   \setbeamercolor{block body}{fg=black,bg=structure!10}
   \begin{block}{\bf Lemme}
}{%
   \end{block}%
}

\newenvironment{remarque}[1][]{%
   \setbeamercolor{block title}{fg=black,bg=structure!20}
   \setbeamercolor{block body}{fg=black,bg=structure!5}
   \begin{block}{Remarque #1}
}{%
   \end{block}%
}


\newenvironment{exemple}[1][]{%
   \setbeamercolor{block title}{fg=black,bg=structure!20}
   \setbeamercolor{block body}{fg=black,bg=structure!5}
   \begin{block}{{\bf Exemple }#1}
}{%
   \end{block}%
}


\newenvironment{miniexercice}[0]{%
   \setbeamercolor{block title}{fg=structure,bg=structure!20}
   \setbeamercolor{block body}{fg=black,bg=structure!5}
   \begin{block}{Mini-exercices}
}{%
   \end{block}%
}


\newenvironment{tp}[0]{%
   \setbeamercolor{block title}{fg=structure,bg=structure!40}
   \setbeamercolor{block body}{fg=black,bg=structure!10}
   \begin{block}{\bf Travaux pratiques}
}{%
   \end{block}%
}
\newenvironment{exercicecours}[1][]{%
   \setbeamercolor{block title}{fg=structure,bg=structure!40}
   \setbeamercolor{block body}{fg=black,bg=structure!10}
   \begin{block}{{\bf Exercice }#1}
}{%
   \end{block}%
}
\newenvironment{algo}[1][]{%
   \setbeamercolor{block title}{fg=structure,bg=structure!40}
   \setbeamercolor{block body}{fg=black,bg=structure!10}
   \begin{block}{{\bf Algorithme}\hfill{\color{gray}\texttt{#1}}}
}{%
   \end{block}%
}


\setbeamertemplate{proof begin}{
   \setbeamercolor{block title}{fg=black,bg=structure!20}
   \setbeamercolor{block body}{fg=black,bg=structure!5}
   \begin{block}{{\footnotesize Démonstration}}
   \footnotesize
   \smallskip}
\setbeamertemplate{proof end}{%
   \end{block}}
\setbeamertemplate{qed symbol}{\openbox}


\makeatother
\usecolortheme[RGB={191,146,10}]{structure}

% Commande spécifique à ce chapitre
\usetikzlibrary{fit}					% fitting shapes to coordinates
\usetikzlibrary{backgrounds}	% drawing the background after the foreground
\usetikzlibrary{matrix}

%%%%%%%%%%%%%%%%%%%%%%%%%%%%%%%%%%%%%%%%%%%%%%%%%%%%%%%%%%%%%
%%%%%%%%%%%%%%%%%%%%%%%%%%%%%%%%%%%%%%%%%%%%%%%%%%%%%%%%%%%%%

\begin{document}


\title{{\bf Matrices}}
\subtitle{Matrices triangulaires, transposition, trace, matrices symétriques}

\begin{frame}
  
  \debutmontitre

  \pause

{\footnotesize
\hfill
\setbeamercovered{transparent=50}
\begin{minipage}{0.6\textwidth}
  \begin{itemize}
    \item<3-> Matrices triangulaires, matrices diagonales
    \item<4-> La transposition
    \item<5-> La trace
    \item<6-> Matrices symétriques
    \item<7-> Matrices antisymétriques  
  \end{itemize}
\end{minipage}
}

\end{frame}

\setcounter{framenumber}{0}


%%%%%%%%%%%%%%%%%%%%%%%%%%%%%%%%%%%%%%%%%%%%%%%%%%%%%%%%%%%%%%%%

\section{Matrices triangulaires, matrices diagonales}

%\begin{frame}
%Soit $A$ une matrice carrée de taille $n \times n$
%\begin{itemize}
%\item $A$ est \defi{triangulaire inférieure}  
%si ses éléments \emph{au-dessus} de la diagonale sont nuls : $i < j \  \Longrightarrow \ a_{ij} = 0$.
%\item $A$ est \defi{triangulaire supérieure}  
%si ses éléments \emph{en-dessous} de la diagonale sont nuls : $i > j \  \Longrightarrow \ a_{ij} = 0$.
%\end{itemize}
%%
%Ces matrices triangulaires ont la forme suivante:
%\[
%\renewcommand*{\arraystretch}{1.5}
%\begin{array}{c}
%\begin{pmatrix}
%a_{11} & 0 &\cdots&\cdots& 0\\
%a_{21}&a_{22}&\ddots&&\vdots\\
%\vdots&\vdots&\ddots&\ddots&\vdots\\
%\vdots & \vdots &&\ddots&0\\
%a_{n1}&a_{n2}&\cdots&\cdots&a_{nn}  
%\end{pmatrix}
%\\
%\text{triangulaire inférieure}
%\end{array}
%\qquad
%\begin{array}{c} \begin{pmatrix}
%a_{11} & a_{12} &\dots&\dots & a_{1n}\\
%0&a_{22}&\dots&\dots &a_{2n}\\
%\vdots&\ddots&\ddots&&\vdots\\
%\vdots&&\ddots&\ddots&\vdots\\
%0&\dots&\dots&0&a_{nn}    
%\end{pmatrix}
%\\
%\text{triangulaire supérieure}
%\end{array}
%\]
%
%\end{frame}

%--------------------------------------------------------------

\begin{frame}
Soit $A$ une matrice carrée de taille $n \times n$
\pause %
\begin{itemize}
\item $A$ est \defi{triangulaire inférieure}  
si ses éléments \emph{au-dessus} de la diagonale sont nuls : $i < j \  \Longrightarrow \ a_{ij} = 0$

\uncover<4->{\item $A$ est \defi{triangulaire supérieure}  
si ses éléments \emph{en-dessous} de la diagonale sont nuls : $i > j \  \Longrightarrow \ a_{ij} = 0$
}
\end{itemize}


\begin{figure}[b]
\centering
\tikzstyle{background}=[rectangle,
		fill=gray!20,
		inner sep=0.2mm,
		rounded corners=3mm]
\begin{tikzpicture}[scale=1]
\uncover<3->{
  \matrix [ampersand replacement=\&,left delimiter=(,right delimiter=)] {
	\node (11) {$a_{11}$} ; \& \node (12) {$0$}; \& \& \& \node (1n) {$0$}; \\
	\node (21) {$a_{21}$} ; \& \node (22) {$a_{22}$} ; \& \& \& \node {\phantom{$\vdots$}} ; \\
	\& \& \node {\phantom{$\ddots$}} ; \& \\
	\& \& \& \node {\phantom{$\ddots$}} ; \& \node (n-1n) {$0$} ; \\
	\node (n1) {$a_{n1}$} ; \& \node (n2) {$a_{n2}$} ; \& \& \&  \node (nn) {$a_{nn}$} ;  \\
    };
	\draw [thick,loosely dotted] (12) -- (1n) -- (n-1n) -- (12) 
		(21) -- (n1) (22) -- (n2) -- (nn) -- (22);
	\node at ($0.5*(n1)+0.5*(nn)+(0,-0.8)$) {Triangulaire inférieure} ;
	\only<3->{
	\begin{pgfonlayer}{background}
		\fill [fill=gray!20, rounded corners=5mm] 
			($(11)+(-0.4,0.8)$) -- ($(n1)+(-0.4,-0.4)$)	-- ($(nn)+(0.8,-0.4)$) -- cycle ;
	\end{pgfonlayer}
	}
}
\uncover<5->{
\matrix [xshift=6cm,ampersand replacement=\&,left delimiter=(,right delimiter=)] {
	\node (s11) {$a_{11}$} ; \& \node (s12) {$a_{12}$}; \& \& \& \node (s1n) {$a_{1n}$}; \\
		\node (s21) {$0$} ; \& \node (s22) {$a_{22}$} ; \& \node {\phantom{$\ddots$}} ; \& \& \node (s2n) {$a_{2n}$} ; \\
	\& \& \node {\phantom{$\ddots$}} ; \& \& \\
	\& \& \& \node {\phantom{$\ddots$}} ; \& \\
	\node (sn1) {$0$} ; \& \& \& \node (snn-1) {$0$} ; \&  \node (snn) {$a_{nn}$} ;  \\
    };
	\draw [thick,loosely dotted] (s21) -- (sn1) -- (snn-1) -- (s21) 
		(s12) -- (s1n) (s22) -- (s2n) -- (snn) -- (s22);
	\node at ($0.5*(sn1)+0.5*(snn)+(0,-0.8)$) {Triangulaire supérieure} ;
	\only<5->{
	\begin{pgfonlayer}{background}
		\fill [fill=gray!20, rounded corners=5mm] 
			($(s11)+(-0.8,0.4)$) -- ($(snn)+(0.4,-0.7)$)	-- ($(s1n)+(0.4,0.4)$) -- cycle ;
	\end{pgfonlayer}
	}
}	
	
\end{tikzpicture}
\end{figure}

\end{frame}


%--------------------------------------------------------------

\begin{frame}
\begin{exemple}[Matrices triangulaires]
%Deux matrices triangulaires inférieures (à gauche), une matrice triangulaire supérieure (à droite) :
\[
\begin{pmatrix}
 4 & 0 & 0\\
 0 & -1 & 0\\
 3 & -2 & 3
\end{pmatrix}\qquad\qquad
\begin{pmatrix} 
 5 & 0\\
 1 & -2
\end{pmatrix}\qquad\qquad
\begin{pmatrix}
 1 & 1 & 0 \\
 0 & -1 & -1\\
 0 & 0 & -1
\end{pmatrix}
\]
\end{exemple}

\bigskip

\pause %
\begin{theoreme}
Une matrice $A$ de taille $n\times n$, triangulaire, est inversible 
si et seulement si ses éléments diagonaux sont tous non nuls 
\end{theoreme}



\end{frame}

%--------------------------------------------------------------


\begin{frame}

Une matrice qui est triangulaire inférieure \evidence{et} triangulaire supérieure 
est une matrice \defi{diagonale}: $i\neq j \ \Longrightarrow \ a_{ij} = 0$
\pause %
\begin{exemple}[Matrices diagonales]
$$\begin{pmatrix}
 -1 &  0 &0\\
 0 & 6 & 0\\
 0 & 0 &0 
 \end{pmatrix} \quad \text{ et } \quad
 \begin{pmatrix}
 2 & 0\\
 0 & 3
\end{pmatrix}$$
\end{exemple}

\pause

\begin{exemple}[Puissances d'une matrice diagonale]
% Si $D$ est une matrice diagonale, il est très facile de calculer ses puissances $D^p$ 
% (par récurrence sur $p$) :
% \pause %
\[
\arraycolsep=3pt
D=\begin{pmatrix} 
\alpha_{1}& 0& \dots &  \dots & 0\cr 
0& \alpha_{2}& 0&  \dots &0\cr
\vdots & \ddots & \ddots & \ddots & \vdots \cr
0& \dots & 0 & \alpha_{n-1} & 0 \cr
0& \dots & \dots & 0 & \alpha_{n}\cr 
\end{pmatrix}
\; \implies \;
D^{p}=\begin{pmatrix} 
\alpha_{1}^{p}& 0& \dots &  \dots & 0\cr 
0& \alpha_{2}^{p}& 0&  \dots &0\cr
\vdots & \ddots & \ddots & \ddots & \vdots \cr
0& \dots & 0 & \alpha_{n-1}^{p} & 0 \cr
0& \dots & \dots & 0 & \alpha_{n}^{p}\cr 
\end{pmatrix}
\]
\smallskip
\end{exemple}
\end{frame}

%--------------------------------------------------------------

% \begin{frame}
% \begin{theoreme}
% Une matrice $A$ de taille $n\times n$, triangulaire, est inversible 
% si et seulement si ses éléments diagonaux sont tous non nuls 
% \end{theoreme}
% \pause
% \begin{proof}
% \begin{overlayarea}{\textwidth}{4.8cm}
% \only<2>{Supposons que $A$ soit triangulaire supérieure. }
% 
% \only<3-4>{\begin{itemize}
%   \only<3>{\item Si les éléments de la diagonale sont tous non nuls, alors
%   la matrice $A$ est déjà sous la forme échelonnée. La forme échelonnée réduite 
%   de $A$ sera la matrice $\Id$, ce qui prouve l'inversibilité.}
%   
%   \item<4-> Si un des éléments diagonaux est nul, soit $a_{\ell\ell}$ le premier.
%   En multipliant les lignes $1$ à $\ell-1$ par l'inverse de leur élément diagonal $a_{ii}$, 
%   on obtient 
% 	\[\left( \begin{smallmatrix}
% 	1 & * & \cdots &&& \cdots& * \\
% 	0& \ddots  & * &\cdots &&\cdots & * \\
% 	0 & 0 & 1 & * && \cdots &* \\
% 	0 & \cdots & 0 & {\color{myred}0} &*&\cdots &*\\
% 	0 & \cdots & 0& 0& * & \cdots &*  \\
% 	\vdots & \vdots & \vdots & \cdots &0&\ddots & \vdots \\
% 	0& \cdots &&&\cdots&0&*   
% 	\end{smallmatrix}\right).
% 	\]
% 	La colonne numéro $\ell$ de la forme échelonnée réduite ne 
% 	contiendra jamais de $1$ comme pivot. La forme échelonnée réduite de $A$ 
% 	ne peut pas être $I_n$ et par conséquent, $A$ n'est pas inversible.
% \end{itemize}
% }
% 
% % \only<5>{Dans le cas d'une matrice triangulaire inférieure, on utilise 
% % la \defi{transposition} (voir section suivante) et on 
% % obtient une matrice triangulaire supérieure. On applique alors 
% % la démonstration ci-dessus.}
% \end{overlayarea}
% \end{proof}
% 
% \end{frame}



%%%%%%%%%%%%%%%%%%%%%%%%%%%%%%%%%%%%%%%%%%%%%%%%%%%%%%%%%%%%%%%%

\section{La transposition}

\begin{frame}
\begin{mydefinition}[Matrice transposée]
Soit $A = (a_{ij})$ une matrice de taille $n\times p$

\uncover<2->{La \defi{matrice transposée} de $A$, notée $A^T$ est la matrice de taille $p \times n$ 
définie par:}
\[
\renewcommand*{\arraystretch}{1.6} \arraycolsep=0.5mm
A = \left(
\begin{array}{cccc}
a_{11} & a_{12} & \dots & a_{1p}\\
a_{21} & a_{22} & \dots & a_{2p}\\
\vdots & \vdots &&\vdots\\
a_{n1} & a_{n2} & \dots & a_{np}
\end{array}\right)
\uncover<2->{
\longmapsto\ 
\renewcommand*{\arraystretch}{1} \arraycolsep=3mm
A^T = \left(
\begin{array}{cccc}
a_{11} & a_{21} & \dots & a_{n1}\\
a_{12} & a_{22} & \dots & a_{n2}\\
\vdots & \vdots &&\vdots\\
a_{1p} & a_{2p} &\dots & a_{np}
\end{array}\right)
}
\]
\end{mydefinition}
\pause\pause
Autrement dit : le coefficient $a_{ij}$ est à la place $(j,i)$ dans $A^T$

\pause
Ou encore : la $i$-ème ligne de $A$ devient la $i$-ème colonne de $A^{T}$ 
% (et réciproquement la $j$-ème colonne de $A^T$ est la $j$-ème ligne de $A$).

\pause
{\bf Notation :} La transposée de la matrice $A$ se note aussi $^{t\!}A$


\end{frame}

%--------------------------------------------------------------

\begin{frame}
%\vspace*{-10mm}
\begin{exemple}
\[
\left( \begin{smallmatrix}
1&2&3\\
4&5&-6\\
-7&8&9\\
\end{smallmatrix}\right)^T
= 
\left( \begin{smallmatrix}
1&4&-7\\
2&5&8\\
3&-6&9\\
\end{smallmatrix}\right)
\quad
\left( \begin{smallmatrix}
 0 & 3\\
 1 & -5\\
 -1 & 2  
\end{smallmatrix}\right)^T=
\left( \begin{smallmatrix}
 0 & 1 & -1\\
 3 &-5 & 2  
\end{smallmatrix}\right)
\]\[
\left( \begin{smallmatrix}
1 & -2 & 5 \end{smallmatrix}\right)^T  = 
\left( \begin{smallmatrix}
 1 \\
 -2\\
 5   
\end{smallmatrix}\right)
\]
\end{exemple} 
\pause %
\begin{theoreme}~
 \begin{enumerate}
    \item $(A + B)^T = A^T + B^T$
    \pause %
    \item $(\alpha \, A)^T = \alpha \, A^T $
    \pause %
    \item $(A^T)^T = A$
    \pause %
    \item \myboxinline{$(AB)^T = B^T A^T$} % \quad (Noter l'ordre inversé)
    \pause %
    \item Si $A$ est inversible, alors $A^T$ l'est aussi et on a $(A^T)^{-1}=(A^{-1})^T$
  \end{enumerate} 
\label{theotranspo}
\end{theoreme}
\end{frame}

%%%%%%%%%%%%%%%%%%%%%%%%%%%%%%%%%%%%%%%%%%%%%%%%%%%%%%%%%%%%%%%%

\section{La trace}

\begin{frame}
\begin{itemize}
  \item Soit $A=(a_{ij})$ une matrice \emph{carrée} de taille $n\times n$
  \pause
  \item Les éléments $a_{11}$, $a_{22}, \ldots, a_{nn}$ sont les \defi{éléments diagonaux}
  \pause
  \item La \defi{diagonale principale} est la diagonale $(a_{11},a_{22}, \ldots, a_{nn})$
  \[
A = \left( \begin{smallmatrix}
{\color{myred}a_{11}} & a_{12} & \dots & a_{1n}\\
 a_{21} & {\color{myred}a_{22}} & \dots & a_{2n}\\
 \vdots& \vdots & {\color{myred}\ddots}  & \vdots\\
 a_{n1} & a_{n2} & \dots & {\color{myred}a_{nn}}  
\end{smallmatrix} \right)
\] 
  
\end{itemize}


\pause %
\begin{mydefinition}
La \defi{trace}  de la matrice $A$ est
la somme des éléments diagonaux de $A$
\mybox{$\tr A = a_{11} + a_{22} + \cdots +a_{nn}$}
\end{mydefinition}
\pause %
\begin{exemple}
\begin{itemize}
  \item Pour
$ A = \left( \begin{smallmatrix}
2 & 1\\
0& 5
\end{smallmatrix}\right)$
alors \ $\tr A = 2 + 5 = 7$

\pause

  \item Pour $B = \left(\begin{smallmatrix}
1 & 1 &2\\
5&2&8\\
11 & 0 & -10
\end{smallmatrix}\right)$ alors \  $\tr B = 1 + 2 -10 =-7$
\end{itemize}
\end{exemple}

\end{frame}

%--------------------------------------------------------------

\begin{frame}
\begin{theoreme}
Soient $A$ et $B$ deux matrices $n \times n$. Alors 
\begin{enumerate}
\item $\tr(A + B)$ = $\tr A$ + $\tr B$
\item $\tr(\alpha A)$ = $\alpha$ $\tr A$ pour tout $\alpha \in \Kk$
\item $\tr(A^T)$ = $\tr A $
\item \myboxinline{$\tr(AB)$ = $\tr(BA)$}
\end{enumerate}
\end{theoreme}

% \setbeamercovered{transparent}
% %
% \begin{theoreme}
% Soient $A$ et $B$ deux matrices $n \times n$. Alors 
% \begin{enumerate}
% \item<1-2> $\tr(A + B)$ = $\tr A$ + $\tr B$
% \item<1,3> $\tr(\alpha A)$ = $\alpha$ $\tr A$ pour tout $\alpha \in \Kk$
% \item<1,4> $\tr(A^T)$ = $\tr A $
% \item<1,5-> \myboxinline{$\tr(AB)$ = $\tr(BA)$}
% \end{enumerate}
% \end{theoreme}
% %
% \visible<2->{
% \begin{proof}
% \begin{overlayarea}{\textwidth}{30mm}
% \begin{enumerate} \vspace*{-4mm}
% \only<1>{\item}	% pour éviter un enumerate sans item
% \only<2>{\setcounter{enumi}{0} \item 
% Pour tout $1 \leq i \leq n$, le coefficient $(i,i)$ de $A+B$ est $a_{ii} + b_{ii}$. 
% Ainsi, on a bien $\tr(A + B)$ = $\tr(A)$ + $\tr(B)$.}
% %
% \only<3>{\setcounter{enumi}{1} \item 
%  On a $\tr(\alpha A) = \alpha a_{11} +\dots+ \alpha a_{nn} 
% = \alpha (a_{11} +\dots+ a_{nn})= \alpha \tr A$.}
% %
% \only<4>{\setcounter{enumi}{2} \item 
%  \'Etant donné que la transposition ne change pas les éléments diagonaux, 
% la trace de $A$ est égale à la trace de $A^T$.}
% %
% \only<5->{\setcounter{enumi}{3} \item
% \only<5>{Soient $c_{ij}$ les coefficients de $AB$ et $d_{ij}$ ceux de $BA$:
% \[
% c_{ii} = a_{i1}b_{1i} + a_{i2}b_{2i} +\dots+ a_{in}b_{ni}
% \]\[
% d_{ii} = b_{i1}a_{1i} + b_{i2} a_{2i} +\dots+ b_{in} a_{ni}
% \]
% }
% \only<6-7>{
% \quad \vspace*{-8mm}
% \begin{figure}[b]
% \centering
% \tikzstyle{background}=[rectangle,
% 		fill=gray!20,
% 		inner sep=0.2mm,
% 		rounded corners=3mm]
% \begin{tikzpicture}[scale=1]
%   \matrix [column sep=2mm,row sep=2mm, ampersand replacement=\&] {
% 	\node (tr) {$\text{tr} (AB)= $}; \&
%   		\node (11) {$\phantom{+}a_{11} b_{11}$}; \&
%   		\node (12) {$+a_{12} b_{21}$}; \&
%   		\node (13) {$+\cdots$}; \&
%   		\node (1n) {$+a_{1n} b_{n1}$}; 
% 		\onslide<6>{\& \node (1c) {$(c_{11})$} ;}
% 		\\
% 		\& \node (21)	{$+a_{21} b_{12}$}; \&
%         \node (22) {$+ a_{22} b_{22}$}; \&
%         \node (23)   {$+ \cdots +$};     \&
%         \node (2n) {$+ a_{2n} b_{n2}$};
% 		\onslide<6>{\& \node (2c) {$(c_{22})$} ;}
%         \\
%        \& \node (31) {$\vdots$} ; \&
%         \node (32) {$\vdots$} ; \&
%         \node (33) {$\vdots$} ; \&
%         \node (3n) {$\vdots$} ; \&
% 		\node (3c) {} ;
%         \\
% 		\& \node (41)	{$+a_{41} b_{14}$}; \&
%         \node (42) {$+ a_{42} b_{24}$}; \&
%         \node (43)   {$+ \cdots +$};     \&
%         \node (4n) {$+ a_{4n} b_{n4}$};
% 		\onslide<6>{\& \node (4c) {$(c_{44})$} ;}
%       \only<7>{		
%       \\
%       a \& 
%       \node (51) {$(d_{11})$}; \&
%       \node (52) {$(d_{22})$}; \&
% 		\&
%       \node (5n) {$(d_{nn})$}; 
%       }
% 	\\
%     };
% 	\only<6>{
%     \begin{pgfonlayer}{background}
%         \node [background, fit=(11) (1n)] {};
%         \node [background, fit=(21) (2n)] {};
%         \node [background, fit=(41) (4n)] {};
%     \end{pgfonlayer}}
%     \only<7>{
%     \begin{pgfonlayer}{background}
%         \node [background, fit=(11) (41)] {};
%         \node [background, fit=(12) (42)] {};
%         \node [background, fit=(1n) (4n)] {};
%     \end{pgfonlayer}}
% \end{tikzpicture}
% \end{figure}
% }
% %Ainsi,
% %$$\begin{array}{cccccc}
% %\tr(AB) & = & a_{11}b_{11} & + a_{12}b_{21} & +\cdots &+ a_{1n}b_{n1}\\
% % & &+ a_{21}b_{12} & + a_{22}b_{22}& +\cdots & + a_{2n}b_{n2}\\
% %& & \vdots &&&\\
% %& & +a_{n1}b_{1n} & + a_{n2}b_{2n}& +\cdots & + a_{nn}b_{nn}.
% %\end{array}$$
% 
% %On peut réarranger les termes pour obtenir
% %
% %$$\begin{array}{cccccc}
% %\tr(AB) &=& a_{11}b_{11} & + a_{21}b_{12} & +\cdots &+ a_{n1}b_{1n}\\
% %&&+  a_{12}b_{21} & + a_{22}b_{22}& +\cdots & + a_{n2}b_{2n}\\
% % && \vdots &&\\
% % &&+ a_{1n}b_{n1} & + a_{2n}b_{n2}& +\cdots & + a_{nn}b_{nn}.
% %\end{array}$$
% %
% \only<8>{En utilisant la commutativité de la multiplication dans $\Kk$, 
% la première colonne devient
% \[
% a_{11} b_{11} + a_{21} b_{12} +\dots+ a_{n1} b_{1n}
% = b_{11}a_{11} + b_{12}a_{21} +\dots+ b_{1n}a_{n1} = d_{11},
% \]
% le coefficient $(1,1)$ de $BA$. Finalement, 
% $$\tr(AB) = d_{11} +\dots+ d_{nn} = \tr(BA).$$}
% }
% \end{enumerate}
% \end{overlayarea}
% \end{proof}
% }
\end{frame}


%%%%%%%%%%%%%%%%%%%%%%%%%%%%%%%%%%%%%%%%%%%%%%%%%%%%%%%%%%%%%%%%

\section{Matrices symétriques}

\begin{frame}
\begin{mydefinition}
Une matrice carrée $A$ de taille $n \times n$ est \defi{symétrique} si elle est égale 
à sa transposée, c'est-à-dire si
$$A = A^T$$ 



\end{mydefinition}
\pause
Autrement dit si \ $a_{ij}=a_{ji}$ \ pour tout $i,j=1, \ldots, n$

%\pause
Les coefficients sont donc symétriques par rapport à la diagonale
\bigskip

\pause %   
\begin{exemple} Les matrices suivantes sont symétriques :
 $$ \begin{pmatrix}
 0 & 2\\
 2 & 4\end{pmatrix} \qquad 
\begin{pmatrix}
 -1 & 0 & 5\\
 0 & 2 & -1\\
 5 & -1 & 0\end{pmatrix}
$$
\end{exemple} 
\end{frame}

%--------------------------------------------------------------

\begin{frame}
\begin{exemple}
Pour une matrice $B$ quelconque, les matrices $B \cdot B^T$ et $B^T \cdot B$ sont symétriques   
\end{exemple}

\bigskip
\pause

\begin{proof}
\[
(BB^T)^T = (B^T)^T B^T = BB^T 
\]
\end{proof}
\end{frame}

%%%%%%%%%%%%%%%%%%%%%%%%%%%%%%%%%%%%%%%%%%%%%%%%%%%%%%%%%%%%%%%%

\section{Matrices antisymétriques}

\begin{frame}
\begin{mydefinition}
Une matrice carrée $A$ de taille $n \times n$ est \defi{antisymétrique} si
$$A^T = -A$$
\end{mydefinition}
\pause
Autrement dit  si \ $a_{ij} = -a_{ji}$ \ pour tout $i,j=1, \ldots, n$

\pause %

\bigskip

\begin{exemple}
$$\begin{pmatrix}
0 & -1\\
1 & 0     
\end{pmatrix} \qquad 
\begin{pmatrix}
0 & 4 & 2\\
-4 & 0 & -5\\
-2 & 5 & 0  
\end{pmatrix}$$  
\end{exemple} 

\pause %

\bigskip
Les éléments diagonaux d'une matrice antisymétrique sont toujours tous nuls
(car en prenant $i=j$, on a $a_{ii} = - a_{ii}$)

\end{frame}

%--------------------------------------------------------------
% 
% \begin{frame}
% \begin{proposition}
% Toute matrice est la somme d'une matrice symétrique et d'une matrice antisymétrique.
% \end{proposition}
% \pause %
% \begin{proof}
% Soit $A$ une matrice. Définissons $B=\frac12(A+A^T)$ et $C=\frac12(A-A^T)$.
% Alors d'une part $A=B+C$ ; d'autre part $B$ est
% symétrique, car $B^T=\frac12(A^T+(A^T)^T)=\frac12(A^T+A)=B$ ; et enfin $C$ est
% antisymétrique, car $C^T=\frac12(A^T-(A^T)^T)=-C$.
% \end{proof}
% \pause %
% \begin{exemple}
% \[
% \begin{pmatrix} 2 & 10 \\ 8 & -3 \end{pmatrix}
% %\quad\text{ alors }\quad A 
% \; = \; \underbrace{\begin{pmatrix}
% 2 & 9\\
% 9 & -3\end{pmatrix}}_{\text{symétrique}} 
% \; + \; \underbrace{\begin{pmatrix}
% 0 & 1\\
% -1 & 0\end{pmatrix}}_{\text{antisymétrique}}
% \]
% \end{exemple}
% \end{frame}

%%%%%%%%%%%%%%%%%%%%%%%%%%%%%%%%%%%%%%%%%%%%%%%%%%%%%%%%%%%%%%%%
\section{Mini-exercices}

\begin{frame}

\begin{miniexercice}
\begin{enumerate}
  \item Montrer que la somme de deux matrices triangulaires supérieures reste triangulaire supérieure.
  Montrer que c'est aussi valable pour le produit.
  
  \item Montrer que si $A$ est triangulaire supérieure, alors $A^T$ est triangulaire inférieure.
  Et si $A$ est diagonale ?
  
  \item Soit $A = \left(\begin{smallmatrix}x_1 \\ x_2 \\ \vdots \\ x_n \end{smallmatrix} \right)$.
  Calculer $A^T \cdot A$, puis $A\cdot A^T$.
        
  \item Soit $A=\left(\begin{smallmatrix}a&b\\c&d\end{smallmatrix}\right)$. 
  Calculer $\tr(A\cdot A^T)$.
  
  \item Soit $A$ une matrice de taille $2\times 2$ inversible. Montrer
  que si $A$ est symétrique, alors $A^{-1}$ aussi. Et si $A$ est antisymétrique ?
  
  \item Montrer que la décomposition d'une matrice sous la forme
  \og symétrique + antisymétrique \fg\ est unique.

\end{enumerate}
\end{miniexercice}

\end{frame}

\end{document}