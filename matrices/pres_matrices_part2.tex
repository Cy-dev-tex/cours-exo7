
%%%%%%%%%%%%%%%%%% PREAMBULE %%%%%%%%%%%%%%%%%%

\documentclass[aspectratio=169,utf8]{beamer}
%\documentclass[aspectratio=169,handout]{beamer}

\usetheme{Boadilla}
%\usecolortheme{seahorse}
\usecolortheme[RGB={245,66,24}]{structure}
\useoutertheme{infolines}

% packages
\usepackage{amsfonts,amsmath,amssymb,amsthm}
\usepackage[utf8]{inputenc}
\usepackage[T1]{fontenc}
\usepackage{lmodern}

\usepackage[francais]{babel}
\usepackage{fancybox}
\usepackage{graphicx}

\usepackage{float}
\usepackage{xfrac}

%\usepackage[usenames, x11names]{xcolor}
\usepackage{tikz}
\usepackage{pgfplots}
\usepackage{datetime}



%-----  Package unités -----
\usepackage{siunitx}
\sisetup{locale = FR,detect-all,per-mode = symbol}

%\usepackage{mathptmx}
%\usepackage{fouriernc}
%\usepackage{newcent}
%\usepackage[mathcal,mathbf]{euler}

%\usepackage{palatino}
%\usepackage{newcent}
% \usepackage[mathcal,mathbf]{euler}



% \usepackage{hyperref}
% \hypersetup{colorlinks=true, linkcolor=blue, urlcolor=blue,
% pdftitle={Exo7 - Exercices de mathématiques}, pdfauthor={Exo7}}


%section
% \usepackage{sectsty}
% \allsectionsfont{\bf}
%\sectionfont{\color{Tomato3}\upshape\selectfont}
%\subsectionfont{\color{Tomato4}\upshape\selectfont}

%----- Ensembles : entiers, reels, complexes -----
\newcommand{\Nn}{\mathbb{N}} \newcommand{\N}{\mathbb{N}}
\newcommand{\Zz}{\mathbb{Z}} \newcommand{\Z}{\mathbb{Z}}
\newcommand{\Qq}{\mathbb{Q}} \newcommand{\Q}{\mathbb{Q}}
\newcommand{\Rr}{\mathbb{R}} \newcommand{\R}{\mathbb{R}}
\newcommand{\Cc}{\mathbb{C}} 
\newcommand{\Kk}{\mathbb{K}} \newcommand{\K}{\mathbb{K}}

%----- Modifications de symboles -----
\renewcommand{\epsilon}{\varepsilon}
\renewcommand{\Re}{\mathop{\text{Re}}\nolimits}
\renewcommand{\Im}{\mathop{\text{Im}}\nolimits}
%\newcommand{\llbracket}{\left[\kern-0.15em\left[}
%\newcommand{\rrbracket}{\right]\kern-0.15em\right]}

\renewcommand{\ge}{\geqslant}
\renewcommand{\geq}{\geqslant}
\renewcommand{\le}{\leqslant}
\renewcommand{\leq}{\leqslant}
\renewcommand{\epsilon}{\varepsilon}

%----- Fonctions usuelles -----
\newcommand{\ch}{\mathop{\text{ch}}\nolimits}
\newcommand{\sh}{\mathop{\text{sh}}\nolimits}
\renewcommand{\tanh}{\mathop{\text{th}}\nolimits}
\newcommand{\cotan}{\mathop{\text{cotan}}\nolimits}
\newcommand{\Arcsin}{\mathop{\text{arcsin}}\nolimits}
\newcommand{\Arccos}{\mathop{\text{arccos}}\nolimits}
\newcommand{\Arctan}{\mathop{\text{arctan}}\nolimits}
\newcommand{\Argsh}{\mathop{\text{argsh}}\nolimits}
\newcommand{\Argch}{\mathop{\text{argch}}\nolimits}
\newcommand{\Argth}{\mathop{\text{argth}}\nolimits}
\newcommand{\pgcd}{\mathop{\text{pgcd}}\nolimits} 


%----- Commandes divers ------
\newcommand{\ii}{\mathrm{i}}
\newcommand{\dd}{\text{d}}
\newcommand{\id}{\mathop{\text{id}}\nolimits}
\newcommand{\Ker}{\mathop{\text{Ker}}\nolimits}
\newcommand{\Card}{\mathop{\text{Card}}\nolimits}
\newcommand{\Vect}{\mathop{\text{Vect}}\nolimits}
\newcommand{\Mat}{\mathop{\text{Mat}}\nolimits}
\newcommand{\rg}{\mathop{\text{rg}}\nolimits}
\newcommand{\tr}{\mathop{\text{tr}}\nolimits}


%----- Structure des exercices ------

\newtheoremstyle{styleexo}% name
{2ex}% Space above
{3ex}% Space below
{}% Body font
{}% Indent amount 1
{\bfseries} % Theorem head font
{}% Punctuation after theorem head
{\newline}% Space after theorem head 2
{}% Theorem head spec (can be left empty, meaning ‘normal’)

%\theoremstyle{styleexo}
\newtheorem{exo}{Exercice}
\newtheorem{ind}{Indications}
\newtheorem{cor}{Correction}


\newcommand{\exercice}[1]{} \newcommand{\finexercice}{}
%\newcommand{\exercice}[1]{{\tiny\texttt{#1}}\vspace{-2ex}} % pour afficher le numero absolu, l'auteur...
\newcommand{\enonce}{\begin{exo}} \newcommand{\finenonce}{\end{exo}}
\newcommand{\indication}{\begin{ind}} \newcommand{\finindication}{\end{ind}}
\newcommand{\correction}{\begin{cor}} \newcommand{\fincorrection}{\end{cor}}

\newcommand{\noindication}{\stepcounter{ind}}
\newcommand{\nocorrection}{\stepcounter{cor}}

\newcommand{\fiche}[1]{} \newcommand{\finfiche}{}
\newcommand{\titre}[1]{\centerline{\large \bf #1}}
\newcommand{\addcommand}[1]{}
\newcommand{\video}[1]{}

% Marge
\newcommand{\mymargin}[1]{\marginpar{{\small #1}}}

\def\noqed{\renewcommand{\qedsymbol}{}}


%----- Presentation ------
\setlength{\parindent}{0cm}

%\newcommand{\ExoSept}{\href{http://exo7.emath.fr}{\textbf{\textsf{Exo7}}}}

\definecolor{myred}{rgb}{0.93,0.26,0}
\definecolor{myorange}{rgb}{0.97,0.58,0}
\definecolor{myyellow}{rgb}{1,0.86,0}

\newcommand{\LogoExoSept}[1]{  % input : echelle
{\usefont{U}{cmss}{bx}{n}
\begin{tikzpicture}[scale=0.1*#1,transform shape]
  \fill[color=myorange] (0,0)--(4,0)--(4,-4)--(0,-4)--cycle;
  \fill[color=myred] (0,0)--(0,3)--(-3,3)--(-3,0)--cycle;
  \fill[color=myyellow] (4,0)--(7,4)--(3,7)--(0,3)--cycle;
  \node[scale=5] at (3.5,3.5) {Exo7};
\end{tikzpicture}}
}


\newcommand{\debutmontitre}{
  \author{} \date{} 
  \thispagestyle{empty}
  \hspace*{-10ex}
  \begin{minipage}{\textwidth}
    \titlepage  
  \vspace*{-2.5cm}
  \begin{center}
    \LogoExoSept{2.5}
  \end{center}
  \end{minipage}

  \vspace*{-0cm}
  
  % Astuce pour que le background ne soit pas discrétisé lors de la conversion pdf -> png
\begin{tikzpicture}
        \fill[opacity=0,green!60!black] (0,0)--++(0,0)--++(0,0)--++(0,0)--cycle; 
\end{tikzpicture}

% toc S'affiche trop tot :
% \tableofcontents[hideallsubsections, pausesections]
}

\newcommand{\finmontitre}{
  \end{frame}
  \setcounter{framenumber}{0}
} % ne marche pas pour une raison obscure

%----- Commandes supplementaires ------

% \usepackage[landscape]{geometry}
% \geometry{top=1cm, bottom=3cm, left=2cm, right=10cm, marginparsep=1cm
% }
% \usepackage[a4paper]{geometry}
% \geometry{top=2cm, bottom=2cm, left=2cm, right=2cm, marginparsep=1cm
% }

%\usepackage{standalone}


% New command Arnaud -- november 2011
\setbeamersize{text margin left=24ex}
% si vous modifier cette valeur il faut aussi
% modifier le decalage du titre pour compenser
% (ex : ici =+10ex, titre =-5ex

\theoremstyle{definition}
%\newtheorem{proposition}{Proposition}
%\newtheorem{exemple}{Exemple}
%\newtheorem{theoreme}{Théorème}
%\newtheorem{lemme}{Lemme}
%\newtheorem{corollaire}{Corollaire}
%\newtheorem*{remarque*}{Remarque}
%\newtheorem*{miniexercice}{Mini-exercices}
%\newtheorem{definition}{Définition}

% Commande tikz
\usetikzlibrary{calc}
\usetikzlibrary{patterns,arrows}
\usetikzlibrary{matrix}
\usetikzlibrary{fadings} 

%definition d'un terme
\newcommand{\defi}[1]{{\color{myorange}\textbf{\emph{#1}}}}
\newcommand{\evidence}[1]{{\color{blue}\textbf{\emph{#1}}}}
\newcommand{\assertion}[1]{\emph{\og#1\fg}}  % pour chapitre logique
%\renewcommand{\contentsname}{Sommaire}
\renewcommand{\contentsname}{}
\setcounter{tocdepth}{2}



%------ Figures ------

\def\myscale{1} % par défaut 
\newcommand{\myfigure}[2]{  % entrée : echelle, fichier figure
\def\myscale{#1}
\begin{center}
\footnotesize
{#2}
\end{center}}


%------ Encadrement ------

\usepackage{fancybox}


\newcommand{\mybox}[1]{
\setlength{\fboxsep}{7pt}
\begin{center}
\shadowbox{#1}
\end{center}}

\newcommand{\myboxinline}[1]{
\setlength{\fboxsep}{5pt}
\raisebox{-10pt}{
\shadowbox{#1}
}
}

%--------------- Commande beamer---------------
\newcommand{\beameronly}[1]{#1} % permet de mettre des pause dans beamer pas dans poly


\setbeamertemplate{navigation symbols}{}
\setbeamertemplate{footline}  % tiré du fichier beamerouterinfolines.sty
{
  \leavevmode%
  \hbox{%
  \begin{beamercolorbox}[wd=.333333\paperwidth,ht=2.25ex,dp=1ex,center]{author in head/foot}%
    % \usebeamerfont{author in head/foot}\insertshortauthor%~~(\insertshortinstitute)
    \usebeamerfont{section in head/foot}{\bf\insertshorttitle}
  \end{beamercolorbox}%
  \begin{beamercolorbox}[wd=.333333\paperwidth,ht=2.25ex,dp=1ex,center]{title in head/foot}%
    \usebeamerfont{section in head/foot}{\bf\insertsectionhead}
  \end{beamercolorbox}%
  \begin{beamercolorbox}[wd=.333333\paperwidth,ht=2.25ex,dp=1ex,right]{date in head/foot}%
    % \usebeamerfont{date in head/foot}\insertshortdate{}\hspace*{2em}
    \insertframenumber{} / \inserttotalframenumber\hspace*{2ex} 
  \end{beamercolorbox}}%
  \vskip0pt%
}


\definecolor{mygrey}{rgb}{0.5,0.5,0.5}
\setlength{\parindent}{0cm}
%\DeclareTextFontCommand{\helvetica}{\fontfamily{phv}\selectfont}

% background beamer
\definecolor{couleurhaut}{rgb}{0.85,0.9,1}  % creme
\definecolor{couleurmilieu}{rgb}{1,1,1}  % vert pale
\definecolor{couleurbas}{rgb}{0.85,0.9,1}  % blanc
\setbeamertemplate{background canvas}[vertical shading]%
[top=couleurhaut,middle=couleurmilieu,midpoint=0.4,bottom=couleurbas] 
%[top=fondtitre!05,bottom=fondtitre!60]



\makeatletter
\setbeamertemplate{theorem begin}
{%
  \begin{\inserttheoremblockenv}
  {%
    \inserttheoremheadfont
    \inserttheoremname
    \inserttheoremnumber
    \ifx\inserttheoremaddition\@empty\else\ (\inserttheoremaddition)\fi%
    \inserttheorempunctuation
  }%
}
\setbeamertemplate{theorem end}{\end{\inserttheoremblockenv}}

\newenvironment{theoreme}[1][]{%
   \setbeamercolor{block title}{fg=structure,bg=structure!40}
   \setbeamercolor{block body}{fg=black,bg=structure!10}
   \begin{block}{{\bf Th\'eor\`eme }#1}
}{%
   \end{block}%
}


\newenvironment{proposition}[1][]{%
   \setbeamercolor{block title}{fg=structure,bg=structure!40}
   \setbeamercolor{block body}{fg=black,bg=structure!10}
   \begin{block}{{\bf Proposition }#1}
}{%
   \end{block}%
}

\newenvironment{corollaire}[1][]{%
   \setbeamercolor{block title}{fg=structure,bg=structure!40}
   \setbeamercolor{block body}{fg=black,bg=structure!10}
   \begin{block}{{\bf Corollaire }#1}
}{%
   \end{block}%
}

\newenvironment{mydefinition}[1][]{%
   \setbeamercolor{block title}{fg=structure,bg=structure!40}
   \setbeamercolor{block body}{fg=black,bg=structure!10}
   \begin{block}{{\bf Définition} #1}
}{%
   \end{block}%
}

\newenvironment{lemme}[0]{%
   \setbeamercolor{block title}{fg=structure,bg=structure!40}
   \setbeamercolor{block body}{fg=black,bg=structure!10}
   \begin{block}{\bf Lemme}
}{%
   \end{block}%
}

\newenvironment{remarque}[1][]{%
   \setbeamercolor{block title}{fg=black,bg=structure!20}
   \setbeamercolor{block body}{fg=black,bg=structure!5}
   \begin{block}{Remarque #1}
}{%
   \end{block}%
}


\newenvironment{exemple}[1][]{%
   \setbeamercolor{block title}{fg=black,bg=structure!20}
   \setbeamercolor{block body}{fg=black,bg=structure!5}
   \begin{block}{{\bf Exemple }#1}
}{%
   \end{block}%
}


\newenvironment{miniexercice}[0]{%
   \setbeamercolor{block title}{fg=structure,bg=structure!20}
   \setbeamercolor{block body}{fg=black,bg=structure!5}
   \begin{block}{Mini-exercices}
}{%
   \end{block}%
}


\newenvironment{tp}[0]{%
   \setbeamercolor{block title}{fg=structure,bg=structure!40}
   \setbeamercolor{block body}{fg=black,bg=structure!10}
   \begin{block}{\bf Travaux pratiques}
}{%
   \end{block}%
}
\newenvironment{exercicecours}[1][]{%
   \setbeamercolor{block title}{fg=structure,bg=structure!40}
   \setbeamercolor{block body}{fg=black,bg=structure!10}
   \begin{block}{{\bf Exercice }#1}
}{%
   \end{block}%
}
\newenvironment{algo}[1][]{%
   \setbeamercolor{block title}{fg=structure,bg=structure!40}
   \setbeamercolor{block body}{fg=black,bg=structure!10}
   \begin{block}{{\bf Algorithme}\hfill{\color{gray}\texttt{#1}}}
}{%
   \end{block}%
}


\setbeamertemplate{proof begin}{
   \setbeamercolor{block title}{fg=black,bg=structure!20}
   \setbeamercolor{block body}{fg=black,bg=structure!5}
   \begin{block}{{\footnotesize Démonstration}}
   \footnotesize
   \smallskip}
\setbeamertemplate{proof end}{%
   \end{block}}
\setbeamertemplate{qed symbol}{\openbox}


\makeatother
\usecolortheme[RGB={191,146,10}]{structure}

% Commande spécifique à ce chapitre
\newcounter{saveenumi}

%%%%%%%%%%%%%%%%%%%%%%%%%%%%%%%%%%%%%%%%%%%%%%%%%%%%%%%%%%%%%
%%%%%%%%%%%%%%%%%%%%%%%%%%%%%%%%%%%%%%%%%%%%%%%%%%%%%%%%%%%%%


\begin{document}


\title{{\bf Matrices}}
\subtitle{Multiplication de matrices}

\begin{frame}
  
  \debutmontitre

  \pause

{\footnotesize
\hfill
\setbeamercovered{transparent=50}
\begin{minipage}{0.6\textwidth}
  \begin{itemize}
    \item<3-> Définition du produit
    \item<4-> Propriétés du produit de matrices
    \item<5-> La matrice identité
    \item<6-> Puissance d'une matrice   
  \end{itemize}
\end{minipage}
}

\end{frame}

\setcounter{framenumber}{0}


%%%%%%%%%%%%%%%%%%%%%%%%%%%%%%%%%%%%%%%%%%%%%%%%%%%%%%%%%%%%%%%%
\section{Définition du produit}

\begin{frame}

\begin{mydefinition}[Produit de deux matrices]
$A=(a_{ij})$ une matrice ${\color<2>{blue}{n}}\times {\color<1>{red}{p}}$ et $B=(b_{ij})$ une matrice ${\color<1>{red}{p}}\times {\color<2>{magenta}{q}}$

\pause
Alors le produit $C=AB$ est une matrice ${\color<2>{blue}{n}}\times {\color<2>{magenta}{q}}$ \pause définie par 

\vspace{-.3cm}

\mybox{$\displaystyle
c_{ij} = \sum_{k=1}^p a_{ik}b_{kj}
$}
\end{mydefinition}

\vspace{-.4cm}

\pause
$$c_{{\color<4->{blue}{i}}{\color<4->{magenta}{j}}}=a_{{\color<4->{blue}{i}}1}b_{1{\color<4->{magenta}{j}}}+a_{{\color<4->{blue}{i}}2}b_{2{\color<4->{magenta}{j}}}+ \dots + a_{{\color<4->{blue}{i}}k}b_{k{\color<4->{magenta}{j}}}+ \dots + a_{{\color<4->{blue}{i}}p}b_{p{\color<4->{magenta}{j}}}
$$ 

\vspace{-.5cm}

\pause

$$\begin{array}{ccl}
&\begin{pmatrix}
&&&{\color{magenta}\times}&&\\
&&&{\color{magenta}\times}&&\\
\hphantom{-}&\hphantom{-}&\hphantom{-}&{\color{magenta}\times}&\hphantom{-}&\hphantom{-}\\
&&&{\color{magenta}\times}&&
\end{pmatrix}&\leftarrow B\\
A\to\begin{pmatrix}
&&&\\
&&&\\
{\color{blue}\times}&{\color{blue}\times}&{\color{blue}\times}&{\color{blue}\times}\\
&&&
\end{pmatrix}
&\begin{pmatrix}
&&&|&&\\
&&&|&&\\
-&-&-&{\color{myred}c_{ij}}&\hphantom{-}&\hphantom{-}\\
&&&&&
\end{pmatrix}&\leftarrow AB\\
\end{array}
$$


\end{frame}

%%%%%%%%%%%%%%%%%%%%%%%%%%%%%%%%%%%%%%%%%%%%%%%%%%%%%%%%%%%%%%%%
\section{Exemples}

\begin{frame}

\begin{exemple}
$$A =\begin{pmatrix}
1 & 2 & 3\cr 
2 & 3 & 4\cr
\end{pmatrix}\qquad B =
\begin{pmatrix}
1&2\cr
-1&1 \cr
1&1\cr
\end{pmatrix}
$$

\pause

$$ \begin{array}{cc}
  & \begin{pmatrix}
{\color<3,5>{myred}1}&{\color<4,6>{myred}2}\cr
{\color<3,5>{myred}-1}&{\color<4,6>{myred}1} \cr
{\color<3,5>{myred}1}&{\color<4,6>{myred}1}\cr
\end{pmatrix}  \\
\begin{pmatrix}
{\color<3,4>{myred}1} & {\color<3,4>{myred}2} & {\color<3,4>{myred}3}\cr 
{\color<5,6>{myred}2} & {\color<5,6>{myred}3}& {\color<5,6>{myred}4}\cr
\end{pmatrix}
   & \begin{pmatrix} \only<2>{c_{11}} \only<3->{\color<3>{blue}2} & \only<-3>{c_{12}} \only<4->{\color<4>{blue}7} \\ \only<-4>{c_{21}}\only<5->{\color<5>{blue}3} & \only<-5>{c_{22}}\only<6->{\color<6>{blue}11} \end{pmatrix}
  \end{array}   
 $$ 
  
  \pause  \pause  \pause   \pause
\end{exemple}


\end{frame}

%%%%%%%%%%%%%%%%%%

\begin{frame}

\begin{exemple}
$$u = \begin{pmatrix} a_1 & a_2 & \cdots & a_n \end{pmatrix} \qquad 
v = \begin{pmatrix} b_1 \\ b_2 \\ \vdots \\ b_n \end{pmatrix}$$

\pause
Alors $u \times v $ est une matrice $1\times 1$
\[
u \times v =  \Big( \ a_1 b_1 + a_2 b_2 + \cdots + a_n b_n \ \Big)
\]
\pause
Ce nombre s'appelle le \defi{produit scalaire} des vecteurs $u$ et $v$

\end{exemple}

\end{frame}


%%%%%%%%%%%%%%%%%%%%%%%%%%%%%%%%%%%%%%%%%%%%%%%%%%%%%%%%%%%%%%%%
\section{Pièges à éviter}

\begin{frame}

\begin{enumerate}
\item  $AB\neq BA$ : le produit de matrices n'est pas commutatif en général 

\pause
\begin{exemple}
$$\begin{pmatrix}
5&1\\3&-2
\end{pmatrix}
\begin{pmatrix}
2&0\\4&3
\end{pmatrix}
\uncover<3->{
=
\begin{pmatrix}
14&3\\-2&-6
\end{pmatrix}
}$$
et
$$
\begin{pmatrix}
2&0\\4&3
\end{pmatrix}
\begin{pmatrix}
5&1\\3&-2
\end{pmatrix}
\uncover<4->{
=
\begin{pmatrix}
10&2\\29&-2
\end{pmatrix}
}
$$
\end{exemple}
\pause\pause
  \item\pause $AB=0$ n'implique pas $A=0$ ou $B=0$
  
\pause
\begin{exemple}
$$
A  = \begin{pmatrix}
 0 & -1\\
 0 & 5\end{pmatrix}\qquad B  = 
\begin{pmatrix}
2 & -3\\
0 & 0 \end{pmatrix}
\qquad \text{et} \qquad 
 AB  =  \begin{pmatrix}
 0 & 0\\
 0 & 0\end{pmatrix}
$$
\end{exemple}

\pause

  \item $AB=AC$ n'implique pas $B=C$
  


%\setcounter{saveenumi}{\theenumi}
\end{enumerate}
\end{frame}

%%%%%%%%%%%%%%%%%%

% \begin{frame}
% \begin{enumerate}
%   \setcounter{enumi}{\thesaveenumi}
%   \item $AB=AC$ n'implique pas $B=C$
%  
% \pause  
% \begin{exemple}
% $$
%  A  =  \begin{pmatrix}
%  0 & -1\\
%  0 & 3\end{pmatrix}
%  \qquad  
%  B  = \begin{pmatrix} 
% 4 & -1\\
% 5 & 4 \end{pmatrix}
%  \qquad 
%  C  = \begin{pmatrix} 
%  2 & 5\\
%  5 & 4\end{pmatrix}
%  $$
%  Alors
%  $$
%  AB  = AC = \begin{pmatrix}
%   -5 & -4\\
%   15 & 12\end{pmatrix}
% $$
% \end{exemple}
% \end{enumerate}
% \end{frame}




%%%%%%%%%%%%%%%%%%%%%%%%%%%%%%%%%%%%%%%%%%%%%%%%%%%%%%%%%%%%%%%%
\section{Propriétés du produit de matrices}

\begin{frame}

\begin{proposition} \ 
\begin{enumerate}\setlength{\itemsep}{10pt}
  \item $A (BC) = (AB) C$ % : associativité du produit

  \item\pause $A(B+C) = AB + AC$ \quad et \quad  $(B+C) A = BA + CA$ % : distributivité du produit par rapport à la somme
 
  \item\pause $A\cdot 0 = 0$ \quad et \quad $0\cdot A= 0$
  \vspace*{2ex}
\end{enumerate}
\end{proposition}
 
 
\end{frame}

%%%%%%%%%%%%%%
% 
% \begin{frame}
% 
% \begin{proof}
% \begin{enumerate}
%   \item  $A=(a_{ij}) \in M_{n,p}(\Kk)$, $B=(b_{ij})\in M_{p,q}(\Kk)$ 
% et $C=(c_{ij})\in M_{q,r}(\Kk)$
% 
% \pause
% Montrons que les matrices $A(BC)$ et $(AB) C$ ont les mêmes coefficients
% \vspace{-.2cm}
% \begin{itemize}
% \item\pause le coefficient $(i,k)$ de $AB$ est $x_{ik}={\displaystyle \sum_{\ell=1}^{p}}a_{i \ell}b_{\ell k}$
% 
% \item\pause le coefficient $(i,j)$ de $(AB)C$ est donc
% \vspace{-.2cm}
% $$\sum_{k=1}^{q}x_{ik}c_{kj}=\sum_{k=1}^{q}
% \left ( \sum_{\ell=1}^{p}a_{i\ell}b_{\ell k} \right )c_{kj}$$
% 
% \item\pause le coefficient $(\ell,j)$ de $BC$ est 
% $y_{\ell j}={\displaystyle \sum_{k=1}^{q}}b_{\ell k}c_{kj}$
% 
% \item\pause  le coefficient $(i,j)$ de $A(BC)$ est donc 
% \vspace{-.2cm}
% $$\sum_{\ell=1}^{p}a_{i\ell}\left (  \sum_{k=1}^{q}b_{\ell k}c_{kj}\right )$$
% 
% \item\pause distributivité et associativité de la multiplication dans $\Kk$
% \vspace{-.6cm}
% \end{itemize}
% \end{enumerate}
% 
% \end{proof}
% 
% \end{frame}

%%%%%%%%%%%%%%%%%%%%%%%%%%%%%%%%%%%%%%%%%%%%%%%%%%%%%%%%%%%%%%%%
\section{La matrice identité}

\begin{frame}

La matrice carrée suivante s'appelle la \defi{matrice identité}, notée $I_n$ ou $I$
 \[
 \Id = \left(
 \begin{array}{cccc}
1 & 0 & \dots & 0\\
0& 1& \dots & 0\\
 \vdots& \vdots & \ddots  & \vdots\\
0 & 0 & \dots &1
\end{array}
\right)
 \]

\bigskip

\pause
On définit le \defi{symbole de Kronecker} $\delta_{i,j}\in \Rr$ : pour $i$ et $j$ entiers
$$\delta_{i,j}=\begin{cases}
0 & \text{ si } \;\; i\neq j \\
1 & \text{ si } \;\; i=j
\end{cases}$$

\pause
\bigskip

\centerline{Alors $I_{n}=\Big(\delta_{i,j}\Big)_{1\leq i,j \leq n}$ }



\end{frame}

%%%%%%%%%%%%%%%%%%

\begin{frame}

\begin{proposition}
Si $A$ est une matrice $n \times p$, alors 
$$ \Id \cdot A = A \qquad \text{et} \qquad A \cdot I_p = A$$  
\end{proposition}

\pause
\begin{proof} 
$A=(a_{ij}) \in M_{n,p}(\Kk)$

\begin{enumerate}
\item Montrons $AI_p = A$

\pause
$AI_{p}\in M_{n,p}(\Kk)$ de terme général $c_{ij}={\displaystyle \sum_{k=1}^{p}a_{ik}\delta_{kj}}$

\begin{itemize}
\item\pause si $k\neq j$ alors $\delta_{kj}=0$

\item\pause si $k=j$ alors $\delta_{kj}=1$

\item\pause donc $c_{ij}=a_{ij}\delta_{jj}=a_{ij}1=a_{ij}$

\end{itemize}
\pause
$AI_{p}$ et $A$ ont le même terme général et sont donc égales

\item\pause De même $I_{n} A=A$
\vspace{-.2cm}
\end{enumerate}
\end{proof}

\end{frame}

 


%%%%%%%%%%%%%%%%%%%%%%%%%%%%%%%%%%%%%%%%%%%%%%%%%%%%%%%%%%%%%%%%
\section{Puissance d'une matrice}

\begin{frame}

Dans $M_{n}(\Kk)$ la multiplication des matrices est une opération interne :
\[
\text{si } A,B \in M_n(\Kk) \quad \text{ alors } A \times B \in M_n(\Kk)
\]

\pause

\begin{itemize}\setlength{\itemsep}{10pt}
  \item En particulier : $A^2 = A \times A$, \ $A^3 = A \times A \times A$, ...
\pause  
  \item $A^p = \underbrace{A \times A \times \cdots \times A}_{p \text{ facteurs}}$
\pause  
  \item $A^0=\Id$
\end{itemize}

% 
% 
% \pause
% \begin{mydefinition}
% Pour tout $A\in M_n(\Kk)$, on définit les puissances successives de $A$ par 
% \[
% \begin{cases}
% A^0=\Id \\
% A^{p+1}=A^p \times A \quad \text{ pour tout } p\in\N
% \end{cases}
% \]
% \end{mydefinition}
% 
% \bigskip
% \pause
% \centerline{$A^p = \underbrace{A \times A \times \cdots \times A}_{p \text{ facteurs}}$}
\end{frame}


\begin{frame}

\begin{exemple}
Calculons $A^{p}$ avec 
$A=\begin{pmatrix}
1 & 0  & 1 \cr
0 & -1 & 0\cr
0 & 0  & 2 \cr
\end{pmatrix}$

\begin{itemize}
\item\pause On calcule $A^{2}$, $A^3$ et $A^{4}$ : \pause \quad
$A^{2}= \begin{pmatrix}
1 & 0 & 3 \cr
0 & 1 & 0\cr
0 & 0 & 4 \cr
\end{pmatrix}$

\pause
$
A^{3}=A^2 \times A =
\begin{pmatrix}
1 & 0 & 7 \cr
0 & -1& 0\cr
0 & 0 & 8 \cr
\end{pmatrix}
$ \quad \pause $
A^{4}=A^3 \times A =
\begin{pmatrix}
1 & 0 & 15 \cr
0 & 1& 0\cr
0 & 0 & 16 \cr
\end{pmatrix}
$

\item\pause On \emph{conjecture} la formule
$A^{p}= \begin{pmatrix}
1 & 0       & 2^p-1 \cr
0 & (-1)^{p}& 0\cr
0 & 0       & 2^p \cr
\end{pmatrix}$

\end{itemize}
\end{exemple}

\end{frame}


\begin{frame}
\begin{exemple}
Démontrons par récurrence $A^{p}= \begin{pmatrix}
1 & 0       & 2^p-1 \cr
0 & (-1)^{p}& 0\cr
0 & 0       & 2^p \cr
\end{pmatrix}$

\begin{enumerate}
\item\pause Vrai pour $p=0$ 
\item\pause On suppose la formule vraie pour un entier $p$

\pause
Alors
\begin{align*}
A ^{p+1}=A ^{p} \times A \uncover<5->{ &= 
\begin{pmatrix}
1 & 0       & 2^p-1 \cr
0 & (-1)^{p}& 0\cr
0 & 0       & 2^p \cr
\end{pmatrix} \times
\begin{pmatrix}
1 & 0 & 1 \cr
0 & -1& 0\cr
0 & 0 & 2 \cr
\end{pmatrix}}\\
\uncover<6->{  &=\begin{pmatrix}
1 & 0       & 2^{p+1}-1 \cr
0 & (-1)^{p+1}& 0\cr
0 & 0       & 2^{p+1} \cr
\end{pmatrix}}
\end{align*}
\item \uncover<7->{ D'après le principe de récurrence, la propriété est démontrée}
\end{enumerate}

\end{exemple}

\end{frame}

%%%%%%%%%%%%%%%%%%%%%%%%%%%%%%%%%%%%%%%%%%%%%%%%%%%%%%%%%%%%%%%%
\section{Formule du binôme}

\begin{frame}
\begin{remarque}
Comme la multiplication n'est pas commutative 
\begin{align*}
(A+B)^2&= A^2+{\color<1>{myred}AB+BA}+B^2\\
&\neq A^2+2AB+B^2
\end{align*}
\end{remarque}

\pause
\begin{proposition}%[Calcul de $(A+B)^{p}$ lorsque $AB=BA$]
$A, B\in M_{n}(\Kk)$ tels que {\color<1->{myred}$AB=BA$}

\pause
Alors pour tout entier $p \ge 0$ 
$$(A+B)^{p}= \sum_{k=0}^{p} \binom{p}{k} A^{p-k}B^{k}$$
où $\binom{p}{k}$ désigne le coefficient du binôme
\end{proposition}

\end{frame}

%%%%%%%%%%%%%%

\begin{frame}

\begin{exemple}
Calculons $A^{p}$ avec $A=\begin{pmatrix}
1&1&1&1\cr 
0 & 1&2&1 \cr
0&0&1&3 \cr
0&0&0&1
\end{pmatrix}$

\begin{itemize}\setlength{\itemsep}{10pt}
\item\pause On pose 
$N=A - I_4= \begin{pmatrix}
0&1&1&1\cr 
0 &0&2&1 \cr
0&0&0&3 \cr
0&0&0&0
\end{pmatrix}$

\pause 

\item Alors $N^{2}= \begin{pmatrix}
0&0&2&4\cr 
0 &0&0&6 \cr
0&0&0&0 \cr
0&0&0&0
\end{pmatrix} \qquad \pause
N^{3}= \begin{pmatrix}
0&0&0&6\cr 
0 &0&0&0 \cr
0&0&0&0 \cr
0&0&0&0
\end{pmatrix}$

\pause 

\item  $N^{4}=0$, la matrice $N$ est dite nilpotente

\end{itemize}
\end{exemple}

\end{frame}

%--------------------------------------------------------------

\begin{frame}
\begin{exemple}

\begin{itemize}
\item Comme $A=\Id+N$ et les matrices $N$ et $\Id$ commutent :
\begin{align*}
A^{p} \uncover<2->{ &= \sum_{k=0}^{p} \binom{p}{k} N^{k} \Id^{p-k}} \uncover<3->{ = \sum_{k=0}^{p} \binom{p}{k} N^{k}}\\
& \uncover<4->{= \displaystyle \sum_{k=0}^{{\color<4,5>{myred}3}} \binom{p}{k} N^{k}}
\uncover<5->{ = \Id+pN+\tfrac{p(p-1)}{2!}N^{2}+ \tfrac{p(p-1)(p-2)}{3!}N^{3}}
\end{align*}
\onslide<4->{car $N^{k}=0$ si $k \geq 4$}

%On utilise que $\Id^k=\Id$ pour tout $k$ et surtout que $N^{k}=0$ si $k \geq 4$. 

\item\uncover<6->{ D'où 
$$A^{p}= \begin{pmatrix}
1&p&p^{2}&p(p^{2}-p+1)\cr 
0 & 1&2p&p(3p-2) \cr
0&0&1&3p \cr
0&0&0&1
\end{pmatrix}$$}
\end{itemize}
\end{exemple}


\end{frame}

%%%%%%%%%%%%%%%%%%%%%%%%%%%%%%%%%%%%%%%%%%%%%%%%%%%%%%%%%%%%%%%%
\section{Mini-exercices}

\begin{frame}

\begin{miniexercice}
\begin{enumerate}
  \item Soient 
  $A=\left(\begin{smallmatrix}0&2&-2\\6&-4&0\end{smallmatrix} \right)$,
  $B=\left(\begin{smallmatrix}2&1&0\\0&1&0\\2&-2&-3\end{smallmatrix} \right)$,
  $C=\left(\begin{smallmatrix}8&2\\-3&2\\-5&5\end{smallmatrix} \right)$,
  $D=\left(\begin{smallmatrix}5\\2\\-1\end{smallmatrix} \right)$,
  $E=\left(\begin{matrix}x&y&z\end{matrix} \right)$.
  Quels produits sont possibles ? Les calculer !
  
  \item Soient $A= \left(\begin{smallmatrix}0&0&1\\0&1&0\\1&1&2 \end{smallmatrix} \right)$
  et $B=\left(\begin{smallmatrix}1&0&0\\0&0&2\\1&-1&0\end{smallmatrix} \right)$.
  Calculer $A^2$, $B^2$, $AB$ et $BA$.
  
  \item Soient $A= \left(\begin{smallmatrix} 2&0&0\\0&2&0\\0&0&2\end{smallmatrix} \right)$
  et $B=\left(\begin{smallmatrix}0&0&0\\2&0&0\\3&1&0\end{smallmatrix} \right)$.
  Calculer $A^p$ et $B^p$ pour tout $p\ge0$. Montrer que $AB=BA$. Calculer $(A+B)^p$.

\end{enumerate}
\end{miniexercice}

\end{frame}

\end{document}