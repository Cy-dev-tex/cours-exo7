
%%%%%%%%%%%%%%%%%% PREAMBULE %%%%%%%%%%%%%%%%%%


\documentclass[12pt]{article}

\usepackage{amsfonts,amsmath,amssymb,amsthm}
\usepackage[utf8]{inputenc}
\usepackage[T1]{fontenc}
\usepackage[francais]{babel}


% packages
\usepackage{amsfonts,amsmath,amssymb,amsthm}
\usepackage[utf8]{inputenc}
\usepackage[T1]{fontenc}
%\usepackage{lmodern}

\usepackage[francais]{babel}
\usepackage{fancybox}
\usepackage{graphicx}

\usepackage{float}

%\usepackage[usenames, x11names]{xcolor}
\usepackage{tikz}
\usepackage{datetime}

\usepackage{mathptmx}
%\usepackage{fouriernc}
%\usepackage{newcent}
\usepackage[mathcal,mathbf]{euler}

%\usepackage{palatino}
%\usepackage{newcent}


% Commande spéciale prompteur

%\usepackage{mathptmx}
%\usepackage[mathcal,mathbf]{euler}
%\usepackage{mathpple,multido}

\usepackage[a4paper]{geometry}
\geometry{top=2cm, bottom=2cm, left=1cm, right=1cm, marginparsep=1cm}

\newcommand{\change}{{\color{red}\rule{\textwidth}{1mm}\\}}

\newcounter{mydiapo}

\newcommand{\diapo}{\newpage
\hfill {\normalsize  Diapo \themydiapo \quad \texttt{[\jobname]}} \\
\stepcounter{mydiapo}}


%%%%%%% COULEURS %%%%%%%%%%

% Pour blanc sur noir :
%\pagecolor[rgb]{0.5,0.5,0.5}
% \pagecolor[rgb]{0,0,0}
% \color[rgb]{1,1,1}



%\DeclareFixedFont{\myfont}{U}{cmss}{bx}{n}{18pt}
\newcommand{\debuttexte}{
%%%%%%%%%%%%% FONTES %%%%%%%%%%%%%
\renewcommand{\baselinestretch}{1.5}
\usefont{U}{cmss}{bx}{n}
\bfseries

% Taille normale : commenter le reste !
%Taille Arnaud
%\fontsize{19}{19}\selectfont

% Taille Barbara
%\fontsize{21}{22}\selectfont

%Taille François
%\fontsize{25}{30}\selectfont

%Taille Pascal
%\fontsize{25}{30}\selectfont

%Taille Laura
%\fontsize{30}{35}\selectfont


%\myfont
%\usefont{U}{cmss}{bx}{n}

%\Huge
%\addtolength{\parskip}{\baselineskip}
}


% \usepackage{hyperref}
% \hypersetup{colorlinks=true, linkcolor=blue, urlcolor=blue,
% pdftitle={Exo7 - Exercices de mathématiques}, pdfauthor={Exo7}}


%section
% \usepackage{sectsty}
% \allsectionsfont{\bf}
%\sectionfont{\color{Tomato3}\upshape\selectfont}
%\subsectionfont{\color{Tomato4}\upshape\selectfont}

%----- Ensembles : entiers, reels, complexes -----
\newcommand{\Nn}{\mathbb{N}} \newcommand{\N}{\mathbb{N}}
\newcommand{\Zz}{\mathbb{Z}} \newcommand{\Z}{\mathbb{Z}}
\newcommand{\Qq}{\mathbb{Q}} \newcommand{\Q}{\mathbb{Q}}
\newcommand{\Rr}{\mathbb{R}} \newcommand{\R}{\mathbb{R}}
\newcommand{\Cc}{\mathbb{C}} 
\newcommand{\Kk}{\mathbb{K}} \newcommand{\K}{\mathbb{K}}

%----- Modifications de symboles -----
\renewcommand{\epsilon}{\varepsilon}
\renewcommand{\Re}{\mathop{\text{Re}}\nolimits}
\renewcommand{\Im}{\mathop{\text{Im}}\nolimits}
%\newcommand{\llbracket}{\left[\kern-0.15em\left[}
%\newcommand{\rrbracket}{\right]\kern-0.15em\right]}

\renewcommand{\ge}{\geqslant}
\renewcommand{\geq}{\geqslant}
\renewcommand{\le}{\leqslant}
\renewcommand{\leq}{\leqslant}

%----- Fonctions usuelles -----
\newcommand{\ch}{\mathop{\mathrm{ch}}\nolimits}
\newcommand{\sh}{\mathop{\mathrm{sh}}\nolimits}
\renewcommand{\tanh}{\mathop{\mathrm{th}}\nolimits}
\newcommand{\cotan}{\mathop{\mathrm{cotan}}\nolimits}
\newcommand{\Arcsin}{\mathop{\mathrm{Arcsin}}\nolimits}
\newcommand{\Arccos}{\mathop{\mathrm{Arccos}}\nolimits}
\newcommand{\Arctan}{\mathop{\mathrm{Arctan}}\nolimits}
\newcommand{\Argsh}{\mathop{\mathrm{Argsh}}\nolimits}
\newcommand{\Argch}{\mathop{\mathrm{Argch}}\nolimits}
\newcommand{\Argth}{\mathop{\mathrm{Argth}}\nolimits}
\newcommand{\pgcd}{\mathop{\mathrm{pgcd}}\nolimits} 

\newcommand{\Card}{\mathop{\text{Card}}\nolimits}
\newcommand{\Ker}{\mathop{\text{Ker}}\nolimits}
\newcommand{\id}{\mathop{\text{id}}\nolimits}
\newcommand{\ii}{\mathrm{i}}
\newcommand{\dd}{\mathrm{d}}
\newcommand{\Vect}{\mathop{\text{Vect}}\nolimits}
\newcommand{\Mat}{\mathop{\mathrm{Mat}}\nolimits}
\newcommand{\rg}{\mathop{\text{rg}}\nolimits}
\newcommand{\tr}{\mathop{\text{tr}}\nolimits}
\newcommand{\ppcm}{\mathop{\text{ppcm}}\nolimits}

%----- Structure des exercices ------

\newtheoremstyle{styleexo}% name
{2ex}% Space above
{3ex}% Space below
{}% Body font
{}% Indent amount 1
{\bfseries} % Theorem head font
{}% Punctuation after theorem head
{\newline}% Space after theorem head 2
{}% Theorem head spec (can be left empty, meaning ‘normal’)

%\theoremstyle{styleexo}
\newtheorem{exo}{Exercice}
\newtheorem{ind}{Indications}
\newtheorem{cor}{Correction}


\newcommand{\exercice}[1]{} \newcommand{\finexercice}{}
%\newcommand{\exercice}[1]{{\tiny\texttt{#1}}\vspace{-2ex}} % pour afficher le numero absolu, l'auteur...
\newcommand{\enonce}{\begin{exo}} \newcommand{\finenonce}{\end{exo}}
\newcommand{\indication}{\begin{ind}} \newcommand{\finindication}{\end{ind}}
\newcommand{\correction}{\begin{cor}} \newcommand{\fincorrection}{\end{cor}}

\newcommand{\noindication}{\stepcounter{ind}}
\newcommand{\nocorrection}{\stepcounter{cor}}

\newcommand{\fiche}[1]{} \newcommand{\finfiche}{}
\newcommand{\titre}[1]{\centerline{\large \bf #1}}
\newcommand{\addcommand}[1]{}
\newcommand{\video}[1]{}

% Marge
\newcommand{\mymargin}[1]{\marginpar{{\small #1}}}



%----- Presentation ------
\setlength{\parindent}{0cm}

%\newcommand{\ExoSept}{\href{http://exo7.emath.fr}{\textbf{\textsf{Exo7}}}}

\definecolor{myred}{rgb}{0.93,0.26,0}
\definecolor{myorange}{rgb}{0.97,0.58,0}
\definecolor{myyellow}{rgb}{1,0.86,0}

\newcommand{\LogoExoSept}[1]{  % input : echelle
{\usefont{U}{cmss}{bx}{n}
\begin{tikzpicture}[scale=0.1*#1,transform shape]
  \fill[color=myorange] (0,0)--(4,0)--(4,-4)--(0,-4)--cycle;
  \fill[color=myred] (0,0)--(0,3)--(-3,3)--(-3,0)--cycle;
  \fill[color=myyellow] (4,0)--(7,4)--(3,7)--(0,3)--cycle;
  \node[scale=5] at (3.5,3.5) {Exo7};
\end{tikzpicture}}
}



\theoremstyle{definition}
%\newtheorem{proposition}{Proposition}
%\newtheorem{exemple}{Exemple}
%\newtheorem{theoreme}{Théorème}
\newtheorem{lemme}{Lemme}
\newtheorem{corollaire}{Corollaire}
%\newtheorem*{remarque*}{Remarque}
%\newtheorem*{miniexercice}{Mini-exercices}
%\newtheorem{definition}{Définition}




%definition d'un terme
\newcommand{\defi}[1]{{\color{myorange}\textbf{\emph{#1}}}}
\newcommand{\evidence}[1]{{\color{blue}\textbf{\emph{#1}}}}



 %----- Commandes divers ------

\newcommand{\codeinline}[1]{\texttt{#1}}

%%%%%%%%%%%%%%%%%%%%%%%%%%%%%%%%%%%%%%%%%%%%%%%%%%%%%%%%%%%%%
%%%%%%%%%%%%%%%%%%%%%%%%%%%%%%%%%%%%%%%%%%%%%%%%%%%%%%%%%%%%%



\begin{document}

\debuttexte


%%%%%%%%%%%%%%%%%%%%%%%%%%%%%%%%%%%%%%%%%%%%%%%%%%%%%%%%%%%
\diapo

\change

\change

Dans cette leçon nous commençons par revoir ce qu'est un intervalle.

\change


Nous énoncerons ensuite un premier résultat de densité : 
$\Qq$ est dense dans $\Rr$ 

Ce qui signifie que tout nombre réel peut être approché d'aussi près que l'on veut par des rationnels.



%%%%%%%%%%%%%%%%%%%%%%%%%%%%%%%%%%%%%%%%%%%%%%%%%%%%%%%%%%%
\diapo

Un intervalle de $\Rr$ est une partie \og{}d'un seul bloc\fg{}, ce qui revient à dire qu'il est sans \og{}trou\fg{}.

C'est un sous-ensemble $I$ de $\Rr$ tel que si 
$I$ contient $a'$ et $b'$, l'ensemble $I$ contient aussi tout $x$ compris entre $a'$ et $b'$.


\change

Un intervalle ouvert est l'ensemble des nombres 
réels compris strictement entre deux bornes $a$ et $b$, ces bornes étant éventuellement infinies.


\change

Par convention, l'ensemble vide est un intervalle.



Il en est de même pour l'ensemble $\Rr$ tout entier.



Même si cela semble évident il faut justifier qu'un intervalle ouvert 
est un intervalle !  Je vous laisse lire, cela se vérifie facilement.



%%%%%%%%%%%%%%%%%%%%%%%%%%%%%%%%%%%%%%%%%%%%%%%%%%%%%%%%%%%
\diapo

La notion de voisinage, que nous allons introduire tout de suite, sera utile lors de l'étude des limites.
Cela généralise en quelque sorte la notion d'intervalle.


Un sous-ensemble $V$ de $\Rr$ est 
un voisinage d'un réel $a$ s'il contient un intervalle ouvert qui lui même contient $a$. 


Sur ce dessin, on voit un voisinage $V$ de $a$ ici en bleu, il est en 
composée de deux morceaux et contient 
un intervalle ouvert $I$ contenant $a$ mais $V$ n'est pas lui même un intervalle ouvert.

%%%%%%%%%%%%%%%%%%%%%%%%%%%%%%%%%%%%%%%%%%%%%%%%%%%%%%%%%%%
\diapo

Nous allons maintenant énoncer le résultat principal de ce chapitre.

\change

Ce théorème affirme que l'ensemble $\Qq$ des rationnels forme un sous-ensemble dense 
dans $\Rr$, c'est-à-dire que tout intervalle ouvert non vide contient au moins un rationnel, 
ou ce qui est équivalent, une infinité.

\change

L'ensemble $\Rr\backslash\Qq$ possède la même propriété : il est aussi dense dans $\Rr$.

\change

La démonstration va se dérouler en trois étapes.
On va d'abord montrer que l'on peut trouver *un* rationnel.

\change

On suppose que l'intervalle est l'intervalle ouvert $]a,b[$ avec $a<b$.

Il s'agit donc de construire un rationnel compris entre $a$ et $b$.

\change

D'après la propriété d'Archimède

il existe un entier $q$ tel que $q>\frac{1}{b-a}$. 

Notez que $b-a$ est strictement positif, mais peut être être petit, donc $q$ lui peut-être très grand.

\change

Posons $p=E(aq)+1$. 


\change

Alors par la définition de la partie entière nous avons
la double inégalité $p-1\leq aq<p$. 

On en déduit d'une part que $a<\frac pq$

\change

Mais d'autre part en divisant par $q$, la double inégalité implique aussi $\frac pq- \frac 1q\leq a$, 

donc $\frac pq \leq a+\frac 1q < a+b-a=b$. 

\change

$\frac pq$ est bien un rationnel et il est compris entre $a$ et $b$ comme nous le voulions.

%%%%%%%%%%%%%%%%%%%%%%%%%%%%%%%%%%%%%%%%%%%%%%%%%%%%%%%%%%%
\diapo

Passons à la deuxième partie de la démonstration

\change

Il s'agit de construire maintenant un irrationnel compris entre $a$ et $b$

\change

Nous avons vu dans la première partie que tout intervalle contient un rationnel

\change

Appliquons ce premier résultat 
à l'intervalle $]a-\sqrt{2},b-\sqrt{2}[$.Il existe donc un 
rationnel $r$ dans cet intervalle

\change

Par translation $r+\sqrt{2}$ est dans l'intervalle $]a,b[$. 

Mais comme $\sqrt 2$ est irrationnel et que $r$ est rationnel alors $r+\sqrt{2}$ est irrationnel.

On a donc montré que l'intervalle $]a,b[$ contient aussi un irrationnel.

\change


On va déduire de l'existence d'un rationnel et d'un irrationnel 
dans tout intervalle ouvert le fait qu'il existe une infinité 
de chaque dans un intervalle ouvert $]a,b[$. 

\change

En effet pour un entier $N\geq 1$, 
 on découpe notre intervalle en $N$ sous-intervalles

\change

Voici ces intervalles : il s'agit du découpage de $]a,b[$ en $N$ petits intervalles
de même longueur.

\change

Par les deux premières parties chaque sous-intervalle contient un rationnel et un irrationnel, 

\change

donc $]a,b[$ contient (au moins) $N$ rationnels et $N$ irrationnels. 

\change

Comme ceci est vrai pour tout entier $N\geq 1$, 

\change

alors $]a,b[$ contient 
une infinité de rationnels et une infinité d'irrationnels.

%%%%%%%%%%%%%%%%%%%%%%%%%%%%%%%%%%%%%%%%%%%%%%%%%%%%%%%%%%%
\diapo

Je vous laisse vous entraîner avec ces exercices, 
pour vérifier que vous avez bien compris le cours.

\end{document}
