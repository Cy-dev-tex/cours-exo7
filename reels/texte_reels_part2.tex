
%%%%%%%%%%%%%%%%%% PREAMBULE %%%%%%%%%%%%%%%%%%


\documentclass[12pt]{article}

\usepackage{amsfonts,amsmath,amssymb,amsthm}
\usepackage[utf8]{inputenc}
\usepackage[T1]{fontenc}
\usepackage[francais]{babel}


% packages
\usepackage{amsfonts,amsmath,amssymb,amsthm}
\usepackage[utf8]{inputenc}
\usepackage[T1]{fontenc}
%\usepackage{lmodern}

\usepackage[francais]{babel}
\usepackage{fancybox}
\usepackage{graphicx}

\usepackage{float}

%\usepackage[usenames, x11names]{xcolor}
\usepackage{tikz}
\usepackage{datetime}

\usepackage{mathptmx}
%\usepackage{fouriernc}
%\usepackage{newcent}
\usepackage[mathcal,mathbf]{euler}

%\usepackage{palatino}
%\usepackage{newcent}


% Commande spéciale prompteur

%\usepackage{mathptmx}
%\usepackage[mathcal,mathbf]{euler}
%\usepackage{mathpple,multido}

\usepackage[a4paper]{geometry}
\geometry{top=2cm, bottom=2cm, left=1cm, right=1cm, marginparsep=1cm}

\newcommand{\change}{{\color{red}\rule{\textwidth}{1mm}\\}}

\newcounter{mydiapo}

\newcommand{\diapo}{\newpage
\hfill {\normalsize  Diapo \themydiapo \quad \texttt{[\jobname]}} \\
\stepcounter{mydiapo}}


%%%%%%% COULEURS %%%%%%%%%%

% Pour blanc sur noir :
%\pagecolor[rgb]{0.5,0.5,0.5}
% \pagecolor[rgb]{0,0,0}
% \color[rgb]{1,1,1}



%\DeclareFixedFont{\myfont}{U}{cmss}{bx}{n}{18pt}
\newcommand{\debuttexte}{
%%%%%%%%%%%%% FONTES %%%%%%%%%%%%%
\renewcommand{\baselinestretch}{1.5}
\usefont{U}{cmss}{bx}{n}
\bfseries

% Taille normale : commenter le reste !
%Taille Arnaud
%\fontsize{19}{19}\selectfont

% Taille Barbara
%\fontsize{21}{22}\selectfont

%Taille François
%\fontsize{25}{30}\selectfont

%Taille Pascal
%\fontsize{25}{30}\selectfont

%Taille Laura
%\fontsize{30}{35}\selectfont


%\myfont
%\usefont{U}{cmss}{bx}{n}

%\Huge
%\addtolength{\parskip}{\baselineskip}
}


% \usepackage{hyperref}
% \hypersetup{colorlinks=true, linkcolor=blue, urlcolor=blue,
% pdftitle={Exo7 - Exercices de mathématiques}, pdfauthor={Exo7}}


%section
% \usepackage{sectsty}
% \allsectionsfont{\bf}
%\sectionfont{\color{Tomato3}\upshape\selectfont}
%\subsectionfont{\color{Tomato4}\upshape\selectfont}

%----- Ensembles : entiers, reels, complexes -----
\newcommand{\Nn}{\mathbb{N}} \newcommand{\N}{\mathbb{N}}
\newcommand{\Zz}{\mathbb{Z}} \newcommand{\Z}{\mathbb{Z}}
\newcommand{\Qq}{\mathbb{Q}} \newcommand{\Q}{\mathbb{Q}}
\newcommand{\Rr}{\mathbb{R}} \newcommand{\R}{\mathbb{R}}
\newcommand{\Cc}{\mathbb{C}} 
\newcommand{\Kk}{\mathbb{K}} \newcommand{\K}{\mathbb{K}}

%----- Modifications de symboles -----
\renewcommand{\epsilon}{\varepsilon}
\renewcommand{\Re}{\mathop{\text{Re}}\nolimits}
\renewcommand{\Im}{\mathop{\text{Im}}\nolimits}
%\newcommand{\llbracket}{\left[\kern-0.15em\left[}
%\newcommand{\rrbracket}{\right]\kern-0.15em\right]}

\renewcommand{\ge}{\geqslant}
\renewcommand{\geq}{\geqslant}
\renewcommand{\le}{\leqslant}
\renewcommand{\leq}{\leqslant}

%----- Fonctions usuelles -----
\newcommand{\ch}{\mathop{\mathrm{ch}}\nolimits}
\newcommand{\sh}{\mathop{\mathrm{sh}}\nolimits}
\renewcommand{\tanh}{\mathop{\mathrm{th}}\nolimits}
\newcommand{\cotan}{\mathop{\mathrm{cotan}}\nolimits}
\newcommand{\Arcsin}{\mathop{\mathrm{Arcsin}}\nolimits}
\newcommand{\Arccos}{\mathop{\mathrm{Arccos}}\nolimits}
\newcommand{\Arctan}{\mathop{\mathrm{Arctan}}\nolimits}
\newcommand{\Argsh}{\mathop{\mathrm{Argsh}}\nolimits}
\newcommand{\Argch}{\mathop{\mathrm{Argch}}\nolimits}
\newcommand{\Argth}{\mathop{\mathrm{Argth}}\nolimits}
\newcommand{\pgcd}{\mathop{\mathrm{pgcd}}\nolimits} 

\newcommand{\Card}{\mathop{\text{Card}}\nolimits}
\newcommand{\Ker}{\mathop{\text{Ker}}\nolimits}
\newcommand{\id}{\mathop{\text{id}}\nolimits}
\newcommand{\ii}{\mathrm{i}}
\newcommand{\dd}{\mathrm{d}}
\newcommand{\Vect}{\mathop{\text{Vect}}\nolimits}
\newcommand{\Mat}{\mathop{\mathrm{Mat}}\nolimits}
\newcommand{\rg}{\mathop{\text{rg}}\nolimits}
\newcommand{\tr}{\mathop{\text{tr}}\nolimits}
\newcommand{\ppcm}{\mathop{\text{ppcm}}\nolimits}

%----- Structure des exercices ------

\newtheoremstyle{styleexo}% name
{2ex}% Space above
{3ex}% Space below
{}% Body font
{}% Indent amount 1
{\bfseries} % Theorem head font
{}% Punctuation after theorem head
{\newline}% Space after theorem head 2
{}% Theorem head spec (can be left empty, meaning ‘normal’)

%\theoremstyle{styleexo}
\newtheorem{exo}{Exercice}
\newtheorem{ind}{Indications}
\newtheorem{cor}{Correction}


\newcommand{\exercice}[1]{} \newcommand{\finexercice}{}
%\newcommand{\exercice}[1]{{\tiny\texttt{#1}}\vspace{-2ex}} % pour afficher le numero absolu, l'auteur...
\newcommand{\enonce}{\begin{exo}} \newcommand{\finenonce}{\end{exo}}
\newcommand{\indication}{\begin{ind}} \newcommand{\finindication}{\end{ind}}
\newcommand{\correction}{\begin{cor}} \newcommand{\fincorrection}{\end{cor}}

\newcommand{\noindication}{\stepcounter{ind}}
\newcommand{\nocorrection}{\stepcounter{cor}}

\newcommand{\fiche}[1]{} \newcommand{\finfiche}{}
\newcommand{\titre}[1]{\centerline{\large \bf #1}}
\newcommand{\addcommand}[1]{}
\newcommand{\video}[1]{}

% Marge
\newcommand{\mymargin}[1]{\marginpar{{\small #1}}}



%----- Presentation ------
\setlength{\parindent}{0cm}

%\newcommand{\ExoSept}{\href{http://exo7.emath.fr}{\textbf{\textsf{Exo7}}}}

\definecolor{myred}{rgb}{0.93,0.26,0}
\definecolor{myorange}{rgb}{0.97,0.58,0}
\definecolor{myyellow}{rgb}{1,0.86,0}

\newcommand{\LogoExoSept}[1]{  % input : echelle
{\usefont{U}{cmss}{bx}{n}
\begin{tikzpicture}[scale=0.1*#1,transform shape]
  \fill[color=myorange] (0,0)--(4,0)--(4,-4)--(0,-4)--cycle;
  \fill[color=myred] (0,0)--(0,3)--(-3,3)--(-3,0)--cycle;
  \fill[color=myyellow] (4,0)--(7,4)--(3,7)--(0,3)--cycle;
  \node[scale=5] at (3.5,3.5) {Exo7};
\end{tikzpicture}}
}



\theoremstyle{definition}
%\newtheorem{proposition}{Proposition}
%\newtheorem{exemple}{Exemple}
%\newtheorem{theoreme}{Théorème}
\newtheorem{lemme}{Lemme}
\newtheorem{corollaire}{Corollaire}
%\newtheorem*{remarque*}{Remarque}
%\newtheorem*{miniexercice}{Mini-exercices}
%\newtheorem{definition}{Définition}




%definition d'un terme
\newcommand{\defi}[1]{{\color{myorange}\textbf{\emph{#1}}}}
\newcommand{\evidence}[1]{{\color{blue}\textbf{\emph{#1}}}}



 %----- Commandes divers ------

\newcommand{\codeinline}[1]{\texttt{#1}}

%%%%%%%%%%%%%%%%%%%%%%%%%%%%%%%%%%%%%%%%%%%%%%%%%%%%%%%%%%%%%
%%%%%%%%%%%%%%%%%%%%%%%%%%%%%%%%%%%%%%%%%%%%%%%%%%%%%%%%%%%%%



\begin{document}

\debuttexte


%%%%%%%%%%%%%%%%%%%%%%%%%%%%%%%%%%%%%%%%%%%%%%%%%%%%%%%%%%%
\diapo

\change

Dans cette leçon, nous commençons par rappeler 

\change

les principales propriétés de l'ensemble des nombres réels.

\change

\change

Puis nous énonçons la propriété d'Archimède, qui affirme que tout réel admet un entier qui lui est supérieur.

\change

Et enfin nous nous intéresserons à la valeur absolue.


%%%%%%%%%%%%%%%%%%%%%%%%%%%%%%%%%%%%%%%%%%%%%%%%%%%%%%%%%%%
\diapo

L'ensemble $\Rr$ des nombres réels, muni des la loi d'addition et de la loi de multiplication, 
est un corps commutatif.

\change

Ceci signifie que ces lois vérifient les propriétés rappelées ci-dessous : 

par exemple l'addition est commutative $a+b=b+a$.

L'addition est associative $(a+b)+c=a+(b+c)$, donc lorsqu'il n'y a que des additions on peut omettre les parenthèses.

De même la multiplication est commutative : $a*b=b*a$.
et elle est associative.


Chaque élément non nul admet un inverse 

Pour l'addition tout élément admet un opposé.

\change

Ces deux lois sont de plus compatibles entre elles, 
au sens que la multiplication est distributive par rapport à l'addition.

Enfin une propriété qui sert souvent sans que l'on s'en rende compte.
Un produit est nul si et seulement si un des facteurs est nul.

%%%%%%%%%%%%%%%%%%%%%%%%%%%%%%%%%%%%%%%%%%%%%%%%%%%%%%%%%%%
\diapo

L'ensemble $\Rr$ est muni d'une structure supplémentaire : 
c'est un ensemble ordonné. La définition abstraite est la suivante.

Vous pouvez garder à l'esprit l'exemple où $E$ est l'ensemble des réels 
et la relation $\mathcal{R}$ est la relation "être inférieur ou égal".

Une relation $\mathcal R$ sur un ensemble $E$ est une partie de 
$E\times E$ l'ensemble des couples d'éléments de $E$.

\change

 Si un couple $(x,y)$ est dans $\mathcal R$, on note $x\mathcal R y$.
 
 et on dit que $x$ et $y$ sont en relation.

\change

Une relation $\mathcal R$ sur un ensemble $E$ est une relation d'ordre si 
elle vérifie les trois points suivants.

(1) elle est réflexive : tout élément est en relation avec lui-même, 

(2) elle est antisymétrique, c'est à dire que $x$ est en relation avec $y$  et $y$ en relation avec $x$ alors cela 
entraîne que $x=y$,

(3) enfin elle est transitive, 
c'est-à-dire si $x\mathcal R y$ et $y\mathcal R z$ alors $x\mathcal R z$.

\change


Une relation $\mathcal R$ sur un ensemble $E$ est une relation d'ordre totale 

si pour tout $x,y\in E$ on a $ x\mathcal R y$ ou bien $y\mathcal R x$.
Cela signifie que l'on peut toujours comparer deux éléments.



%%%%%%%%%%%%%%%%%%%%%%%%%%%%%%%%%%%%%%%%%%%%%%%%%%%%%%%%%%%
\diapo

L'ensemble $\Rr$ possède cette propriété d'être muni 
d'une relation d'ordre naturelle "inférieur ou égal" et cette relation d'ordre est même totale.

\change

On rappelle que ceci signifie en particulier que cette relation d'ordre est 

réflexive, en effet pour tout $x$ $x \le x$

antisymétrique   si $x \le y$ et $y\le x$ alors on bien $x=y$

transitive $x\le y$ et $y\le z$ alors $x\le z$

\change

Cette relation d'ordre "inférieur ou égal" possède en outre les propriétés bien connues 
par rapport à l'addition et à la multiplication : 

on peut additionner 
deux inégalités, ou multiplier une inégalité par un réel positif. 

Faites bien attention : Si on multiplie une inégalité par un réel négatif, 
alors il faut en changer le sens de l'inégalité.

%%%%%%%%%%%%%%%%%%%%%%%%%%%%%%%%%%%%%%%%%%%%%%%%%%%%%%%%%%%
\diapo

Nous abordons à présent la principale propriété de cette section : 

l'ensemble $\Rr$ est archimédien, c'est-à-dire que pour chaque réel $x$ on peut toujours 
lui trouver un entier $n$ strictement plus grand.

Cette propriété peut paraître comme évidente, mais elle est fondamentale.

\change

C'est en fait cette propriété qui permet de définir la partie entière.

\change

Pour tout réel $x$, il existe un unique entier $E(x)$ tel que 
$x$ soit compris entre $E(x)$, au sens large, et $E(x)+1$, au sens strict.

Retenez bien que $E(x)$ est un entier et qu'il vérifie cette double inégalité,
large à gauche et stricte à droite.

\change

Par exemple la partie entière de $2,853$ est $2$, 

\change

celle de $\pi$ est $3$, 

\change

La partie entière est l'entier le plus proche en dessous du réel $x$,
donc faites attention pour les nombres négatifs :

la partie entière de  $-3,5$ est bien $-4$.

\change

Autre exemple : la partie entière d'un nombre $x$ est $3$ si et seulement si $x$ 
est compris entre $3$, au sens large, et $4$, au sens strict.

%%%%%%%%%%%%%%%%%%%%%%%%%%%%%%%%%%%%%%%%%%%%%%%%%%%%%%%%%%%
\diapo

Voici le graphe de la fonction partie entière (à connaître !) :

à $x$ on associe partie entière de $x$.

\change

On peut, grâce à une lecture graphique, retrouver que la partie entière de $2,853$ est $2$.

%%%%%%%%%%%%%%%%%%%%%%%%%%%%%%%%%%%%%%%%%%%%%%%%%%%%%%%%%%%

\diapo

Voici comment on peut calculer la partie entière de $\sqrt{10}$.

\change

$3^2 = 9 < 10$  

\change

La fonction racine carrée est une fonction strictement croissante
donc en prenant les racines on obtient $3=\sqrt{3^2} < \sqrt{10}$.


\change

$4^2=16 > 10$ 

\change

donc en prenant les racines $4=\sqrt{4^2} > \sqrt{10}$.

\change

En conclusion $3 < \sqrt{10} < 4$ donc $E\big(\sqrt{10}\big)=3$.

\change

Revenons sur la démonstration de l'existence et l'unicité de la 
partie entière d'un nombre réel $x$.

Commençons par l'existence. On fixe $x\geq 0$.
On cherche un entier $E(x)$  tel que $E(x)\leq x <E(x)+1$.

\change 

Par la propriété d'Archimède il existe entier $n$ strictement plus grand que $x$.

\change 

[grand K]

Notons  grand $K$ l'ensemble des entiers petit $k$ qui sont plus petit que $x$.
Cet ensemble est donc fini. 

\change

Notons $k_{max}=\max K$ son plus grand élément.

\change 

On a d'une part $k_{max}\leq x$ car $k_{max}$ est un élément de grand $K$. 

\change

Et d'autre part $k_{max}+1 > x$ car $k_{max}+1$ n'appartient pas à $K$
(c'est $k_{max}$ le plus grand élément de $K$). 

\change 

On vient de prouver que $k_{max}\leq x< k_{max}+1$, 

on peut prendre $E(x)=k_{max}$, qui convient.

\change 

Montrons à présent l'unicité.


 Soient $k$ et $\ell$ sont deux entiers vérifiant la propriété de la partie entière, 
 c'est-à-dire  $k\leq x< k+1$ et aussi $\ell\leq x< \ell+1$.

 \change 
 
 On en déduit d'abord que $k \leq x < \ell+1$, en particulier $k < \ell+1$.

 \change

 On a aussi par symétrie $\ell<k+1$.

 \change

 Et donc en conclusion $\ell-1<k<\ell+1$. 
 
 \change
 
 Mais il n'y a qu'un seul entier compris strictement entre $\ell-1$
et $\ell+1$, c'est $\ell$. Ainsi $k=\ell$. 


%%%%%%%%%%%%%%%%%%%%%%%%%%%%%%%%%%%%%%%%%%%%%%%%%%%%%%%%%%%
\diapo

Venons en à présent à la définition de la valeur absolue d'un réel.

La valeur absolue de $x$ est $x$ lui-même si $x\geq 0$, 


par contre si $x$ est négatif alors la valeur absolue de $x$ est $-x$.

(comme $x$ est négatif alors $-x$ est positif).

Donc quelque soit le signe de $x$, valeur absolue de $x$ est positif ou nul.


\change

Voici le graphe de la fonction valeur absolue. 

C'est une fonction paire, son graphe est symétrique par rapport à l'axe des ordonnées.


%%%%%%%%%%%%%%%%%%%%%%%%%%%%%%%%%%%%%%%%%%%%%%%%%%%%%%%%%%%
\diapo

Voici quelques propriétés de la valeur absolue. 

On l'a déjà noté : la valeur absolue d'un nombre $x$ est toujours positive ou nulle. 

Elle est même strictement positive si $x\neq 0$.

La valeur absolue du réel $x$ est égale à celle de son opposé $-x$.

\change

La valeur absolue de $x$ peut s'exprimer aussi comme la racine carrée du carré de $x$.

\change

La valeur absolue d'un produit est le produit des valeurs absolues.

\change

L'inégalité triangulaire  est très importante : 

$|x+y|\leq |x|+|y|$

En d'autres termes la valeur absolue d'une somme est plus petite que 
la somme des valeurs absolues.

\change 

On en déduit la formule suivante sur la valeur absolue d'une différence, 
qui est appelée seconde inégalité triangulaire :

$\big||x|-|y|\big|\leq |x-y|$


\change

Enfin sur la droite numérique, $|x-y|$ représente la distance entre les
réels $x$ et $y$ ;


en particulier $|x|$ représente la distance entre les réels $x$ et $0$.


%%%%%%%%%%%%%%%%%%%%%%%%%%%%%%%%%%%%%%%%%%%%%%%%%%%%%%%%%%%
\diapo

Nous allons prouver l'inégalité triangulaire, pour deux réels $x$ et $y$.

\change

Par définition, on a les inégalités :
$-|x|\leq x \leq |x|$ \quad et \quad $-|y|\leq y \leq |y|$.

\change

En additionnant, on en déduit que 
$-\left(|x|+|y|\right)\leq x+y\leq |x|+|y|$.

\change

Et ceci montre que $|x+y|\leq |x|+|y|$.

\change

Voici une caractérisation importante de la valeur absolue :
$|x-a|<r$ si et seulement si $a-r< x <a+r$.

\change

Autrement dit $|x-a|<r$ si et seulement si $x$ est dans l'intervalle $]a-r,a+r[$.

\change

Graphiquement, sur la droite numérique, cette condition signifie 
que la distance de $x$ à $a$ est strictement plus petite que $r$.



%%%%%%%%%%%%%%%%%%%%%%%%%%%%%%%%%%%%%%%%%%%%%%%%%%%%%%%%%%%
\diapo

Entraînez avec ces exercices, 
pour vérifier que vous avez bien compris le cours.

\end{document}
