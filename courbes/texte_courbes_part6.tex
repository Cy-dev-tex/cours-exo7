
%%%%%%%%%%%%%%%%%% PREAMBULE %%%%%%%%%%%%%%%%%%


\documentclass[12pt]{article}

\usepackage{amsfonts,amsmath,amssymb,amsthm}
\usepackage[utf8]{inputenc}
\usepackage[T1]{fontenc}
\usepackage[francais]{babel}


% packages
\usepackage{amsfonts,amsmath,amssymb,amsthm}
\usepackage[utf8]{inputenc}
\usepackage[T1]{fontenc}
%\usepackage{lmodern}

\usepackage[francais]{babel}
\usepackage{fancybox}
\usepackage{graphicx}

\usepackage{float}

%\usepackage[usenames, x11names]{xcolor}
\usepackage{tikz}
\usepackage{datetime}

\usepackage{mathptmx}
%\usepackage{fouriernc}
%\usepackage{newcent}
\usepackage[mathcal,mathbf]{euler}

%\usepackage{palatino}
%\usepackage{newcent}


% Commande spéciale prompteur

%\usepackage{mathptmx}
%\usepackage[mathcal,mathbf]{euler}
%\usepackage{mathpple,multido}

\usepackage[a4paper]{geometry}
\geometry{top=2cm, bottom=2cm, left=1cm, right=1cm, marginparsep=1cm}

\newcommand{\change}{{\color{red}\rule{\textwidth}{1mm}\\}}

\newcounter{mydiapo}

\newcommand{\diapo}{\newpage
\hfill {\normalsize  Diapo \themydiapo \quad \texttt{[\jobname]}} \\
\stepcounter{mydiapo}}


%%%%%%% COULEURS %%%%%%%%%%

% Pour blanc sur noir :
%\pagecolor[rgb]{0.5,0.5,0.5}
% \pagecolor[rgb]{0,0,0}
% \color[rgb]{1,1,1}



%\DeclareFixedFont{\myfont}{U}{cmss}{bx}{n}{18pt}
\newcommand{\debuttexte}{
%%%%%%%%%%%%% FONTES %%%%%%%%%%%%%
\renewcommand{\baselinestretch}{1.5}
\usefont{U}{cmss}{bx}{n}
\bfseries

% Taille normale : commenter le reste !
%Taille Arnaud
%\fontsize{19}{19}\selectfont

% Taille Barbara
%\fontsize{21}{22}\selectfont

%Taille François
%\fontsize{25}{30}\selectfont

%Taille Pascal
%\fontsize{25}{30}\selectfont

%Taille Laura
%\fontsize{30}{35}\selectfont


%\myfont
%\usefont{U}{cmss}{bx}{n}

%\Huge
%\addtolength{\parskip}{\baselineskip}
}


% \usepackage{hyperref}
% \hypersetup{colorlinks=true, linkcolor=blue, urlcolor=blue,
% pdftitle={Exo7 - Exercices de mathématiques}, pdfauthor={Exo7}}


%section
% \usepackage{sectsty}
% \allsectionsfont{\bf}
%\sectionfont{\color{Tomato3}\upshape\selectfont}
%\subsectionfont{\color{Tomato4}\upshape\selectfont}

%----- Ensembles : entiers, reels, complexes -----
\newcommand{\Nn}{\mathbb{N}} \newcommand{\N}{\mathbb{N}}
\newcommand{\Zz}{\mathbb{Z}} \newcommand{\Z}{\mathbb{Z}}
\newcommand{\Qq}{\mathbb{Q}} \newcommand{\Q}{\mathbb{Q}}
\newcommand{\Rr}{\mathbb{R}} \newcommand{\R}{\mathbb{R}}
\newcommand{\Cc}{\mathbb{C}} 
\newcommand{\Kk}{\mathbb{K}} \newcommand{\K}{\mathbb{K}}

%----- Modifications de symboles -----
\renewcommand{\epsilon}{\varepsilon}
\renewcommand{\Re}{\mathop{\text{Re}}\nolimits}
\renewcommand{\Im}{\mathop{\text{Im}}\nolimits}
%\newcommand{\llbracket}{\left[\kern-0.15em\left[}
%\newcommand{\rrbracket}{\right]\kern-0.15em\right]}

\renewcommand{\ge}{\geqslant}
\renewcommand{\geq}{\geqslant}
\renewcommand{\le}{\leqslant}
\renewcommand{\leq}{\leqslant}

%----- Fonctions usuelles -----
\newcommand{\ch}{\mathop{\mathrm{ch}}\nolimits}
\newcommand{\sh}{\mathop{\mathrm{sh}}\nolimits}
\renewcommand{\tanh}{\mathop{\mathrm{th}}\nolimits}
\newcommand{\cotan}{\mathop{\mathrm{cotan}}\nolimits}
\newcommand{\Arcsin}{\mathop{\mathrm{Arcsin}}\nolimits}
\newcommand{\Arccos}{\mathop{\mathrm{Arccos}}\nolimits}
\newcommand{\Arctan}{\mathop{\mathrm{Arctan}}\nolimits}
\newcommand{\Argsh}{\mathop{\mathrm{Argsh}}\nolimits}
\newcommand{\Argch}{\mathop{\mathrm{Argch}}\nolimits}
\newcommand{\Argth}{\mathop{\mathrm{Argth}}\nolimits}
\newcommand{\pgcd}{\mathop{\mathrm{pgcd}}\nolimits} 

\newcommand{\Card}{\mathop{\text{Card}}\nolimits}
\newcommand{\Ker}{\mathop{\text{Ker}}\nolimits}
\newcommand{\id}{\mathop{\text{id}}\nolimits}
\newcommand{\ii}{\mathrm{i}}
\newcommand{\dd}{\mathrm{d}}
\newcommand{\Vect}{\mathop{\text{Vect}}\nolimits}
\newcommand{\Mat}{\mathop{\mathrm{Mat}}\nolimits}
\newcommand{\rg}{\mathop{\text{rg}}\nolimits}
\newcommand{\tr}{\mathop{\text{tr}}\nolimits}
\newcommand{\ppcm}{\mathop{\text{ppcm}}\nolimits}

%----- Structure des exercices ------

\newtheoremstyle{styleexo}% name
{2ex}% Space above
{3ex}% Space below
{}% Body font
{}% Indent amount 1
{\bfseries} % Theorem head font
{}% Punctuation after theorem head
{\newline}% Space after theorem head 2
{}% Theorem head spec (can be left empty, meaning ‘normal’)

%\theoremstyle{styleexo}
\newtheorem{exo}{Exercice}
\newtheorem{ind}{Indications}
\newtheorem{cor}{Correction}


\newcommand{\exercice}[1]{} \newcommand{\finexercice}{}
%\newcommand{\exercice}[1]{{\tiny\texttt{#1}}\vspace{-2ex}} % pour afficher le numero absolu, l'auteur...
\newcommand{\enonce}{\begin{exo}} \newcommand{\finenonce}{\end{exo}}
\newcommand{\indication}{\begin{ind}} \newcommand{\finindication}{\end{ind}}
\newcommand{\correction}{\begin{cor}} \newcommand{\fincorrection}{\end{cor}}

\newcommand{\noindication}{\stepcounter{ind}}
\newcommand{\nocorrection}{\stepcounter{cor}}

\newcommand{\fiche}[1]{} \newcommand{\finfiche}{}
\newcommand{\titre}[1]{\centerline{\large \bf #1}}
\newcommand{\addcommand}[1]{}
\newcommand{\video}[1]{}

% Marge
\newcommand{\mymargin}[1]{\marginpar{{\small #1}}}



%----- Presentation ------
\setlength{\parindent}{0cm}

%\newcommand{\ExoSept}{\href{http://exo7.emath.fr}{\textbf{\textsf{Exo7}}}}

\definecolor{myred}{rgb}{0.93,0.26,0}
\definecolor{myorange}{rgb}{0.97,0.58,0}
\definecolor{myyellow}{rgb}{1,0.86,0}

\newcommand{\LogoExoSept}[1]{  % input : echelle
{\usefont{U}{cmss}{bx}{n}
\begin{tikzpicture}[scale=0.1*#1,transform shape]
  \fill[color=myorange] (0,0)--(4,0)--(4,-4)--(0,-4)--cycle;
  \fill[color=myred] (0,0)--(0,3)--(-3,3)--(-3,0)--cycle;
  \fill[color=myyellow] (4,0)--(7,4)--(3,7)--(0,3)--cycle;
  \node[scale=5] at (3.5,3.5) {Exo7};
\end{tikzpicture}}
}



\theoremstyle{definition}
%\newtheorem{proposition}{Proposition}
%\newtheorem{exemple}{Exemple}
%\newtheorem{theoreme}{Théorème}
\newtheorem{lemme}{Lemme}
\newtheorem{corollaire}{Corollaire}
%\newtheorem*{remarque*}{Remarque}
%\newtheorem*{miniexercice}{Mini-exercices}
%\newtheorem{definition}{Définition}




%definition d'un terme
\newcommand{\defi}[1]{{\color{myorange}\textbf{\emph{#1}}}}
\newcommand{\evidence}[1]{{\color{blue}\textbf{\emph{#1}}}}



 %----- Commandes divers ------

\newcommand{\codeinline}[1]{\texttt{#1}}

%%%%%%%%%%%%%%%%%%%%%%%%%%%%%%%%%%%%%%%%%%%%%%%%%%%%%%%%%%%%%
%%%%%%%%%%%%%%%%%%%%%%%%%%%%%%%%%%%%%%%%%%%%%%%%%%%%%%%%%%%%%


\begin{document}

\debuttexte


%%%%%%%%%%%%%%%%%%%%%%%%%%%%%%%%%%%%%%%%%%%%%%%%%%%%%%%%%%%
\diapo


Dans cette dernière partie du chapitre sur les courbes paramétrées, 
nous allons regarder de près quelques exemples 
de courbes données par une équation polaire.

\change

Voici le plan de cette partie.

\change

Nous commencerons par réduire le domaine d'étude de la courbe,

\change 
puis nous donnerons le plan d'étude,

\change 
et nous terminerons par un exemple en détails : la cardioïde.

%%%%%%%%%%%%%%%%%%%%%%%%%%%%%%%%%%%%%%%%%%%%%%%%%%%%%%%%%%%
\diapo

On commence par expliquer comment réduire le domaine d'étude. 
Pour cela, on doit connaître l'effet de transformations géométriques 
usuelles sur les coordonnées polaires d'un point.  
Le plan est rapporté à un repère orthonormé direct, $M$ étant le point de coordonnées polaires $[r:\theta]$.

La réflexion d'axe $(Ox)$ envoie le point de coordonnées polaires 
$[r:\theta]$ sur le point de coordonnées polaires $[r:-\theta]$.
 
\change


La réflexion d'axe $(Oy)$ enverrait le point $[r:\theta]$ 
sur le point $[r:\pi -\theta]$.
 
\change


La symétrie centrale de centre $O$ envoie le point de 
coordonnées polaires $[r:\theta]$  sur le point de 
coordonnées polaires $[r:\theta+\pi]$, qui est aussi le point de coordonnées polaires $[-r:\theta]$.



%%%%%%%%%%%%%%%%%%%%%%%%%%%%%%%%%%%%%%%%%%%%%%%%%%%%%%%%%%%
\diapo


On continue avec d'autres transformations utiles à la réduction du domaine d'étude.

La réflexion d'axe la droite $D$ d'équation $(y=x)$ envoie 
le point de coordonnées polaires $[r:\theta]$ sur le point de coordonnées polaires $[r:\frac{\pi}{2}-\theta]$.
 
\change
La rotation d'angle $\frac{\pi}{2}$ autour de $O$ envoie 
le point de coordonnées polaires $[r:\theta]$ sur le point 
de coordonnées polaires $[r:\theta+\frac{\pi}{2}]$.

Et plus généralement la rotation la rotation d'angle $\varphi$ 
envoie $[r:\theta]$ sur le point $[r:\theta+\varphi]$.




%%%%%%%%%%%%%%%%%%%%%%%%%%%%%%%%%%%%%%%%%%%%%%%%%%%%%%%%%%%
\diapo

Voyons comment réduire le domaine d'étude sur l'exemple de la 
courbe d'équation polaire 
$$r=1+2\cos^2\theta.$$

\change

On commence par observer que 
La fonction $r$ est $2\pi$-périodique. 

\change
$$M(\theta+2\pi)=\big[r(\theta+2\pi):\theta+2\pi\big]=\big[r(\theta):\theta\big]=M(\theta).$$

\change
La courbe complète est donc obtenue quand $\theta$ décrit un 
intervalle de longueur $2\pi$ comme $[-\pi,\pi]$. 

\change
La fonction $r$ est paire. 

\change
Donc, 
$$M(-\theta)=\big[r(-\theta):-\theta\big]=\big[r(\theta):-\theta\big]=s_{(Ox)}\big(M(\theta)\big).$$

\change
Il suffit donc d'étudier et construire la courbe sur $[0,\pi]$, 
puis de faire une réflexion d'axe $(Ox)$.


\change
Enfin 
$r(\pi-\theta)=r(\theta)$. 


\change
Ainsi 
$$M(\pi-\theta)=\big[r(\pi-\theta):\pi-\theta\big]=\big[r(\theta):\pi-\theta\big]=s_{(Oy)}\big(M(\theta)\big).$$

\change
On étudie et construit la courbe seulement su l'intervalle $[0,\frac{\pi}{2}]$, 
puis on obtiendra la courbe sur $[0,\pi]$ par réflexion d'axe $(Oy)$. 


\change
Voici la courbe sur l'intervalle $[0,\frac{\pi}{2}]$,

\change
après la symétrie verticale on la courbe sur $[0,\pi]$.

\change
Et après la symétrie horizontale on a la courbe sur $[-\pi,\pi]$, donc sur $\Rr$.


%%%%%%%%%%%%%%%%%%%%%%%%%%%%%%%%%%%%%%%%%%%%%%%%%%%%%%%%%%%
\diapo

Voici un plan générique d'étude d'une courbe donnée par une équation en coordonnées polaires.
  
On commence par  déterminer le \evidence{domaine de définition} et réduire le \evidence{domaine d'étude} 
en détaillant à chaque fois les transformations géométriques permettant de reconstituer la courbe.

\change
 \evidence{Puis on détermine les passages par l'origine.} On résout l'équation $r(\theta)=0$ 
et on précise les tangentes en ces points.

\change
On passe ensuite à l'étude des \evidence{variations} de la fonction $r$ ainsi que du \evidence{signe} de 
la fonction $r$. 
Ce signe a une influence sur le tracé 
de la courbe. 

\change
On détermine les \evidence{tangentes parallèles aux axes.} Par exemple 
pour trouver les tangentes horizontales
il s'agit de résoudre $y'=0$, ce qui s'écrit ici $\big(r\sin(\theta)\big)'=0$.


\change
On procède ensuite à \evidence{l'étude des branches infinies.}  
Le plus simple est de se ramener à l'étude des branches 
infinies d'une courbe paramétrée classique. 

\change
On termine par la \evidence{construction de la courbe,}

\change
et éventuellement 
par la recherche de \evidence{points multiples} 
si le tracé de la courbe le suggère. 

%%%%%%%%%%%%%%%%%%%%%%%%%%%%%%%%%%%%%%%%%%%%%%%%%%%%%%%%%%%
\diapo

Appliquons ce plan d'étude à la \defi{cardioïde}, 
courbe dont l'équation polaire 
$$r=1-\cos\theta.$$


\change
  On commence par déterminer le domaine d'étude. 
  La fonction $r$ est $2\pi$-périodique, 
  donc on l'étudie sur $[-\pi,\pi]$, mais 
  comme $r$ est de plus une fonction paire, on se 
  limite à l'intervalle $[0,\pi]$, la courbe étant 
  symétrique par rapport à l'axe des abscisses.

\change
  
   On peut facilement localiser la courbe. 
   Comme $0 \le r \le 2$, la courbe est bornée, 
   incluse dans le disque de rayon $2$, centré à l'origine. 
   Il n'y a pas de branches infinies.

\change

 Étudions les passages par l'origine. 
 On résout  $r(\theta)=0$, ce qui équivaut à  
 $\cos \theta  = 1$ ou encore $\theta = 0$ (toujours avec 
  notre restriction $\theta \in [0,\pi]$). 
  La courbe passe par l'origine uniquement pour $\theta = 0$.

\change
Étudions à présent les variations de $r$. 
La fonction $r$ est croissante sur $[0,\pi]$ 
avec $r(0)=0$, $r(\pi) = 2$. Conséquence : $r$ est positif 
et croît, on tourne dans le sens direct en s'écartant de l'origine.
 
\change
 Recherchons à présent les tangentes parallèles aux axes.

 \change
Pour cela on utilise la représentation paramétrique de la courbe.
On calcule
  $$x(\theta) = r \cos \theta = \cos \theta - \cos^2\theta $$ 

 \change  
  puis
 $$ x'(\theta) = \sin\theta ( 2\cos\theta -1),$$ 
 
 
 \change
 
 et on en déduit que
  $$x'(\theta) = 0 \iff \theta = 0, \quad \theta = \frac\pi3, \quad \theta = \pi$$
 
 \change
 
 \change
 
 \change
 
 \change 
 Je vous laisse  vérifier les calculs pour trouver 
 les points où $y'(\theta)=0$.
 
 Bilan :  en $\theta = \frac\pi3$ et $\theta = \pi$, $x'$ s'annule mais pas $y'$ : 
 la tangente est verticale.
 
 en $\theta = \frac{2\pi}{3}$, $y'$ s'annule mais pas $x'$ : 
 la tangente est horizontale.

 En $\theta=0$ c'est un point singulier.
 
 %%%%%%%%%%%%%%%%%%%%%%%%%%%%%%%%%%%%%%%%%%%%%%%%%%%%%%%%%%%
\diapo

On passe ensuite au \textbf{comportement de la courbe à l'origine.} 

\change
La courbe passe à l'origine   
pour $\theta_0=0$,

  \change
la tangente y est donc 
la droite d'équation polaire $\theta=0$, 
c'est-à-dire l'axe des abscisses.  

  \change
Comme $r(\theta)\ge0$, la courbe ne franchit pas l'origine, 
il s'agit d'un point de rebroussement.

  \change

  On peut alors passer au tracé du graphe.
  
  \change
  
  On commence par placer les points correspondant aux paramètres
  $\theta = 0 , \pi/3, 2\pi/3$ et $\pi$.
  
  \change
  Puis les tangentes en ces points.
  
  \change
  On se souvient que le rayon $r$ est croissant de $0$ à $2$,
  ce qui permet de tracer la courbe.
   
  \change
  On termine par la symétrie.
 







%%%%%%%%%%%%%%%%%%%%%%%%%%%%%%%%%%%%%%%%%%%%%%%%%%%%%%%%%%%
\diapo

Voici pour finir quelques exercices pour vous 
entrainer à l'étude et au tracé des courbes en coordonnées polaires.


\end{document}
