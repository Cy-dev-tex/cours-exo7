
%%%%%%%%%%%%%%%%%% PREAMBULE %%%%%%%%%%%%%%%%%%


\documentclass[12pt]{article}

\usepackage{amsfonts,amsmath,amssymb,amsthm}
\usepackage[utf8]{inputenc}
\usepackage[T1]{fontenc}
\usepackage[francais]{babel}


% packages
\usepackage{amsfonts,amsmath,amssymb,amsthm}
\usepackage[utf8]{inputenc}
\usepackage[T1]{fontenc}
%\usepackage{lmodern}

\usepackage[francais]{babel}
\usepackage{fancybox}
\usepackage{graphicx}

\usepackage{float}

%\usepackage[usenames, x11names]{xcolor}
\usepackage{tikz}
\usepackage{datetime}

\usepackage{mathptmx}
%\usepackage{fouriernc}
%\usepackage{newcent}
\usepackage[mathcal,mathbf]{euler}

%\usepackage{palatino}
%\usepackage{newcent}


% Commande spéciale prompteur

%\usepackage{mathptmx}
%\usepackage[mathcal,mathbf]{euler}
%\usepackage{mathpple,multido}

\usepackage[a4paper]{geometry}
\geometry{top=2cm, bottom=2cm, left=1cm, right=1cm, marginparsep=1cm}

\newcommand{\change}{{\color{red}\rule{\textwidth}{1mm}\\}}

\newcounter{mydiapo}

\newcommand{\diapo}{\newpage
\hfill {\normalsize  Diapo \themydiapo \quad \texttt{[\jobname]}} \\
\stepcounter{mydiapo}}


%%%%%%% COULEURS %%%%%%%%%%

% Pour blanc sur noir :
%\pagecolor[rgb]{0.5,0.5,0.5}
% \pagecolor[rgb]{0,0,0}
% \color[rgb]{1,1,1}



%\DeclareFixedFont{\myfont}{U}{cmss}{bx}{n}{18pt}
\newcommand{\debuttexte}{
%%%%%%%%%%%%% FONTES %%%%%%%%%%%%%
\renewcommand{\baselinestretch}{1.5}
\usefont{U}{cmss}{bx}{n}
\bfseries

% Taille normale : commenter le reste !
%Taille Arnaud
%\fontsize{19}{19}\selectfont

% Taille Barbara
%\fontsize{21}{22}\selectfont

%Taille François
%\fontsize{25}{30}\selectfont

%Taille Pascal
%\fontsize{25}{30}\selectfont

%Taille Laura
%\fontsize{30}{35}\selectfont


%\myfont
%\usefont{U}{cmss}{bx}{n}

%\Huge
%\addtolength{\parskip}{\baselineskip}
}


% \usepackage{hyperref}
% \hypersetup{colorlinks=true, linkcolor=blue, urlcolor=blue,
% pdftitle={Exo7 - Exercices de mathématiques}, pdfauthor={Exo7}}


%section
% \usepackage{sectsty}
% \allsectionsfont{\bf}
%\sectionfont{\color{Tomato3}\upshape\selectfont}
%\subsectionfont{\color{Tomato4}\upshape\selectfont}

%----- Ensembles : entiers, reels, complexes -----
\newcommand{\Nn}{\mathbb{N}} \newcommand{\N}{\mathbb{N}}
\newcommand{\Zz}{\mathbb{Z}} \newcommand{\Z}{\mathbb{Z}}
\newcommand{\Qq}{\mathbb{Q}} \newcommand{\Q}{\mathbb{Q}}
\newcommand{\Rr}{\mathbb{R}} \newcommand{\R}{\mathbb{R}}
\newcommand{\Cc}{\mathbb{C}} 
\newcommand{\Kk}{\mathbb{K}} \newcommand{\K}{\mathbb{K}}

%----- Modifications de symboles -----
\renewcommand{\epsilon}{\varepsilon}
\renewcommand{\Re}{\mathop{\text{Re}}\nolimits}
\renewcommand{\Im}{\mathop{\text{Im}}\nolimits}
%\newcommand{\llbracket}{\left[\kern-0.15em\left[}
%\newcommand{\rrbracket}{\right]\kern-0.15em\right]}

\renewcommand{\ge}{\geqslant}
\renewcommand{\geq}{\geqslant}
\renewcommand{\le}{\leqslant}
\renewcommand{\leq}{\leqslant}

%----- Fonctions usuelles -----
\newcommand{\ch}{\mathop{\mathrm{ch}}\nolimits}
\newcommand{\sh}{\mathop{\mathrm{sh}}\nolimits}
\renewcommand{\tanh}{\mathop{\mathrm{th}}\nolimits}
\newcommand{\cotan}{\mathop{\mathrm{cotan}}\nolimits}
\newcommand{\Arcsin}{\mathop{\mathrm{Arcsin}}\nolimits}
\newcommand{\Arccos}{\mathop{\mathrm{Arccos}}\nolimits}
\newcommand{\Arctan}{\mathop{\mathrm{Arctan}}\nolimits}
\newcommand{\Argsh}{\mathop{\mathrm{Argsh}}\nolimits}
\newcommand{\Argch}{\mathop{\mathrm{Argch}}\nolimits}
\newcommand{\Argth}{\mathop{\mathrm{Argth}}\nolimits}
\newcommand{\pgcd}{\mathop{\mathrm{pgcd}}\nolimits} 

\newcommand{\Card}{\mathop{\text{Card}}\nolimits}
\newcommand{\Ker}{\mathop{\text{Ker}}\nolimits}
\newcommand{\id}{\mathop{\text{id}}\nolimits}
\newcommand{\ii}{\mathrm{i}}
\newcommand{\dd}{\mathrm{d}}
\newcommand{\Vect}{\mathop{\text{Vect}}\nolimits}
\newcommand{\Mat}{\mathop{\mathrm{Mat}}\nolimits}
\newcommand{\rg}{\mathop{\text{rg}}\nolimits}
\newcommand{\tr}{\mathop{\text{tr}}\nolimits}
\newcommand{\ppcm}{\mathop{\text{ppcm}}\nolimits}

%----- Structure des exercices ------

\newtheoremstyle{styleexo}% name
{2ex}% Space above
{3ex}% Space below
{}% Body font
{}% Indent amount 1
{\bfseries} % Theorem head font
{}% Punctuation after theorem head
{\newline}% Space after theorem head 2
{}% Theorem head spec (can be left empty, meaning ‘normal’)

%\theoremstyle{styleexo}
\newtheorem{exo}{Exercice}
\newtheorem{ind}{Indications}
\newtheorem{cor}{Correction}


\newcommand{\exercice}[1]{} \newcommand{\finexercice}{}
%\newcommand{\exercice}[1]{{\tiny\texttt{#1}}\vspace{-2ex}} % pour afficher le numero absolu, l'auteur...
\newcommand{\enonce}{\begin{exo}} \newcommand{\finenonce}{\end{exo}}
\newcommand{\indication}{\begin{ind}} \newcommand{\finindication}{\end{ind}}
\newcommand{\correction}{\begin{cor}} \newcommand{\fincorrection}{\end{cor}}

\newcommand{\noindication}{\stepcounter{ind}}
\newcommand{\nocorrection}{\stepcounter{cor}}

\newcommand{\fiche}[1]{} \newcommand{\finfiche}{}
\newcommand{\titre}[1]{\centerline{\large \bf #1}}
\newcommand{\addcommand}[1]{}
\newcommand{\video}[1]{}

% Marge
\newcommand{\mymargin}[1]{\marginpar{{\small #1}}}



%----- Presentation ------
\setlength{\parindent}{0cm}

%\newcommand{\ExoSept}{\href{http://exo7.emath.fr}{\textbf{\textsf{Exo7}}}}

\definecolor{myred}{rgb}{0.93,0.26,0}
\definecolor{myorange}{rgb}{0.97,0.58,0}
\definecolor{myyellow}{rgb}{1,0.86,0}

\newcommand{\LogoExoSept}[1]{  % input : echelle
{\usefont{U}{cmss}{bx}{n}
\begin{tikzpicture}[scale=0.1*#1,transform shape]
  \fill[color=myorange] (0,0)--(4,0)--(4,-4)--(0,-4)--cycle;
  \fill[color=myred] (0,0)--(0,3)--(-3,3)--(-3,0)--cycle;
  \fill[color=myyellow] (4,0)--(7,4)--(3,7)--(0,3)--cycle;
  \node[scale=5] at (3.5,3.5) {Exo7};
\end{tikzpicture}}
}



\theoremstyle{definition}
%\newtheorem{proposition}{Proposition}
%\newtheorem{exemple}{Exemple}
%\newtheorem{theoreme}{Théorème}
\newtheorem{lemme}{Lemme}
\newtheorem{corollaire}{Corollaire}
%\newtheorem*{remarque*}{Remarque}
%\newtheorem*{miniexercice}{Mini-exercices}
%\newtheorem{definition}{Définition}




%definition d'un terme
\newcommand{\defi}[1]{{\color{myorange}\textbf{\emph{#1}}}}
\newcommand{\evidence}[1]{{\color{blue}\textbf{\emph{#1}}}}



 %----- Commandes divers ------

\newcommand{\codeinline}[1]{\texttt{#1}}

%%%%%%%%%%%%%%%%%%%%%%%%%%%%%%%%%%%%%%%%%%%%%%%%%%%%%%%%%%%%%
%%%%%%%%%%%%%%%%%%%%%%%%%%%%%%%%%%%%%%%%%%%%%%%%%%%%%%%%%%%%%


\begin{document}

\debuttexte


%%%%%%%%%%%%%%%%%%%%%%%%%%%%%%%%%%%%%%%%%%%%%%%%%%%%%%%%%%%
\diapo

\change

Dans cette première partie du chapitre sur les courbes paramétrées, nous allons introduire quelques notions fondamentales.

\change

Nous nous appuyons sur l'exemple de la cycloïde.

\change

Après l'avoir présenté, nous donnerons la définition générale d'une courbe paramétrée.

\change

Nous expliquerons ensuite comment réduire le domaine d'étude,

\change 

puis nous conclurons par la distinction entre points simples et points multiples. 

%%%%%%%%%%%%%%%%%%%%%%%%%%%%%%%%%%%%%%%%%%%%%%%%%%%%%%%%%%%
\diapo

Voici une roue de vélo, et un point sur la roue. Lorsque la roue avance, le point bouge.

\change
\change
\change
\change

La \defi{cycloïde} est la courbe que parcourt ce point fixé, lorsque le vélo avance. 
Les coordonnées $(x,y)$ 
de ce point $M$ varient
en fonction du temps $t$, 

\change
et sont données par les équations suivantes, 
où $r$ est le rayon de la roue.

% \change
% 
% Comme nous le verrons, la trajectoire du point $M$ a 
% l'allure de la courbe plane tracée en rouge sur ce graphique. 
% On a aussi tracé en bleu la roue en quelques instants.

%%%%%%%%%%%%%%%%%%%%%%%%%%%%%%%%%%%%%%%%%%%%%%%%%%%%%%%%%%%
\diapo

La cycloïde a des propriétés remarquables. Par exemple, la 
cycloïde renversée est une courbe \emph{brachistochrone} : 
c'est-à-dire que c'est la courbe 
qui permet à une bille d'arriver le plus rapidement possible d'un point $A$ à un point $B$.

Dans cette expérience deux billes sont lâchées en $A$ à l'instant $t_0$, 

Une bille bleue qui roule sur le segment $[AB]$.

Une bille rouge qui roule sur la cycloide renversée.

\change

\change

\change

\change
C'est la bille rouge qui arrive en premier,
Contrairement à ce que l'on pourrait croire !


Recommençons :

\change

\change

\change

\change



La bille bleue sur le segment $[AB]$ a une accélération constante.

La bille rouge suit la trajectoire de la cycloïde renversée, 
ayant une tangente verticale en $A$ et passant par $B$.

La bille rouge accélère beaucoup au début et elle atteint $B$ bien avant l'autre bille 
[$t_4$].
Notez que la bille rouge passe même par des positions en-dessous de la position d'arrivée $B$ [$t_3$].



%%%%%%%%%%%%%%%%%%%%%%%%%%%%%%%%%%%%%%%%%%%%%%%%%%%%%%%%%%%
\diapo


Voici la définition d'une \emph{courbe paramétrée} : c'est une application qui, 
à un réel (le \emph{paramètre}), associe \emph{un point} du plan. 
On parle aussi d'\defi{arc paramétré}.

\change
A une valeur réelle $t$, on associe un point $f(t)$.

Plutôt que $f(t)$, on notera de manière plus géométrique ce point $M(t)$,
ou par ces coordonnées $(x(t),y(t))$.



%%%%%%%%%%%%%%%%%%%%%%%%%%%%%%%%%%%%%%%%%%%%%%%%%%%%%%%%%%%
\diapo

Voici deux premiers exemples de courbes paramétrées : 
la paramétrisation du cercle trigonométrique par les 
fonctions $\cos$ et $\sin$, 

\change
et voici la paramétrisation $(2t-3,3t+1)$, c'est la paramétrisation d'une
droite, dont on détermine facilement un point et un vecteur directeur.

%%%%%%%%%%%%%%%%%%%%%%%%%%%%%%%%%%%%%%%%%%%%%%%%%%%%%%%%%%%
\diapo

On peut aussi paramétrer un segment de manière naturelle 
en faisant varier le paramètre, noté ici $\lambda$, 
dans l'intervalle $[0,1]$. 
En $\lambda=0$ le point est en $A$, en $\lambda=1$ il est en $B$.

\change
D'une manière générale, tous les graphes de fonctions sont de 
manière naturelle paramétrées, puisqu'il s'agit 
de l'ensemble des points du plan de coordonnées $(t,f(t))$. 
Attention cependant au fait que la réciproque n'est pas vraie : 
toute courbe paramétrée n'est pas la courbe représentative 
d'une fonction, comme le montre l'exemple du cercle.

%%%%%%%%%%%%%%%%%%%%%%%%%%%%%%%%%%%%%%%%%%%%%%%%%%%%%%%%%%%
\diapo

Il est important de comprendre qu'une courbe paramétrée 
ne se réduit pas au graphe, malgré le vocabulaire utilisé, 
mais c'est bel et bien \emph{une application}.

Pour insister sur la distinction, on introduit le notion 
de \emph{support} d'une courbe paramétrée comme l'ensemble 
des points $M(t)$ lorsque $t$ décrit l'ensemble de définition.

\change

Des courbes paramétrées différentes peuvent avoir un même support.

Par exemple on peut considérer la paramétrisation du cercle qui 
à $t$ associe le couple $(\cos(t),\sin(t))$ sur 
l'intervalle $[0,2\pi]$.
Mais on pourrait aussi considérer la paramétrisation définie 
sur l'intervalle $[0,4\pi]$, cela correspond au fait 
de faire deux fois le tour. 
On pourrait aussi paramétrer le cercle en associant à $t$ 
le couple $(\cos(2t),\sin(2t))$ par exemple sur 
l'intervalle $[0,\pi]$, ce qui correspondrait au fait de faire un tour, 
mais deux fois plus vite.

\change

Plus surprenant, la courbe 
$$t\mapsto \left(\frac{1-t^2}{1+t^2},\frac{2t}{1+t^2}\right), \qquad t\in\Rr,$$
est une paramétrisation du cercle privé du point $(-1,0)$, 
avec des coordonnées qui sont des fractions rationnelles.

Ainsi, la seule donnée du support ne suffit pas à définir un arc 
paramétré, qui est donc plus qu'un simple dessin :
c'est une \emph{courbe munie d'un mode de parcours}.


%%%%%%%%%%%%%%%%%%%%%%%%%%%%%%%%%%%%%%%%%%%%%%%%%%%%%%%%%%%
\diapo

Rappelons maintenant l'effet de quelques transformations 
géométriques usuelles sur le point $M(x,y)$ dans un repère orthonormé.


La translation de vecteur $(a,b)$ envoie $(x,y)$  sur $(x+a,y+b)$.


\change
C'est simplement l'addition 
du couple $(a,b)$ au couple $(x,y)$. .

\change
La réflexion par rapport à l'axe des abscisses envoie $(x,y)$ sur $(x,-y)$

\change
elle a pour effet de changer le signe de l'ordonnée, 

\change
et la réflexion par rapport à l'axe des ordonnées envoie $(x,y)$ sur $(-x,y)$

\change
La réflexion par rapport à la diagonale principale 
envoie $(x,y)$ sur $(y,x)$ : elle échange les coordonnées.



%%%%%%%%%%%%%%%%%%%%%%%%%%%%%%%%%%%%%%%%%%%%%%%%%%%%%%%%%%%
\diapo

Voici d'autres exemples d'expressions analytiques de transformations géométriques.

\change

La symétrie centrale de centre l'origine envoie $(x,y)$ sur $(-x,-y)$ ;
on change le signe des deux coordonnées.


\change
\change

La rotation de centre l'origine et d'angle $+\pi/2$ 
envoie $(x,y)$ sur $(-y,x)$.


%Les deux premiers exemples de chaque type sont illustrés sur les graphiques suivants.

%%%%%%%%%%%%%%%%%%%%%%%%%%%%%%%%%%%%%%%%%%%%%%%%%%%%%%%%%%%
\diapo



On cherche à simplifier 
le domaine d'étude des courbes paramétrées.
Examinons un premier exemple 
donné par les équations suivantes :

$$\left\{
\begin{array}{l}
x(t)=t-\frac32\sin t\\
y(t)=1-\frac32\cos t
\end{array}
\right.$$

\change

Pour cela, on cherche des propriétés particulières de $M(t)$. 
L'étude de la périodicité, comme on va le voir, 
permet de se restreindre à un intervalle de longueur $2\pi$. 

En effet $M(t+2\pi)$

\change
c'est
$(t-\tfrac32\sin t,1-\tfrac32\cos t)+(2\pi,0)$

car sinus et cosinus sont $2\pi$-périodique.

\change
$M(t+2\pi)$ est donc l'image de $M(t)$ par la translation de vecteur $\vec{u}$

qui est le vecteur horizontal $(2\pi,0)$.

\change
Il suffit donc d'étudier la courbe sur n'importe quel intervalle de longueur $2\pi$, 
on choisit l'intervalle $[-\pi,\pi]$.


\change
Mais on remarque aussi, par un calcul direct, que $M(-t)$ 

\change
c'est $(-x(t),y(t))$

\change
$M(-t)$ est donc le symétrique de $M(t)$  par rapport à l'axe des ordonnées. 

\change
On peut donc limiter notre étude à l'intervalle $[0,\pi]$.

\change
Voici la courbe sur l'intervalle $[0,\pi]$ [première figure].
On remonte maintenant nos transformations.

\change
On effectue la réflexion d'axe $(Oy)$, ce qui donne la courbe
sur l'intervalle $[-\pi,\pi]$ [deuxième figure].

\change
On obtient la courbe complète à partir de la portion déjà tracée
en effectuant les translations de vecteurs $k\vec{u}$, 
pour tout les entiers $k$ positifs ou négatifs.



%%%%%%%%%%%%%%%%%%%%%%%%%%%%%%%%%%%%%%%%%%%%%%%%%%%%%%%%%%%
\diapo

Le deuxième exemple de domaine d'étude est celui d'une \defi{courbe de Lissajous} 
dont les coordonnées sont données par $(\sin(2t),\sin(3t))$.

\change

Tout d'abord la fonction $M(t)$ est $2\pi$-périodique et on obtient donc la courbe 
complète quand $t$ décrit un intervalle de longueur $2\pi$, 

\change
qu'on va choisir égal à $[-\pi,\pi]$.

\change
D'autre part, on voit immédiatement que $M(-t)$ et $M(t)$ 
sont symétriques par rapport à l'origine.
En effet $M(-t)$

\change
$= \big(-\sin(2t),-\sin(3t)\big)$ 

\change
c'est $-M(t)$.

\change
On se limite donc à l'étude sur $[0,\pi]$.

\change
Et ce n'est pas fini !
On calcule $M(\pi-t)$ 

\change
$=\big(\sin(2\pi-2t),\sin(3\pi-3t)\big)$

\change
$=\big(\sin(-2t),\sin(\pi-3t)\big)$

\change
$=\big(-\sin(2t),\sin(3t)\big)$

\change
$=s_{(Oy)}\big(M(t)\big)$.

C'est le symétrique de $M(t)$ par rapport à l'axe des ordonnées. 

\change
On étudie et on construit la courbe seulement sur $t\in[0,\frac{\pi}{2}]$.

\change
Voici la courbe sur $[0,\frac{\pi}{2}]$.

[première figure]

\change
par la réflexion d'axe $(Oy)$ on obtient la courbe $[0,\pi]$
[deuxième figure].

\change
en appliquant une symétrie centrale de centre $O$ 
à la portion de courbe déjà tracée sur $[0,\pi]$, on obtient la courbe 
sur $[-\pi,\pi]$, et donc toute la courbe définie sur $\Rr$.

%%%%%%%%%%%%%%%%%%%%%%%%%%%%%%%%%%%%%%%%%%%%%%%%%%%%%%%%%%%
\diapo

Considérons une courbe paramétrée.

On définit la multiplicité d'un point $A$ du plan sur une 
courbe paramétrée comme le nombre de paramètres $t$ 
pour lesquels $M(t)=A$.

\change
\begin{itemize}
\item Si $A$ est atteint une et une seule fois, sa multiplicité est $1$ 
et on dit que le point $A$ est un \defi{point simple} 
de la courbe [première figure].

\item Si $A$ est atteint pour exactement deux valeurs distinctes du paramètre, 
on dit que $A$ est un \defi{point double} de la courbe [deuxième figure].

\item On parle de même de \defi{points triples} [troisième figure]
\defi{quadruples}, 
ou plus généralement de points \defi{multiples} 
dès que le point est atteint au moins deux fois [deuxième et troisième figure].

\end{itemize}


%%%%%%%%%%%%%%%%%%%%%%%%%%%%%%%%%%%%%%%%%%%%%%%%%%%%%%%%%%%
\diapo

Voici un exemple : on cherche les points multiples de 
la courbe paramétrée donnée par : 

$$\left\{
\begin{array}{l}
x(t)=2t+t^2\\
y(t)=2t-\frac{1}{t^2}
\end{array}
\right.$$

\change

Voici la courbe, il s'agit de trouver les points 
où la courbe se coupe elle-même. Ici on va calculer
qu'il n'y a qu'un point double, celui là !

\change
Pour trouver les points multiples d'une courbe,
on cherche les couples $(t,u)$ tq
$M(t)=M(u)$. On convient que $t>u$ afin de ne
pas compter la solution redondante $(u,t)$ en plus de $(t,u)$.

\change
Quand est-ce que $M(t)=M(u)$ ?

\change
c'est le cas ssi 
$\begin{array}{l}
2t+t^2=2u+u^2\\
2t-\frac{1}{t^2}=2u-\frac{1}{u^2}
\end{array}$.

\change
On regroupe les termes de même nature.

\change
La suite des calculs, c'est de factoriser, puis diviser par $t-u$.
Je vous laisse poursuivre les calculs. 
Mais sachez qu'en général les calculs peuvent être compliqués !

\change
Dans ce cas particulier, on trouve après calcul une seule solution :
$t=-1+\sqrt{2}\quad\text{et}\quad u=-1-\sqrt{2}$

\change

En ces deux valeurs du paramètre, le point est le point de coordonnées 
$(1,-5)$ [figure].



%%%%%%%%%%%%%%%%%%%%%%%%%%%%%%%%%%%%%%%%%%%%%%%%%%%%%%%%%%%
\diapo

Voici quelques exercices pour vous entraîner sur 
ces notions fondamentales des courbes paramétrées.





\end{document}
