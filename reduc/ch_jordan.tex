
\documentclass[12pt, class=report,crop=false]{standalone}
\usepackage[screen]{../exo7book}

% Commandes spécifiques à ce chapitre
% \newcommand{\GL}{GL}



\begin{document}

%====================================================================
\chapitre{Décomposition de Dunford et réduction de Jordan}
%====================================================================


Nous avons vu que les matrices ne sont pas toutes diagonalisables. 
On peut néanmoins décomposer certaines d'entre elles, en une forme la plus simple possible.
Nous verrons trois décompositions. 
\begin{itemize}
  \item La trigonalisation : transformer une matrice en une matrice triangulaire.
  \item La décomposition de Dunford : écrire une matrice comme la somme d'une matrice diagonalisable et d'une matrice nilpotente.
  \item La réduction de Jordan : transformer une matrice en une matrice diagonale par blocs.
\end{itemize}

$\Kk$ sera le corps $\Rr$ ou $\Cc$, $E$ un $\Kk$-espace vectoriel de dimension finie.


%%%%%%%%%%%%%%%%%%%%%%%%%%%%%%%%%%%%%%%%%%%%%%%%%%%%%
\section{Trigonalisation}

Nous allons montrer que toute matrice, dont le polynôme caractéristique est scindé, est semblable à une matrice triangulaire.

%----------------------------------------------------
\subsection{Trigonalisation}


On rappelle qu'une matrice $A=(a_{i,j})_{1\leq i,j\leq n}$ est \defi{triangulaire supérieure} 
si $a_{i,j}=0$ dès que $i> j$ :
$$A=\begin{pmatrix}
a_{1,1}&a_{1,2}&\cdots &a_{1,n} \cr 
0&a_{2,2} & \ddots &\vdots \cr
\vdots &\ddots &\ddots& a_{n-1,n}\cr
0&\cdots&0&a_{n,n}
\end{pmatrix}.$$
Les coefficients sous la diagonale sont tous nuls. Ceux sur la diagonale ou au-dessus peuvent être nuls ou pas.

\begin{definition}
\sauteligne
\begin{itemize}
  \item Une matrice $A \in M_n(\Kk)$ est \defi{trigonalisable} sur $\Kk$
  s'il existe une matrice inversible $P \in M_n(\Kk)$ inversible telle que $P^{-1}AP$ soit triangulaire supérieure.
  
  \item Un endomorphisme $f$ de $E$ est \defi{trigonalisable} s'il existe une
base de $E$ dans laquelle la matrice de $f$ soit triangulaire supérieure.
\end{itemize}
\end{definition} 

% Note : $\GL_n(\Kk)$ est l'ensemble des matrices inversibles de $M_n(\Kk)$.
Bien sûr, une matrice diagonalisable est en particulier trigonalisable.


\begin{theoreme}
Une matrice $A \in M_n(\Kk)$ (resp. un endomorphisme $f$) est trigonalisable sur $\Kk$ si et seulement 
si son polynôme caractéristique $\chi_A$ (resp. $\chi_f$) est scindé sur $\Kk$.
\end{theoreme} 

On rappelle qu'un polynôme est scindé sur $\Kk$ s'il se décompose en produit de facteurs linéaires dans $\Kk[X]$.

Remarquons que si $\Kk=\Cc$, par le théorème de d'Alembert-Gauss, on a : 
\begin{corollaire}
Toute matrice $A \in M_n(\Cc)$ est trigonalisable sur $\Cc$.
\end{corollaire}

Ce n'est pas le cas si $\Kk=\Rr$.
\begin{exemple}
Soit $A = 
\begin{pmatrix}
0&-1\\
1&0
\end{pmatrix} \in M_2(\Rr)$.
Alors $\chi_A(X) = X^2+1$. Ce polynôme n'est pas scindé sur $\Rr$, donc $A$ n'est pas trigonalisable sur $\Rr$. Si on considère cette même matrice $A$ comme élément de $M_2(\Cc)$, alors elle est trigonalisable (et ici même diagonalisable) sur $\Cc$ : il existe $P \in M_2(\Cc)$ inversible telle que $P^{-1}AP$ soit triangulaire supérieure. 
\end{exemple}



%----------------------------------------------------
\subsection{Preuve}

\begin{proof}~
\begin{itemize}

\item $\Longrightarrow$. 
Si $f$ est trigonalisable, il existe une base de $E$ dans laquelle la matrice de $f$ s'écrit
$$A=\begin{pmatrix}a_{1,1}&a_{1,2}&\cdots &a_{1,n} \cr 0&a_{2,2} &\ddots &\vdots \cr\vdots &\ddots &\ddots& a_{n-1,n}\cr0&\cdots&0&a_{n,n}\end{pmatrix}.$$
On a alors
$$\chi_f(X)=\chi_A(X)=\prod_{i=1}^n(a_{i,i}-X),$$
ce qui prouve que $\chi_f$ se décompose en produit de facteurs linéaires dans $\Kk[X]$.

\item $\Longleftarrow$.
La démonstration se fait par récurrence sur la dimension $n$ de l'espace vectoriel $E$. 
Si $n=1$, il n'y a rien à démontrer. Supposons le résultat vrai pour $n-1$, $n\ge2$ étant arbitrairement fixé. Le polynôme $\chi_f$ ayant au moins une racine dans $\Kk$, 
notons $\lambda$ l'une d'entre elles et $v_1$ un vecteur propre associé. 
Soit $F$ l'hyperplan supplémentaire de la droite $\Kk v_1$ : on a donc $E = \Kk v_1 \oplus F$. 
On considère alors une base $(v_1,v_2,\ldots,v_n)$ de $E$ avec, pour $2 \le i \le n$, 
$v_i\in F$. La matrice de $f$ dans cette base s'écrit
$$\left(\begin{array}{c|ccc}
\lambda &\ \  &\cdots&\ \  \cr \hline
0 &\ &\ & \cr
\vdots& & B &\cr
0&\ &\ &\  
\end{array}\right)$$
où $B$ est une matrice carrée de taille $(n-1)\times(n-1)$. On a 
$$\chi_f(X)=(\lambda-X)\det(B-X I_{n-1})=(\lambda-X)\chi_B(X).$$
Notons $g$ la restriction de $f$ à $F$ : la matrice de $g$ dans la base $(v_2,\ldots,v_n)$ est égale à $B$. Par hypothèse de récurrence, $g$ (et donc $B$) est trigonalisable : en effet, $\chi_f(X)=(\lambda-X)\chi_g(X)$,
 et comme $\chi_f$ est supposé scindé sur $\Kk$, $\chi_g$ l'est également. 
 Par conséquent, il existe une base $(w_2,\ldots,w_n)$ de $F$ dans laquelle la matrice de $g$ est triangulaire supérieure. Ainsi, dans la base $(v_1,w_2,\ldots,w_n)$, la matrice de $f$ est triangulaire supérieure.
\end{itemize}
\end{proof} 



%----------------------------------------------------
\subsection{Exemple}

\begin{exemple}
Soit $$A=\begin{pmatrix}1&4&-2\cr0&6&-3\cr-1&4&0\end{pmatrix} \in M_3(\Rr).$$
Démontrons que $A$ est trigonalisable sur $\Rr$ et trouvons une matrice $P$ telle que $P^{-1}AP$ soit triangulaire supérieure.

\begin{enumerate}
  \item  Commençons par calculer le polynôme caractéristique de $A$ :
$$\chi_A(X)=\begin{vmatrix}1-X&4&-2\cr 0&6-X&-3\cr-1&4&-X\end{vmatrix}=\cdots=(3-X)(2-X)^2$$

Comme $\chi_A$ est scindé sur $\Rr$, la matrice est trigonalisable sur $\Rr$. (Nous verrons plus tard si elle est diagonalisable ou pas.)

  \item Les racines du polynôme caractéristique sont les réels $3$ (avec la multiplicité $1$), et $2$ (avec la multiplicité $2$). 

Déterminons les sous-espaces propres associés.
  \begin{itemize}
    \item Soit $E_3$ le sous-espace propre associé à la valeur propre simple $3$ :
$E_3=\{v=(x,y,z)\in\Rr^3 \mid A v = 3v\}$.

$$v\in E_3 
\iff Av = 3v \iff \left\{\begin{array}{rcl}
x+4y-2z&=&3x\cr 
6y-3z&=&3y\cr 
-x+4y&=&3z
\end{array}\right.\iff x=y=z$$
$E_3$ est donc la droite vectorielle engendrée par le vecteur $v_1=(1,1,1)$.

    \item Soit $E_2$ le sous-espace propre associé à la valeur propre double $2$ :
$E_2=\{v=(x,y,z)\in\Rr^3 \mid A v = 2v\}$.
$$v\in E_2\iff Av = 2v \iff  
\left\{\begin{array}{rcl}
x+4y-2z&=&2x\cr
6y-3z&=&2y\cr
-x+4y&=&2z
\end{array}\right.\iff \left\{\begin{array}{rcl}x&=&z\cr 4y&=&3z\end{array}\right.$$
$E_2$ est donc la droite vectorielle engendrée par $v_2=(4,3,4)$.

La dimension de $E_2$ est égale à $1$ alors que la multiplicité de la valeur propre $2$ correspondante est égale à $2$. Par conséquent, on sait que la matrice $A$ ne sera pas diagonalisable.

    \item Soit $v_3=(0,0, 1)$. 
    Les vecteurs $(v_1, v_2, v_3)$ forment une base de $\Rr^3$. 
    La matrice de passage (constituée des $v_i$ écrits en colonne) est  
    $$P=\begin{pmatrix}1&4&0\cr 1&3&0\cr 1&4&1\end{pmatrix}\qquad  \text{ et } \qquad 
P^{-1}=\begin{pmatrix}-3&4&0\cr 1&-1&0\cr -1&0&1\end{pmatrix}.$$ 
On a $Av_1=3v_1$ et $Av_2=2v_2$. Il reste à exprimer $Av_3$ dans la base $(v_1, v_2, v_3)$ :
$$Av_3 = A(0,0,1) = (-2,-3,0) = -2(-3v_1+v_2-v_3)-3(4v_1-v_2) = -6v_1+v_2+2v_3.$$


  \end{itemize}
  
  \item  Ainsi, l'endomorphisme qui a pour matrice $A$ dans la base canonique de $\Rr^3$ a pour matrice $T$ dans la base $(v_1, v_2, v_3)$, où
$$T = \begin{pmatrix}3&0&-6\cr 0&2&1\cr 0&0&2\end{pmatrix}.$$
On aurait aussi pu calculer $T$ par la formule $T=P^{-1}AP$.

     \item Note. D'autres choix pour $v_3$ sont possibles. Ici, n'importe quel vecteur $v'_3$ complétant $(v_1,v_2)$ en une base de $\Rr^3$ conviendrait. Par contre, un autre choix conduirait à une matrice triangulaire $T'$ différente (pour la dernière colonne).
\end{enumerate}
\end{exemple} 

 
%----------------------------------------------------
\begin{miniexercices}
\sauteligne
\begin{enumerate}
  \item La matrice 
  $A = \left(\begin{smallmatrix}
  28 & -27 \\ 12 & -8
  \end{smallmatrix}\right)$ 
  est-elle trigonalisable sur $\Rr$ ? Si oui, trouver $P$ telle que 
  $P^{-1}AP$ soit triangulaire supérieure. 
  Même question avec :
$$\left(\begin{smallmatrix}
4 & -3 \\ 7 & 1
\end{smallmatrix}\right) \qquad
\left(\begin{smallmatrix}
-7 & -1 & 7 \\ 5 & 2 & -4 \\ -2 & -1 & 2
\end{smallmatrix}\right) \qquad
\left(\begin{smallmatrix}
4 & 3 & 7 & 0 \\
1 & 2 & 2 & 0 \\
-1 & 0 & 0 & 0 \\
-4 & -7 & -14 & -3
\end{smallmatrix}\right)$$  
   
  \item Trouver deux matrices $T, T' \in M_3(\Rr)$  qui soient distinctes, triangulaires supérieures et semblables.
  
\end{enumerate}
\end{miniexercices}




%%%%%%%%%%%%%%%%%%%%%%%%%%%%%%%%%%%%%%%%%%%%%%%%%%%%%
\section{Sous-espaces caractéristiques}


%----------------------------------------------------
\subsection{Lemme des noyaux}

Commençons par démontrer le lemme suivant :

\begin{lemme}[Lemme des noyaux]
Soit $f$ un endomorphisme de $E$.
Soient $P$ et $Q$ des polynômes de $\Kk[X]$, \evidence{premiers entre eux}. Alors :
\mybox{$\Ker(PQ)(f) = \Ker P(f)\oplus\Ker Q(f)$}
\end{lemme}
 
Généralisation : soient $P_1,\ldots,P_r$ des polynômes deux à deux premiers entre eux. Alors :
\mybox{$\Ker (P_1 \cdots P_r)(f) = \Ker(P_1(f)) \oplus \cdots \oplus \Ker (P_r(f))$}

On a bien sûr des énoncés similaires avec les matrices.

\bigskip


\evidence{Rappels.} 

\begin{itemize}
  \item Soient $P,Q \in \Kk[X]$. On dit que $P(X)$ et $Q(X)$ sont \defi{premiers entre eux} dans $\Kk[X]$ si les seuls polynômes qui divisent à la fois $P$ et $Q$ sont les polynômes constants.

  \item En particulier, sur $\Cc$, deux polynômes sont premiers entre eux si et seulement s'ils n'ont pas de racine commune.

  \item Le théorème de Bézout s'énonce ainsi :
\[P \text{ et } Q \text{ sont premiers entre eux } \iff  \exists A,B \in \Kk[X] \quad AP + BQ = 1 .\]
\end{itemize}

\bigskip

\begin{proof}
Soient $P$ et $Q$ deux polynômes premiers entre eux. Alors, d'après le théorème de Bézout, il existe des polynômes $A$ et $B$ tels que $AP+BQ=1$. On a donc, pour tout endomorphisme $f$ : 
$$A(f)\circ P(f)+B(f)\circ Q(f) = \id_E.$$
Autrement dit, pour tout $x\in E$ :
$$A(f)\circ P(f)(x)+B(f)\circ Q(f)(x)=x.$$


\begin{itemize}
  \item Montrons que $\Ker P(f) \cap \Ker Q(f) = \{0\}$.
  
Soit $x\in\Ker P(f)\cap\Ker Q(f)$. On a 
$$A(f)\circ \underbrace{P(f)(x)}_{=0}+B(f)\circ\underbrace{Q(f)(x)}_{=0}=x,$$
donc $x=0$, ce qui prouve $\Ker P(f)\cap\Ker Q(f)=\{0\}$.


  \item Montrons que $\Ker(PQ)(f) = \Ker P(f) + \Ker Q(f)$ par double inclusion.
  \begin{itemize}

    
    \item Preuve de $\Ker(PQ)(f) \subset \Ker P(f) + \Ker Q(f)$.

Soit $x\in\Ker(PQ)(f)$. 
On a, toujours en raison du théorème de Bézout, 
$$x  = A(f)\circ P(f)(x)+B(f)\circ Q(f)(x).$$

Montrons que $A(f)\circ P(f)(x) \in \Ker Q(f)$. En effet :
\[Q(f) \circ A(f) \circ P(f)(x) 
= A(f) \circ  P(f) \circ Q(f) (x) 
= A(f) \circ \big((PQ)(f)\big) (x)
= 0.\]
On a utilisé que les polynômes d'endomorphisme en $f$ commutent et
que $(PQ)(f) (x) = 0$.

De même, $B(f)\circ Q(f)(x) \in \Ker P(f)$.
Ainsi,
$$x = \underbrace{A(f)\circ P(f)(x)}_{\in\Ker Q(f)}+\underbrace{B(f)\circ Q(f)(x)}_{\in\Ker P(f)},$$
et donc $x \in \Ker P(f) + \Ker Q(f)$. 

    \item Preuve de $\Ker P(f) + \Ker Q(f) \subset \Ker(PQ)(f)$.
    Soient $y \in \Ker P(f)$ et $z\in \Ker Q(f)$. Alors :
    $$PQ(f)(y+z) = Q(f) \circ \underbrace{P(f)(y)}_{=0}+P(f)\circ \underbrace{Q(f)(z)}_{=0} = 0,$$
    et donc $y+z\in \Ker(PQ)(f).$
    
\end{itemize}
  \item Conclusion : $\Ker(PQ)(f) = \Ker P(f) \oplus \Ker Q(f)$.
\end{itemize}
\end{proof}



%----------------------------------------------------
\subsection{Sous-espaces caractéristiques}

Nous avons vu que, lorsque $f$ est diagonalisable, 
on a $E=E_{\lambda_1}\oplus\cdots\oplus E_{\lambda_r}$ avec 
$E_{\lambda_i}=\Ker(f-\lambda_i\id_E)$ le sous-espace propre associé 
à la valeur propre $\lambda_i$. Nous allons démontrer que même si $f$ 
n'est pas diagonalisable, mais si son polynôme caractéristique est scindé sur $\Kk$, on peut écrire
$$E=\Ker(f-\lambda_1\id_E)^{m_1}\oplus\cdots\oplus\Ker(f-\lambda_r\id_E)^{m_r},$$
où $m_i$ est la multiplicité de la valeur propre $\lambda_i$ comme 
racine du polynôme caractéristique de $f$.


\begin{definition}
Soit $f$ un endomorphisme de $E$. Soit $\lambda$ une valeur propre de $f$ et soit $m$ sa multiplicité
en tant que racine de $\chi_f$. Le \defi{sous-espace caractéristique} de $f$ pour la 
valeur propre $\lambda$ est 
\mybox{$N_\lambda=\Ker\big( (f-\lambda\id_E)^m \big)$.}
\end{definition} 

Pour $\lambda$ valeur propre de $f$, on a $E_\lambda \subset N_\lambda$, car 
$\Ker(f-\lambda\id_E) \subset \Ker(f-\lambda\id_E)^{k}$ quel que soit $k \ge 1$.

\begin{exemple}
\label{ex:sscar}
Soit $$A = \begin{pmatrix}
-2 & 3 & 0 & 0 \\
-3 & 4 & 0 & 0 \\
1 & 1 & 1 & 0 \\
9 & -5 & -1 & 3
\end{pmatrix} \in M_4(\Rr).$$


Calculons les sous-espaces caractéristiques de $A$.
\begin{itemize}
  \item Pour déterminer ses valeurs propres, on calcule d'abord son polynôme caractéristique :
  $$\chi_A(X) = \det(A-X I_4) = \cdots = (X-3)(X-1)^3$$
  La valeur propre $3$ est de multiplicité $1$ et la valeur propre $1$ est de multiplicité $3$.
  
  \item \textbf{Sous-espace caractéristique associé à $\lambda=3$.}
  
  Comme la multiplicité de cette valeur propre est $1$ alors le sous-espace caractéristique est aussi le sous-espace propre : $N_3 = \Ker(A-3I_4)^1 = E_3$. 
  Ainsi, $N_3 = \{ v \in \Rr^4 \mid (A-3I_4)v = 0 \}$. 
  Comme $N_3=E_3$ est de dimension $1$ et $v_1=(0,0,0,1) \in N_3$, alors
  $$N_3 = \Rr v_1.$$
    
  
  \item \textbf{Sous-espace caractéristique associé à $\lambda=1$.}

   La multiplicité de cette valeur propre est $3$, donc
   $N_1 = \Ker (A-I_4)^3$. On a :
   $$A-I_4 = \begin{pmatrix}
-3 & 3 & 0 & 0 \\
-3 & 3 & 0 & 0 \\
1 & 1 & 0 & 0 \\
9 & -5 & -1 & 2
   \end{pmatrix}\qquad
   (A-I_4)^2 = \begin{pmatrix}
0 & 0 & 0 & 0 \\
0 & 0 & 0 & 0 \\
-6 & 6 & 0 & 0 \\
5 & 1 & -2 & 4   
   \end{pmatrix}\qquad
   (A-I_4)^3 = \begin{pmatrix}
0 & 0 & 0 & 0 \\
0 & 0 & 0 & 0 \\
0 & 0 & 0 & 0 \\
16 & -4 & -4 & 8   
   \end{pmatrix} 
$$ 
On cherche une base de $N_1 = \{ v \in \Rr^4 \mid (A-I_4)^3v = 0 \}$.
C'est un espace vectoriel de dimension $3$, dont par exemple $(v_2,v_3,v_4)$ est une base, avec
$$v_2 = (1,4,0,0) \qquad v_3 = (1,0,4,0) \qquad v_4 = (1,0,0,-2),$$
et donc 
$$N_1 = \Vect(v_2,v_3,v_4).$$
\end{itemize}
\end{exemple}

\begin{theoreme}
Soit $f$ un endomorphisme de $E$ tel que $\chi_f$ est scindé sur $\Kk$.
Notons $\chi_f(X)=\pm(X-\lambda_1)^{m_1}\cdots(X-\lambda_r)^{m_r}$ et, pour $1\leq i\leq r$, $N_{\lambda_i}$ le sous-espace caractéristique associé à la
valeur propre $\lambda_i$. Alors :

\begin{enumerate}
  \item Chaque $N_{\lambda_i}$ est stable par $f$.
  \item $E=N_{\lambda_1}\oplus\cdots\oplus N_{\lambda_r}$.
  \item $\dim N_{\lambda_i}=m_i$.
\end{enumerate}
\end{theoreme} 

Autrement dit, l'espace vectoriel $E$ est la somme directe des sous-espaces caractéristiques. En plus, la dimension du sous-espace caractéristique associé à la valeur propre $\lambda$ est la multiplicité de $\lambda$ comme racine du polynôme caractéristique. 

\begin{exemple}
Reprenons l'exemple \ref{ex:sscar} :
\begin{itemize}	
  \item $3$ est valeur propre de multiplicité $1$, et on a bien $\dim N_3 = 1$,
  \item $1$ est valeur propre de multiplicité $3$, et on a bien $\dim N_1 = 3$,
  \item on a bien $\Rr^4 = N_3 \oplus N_1.$
\end{itemize}
\end{exemple}

\begin{proof}~
\begin{enumerate}
  \item Soit $x \in N_\lambda = \Ker (f-\lambda \id_E)^m$.
  On a $(f-\lambda\id_E)^{m}(x)=0$. Or 
$$(f-\lambda\id_E)^{m}\circ f(x)=f\circ(f-\lambda\id_E)^{m}(x)=0,$$
d'où $f(x)\in N_\lambda$.

  \item C'est une application du lemme des noyaux. On rappelle que
$$\chi_f(X)=\pm(X-\lambda_1)^{m_1}\cdots(X-\lambda_r)^{m_r}.$$
Les polynômes $(X-\lambda_i)^{m_i}$ sont premiers entre eux puisque 
les valeurs propres sont distinctes. Par le lemme des noyaux, on obtient
$$\Ker \chi_f(f) 
= \Ker (f-\lambda_1\id_E)^{m_1} \oplus \cdots \oplus \Ker (f-\lambda_r\id_E)^{m_r}
= N_{\lambda_1} \oplus\cdots\oplus N_{\lambda_r}.$$
Or, d'après le théorème de Cayley-Hamilton, on a $\chi_f(f)=0$, donc $\Ker \chi_f(f)=E$, 
d'où le résultat.
 
  \item Notons $g_i=f_{|N_{\lambda_i}}$ pour $1\le i \le r$. 
  Pour $i\neq j$, $N_{\lambda_i}\cap N_{\lambda_j}=\{0\}$. Or $E_{\lambda_i}\subset N_{\lambda_i}$, 
  donc la seule valeur propre possible de $g_i$ est $\lambda_i$. 
  Le polynôme caractéristique de $g_i$ est scindé (car il divise celui de $f$) et sa seule racine est la seule valeur propre de $g_i$, c'est-à-dire $\lambda_i$. 
  Ainsi, $\chi_{g_i}(X)= \pm(X-\lambda_i)^{n_i}$ (où $n_i = \dim N_{\lambda_i}$). 
  De plus,
$$\pm(X-\lambda_1)^{m_1}\cdots(X-\lambda_r)^{m_r} = \chi_f(X) 
= \chi_{g_1}(X)\cdots \chi_{g_r}(X) = \pm (X-\lambda_1)^{n_1}\cdots(X-\lambda_r)^{n_r}.$$
D'où, en identifiant les exposants des facteurs irréductibles, $n_i=\dim N_i=m_i$, pour $1\le i\le r$.
\end{enumerate}
\end{proof}

 
%----------------------------------------------------
\begin{miniexercices}
\sauteligne
\begin{enumerate}
  \item Calculer les sous-espaces caractéristiques de la matrice  
  $A = \left(\begin{smallmatrix}
-2 & -2 & -1 & -3 \\
0 & -2 & 0 & 3 \\
0 & 1 & -2 & 1 \\
0 & -3 & 0 & 4
  \end{smallmatrix}\right)$.
  Même exercice avec
  $B = \left(\begin{smallmatrix}
1 & -2 & 2 & 2 & 2 \\
2 & 6 & -5 & -3 & -4 \\
2 & 3 & -2 & -3 & -5 \\
0 & 1 & -1 & 2 & 0 \\
0 & -1 & 1 & 1 & 4
  \end{smallmatrix}\right)$.

  \item Soient $f \in \mathcal{L}(E)$ et $\lambda \in \Kk$. Montrer que $\Ker(f-\lambda \id_E) \subset \Ker(f-\lambda \id_E)^2 \subset \Ker(f-\lambda \id_E)^3 \subset \cdots$.
  
  \item En utilisant le lemme des noyaux, prouver ce résultat du chapitre \og{}Polynômes d'endomorphismes\fg{} :
  \begin{quote}
  \og{}\textbf{Théorème.} Un endomorphisme $f \in\mathcal{L}(E)$ est diagonalisable sur $\Kk$ si et seulement si son polynôme minimal est scindé à racines simples dans $\Kk$.\fg{}
  \end{quote}
\end{enumerate}
\end{miniexercices}




%%%%%%%%%%%%%%%%%%%%%%%%%%%%%%%%%%%%%%%%%%%%%%%%%%%%%
\section{Décomposition de Dunford}

Nous allons montrer que toute matrice, dont le polynôme caractéristique est scindé, peut s'écrire comme somme d'une matrice diagonalisable et d'une matrice nilpotente. Autrement dit, cette matrice est semblable à la somme d'une matrice diagonale et d'une matrice nilpotente.

%----------------------------------------------------
\subsection{\'Enoncé}

\begin{definition}
On dit qu'un endomorphisme $f$ (resp. une matrice $A$) 
est \defi{nilpotent(e)} s'il existe $k\in\Nn^*$ tel
que $f^k=0$ (resp. $A^k=0$).
\end{definition} 



Nous allons démontrer que les endomorphismes nilpotents et 
les endomorphismes diagonalisables permettent de décrire tous les 
endomorphismes dont le polynôme caractéristique est scindé sur $\Kk$ (c'est-à-dire ceux trigonalisables). 



\begin{theoreme}[Décomposition de Dunford]
\label{th:dunford1}
Soit $f$ un endomorphisme de $E$ tel que $\chi_f$ soit scindé sur $\Kk$. 
Alors il existe un unique couple $(n,d)$ d'endomorphismes tel que : 
\begin{itemize}
  \item[i)] $n$ est nilpotent et $d$ est diagonalisable,
  \item[ii)] $f=n+d$,
  \item[iii)] $n\circ d=d\circ n$.   
\end{itemize}
\end{theoreme}

De plus, on pourrait montrer que $d$ et $n$ sont des polynômes en $f$.
En particulier, si $\Kk=\Cc$, cette décomposition existe toujours.
Le théorème peut encore s'écrire :

\begin{theoreme}[Décomposition de Dunford]
\label{th:dunford2}
Pour toute matrice $A\in M_n(\Kk)$ ayant un polynôme caractéristique scindé sur $\Kk$, il existe une 
unique matrice $N$ nilpotente et une unique matrice $\Delta$ diagonalisable telles que 
$$A=N+\Delta \quad \text{ et } \quad N\Delta=\Delta N.$$
\end{theoreme} 
 
\bigskip 
 
\evidence{Remarque importante.} \\
Attention ! $\Delta$ est une matrice \evidence{diagonalisable}, pas nécessairement une matrice diagonale.

Comme $\Delta$ est diagonalisable, alors il existe une matrice inversible $P$ et une matrice \evidence{diagonale} $D$ telles que $D = P^{-1}\Delta P$. Si on note $N' = P^{-1}NP$ alors $N'$ est encore nilpotente et $N' D= DN'$. Une autre façon d'écrire la décomposition de Dunford est alors $P^{-1}AP = D + N'$. C'est dire que $A$ est semblable à la somme d'une matrice diagonale avec une matrice nilpotente.
 
\bigskip

Comme conséquence directe :
\begin{corollaire}
Soit $f$ un endomorphisme avec une décomposition de Dunford $f= d+n$, avec
$d$ diagonalisable, $n$ nilpotent et $d \circ n = n \circ d$. Alors :
\begin{itemize}
  \item $f$ diagonalisable $\iff$ $f=d$ $\iff$ $n=0$ ;
  \item $f$ nilpotent $\iff$ $f=n$ $\iff$ $d=0$.
\end{itemize}
\end{corollaire}


 
 
%----------------------------------------------------
\subsection{Lemmes}

 
Avant de démontrer ces théorèmes, nous allons démontrer 
des lemmes dont les résultats nous seront utiles.


\begin{lemme}
\label{lem:jordan1}
Si $f$ est nilpotent, alors $0$ est son unique valeur propre et on a 
$$\chi_f(X)=(-1)^n X^n.$$
\end{lemme} 

\begin{proof}~
\begin{itemize}
  \item Notons $A$ la matrice de $f$ dans une base. 
  Comme $f$ est nilpotent, il existe $k\ge1$ tel que $A^k=0$. Cela implique 
  $\det(A^k) = 0$, donc $\det(A) = 0$. La matrice $A$ n'est donc pas inversible : cela entraîne que $f$ n'est pas bijectif, et comme $f$ est un endomorphisme, $f$ n'est pas non plus injectif (la dimension de l'espace de départ égale la dimension de l'espace d'arrivée).
  Ainsi, $\Ker f\neq \{0\}$, ce qui est exactement dire que $0$ est une valeur propre de $f$.
  
  \item Supposons que $\lambda$ soit une valeur propre de $f$ : il existe alors $x\neq 0$ tel que $f(x)=\lambda x$. Par récurrence sur $n$,  $f^n(x)=\lambda^n x$. Or, $f$ est supposé nilpotent, il existe donc $k\in\Nn^*$ tel que $f^k=0$. D'où $\lambda^k x=0$, ce qui implique $\lambda^k=0$ (car $x\neq 0$), donc $\lambda =0$. 
  
  \item Ainsi, $0$ est la seule valeur propre de $f$, donc la seule racine de son polynôme caractéristique. 
  
  
  \item Si $\Kk = \Rr$, on considère l'endomorphisme comme aussi défini sur $\Cc$. En termes de matrice, cela revient à dire que la matrice à coefficients réels $A$ peut être vue aussi comme à coefficients complexes. Or, sur $\Cc$, un polynôme dont la seule racine est $0$ est de la forme $\alpha X^n$, donc $\chi_f(X)=(-1)^nX^n$
  puisque le coefficient dominant d'un polynôme caractéristique est $(-1)^n$.

  
\end{itemize}
\end{proof}

\begin{lemme}
\label{lem:jordan2bis}
Soit $f$ un endomorphisme de $E$ diagonalisable. On note $\lambda_1,\ldots,\lambda_r$ ses valeurs propres 
et $E_{\lambda_1},\ldots,E_{\lambda_r}$ les sous-espaces propres correspondants. 
Si $F$ est un sous-espace de $E$ stable par $f$, alors on a 
$$F = (F\cap E_{\lambda_1})\oplus\cdots\oplus (F\cap E_{\lambda_r}).$$
\end{lemme} 

\begin{proof}
Soit $x\in F$. Comme $x\in E=E_{\lambda_1}\oplus\cdots\oplus E_{\lambda_r}$, il existe $x_1,\dots,x_r$ uniques, avec $x_i\in E_{\lambda_i}$ pour $1\leq i\leq r$, tels que $x=x_1+\cdots+x_r$.

Le sous-espace $F$ est stable par $f$, donc également par $P(f)$ pour tout polynôme $P\in K[X]$. 
On rappelle que comme $x_i \in E_{\lambda_i}$ alors $f(x_i)=\lambda_i x_i$ et plus généralement $P(f)(x_i) = P(\lambda_i) x_i$.
Pour $1\leq i\leq r$, on définit
$$P_i(X)=\prod_{\substack{k=1\\k\neq i}}^r(X-\lambda_k).$$
On a ainsi $P_i(\lambda_j)=0$ si $i\neq j$, et $P_i(\lambda_i)\neq 0$. 

On peut alors écrire
$$P_i(f)(x)
= P_i(f)(x_1+\cdots+x_r)
= P_i(\lambda_1)x_1+\cdots+P_i(\lambda_r)x_r
= P_i(\lambda_i)x_i.$$
Or $P_i(f)(x)\in F$ par stabilité de $f$, donc $x_i\in F$. Ainsi, pour tout $1\le i \le r$, $x_i\in F\cap E_{\lambda_i}$, d'où le résultat.
\end{proof}


\begin{lemme}
\label{lem:jordan2}
Si $f$ est diagonalisable et $F$ est un sous-espace vectoriel de $E$, stable par $f$, alors la restriction 
de $f$ à $F$ est aussi diagonalisable.
\end{lemme}

Pour une autre preuve, voir la fin du chapitre \og{}Polynômes d'endomorphismes\fg{}.

\begin{proof}
Notons $g$ la restriction de $f$ au sous-espace $F$ : $g=f_{|F}$ qui est bien définie car $F$ est stable par $f$. Soient $\lambda_1,\dots,\lambda_r$ les valeurs propres de $f$. 
L'endomorphisme $f$ étant supposé diagonalisable, on a 
$$E=E_{\lambda_1}\oplus\dots\oplus E_{\lambda_r},$$
où les $E_{\lambda_i}$ sont les sous-espaces propres de $f$. 
On obtient grâce au lemme \ref{lem:jordan2bis} que
$$F = (F\cap E_{\lambda_1})\oplus\dots\oplus (F\cap E_{\lambda_r}).$$

Or, quel que soit $\mu\in\Kk$, $\Ker(g-\mu\id_F)=F\cap\Ker(f-\mu\id_E)$,
donc les valeurs propres de $g$, notées $\mu_1,\dots,\mu_s$, appartiennent à l'ensemble 
$\{\lambda_1,\dots,\lambda_r\}$. 
Ces $\mu_i$ forment exactement l'ensemble des valeurs de $\{\lambda_1,\dots,\lambda_r\}$ pour lesquelles $F\cap E_{\lambda_i} \neq \{0\}$.
On a donc 
$$F= \Ker(g-\mu_1\id_F)\oplus \cdots \oplus \Ker(g-\mu_s\id_F),$$
ce qui prouve que $g$ est diagonalisable.
\end{proof}


\begin{lemme}
\label{lem:jordan3}
Soient $f$ et $g$ deux endomorphismes diagonalisables. On suppose que 
$f\circ g=g\circ f$. Alors il existe une base commune de vecteurs propres de $f$ et de $g$. 
\end{lemme}

Autrement dit, si $A,B \in M_n(\Kk)$ sont diagonalisables et commutent, alors on peut les diagonaliser dans  une base commune, c'est-à-dire qu'il existe une matrice $P \in M_n(\Kk)$ inversible telle que $P^{-1}AP$ et $P^{-1}BP$ soient toutes les deux diagonales.


\begin{proof}
Soient $\lambda_1,\dots,\lambda_r$ les valeurs propres de $f$. Notons 
$E_{\lambda_i} = \{x\in E \mid f(x)=\lambda_i x\}$. 
On a alors, pour $x\in E_{\lambda_i}$,
$$f(g(x))=g(f(x))=g(\lambda_i x)=\lambda_i g(x),$$
donc $g(x)\in E_{\lambda_i}$, ce qui prouve que $E_{\lambda_i}$ est stable par $g$.
D'après le lemme \ref{lem:jordan2}, la restriction de $g$ à $E_{\lambda_i}$ est donc diagonalisable. On considère, dans $E_{\lambda_i}$, une base $\mathcal{B}_i$ de vecteurs propres de $g$. Noter que ce sont aussi des vecteurs propres de $f$ (car ils sont dans $E_{\lambda_i}$).
Comme $f$ est diagonalisable, on a
$$E=\underbrace{E_{\lambda_1}}_{\mathcal{B}_1}\oplus\cdots\oplus\underbrace{E_{\lambda_r}}_{\mathcal{B}_r}.$$
La base $\mathcal{B}=\mathcal{B}_1\cup\cdots\cup\mathcal{B}_r$ est donc une base de $E$ formée de vecteurs qui sont des vecteurs propres à la fois pour $f$ et pour $g$.
\end{proof}



%----------------------------------------------------
\subsection{Preuve}

Passons à la démonstration du théorème de décomposition de Dunford. Voici l'idée générale :
\begin{itemize}
  \item On décompose l'espace vectoriel $E$ en somme des sous-espaces caractéristiques
$N_{\lambda_i}$, où les $\lambda_i$ sont les valeurs propres de $f$. 
  \item Sur chacun de ces sous-espaces, on décompose la restriction de $f$ en $d_i+n_i$, avec $d_i = \lambda_i \id_{N_{\lambda_i}}$ qui est bien sûr diagonalisable.
  \item On montre que $n_i$, qui est $f-d_i$ restreint à $N_{\lambda_i}$, est nilpotent.
  \item Comme $d_i$ est $\lambda_i$ fois l'identité, alors $d_i$ commute en particulier 
  avec $n_i$.
\end{itemize}

\begin{proof}~

\textbf{Construction.}
\begin{itemize}
  \item Soit $\chi_f$ le polynôme caractéristique de $f$ qui, par hypothèse, est scindé sur $\Kk$.
 Notons $\lambda_i$ une valeur propre de $f$, et $m_i$ sa multiplicité en tant que racine du polynôme caractéristique :
$$\chi_f(X)=\pm\prod_{i=1}^r(X-\lambda_i)^{m_i}.$$

  \item Soient $N_{\lambda_1},\ldots, N_{\lambda_r}$ les sous-espaces caractéristiques de $f$. Pour $1\le i\le r$, on a 
$$N_{\lambda_i} = \Ker(f-\lambda_i\id_E)^{m_i}\quad  \text{ et }\quad 
E=N_{\lambda_1}\oplus\cdots\oplus N_{\lambda_r}.$$

  \item Nous allons définir l'endomorphisme $d$ sur chaque $N_{\lambda_i}$ de la manière suivante : pour tout $x\in N_{\lambda_i}$, on pose
$$d(x)=\lambda_i x.$$
L'espace vectoriel $E$ étant somme directe des $N_{\lambda_i}$, 
$d$ est défini sur $E$ tout entier. En effet, si $x \in E$ est décomposé en
 $x=x_1+\cdots+x_r$ avec $x_i\in N_{\lambda_i}$ (pour $1\le i\le r$), alors
 $$d(x) = d(x_1+\cdots+x_r) = d(x_1) + \cdots + d(x_r)
 = \lambda_1 x_1 + \cdots + \lambda_r x_r.$$
Pour $1\le i\le r$, on a  $d_i=d_{|N_{\lambda_i}}=\lambda_i\id_{N_{\lambda_i}}$. 

On pose enfin 
$$n(x)=f(x)-d(x).$$

Il nous reste à vérifier que $n$ et $d$ conviennent.
\end{itemize}

\medskip

\textbf{Propriétés.}
\begin{enumerate}
  \item Par construction, $d$ est diagonalisable. 
  En effet, fixons une base pour chaque sous-espace $N_{\lambda_i}$. Pour chaque vecteur $x$ de cette base,
  $d(x)=\lambda_i x$. Comme $E$ est somme directe des $N_{\lambda_i}$
  alors, dans la base de $E$ formée de l'union des bases des $N_{\lambda_i}$ ($1\le i\le r$), la matrice de $d$ est diagonale.

  \item On a défini $n=f-d$. $N_{\lambda_i}$ est stable par $n$ (car c'est vrai pour $f$  et $d$). 
  On pose $n_i=n_{|N_{\lambda_i}}=f_{|N_{\lambda_i}}-\lambda_i\id_{N_{\lambda_i}}$. 
  Alors, par définition, $N_{\lambda_i} = \Ker(n_i^{m_i})$, et donc $n_i^{m_i}=0$.
  % La seule valeur propre de $n_i$ est $0$ et $\dim N_{\lambda_i} = m_i$. On a donc par le théorème de Cayley-Hamilton, $n_i^{m_i}=0$. 
  Ainsi, en posant $m=\max {m_i}$ ($1\le i \le r$), puisque $n^m$ s'annule sur chaque $N_{\lambda_i}$, alors $n^m=0$, ce qui prouve que $n$ est nilpotent.

  \item On va vérifier que $d\circ n=n\circ d$. Si $x\in E$, il se décompose
  en $x=x_1+\cdots+x_r$ avec $x_i\in N_{\lambda_i}$ pour $1\le i\le r$.
  Sur chaque $N_{\lambda_i}$, $d_{|N_{\lambda_i}}=\lambda_i\id_{N_{\lambda_i}}$ donc commute avec tout endomorphisme. 
  En particulier, $d\circ n(x_i)=n\circ d(x_i)$ puisque $N_{\lambda_i}$ est stable par $n$. On a donc
  $$d\circ n(x)= d\circ n(x_1+\cdots+x_r)
  =d\circ n(x_1)+\cdots+d\circ n(x_r)=n\circ d(x_1)+\cdots+n\circ d(x_r)=n\circ d(x).$$
  Ainsi, $d$ et $n$ commutent.

  \item  Il reste à prouver l'unicité. Supposons que $(n,d)$ soit le couple construit ci-dessus et soit $(n',d')$ un autre couple vérifiant les propriétés (i), (ii), (iii) de la décomposition de Dunford.
  \begin{itemize}
    \item Montrons que $d$ et $d'$ commutent, ainsi que $n$ et $n'$.
    
    On a $f=d+n=d'+n'$, d'où 
    $$d\circ f=d\circ (d+n) = d\circ d+d\circ n
    =d\circ d+n\circ d = (d+n)\circ d =f\circ d.$$ 
    Ainsi, $f$ et $d$ commutent et on montre de même que $f$ et $d'$ commutent.
    Montrons que $N_{\lambda_i}$ est stable par $d'$. 
    Soit $x\in N_{\lambda_i}= \Ker (f-\lambda_i\id_E)^{m_i}$. Alors
$$(f-\lambda_i\id_E)^{m_i}\circ d'(x)=d'\circ(f-\lambda_i\id_E)^{m_i}(x)=0.$$
Donc $d'(x) \in N_{\lambda_i}$.

    Par construction, $d_{|N_{\lambda_i}}=\lambda_i\id_{N_{\lambda_i}}$, donc $d$ et $d'$ commutent sur chaque 
    $N_{\lambda_i}$, donc sur $E$ tout entier.
    
    Or $n=f-d$ et $n'=f-d'$ donc, comme $d$ et $d'$ commutent et $f$ commute avec $d$ et $d'$, alors $n$ et $n'$ commutent également. 
    
    \item Comme $d$ et $d'$ commutent, d'après le lemme \ref{lem:jordan3}, il existe une base commune de vecteurs propres. En particulier, $d-d'$ est diagonalisable. 
    
    \item Comme les endomorphismes $n$ et $n'$ sont nilpotents et commutent, $n-n'$ est également nilpotent. En effet, si $p$ et $q$ sont des entiers tels que $n^p=0$ et $n'^q=0$, alors $(n-n')^{p+q}=0$ (écrire la formule du binôme de Newton et voir que, dans chaque terme $n^k n'^{p+q-k}$, on a $k \ge p$ ou $p+q-k\ge q$).
    
    
    \item  Ainsi $d-d'=n'-n$ est un endomorphisme qui est à la fois diagonalisable et nilpotent.
    Comme il est nilpotent, sa seule valeur propre est $0$ (c'est le lemme \ref{lem:jordan1}).
    Et comme il est diagonalisable, c'est nécessairement l'endomorphisme nul. On a donc $d-d'=n'-n=0$.
    
    \item Conclusion : $d=d'$, $n=n'$, ce qui termine la preuve de l'unicité. 
    
  \end{itemize}
   
\end{enumerate}
\end{proof}



%----------------------------------------------------
\subsection{Exemples}


%----------------------------------------------------
\evidence{Piège classique}

Attention, une matrice triangulaire peut toujours s'écrire 
comme somme d'une matrice diagonale et d'une matrice nilpotente, 
mais, en général, celles-ci ne commutent pas. 

Retenez bien ce contre-exemple pour éviter ce piège classique.
\begin{exemple}
Soient 
$$A = \begin{pmatrix}
1 & 3\\
0 & 2
\end{pmatrix}
\qquad \tilde D =  \begin{pmatrix}
1 & 0\\
0 & 2
\end{pmatrix}
\qquad \tilde N =  \begin{pmatrix}
0 & 3\\
0 & 0
\end{pmatrix}.$$
Alors on a bien $A = \tilde D + \tilde N$, $\tilde D$ est diagonale,
$\tilde N$ est nilpotente ($\tilde N^2 = 0$). Pourtant, ce n'est pas la décomposition de Dunford car les matrices ne commutent pas : $\tilde D \tilde N \neq \tilde N \tilde D$.
 
La décomposition de Dunford est tout simplement $D = A$, et $N$ est la matrice nulle.
$D$ est bien diagonalisable (son polynôme caractéristique est scindé à racines simples)
mais n'est pas diagonale ; $N$ est nilpotente et $DN=ND$. 
\end{exemple}

\bigskip
%----------------------------------------------------
\evidence{Pratique de la décomposition}

La méthode pour trouver la décomposition de Dunford d'une matrice $A \in M_n(\Kk)$ consiste à suivre les étapes de la preuve.
\begin{enumerate}

  \item On calcule le polynôme caractéristique $\chi_A$ de $A$ : il doit être scindé. On calcule ses racines, qui sont les valeurs propres de $A$.
  
  \item  Pour chaque valeur propre $\lambda$, de multiplicité $m$ comme racine de $\chi_A$,
  on note $N_\lambda = \Ker (A-\lambda I_n)^m$. C'est un espace vectoriel de dimension $m$. On détermine
  $m$ vecteurs formant une base de $N_\lambda$.
L'union de toutes les bases $\mathcal{B}_\lambda$ des $N_\lambda$ forme une base $\mathcal{B}
  = (v_1,\ldots,v_n)$ de $\Kk^n$.
   
  \item On définit l'endomorphisme $d$ par $d(v_i) = \lambda v_i$ pour chaque $v_i \in N_\lambda$.  
  (Dans la base $\mathcal{B}$, la matrice de $d$ est diagonale.)
  On note $\mathcal{B}_0 = (e_1,\ldots,e_n)$ la base canonique de $\Kk^n$. ($A$
  est la matrice de l'endomorphisme $f$ dans la base $\mathcal{B}_0$.) 
  $\Delta$ sera la matrice de $d$ dans la base $\mathcal{B}_0$,
  c'est-à-dire que les colonnes de $\Delta$ sont les coordonnées des $d(e_i)$ exprimées dans la base $(e_1,\ldots,e_n)$. 
  
  \item On pose $N = A - \Delta$. Par la démonstration du théorème \ref{th:dunford2}, $\Delta$ est diagonalisable, $N$ est nilpotente et $\Delta  N = N \Delta$. La matrice de passage $P$ de la base $\mathcal{B}$ vers la base canonique $\mathcal{B}_0$ transforme $\Delta$ en une matrice diagonale $D = P^{-1}\Delta P$. 
\end{enumerate}


On commence par un exemple où les calculs sont assez simples.
\begin{exemple}
Calculons la décomposition de Dunford de la matrice 
$$A=\begin{pmatrix}1&1&1\cr 0&1&1\cr 0&0&2\end{pmatrix} \in M_3(\Rr).$$


\'Evitons le piège d'écrire :
$$A=\underbrace{\begin{pmatrix}1&0&0\cr 0&1&0\cr 0&0&2\end{pmatrix}}_{\tilde D}
+\underbrace{\begin{pmatrix}0&1&1\cr 0&0&1\cr 0&0&0\end{pmatrix}}_{\tilde N}$$
$\tilde D$ est diagonale (donc diagonalisable) et $\tilde N$ est nilpotente,
mais on peut vérifier que les matrices \emph{ne commutent pas} :  
$\tilde D \tilde N \neq \tilde N \tilde D$.
%$DN=\left(\begin{matrix}0&1&1\cr 0&0&1\cr 0&0&0\end{matrix}\right)\neq ND=\left(\begin{matrix}0&1&2\cr 0&0&2\cr 0&0&0\end{matrix}\right).$$

\bigskip 
Calculons maintenant la décomposition de Dunford de $A$.

\begin{enumerate}
  \item Le polynôme caractéristique $\chi_A$ est égal à
$$\chi_A(X) 
= \det(A-XI_3) 
= \begin{vmatrix}1-X & 1 & 1\cr 0 & 1-X & 1\cr 0 & 0 & 2-X\end{vmatrix}
= -(X-1)^2(X-2).$$
Nous avons donc deux valeurs propres qui sont $\lambda_1 = 1$ et $\lambda_2 = 2$. 
La valeur propre $1$ est de multiplicité $m_1=2$, alors que, pour la valeur propre $2$, $m_2=1$.

  \item On note $N_1 = \Ker(A-I_3)^2$ et $N_2 = \Ker(A-2I_3)$. 
 L'espace vectoriel $\Rr^3$ s'écrit comme somme directe
$$\Rr^3=\Ker(A-I_3)^2\oplus\Ker(A-2I_3).$$
Déterminons ces sous-espaces caractéristiques. 
  \begin{itemize}
    \item Calcul de $N_1 = \Ker(A-I_3)^2$.
    
  On sait que c'est un espace vectoriel de dimension $m_1=2$. On calcule
$$A-I_3 = \begin{pmatrix}0&1&1\cr0&0&1\cr0&0&1\end{pmatrix}\qquad \qquad 
(A-I_3)^2 = \begin{pmatrix}0&0&2\cr0&0&1\cr0&0&1\end{pmatrix}.$$
Ainsi $N_2 = \Ker(A-I_3)^2$ est le plan vectoriel engendré par les vecteurs $v_1=(1,0,0)$ et $v_2=(0,1,0)$.
Donc 
$$N_1=\Ker(A-I_3)^2=\Rr v_1 + \Rr v_2.$$

Au passage, notons que la matrice $A$ n'est pas diagonalisable : en effet,
la valeur propre $1$ est de multiplicité $2$, mais 
$E_1 = \Ker(A-I_3) =\{v \in \Rr^3 \mid Av=v\}$ est de dimension seulement $1$ (en fait
$E_1 = \Rr v_1$).

    \item Calcul de $N_2 = \Ker(A-2I_3)$.
    
    On sait que c'est un espace vectoriel de dimension $m_2=1$.
    Pour déterminer le noyau $\Ker(A-2I_3)=\{v\in \Rr^3 \mid Av=2v\}$, si $v=(x,y,z)$, on résout :
$$\left\{\begin{array}{rcl}
x+y+z&=& 2x\cr
y+z&=&2y\cr
2z&=&2z
\end{array}\right.\iff\left\{\begin{array}{rcl}
-x+y+z&=&0\cr
-y+z&=&0
\end{array}\right.\iff\left\{\begin{array}{rcl}
x&=&2z\cr
y&=& z
\end{array}\right.$$
Le sous-espace $N_2=\Ker(A-2I_3)$ est donc la droite vectorielle engendrée par le vecteur $v_3=(2,1,1)$ :
$N_2 = \Rr v_3$.


    
    \item La famille $\mathcal{B} = (v_1,v_2,v_3)$ est une base de $\Rr^3$ :
    $$\Rr^3 = \underbrace{\Rr v_1 \oplus \Rr v_2}_{N_1} \oplus \underbrace{\Rr v_3}_{N_2}.$$
  
  \end{itemize}

  \item 
  On définit l'endomorphisme $d$  par $d(v_1)=v_1$, $d(v_2)= v_2$ (car $v_1,v_2 \in N_1$)
  et $d(v_3) = 2v_3$ (car $v_3 \in N_2$).
  Dans la base $\mathcal{B}$, la matrice de $d$ est donc la matrice diagonale
  $D = \left(\begin{smallmatrix}1&0&0\cr 0&1&0\cr 0&0&2\end{smallmatrix}\right).$
  Or nous voulons la matrice de $d$ dans la base canonique $\mathcal{B}_0 
  = (e_1,e_2,e_3)$. (Les calculs vont être assez simples car $e_1=v_1$ et $e_2=v_2$.)
  
  \begin{itemize}
    \item $d(e_1) = d(1,0,0) = (1,0,0) = e_1$.
    \item $d(e_2) = d(0,1,0) = (0,1,0) = e_2$.
    \item On a $v_3 = (2,1,1) = 2e_1+e_2+e_3$ et aussi $e_3 = (0,0,1) = -2v_1-v_2+v_3$.
    Donc 
    \begin{align*}
    d(e_3) &= d(-2v_1-v_2+v_3) 
    = -2d(v_1)-d(v_2) +d(v_3)
    = -2v_1-v_2+2v_3 \\
    &= -2e_1-e_2+2(2e_1+e_2+e_3)
    = 2e_1 + e_2 + 2e_3.
    \end{align*}
  \end{itemize}
    Donc
    $$\Delta = \Mat_{\mathcal{B}_0} d = 
    \bordermatrix{
       &d(e_1)&d(e_2)&d(e_3)\cr
    e_1&1&0&2\cr 
    e_2&0&1&1\cr 
    e_3&0&0&2
    }
    .$$
    
  \item  On pose 
  $$N = A - \Delta
  = \begin{pmatrix}0&1&-1\cr 0&0&0\cr 0&0&0\end{pmatrix}.$$
  La décomposition de Dunford est $A = \Delta + N$. 
  La preuve du théorème de la décomposition affirme que $\Delta$ est diagonalisable,
  $N$ est nilpotente et $\Delta N = N \Delta$ (c'est un bon exercice de le vérifier
  à la main).
  
  \item On note $P$ la matrice de passage de la base $\mathcal{B}_0$ vers la base $\mathcal{B}$. $P$ contient donc, en colonnes, les vecteurs de la nouvelle base $\mathcal{B} = (v_1,v_2,v_3)$ exprimés dans l'ancienne base 
 $\mathcal{B}_0 = (e_1,e_2,e_3)$.  Comme
 $v_1 = e_1$, $v_2 = e_2$ et $v_3 =  2e_1+e_2+e_3$, alors
 $$P=\bordermatrix{
    &v_1&v_2&v_3\cr
 e_1&1&0&2\cr  
 e_2&0&1&1\cr 
 e_3&0&0&1}\qquad \text{ et on calcule } \quad P^{-1}=\begin{pmatrix}1&0&-2\cr 0&1&-1\cr 0&0&1\end{pmatrix}.$$
 Si besoin, on peut diagonaliser $\Delta$ : 
 $$D = P^{-1} \Delta P = \begin{pmatrix}1&0&0\cr 0&1&0\cr 0&0&2\end{pmatrix}$$
 
\end{enumerate}
\end{exemple}


\textbf{Remarque.} $D$ et $N$ sont uniques, mais il y a plusieurs choix possibles pour les  vecteurs $v_i$ et donc pour la matrice $P$.


\bigskip
%----------------------------------------------------
\evidence{Pratique de la décomposition (suite)}


\begin{exemple}
Calculons la décomposition de Dunford de
$$A=\begin{pmatrix}2&1&-1\cr 3&3&-4\cr 3&1&-2\end{pmatrix} \in M_3(\Rr).$$

\begin{enumerate}
  \item On trouve $\chi_A(X)=-(X+1)(X-2)^2$.
 La valeur propre $-1$ est de multiplicité $1$, et la valeur propre $2$ est de multiplicité $2$.  
 
  \item 
  \begin{itemize}
    \item On calcule $N_{-1}=\Ker(A+I_3) = \Rr v_1$ où $v_1 =(0,1,1)$ (c'est bien un espace vectoriel de dimension $m_{-1} = 1$).
    
    \item Calcul de $N_2=\Ker(A-2I_3)^2$ qui va être de dimension $m_2=2$ :
    
    $$A-2I_3 = \begin{pmatrix}0&1&-1\cr 3&1&-4\cr 3&1&-4\end{pmatrix}
    \qquad \text{ et } \qquad 
    (A-2I_3)^2 = \begin{pmatrix}0&0&0\cr -9&0&9\cr -9&0&9\end{pmatrix}.$$
    Pour une base de $N_2$, on choisit d'abord 
    $v_2 \in E_2 = \Ker(A-2I_3) \subset N_2$, 
    par exemple $v_2 = (1,1,1)$.
    On cherche $v_3 \in N_2$, linéairement indépendant de $v_2$.
    Par exemple, $v_3 = (1,0,1)$. 
    
    \item La famille $\mathcal{B} = (v_1,v_2,v_3)$ est une base de $\Rr^3$ :
    $$\Rr^3 = \underbrace{\Rr v_1 }_{N_{-1}} \oplus \underbrace{\Rr v_2 \oplus \Rr v_3}_{N_2}.$$
    
  \end{itemize}  
  
  \item On note $\mathcal{B}_0 = (e_1,e_2,e_3)$ la base canonique de $\Rr^3$.
  La matrice de passage $P$ de la base $\mathcal{B}_0$ vers la base $\mathcal{B}$ s'obtient en écrivant les vecteurs $v_i$ en colonnes :
  $$P=\begin{pmatrix}0&1&1\cr 1&1&0\cr1&1&1\end{pmatrix}
  \qquad \text{ et donc } \qquad 
  P^{-1} = \begin{pmatrix}-1 & 0 & 1 \\1 & 1 & -1 \\0 & -1 & 1\end{pmatrix}.$$
  
  
  \item  On définit l'endomorphisme $d$  par $d(v_1)=-v_1$ (car $v_1 \in N_{-1}$),
  et $d(v_2)= 2v_2$, $d(v_3) = 2v_3$ (car $v_2,v_3 \in N_2$).
  Dans la base $\mathcal{B} = (v_1,v_2,v_3)$, la matrice de $d$ est la matrice diagonale
  $$D = \begin{pmatrix}-1&0&0\cr 0&2&0\cr 0&0&2\end{pmatrix}.$$
  
  
  La matrice de $d$ dans la base $\mathcal{B}_0$ est
  obtenue en exprimant $d(e_i)$ dans la base $(e_1,e_2,e_3)$ ou, ce qui revient au même, par 
  $$\Delta = P D P^{-1} = \begin{pmatrix}2 & 0 & 0 \\3 & 2 & -3 \\3 & 0 & -1\end{pmatrix}.$$
  
  \item On pose 
  $$N = A - \Delta=\begin{pmatrix}  0 & 1 & -1 \\0 & 1 & -1 \\0 & 1 & -1\end{pmatrix}.$$
  La décomposition de Dunford est $A = \Delta + N$. 
  On a bien $\Delta$ diagonalisable car $D = P^{-1}\Delta P$. Pour vous rassurer, vérifiez 
  que $N^2 = 0$ et que $\Delta N = N\Delta$. 
  \end{enumerate}
\end{exemple}

 
\bigskip
%----------------------------------------------------
\evidence{Application au calcul de puissance.}


La décomposition de Dunford est utile pour calculer les puissances d'une matrice.
Nous verrons d'autres applications dans le chapitre \og{}Systèmes différentiels\fg{}.
Voyons les étapes pour calculer $A^p$, où  $A\in M_n(\Kk)$ :
\begin{enumerate}
  \item \'Ecrire la décomposition de Dunford $A = \Delta + N$.  
  
  \item Diagonaliser $\Delta$ : $D = P^{-1}\Delta P$ avec $D$ matrice diagonale.
  Comme $D$ est une matrice diagonale, on calcule facilement $D^k$, pour tout $k\ge0$.
  
  \item On note $N' = P^{-1}NP$. La matrice $N'$ est encore une matrice nilpotente. 
  On calcule les puissances successives $N'^2$, $N'^3$,\ldots{} sachant qu'à partir d'un certain rang tous les $N'^k$ sont nuls.
  
  \item Comme $D$ et $N'$ commutent (car $\Delta$ et $N$ commutent), alors 
  on applique la formule du binôme de Newton :
$$(D+N')^p=\sum_{k=0}^p \binom{p}{k} D^{p-k} N'^{k}.$$
On a vu qu'à partir d'un certain rang les matrices $N'^k$ sont toutes nulles. La somme a donc peu de termes non nuls.

  \item On a $A = \Delta + N = P D P^{-1} + P N' P^{-1} = P(D+N')P^{-1}$.
  Donc $A^p = \big(P(D+N')P^{-1}\big)^p = P (D+N')^p P^{-1}$.
\end{enumerate}


Reprenons l'exemple précédent.

\begin{exemple}
Calculons $A^p$, quel que soit $p\in \Nn$, pour 
$$A=\begin{pmatrix}2&1&-1\cr 3&3&-4\cr 3&1&-2\end{pmatrix} \in M_3(\Rr).$$

\begin{enumerate}
  \item Nous avons déjà calculé la décomposition de Dunford de $A$ :
  $$\Delta = \begin{pmatrix}2 & 0 & 0 \\3 & 2 & -3 \\3 & 0 & -1\end{pmatrix}\qquad \qquad
N = \begin{pmatrix}  0 & 1 & -1 \\0 & 1 & -1 \\0 & 1 & -1\end{pmatrix}.$$
  \item On a aussi calculé la matrice de passage $P$ qui diagonalise $\Delta$.
 $$\text{Avec }\quad P=\begin{pmatrix}0&1&1\cr 1&1&0\cr1&1&1\end{pmatrix} \qquad \text{ on a } \quad
   D = P^{-1}\Delta P = \begin{pmatrix}-1&0&0\cr 0&2&0\cr 0&0&2\end{pmatrix}.$$
   Et donc, pour tout $k\ge0$ :
   $$D^k = \begin{pmatrix}(-1)^k&0&0\cr 0&2^k&0\cr 0&0&2^k\end{pmatrix}.$$
  
  \item Pour la partie nilpotente, on pose
   $$N' = P^{-1}NP = \begin{pmatrix}
   0 & 0 & 0 \\0 & 0 & -1 \\0 & 0 & 0\end{pmatrix} \qquad \text{ avec } N'^2 = 0.$$

\item La formule du binôme de Newton se réduit donc à seulement deux termes :
$$(D+N')^p = D^p + \binom{p}{1} D^{p-1}N' + \binom{p}{2} D^{p-2}N'^2 + \cdots
= D^p + pD^{p-1}N'$$
Donc 
$$(D+N')^p =
 \begin{pmatrix}(-1)^p&0&0\cr 0&2^p&0\cr 0&0&2^p\end{pmatrix}
+p\begin{pmatrix}0&0&0\cr 0&0&-2^{p-1}\cr 0&0&0\end{pmatrix}
= \begin{pmatrix}(-1)^p&0&0\cr 0&2^p&-p2^{p-1}\cr 0&0&2^p\end{pmatrix}.$$

  \item Ainsi, pour $p\ge0$ :
  $$A^p = P (D+N')^p P^{-1}
  = \begin{pmatrix}
  2^{p} & p2^{p - 1}  & -p2^{p - 1}  \\
2^{p} - (-1)^{p} & p2^{p - 1}  + 2^{p} & -p2^{p - 1} 
- 2^{p} + (-1)^{p} \\
2^{p} - (-1)^{p} & p2^{p - 1}  & -p2^{p - 1} +
(-1)^{p}\end{pmatrix}$$
\end{enumerate}
\end{exemple}


%----------------------------------------------------
\begin{miniexercices}
\sauteligne
\begin{enumerate}
\item Montrer que la décomposition de Dunford de la matrice
$A = \left(\begin{smallmatrix}
-6 & -5 & -5 \\
2 & 0 & 6 \\
-2 & 4 & -2
  \end{smallmatrix}\right)$
  est $A = \Delta + N$ avec
  $\Delta = \left(\begin{smallmatrix}
-6 & -5 & -5 \\
0 & -1 & 5 \\
0 & 5 & -1
  \end{smallmatrix}\right)$
  et $N = \left(\begin{smallmatrix}
0 & 0 & 0 \\
2 & 1 & 1 \\
-2 & -1 & -1
  \end{smallmatrix}\right)$.
  Même exercice avec
$$A' = \left(\begin{smallmatrix}
1 & 2 & 3 \\
0 & -2 & 3 \\
0 & 0 & 1
  \end{smallmatrix}\right)\qquad
\Delta' =  \left(\begin{smallmatrix}
1 & 2 & -2 \\
0 & -2 & 3 \\
0 & 0 & 1
  \end{smallmatrix}\right) \qquad
N' = \left(\begin{smallmatrix}
0 & 0 & 5 \\
0 & 0 & 0 \\
0 & 0 & 0
  \end{smallmatrix}\right). 
  $$
  
  
\item Calculer la décomposition de Dunford de 
$\left(\begin{smallmatrix}
-3 & 3 & 6 & -2 \\
-3 & 3 & 6 & -2 \\
1 & -1 & 0 & 0 \\
-7 & 9 & 10 & -4
  \end{smallmatrix}\right)$.
  Même exercice avec
 $\left(\begin{smallmatrix}
 1 & 0 & 3 & -2 \\
3 & 4 & -3 & 0 \\
3 & 0 & 1 & -2 \\
3 & 0 & -3 & 4
  \end{smallmatrix}\right)$.
  
  
  
  
  \item Montrer que la décomposition de Dunford de la matrice
$A = \left(\begin{smallmatrix}
7 & 1 \\
-1 & 5
  \end{smallmatrix}\right)$
  est $A = \Delta + N$ avec
  $\Delta = \left(\begin{smallmatrix}
6 & 0 \\
0 & 6
  \end{smallmatrix}\right)$
  et $N = \left(\begin{smallmatrix}
1 & 1 \\
-1 & -1
  \end{smallmatrix}\right)$.
  Calculer $A^p$ pour tout $p\ge0$. 
  
  Même exercice avec 
  $$A' = \left(\begin{smallmatrix}
8 & 4 & 0 \\
-1 & 4 & 0 \\
0 & 0 & -9
  \end{smallmatrix}\right)\qquad
\Delta' =  \left(\begin{smallmatrix}
6 & 0 & 0 \\
0 & 6 & 0 \\
0 & 0 & -9
  \end{smallmatrix}\right) \qquad
N' = \left(\begin{smallmatrix}
2 & 4 & 0 \\
-1 & -2 & 0 \\
0 & 0 & 0
  \end{smallmatrix}\right).  
  $$
\end{enumerate}
\end{miniexercices}



%%%%%%%%%%%%%%%%%%%%%%%%%%%%%%%%%%%%%%%%%%%%%%%%%%%%%
\section{Réduction de Jordan}


Nous allons montrer que toute matrice, dont le polynôme caractéristique est scindé, est semblable à une matrice diagonale par blocs, avec des blocs \og{}presque\fg{} diagonaux.


%----------------------------------------------------
\subsection{Blocs et matrices de Jordan}


\begin{definition}
Un \defi{bloc de Jordan} est une matrice de la forme
\[J(\lambda) = 
\begin{pmatrix}
\lambda&1&0&\hphantom{\;}0\hphantom{\;}\\
0&\lambda&\ddots&0\\
\vdots&\ddots&\ddots&1\\
0&\ldots&0&\lambda
\end{pmatrix}
\in M_p(\Kk)\]
avec $\lambda \in \Kk$ et $p \ge 1$.
\end{definition}

C'est donc une matrice triangulaire supérieure, avec des coefficients $\lambda$ sur la diagonale, des $1$ juste au-dessus de la diagonale, puis des $0$ encore au-dessus.


\begin{exercicecours}
Pour un bloc de Jordan de taille $p\times p$, calculer $(J(\lambda)-\lambda I_p)^k$ pour tout $k\ge 1$.
En particulier, montrer que, pour 
$(J(\lambda)-\lambda I_p)^{p-1}$, il ne reste plus qu'un coefficient $1$, tout en haut à droite, 
et que $(J(\lambda)-\lambda I_p)^{p}$ est la matrice nulle.
\end{exercicecours}


\begin{exercicecours}
Pour un bloc de Jordan $J(\lambda) \in M_p(\Kk)$,
montrer que son polynôme caractéristique 
est $(-1)^p(X-\lambda)^p$ et que son polynôme minimal 
est $(X-\lambda)^p$.
\end{exercicecours}


\begin{definition}
Une \defi{matrice de Jordan} est une matrice diagonale par blocs de la forme
\[\left(\begin{array}{c|c|c|c}
J_1(\lambda_1)&O&O&O\\\hline
O&J_2(\lambda_2)&O&O\\\hline
O&O&\ddots&O\\\hline
O&O&O&J_r(\lambda_r)
\end{array}\right)\]
où les $J_i(\lambda_i)$ sont des blocs de Jordan.
\end{definition}

Les blocs de Jordan peuvent être de tailles différentes, et les valeurs $\lambda_i \in \Kk$ sont quelconques (certaines d'entre elles peuvent être égales).
La notation $O$ désigne une matrice nulle (elles peuvent être de tailles différentes).


\begin{exemple}
Voici une matrice de Jordan : 



\[
\left(\begin{array}{cc|ccc|c|ccc}
-3	&	1	&	0	&	0	&	0	&	0	&	0	&	0	&	0\\
0	&	-3	&	0	&	0	&	0	&	0	&	0	&	0	&	0\\ \hline
0	&	0	&	-3	&	1	&	0	&	0	&	0	&	0	&	0\\
0	&	0	&	0	&	-3	& 	1	&	0	&	0	&	0	&	0\\
0	&	0	&	0	&	0	& 	-3	&	0	&	0	&	0	&	0\\ \hline
0	&	0	&	0	&	0	& 	0	&	2	&	0	&	0	&	0\\ \hline
0	&	0	&	0	&	0	&	0	&	0	&	5	&	1	&	0\\
0	&	0	&	0	&	0	&	0	&	0	&	0	&	5	& 	1\\
0	&	0	&	0	&	0	&	0	&	0	&	0	&	0	& 	5\\
\end{array}\right)
\]

\bigskip
Elle est formée de :
\begin{itemize}
  \item un bloc de Jordan $2 \times 2$ associé à la valeur propre $-3$,
  \item un bloc de Jordan $3 \times 3$ associé à cette même valeur $-3$, 
  \item un bloc de Jordan $1 \times 1$ associé à la valeur $2$, 
  \item un bloc de Jordan $3 \times 3$ associé à la valeur $5$. 
\end{itemize}
\end{exemple}



%----------------------------------------------------
\subsection{\'Enoncé}

\begin{theoreme}
Si $A \in M_n(\Kk)$ a son polynôme caractéristique scindé sur $\Kk$, alors $A$ est semblable (sur $\Kk$) à une matrice de Jordan, appelée \defi{réduite de Jordan} de $A$. Il existe donc $P \in M_n(\Kk)$ inversible telle que
\[ P^{-1}AP = \left(\begin{array}{c|c|c}
J_1&O&O\\\hline
O&\ddots&O\\\hline
O&O&J_r
\end{array}\right)\]
où les $J_i$ sont des blocs de Jordan.
\end{theoreme}

Ce qui se reformule aussi :
\begin{theoreme}
Soit $f$ un endomorphisme de $E$ dont le polynôme caractéristique $\chi_f(X)$ est scindé sur $\Kk$.

Il existe une base $\mathcal{B}$ de $E$ où la matrice de $f$ est de Jordan, c'est-à-dire  :
\[ \Mat_{\mathcal{B}} (f) = 
\left(\begin{array}{c|c|c}
J_1&O&O\\\hline
O&\ddots&O\\\hline
O&O&J_r
\end{array}\right)\]
\end{theoreme}

Nous admettons ces théorèmes, mais nous verrons sur des exemples comment obtenir la matrice de Jordan.

\bigskip


Voici des remarques importantes :
\begin{itemize}
  \item Les $\lambda$ qui apparaissent dans les blocs de Jordan sont les valeurs propres de $A$ (ou de $f$) et donc les 
racines du polynôme caractéristique.

  \item Une même valeur $\lambda$ peut apparaître dans plusieurs
  blocs différents.
    
  \item En particulier, ce théorème s'applique à toutes les matrices complexes.  
  
  \item \textbf{Unicité.} Cette décomposition est unique dans le sens où le nombre et la taille des blocs de Jordan ne dépendent que de $A$ (ou de $f$). Par contre, on s'autorise à permuter les blocs de Jordan entre eux.

\end{itemize}

\bigskip

Voici d'autres remarques qui découlent du théorème :
\begin{itemize}
  \item Le nombre de blocs associés à la valeur propre $\lambda$ est égal à
  la dimension du sous-espace propre $E_\lambda$.
    
  \item La somme des tailles des blocs de Jordan associés à $\lambda$
  est la multiplicité de $\lambda$ comme racine du polynôme caractéristique.
  
  \item La taille du plus grand bloc de Jordan associé à $\lambda$ est la multiplicité de $\lambda$ comme racine du polynôme minimal.
\end{itemize}


%----------------------------------------------------
\subsection{Exemples}

Voici une méthode basique pour trouver la réduite de Jordan d'une matrice $A \in M_n(\Kk)$, 
ainsi qu'une matrice de passage :

\begin{itemize}
  \item Calculer le polynôme caractéristique et les valeurs propres de $A$.
  
  \item Pour chaque valeur propre $\lambda$, calculer le sous-espace propre
  $E_\lambda = \Ker(A-\lambda I_n)$ et trouver une base de $E_\lambda$.
  Le nombre de blocs de Jordan associés à $\lambda$ est $\dim E_\lambda$.
    
  \item Pour chaque vecteur propre de la base de $E_\lambda$, on construit le bloc de Jordan associé :
  \begin{itemize}
      \item Si $v_1 \in E_\lambda$ est un vecteur propre de la base de $E_\lambda$, alors on cherche
  $v_2 \in \Kk^n$ tel que $(A-\lambda I_n)v_2 = v_1$. 
      \item Puis on cherche s'il existe
  $v_3 \in \Kk^n$ tel que $(A-\lambda I_n)v_3 = v_2$.
      \item On arrête le processus lorsqu'il n'y pas de solution. 
      \item On a $Av_1 = \lambda v_1$, puis $Av_2 = v_1 + \lambda v_2$, \ldots,
  $Av_p = v_{p-1} + \lambda v_p$. 
      \item Donc, dans le sous-espace engendré par ces $(v_1,v_2,\ldots,v_p)$, la matrice associée à $A$, dans cette base, est exactement le bloc de Jordan :
  $$J(\lambda) = 
  \bordermatrix{
     & Av_1    & Av_2 & \cdots & Av_p \cr
  v_1& \lambda&1&0&0 \cr
  v_2& 0&\lambda&\ddots&0\cr
   \vdots  & \vdots&\ddots&\ddots&1 \cr
  v_p& 0&\ldots&0&\lambda }
  $$ 
  
      \item On peut aussi savoir quand s'arrêter en utilisant le fait que le bloc de Jordan est toujours d'une taille $p$ inférieure ou égale à la multiplicité de $\lambda$
  comme racine du polynôme caractéristique (et même du polynôme minimal).
  \end{itemize}
  
  
  \item On recommence avec $v'_1$, un autre vecteur de la base de $E_\lambda$ : on construit $v'_2$, $v'_3$,\ldots{} ce qui conduit à un autre bloc de Jordan pour la valeur $\lambda$. On procède ainsi de suite pour tous les vecteurs de la base de $E_\lambda$ et ensuite bien sûr pour les autres valeurs propres. 
\end{itemize}  
 
  


\begin{exemple}
Soit 
$$A = \begin{pmatrix}
4 & 3 & -2 \\
-3 & -1 & 3 \\
2 & 3 & 0
\end{pmatrix}
\in M_3(\Rr).$$
Calculons sa réduite de Jordan $J$ et une matrice de passage $P$ telle que 
$P^{-1}AP = J$.

\begin{enumerate}
  \item On commence par calculer le polynôme caractéristique de $A$ :
  $$\chi_A(X) = \det(A-XI_3) = 
  \begin{vmatrix}
  4-X & 3 & -2 \\
  -3 & -1-X & 3 \\
  2 & 3 & -X  
  \end{vmatrix}
  = - (X+1)(X-2)^2$$
  
  Il y a donc deux valeurs propres : $-1$ et $2$.
  
  \item \textbf{Valeur propre $-1$.}
  
  La valeur propre $-1$ est de multiplicité $1$. Le sous-espace propre associé $E_{-1} = \Ker(A+I_3) = \{ v \in \Rr^3 \mid Av = -v\}$ sera de dimension $1$. Après calculs, on trouve que $E_{-1} = \Rr v_1$ où
  $v_1 = (-1,1,-1)$. Comme la multiplicité de $-1$ comme racine de $\chi_A(X)$ est $1$, alors
  la valeur propre $-1$ sera juste associée à un bloc de Jordan de taille $1\times 1$.
  
  \item \textbf{Valeur propre $2$.}
  
  La valeur propre $2$ est de multiplicité $2$. Il faut déterminer le sous-espace propre associé $E_{2} = \Ker(A-2I_3) = \{ v \in \Rr^3 \mid Av = 2v\}$. 
  Après calculs, on trouve que $E_{2} = \Rr v_2$ où
  $v_2 = (1,0,1)$. Comme $E_2$ est un espace vectoriel de dimension $1$, alors que $2$ est racine de multiplicité $2$, la matrice $A$ n'est pas diagonalisable et on sait alors que la valeur propre $2$ sera associée à un bloc de Jordan de taille $2\times 2$.

  
  \item \textbf{Bloc de Jordan.}

  On cherche $v_3 \in \Rr^3$ tel que $(A-2I_3)v_3 = v_2$.
  Si $v_3 = (x,y,z)$ alors :
  $$(A-2I_3)v_3 = v_2 
  \iff 
  \begin{pmatrix}
  2 & 3 & -2 \\
  -3 & -3 & 3 \\
  2 & 3 & -2
  \end{pmatrix}
  \begin{pmatrix}
  x \\ y \\ z
  \end{pmatrix}
  = \begin{pmatrix}1\\0 \\1\end{pmatrix}
  \iff   
  \left\{\begin{array}{rcl}
  2x  + 3y -2z &=& 1\\
  -3x -3y + 3z &=& 0 \\
  2x  + 3y -2z &=& 1
  \end{array}\right.
  $$
  $$
  \iff
  \left\{\begin{array}{rcl}
  2x  + 3y -2z &=& 1\\
  x +y - z &=& 0 \\
  \end{array}\right.
 \iff
  \left\{\begin{array}{rcl}
  2x  + 3y &=& 1+2z\\
  x + y  &=& z \\
  \end{array}\right. 
  \iff
 \left\{\begin{array}{rcl}
  x &=& -1+z\\
  y &=& 1 \\
  \end{array}\right.   
  $$
  En prenant par exemple $z=0$ (n'importe quelle valeur conviendrait), on choisit
  $v_3 = (-1,1,0)$.
  
  
  \item \textbf{Matrice de Jordan.}
  
  Dans la base $(v_1,v_2,v_3)$, on a 
  $Av_1 = -v_1$, $A v_ 2 = 2v_2$, et
  comme $(A-2I_3) v_3 = v_2$ alors $A v_3 = v_2 + 2v_3$.
  
  La matrice associée à $A$ dans la base  $(v_1,v_2,v_3)$ est
  donc :
  $$J = %\begin{pmatrix}
  \bordermatrix{
  & Av_1 & Av_2 & Av_3 \cr
  v_1&-1 & 0 & 0 \cr
  v_2&0 & 2 & 1 \cr
  v_3&0 & 0 & 2}
  %\end{pmatrix}
  $$

  Autrement dit, $J = P^{-1}AP$ où $P$ est la matrice dont les colonnes sont les vecteurs $v_1, v_2, v_3$ exprimés dans la base canonique :
$$P =  \begin{pmatrix}
-1 & 1 & -1 \\
1 & 0 & 1 \\
-1 & 1 & 0
\end{pmatrix} \qquad 
\text{ et on a } \quad
P^{-1} = \begin{pmatrix}
1 & 1 & -1 \\
1 & 1 & 0 \\
-1 & 0 & 1
\end{pmatrix}.
$$
%
%  \item \textbf{Seconde méthode.}
%  
%  On calcule $A-2I_3$ et $(A-2I_3)^2$ (on s'arrête là car la multiplicité de la valeur propre $2$ est $2$) :
%$$A-2I_3 =  \begin{pmatrix}
%
%2 & 3 & -2 \\
%-3 & -3 & 3 \\
%2 & 3 & -2  
%\end{pmatrix} \qquad 
%\text{ et } \quad 
%(A-2I_3)^2 =  \begin{pmatrix}
% -9 & -9 & 9 \\
%9 & 9 & -9 \\
%-9 & -9 & 9
%\end{pmatrix}
%$$
%On cherche $v'_3$ qui soit dans  $\Ker (A-2I_3)^2$ mais pas dans $\Ker(A-2I_3)$.
%C'est-à-dire $(A-2I_3)^2v'_3 = 0$ mais $(A-2I_3) v'_3 \neq 0$.
%Par exemple $v'_3 = (2,2,4)$ convient.
%On pose 
%$$v'_2 = (A-2I_3)v'_3 = (2,0,2).$$
%
%Comme $(A-2I_3) v'_2 = (A-2I_3)^2 v'_3 = 0$, donc $Av'_2 = 2v'_2$.
%
%Comme  $(A-2I_3)v'_3 = v'_2$, alors $Av'_3 = v'_2 + 2v'_3$.
%La matrice associée à $A$ dans la base  $(v_1,v_2,v_3)$ est donc la même matrice de Jordan 
%qu'auparavant :
%  $$J = %\begin{pmatrix}
%  \bordermatrix{
%  & Av_1 & Av_2' & Av_3' \cr
%  v_1&-1 & 0 & 0 \cr
%  v_2'&0 & 2 & 1 \cr
%  v_3'&0 & 0 & 2}
%  %\end{pmatrix}
%  $$
%  Par contre la matrice de passage est différente :
%$$P' =  \begin{pmatrix}
%-1 & 2 & 2 \\
%1 & 0 & 2 \\
%-1 & 2 & 4
%\end{pmatrix} \qquad 
%\text{ et on a } \quad
%P'^{-1} = \frac12\begin{pmatrix}
%2 & 2 & -2 \\
%3 & 1 & -2 \\
%-1 & 0 & 1
%\end{pmatrix}
%$$  


\end{enumerate}
\end{exemple}

\bigskip

On recommence avec un exemple plus compliqué !

\begin{exemple}
Soit 
$$A = \begin{pmatrix}
5 & 0 & 4 & -2 & -3 \\
-2 & 3 & -3 & 2 & 4 \\
0 & 0 & 3 & 0 & 0 \\
0 & 0 & 0 & 3 & 1 \\
1 & 0 & 2 & -1 & 1
\end{pmatrix}
\in M_5(\Rr).$$
Calculons sa réduite de Jordan $J$.

\begin{enumerate}
  \item Le polynôme caractéristique de $A$ vaut
  $$\chi_A(X) = \det(A-XI_5) = \cdots = -(X-3)^5.$$
  
  Il y a donc une seule valeur propre : $\lambda = 3$.
  
  \item \textbf{Valeur propre $3$.}
  
  La valeur propre $3$ est de multiplicité $5$. 
  Il faut ensuite calculer le sous-espace propre associé : 
  $E_{3} = \Ker(A-3I_5) = \{ v \in \Rr^5 \mid Av = 3v\}$.
  On calcule $A-3I_5$, et on trouve une base $(v_1,v_3)$ de $E_3$ :
  $$v_1 =(0,1,0,0,0) \qquad \text{ et } \qquad  v_3 = (1,0,0,1,0).$$
  Ainsi $\dim E_3 = 2$ alors que la multiplicité de $\lambda$ est $5$. La matrice $A$ n'est pas diagonalisable. 
  Il y aura donc deux blocs de Jordan associés à la valeur propre $3$. (Cela peut être un de taille $1\times1$ avec un de taille $4\times 4$, ou bien $2\times 2$ avec $3\times 3$.)
  
  
  \item \textbf{Matrice de Jordan.}

  \begin{itemize}
    \item On cherche si on peut trouver $v_2 \in \Rr^5$ tel que $(A-3I_5)v_2 = v_1$.
  Une solution possible est $v_2 = (-2,0,1,0,0)$.
  Le processus s'arrête là, car il n'y a aucune solution $v$ au système $(A-3I_5)v=v_2$.
  (En effet, la troisième ligne de $A-3I_5$ est nulle, alors que la troisième coordonnée de $v_2$ ne l'est pas.) On obtient donc un bloc de Jordan $2\times2$.  
  
    \item On fait le même travail pour l'autre bloc de Jordan, en partant du
    vecteur $v_3 =  (1,0,0,1,0) \in E_3$.    
     On cherche maintenant $v_4 \in \Rr^5$ tel que $(A-3I_5)v_4 = v_3$.
    Après calculs, on trouve $v_4 = (2, 0, 0, 0, 1)$. 
    On cherche $v_5$ tel que $(A-3I_5)v_5 = v_4$. On trouve
    $v_5 = (-3, 0, 2, 0, 0)$. Le processus s'arrête là, ce qui correspond à un bloc de Jordan de taille 
    $3\times 3$.
    
  \end{itemize}
  On a $Av_1 = 3v_1$, $Av_2 = v_1+3v_2$, 
  $Av_3 = 3v_3$, $Av_4 = v_3+3v_4$, $Av_5 = v_4+3v_5$.
   
  Dans la base $(v_1,v_2,v_3,v_4,v_5)$, la matrice associée à $A$ est :
  $$J = \left(\begin{array}{cc|ccc}
  3 & 1 & 0 & 0 & 0 \\
0 & 3 & 0 & 0 & 0 \\\hline
0 & 0 & 3 & 1 & 0 \\
0 & 0 & 0 & 3 & 1 \\
0 & 0 & 0 & 0 & 3
  \end{array}\right)$$
  
Ainsi $J = P^{-1}AP$, où la matrice $P$ est la matrice dont les colonnes sont les vecteurs 
$v_1, \ldots, v_5$ exprimés dans la base canonique.


\end{enumerate}
\end{exemple}

%----------------------------------------------------
\subsection{Applications}

\begin{exemple}
Si $A \in M_n(\Cc)$, alors $A$ est semblable à sa transposée $A^T$.

En effet, il suffit de le vérifier lorsque $A$ est un bloc de Jordan.
\end{exemple}

\begin{exemple}
Si $N \in M_4(\Kk)$ est nilpotente (c'est-à-dire $N^4 = 0$), alors $N$ est semblable à une et une seule des $5$ matrices suivantes :
\[
\begin{pmatrix}
0&0&0&0\\
0&0&0&0\\
0&0&0&0\\
0&0&0&0
\end{pmatrix}
\quad 
\begin{pmatrix}
0&1&0&0\\
0&0&0&0\\
0&0&0&0\\
0&0&0&0
\end{pmatrix} \quad
\begin{pmatrix}
0&1&0&0\\
0&0&0&0\\
0&0&0&1\\
0&0&0&0
\end{pmatrix} \quad
\begin{pmatrix}
0&1&0&0\\
0&0&1&0\\
0&0&0&0\\
0&0&0&0
\end{pmatrix} \quad
\begin{pmatrix}
0&1&0&0\\
0&0&1&0\\
0&0&0&1\\
0&0&0&0
\end{pmatrix}
 \] 

Il y a une infinité de matrices nilpotentes $4 \times 4$, mais il n'y en a que $5$ à similitude près.

\end{exemple}

\begin{exemple}
Si $A \in M_3(\Kk)$ a pour polynôme caractéristique $ \chi_A(X) = -(X-1)(X-2)^2$, alors $A$ est semblable 
à l'une des deux matrices suivantes :
\[
\begin{pmatrix}
1&0&0\\
0&2&0\\
0&0&2
\end{pmatrix} \qquad
\begin{pmatrix}
1&0&0\\
0&2&1\\
0&0&2
\end{pmatrix}
\]
\end{exemple}


 
%----------------------------------------------------
\begin{miniexercices}
\sauteligne
\begin{enumerate}
  \item Montrer que $J(\lambda)-\mu I_p$ est inversible si $\mu \neq \lambda$.
  
  \item Montrer qu'une matrice de Jordan est diagonalisable 
  si et seulement si ses blocs sont tous de taille $1$.
  
  \item Montrer qu'un bloc de Jordan $J$ est semblable à sa transposée $J^T$.

  \item Calculer la réduction de Jordan de la matrice 
  $A = \left(\begin{smallmatrix}
  2 & 2 & -8 \\8 & -4 & -17 \\-1 & 1 & 0 \end{smallmatrix}\right)$, 
  c'est-à-dire trouver $P$ telle que $J = P^{-1}AP$ soit une matrice de Jordan.
  Même exercice avec :
  $$
  \left(\begin{smallmatrix} 
  3 & -1 & -1 \\ -1 & 5 & 1 \\ 2 & 0 & 4
  \end{smallmatrix}\right) \quad
  \left(\begin{smallmatrix} 
  -2 & 2 & -4 & 9 \\ -2 & 2 & -3 & 6 \\ 1 & -1 & 2 & -3 \\ 0 & 0 & 0 & 1
  \end{smallmatrix}\right) \quad
  \left(\begin{smallmatrix} 
  2 & -3 & 3 & -2 & 0 \\
  2 & -1 & 0 & 1 & -2 \\
  -3 & 0 & 5 & -6 & 3 \\
  -4 & 0 & 5 & -6 & 4 \\
  -2 & -3 & 7 & -6 & 4
  \end{smallmatrix}\right) \quad
  $$
  
    
  
  
\end{enumerate}
\end{miniexercices}






\auteurs{

D'après un cours de Sandra Delaunay et un cours d'Alexis Tchoudjem.

Revu et augmenté par Arnaud Bodin.

Relu par Stéphanie Bodin et Vianney Combet.
}


\finchapitre 
\end{document}


