
\documentclass[11pt, class=report,crop=false]{standalone}
\usepackage[screen]{../exo7book}

% Commandes spécifiques à ce chapitre
%\newcommand{\Sp}{\text{sp}}
%\newcommand{\GL}{GL}

\begin{document}

%====================================================================
\chapitre{Valeurs propres, vecteurs propres}
%====================================================================


Dans ce chapitre, nous allons définir et étudier les valeurs propres et 
les vecteurs propres d'une matrice.
Ce chapitre peut être vu comme un cours minimal pour comprendre la diagonalisation ou comme une 
introduction à la théorie de la réduction des endomorphismes.

\bigskip

\textbf{Notations.}


$\Kk$ est un corps. Dans les exemples de ce chapitre, $\Kk$ sera $\Rr$ ou $\Cc$.
Les matrices seront des éléments de $M_n(\Kk)$, c'est-à-dire des matrices carrées, de taille $n\times n$, à coefficients dans $\Kk$. 





%%%%%%%%%%%%%%%%%%%%%%%%%%%%%%%%%%%%%%%%%%%%%%%%%%%%%
\section{Valeurs propres et vecteurs propres}


%----------------------------------------------------
\subsection{Motivation}

Voici deux transformations simples définies par une matrice :

\begin{enumerate}
  \item $$h  : \begin{pmatrix}x\cr y\end{pmatrix} 
\mapsto \begin{pmatrix}2&0\cr0&2\end{pmatrix} \begin{pmatrix}x\cr y\end{pmatrix}
= \begin{pmatrix}2x\cr 2y\end{pmatrix}$$

L'application $h$ est une homothétie de $\Rr^2$ (centrée à l'origine). Si $D$ est une droite passant par l'origine, alors elle est globalement invariante par cette transformation, c'est-à-dire si
$P \in D$ alors $h(P) \in D$ (mais on n'a pas $h(P)=P$). 
On retient ici que n'importe quel vecteur $\left(\begin{smallmatrix}x\cr y\end{smallmatrix}\right)$ est envoyé sur son double $2\left(\begin{smallmatrix}x\cr y\end{smallmatrix}\right)$.

  \item $$k  : \begin{pmatrix}x\cr y\end{pmatrix} 
\mapsto \begin{pmatrix}2&0\cr0&3\end{pmatrix} \begin{pmatrix}x\cr y\end{pmatrix}
= \begin{pmatrix}2x\cr 3y\end{pmatrix}$$
L'application $k$ n'est plus une homothétie. Cependant l'axe $(Ox)$ est globalement invariant par $k$ ; 
de même, l'axe $(Oy)$ est globalement invariant.
On retient qu'un vecteur du type $\left(\begin{smallmatrix}x\cr 0\end{smallmatrix}\right)$ est envoyé sur son double $2\left(\begin{smallmatrix}x\cr 0\end{smallmatrix}\right)$, alors qu'un vecteur du type $\left(\begin{smallmatrix}0\cr y\end{smallmatrix}\right)$ est envoyé sur son triple $3\left(\begin{smallmatrix}0\cr y\end{smallmatrix}\right)$.
\end{enumerate}


Pour une matrice quelconque, il s'agit de voir comment on se ramène à ces situations géométriques simples. C'est ce qui nous amène à la notion de vecteurs propres et valeurs propres.


%----------------------------------------------------
\subsection{Définitions}



\begin{definition}
Soit $A \in M_n(\Kk)$.

\begin{itemize}
  \item $\lambda$ est dite \defi{valeur propre} de la matrice $A$ s'il
existe un vecteur non nul $X \in \Kk^n$ tel que 
\mybox{$AX = \lambda X$.}
  
  \item Le vecteur $X$ est alors appelé \defi{vecteur propre} de $A$ associé à la valeur propre $\lambda$.
\end{itemize}
\end{definition}

%----------------------------------------------------
\subsection{Exemples}
 
 
\begin{exemple}
Soit $A \in M_3(\Rr)$ la matrice 
$$A = \begin{pmatrix}
1 & 3 & 3 \\
-2 & 11 & -2 \\
8 & -7 & 6
\end{pmatrix}.$$
\begin{itemize}
  \item Vérifions que $X_1 = \begin{pmatrix}-1\\0\\1\end{pmatrix}$ 
  est vecteur propre de $A$.
  
  En effet,
 $$A X_1 = \begin{pmatrix}
1 & 3 & 3 \\
-2 & 11 & -2 \\
8 & -7 & 6
\end{pmatrix} 
\begin{pmatrix}-1\\0\\1\end{pmatrix}
= \begin{pmatrix}2\\0\\-2\end{pmatrix}
= -2 \begin{pmatrix}-1\\0\\1\end{pmatrix}
= -2 X_1.$$

Donc $X_1$ est un vecteur propre de $A$ associé à la valeur propre $\lambda_1 = -2$.

 \item Vérifions que $X_2 = \begin{pmatrix}0\\1\\-1\end{pmatrix}$ 
  est vecteur propre de $A$.
  
  On calcule $AX_2$ et on vérifie que :
  $$AX_2 = 13 X_2$$
  Donc $X_2$ est un vecteur propre de $A$ associé à la valeur propre $\lambda_2 = 13$.

  \item Vérifions que $\lambda_3 = 7$ est valeur propre de $A$.
  
  Il s'agit donc de trouver un vecteur $X_3 = \begin{pmatrix}x_1\\x_2\\x_3\end{pmatrix}$
  tel que $AX_3 = 7 X_3$.
  
  \begin{align*}
  AX_3 = 7 X_3 
    & \iff  \begin{pmatrix}
1 & 3 & 3 \\
-2 & 11 & -2 \\
8 & -7 & 6
\end{pmatrix} 
\begin{pmatrix}x_1\\x_2\\x_3\end{pmatrix}
= 7\begin{pmatrix}x_1\\x_2\\x_3\end{pmatrix} \\
    & \iff  
\begin{pmatrix}
x_1 + 3x_2 + 3x_3 \\
-2x_1 + 11x_2  -2x_3 \\
8x_1 -7x_2 + 6x_3\end{pmatrix}
= \begin{pmatrix}7x_1\\7x_2\\7x_3\end{pmatrix} \\
& \iff  
\left\{ 
\begin{array}{rcl}
-6x_1 + 3x_2 + 3x_3 &=& 0\\
-2x_1 + 4x_2 - 2x_3 &=& 0\\
8x_1 -7x_2 - x_3 &=& 0
\end{array}
\right.
\end{align*}

On résout ce système linéaire et on trouve comme ensemble de solutions :
$\left\{ 
\left(\begin{smallmatrix} t \\ t \\ t \end{smallmatrix}\right) \mid t \in \Rr
\right\}$.
Autrement dit, les solutions sont engendrées par le vecteur 
$X_3 = \begin{pmatrix}1\\1\\1\end{pmatrix}$.

On vient de calculer que $AX_3 = 7X_3$.
Ainsi $X_3$ est un vecteur propre de $A$ associé à la valeur propre $\lambda_3 = 7$.
\end{itemize}  
\end{exemple}
  


\begin{exemple}
Soit
$$A = \begin{pmatrix}0 & 1\\1 &1\end{pmatrix}.$$
Le réel $\alpha =\frac{1+\sqrt{5}}{2}$ est une valeur propre de $A$. En effet :
\[A\begin{pmatrix}1\\\alpha\end{pmatrix}
= \alpha \begin{pmatrix}1\\\alpha\end{pmatrix}\]
\end{exemple}



\begin{exemple}
Soit 
$$A_\theta = \begin{pmatrix}
\cos\theta & -\sin\theta & 0 \\
\sin\theta &  \cos\theta & 0 \\
0          & 0           & 1 \\
\end{pmatrix}$$
la matrice de la rotation de l'espace d'angle $\theta$ et d'axe $(Oz)$.
Soit $X_3 = \left( \begin{smallmatrix} 0 \\ 0 \\ 1 \end{smallmatrix}\right)$.  
Alors $A_\theta X_3=X_3$. Donc $X_3$ est un vecteur propre de $A_\theta$ et la valeur propre associée est $1$.
\end{exemple}


\begin{exemple}
Soit
$$A = \begin{pmatrix}0 & 1\\-1 &-1\end{pmatrix}.$$
Le complexe $j =-\frac{1}{2} + \ii \frac{\sqrt{3}}{2} = e^{\ii\frac{2\pi}{3}}$ est une valeur propre de $A$. En effet :
\[A\begin{pmatrix}1\\j\end{pmatrix} 
= j\begin{pmatrix}1\\j\end{pmatrix}\]
\end{exemple}


%----------------------------------------------------
\subsection{Cas d'une matrice diagonale}


Le cas idéal est celui d'une matrice diagonale. 
Il est en effet très facile de lui trouver des valeurs propres et des vecteurs propres. 
%Il n'y a rien à faire pour trouver les valeurs propres et les vecteurs propres. 
C'est le but de la \og{}diagonalisation\fg{} de se ramener à ce cas !  

\begin{exemple}[Cas d'une matrice diagonale]
Soit $A$ la matrice diagonale 
$$A=\begin{pmatrix}
\lambda_{1}& 0& \cdots &  \cdots & 0\cr
0& \lambda_{2}& 0&  \cdots &0\cr
\vdots & \ddots & \ddots & \ddots & \vdots \cr
0& \cdots & 0 & \lambda_{n-1} & 0 \cr
0& \cdots & \cdots & 0 & \lambda_{n}\cr
\end{pmatrix}.$$
Alors les scalaires $\lambda_1, \ldots,\lambda_n$ sont des valeurs propres de $A$, admettant respectivement comme vecteurs propres associés : 
$$X_1 = \begin{pmatrix} 1 \\ 0 \\ \vdots \\ 0 \end{pmatrix} \qquad \cdots \qquad
X_n = \begin{pmatrix} 0 \\ \vdots \\ 0 \\ 1\end{pmatrix}.$$

La preuve est immédiate. Le vecteur $X_i = \left(\begin{smallmatrix}\vdots\\0\\1\\0 \\\vdots\end{smallmatrix}\right) $ (où toutes les coordonnées sont nulles, sauf $1$ en position $i$) vérifie en effet
$$AX_i = 
\begin{pmatrix}
\lambda_{1}& 0& \cdots &  \cdots & 0\cr
0& \lambda_{2}& 0&  \cdots &0\cr
\vdots & \ddots & \ddots & \ddots & \vdots \cr
0& \cdots & 0 & \lambda_{n-1} & 0 \cr
0& \cdots & \dots & 0 & \lambda_{n}\cr
\end{pmatrix} 
\begin{pmatrix}
\vdots\\0\\1\\0 \\\vdots
\end{pmatrix}
= 
 \begin{pmatrix}
\vdots\\0\\\lambda_i\\0 \\\vdots
\end{pmatrix}
= \lambda_i X_i.$$

Conclusion : les éléments diagonaux sont des valeurs propres de $A$ ! On verra plus loin dans ce chapitre qu'il n'y a pas d'autres valeurs propres.
\end{exemple} 

%----------------------------------------------------
\subsection{Les vecteurs propres forment une famille libre}

\begin{theoreme}
\label{th:vplibre}
Soit $A \in M_n(\Kk)$.
Soient $\lambda_1,\lambda_2,\dots,\lambda_r$ des valeurs propres distinctes de 
$A$, et soit $X_i$ un vecteur propre associé à $\lambda_i$ (pour $1\leq i\leq r$).
Alors les vecteurs $X_1,X_2,\dots,X_r$ sont linéairement indépendants.
\end{theoreme}



\begin{proof}
Raisonnons par récurrence sur $r$, le nombre de valeurs propres.

\begin{itemize}
  \item \textbf{Initialisation.}
    Pour $r=1$, la famille constituée du seul vecteur $\{X_1\}$ est toujours une famille libre, car $X_1$ est non nul.
  
  
  \item \textbf{Hérédité.}
    Supposons que $X_1,X_2,\dots,X_{r-1}$ soient des vecteurs propres linéairement indépendants (où $r\ge 2$ est fixé). Soit $\lambda_r$ une autre valeur propre, et soit $X_r$ un vecteur propre associé. 
  Soient $\alpha_1,\alpha_2,\ldots,\alpha_{r-1}, \alpha_r \in \Kk$ tels que
\begin{equation}
\label{eq:vp1}
\alpha_1 X_1 + \alpha_2 X_2 +\cdots + \alpha_{r-1} X_{r-1} + \alpha_r X_r = 0.
\end{equation}
Ainsi
$$A \big(\alpha_1 X_1 + \alpha_2 X_2 +\cdots + \alpha_{r-1} X_{r-1} + \alpha_r X_r \big) = 0 $$
donc
$$\alpha_1 A X_1 + \alpha_2 A X_2 +\cdots + \alpha_{r-1} A X_{r-1} + \alpha_r A X_r  = 0.$$
Comme les $X_i$ sont des vecteurs propres, alors
\begin{equation}
\label{eq:vp2}
\alpha_1 \lambda_1 X_1 + \alpha_2 \lambda_2 X_2 +\cdots + \alpha_{r-1} \lambda_{r-1} X_{r-1} + \alpha_r \lambda_r X_r  = 0.
\end{equation}

\`A partir des équations (\ref{eq:vp1}) et (\ref{eq:vp2}), on calcule
l'expression $(\ref{eq:vp2}) - \lambda_r (\ref{eq:vp1})$ :
\begin{equation}
\label{eq:vp3}
\alpha_1 (\lambda_1-\lambda_r)  X_1 + \alpha_2 (\lambda_2-\lambda_r) X_2 +\cdots + \alpha_{r-1} (\lambda_{r-1}-\lambda_{r}) X_{r-1} = 0
\end{equation}
(le vecteur $X_r$ n'apparaît plus dans cette identité).
Comme la famille $\{ X_1,X_2,\dots,X_{r-1}\}$ est une famille libre, alors la combinaison linéaire nulle (\ref{eq:vp3}) implique que tous ses coefficients sont nuls :
$$\alpha_1 (\lambda_1-\lambda_r) = 0 \qquad \alpha_2 (\lambda_2-\lambda_r) = 0 \qquad 
\cdots \qquad \alpha_{r-1} (\lambda_{r-1}-\lambda_{r}) = 0$$
Mais comme les valeurs propres sont distinctes alors $\lambda_i - \lambda_r \neq 0$ (pour $i=1,\ldots,r-1$).
Ainsi 
$$\alpha_1  = 0 \qquad \alpha_2 = 0 \qquad 
\cdots \qquad \alpha_{r-1} =0.$$

En se souvenant qu'un vecteur propre n'est pas le vecteur nul, alors l'équation (\ref{eq:vp1}) implique en plus
$$\alpha_r = 0.$$

Bilan : on vient de prouver que la famille $\{ X_1,X_2,\dots,X_r\}$ est une famille libre.

  \item \textbf{Conclusion.} Par le principe de récurrence, des vecteurs propres associés à des valeurs propres distinctes sont linéairement indépendants.
   
\end{itemize}


\end{proof}


 
 
%----------------------------------------------------
\begin{miniexercices}
\sauteligne
\begin{enumerate}
  \item Soit $A = \left(\begin{smallmatrix}2 & 5 \\3 & 4 \end{smallmatrix}\right)$.
  Montrer que $X_1 = \left(\begin{smallmatrix}1\\1 \end{smallmatrix}\right)$ et
  $X_2 = \left(\begin{smallmatrix}5\\-3 \end{smallmatrix}\right)$ sont des vecteurs propres de $A$. Quelles sont les valeurs propres associées ?
  Même question avec 
  $A = \left(\begin{smallmatrix} 2 & -1 \\-1 & 3\end{smallmatrix}\right)$, 
  $X_1 = \left(\begin{smallmatrix}2\\ \sqrt{5}-1\end{smallmatrix}\right)$,
  $X_2 = \left(\begin{smallmatrix}2\\ -\sqrt{5}-1\end{smallmatrix}\right)$.
  
  \item Soit $A = \left(\begin{smallmatrix}
  -1 & -1 & 0 \\
0 & -2 & 0 \\
-1 & 0 & 0\end{smallmatrix}\right)$.
Montrer que $\lambda_1 = -2$, $\lambda_2 = -1$ et $\lambda_3 = 0$ sont valeurs propres de $A$.
Pour chaque valeur propre, trouver un vecteur propre associé.
    
  \item Quelles sont les valeurs propres et les vecteurs propres de la matrice identité
  $I_n$ ? Et de la matrice nulle $0_n$ ?
  
  \item Montrer qu'une matrice $A \in M_n(\Kk)$ a au plus $n$ valeurs propres distinctes (utiliser un résultat du cours).
  
  \item Soit $A = \left(\begin{smallmatrix}
  5 & -7 & 7 \\
0 & 5 & 0 \\
0 & 7 & -2\end{smallmatrix}\right)$.
  Montrer que les vecteurs 
  $X_1 = \left(\begin{smallmatrix}3\\-1\\-1 \end{smallmatrix}\right)$, $X_2 = \left(\begin{smallmatrix}0\\2\\2\end{smallmatrix}\right)$, $X_3 =  \left(\begin{smallmatrix}5\\1\\1\end{smallmatrix}\right)$ 
  sont vecteurs propres de $A$. Montrer que 
  $\{X_1 , X_2 , X_3\}$ \emph{ne forme pas} une famille libre. Est-ce que cela contredit un résultat
  du cours ? 
\end{enumerate}
\end{miniexercices}

%%%%%%%%%%%%%%%%%%%%%%%%%%%%%%%%%%%%%%%%%%%%%%%%%%%%%
\section{Polynôme caractéristique}

Comment trouver les valeurs propres d'une matrice parmi tous les éléments de $\Kk$  ?

%----------------------------------------------------
\subsection{Caractérisation des valeurs propres}

Voici le résultat fondamental pour déterminer les valeurs propres.

\begin{proposition}
\label{prop:vprac}
Soient $A \in M_n(\Kk)$ et $\lambda \in \Kk$. Alors :
\[\lambda \text{ est une valeur propre de $A$ } \iff \det (A-\lambda I_n) = 0 .\]
\end{proposition}

Rappel : $I_n$ est la matrice identité de taille $n\times n$ ; quel que soit le vecteur $X \in \Kk^n$,
$I_n \cdot X = X$.

\begin{proof}

\begin{align*}
\lambda \text{ est une valeur propre de } A
&\quad\iff\quad \exists X \in \Kk^n \setminus\{0\}, \quad AX = \lambda X \\
&\quad\iff\quad \exists X \in \Kk^n \setminus\{0\}, \quad  (A -\lambda I_n)X = 0 \\
&\quad\iff\quad \Ker(A-\lambda I_n) \neq \{0\} \\
&\quad\iff\quad A- \lambda I_n \text{ n'est pas injective} \\
&\quad\iff\quad A -\lambda I_n \text{ n'est pas inversible} \\
&\quad\iff\quad \det (A-\lambda I_n) = 0 .
\end{align*}

\end{proof}


%----------------------------------------------------
\subsection{Polynôme caractéristique}

Nous transformons la caractérisation précédente en un outil pratique :

\begin{definition}
Soit $ A \in M_n(\Kk)$.
Le \defi{polynôme caractéristique} de $A$ est
\mybox{$\chi_A(X) = \det(A - XI_n).$ }
\end{definition}



La proposition \ref{prop:vprac} devient alors :
\mybox{$\lambda \text{ valeur propre de } A \quad \iff \quad
\chi_A(\lambda) = 0$}


\begin{remarque*}
\sauteligne
\begin{itemize}
  \item La matrice $A - XI_n$ est à coefficients dans $\Kk[X]$, donc son déterminant $\chi_A(X)$ appartient à $\Kk[X]$. 
  
  \item On notera aussi souvent $\chi_A(\lambda) = \det(A - \lambda I_n)$ comme un polynôme
  en $\lambda$, plutôt que $\chi_A(X)$.
    
  \item Pour tout $\lambda \in \Kk$, $\det (\lambda I_n-A) = (-1)^n\chi_A(\lambda)$.
\end{itemize}
\end{remarque*}


\medskip


La matrice $A$ étant de taille $n \times n$, alors le polynôme 
caractéristique de $A$ est un polynôme de degré $n$. Cela conduit aux propriétés suivantes :
\begin{corollaire}
\sauteligne
\begin{itemize}
  \item Soit $A \in M_n(\Kk)$. Alors $A$ admet au plus $n$ valeurs propres.
  
  \item Si le corps $\Kk=\Cc$, alors toute matrice $A \in M_n(\Cc)$ admet 
au moins une valeur propre.
\end{itemize}
\end{corollaire}
 
En effet, nous savons qu'un polynôme de degré $n$ a au plus $n$ racines. Et sur $\Cc$ un polynôme non constant admet toujours au moins une racine.


\begin{proposition}
Deux matrices semblables ont le même polynôme caractéristique.
Autrement dit, si $B = P^{-1}AP$ alors $\chi_A(X) = \chi_B(X)$.
\end{proposition}


\begin{proof}
On écrit 
$$B-XI_n=P^{-1}AP-X(P^{-1}I_nP)=P^{-1}(A-XI_n)P.$$
Mais on sait que le déterminant vérifie $\det(MN) = \det(M)\cdot\det(N)$ pour toutes matrices $M$ et $N$
et $\det(M^{-1}) = \frac{1}{\det(M)}$ pour une matrice $M$ inversible.
Donc
$$\chi_B(X) 
= \det(B-XI_n)
= \frac{1}{\det(P)} \cdot \det(A-XI_n) \cdot \det(P)
= \det(A-XI_n) 
= \chi_A(X).$$
\end{proof}

%----------------------------------------------------
\subsection{Exemples}

Reprenons les exemples du début de ce chapitre et traitons-les en utilisant le polynôme caractéristique.

\begin{exemple}
$$A = \begin{pmatrix}
1 & 3 & 3 \\
-2 & 11 & -2 \\
8 & -7 & 6
\end{pmatrix}$$
Alors 

\begin{align*}
\chi_A(X) 
 & = \det(A-XI_3) \\ 
 & =  \det \left(
\begin{pmatrix}
1 & 3 & 3 \\
-2 & 11 & -2 \\
8 & -7 & 6
\end{pmatrix}
- X 
\begin{pmatrix}
1 & 0 & 0 \\
0 & 1 & 0 \\
0 & 0 & 1
\end{pmatrix}
\right) \\
 & = \begin{vmatrix}
1-X & 3 & 3 \\
-2 & 11-X & -2 \\
8 & -7 & 6-X
\end{vmatrix} \\
 & = \cdots \\
 & = -X^3 + 18X^2 -51X - 182 \\
 & = -(X+2)(X-7)(X-13).
 \end{align*}
Donc les valeurs propres sont :
$$-2, \ \ 7\  \text{ et }\  13.$$ 
\end{exemple}

\begin{exemple}
$$A = \begin{pmatrix}0 & 1\\1 &1\end{pmatrix}$$
Alors 
\[\chi_A(X) 
= \begin{vmatrix}0-X\  & \ 1\\1\  & \ 1-X\end{vmatrix}
= X^2 -X -1 
= \left(X-\frac{1-\sqrt{5}}{2}\right)\left(X-\frac{1+\sqrt{5}}{2}\right)\]
donc les valeurs propres sont :
$$\frac{1-\sqrt{5}}{2} \qquad \text{ et } \qquad \frac{1+\sqrt{5}}{2}.$$ 
\end{exemple}


\begin{exemple}
$$A = \begin{pmatrix}0 & 1\\-1 &-1\end{pmatrix}$$
Alors :
\[\chi_A(X) 
= \begin{vmatrix}0-X\  & \ 1\\-1\  & \ -1 -X\end{vmatrix}
= X^2 + X +1 
= (X-j)(X-\bar j)\]
où $j = -\frac{1}{2}+\ii\frac{\sqrt{3}}{2}$. 
Bilan :
\begin{itemize}
  \item sur $\Kk = \Rr$, la matrice $A$ n'a aucune valeur propre,
  \item sur $\Kk = \Cc$, la matrice $A$ a deux valeurs propres : $j$ et $\bar j$. 
\end{itemize}
\end{exemple}


\begin{exemple}
$$A = \begin{pmatrix}
\cos \theta & -\sin \theta\\
\sin \theta &\cos \theta
\end{pmatrix}$$
Alors :
\[\chi_A(X) 
= \begin{vmatrix}
\cos \theta -X \  & \ -\sin \theta\\
\sin \theta \  & \ \cos \theta-X
\end{vmatrix}
= X^2-2\cos(\theta) X +1 
= (X-e^{\ii\theta}) (X-e^{-\ii\theta}).\]
Sur $\Cc$, les valeurs propres sont $e^{-\ii\theta}$ et $e^{\ii\theta}$.
\end{exemple}


\begin{exemple}[Cas des matrices triangulaires]
Soit $A$ une matrice triangulaire supérieure
$$A = 
\begin{pmatrix}
a_{11} & a_{12} &\cdots&\cdots&\cdots & a_{1n}\\
0&a_{22}&\cdots&\cdots&\cdots&a_{2n}\\
\vdots&\ddots&\ddots&&&\vdots\\
\vdots&&\ddots&\ddots&&\vdots\\
\vdots & &&\ddots&\ddots&\vdots\\
0&\cdots&\cdots&\cdots&0&a_{nn}
\end{pmatrix}.
$$
Alors 
$$\chi_A(X)
= \begin{vmatrix}
a_{11}-X & a_{12} &\cdots&\cdots&\cdots & a_{1n}\\
0&a_{22}-X&\cdots&\cdots&\cdots&a_{2n}\\
\vdots&\ddots&\ddots&&&\vdots\\
\vdots&&\ddots&\ddots&&\vdots\\
\vdots & &&\ddots&\ddots&\vdots\\
0&\cdots&\cdots&\cdots&0&a_{nn}-X
\end{vmatrix}
= \prod_{i=1}^n(a_{ii}-X) =  (-1)^n \prod_{i=1}^n(X-a_{ii}).$$
Le calcul de ce déterminant se fait d'abord en développant la première colonne, puis par récurrence.
L'ensemble des valeurs propres est donc $\left\{a_{ii} \mid 1\le i \le n\right\}$.
On retient :
\mybox{Les valeurs propres d'une matrice triangulaire sont exactement ses éléments diagonaux.}
\end{exemple}



%----------------------------------------------------
\subsection{Coefficients du polynôme caractéristique}

La somme et le produit des valeurs propres peuvent se calculer facilement à partir de la matrice.

\begin{proposition}
\label{prop:vptrdet}
Si une matrice $A \in M_n(\Kk)$ admet $n$ valeurs propres alors :

\mybox{La somme des valeurs propres vaut $\tr A$.}

\mybox{Le produit des valeurs propres vaut $\det A$.}
\end{proposition}

Rappel : la \defi{trace} de $A$, $\tr A$, est la somme des coefficients diagonaux de $A$.


Cette proposition est évidente pour une matrice diagonale. Elle se démontre aussi facilement si la matrice est diagonalisable. Nous allons ici déduire cette proposition 
d'un résultat plus général sur les coefficients du polynôme caractéristique.


\begin{proposition}
\label{prop:polcartrdet}
Soit $A \in M_n(\Kk)$. Soit $\chi_A(X)$ son polynôme caractéristique.
C'est un polynôme de degré $n$ qui vérifie :
$$\chi_A(X)=(-1)^n X^n + (-1)^{n-1}(\tr A)X^{n-1}+\dots+\det A.$$
\end{proposition} 


\begin{exemple}
Voici le cas $n=2$. Soit $A=\left(\begin{matrix}a&b\cr c&d\end{matrix}\right)$.
On a $$\chi_A(X)=\left|\begin{matrix}a-X&b\cr c& d-X\end{matrix}\right|
=(a-X)(d-X)-bc=X^2 -\underbrace{(a+d)}_{\tr A}X+
\underbrace{ad-bc}_{\det A}.$$
\end{exemple}



\begin{exemple}
Si $\lambda_1,\lambda_2,\ldots,\lambda_n \in \Kk$ sont les valeurs propres de $A$, alors le polynôme caractéristique s'écrit aussi
$$\chi_A(X) = (-1)^n (X-\lambda_1)(X-\lambda_2)\cdots(X-\lambda_n).$$
En développant cette expression, on trouve
$$\chi_A(X) = (-1)^n \Big(X^n - (\lambda_1+\cdots+\lambda_n) X^{n-1}+\cdots + (-1)^n(\lambda_1\cdot\lambda_2\cdots\lambda_n)\Big).$$
En identifiant ce polynôme avec l'expression de la proposition \ref{prop:polcartrdet}, on obtient la proposition \ref{prop:vptrdet}.
\end{exemple}

Voici les grandes lignes de la preuve de la proposition \ref{prop:polcartrdet} (on reviendra sur cette preuve dans le chapitre suivant \og{}Diagonalisation\fg{}).
\begin{proof}
Si $A=(a_{ij})_{1\le i,j\le n}$, on a
$$\chi_A(X)
= \begin{vmatrix}
a_{11}-X&a_{12}&\cdots&a_{1n}\cr 
a_{21}&a_{22}-X& \cdots & a_{2n} \cr
\vdots& \vdots & \ddots &  \vdots \cr 
a_{n1}&a_{n2} &\cdots&a_{nn}-X\end{vmatrix}.$$
On a alors :
\begin{itemize}
  \item Le polynôme $\chi_A(X)$ est de degré $n$.
  \item Les termes de degré $n$ et $n-1$ proviennent du produit 
$$(a_{11}-X)\cdots(a_{nn}-X)=(-1)^nX^n+(-1)^{n-1}(\tr A)X^{n-1}+\cdots$$   
  \item Le terme constant, quant à lui, est donné par $\chi_A(0)=\det A$.
\end{itemize}
\end{proof}



%----------------------------------------------------
\subsection{Matrice compagnon}


Est-ce que n'importe quel polynôme peut être réalisé comme polynôme caractéristique ? La réponse est oui !

\begin{proposition}
Soit $P(X) = X^n+c_{n-1}X^{n-1}+ \cdots +c_1X+c_0 \in \Kk[X]$.
Soit $A \in M_n(\Kk)$ la \defi{matrice compagnon} du polynôme $P$ :
$$A = 
\begin{pmatrix}
0&\cdots&\cdots&0&-c_{0}\\
1&\ddots&&\vdots&\vdots\\
0&\ddots&\ddots&\vdots&\vdots\\
\vdots&\ddots&\ddots&0&\vdots\\
0&\cdots&0&1&-c_{n-1}
\end{pmatrix}$$
Alors 
\mybox{$\chi_{A}(X) = (-1) ^n P(X) .$}
\end{proposition}


\begin{proof}
La preuve est par récurrence sur $n \ge 1$. 

\textbf{Initialisation.}
Si $n=1$ alors la matrice compagnon
est la matrice $A = (-c_0)$, de taille $1 \times 1$. Ainsi $\det(A-XI_1) = \det (-c_0-X) = -X-c_0 = -P(X)$.


\textbf{Hérédité.} Fixons $n\ge2$ et supposons la formule vraie pour des matrices compagnons de taille $(n-1)\times(n-1)$.
Alors :
\begin{align*}
\chi_{A}(X)
 & \quad = \quad \det(A-XI_n) \\
 & \quad =  \quad\begin{vmatrix}
-X&0&\cdots&0&-c_{0}\\
1&\ddots&\ddots&\vdots&\vdots\\
0&\ddots&\ddots&0&\vdots\\
\vdots&\ddots&\ddots&-X&\vdots\\
0&\cdots&0&1&-X-c_{n-1}
\end{vmatrix} \qquad \text{on développe par rapport à la première ligne :} \\
& \quad = \quad (-X)  
\underbrace{\begin{vmatrix}
-X&0&\cdots&0&-c_{1}\\
1&\ddots&\ddots&\vdots&\vdots\\
0&\ddots&\ddots&0&\vdots\\
\vdots&\ddots&\ddots&-X&\vdots\\
0&\cdots&0&1&-X-c_{n-1}
\end{vmatrix}}_{
\substack{\text{ polynôme caractéristique associé à une }\\
\text{ matrice compagnon de taille  $(n-1)\times(n-1)$}}
}
+(-1)^{n} c_0
\underbrace{
\begin{vmatrix}
1&-X&0&\cdots&0\\
0&\ddots&\ddots&\ddots&\vdots\\
\vdots&\ddots&\ddots&\ddots&0\\
\vdots&&\ddots&\ddots&-X\\
0&\cdots&\cdots&0&1
\end{vmatrix}}_{ = 1 }\\
& \quad  = \quad (-X)\Big( (-1) ^{n-1}(X^{n-1}+c_{n-1}X^{n-2}+\cdots+c_1 ) \Big) + (-1)^{n} c_0  \times 1 \qquad \text{ par hypothèse de récurrence} \\
& \quad  = \quad (-1)^n(X^n + c_{n-1}X^{n-1} + \cdots +c_1 X + c_0) \\
& \quad = \quad (-1) ^n P(X)
\end{align*}

\textbf{Conclusion.}
Par le principe de récurrence, la proposition est vérifiée pour tout $n\ge1$.

\end{proof}

\begin{exemple}
Soit $A$ la matrice :
$$A = 
\begin{pmatrix}
0&\cdots&\cdots&0&1\\
1&\ddots&&\vdots&0\\
0&\ddots&\ddots&\vdots&\vdots\\
\vdots&\ddots&\ddots&0&\vdots\\
0&\cdots&0&1&0
\end{pmatrix}
\in M_n(\Cc)
$$
Alors $\chi_A(X) = (-1)^n(X^n-1)$.
Les valeurs propres de $A$ sont donc les racines $n$-ièmes de l'unité.
\end{exemple}




%----------------------------------------------------
\begin{miniexercices}
\sauteligne
\begin{enumerate}
  \item Calculer le polynôme caractéristique des matrices suivantes, et en déduire leurs valeurs propres : 
  $A = \left(\begin{smallmatrix}2 & 3 \\7 & 4 \end{smallmatrix}\right)$,
  $B = \left(\begin{smallmatrix}-6 & 3 & -7 \\5 & 0 & 5 \\5 & -1 & 6 \end{smallmatrix}\right)$,   
  $C = \left(\begin{smallmatrix}-3 & -1 & 0 & 0 \\1 & -3 & 0 & 0 \\0 & 0 & -1 & 2 \\0 & 0 & 5 & 2 \end{smallmatrix}\right)$. 
   
  \item Calculer, à la main, les coefficients du polynôme caractéristique d'une matrice $A$ de taille
  $3\times3$ :
  $A = \left(\begin{smallmatrix}a_{11}&a_{12}&a_{13}\\a_{21}&a_{22}&a_{23}\\a_{31}&a_{32}&a_{33}\\ \end{smallmatrix}\right)$. 
  
  \item Trouver plusieurs matrices de taille $2\times2$ dont la somme des valeurs propres fait $6$ et le produit des valeurs propres fait $2$.
  
  \item Trouver une matrice, ni diagonale ni triangulaire, dont le polynôme caractéristique est 
  $(X-1)^2(X^2+X+1)$. 
  Trouver une matrice, ni diagonale ni triangulaire, dont l'ensemble des valeurs propres est $\{0,1,-1,\ii,-\ii\}$.

\end{enumerate}
\end{miniexercices}



%%%%%%%%%%%%%%%%%%%%%%%%%%%%%%%%%%%%%%%%%%%%%%%%%%%%%
\section{Diagonaliser à la main}

Nous avons vu qu'une matrice $A$ et une matrice semblable $P^{-1}AP$ ont le même polynôme caractéristique, donc les mêmes valeurs propres. Nous allons voir un critère simple (mais pas le plus général) pour obtenir une matrice $P^{-1}AP$ qui est diagonale, les coefficients diagonaux étant les valeurs propres de $A$.

%----------------------------------------------------
\subsection{Matrice diagonalisable}


\begin{definition}
Soit $A \in M_n(\Kk)$. On dit que $A$ est \defi{diagonalisable} 
sur $\Kk$ s'il existe une matrice $P \in M_n(\Kk)$ inversible telle que 
$P^{-1}AP$ soit une matrice diagonale.
\end{definition} 


Il y a beaucoup d'intérêt à se ramener à une matrice diagonale. Voici juste une application.

\begin{exemple}
Soient 
$$A = \begin{pmatrix}
1 & 2 \\
3 & -4
\end{pmatrix}\qquad
P = \begin{pmatrix}
1 & 1 \\
-3 & \frac{1}{2}
\end{pmatrix}\qquad
P^{-1}=\begin{pmatrix}
\frac{1}{7} & -\frac{2}{7} \\
\frac{6}{7} & \frac{2}{7}
\end{pmatrix}.$$
Alors 
$$D = P^{-1}AP =
\begin{pmatrix}
-5 & 0 \\
0 & 2
\end{pmatrix}
\qquad \text{ donc } \qquad
D^k=\begin{pmatrix}
(-5)^k & 0 \\
0 & 2^k
\end{pmatrix}.
$$
La matrice $D$ étant diagonale, la matrice $A$ est diagonalisable.
Calculons maintenant $A^k$.

Comme $D = P^{-1}AP$ alors $A = PDP^{-1}$, donc
$$A^k = (PDP^{-1})^k = \underbrace{(PDP^{-1})(PDP^{-1}) \cdots (PDP^{-1})}_{k \text{ occurrences}} = P D^k P^{-1}.$$
D'où
$$A^k=
\begin{pmatrix}
\frac{6}{7} \times 2^{k} + \frac{1}{7} \times \left(-5\right)^{k} &
\frac{2}{7} \times 2^{k} - \frac{2}{7} \times \left(-5\right)^{k} \\
\frac{3}{7} \times 2^{k} - \frac{3}{7} \times \left(-5\right)^{k} &
\frac{1}{7} \times 2^{k} + \frac{6}{7} \times \left(-5\right)^{k}
\end{pmatrix}.
$$
\end{exemple}


%----------------------------------------------------
\subsection{Un critère pour diagonaliser}

Comment trouver cette matrice $P$ qui permet de diagonaliser ?

Nous allons donner une méthode sous l'hypothèse que la matrice admette $n$ valeurs propres distinctes.
Nous verrons des critères plus puissants dans les chapitres suivants.

\begin{proposition}
\label{prop:diagvpdis}
Soit $A \in M_n(\Kk)$ une matrice. Supposons que $A$ admette $n$ valeurs propres distinctes $\lambda_1,\ldots,\lambda_n \in \Kk$. Soient $X_1,\ldots,X_n$ des vecteurs propres associés. Alors :
\begin{itemize}
  \item La matrice $A$ est diagonalisable.
  
  \item Si on note $P$ la matrice dont les vecteurs colonnes sont les $(X_1,\ldots,X_n)$, alors $D=P^{-1}AP$ est la matrice diagonale formée des valeurs propres de $A$.
\end{itemize}
\end{proposition}

\begin{proof}~

\begin{itemize}
  \item Par le théorème \ref{th:vplibre}, la famille $(X_1,\ldots,X_n)$ est une famille libre, car les valeurs propres sont distinctes. Comme cette famille possède $n$ éléments, c'est une base de $\Kk^n$.
  
  \item La matrice $P$ est inversible car ses vecteurs colonnes forment une base de $\Kk^n$.
  On pose $D=P^{-1}AP$.
  
  \item Notons $(Y_1,\ldots,Y_n)$ la base canonique de $\Kk^n$ :  $Y_i = \left(\begin{smallmatrix}\vdots\\0\\1\\0 \\\vdots\end{smallmatrix}\right)$ (où le $1$ est en position $i$).
 On remarque que la multiplication $M Y_i$ renvoie la $i$-ème colonne de la matrice $M$. En particulier, $PY_i = X_i$ et donc $Y_i = P^{-1}X_i$. 
  
  Calculons $DY_i$ pour chaque $1\le i \le n$ :  
  \begin{align*}
  DY_i & = (P^{-1}AP) Y_i 
        = (P^{-1}A)P Y_i \\
       & = (P^{-1}A)X_i 
        = P^{-1} (AX_i) \\
       & = P^{-1} (\lambda_i X_i) 
        = \lambda_i P^{-1} X_i \\
       &= \lambda_i Y_i
  \end{align*}
  Nous avons utilisé le fait que $X_i$ est un vecteur propre associé à la valeur propre $\lambda_i$.
  Ainsi on vient de prouver que la $i$-ème colonne de $D$, qui est $DY_i$, est aussi
  $\lambda_i Y_i = \left(\begin{smallmatrix}\vdots\\0\\\lambda_i\\0 \\\vdots\end{smallmatrix}\right)$.
  
  La matrice $D$ est donc bien la matrice diagonale dont les coefficients sont les $\lambda_i$ :
  $$D = \begin{pmatrix}
\lambda_{1}&0&\cdots&0 \cr
0&\lambda_{2}&\ddots&\vdots\cr
\vdots&\ddots& \ddots &0\cr
0&\cdots&0& \lambda_{n}
\end{pmatrix}$$ 
\end{itemize}

\end{proof}

%----------------------------------------------------
\subsection{Méthode pour diagonaliser}


Soit $A \in M_n(\Kk)$ une matrice carrée $n \times n$. 
Pour essayer de la diagonaliser :
\begin{enumerate}
  \item On calcule d'abord son polynôme caractéristique $\chi_A(X)$.
  \item On cherche les racines de $\chi_A(X)$ : ce sont les valeurs propres de $A$.
\end{enumerate} 


\emph{On suppose ici que l'on a trouvé $n$ valeurs propres distinctes : $\lambda_1,\ldots,\lambda_n$.}

\begin{enumerate}  
  \setcounter{enumi}{2}
  \item Pour chaque valeur propre $\lambda_i$ de $A$, on cherche un vecteur propre $X_i$.
  
  \item Soit $P$ la matrice dont les vecteurs colonnes sont les $(X_1,\ldots,X_n)$.
  Alors $D=P^{-1}AP$ est la matrice diagonale :
  $$D = \begin{pmatrix}
\lambda_{1}&0&\cdots&0 \cr
0&\lambda_{2}&\ddots&\vdots\cr
\vdots&\ddots& \ddots &0\cr
0&\cdots&0& \lambda_{n}
\end{pmatrix}$$
 
 
 \end{enumerate} 
 

%----------------------------------------------------
\subsection{Exemples}

\begin{exemple}
Diagonalisons la matrice 
$$A=\begin{pmatrix}4&-2\cr1&1\cr\end{pmatrix}.$$
\begin{enumerate}
  \item Pour cela, on détermine ses valeurs propres :
$$\det(A-\lambda I)=\begin{vmatrix}4-\lambda &-2\cr1&1-\lambda\end{vmatrix}=(4-\lambda)(1-\lambda)+2=\lambda^2-5\lambda+6=(\lambda-2)(\lambda-3).$$
  
  \item Ainsi, la matrice $A$ admet deux valeurs propres distinctes, qui sont $\lambda_1=2$ et $\lambda_2=3$. Par la proposition \ref{prop:diagvpdis}, elle est diagonalisable.
  
  \item Déterminons une base de vecteurs propres. Tout d'abord pour $\lambda_1 = 2$ :
$$\begin{pmatrix}4&-2\cr1&1\cr\end{pmatrix}\begin{pmatrix}x\cr y\end{pmatrix}=
\begin{pmatrix}2x\cr 2y\end{pmatrix}\iff x=y,$$
d'où le vecteur propre $X_1=\left(\begin{smallmatrix}1\\1\end{smallmatrix}\right)$ associé à la valeur propre $\lambda_1=2$. 
Pour la valeur propre $\lambda_2=3$ : 
$$\begin{pmatrix}4&-2\cr1&1\cr\end{pmatrix}\begin{pmatrix}x\cr y\end{pmatrix}=
\begin{pmatrix}3x\cr 3y\end{pmatrix}\iff x=2y,$$
d'où le vecteur propre $X_2=\left(\begin{smallmatrix}2\\1\end{smallmatrix}\right)$ associé à la valeur propre $\lambda_2=3$.
  
  \item Dans la base $(X_1, X_2)$, la matrice s'écrit
$$D=\left(\begin{matrix}2&0\cr0&3\cr\end{matrix}\right).$$
Plus précisément $D = P^{-1}AP$, où
$$P=\begin{pmatrix}1&2\cr1&1\cr\end{pmatrix}
\quad \text{ et } \quad 
P^{-1}=\begin{pmatrix}-1&2\cr1&-1\cr\end{pmatrix}.$$
  
\end{enumerate}
\end{exemple}


\begin{exemple}
Soit 
$$A=
\begin{pmatrix}
1 & 0 & 0 \\
5 & -4 & 0 \\
6 & -6 & 2
\end{pmatrix}.$$
Démontrons que $A$ est diagonalisable et trouvons une matrice $P$ telle que $P^{-1}AP$ soit diagonale.


\begin{enumerate}
  \item Commençons par calculer le polynôme caractéristique de $A$ :
$$\chi_A(X)=\begin{vmatrix}
1-X & 0 & 0 \\
5 & -4-X & 0 \\
6 & -6 & 2-X
\end{vmatrix}
=-(X-1)(X+4)(X-2)$$


  \item Les racines du polynôme caractéristique, et donc les valeurs propres de $A$, sont les réels 
  $\lambda_1 = 1$, $\lambda_2 = -4$ et $\lambda_3 = 2$. Il y a trois valeurs propres distinctes, et donc par la proposition \ref{prop:diagvpdis}, la matrice est diagonalisable.
  
  \item  Déterminons des vecteurs propres associés.
Par exemple, pour $\lambda_2 = -4$ :
$$\begin{pmatrix}
1 & 0 & 0 \\
5 & -4 & 0 \\
6 & -6 & 2
\cr\end{pmatrix}\begin{pmatrix}x\cr y \cr z\end{pmatrix}=
\begin{pmatrix}-4x\cr -4y \cr -4z\end{pmatrix}
\iff 
\left\{ 
\begin{array}{rcl}
5x &=& 0 \\
5x &=& 0 \\
6x-6y+6z &=& 0 \\ 
\end{array}
\right.
\iff 
\left\{ 
\begin{array}{rcl}
x &=& 0 \\
y &=& z \\ 
\end{array}
\right.$$
L'ensemble des solutions est donc 
$\left\{ \left(\begin{smallmatrix}0\\t\\t\end{smallmatrix}\right) \mid t \in \Rr \right\}$.
D'où le vecteur propre $X_2=\left(\begin{smallmatrix}0\\1\\1\end{smallmatrix}\right)$ associé à la valeur propre $\lambda_2=-4$.   
  
  
De même, on trouve $X_1$ (associé à $\lambda_1 = 1$) et $X_3$ (associé à $\lambda_3 = 2$) :
 $$
 X_1 = \begin{pmatrix}1\\1\\0\end{pmatrix}\qquad
 X_2 = \begin{pmatrix}0\\1\\1\end{pmatrix}\qquad 
 X_3 = \begin{pmatrix}0\\0\\1\end{pmatrix}$$
 
  \item  
 La base $(X_1, X_2,X_3)$ diagonalise la matrice $A$, c'est-à-dire
 $D = P^{-1}AP$ avec  
$$D=\begin{pmatrix}1&0&0\cr0&-4&0\cr0&0&2\end{pmatrix} \quad \text{ et  } \quad 
P=\begin{pmatrix}
1 & 0 & 0 \\
1 & 1 & 0 \\
0 & 1 & 1
\end{pmatrix}
\quad
P^{-1}=\begin{pmatrix}
1 & 0 & 0 \\
-1 & 1 & 0 \\
1 & -1 & 1
\end{pmatrix}
.$$    
\end{enumerate}  
\end{exemple}


%----------------------------------------------------
\begin{miniexercices}
\sauteligne
\begin{enumerate}
  \item 
  Soit $A=\left(\begin{smallmatrix}
-4 & 5 & 0 \\
0 & 1 & 0 \\
-7 & 7 & 3  
\end{smallmatrix}\right)$. Calculer les valeurs propres de $A$. Trouver un vecteur propre pour chaque valeur propre. Trouver une matrice $P$ telle que $D = P^{-1} A P$ soit une matrice diagonale. Calculer $P^{-1}$. Calculer $D^k$ ($k\in\Nn$). En déduire $A^k$.
  
  \item Même exercice avec $A = \left(\begin{smallmatrix}
  1 & -2 & 0 & 0 \\
-1 & 1 & 0 & 0 \\
0 & 0 & 1 & -3 \\
0 & 0 & -3 & 1
 \end{smallmatrix}\right)$.
 
  \item Dans le cas d'une matrice $A \in M_n(\Kk)$ ayant $n$ valeurs propres distinctes, donner une nouvelle preuve des assertions : \og{}La somme des valeurs propres vaut $\tr A$.\fg{}
  et \og{}Le produit des valeurs propres vaut $\det A$.\fg{}
  
  \item Soit $T$ une matrice triangulaire, non diagonale, dont les coefficients sur la diagonale sont tous $1$. Montrer qu'elle n'est pas diagonalisable. Indications : (a) Quelles sont les valeurs propres de $T$ ? (b) Si $T$ était diagonalisable en une matrice $D$, que vaudrait $D$ ? (c) \`A quoi pourrait être semblable cette matrice $D$ ? 
\end{enumerate}
\end{miniexercices}



\auteurs{
\\
D'après un cours de Sandra Delaunay et un cours d'Alexis Tchoudjem.

Revu et augmenté par Arnaud Bodin.

Relu par Stéphanie Bodin et Vianney Combet.
}


\finchapitre 
\end{document}


