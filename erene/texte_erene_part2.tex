
%%%%%%%%%%%%%%%%%% PREAMBULE %%%%%%%%%%%%%%%%%%


\documentclass[12pt]{article}

\usepackage{amsfonts,amsmath,amssymb,amsthm}
\usepackage[utf8]{inputenc}
\usepackage[T1]{fontenc}
\usepackage[francais]{babel}


% packages
\usepackage{amsfonts,amsmath,amssymb,amsthm}
\usepackage[utf8]{inputenc}
\usepackage[T1]{fontenc}
%\usepackage{lmodern}

\usepackage[francais]{babel}
\usepackage{fancybox}
\usepackage{graphicx}

\usepackage{float}

%\usepackage[usenames, x11names]{xcolor}
\usepackage{tikz}
\usepackage{datetime}

\usepackage{mathptmx}
%\usepackage{fouriernc}
%\usepackage{newcent}
\usepackage[mathcal,mathbf]{euler}

%\usepackage{palatino}
%\usepackage{newcent}


% Commande spéciale prompteur

%\usepackage{mathptmx}
%\usepackage[mathcal,mathbf]{euler}
%\usepackage{mathpple,multido}

\usepackage[a4paper]{geometry}
\geometry{top=2cm, bottom=2cm, left=1cm, right=1cm, marginparsep=1cm}

\newcommand{\change}{{\color{red}\rule{\textwidth}{1mm}\\}}

\newcounter{mydiapo}

\newcommand{\diapo}{\newpage
\hfill {\normalsize  Diapo \themydiapo \quad \texttt{[\jobname]}} \\
\stepcounter{mydiapo}}


%%%%%%% COULEURS %%%%%%%%%%

% Pour blanc sur noir :
%\pagecolor[rgb]{0.5,0.5,0.5}
% \pagecolor[rgb]{0,0,0}
% \color[rgb]{1,1,1}



%\DeclareFixedFont{\myfont}{U}{cmss}{bx}{n}{18pt}
\newcommand{\debuttexte}{
%%%%%%%%%%%%% FONTES %%%%%%%%%%%%%
\renewcommand{\baselinestretch}{1.5}
\usefont{U}{cmss}{bx}{n}
\bfseries

% Taille normale : commenter le reste !
%Taille Arnaud
%\fontsize{19}{19}\selectfont

% Taille Barbara
%\fontsize{21}{22}\selectfont

%Taille François
%\fontsize{25}{30}\selectfont

%Taille Pascal
%\fontsize{25}{30}\selectfont

%Taille Laura
%\fontsize{30}{35}\selectfont


%\myfont
%\usefont{U}{cmss}{bx}{n}

%\Huge
%\addtolength{\parskip}{\baselineskip}
}


% \usepackage{hyperref}
% \hypersetup{colorlinks=true, linkcolor=blue, urlcolor=blue,
% pdftitle={Exo7 - Exercices de mathématiques}, pdfauthor={Exo7}}


%section
% \usepackage{sectsty}
% \allsectionsfont{\bf}
%\sectionfont{\color{Tomato3}\upshape\selectfont}
%\subsectionfont{\color{Tomato4}\upshape\selectfont}

%----- Ensembles : entiers, reels, complexes -----
\newcommand{\Nn}{\mathbb{N}} \newcommand{\N}{\mathbb{N}}
\newcommand{\Zz}{\mathbb{Z}} \newcommand{\Z}{\mathbb{Z}}
\newcommand{\Qq}{\mathbb{Q}} \newcommand{\Q}{\mathbb{Q}}
\newcommand{\Rr}{\mathbb{R}} \newcommand{\R}{\mathbb{R}}
\newcommand{\Cc}{\mathbb{C}} 
\newcommand{\Kk}{\mathbb{K}} \newcommand{\K}{\mathbb{K}}

%----- Modifications de symboles -----
\renewcommand{\epsilon}{\varepsilon}
\renewcommand{\Re}{\mathop{\text{Re}}\nolimits}
\renewcommand{\Im}{\mathop{\text{Im}}\nolimits}
%\newcommand{\llbracket}{\left[\kern-0.15em\left[}
%\newcommand{\rrbracket}{\right]\kern-0.15em\right]}

\renewcommand{\ge}{\geqslant}
\renewcommand{\geq}{\geqslant}
\renewcommand{\le}{\leqslant}
\renewcommand{\leq}{\leqslant}

%----- Fonctions usuelles -----
\newcommand{\ch}{\mathop{\mathrm{ch}}\nolimits}
\newcommand{\sh}{\mathop{\mathrm{sh}}\nolimits}
\renewcommand{\tanh}{\mathop{\mathrm{th}}\nolimits}
\newcommand{\cotan}{\mathop{\mathrm{cotan}}\nolimits}
\newcommand{\Arcsin}{\mathop{\mathrm{Arcsin}}\nolimits}
\newcommand{\Arccos}{\mathop{\mathrm{Arccos}}\nolimits}
\newcommand{\Arctan}{\mathop{\mathrm{Arctan}}\nolimits}
\newcommand{\Argsh}{\mathop{\mathrm{Argsh}}\nolimits}
\newcommand{\Argch}{\mathop{\mathrm{Argch}}\nolimits}
\newcommand{\Argth}{\mathop{\mathrm{Argth}}\nolimits}
\newcommand{\pgcd}{\mathop{\mathrm{pgcd}}\nolimits} 

\newcommand{\Card}{\mathop{\text{Card}}\nolimits}
\newcommand{\Ker}{\mathop{\text{Ker}}\nolimits}
\newcommand{\id}{\mathop{\text{id}}\nolimits}
\newcommand{\ii}{\mathrm{i}}
\newcommand{\dd}{\mathrm{d}}
\newcommand{\Vect}{\mathop{\text{Vect}}\nolimits}
\newcommand{\Mat}{\mathop{\mathrm{Mat}}\nolimits}
\newcommand{\rg}{\mathop{\text{rg}}\nolimits}
\newcommand{\tr}{\mathop{\text{tr}}\nolimits}
\newcommand{\ppcm}{\mathop{\text{ppcm}}\nolimits}

%----- Structure des exercices ------

\newtheoremstyle{styleexo}% name
{2ex}% Space above
{3ex}% Space below
{}% Body font
{}% Indent amount 1
{\bfseries} % Theorem head font
{}% Punctuation after theorem head
{\newline}% Space after theorem head 2
{}% Theorem head spec (can be left empty, meaning ‘normal’)

%\theoremstyle{styleexo}
\newtheorem{exo}{Exercice}
\newtheorem{ind}{Indications}
\newtheorem{cor}{Correction}


\newcommand{\exercice}[1]{} \newcommand{\finexercice}{}
%\newcommand{\exercice}[1]{{\tiny\texttt{#1}}\vspace{-2ex}} % pour afficher le numero absolu, l'auteur...
\newcommand{\enonce}{\begin{exo}} \newcommand{\finenonce}{\end{exo}}
\newcommand{\indication}{\begin{ind}} \newcommand{\finindication}{\end{ind}}
\newcommand{\correction}{\begin{cor}} \newcommand{\fincorrection}{\end{cor}}

\newcommand{\noindication}{\stepcounter{ind}}
\newcommand{\nocorrection}{\stepcounter{cor}}

\newcommand{\fiche}[1]{} \newcommand{\finfiche}{}
\newcommand{\titre}[1]{\centerline{\large \bf #1}}
\newcommand{\addcommand}[1]{}
\newcommand{\video}[1]{}

% Marge
\newcommand{\mymargin}[1]{\marginpar{{\small #1}}}



%----- Presentation ------
\setlength{\parindent}{0cm}

%\newcommand{\ExoSept}{\href{http://exo7.emath.fr}{\textbf{\textsf{Exo7}}}}

\definecolor{myred}{rgb}{0.93,0.26,0}
\definecolor{myorange}{rgb}{0.97,0.58,0}
\definecolor{myyellow}{rgb}{1,0.86,0}

\newcommand{\LogoExoSept}[1]{  % input : echelle
{\usefont{U}{cmss}{bx}{n}
\begin{tikzpicture}[scale=0.1*#1,transform shape]
  \fill[color=myorange] (0,0)--(4,0)--(4,-4)--(0,-4)--cycle;
  \fill[color=myred] (0,0)--(0,3)--(-3,3)--(-3,0)--cycle;
  \fill[color=myyellow] (4,0)--(7,4)--(3,7)--(0,3)--cycle;
  \node[scale=5] at (3.5,3.5) {Exo7};
\end{tikzpicture}}
}



\theoremstyle{definition}
%\newtheorem{proposition}{Proposition}
%\newtheorem{exemple}{Exemple}
%\newtheorem{theoreme}{Théorème}
\newtheorem{lemme}{Lemme}
\newtheorem{corollaire}{Corollaire}
%\newtheorem*{remarque*}{Remarque}
%\newtheorem*{miniexercice}{Mini-exercices}
%\newtheorem{definition}{Définition}




%definition d'un terme
\newcommand{\defi}[1]{{\color{myorange}\textbf{\emph{#1}}}}
\newcommand{\evidence}[1]{{\color{blue}\textbf{\emph{#1}}}}



 %----- Commandes divers ------

\newcommand{\codeinline}[1]{\texttt{#1}}

%%%%%%%%%%%%%%%%%%%%%%%%%%%%%%%%%%%%%%%%%%%%%%%%%%%%%%%%%%%%%
%%%%%%%%%%%%%%%%%%%%%%%%%%%%%%%%%%%%%%%%%%%%%%%%%%%%%%%%%%%%%



\begin{document}

\debuttexte


%%%%%%%%%%%%%%%%%%%%%%%%%%%%%%%%%%%%%%%%%%%%%%%%%%%%%%%%%%%
\diapo

\change

Après avoir vu la structure de l'espace vectoriel $\Rr^n$
nous allons voir quelles fonctions relie deux tels espaces.

\change
Plus précisément nous allons définir ce qu'est une application linéaire

\change
et nous verrons de nombreux exemples géométriques.


%%%%%%%%%%%%%%%%%%%%%%%%%%%%%%%%%%%%%%%%%%%%%%%%%%%%%%%%%%%
\diapo

Commençons par rappeler ce qu'est une application de $\Rr^p$ dans $\Rr^n$.

\change

Considérons $n$ fonctions qui ont chacune $p$ variables 
réelles et sont à valeurs réelles.

Notons les 

$f_1 \qquad f_2r \qquad \cdots \qquad f_n $

\change

Répétons que chaque $f_i$ est une fonction qui va de $\Rr^p$ dans $\Rr$.

Aux $p$-uplet $(x_1,x_2,\ldots,x_p)$ on associe un réel $f_i(x_1,\ldots,x_p)$.

\change
On construit une application
$$f : \Rr^p \longrightarrow \Rr^n$$

\change
définie par
$$f(x_1,\dots , x_p)  = \left(f_1 (x_1, \dots , x_p), \dots , f_n (x_1, \dots , x_p)\right).$$


%%%%%%%%%%%%%%%%%%%%%%%%%%%%%%%%%%%%%%%%%%%%%%%%%%%%%%%%%%%
\diapo

Repartons d'une application $f: \Rr^p  \longrightarrow \Rr^n$.

On note ici $y_i$ chacune des composantes de la fonction $f$.


Une telle application $f$
est une \defi{application linéaire} si 
 $$ 
 \left\{
\begin{array}{ccccccccc}
y_1 & = &a_{11}x_1 &+ &a_{12} x_2 & + & \cdots & + & a_{1p}x_p\\
y_2 & = & a_{21} x_1 & + & a_{22} x_2 & + & \cdots & + & a_{2p} x_p\\
\vdots &&\vdots &&\vdots & & & &\vdots\\
y_n & = & a_{n1}x_1 & + & a_{n2}x_2 &+&\cdots & +& a_{np} x_p\, .
\end{array}\right.
$$

C'est-à-dire chacune des équations définissant les composantes
est une équation linéaire.

Notez que l'on a toujours $f(0,\ldots,0) = (0,\ldots,0)$.

\change

Le point crucial de cette partie, est de comprendre qu'une
application linéaire correspond à une matrice.

En effet en notation matricielle, l'application linéaire $f$
s'écrit ainsi
$f\begin{pmatrix} x_1\\x_2 \\ \vdots \\x_p \end{pmatrix}$

dont nous avons noter les composantes 
$\begin{pmatrix} y_1\\ y_2\\ \vdots\\ y_n  \end{pmatrix}$.

C'est en fait le produit de la matrice  
$(a_{ij}$
par le vecteur colonne.
$
\begin{pmatrix} x_1\\x_2 \\ \vdots \\x_p \end{pmatrix},
$.

Par exemple en calculant la première composante 
on trouve $a_{11}x_1 + a_{12} x_2  +  \cdots  +  a_{1p}x_p$
qui est notre composante $y_1$.

\change
Si on note $X$ le vecteur colonne $\begin{pmatrix} x_1\\ \vdots \\x_p \end{pmatrix}$ 

et 
$A$ la matrice des coefficients $(a_{ij})$ définissant l'application linéaire $f$,

$A$ est donc une matrice à $n$ ligne et $p$ colonnes.

Alors l'application linéaire s'écrit simplement
$f(X)=AX.$

\change
Ce qu'il faut retenir,c'est qu'une application linéaire  de $\Rr^p \to \Rr^n$ 
peut s'écrire $X \mapsto A X$. La matrice $A$ est 
appelée la \defi{matrice de l'application linéaire} $f$. 

Plus précisément la matrice $A$ est la matrice de l'application linéaire $f$
  de la base canonique de $\Rr^p$ vers la base canonique de $\Rr^n$.


%%%%%%%%%%%%%%%%%%%%%%%%%%%%%%%%%%%%%%%%%%%%%%%%%%%%%%%%%%%
\diapo

Voici une exemple.
$f$ est la fonction $\Rr^4 \longrightarrow \Rr^3$ définie par
$$\left\{\begin{array}{rcl}
y_1 & = & -2x_1 + 5x_2 + 2x_3 - 7x_4\\
y_2 & = & 4x_1 + 2x_2 - 3x_3 + 3x_4\\
y_3 & = & 7x_1 - 3x_2 + 9x_3  
\end{array}\right.$$

C'est bien une application linéaire, vous voyez que chacune des composantes 
qui définie $f$ est bien une équation linéaire.

L'écriture matricielle est :
$$\begin{pmatrix} y_1\\ y_2\\ y_3 \end{pmatrix} =
\begin{pmatrix}
-2 & 5 & 2 & -7\\
4 & 2 & -3 & 3\\
7 & -3 & 9 & 0  
\end{pmatrix}\quad
\begin{pmatrix}x_1\\ x_2\\ x_3\\ x_4 \end{pmatrix}.$$


Donc cette matrice est la matrice de l'application linéaire.


%%%%%%%%%%%%%%%%%%%%%%%%%%%%%%%%%%%%%%%%%%%%%%%%%%%%%%%%%%%
\diapo

Continuons avec deux exemple particulier.


L'application linéaire identité $\Rr^n \to \Rr^n$, 
qui est définie par  $(x_1,\ldots,x_n) \mapsto (x_1,\ldots,x_n)$

Alors sa matrice associée est la matrice identité $I_n$ 

En effet on a bien $I_n$ fois le vecteur $X$ est égal à $X$).
  
\change

Pour l'application linéaire nulle, cette fois de $\Rr^p \to \Rr^n$, 
qui à n'importe quel $(x_1,\ldots,x_p)$ associe toujours $(0,\ldots,0)$.

Vous l'aurez deviner la matrice associée à l'application linéaire nulle
est la matrice nulle.

%%%%%%%%%%%%%%%%%%%%%%%%%%%%%%%%%%%%%%%%%%%%%%%%%%%%%%%%%%%
\diapo

Nous passons maintenant à une liste d'exemples géométriques, que vous connaissez bien,
et qui sont en fait des applications linéaires dont on va calculer les matrices.

Commençons par la fonction
$$f : \Rr^2 \longrightarrow \Rr^2 \qquad \begin{pmatrix} x \\ y \end{pmatrix} \mapsto \begin{pmatrix} -x \\ y \end{pmatrix}$$

C'est la réflexion par rapport à l'axe des ordonnées $(Oy)$, 

[sur le dessin] le vecteur $\begin{pmatrix} x \\ y \end{pmatrix}$ a pour image le vecteur 
$\begin{pmatrix} -x \\ y \end{pmatrix}$.

Cette application est une application linéaire et sa matrice est 
$$\begin{pmatrix}-1 & 0 \\ 0 & 1 \end{pmatrix}$$

En effet si on calcule 
$$
\begin{pmatrix} -1 & 0 \\ 0 & 1 \end{pmatrix}
\begin{pmatrix} x \\ y \end{pmatrix}$$
alors on retrouve bien le vecteur $\begin{pmatrix} -x \\ y \end{pmatrix}$.


%%%%%%%%%%%%%%%%%%%%%%%%%%%%%%%%%%%%%%%%%%%%%%%%%%%%%%%%%%%
\diapo

On calcule de même que pour la réflexion par rapport à l'axe 
des abscisses la matrice est :
\[\begin{pmatrix} 1 & 0\\ 0 & -1 \end{pmatrix}.\]


%%%%%%%%%%%%%%%%%%%%%%%%%%%%%%%%%%%%%%%%%%%%%%%%%%%%%%%%%%%
\diapo

[Sur le dessin] La réflexion par rapport à la droite $(y=x)$ est donnée par
l'application $f$ qui échange $x$ et $y$.
 $(x,y)$ est envoyé sur $(y,x)$.

Vérifiez que sa matrice est 
\[\begin{pmatrix}0 & 1\\1 & 0\end{pmatrix}.\]

%%%%%%%%%%%%%%%%%%%%%%%%%%%%%%%%%%%%%%%%%%%%%%%%%%%%%%%%%%%
\diapo

On continue avec les homothéties.

L'homothétie de rapport $\lambda$, centrée à l'origine est définie par  
$$f: \Rr^2 \longrightarrow \Rr^2, \qquad 
\begin{pmatrix}x\\y\end{pmatrix} \mapsto \begin{pmatrix} \lambda x \\ \lambda y \end{pmatrix}.$$

On peut donc ré-écrire cette application en terme de matrice :
$f\left(\begin{smallmatrix} x \\ y \end{smallmatrix}\right) = 
\left(\begin{smallmatrix} \lambda & 0 \\ 0 & \lambda \end{smallmatrix}\right)
\left(\begin{smallmatrix} x \\ y \end{smallmatrix}\right)$.
Ainsi la matrice de l'homothétie est :
\[\begin{pmatrix} \lambda & 0 \\ 0 & \lambda \end{pmatrix}.\]
C'est juste $\lambda$ fois la matrice identité.

L'homothétie est une application linéaire.

Mais attention : une translation n'est pas une une application linéaire.
En effet si on considère une translation de vecteur non nul
alors l'origine n'est pas envoyée sur l'origine,
donc une translation ne peut pas être une application linéaire.

%%%%%%%%%%%%%%%%%%%%%%%%%%%%%%%%%%%%%%%%%%%%%%%%%%%%%%%%%%%
\diapo

Soit $f: \Rr^2 \longrightarrow \Rr^2$ la rotation d'angle $\theta$, centrée à l'origine,
c'est une application linéaire.



Si on fixe un vecteur $\left(\begin{smallmatrix}x \\ y \end{smallmatrix}\right)$ 
et que l'on note $\left(\begin{smallmatrix}x' \\ y' \end{smallmatrix}\right)$ 
son image par la rotation d'angle $\theta$

alors on obtient :
$$\left\{\begin{array}{rcl}
x' & = & x \cos  \theta - y \sin\theta\\
y' & = & x\sin \, \theta + y \cos\theta
\end{array}\right.$$

et donc en terme de matrices :
$$
\begin{pmatrix}x' \\ y' \end{pmatrix}
= \begin{pmatrix}
\cos\theta & -\sin\theta\\
\sin\theta & \cos\theta
\end{pmatrix}
\begin{pmatrix}x \\ y \end{pmatrix}.
$$ 

Autrement dit, la rotation d'angle $\theta$ est donnée par la matrice 
$$\begin{pmatrix}
\cos\theta & -\sin\theta\\
\sin\theta & \cos\theta
\end{pmatrix}.$$

C'est la seule formule de cette leçon qui ne se retrouve pas 
pas un petit dessin. Il faut donc l'apprendre par coeur en prenant soin de ne pas intervertir
sinus et cosinus, ni les signes.

%%%%%%%%%%%%%%%%%%%%%%%%%%%%%%%%%%%%%%%%%%%%%%%%%%%%%%%%%%%
\diapo

Définissons l'application
$f: \Rr^2  \longrightarrow \Rr^2$
par $\begin{pmatrix}x\\y\end{pmatrix} \mapsto 
\begin{pmatrix}x\\0\end{pmatrix}$.

Cette application est la projection orthogonale sur l'axe $(Ox)$,

[Sur le dessin] 
Ici le vecteur $(x,y)$ est projeté orthogonalement sur l'axe des abscisses
en $(x,0)$.

C'est une application linéaire donnée par la matrice 
$$\begin{pmatrix} 1 & 0 \\ 0 & 0 \end{pmatrix}.$$

\change

On fait la même chose dans l'espace : l'application linéaire 
$$f: \Rr^3  \longrightarrow \Rr^3, \qquad
\begin{pmatrix}x\\y\\z\end{pmatrix}  \mapsto \begin{pmatrix}x\\y\\0\end{pmatrix}$$
est la projection orthogonale sur le plan $(Oxy)$ 

ici le vecteur de l’espace $(x,y,z)$ est projeté sur le plan horizontal en $(x,y,0)$ 

La matrice de cette projection est 
$$\begin{pmatrix}
1 & 0 & 0\\
0 & 1 & 0\\
0 & 0 & 0
\end{pmatrix}.$$

%%%%%%%%%%%%%%%%%%%%%%%%%%%%%%%%%%%%%%%%%%%%%%%%%%%%%%%%%%%
\diapo

On termine par l'application 
$$
f: \Rr^3  \longrightarrow \Rr^3, \qquad \begin{pmatrix}x\\y\\z\end{pmatrix} \mapsto \begin{pmatrix}x\\y\\-z\end{pmatrix}$$

C'est la réflexion de l'espace par rapport au plan $(Oxy)$. 
C'est une application linéaire et sa matrice est
\[ \begin{pmatrix}
1 & 0 & 0\\
0 & 1 & 0\\
0 & 0 & -1 
\end{pmatrix}
. \]

%%%%%%%%%%%%%%%%%%%%%%%%%%%%%%%%%%%%%%%%%%%%%%%%%%%%%%%%%%%
\diapo

Voici quelques exercices pour conclure...

\end{document}
