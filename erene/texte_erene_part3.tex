
%%%%%%%%%%%%%%%%%% PREAMBULE %%%%%%%%%%%%%%%%%%


\documentclass[12pt]{article}

\usepackage{amsfonts,amsmath,amssymb,amsthm}
\usepackage[utf8]{inputenc}
\usepackage[T1]{fontenc}
\usepackage[francais]{babel}


% packages
\usepackage{amsfonts,amsmath,amssymb,amsthm}
\usepackage[utf8]{inputenc}
\usepackage[T1]{fontenc}
%\usepackage{lmodern}

\usepackage[francais]{babel}
\usepackage{fancybox}
\usepackage{graphicx}

\usepackage{float}

%\usepackage[usenames, x11names]{xcolor}
\usepackage{tikz}
\usepackage{datetime}

\usepackage{mathptmx}
%\usepackage{fouriernc}
%\usepackage{newcent}
\usepackage[mathcal,mathbf]{euler}

%\usepackage{palatino}
%\usepackage{newcent}


% Commande spéciale prompteur

%\usepackage{mathptmx}
%\usepackage[mathcal,mathbf]{euler}
%\usepackage{mathpple,multido}

\usepackage[a4paper]{geometry}
\geometry{top=2cm, bottom=2cm, left=1cm, right=1cm, marginparsep=1cm}

\newcommand{\change}{{\color{red}\rule{\textwidth}{1mm}\\}}

\newcounter{mydiapo}

\newcommand{\diapo}{\newpage
\hfill {\normalsize  Diapo \themydiapo \quad \texttt{[\jobname]}} \\
\stepcounter{mydiapo}}


%%%%%%% COULEURS %%%%%%%%%%

% Pour blanc sur noir :
%\pagecolor[rgb]{0.5,0.5,0.5}
% \pagecolor[rgb]{0,0,0}
% \color[rgb]{1,1,1}



%\DeclareFixedFont{\myfont}{U}{cmss}{bx}{n}{18pt}
\newcommand{\debuttexte}{
%%%%%%%%%%%%% FONTES %%%%%%%%%%%%%
\renewcommand{\baselinestretch}{1.5}
\usefont{U}{cmss}{bx}{n}
\bfseries

% Taille normale : commenter le reste !
%Taille Arnaud
%\fontsize{19}{19}\selectfont

% Taille Barbara
%\fontsize{21}{22}\selectfont

%Taille François
%\fontsize{25}{30}\selectfont

%Taille Pascal
%\fontsize{25}{30}\selectfont

%Taille Laura
%\fontsize{30}{35}\selectfont


%\myfont
%\usefont{U}{cmss}{bx}{n}

%\Huge
%\addtolength{\parskip}{\baselineskip}
}


% \usepackage{hyperref}
% \hypersetup{colorlinks=true, linkcolor=blue, urlcolor=blue,
% pdftitle={Exo7 - Exercices de mathématiques}, pdfauthor={Exo7}}


%section
% \usepackage{sectsty}
% \allsectionsfont{\bf}
%\sectionfont{\color{Tomato3}\upshape\selectfont}
%\subsectionfont{\color{Tomato4}\upshape\selectfont}

%----- Ensembles : entiers, reels, complexes -----
\newcommand{\Nn}{\mathbb{N}} \newcommand{\N}{\mathbb{N}}
\newcommand{\Zz}{\mathbb{Z}} \newcommand{\Z}{\mathbb{Z}}
\newcommand{\Qq}{\mathbb{Q}} \newcommand{\Q}{\mathbb{Q}}
\newcommand{\Rr}{\mathbb{R}} \newcommand{\R}{\mathbb{R}}
\newcommand{\Cc}{\mathbb{C}} 
\newcommand{\Kk}{\mathbb{K}} \newcommand{\K}{\mathbb{K}}

%----- Modifications de symboles -----
\renewcommand{\epsilon}{\varepsilon}
\renewcommand{\Re}{\mathop{\text{Re}}\nolimits}
\renewcommand{\Im}{\mathop{\text{Im}}\nolimits}
%\newcommand{\llbracket}{\left[\kern-0.15em\left[}
%\newcommand{\rrbracket}{\right]\kern-0.15em\right]}

\renewcommand{\ge}{\geqslant}
\renewcommand{\geq}{\geqslant}
\renewcommand{\le}{\leqslant}
\renewcommand{\leq}{\leqslant}

%----- Fonctions usuelles -----
\newcommand{\ch}{\mathop{\mathrm{ch}}\nolimits}
\newcommand{\sh}{\mathop{\mathrm{sh}}\nolimits}
\renewcommand{\tanh}{\mathop{\mathrm{th}}\nolimits}
\newcommand{\cotan}{\mathop{\mathrm{cotan}}\nolimits}
\newcommand{\Arcsin}{\mathop{\mathrm{Arcsin}}\nolimits}
\newcommand{\Arccos}{\mathop{\mathrm{Arccos}}\nolimits}
\newcommand{\Arctan}{\mathop{\mathrm{Arctan}}\nolimits}
\newcommand{\Argsh}{\mathop{\mathrm{Argsh}}\nolimits}
\newcommand{\Argch}{\mathop{\mathrm{Argch}}\nolimits}
\newcommand{\Argth}{\mathop{\mathrm{Argth}}\nolimits}
\newcommand{\pgcd}{\mathop{\mathrm{pgcd}}\nolimits} 

\newcommand{\Card}{\mathop{\text{Card}}\nolimits}
\newcommand{\Ker}{\mathop{\text{Ker}}\nolimits}
\newcommand{\id}{\mathop{\text{id}}\nolimits}
\newcommand{\ii}{\mathrm{i}}
\newcommand{\dd}{\mathrm{d}}
\newcommand{\Vect}{\mathop{\text{Vect}}\nolimits}
\newcommand{\Mat}{\mathop{\mathrm{Mat}}\nolimits}
\newcommand{\rg}{\mathop{\text{rg}}\nolimits}
\newcommand{\tr}{\mathop{\text{tr}}\nolimits}
\newcommand{\ppcm}{\mathop{\text{ppcm}}\nolimits}

%----- Structure des exercices ------

\newtheoremstyle{styleexo}% name
{2ex}% Space above
{3ex}% Space below
{}% Body font
{}% Indent amount 1
{\bfseries} % Theorem head font
{}% Punctuation after theorem head
{\newline}% Space after theorem head 2
{}% Theorem head spec (can be left empty, meaning ‘normal’)

%\theoremstyle{styleexo}
\newtheorem{exo}{Exercice}
\newtheorem{ind}{Indications}
\newtheorem{cor}{Correction}


\newcommand{\exercice}[1]{} \newcommand{\finexercice}{}
%\newcommand{\exercice}[1]{{\tiny\texttt{#1}}\vspace{-2ex}} % pour afficher le numero absolu, l'auteur...
\newcommand{\enonce}{\begin{exo}} \newcommand{\finenonce}{\end{exo}}
\newcommand{\indication}{\begin{ind}} \newcommand{\finindication}{\end{ind}}
\newcommand{\correction}{\begin{cor}} \newcommand{\fincorrection}{\end{cor}}

\newcommand{\noindication}{\stepcounter{ind}}
\newcommand{\nocorrection}{\stepcounter{cor}}

\newcommand{\fiche}[1]{} \newcommand{\finfiche}{}
\newcommand{\titre}[1]{\centerline{\large \bf #1}}
\newcommand{\addcommand}[1]{}
\newcommand{\video}[1]{}

% Marge
\newcommand{\mymargin}[1]{\marginpar{{\small #1}}}



%----- Presentation ------
\setlength{\parindent}{0cm}

%\newcommand{\ExoSept}{\href{http://exo7.emath.fr}{\textbf{\textsf{Exo7}}}}

\definecolor{myred}{rgb}{0.93,0.26,0}
\definecolor{myorange}{rgb}{0.97,0.58,0}
\definecolor{myyellow}{rgb}{1,0.86,0}

\newcommand{\LogoExoSept}[1]{  % input : echelle
{\usefont{U}{cmss}{bx}{n}
\begin{tikzpicture}[scale=0.1*#1,transform shape]
  \fill[color=myorange] (0,0)--(4,0)--(4,-4)--(0,-4)--cycle;
  \fill[color=myred] (0,0)--(0,3)--(-3,3)--(-3,0)--cycle;
  \fill[color=myyellow] (4,0)--(7,4)--(3,7)--(0,3)--cycle;
  \node[scale=5] at (3.5,3.5) {Exo7};
\end{tikzpicture}}
}



\theoremstyle{definition}
%\newtheorem{proposition}{Proposition}
%\newtheorem{exemple}{Exemple}
%\newtheorem{theoreme}{Théorème}
\newtheorem{lemme}{Lemme}
\newtheorem{corollaire}{Corollaire}
%\newtheorem*{remarque*}{Remarque}
%\newtheorem*{miniexercice}{Mini-exercices}
%\newtheorem{definition}{Définition}




%definition d'un terme
\newcommand{\defi}[1]{{\color{myorange}\textbf{\emph{#1}}}}
\newcommand{\evidence}[1]{{\color{blue}\textbf{\emph{#1}}}}



 %----- Commandes divers ------

\newcommand{\codeinline}[1]{\texttt{#1}}

%%%%%%%%%%%%%%%%%%%%%%%%%%%%%%%%%%%%%%%%%%%%%%%%%%%%%%%%%%%%%
%%%%%%%%%%%%%%%%%%%%%%%%%%%%%%%%%%%%%%%%%%%%%%%%%%%%%%%%%%%%%

% Commande spécifique à ce chapitre
%\newcommand{\Mat}{\mathop{\text{Mat}}\nolimits}

\begin{document}

\debuttexte


%%%%%%%%%%%%%%%%%%%%%%%%%%%%%%%%%%%%%%%%%%%%%%%%%%%%%%%%%%%
\diapo

\change
Pour cette dernière partie consacrée à $\Rr^n$ nous approfondissons
notre étude des applications linéaires.

\change
On commence par le lien entre composition d'applications linéaires 
et produit de matrices,

\change
On continue avec le lien entre la bijection réciproque d'une application linéaire 
et l'inverse de matrice,

\change
On termine avec une caractérisation importante des applications linéaires :
les applications linéaires préservent la somme et la multiplication par un scalaire.


%%%%%%%%%%%%%%%%%%%%%%%%%%%%%%%%%%%%%%%%%%%%%%%%%%%%%%%%%%%
\diapo

Considérons deux applications linéaires :
$$f : \Rr^p \longrightarrow \Rr^n \qquad \text{ et } \qquad g : \Rr^q \longrightarrow \Rr^p$$

\change
Comme $\Rr^p$ est l'ensemble d'arrivée de $g$ et aussi l'ensemble de départ de $f$,
on peut donc regarder leur composition :
$$
f \circ g :  \Rr^q \longrightarrow \Rr^n.
$$

\change
Un premier résultat est que $f \circ g$ est une application linéaire.
Mais nous allons faire mieux, nous allons calculer la matrice de l'application linéaire
$f\circ g$ en fonction de celle de $f$ et de $g$.

\change
Pour cela notons : 
$A$ la matrice associée à $f$,

\change
$B$ la matrice associée à $g$,

\change
Et on cherche que vaut $C$ la matrice associée à $f \circ g$.

\change
Le résultat important est que la matrice associée à $f\circ g$ est $C=AB$.

\change
Autrement dit, la matrice associée à la composition de deux applications linéaires est 
égale au produit de leurs matrices :

$\Mat(f \circ g) = \Mat (f) \times \Mat (g)$

\change
La preuve est toute simple :
Pour n'importe quel vecteur $X \in \Rr^q$ :
$(f \circ g)(X)$ c'est $f \big(g(X)\big)$

Par définition de l'application linéaire $g$ et de sa matrice associée $B$ on a
$g(X)=BX$

Donc $(f \circ g)(X)=f\big( BX \big)$.

Aussi par définition de l'application linéaire $f$ et de sa matrice associée $A$

on a $f$ qq chose = $B$ fois ce qq chose.

Donc $f\big( BX \big)= A(BX)$

ce qui donne au final que

$(f \circ g)(X)= (AB)X.$

Ainsi la matrice de $f \circ g$ est bien le produit des matrices $AB$.


En fait le produit de matrices, qui au premier abord peut sembler 
bizarre et artificiel, est défini exactement pour vérifier cette relation.


%%%%%%%%%%%%%%%%%%%%%%%%%%%%%%%%%%%%%%%%%%%%%%%%%%%%%%%%%%%
\diapo


Voyons un exemple de composition.

Soit $f: \Rr^2 \longrightarrow \Rr^2$ la réflexion par rapport à la droite 
$(y = x)$ 

\change
La matrice de cette réflexion est
$$ A = \Mat(f) = \begin{pmatrix} 0 & 1 \\ 1 & 0 \end{pmatrix}$$

\change
Soit maintenant $g: \Rr^2 \longrightarrow \Rr^2$ 
la rotation d'angle $\theta=\frac\pi3$ (centrée à l'origine).

\change
La matrice de cette rotation est
$$B = \Mat(g) =  
\begin{pmatrix}
\cos\theta & -\sin\theta\\
\sin\theta & \cos\theta
\end{pmatrix}
=
\begin{pmatrix} 
\frac12 & -\frac{\sqrt3}{2} \\  
\frac{\sqrt3}{2} & \frac12
\end{pmatrix}.
$$


\change
Calculons la matrice de la composition $f\circ g$.
Notons $C$ cette matrice 
$\Mat (f\circ g)$.

Alors par la propriété précédente $\Mat (f\circ g) = \Mat (f) \times \Mat (g)$

Donc $C=$ matrice de $f$ fois matrice de $g$.

Ce qui après calcul donne la matrice de $f\circ g$ :
$$\begin{pmatrix} \frac{\sqrt3}{2} & \frac12  \\ \frac12 & -\frac{\sqrt3}{2}\end{pmatrix}.
$$



%%%%%%%%%%%%%%%%%%%%%%%%%%%%%%%%%%%%%%%%%%%%%%%%%%%%%%%%%%%
\diapo


Fixons $X=\left(\begin{smallmatrix}1\\0\end{smallmatrix} \right)$ et calculons l'image de $X$ par $f\circ g$.

On calcul d'abord $g(X)$, qui est ici, qui s'obtient par la rotation d'angle $\pi/3$.

Puis on applique $f$ c-à-d la réflexion par rapport à la droite $(y=x)$.

Et on obtient $f\circ g(X)$.


Attention l'ordre de la composition est important :
$g\circ f(X)$ donne un autre résultat.

En effet $f(X)$ est ici : on fait d'abord la réflexion, 
puis on applique la rotation $g$ pour obtenir $g\circ f(X)$ là.


%%%%%%%%%%%%%%%%%%%%%%%%%%%%%%%%%%%%%%%%%%%%%%%%%%%%%%%%%%%
\diapo

On peut poursuivre l'exemple en calculant maintenant la composition $g\circ f$.

Sa matrice est $D = \Mat (g \circ f)$ ce qui vaut *$\Mat (g) \times \Mat (f)$*

\change
Les calculs donnent 
$$
\begin{pmatrix}-\frac{\sqrt3}{2} & \frac12 \\  \frac12 & \frac{\sqrt3}{2} \end{pmatrix}.
$$

\change
Les matrices $C=AB$ et $D=BA$ sont distinctes, 

\change
ce qui illustre que la 
composition d'applications linéaires, comme la multiplication des matrices, 
n'est pas commutative en général. 

%%%%%%%%%%%%%%%%%%%%%%%%%%%%%%%%%%%%%%%%%%%%%%%%%%%%%%%%%%%
\diapo

Nous allons vu le lien entre composition d'applications linéaires et produit de matrice,

passons maintenant au lien entre bijection d'application et inverse de matrice.

Théorème : "Une application linéaire $f : \Rr^n \to *\Rr^n*$ est bijective
si et seulement si sa matrice associée $A$ 
 est inversible."
 
\change
Et la formule à retenir est que si $f$ est bijective, alors 
$\Mat(f^{-1}) = \big(\Mat (f) \big)^{-1}$ :

la matrice de l'inverse est l'inverse de la matrice.

\change
Justifions cette dernière formule :
L'application $f$ est définie par $f(X) = AX$.

\change
Donc si $f$ est bijective, 

\change
alors d'une part $f(X)=Y \iff X = f^{-1}(Y)$,

\change
mais d'autre part $AX=Y \iff X = A^{-1}Y$.

\change
Donc $f^{-1}(Y)=A^{-1}Y$

\change
Et la matrice de $f^{-1}$ est bien $A^{-1}$.

%%%%%%%%%%%%%%%%%%%%%%%%%%%%%%%%%%%%%%%%%%%%%%%%%%%%%%%%%%%
\diapo

Si $f: \Rr^2 \longrightarrow \Rr^2$ est la rotation d'angle $\theta$. 

\change
Alors nous savons que 
$$\Mat(f) = 
\begin{pmatrix}
\cos\, \theta & & -\sin \theta\\
\sin\, \theta &&\cos\, \theta  
\end{pmatrix}$$

\change
On sait aussi que  $f^{-1}: \Rr^2 \longrightarrow \Rr^2$ est la rotation d'angle $-\theta$.

\change
On peut le reprouver de la façon suivante :

Vu que la $\Mat(f^{-1}) = \big(\Mat (f) \big)^{-1}$.

\change
Alors 
$\Mat(f^{-1})$ = 1/déterminant fois $
= \begin{pmatrix}
\cos\, \theta &&\sin \,\theta\\
-\sin\, \theta && \cos\,\theta    
\end{pmatrix}$ obtenu en échangeant les termes de la diagonales principale,
mais ce sont tous les deux de $\cos \theta$. Et en changeant le signe des autres coefficients.

De plus le déterminant vaut $\cos^2 + \sin^2$ donc $1$.

Donc l'inverse de la matrice vaut ceci, ce que l'on peut récrire :
$\begin{pmatrix}
\cos\, (-\theta ) && -\sin\, (-\theta )\\
\sin\, (-\theta ) && \cos (-\theta )    
  \end{pmatrix}.
$.

On retrouve la matrice de la rotation d'angle $-\theta$.


%%%%%%%%%%%%%%%%%%%%%%%%%%%%%%%%%%%%%%%%%%%%%%%%%%%%%%%%%%%
\diapo

Voici les deux propriétés fondamentales qui caractérise une application linéaire.


Une application $f: \Rr^p \longrightarrow \Rr^n$ sera linéaire
est équivalent à ce que 
pour tous les vecteurs $u$, $v$ de $\Rr^p$ et pour tout scalaire $\lambda \in \Rr$,
on a
\begin{enumerate}
  \item[(i)]  $f(u+v) = f(u) + f(v)$,
  \item[(ii)] $ f(\lambda u) = \lambda f(u)\, .$
\end{enumerate} 


Reprenons : si $f$ est une application linéaire alors
\begin{enumerate}
  \item[(i)]  $f(u+v) = f(u) + f(v)$,
  \item[(ii)] $ f(\lambda u) = \lambda f(u)\, .$
\end{enumerate} 

Et réciproquement si une application $f$ vérifie les deux points
\begin{enumerate}
  \item[(i)]  $f(u+v) = f(u) + f(v)$,
  \item[(ii)] $ f(\lambda u) = \lambda f(u)\, .$
\end{enumerate} 
alors c'est une application linéaire.


Dans le cadre général des espaces vectoriels, ce sont ces deux propriétés 
qui définissent une application linéaire.

%%%%%%%%%%%%%%%%%%%%%%%%%%%%%%%%%%%%%%%%%%%%%%%%%%%%%%%%%%%
\diapo

Les $p$ vecteurs
$$
e_1 = \begin{pmatrix} 1\\0\\\vdots\\0\end{pmatrix}\qquad 
e_2 = \begin{pmatrix} 0\\1\\0\\\vdots\\0\end{pmatrix}\qquad\cdots\qquad
e_p = \begin{pmatrix} 0\\\vdots\\0\\1\end{pmatrix}
$$
sont ce que l'on appelle les \defi{vecteurs de la base canonique} de $\Rr^p$.

\change

Le théorème précédent permet maintenant de calculer facilement la matrice d'une application linéaire
dans les bases canoniques de $\Rr^p$ et $\Rr^n$.

Soit $f: \Rr^p \longrightarrow \Rr^n$ une application linéaire, 
et notons $e_1, \ldots , e_p$ 
les vecteurs de base canonique de l'espace de départ c-à-d $\Rr^p$. 

Alors la matrice de $f$ est la matrice construite par les vecteurs colonnes :
$\begin{pmatrix} f(e_1)&f(e_2)& \cdots& f(e_p)\end{pmatrix}$.


Autrement dit les vecteurs colonnes de la $\Mat(f)$
sont les images par $f$ des vecteurs de la base canonique $(e_1, \ldots , e_p)$.


%%%%%%%%%%%%%%%%%%%%%%%%%%%%%%%%%%%%%%%%%%%%%%%%%%%%%%%%%%%
\diapo

Considérons l'application linéaire $f : \Rr^3 \to \Rr^4$ définie 
par ces formules :

\change
Voici les vecteurs de la base canonique de l'espace de départ $\Rr^3$.

\change

  Calculons les images par $f$ de ces vecteurs :
  
  On trouve facilement :
  
$$f\begin{pmatrix} 1 \\ 0 \\ 0 \end{pmatrix} = \begin{pmatrix} 2 \\ -1 \\ 5 \\ 0 \end{pmatrix}\qquad 
  f\begin{pmatrix} 0 \\ 1 \\ 0 \end{pmatrix} = \begin{pmatrix} 1 \\ -4 \\ 1 \\ 3 \end{pmatrix}\qquad
  f\begin{pmatrix} 0 \\ 0 \\ 1 \end{pmatrix} = \begin{pmatrix} -1 \\ 0 \\ 1 \\ 2 \end{pmatrix}.
  $$

\change
Donc la matrice de $f$ entre les bases canoniques est :
$$\Mat(f) = 
\begin{pmatrix}
2 & 1 & -1 \\
-1& -4& 0  \\
 5& 1 & 1 \\
 0 & 3 & 2 \\              
\end{pmatrix}.$$
On retrouve bien sûr la matrice associée à la définition de $f$ !

%%%%%%%%%%%%%%%%%%%%%%%%%%%%%%%%%%%%%%%%%%%%%%%%%%%%%%%%%%%
\diapo

On termine en reprenant un exemple déjà vu précédemment :

$f : \Rr^2 \to \Rr^2$ est la réflexion par rapport à la droite $(y=x)$

\change
$g$ est la rotation du plan d'angle $\frac\pi3$ centrée à l'origine.

\change
Et on souhaite retrouver la matrice de l'application $f\circ g$.

\change
La base canonique de $\Rr^2$ est bien sûr formée des vecteurs
$\left(\begin{smallmatrix} 1 \\ 0 \end{smallmatrix} \right)$ 
et $\left(\begin{smallmatrix} 0 \\ 1 \end{smallmatrix} \right)$.

\change
Il nous faut calculer les images de ces 2 vecteurs par $f\circ g$.

Rien de compliquer : 
$\begin{pmatrix} 1 \\ 0 \end{pmatrix}$
s'envoie sur $\begin{pmatrix} \frac12 \\ \frac{\sqrt{3}}{2} \end{pmatrix}$
par la rotation d'angle $\pi/3$.

Donc 
$$f\circ g \begin{pmatrix} 1 \\ 0 \end{pmatrix}
= f \begin{pmatrix} \frac12 \\ \frac{\sqrt{3}}{2} \end{pmatrix}$$

puis par la réflexion donne 
$\begin{pmatrix} \frac{\sqrt{3}}{2} \\ \frac12  \end{pmatrix}$.

\change
Même chose pour
$f\circ g \begin{pmatrix} 0 \\ 1 \end{pmatrix}
= f \begin{pmatrix} - \frac12 \\ \frac{\sqrt{3}}{2}  \end{pmatrix}
= \begin{pmatrix} \frac{\sqrt{3}}{2} \\ -\frac12  \end{pmatrix}$

\change
La matrice de $f \circ g$ est :
donc composé du premier vecteur colonne :
$f\circ g \begin{pmatrix} 1 \\ 0 \end{pmatrix}$ et du second vecteur colonne :
$f\circ g \begin{pmatrix} 0 \\ 1 \end{pmatrix}$.

On retrouve bien la même matrice que précédemment.



%%%%%%%%%%%%%%%%%%%%%%%%%%%%%%%%%%%%%%%%%%%%%%%%%%%%%%%%%%%
\diapo

Voici les derniers exercices pour ce chapitre sur l'espace vectoriel $\Rr^n$.


\end{document}
