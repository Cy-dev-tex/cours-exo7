
%%%%%%%%%%%%%%%%%% PREAMBULE %%%%%%%%%%%%%%%%%%

\documentclass[aspectratio=169,utf8]{beamer}
%\documentclass[aspectratio=169,handout]{beamer}

\usetheme{Boadilla}
%\usecolortheme{seahorse}
\usecolortheme[RGB={245,66,24}]{structure}
\useoutertheme{infolines}

% packages
\usepackage{amsfonts,amsmath,amssymb,amsthm}
\usepackage[utf8]{inputenc}
\usepackage[T1]{fontenc}
\usepackage{lmodern}

\usepackage[francais]{babel}
\usepackage{fancybox}
\usepackage{graphicx}

\usepackage{float}
\usepackage{xfrac}

%\usepackage[usenames, x11names]{xcolor}
\usepackage{tikz}
\usepackage{pgfplots}
\usepackage{datetime}



%-----  Package unités -----
\usepackage{siunitx}
\sisetup{locale = FR,detect-all,per-mode = symbol}

%\usepackage{mathptmx}
%\usepackage{fouriernc}
%\usepackage{newcent}
%\usepackage[mathcal,mathbf]{euler}

%\usepackage{palatino}
%\usepackage{newcent}
% \usepackage[mathcal,mathbf]{euler}



% \usepackage{hyperref}
% \hypersetup{colorlinks=true, linkcolor=blue, urlcolor=blue,
% pdftitle={Exo7 - Exercices de mathématiques}, pdfauthor={Exo7}}


%section
% \usepackage{sectsty}
% \allsectionsfont{\bf}
%\sectionfont{\color{Tomato3}\upshape\selectfont}
%\subsectionfont{\color{Tomato4}\upshape\selectfont}

%----- Ensembles : entiers, reels, complexes -----
\newcommand{\Nn}{\mathbb{N}} \newcommand{\N}{\mathbb{N}}
\newcommand{\Zz}{\mathbb{Z}} \newcommand{\Z}{\mathbb{Z}}
\newcommand{\Qq}{\mathbb{Q}} \newcommand{\Q}{\mathbb{Q}}
\newcommand{\Rr}{\mathbb{R}} \newcommand{\R}{\mathbb{R}}
\newcommand{\Cc}{\mathbb{C}} 
\newcommand{\Kk}{\mathbb{K}} \newcommand{\K}{\mathbb{K}}

%----- Modifications de symboles -----
\renewcommand{\epsilon}{\varepsilon}
\renewcommand{\Re}{\mathop{\text{Re}}\nolimits}
\renewcommand{\Im}{\mathop{\text{Im}}\nolimits}
%\newcommand{\llbracket}{\left[\kern-0.15em\left[}
%\newcommand{\rrbracket}{\right]\kern-0.15em\right]}

\renewcommand{\ge}{\geqslant}
\renewcommand{\geq}{\geqslant}
\renewcommand{\le}{\leqslant}
\renewcommand{\leq}{\leqslant}
\renewcommand{\epsilon}{\varepsilon}

%----- Fonctions usuelles -----
\newcommand{\ch}{\mathop{\text{ch}}\nolimits}
\newcommand{\sh}{\mathop{\text{sh}}\nolimits}
\renewcommand{\tanh}{\mathop{\text{th}}\nolimits}
\newcommand{\cotan}{\mathop{\text{cotan}}\nolimits}
\newcommand{\Arcsin}{\mathop{\text{arcsin}}\nolimits}
\newcommand{\Arccos}{\mathop{\text{arccos}}\nolimits}
\newcommand{\Arctan}{\mathop{\text{arctan}}\nolimits}
\newcommand{\Argsh}{\mathop{\text{argsh}}\nolimits}
\newcommand{\Argch}{\mathop{\text{argch}}\nolimits}
\newcommand{\Argth}{\mathop{\text{argth}}\nolimits}
\newcommand{\pgcd}{\mathop{\text{pgcd}}\nolimits} 


%----- Commandes divers ------
\newcommand{\ii}{\mathrm{i}}
\newcommand{\dd}{\text{d}}
\newcommand{\id}{\mathop{\text{id}}\nolimits}
\newcommand{\Ker}{\mathop{\text{Ker}}\nolimits}
\newcommand{\Card}{\mathop{\text{Card}}\nolimits}
\newcommand{\Vect}{\mathop{\text{Vect}}\nolimits}
\newcommand{\Mat}{\mathop{\text{Mat}}\nolimits}
\newcommand{\rg}{\mathop{\text{rg}}\nolimits}
\newcommand{\tr}{\mathop{\text{tr}}\nolimits}


%----- Structure des exercices ------

\newtheoremstyle{styleexo}% name
{2ex}% Space above
{3ex}% Space below
{}% Body font
{}% Indent amount 1
{\bfseries} % Theorem head font
{}% Punctuation after theorem head
{\newline}% Space after theorem head 2
{}% Theorem head spec (can be left empty, meaning ‘normal’)

%\theoremstyle{styleexo}
\newtheorem{exo}{Exercice}
\newtheorem{ind}{Indications}
\newtheorem{cor}{Correction}


\newcommand{\exercice}[1]{} \newcommand{\finexercice}{}
%\newcommand{\exercice}[1]{{\tiny\texttt{#1}}\vspace{-2ex}} % pour afficher le numero absolu, l'auteur...
\newcommand{\enonce}{\begin{exo}} \newcommand{\finenonce}{\end{exo}}
\newcommand{\indication}{\begin{ind}} \newcommand{\finindication}{\end{ind}}
\newcommand{\correction}{\begin{cor}} \newcommand{\fincorrection}{\end{cor}}

\newcommand{\noindication}{\stepcounter{ind}}
\newcommand{\nocorrection}{\stepcounter{cor}}

\newcommand{\fiche}[1]{} \newcommand{\finfiche}{}
\newcommand{\titre}[1]{\centerline{\large \bf #1}}
\newcommand{\addcommand}[1]{}
\newcommand{\video}[1]{}

% Marge
\newcommand{\mymargin}[1]{\marginpar{{\small #1}}}

\def\noqed{\renewcommand{\qedsymbol}{}}


%----- Presentation ------
\setlength{\parindent}{0cm}

%\newcommand{\ExoSept}{\href{http://exo7.emath.fr}{\textbf{\textsf{Exo7}}}}

\definecolor{myred}{rgb}{0.93,0.26,0}
\definecolor{myorange}{rgb}{0.97,0.58,0}
\definecolor{myyellow}{rgb}{1,0.86,0}

\newcommand{\LogoExoSept}[1]{  % input : echelle
{\usefont{U}{cmss}{bx}{n}
\begin{tikzpicture}[scale=0.1*#1,transform shape]
  \fill[color=myorange] (0,0)--(4,0)--(4,-4)--(0,-4)--cycle;
  \fill[color=myred] (0,0)--(0,3)--(-3,3)--(-3,0)--cycle;
  \fill[color=myyellow] (4,0)--(7,4)--(3,7)--(0,3)--cycle;
  \node[scale=5] at (3.5,3.5) {Exo7};
\end{tikzpicture}}
}


\newcommand{\debutmontitre}{
  \author{} \date{} 
  \thispagestyle{empty}
  \hspace*{-10ex}
  \begin{minipage}{\textwidth}
    \titlepage  
  \vspace*{-2.5cm}
  \begin{center}
    \LogoExoSept{2.5}
  \end{center}
  \end{minipage}

  \vspace*{-0cm}
  
  % Astuce pour que le background ne soit pas discrétisé lors de la conversion pdf -> png
\begin{tikzpicture}
        \fill[opacity=0,green!60!black] (0,0)--++(0,0)--++(0,0)--++(0,0)--cycle; 
\end{tikzpicture}

% toc S'affiche trop tot :
% \tableofcontents[hideallsubsections, pausesections]
}

\newcommand{\finmontitre}{
  \end{frame}
  \setcounter{framenumber}{0}
} % ne marche pas pour une raison obscure

%----- Commandes supplementaires ------

% \usepackage[landscape]{geometry}
% \geometry{top=1cm, bottom=3cm, left=2cm, right=10cm, marginparsep=1cm
% }
% \usepackage[a4paper]{geometry}
% \geometry{top=2cm, bottom=2cm, left=2cm, right=2cm, marginparsep=1cm
% }

%\usepackage{standalone}


% New command Arnaud -- november 2011
\setbeamersize{text margin left=24ex}
% si vous modifier cette valeur il faut aussi
% modifier le decalage du titre pour compenser
% (ex : ici =+10ex, titre =-5ex

\theoremstyle{definition}
%\newtheorem{proposition}{Proposition}
%\newtheorem{exemple}{Exemple}
%\newtheorem{theoreme}{Théorème}
%\newtheorem{lemme}{Lemme}
%\newtheorem{corollaire}{Corollaire}
%\newtheorem*{remarque*}{Remarque}
%\newtheorem*{miniexercice}{Mini-exercices}
%\newtheorem{definition}{Définition}

% Commande tikz
\usetikzlibrary{calc}
\usetikzlibrary{patterns,arrows}
\usetikzlibrary{matrix}
\usetikzlibrary{fadings} 

%definition d'un terme
\newcommand{\defi}[1]{{\color{myorange}\textbf{\emph{#1}}}}
\newcommand{\evidence}[1]{{\color{blue}\textbf{\emph{#1}}}}
\newcommand{\assertion}[1]{\emph{\og#1\fg}}  % pour chapitre logique
%\renewcommand{\contentsname}{Sommaire}
\renewcommand{\contentsname}{}
\setcounter{tocdepth}{2}



%------ Figures ------

\def\myscale{1} % par défaut 
\newcommand{\myfigure}[2]{  % entrée : echelle, fichier figure
\def\myscale{#1}
\begin{center}
\footnotesize
{#2}
\end{center}}


%------ Encadrement ------

\usepackage{fancybox}


\newcommand{\mybox}[1]{
\setlength{\fboxsep}{7pt}
\begin{center}
\shadowbox{#1}
\end{center}}

\newcommand{\myboxinline}[1]{
\setlength{\fboxsep}{5pt}
\raisebox{-10pt}{
\shadowbox{#1}
}
}

%--------------- Commande beamer---------------
\newcommand{\beameronly}[1]{#1} % permet de mettre des pause dans beamer pas dans poly


\setbeamertemplate{navigation symbols}{}
\setbeamertemplate{footline}  % tiré du fichier beamerouterinfolines.sty
{
  \leavevmode%
  \hbox{%
  \begin{beamercolorbox}[wd=.333333\paperwidth,ht=2.25ex,dp=1ex,center]{author in head/foot}%
    % \usebeamerfont{author in head/foot}\insertshortauthor%~~(\insertshortinstitute)
    \usebeamerfont{section in head/foot}{\bf\insertshorttitle}
  \end{beamercolorbox}%
  \begin{beamercolorbox}[wd=.333333\paperwidth,ht=2.25ex,dp=1ex,center]{title in head/foot}%
    \usebeamerfont{section in head/foot}{\bf\insertsectionhead}
  \end{beamercolorbox}%
  \begin{beamercolorbox}[wd=.333333\paperwidth,ht=2.25ex,dp=1ex,right]{date in head/foot}%
    % \usebeamerfont{date in head/foot}\insertshortdate{}\hspace*{2em}
    \insertframenumber{} / \inserttotalframenumber\hspace*{2ex} 
  \end{beamercolorbox}}%
  \vskip0pt%
}


\definecolor{mygrey}{rgb}{0.5,0.5,0.5}
\setlength{\parindent}{0cm}
%\DeclareTextFontCommand{\helvetica}{\fontfamily{phv}\selectfont}

% background beamer
\definecolor{couleurhaut}{rgb}{0.85,0.9,1}  % creme
\definecolor{couleurmilieu}{rgb}{1,1,1}  % vert pale
\definecolor{couleurbas}{rgb}{0.85,0.9,1}  % blanc
\setbeamertemplate{background canvas}[vertical shading]%
[top=couleurhaut,middle=couleurmilieu,midpoint=0.4,bottom=couleurbas] 
%[top=fondtitre!05,bottom=fondtitre!60]



\makeatletter
\setbeamertemplate{theorem begin}
{%
  \begin{\inserttheoremblockenv}
  {%
    \inserttheoremheadfont
    \inserttheoremname
    \inserttheoremnumber
    \ifx\inserttheoremaddition\@empty\else\ (\inserttheoremaddition)\fi%
    \inserttheorempunctuation
  }%
}
\setbeamertemplate{theorem end}{\end{\inserttheoremblockenv}}

\newenvironment{theoreme}[1][]{%
   \setbeamercolor{block title}{fg=structure,bg=structure!40}
   \setbeamercolor{block body}{fg=black,bg=structure!10}
   \begin{block}{{\bf Th\'eor\`eme }#1}
}{%
   \end{block}%
}


\newenvironment{proposition}[1][]{%
   \setbeamercolor{block title}{fg=structure,bg=structure!40}
   \setbeamercolor{block body}{fg=black,bg=structure!10}
   \begin{block}{{\bf Proposition }#1}
}{%
   \end{block}%
}

\newenvironment{corollaire}[1][]{%
   \setbeamercolor{block title}{fg=structure,bg=structure!40}
   \setbeamercolor{block body}{fg=black,bg=structure!10}
   \begin{block}{{\bf Corollaire }#1}
}{%
   \end{block}%
}

\newenvironment{mydefinition}[1][]{%
   \setbeamercolor{block title}{fg=structure,bg=structure!40}
   \setbeamercolor{block body}{fg=black,bg=structure!10}
   \begin{block}{{\bf Définition} #1}
}{%
   \end{block}%
}

\newenvironment{lemme}[0]{%
   \setbeamercolor{block title}{fg=structure,bg=structure!40}
   \setbeamercolor{block body}{fg=black,bg=structure!10}
   \begin{block}{\bf Lemme}
}{%
   \end{block}%
}

\newenvironment{remarque}[1][]{%
   \setbeamercolor{block title}{fg=black,bg=structure!20}
   \setbeamercolor{block body}{fg=black,bg=structure!5}
   \begin{block}{Remarque #1}
}{%
   \end{block}%
}


\newenvironment{exemple}[1][]{%
   \setbeamercolor{block title}{fg=black,bg=structure!20}
   \setbeamercolor{block body}{fg=black,bg=structure!5}
   \begin{block}{{\bf Exemple }#1}
}{%
   \end{block}%
}


\newenvironment{miniexercice}[0]{%
   \setbeamercolor{block title}{fg=structure,bg=structure!20}
   \setbeamercolor{block body}{fg=black,bg=structure!5}
   \begin{block}{Mini-exercices}
}{%
   \end{block}%
}


\newenvironment{tp}[0]{%
   \setbeamercolor{block title}{fg=structure,bg=structure!40}
   \setbeamercolor{block body}{fg=black,bg=structure!10}
   \begin{block}{\bf Travaux pratiques}
}{%
   \end{block}%
}
\newenvironment{exercicecours}[1][]{%
   \setbeamercolor{block title}{fg=structure,bg=structure!40}
   \setbeamercolor{block body}{fg=black,bg=structure!10}
   \begin{block}{{\bf Exercice }#1}
}{%
   \end{block}%
}
\newenvironment{algo}[1][]{%
   \setbeamercolor{block title}{fg=structure,bg=structure!40}
   \setbeamercolor{block body}{fg=black,bg=structure!10}
   \begin{block}{{\bf Algorithme}\hfill{\color{gray}\texttt{#1}}}
}{%
   \end{block}%
}


\setbeamertemplate{proof begin}{
   \setbeamercolor{block title}{fg=black,bg=structure!20}
   \setbeamercolor{block body}{fg=black,bg=structure!5}
   \begin{block}{{\footnotesize Démonstration}}
   \footnotesize
   \smallskip}
\setbeamertemplate{proof end}{%
   \end{block}}
\setbeamertemplate{qed symbol}{\openbox}


\makeatother
\usecolortheme[RGB={0,0,102}]{structure}
   
%%%%%%%%%%%%%%%%%%%%%%%%%%%%%%%%%%%%%%%%%%%%%%%%%%%%%%%%%%%%%
%%%%%%%%%%%%%%%%%%%%%%%%%%%%%%%%%%%%%%%%%%%%%%%%%%%%%%%%%%%%%


\begin{document}


\title{{\bf La chaînette}}
\subtitle{\'Equation de la chaînette}

\begin{frame}
  
  \debutmontitre

  \pause

{\footnotesize
\hfill
\setbeamercovered{transparent=50}
\begin{minipage}{0.6\textwidth}
  \begin{itemize}
    \item<3-> Découpage infinitésimal de la chaînette
    \item<4-> Principe fondamental de la mécanique
    \item<5-> Tension horizontale
    \item<6-> Tension verticale et poids
    \item<7-> Calcul de l'équation
  \end{itemize}
\end{minipage}
}

\end{frame}

\setcounter{framenumber}{0}


%%%%%%%%%%%%%%%%%%%%%%%%%%%%%%%%%%%%%%%%%%%%%%%%%%%%%%%%%%%%%%%%
\section{Découpage infinitésimal de la chaînette}

\begin{frame}

\begin{minipage}{0.59\textwidth}
\begin{itemize}
  \uncover<3->{\item \evidence{Le poids $\vec P$}}
  \begin{itemize}
    \uncover<4->{\item Force verticale}
    \uncover<5->{\item $\vec P = -P \vec j = - \mu\cdot d\ell \cdot g \cdot \vec j$}
    \uncover<6->{\item $\mu$ est la masse linéique}
    \uncover<7->{\item masse de notre petit bout est $\mu \cdot d\ell$}
    \uncover<8->{\item $g \approx 9,81 \; m/s^2$}
  \end{itemize}

  \uncover<9->{\item \evidence{La tension à gauche $\vec T(x)$}}
  \begin{itemize}
    \uncover<10->{\item S'applique au point d'abscisse $x$}
    \uncover<11->{\item Force tangente à la chaînette}
  \end{itemize}

  \uncover<12->{\item  \evidence{La tension à droite $-\vec T(x+dx)$}} 
  \begin{itemize}
    \uncover<13->{\item S'applique au point d'abscisse $x+dx$}
    \uncover<14->{\item S'oppose à la tension à gauche du morceau suivant}
    \uncover<15->{\item C'est donc $-\vec T(x+dx)$}
  \end{itemize}
\end{itemize}
\end{minipage}
\hspace*{-2.1em}
\begin{minipage}{0.29\textwidth}
\shorthandoff{:}
\myfigure{1}{
\tikzinput{fig_chainette08-pres}
}
\shorthandon{:}
\end{minipage}
\end{frame}



%%%%%%%%%%%%%%%%%%%%%%%%%%%%%%%%%%%%%%%%%%%%%%%%%%%%%%%%%%%%%%%%
\section{Principe fondamental de la mécanique}

\begin{frame}
Par le principe fondamental de la mécanique, la somme des forces est nulle
\uncover<3->{: \myboxinline{$\vec P + \vec T(x)-\vec T(x+dx) = \vec 0$}}

\uncover<2->{
\begin{minipage}{0.49\textwidth}
\shorthandoff{:}
\myfigure{0.7}{
\tikzinput{fig_chainette08}
}
\shorthandon{:}
\end{minipage}
}
\pause\pause\pause
\begin{minipage}{0.49\textwidth}
\shorthandoff{:}
\myfigure{0.9}{
\tikzinput{fig_chainette18}
}
\shorthandon{:}
\pause
\hfil $\vec T(x) = -T_h(x)\vec i - T_v(x) \vec j$  
\end{minipage}

\bigskip
\pause

$$-P \vec j - T_h(x)\vec i - T_v(x) \vec j - \left( - T_h(x+dx)\vec i - T_v(x+dx)  \vec j \right) = \vec 0$$

\pause
\mybox{$
\left\lbrace
\begin{array}{rcl}
T_h(x+dx)-T_h(x) &=& 0 \\
\pause
T_v(x+dx) - T_v(x) - P &=& 0 \\
\end{array}
\right.
$
\pause
}
\end{frame}


%%%%%%%%%%%%%%%%%%%%%%%%%%%%%%%%%%%%%%%%%%%%%%%%%%%%%%%%%%%%%%%%
\section{Tension horizontale}

\begin{frame}


\begin{lemme}
\label{lem:Th}
La \evidence{tension horizontale} est indépendante de $x$ :
$$T_h(x) = T_h$$
\end{lemme}
\pause

\begin{proof}
\begin{itemize}
\setlength{\itemsep}{6pt}
  \uncover<3->{\item Principe fondamental : $T_h(x+dx)-T_h(x)=0$}
  
  \uncover<4->{\item $\displaystyle \frac{T_h(x+dx)-T_h(x)}{x+dx-x}=0$}
  
  \uncover<5->{\item Lorsque $dx$ tend vers $0$ : $\frac{T_h(x+dx)-T_h(x)}{dx}$ tend $T'_h(x)$}
  
  \uncover<6->{\item Bilan : $T_h'(x)=0$}
  
  \uncover<7->{\item $T_h(x)$ est donc une fonction constante}
\end{itemize}

\end{proof}
\vspace*{-22ex}
\hfill
\begin{minipage}{0.39\textwidth}
\shorthandoff{:}
\myfigure{0.7}{
\tikzinput{fig_chainette08}
}
\shorthandon{:}        
\end{minipage}
\end{frame}


%%%%%%%%%%%%%%%%%%%%%%%%%%%%%%%%%%%%%%%%%%%%%%%%%%%%%%%%%%%%%%%%
\section{Tension verticale et poids}

\begin{frame}


\begin{itemize}
\setlength{\itemsep}{6pt}
  \item Notons $y(x)$ l'équation de la chaînette
  
  \uncover<2->{\item Chaque morceau infinitésimal est considéré comme rectiligne}
  \uncover<3->{\item Théorème de Pythagore : $d \ell^2 = dx^2 + dy^2$}
  \uncover<4->{\item $\left(\frac{d\ell}{dx}\right)^2=1+ \left(\frac{dy}{dx}\right)^2$}
  \uncover<5->{\item $\frac{d\ell}{dx}=\sqrt{1+ \left(\frac{dy}{dx}\right)^2}$}
 \end{itemize} 
\vspace*{-12ex}
\shorthandoff{:}
\myfigure{1.5}{
\tikzinput{fig_chainette09}
}
\shorthandon{:}
 
\end{frame}


\begin{frame} 

\evidence{Tension verticale}

\begin{itemize}
\setlength{\itemsep}{6pt}
  \uncover<3->{\item Principe fondamental : $T_v(x+dx) - T_v(x) - P = 0$}
  \uncover<4->{\item Le poids $P= \mu g d\ell$}
  \uncover<5->{\item $T_v(x+dx)-T_v(x)= \mu g d\ell$}
  \uncover<6->{\item $\frac{T_v(x+dx)-T_v(x)}{dx} = \mu g \frac {d\ell}{dx}$}\uncover<7->{$= \mu g \sqrt{1+ \left(\frac{dy}{dx}\right)^2}$}
  \uncover<8->{\item $\frac{T_v(x+dx)-T_v(x)}{dx}$ vaut à la limite $T_v'(x)$}
  \uncover<9->{\item $\frac{dy}{dx}$ vaut à la limite $y'(x)$}
  \uncover<10->{\item $T_v'(x) = \mu g \sqrt{1+ y'(x)^2}$}
\end{itemize}

\vspace*{-18ex}
\hfill
\uncover<2->{
\begin{minipage}{0.6\textwidth}
\shorthandoff{:}
\myfigure{1}{
\tikzinput{fig_chainette08}
}
\shorthandon{:}        
\end{minipage}
}



\end{frame}

%%%%%%%%%%%%%%%%%%%%%%%%%%%%%%%%%%%%%%%%%%%%%%%%%%%%%%%%%%%%%%%%
\section{Calcul de l'équation}

\begin{frame}
\begin{theoreme}
\label{th:chainette}
Une équation de la chaînette est 
\mybox{$\displaystyle y(x) = a \ch \left( \frac x a\right)$}
\pause
où $a$ est une constante qui vaut $a = \frac {T_h}{\mu g}$
\end{theoreme}
\end{frame}


\begin{frame}
\evidence{1. Lien tension verticale/tension horizontale}

\pause 
\begin{itemize}
\setlength{\itemsep}{6pt}
  \uncover<3->{\item $T_h(x) = T(x) \cos \alpha(x)$}

  \uncover<4->{\item $T_v(x) = T(x) \sin \alpha(x)$}
  
  \uncover<5->{\item $T_v(x) = T_h(x) \tan \alpha(x)$}
  
  \uncover<7->{\item $\tan \alpha(x) = \frac{dy}{dx} = y'(x)$}
  
  \uncover<8->{\item $T_v(x) = T_h(x) \cdot y'(x)$}
\end{itemize}
\vspace*{-8ex}
\shorthandoff{:}
\myfigure{1}{
\uncover<2->{
\tikzinput{fig_chainette19}
}
\uncover<6->{
\hspace*{-2em}
\tikzinput{fig_chainette09-bis}
}
}
\shorthandon{:}


\end{frame}


\begin{frame}
\evidence{2. \'Equations différentielles}

\pause
\begin{itemize}
\setlength{\itemsep}{6pt}
  \item $T_v(x) = T_h(x) \cdot y'(x)$
  \pause
  \item $T_v'(x) = T_h \cdot y''(x)$
  \pause
  \item On a vu $T_v'(x) = \mu g \sqrt{1+ y'(x)^2}$
  \pause
  \item Ainsi   $\mu g \sqrt{1+ y'(x)^2} = T_h \cdot y''(x)$
  \pause
  \item Posons la constante $a =\frac{T_h}{\mu g}$
  \pause
  \item Posons $z(x)= y'(x)$
  \pause
  \item $\sqrt{1+ z(x)^2} = a \cdot z'(x)$
  \pause
  \item $\frac{z'(x)}{\sqrt{1+z(x)^2}} = \frac 1 a$
\end{itemize}

\end{frame}


\begin{frame}
\evidence{3. Solutions de l'équation différentielle}

\pause
\begin{itemize}
\setlength{\itemsep}{6pt}
  \item $\frac{z'(x)}{\sqrt{1+z(x)^2}} = \frac 1 a$
  \pause
  \item Une primitive de $\frac{z'(x)}{\sqrt{1+z(x)^2}}$ est $\Argsh z(x)$
  \pause
  \item Donc $\Argsh z(x) = \frac x a + \alpha$
  \pause
  \item Ainsi $y'(x) = z(x) = \sh\left(\frac x a + \alpha\right)$
  \pause
  \item $y(x) = a \ch \left(\frac x a + \alpha\right) + \beta$
\end{itemize}

\end{frame}


\begin{frame}
\evidence{4. Choix des constantes}

\pause
\begin{itemize}
\setlength{\itemsep}{6pt}
  \uncover<2->{\item $y(x) = a \ch \left(\frac x a + \alpha\right) + \beta$}
  \uncover<4->{\item Le point le plus bas de la chaînette est $(0,a)$}
  \uncover<5->{\item Alors $y(0)=a$ et $y'(0)=0$}
  \uncover<6->{\item $\alpha=0$ et $\beta=0$}
  \uncover<7->{\item L'équation est $y(x) = a \ch \left(\frac x a\right)$}
\end{itemize}
\uncover<3->{
\shorthandoff{:}
\myfigure{1}{
\tikzinput{fig_chainette10}
}
\shorthandon{:}
}

\end{frame}



\end{document}