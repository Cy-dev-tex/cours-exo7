
%%%%%%%%%%%%%%%%%% PREAMBULE %%%%%%%%%%%%%%%%%%


\documentclass[12pt]{article}

\usepackage{amsfonts,amsmath,amssymb,amsthm}
\usepackage[utf8]{inputenc}
\usepackage[T1]{fontenc}
\usepackage[francais]{babel}


% packages
\usepackage{amsfonts,amsmath,amssymb,amsthm}
\usepackage[utf8]{inputenc}
\usepackage[T1]{fontenc}
%\usepackage{lmodern}

\usepackage[francais]{babel}
\usepackage{fancybox}
\usepackage{graphicx}

\usepackage{float}

%\usepackage[usenames, x11names]{xcolor}
\usepackage{tikz}
\usepackage{datetime}

\usepackage{mathptmx}
%\usepackage{fouriernc}
%\usepackage{newcent}
\usepackage[mathcal,mathbf]{euler}

%\usepackage{palatino}
%\usepackage{newcent}


% Commande spéciale prompteur

%\usepackage{mathptmx}
%\usepackage[mathcal,mathbf]{euler}
%\usepackage{mathpple,multido}

\usepackage[a4paper]{geometry}
\geometry{top=2cm, bottom=2cm, left=1cm, right=1cm, marginparsep=1cm}

\newcommand{\change}{{\color{red}\rule{\textwidth}{1mm}\\}}

\newcounter{mydiapo}

\newcommand{\diapo}{\newpage
\hfill {\normalsize  Diapo \themydiapo \quad \texttt{[\jobname]}} \\
\stepcounter{mydiapo}}


%%%%%%% COULEURS %%%%%%%%%%

% Pour blanc sur noir :
%\pagecolor[rgb]{0.5,0.5,0.5}
% \pagecolor[rgb]{0,0,0}
% \color[rgb]{1,1,1}



%\DeclareFixedFont{\myfont}{U}{cmss}{bx}{n}{18pt}
\newcommand{\debuttexte}{
%%%%%%%%%%%%% FONTES %%%%%%%%%%%%%
\renewcommand{\baselinestretch}{1.5}
\usefont{U}{cmss}{bx}{n}
\bfseries

% Taille normale : commenter le reste !
%Taille Arnaud
%\fontsize{19}{19}\selectfont

% Taille Barbara
%\fontsize{21}{22}\selectfont

%Taille François
%\fontsize{25}{30}\selectfont

%Taille Pascal
%\fontsize{25}{30}\selectfont

%Taille Laura
%\fontsize{30}{35}\selectfont


%\myfont
%\usefont{U}{cmss}{bx}{n}

%\Huge
%\addtolength{\parskip}{\baselineskip}
}


% \usepackage{hyperref}
% \hypersetup{colorlinks=true, linkcolor=blue, urlcolor=blue,
% pdftitle={Exo7 - Exercices de mathématiques}, pdfauthor={Exo7}}


%section
% \usepackage{sectsty}
% \allsectionsfont{\bf}
%\sectionfont{\color{Tomato3}\upshape\selectfont}
%\subsectionfont{\color{Tomato4}\upshape\selectfont}

%----- Ensembles : entiers, reels, complexes -----
\newcommand{\Nn}{\mathbb{N}} \newcommand{\N}{\mathbb{N}}
\newcommand{\Zz}{\mathbb{Z}} \newcommand{\Z}{\mathbb{Z}}
\newcommand{\Qq}{\mathbb{Q}} \newcommand{\Q}{\mathbb{Q}}
\newcommand{\Rr}{\mathbb{R}} \newcommand{\R}{\mathbb{R}}
\newcommand{\Cc}{\mathbb{C}} 
\newcommand{\Kk}{\mathbb{K}} \newcommand{\K}{\mathbb{K}}

%----- Modifications de symboles -----
\renewcommand{\epsilon}{\varepsilon}
\renewcommand{\Re}{\mathop{\text{Re}}\nolimits}
\renewcommand{\Im}{\mathop{\text{Im}}\nolimits}
%\newcommand{\llbracket}{\left[\kern-0.15em\left[}
%\newcommand{\rrbracket}{\right]\kern-0.15em\right]}

\renewcommand{\ge}{\geqslant}
\renewcommand{\geq}{\geqslant}
\renewcommand{\le}{\leqslant}
\renewcommand{\leq}{\leqslant}

%----- Fonctions usuelles -----
\newcommand{\ch}{\mathop{\mathrm{ch}}\nolimits}
\newcommand{\sh}{\mathop{\mathrm{sh}}\nolimits}
\renewcommand{\tanh}{\mathop{\mathrm{th}}\nolimits}
\newcommand{\cotan}{\mathop{\mathrm{cotan}}\nolimits}
\newcommand{\Arcsin}{\mathop{\mathrm{Arcsin}}\nolimits}
\newcommand{\Arccos}{\mathop{\mathrm{Arccos}}\nolimits}
\newcommand{\Arctan}{\mathop{\mathrm{Arctan}}\nolimits}
\newcommand{\Argsh}{\mathop{\mathrm{Argsh}}\nolimits}
\newcommand{\Argch}{\mathop{\mathrm{Argch}}\nolimits}
\newcommand{\Argth}{\mathop{\mathrm{Argth}}\nolimits}
\newcommand{\pgcd}{\mathop{\mathrm{pgcd}}\nolimits} 

\newcommand{\Card}{\mathop{\text{Card}}\nolimits}
\newcommand{\Ker}{\mathop{\text{Ker}}\nolimits}
\newcommand{\id}{\mathop{\text{id}}\nolimits}
\newcommand{\ii}{\mathrm{i}}
\newcommand{\dd}{\mathrm{d}}
\newcommand{\Vect}{\mathop{\text{Vect}}\nolimits}
\newcommand{\Mat}{\mathop{\mathrm{Mat}}\nolimits}
\newcommand{\rg}{\mathop{\text{rg}}\nolimits}
\newcommand{\tr}{\mathop{\text{tr}}\nolimits}
\newcommand{\ppcm}{\mathop{\text{ppcm}}\nolimits}

%----- Structure des exercices ------

\newtheoremstyle{styleexo}% name
{2ex}% Space above
{3ex}% Space below
{}% Body font
{}% Indent amount 1
{\bfseries} % Theorem head font
{}% Punctuation after theorem head
{\newline}% Space after theorem head 2
{}% Theorem head spec (can be left empty, meaning ‘normal’)

%\theoremstyle{styleexo}
\newtheorem{exo}{Exercice}
\newtheorem{ind}{Indications}
\newtheorem{cor}{Correction}


\newcommand{\exercice}[1]{} \newcommand{\finexercice}{}
%\newcommand{\exercice}[1]{{\tiny\texttt{#1}}\vspace{-2ex}} % pour afficher le numero absolu, l'auteur...
\newcommand{\enonce}{\begin{exo}} \newcommand{\finenonce}{\end{exo}}
\newcommand{\indication}{\begin{ind}} \newcommand{\finindication}{\end{ind}}
\newcommand{\correction}{\begin{cor}} \newcommand{\fincorrection}{\end{cor}}

\newcommand{\noindication}{\stepcounter{ind}}
\newcommand{\nocorrection}{\stepcounter{cor}}

\newcommand{\fiche}[1]{} \newcommand{\finfiche}{}
\newcommand{\titre}[1]{\centerline{\large \bf #1}}
\newcommand{\addcommand}[1]{}
\newcommand{\video}[1]{}

% Marge
\newcommand{\mymargin}[1]{\marginpar{{\small #1}}}



%----- Presentation ------
\setlength{\parindent}{0cm}

%\newcommand{\ExoSept}{\href{http://exo7.emath.fr}{\textbf{\textsf{Exo7}}}}

\definecolor{myred}{rgb}{0.93,0.26,0}
\definecolor{myorange}{rgb}{0.97,0.58,0}
\definecolor{myyellow}{rgb}{1,0.86,0}

\newcommand{\LogoExoSept}[1]{  % input : echelle
{\usefont{U}{cmss}{bx}{n}
\begin{tikzpicture}[scale=0.1*#1,transform shape]
  \fill[color=myorange] (0,0)--(4,0)--(4,-4)--(0,-4)--cycle;
  \fill[color=myred] (0,0)--(0,3)--(-3,3)--(-3,0)--cycle;
  \fill[color=myyellow] (4,0)--(7,4)--(3,7)--(0,3)--cycle;
  \node[scale=5] at (3.5,3.5) {Exo7};
\end{tikzpicture}}
}



\theoremstyle{definition}
%\newtheorem{proposition}{Proposition}
%\newtheorem{exemple}{Exemple}
%\newtheorem{theoreme}{Théorème}
\newtheorem{lemme}{Lemme}
\newtheorem{corollaire}{Corollaire}
%\newtheorem*{remarque*}{Remarque}
%\newtheorem*{miniexercice}{Mini-exercices}
%\newtheorem{definition}{Définition}




%definition d'un terme
\newcommand{\defi}[1]{{\color{myorange}\textbf{\emph{#1}}}}
\newcommand{\evidence}[1]{{\color{blue}\textbf{\emph{#1}}}}



 %----- Commandes divers ------

\newcommand{\codeinline}[1]{\texttt{#1}}
\newcommand{\Sage}{\texttt{Sage}}
%%%%%%%%%%%%%%%%%%%%%%%%%%%%%%%%%%%%%%%%%%%%%%%%%%%%%%%%%%%%%
%%%%%%%%%%%%%%%%%%%%%%%%%%%%%%%%%%%%%%%%%%%%%%%%%%%%%%%%%%%%%


\begin{document}

\debuttexte


%%%%%%%%%%%%%%%%%%%%%%%%%%%%%%%%%%%%%%%%%%%%%%%%%%%%%%%%%%%
\diapo

La visualisation est une étape importante dans l'élaboration 
des preuves en mathématiques. 

~


L'avènement des logiciels de calcul 
formel possédant une interface graphique évoluée a 
rendu cette phase attrayante. 

~

Nous allons explorer 
quelques possibilités graphiques offertes par \Sage.

\change

\change
Nous commencerons par les courbes paramétrées

\change
puis les courbes en coordonnées polaires

\change
ensuite les courbes définies par une équation implicite

\change
et nous terminerons en décrivant brièvement plusieurs façons de tracer des surfaces.



%%%%%%%%%%%%%%%%%%%%%%%%%%%%%%%%%%%%%%%%%%%%%%%%%%%%%%%%%%%
\diapo


Nous avons déjà vu comment tracer les graphes de fonctions avec la commande \codeinline{plot}.

~

Par exemple cette commande 
trace le graphe de la fonction $f$ définie par $f(x)=\sin(x)\exp(x)$ sur l'intervalle $[-3,3]$.

~

Au delà des graphes de fonctions, \Sage\ permet le tracé de courbes et de surfaces.% définies par d'autres méthodes.

\change

Commençons par tracer la courbe paramétrée plane donnée par les points 
de coordonnées $(f(t), g(t))$ pour un paramètre $t$ variant 
dans l'intervalle $[a,b]$. 

\change
La commande est \codeinline{parametric\_plot()} 

\change
Voici le code qui permet d'afficher la lemniscate de Bernoulli :

on définit tout d'abord $x$ puis $y$ en fonction du paramètre $t$, 

la machine calcule dans un premier temps les éléments du graphe et enfin l'affiche.



%%%%%%%%%%%%%%%%%%%%%%%%%%%%%%%%%%%%%%%%%%%%%%%%%%%%%%%%%%%
\diapo

A votre tour de tracer la spirale de Fermat définie ainsi,%ayant cette équation


~

puis la courbe dite du papillon.



%%%%%%%%%%%%%%%%%%%%%%%%%%%%%%%%%%%%%%%%%%%%%%%%%%%%%%%%%%%
\diapo

Voici les deux courbes que vous devriez avoir obtenues.

\change

Les commandes de tracé possèdent de nombreuses options, 
que vous pouvez découvrir grâce 
à la commande \codeinline{help(parametric\_plot)}. 

\change
Par exemple :

il est possible d'imposer un repère orthonormé 

\change
il est possible de changer le nombre de points du tracé

\change
il est possible de ne pas afficher les axes
  
\change
ou d'afficher une grille.
 
\change
et bien sûr changer la couleur du trait !

%%%%%%%%%%%%%%%%%%%%%%%%%%%%%%%%%%%%%%%%%%%%%%%%%%%%%%%%%%%
\diapo

Abordons maintenant le cas des courbes définies en coordonnées polaires. 

~

Il s'agit de l'ensemble des points de coordonnées polaires $[r(t):t]$ pour $t$ variant dans un intervalle $[a,b]$.

$r(t)$ est le rayon (positif ou négatif) et $t$ est l'angle.

\change
La commande qui permet de calculer les éléments du tracé du graphe d'une courbe en coordonnées polaires est \codeinline{polar\_plot()}, 



\change
Voici par exemple la courbe du folium de Dürer d'équation polaire 
$r(t) = \sin \frac t 2$.

\change
et le code qui permet de calculer puis d'afficher cette courbe !


%%%%%%%%%%%%%%%%%%%%%%%%%%%%%%%%%%%%%%%%%%%%%%%%%%%%%%%%%%%
\diapo

Voici deux courbes en coordonnées polaires, dont vous pourrez rechercher l'origine historique. 

\change
D'abord la courbe du Lituus d'équation polaire définie par la relation
  $r(t)^2 = \frac{1}{t}$
  
\change  
Puis la cochléoïde d'équation polaire 
  $r(t) = \frac{\sin t}{t}$


\change

%[[à raccourcir ??]]

Avant de vous laisser chercher, voici une petite indication : 

~

comme les courbes ne sont pas définies sur un intervalle mais sur l'union de deux intervalles, il est conseillé de calculer chaque tracé en deux morceaux qui seront affichés simultanément.
%la moitié de courbe correspondant aux réels négatifs, on obtient un graphe $G_1$, puis la moitié de courbe correspondant aux réels positifs, obtient un graphe $G_2$.
%On superpose les graphes avec \codeinline{G = G1 + G2}, et on affiche \codeinline{G}.


%%%%%%%%%%%%%%%%%%%%%%%%%%%%%%%%%%%%%%%%%%%%%%%%%%%%%%%%%%%
\diapo

Une courbe peut avoir plusieurs types de définitions, comme par exemple une équation implicite. %représentations, 
Voici l'exemple bien connu du cercle d'équation $x^2+y^2=1$.
%qui peut être défini comme l'ensemble des couples $x$, $y$ vérifiant l'équation : $x^2+y^2 = 1$.

\change
En général, l'équation $f(x,y)=0$, définit une courbe :

\change
c'est l'ensemble des couples $(x,y)$ 
dans $[a,b]\times[c,d]$ qui vérifient l'équation $f(x,y)=0$.

\change
La commande qui permet de tracer une courbe donnée par une équation implicite est \codeinline{implicit\_plot()}.

\change
Voici une autre courbe en forme de papillon, 

\change
elle est cette fois définie par l'équation algébrique $x^6+y^6-x^2=0$. 

~

Noter qu'il est possible de sauvegarder cette image dans un fichier externe avec la commande {\sl save}.



%%%%%%%%%%%%%%%%%%%%%%%%%%%%%%%%%%%%%%%%%%%%%%%%%%%%%%%%%%%
\diapo


Dans ce tp, nous allons construire une fonction qui renvoie le tracé de la courbe définie par
 l'équation $y^2-x^3+x+c = 0$
  en fonction d'un paramètre $c$ réel.%\in \Rr$.

~


Pour visualiser l'évolution de l'allure de la courbe lorsque le paramètre $c$ varie, il est possible de construire 
une animation avec la commande  \codeinline{animate}, 


~


Voici par exemple trois courbes pour trois valeurs différentes de $c$. %RAF


\change

\change

\change

%%%%%%%%%%%%%%%%%%%%%%%%%%%%%%%%%%%%%%%%%%%%%%%%%%%%%%%%%%%
\diapo


Nous allons maintenant découvrir, à l'aide de l'énoncé suivant, différentes méthodes 
pour tracer des surfaces avec Sage.

~

Cela débute avec le tracé du graphe d'une fonction de deux variables ;

~

pour cela nous utiliserons la commande \codeinline{plot3d}.

\change 

Dans le cas d'une surface définie par une équation implicite, il suffira d'utiliser la commande
 \codeinline{implicit\_plot3d}.

\change

Enfin, pour tracer une nappe dont les trois coordonnées dépendent de deux paramètres , ici $s$ et $t$.

nous utiliserons la commande \codeinline{parametric\_plot3d}.
 


%%%%%%%%%%%%%%%%%%%%%%%%%%%%%%%%%%%%%%%%%%%%%%%%%%%%%%%%%%%
\diapo

Ce n'est pas fini !

~


Voici encore deux surfaces  à tracer :
la première est définie en coordonnées cylindriques,

\change
la seconde est définie en coordonnées sphériques.


%%%%%%%%%%%%%%%%%%%%%%%%%%%%%%%%%%%%%%%%%%%%%%%%%%%%%%%%%%%
\diapo

Voici, sans préciser le code utilisé, une présentation des graphes demandés. 

~


1. Ici le graphe de la fonction de deux variables,

\change

2. la surface d'Enneper définie par une équation implicite,

\change

3. la nappe paramétrée,

\change

4. la surface paramétrée définie en coordonnées cylindriques,

\change

5. et enfin la surface paramétrée définie en coordonnées sphériques.

\end{document}
