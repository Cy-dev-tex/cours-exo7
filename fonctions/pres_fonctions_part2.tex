
%%%%%%%%%%%%%%%%%% PREAMBULE %%%%%%%%%%%%%%%%%%

\documentclass[aspectratio=169,utf8]{beamer}
%\documentclass[aspectratio=169,handout]{beamer}

\usetheme{Boadilla}
%\usecolortheme{seahorse}
\usecolortheme[RGB={245,66,24}]{structure}
\useoutertheme{infolines}

% packages
\usepackage{amsfonts,amsmath,amssymb,amsthm}
\usepackage[utf8]{inputenc}
\usepackage[T1]{fontenc}
\usepackage{lmodern}

\usepackage[francais]{babel}
\usepackage{fancybox}
\usepackage{graphicx}

\usepackage{float}
\usepackage{xfrac}

%\usepackage[usenames, x11names]{xcolor}
\usepackage{tikz}
\usepackage{pgfplots}
\usepackage{datetime}



%-----  Package unités -----
\usepackage{siunitx}
\sisetup{locale = FR,detect-all,per-mode = symbol}

%\usepackage{mathptmx}
%\usepackage{fouriernc}
%\usepackage{newcent}
%\usepackage[mathcal,mathbf]{euler}

%\usepackage{palatino}
%\usepackage{newcent}
% \usepackage[mathcal,mathbf]{euler}



% \usepackage{hyperref}
% \hypersetup{colorlinks=true, linkcolor=blue, urlcolor=blue,
% pdftitle={Exo7 - Exercices de mathématiques}, pdfauthor={Exo7}}


%section
% \usepackage{sectsty}
% \allsectionsfont{\bf}
%\sectionfont{\color{Tomato3}\upshape\selectfont}
%\subsectionfont{\color{Tomato4}\upshape\selectfont}

%----- Ensembles : entiers, reels, complexes -----
\newcommand{\Nn}{\mathbb{N}} \newcommand{\N}{\mathbb{N}}
\newcommand{\Zz}{\mathbb{Z}} \newcommand{\Z}{\mathbb{Z}}
\newcommand{\Qq}{\mathbb{Q}} \newcommand{\Q}{\mathbb{Q}}
\newcommand{\Rr}{\mathbb{R}} \newcommand{\R}{\mathbb{R}}
\newcommand{\Cc}{\mathbb{C}} 
\newcommand{\Kk}{\mathbb{K}} \newcommand{\K}{\mathbb{K}}

%----- Modifications de symboles -----
\renewcommand{\epsilon}{\varepsilon}
\renewcommand{\Re}{\mathop{\text{Re}}\nolimits}
\renewcommand{\Im}{\mathop{\text{Im}}\nolimits}
%\newcommand{\llbracket}{\left[\kern-0.15em\left[}
%\newcommand{\rrbracket}{\right]\kern-0.15em\right]}

\renewcommand{\ge}{\geqslant}
\renewcommand{\geq}{\geqslant}
\renewcommand{\le}{\leqslant}
\renewcommand{\leq}{\leqslant}
\renewcommand{\epsilon}{\varepsilon}

%----- Fonctions usuelles -----
\newcommand{\ch}{\mathop{\text{ch}}\nolimits}
\newcommand{\sh}{\mathop{\text{sh}}\nolimits}
\renewcommand{\tanh}{\mathop{\text{th}}\nolimits}
\newcommand{\cotan}{\mathop{\text{cotan}}\nolimits}
\newcommand{\Arcsin}{\mathop{\text{arcsin}}\nolimits}
\newcommand{\Arccos}{\mathop{\text{arccos}}\nolimits}
\newcommand{\Arctan}{\mathop{\text{arctan}}\nolimits}
\newcommand{\Argsh}{\mathop{\text{argsh}}\nolimits}
\newcommand{\Argch}{\mathop{\text{argch}}\nolimits}
\newcommand{\Argth}{\mathop{\text{argth}}\nolimits}
\newcommand{\pgcd}{\mathop{\text{pgcd}}\nolimits} 


%----- Commandes divers ------
\newcommand{\ii}{\mathrm{i}}
\newcommand{\dd}{\text{d}}
\newcommand{\id}{\mathop{\text{id}}\nolimits}
\newcommand{\Ker}{\mathop{\text{Ker}}\nolimits}
\newcommand{\Card}{\mathop{\text{Card}}\nolimits}
\newcommand{\Vect}{\mathop{\text{Vect}}\nolimits}
\newcommand{\Mat}{\mathop{\text{Mat}}\nolimits}
\newcommand{\rg}{\mathop{\text{rg}}\nolimits}
\newcommand{\tr}{\mathop{\text{tr}}\nolimits}


%----- Structure des exercices ------

\newtheoremstyle{styleexo}% name
{2ex}% Space above
{3ex}% Space below
{}% Body font
{}% Indent amount 1
{\bfseries} % Theorem head font
{}% Punctuation after theorem head
{\newline}% Space after theorem head 2
{}% Theorem head spec (can be left empty, meaning ‘normal’)

%\theoremstyle{styleexo}
\newtheorem{exo}{Exercice}
\newtheorem{ind}{Indications}
\newtheorem{cor}{Correction}


\newcommand{\exercice}[1]{} \newcommand{\finexercice}{}
%\newcommand{\exercice}[1]{{\tiny\texttt{#1}}\vspace{-2ex}} % pour afficher le numero absolu, l'auteur...
\newcommand{\enonce}{\begin{exo}} \newcommand{\finenonce}{\end{exo}}
\newcommand{\indication}{\begin{ind}} \newcommand{\finindication}{\end{ind}}
\newcommand{\correction}{\begin{cor}} \newcommand{\fincorrection}{\end{cor}}

\newcommand{\noindication}{\stepcounter{ind}}
\newcommand{\nocorrection}{\stepcounter{cor}}

\newcommand{\fiche}[1]{} \newcommand{\finfiche}{}
\newcommand{\titre}[1]{\centerline{\large \bf #1}}
\newcommand{\addcommand}[1]{}
\newcommand{\video}[1]{}

% Marge
\newcommand{\mymargin}[1]{\marginpar{{\small #1}}}

\def\noqed{\renewcommand{\qedsymbol}{}}


%----- Presentation ------
\setlength{\parindent}{0cm}

%\newcommand{\ExoSept}{\href{http://exo7.emath.fr}{\textbf{\textsf{Exo7}}}}

\definecolor{myred}{rgb}{0.93,0.26,0}
\definecolor{myorange}{rgb}{0.97,0.58,0}
\definecolor{myyellow}{rgb}{1,0.86,0}

\newcommand{\LogoExoSept}[1]{  % input : echelle
{\usefont{U}{cmss}{bx}{n}
\begin{tikzpicture}[scale=0.1*#1,transform shape]
  \fill[color=myorange] (0,0)--(4,0)--(4,-4)--(0,-4)--cycle;
  \fill[color=myred] (0,0)--(0,3)--(-3,3)--(-3,0)--cycle;
  \fill[color=myyellow] (4,0)--(7,4)--(3,7)--(0,3)--cycle;
  \node[scale=5] at (3.5,3.5) {Exo7};
\end{tikzpicture}}
}


\newcommand{\debutmontitre}{
  \author{} \date{} 
  \thispagestyle{empty}
  \hspace*{-10ex}
  \begin{minipage}{\textwidth}
    \titlepage  
  \vspace*{-2.5cm}
  \begin{center}
    \LogoExoSept{2.5}
  \end{center}
  \end{minipage}

  \vspace*{-0cm}
  
  % Astuce pour que le background ne soit pas discrétisé lors de la conversion pdf -> png
\begin{tikzpicture}
        \fill[opacity=0,green!60!black] (0,0)--++(0,0)--++(0,0)--++(0,0)--cycle; 
\end{tikzpicture}

% toc S'affiche trop tot :
% \tableofcontents[hideallsubsections, pausesections]
}

\newcommand{\finmontitre}{
  \end{frame}
  \setcounter{framenumber}{0}
} % ne marche pas pour une raison obscure

%----- Commandes supplementaires ------

% \usepackage[landscape]{geometry}
% \geometry{top=1cm, bottom=3cm, left=2cm, right=10cm, marginparsep=1cm
% }
% \usepackage[a4paper]{geometry}
% \geometry{top=2cm, bottom=2cm, left=2cm, right=2cm, marginparsep=1cm
% }

%\usepackage{standalone}


% New command Arnaud -- november 2011
\setbeamersize{text margin left=24ex}
% si vous modifier cette valeur il faut aussi
% modifier le decalage du titre pour compenser
% (ex : ici =+10ex, titre =-5ex

\theoremstyle{definition}
%\newtheorem{proposition}{Proposition}
%\newtheorem{exemple}{Exemple}
%\newtheorem{theoreme}{Théorème}
%\newtheorem{lemme}{Lemme}
%\newtheorem{corollaire}{Corollaire}
%\newtheorem*{remarque*}{Remarque}
%\newtheorem*{miniexercice}{Mini-exercices}
%\newtheorem{definition}{Définition}

% Commande tikz
\usetikzlibrary{calc}
\usetikzlibrary{patterns,arrows}
\usetikzlibrary{matrix}
\usetikzlibrary{fadings} 

%definition d'un terme
\newcommand{\defi}[1]{{\color{myorange}\textbf{\emph{#1}}}}
\newcommand{\evidence}[1]{{\color{blue}\textbf{\emph{#1}}}}
\newcommand{\assertion}[1]{\emph{\og#1\fg}}  % pour chapitre logique
%\renewcommand{\contentsname}{Sommaire}
\renewcommand{\contentsname}{}
\setcounter{tocdepth}{2}



%------ Figures ------

\def\myscale{1} % par défaut 
\newcommand{\myfigure}[2]{  % entrée : echelle, fichier figure
\def\myscale{#1}
\begin{center}
\footnotesize
{#2}
\end{center}}


%------ Encadrement ------

\usepackage{fancybox}


\newcommand{\mybox}[1]{
\setlength{\fboxsep}{7pt}
\begin{center}
\shadowbox{#1}
\end{center}}

\newcommand{\myboxinline}[1]{
\setlength{\fboxsep}{5pt}
\raisebox{-10pt}{
\shadowbox{#1}
}
}

%--------------- Commande beamer---------------
\newcommand{\beameronly}[1]{#1} % permet de mettre des pause dans beamer pas dans poly


\setbeamertemplate{navigation symbols}{}
\setbeamertemplate{footline}  % tiré du fichier beamerouterinfolines.sty
{
  \leavevmode%
  \hbox{%
  \begin{beamercolorbox}[wd=.333333\paperwidth,ht=2.25ex,dp=1ex,center]{author in head/foot}%
    % \usebeamerfont{author in head/foot}\insertshortauthor%~~(\insertshortinstitute)
    \usebeamerfont{section in head/foot}{\bf\insertshorttitle}
  \end{beamercolorbox}%
  \begin{beamercolorbox}[wd=.333333\paperwidth,ht=2.25ex,dp=1ex,center]{title in head/foot}%
    \usebeamerfont{section in head/foot}{\bf\insertsectionhead}
  \end{beamercolorbox}%
  \begin{beamercolorbox}[wd=.333333\paperwidth,ht=2.25ex,dp=1ex,right]{date in head/foot}%
    % \usebeamerfont{date in head/foot}\insertshortdate{}\hspace*{2em}
    \insertframenumber{} / \inserttotalframenumber\hspace*{2ex} 
  \end{beamercolorbox}}%
  \vskip0pt%
}


\definecolor{mygrey}{rgb}{0.5,0.5,0.5}
\setlength{\parindent}{0cm}
%\DeclareTextFontCommand{\helvetica}{\fontfamily{phv}\selectfont}

% background beamer
\definecolor{couleurhaut}{rgb}{0.85,0.9,1}  % creme
\definecolor{couleurmilieu}{rgb}{1,1,1}  % vert pale
\definecolor{couleurbas}{rgb}{0.85,0.9,1}  % blanc
\setbeamertemplate{background canvas}[vertical shading]%
[top=couleurhaut,middle=couleurmilieu,midpoint=0.4,bottom=couleurbas] 
%[top=fondtitre!05,bottom=fondtitre!60]



\makeatletter
\setbeamertemplate{theorem begin}
{%
  \begin{\inserttheoremblockenv}
  {%
    \inserttheoremheadfont
    \inserttheoremname
    \inserttheoremnumber
    \ifx\inserttheoremaddition\@empty\else\ (\inserttheoremaddition)\fi%
    \inserttheorempunctuation
  }%
}
\setbeamertemplate{theorem end}{\end{\inserttheoremblockenv}}

\newenvironment{theoreme}[1][]{%
   \setbeamercolor{block title}{fg=structure,bg=structure!40}
   \setbeamercolor{block body}{fg=black,bg=structure!10}
   \begin{block}{{\bf Th\'eor\`eme }#1}
}{%
   \end{block}%
}


\newenvironment{proposition}[1][]{%
   \setbeamercolor{block title}{fg=structure,bg=structure!40}
   \setbeamercolor{block body}{fg=black,bg=structure!10}
   \begin{block}{{\bf Proposition }#1}
}{%
   \end{block}%
}

\newenvironment{corollaire}[1][]{%
   \setbeamercolor{block title}{fg=structure,bg=structure!40}
   \setbeamercolor{block body}{fg=black,bg=structure!10}
   \begin{block}{{\bf Corollaire }#1}
}{%
   \end{block}%
}

\newenvironment{mydefinition}[1][]{%
   \setbeamercolor{block title}{fg=structure,bg=structure!40}
   \setbeamercolor{block body}{fg=black,bg=structure!10}
   \begin{block}{{\bf Définition} #1}
}{%
   \end{block}%
}

\newenvironment{lemme}[0]{%
   \setbeamercolor{block title}{fg=structure,bg=structure!40}
   \setbeamercolor{block body}{fg=black,bg=structure!10}
   \begin{block}{\bf Lemme}
}{%
   \end{block}%
}

\newenvironment{remarque}[1][]{%
   \setbeamercolor{block title}{fg=black,bg=structure!20}
   \setbeamercolor{block body}{fg=black,bg=structure!5}
   \begin{block}{Remarque #1}
}{%
   \end{block}%
}


\newenvironment{exemple}[1][]{%
   \setbeamercolor{block title}{fg=black,bg=structure!20}
   \setbeamercolor{block body}{fg=black,bg=structure!5}
   \begin{block}{{\bf Exemple }#1}
}{%
   \end{block}%
}


\newenvironment{miniexercice}[0]{%
   \setbeamercolor{block title}{fg=structure,bg=structure!20}
   \setbeamercolor{block body}{fg=black,bg=structure!5}
   \begin{block}{Mini-exercices}
}{%
   \end{block}%
}


\newenvironment{tp}[0]{%
   \setbeamercolor{block title}{fg=structure,bg=structure!40}
   \setbeamercolor{block body}{fg=black,bg=structure!10}
   \begin{block}{\bf Travaux pratiques}
}{%
   \end{block}%
}
\newenvironment{exercicecours}[1][]{%
   \setbeamercolor{block title}{fg=structure,bg=structure!40}
   \setbeamercolor{block body}{fg=black,bg=structure!10}
   \begin{block}{{\bf Exercice }#1}
}{%
   \end{block}%
}
\newenvironment{algo}[1][]{%
   \setbeamercolor{block title}{fg=structure,bg=structure!40}
   \setbeamercolor{block body}{fg=black,bg=structure!10}
   \begin{block}{{\bf Algorithme}\hfill{\color{gray}\texttt{#1}}}
}{%
   \end{block}%
}


\setbeamertemplate{proof begin}{
   \setbeamercolor{block title}{fg=black,bg=structure!20}
   \setbeamercolor{block body}{fg=black,bg=structure!5}
   \begin{block}{{\footnotesize Démonstration}}
   \footnotesize
   \smallskip}
\setbeamertemplate{proof end}{%
   \end{block}}
\setbeamertemplate{qed symbol}{\openbox}


\makeatother
\usecolortheme[RGB={66,15,15}]{structure}

%%%%%%%%%%%%%%%%%%%%%%%%%%%%%%%%%%%%%%%%%%%%%%%%%%%%%%%%%%%%%
%%%%%%%%%%%%%%%%%%%%%%%%%%%%%%%%%%%%%%%%%%%%%%%%%%%%%%%%%%%%%


\begin{document}


\title{{\bf Limites et fonctions continues}}
\subtitle{Limites}

\begin{frame}
  
  \debutmontitre

  \pause

{\footnotesize
\hfill
\setbeamercovered{transparent=50}
\begin{minipage}{0.6\textwidth}
  \begin{itemize}
    \item<3-> Définitions
    \item<4-> Propriétés
   \end{itemize}
\end{minipage}
}

\end{frame}

\setcounter{framenumber}{0}


%%%%%%%%%%%%%%%%%%%%%%%%%%%%%%%%%%%%%%%%%%%%%%%%%%%%%%%%%%%%%%%%

%
%\section{Motivation}
%
%
%\begin{frame}
%
%\end{frame}


%---------------------------------------------------------------
\section{Limite en un point} 

\begin{frame}

Soit $I$ un intervalle et $x_0$ un élément ou une extrémité de $I$

\pause

\begin{mydefinition}
Soit $\ell\in\Rr$. On dit que $f:I\to\Rr$ a pour \defi{limite $\ell$ en $x_0$} si 
\mybox{$
\forall \epsilon>0 \quad \exists \delta>0 \quad \forall x\in I \quad \vert x-x_0\vert <\delta 
\implies \vert f(x)-\ell\vert <\epsilon
$}
\end{mydefinition}

\pause
\myfigure{.88}{
\tikzinput{fig_fonctions4}
}

\end{frame}


\begin{frame}

$f$ a pour limite $\ell$ en $x_0$ si 
\mybox{$
\forall \epsilon>0 \quad \exists \delta>0 \quad \forall x\in I \quad \vert x-x_0\vert <\delta 
\implies \vert f(x)-\ell\vert <\epsilon
$}

\pause

\begin{remarque}
\begin{itemize}
  \item On note \quad $\displaystyle\lim_{x\to x_0}f(x)=\ell$  \quad ou  \quad $\displaystyle\lim_{x_0} f=\ell$  
\pause

  \item $\vert x-x_0\vert <\delta \  \iff  x \in \, ]x_0 - \delta, x_0+\delta[$ 
  \pause
  
 $\vert f(x)-\ell\vert <\epsilon \iff   f(x) \in \, ]\ell - \epsilon, \ell+\epsilon[$
\pause
 
  \item $\forall \epsilon>0 \quad \exists \delta>0 \quad \forall x\in I \quad \vert x-x_0\vert \le \delta 
\implies \vert f(x)-\ell\vert \le \epsilon$
\pause

  \item Attention à l'ordre des quantificateurs ! 
  
  \pause $\forall \epsilon>0 \quad \exists   \delta{\color{red}(\epsilon)} >0 \quad \ldots$

  \end{itemize}
\end{remarque}
 
\end{frame}



\begin{frame}

\begin{exemple}
\begin{itemize}
\item $\displaystyle\lim_{x\to x_0} \sqrt{x} = \sqrt{x_0}$ pour tout $x_0\geq0$
\pause

\item \onslide<3->la fonction partie entière $E$ n'a pas de limite aux points $x_0\in\Zz$
\end{itemize}
\end{exemple}

\pause\pause
\myfigure{0.7}{
\onslide<2->\tikzinput{fig_fonctionsA08} \quad
\onslide<4>\tikzinput{fig_fonctionsA09}
}

  
\end{frame}



\begin{frame}

\begin{mydefinition}
\defi{$f$ a pour limite $+\infty$ en $x_0$}, noté  $\displaystyle\lim_{x\to x_0}f(x)=+\infty$, si 
\[
\forall A>0 \quad \exists \delta>0 \quad \forall x\in I \quad \vert x-x_0\vert <\delta \implies f(x)>A
\]

\end{mydefinition}

\pause

 \onslide<3> \defi{$f$ a pour limite $-\infty$ en $x_0$} si 
\[
\forall A>0 \quad \exists \delta>0 \quad \forall x\in I \quad \vert x-x_0\vert <\delta \implies f(x)<-A
\]


\pause
\onslide<2-3>\myfigure{0.65}{
\tikzinput{fig_fonctionsA10}
}
  
\end{frame}



%---------------------------------------------------------------
\section{Limite en l'infini}

\begin{frame}

Soit $f:I\to \Rr$,   $I=]a,+\infty[$

\pause

\begin{mydefinition}
Soit $\ell\in\Rr$. \defi{$f$ a pour limite $\ell$ en $+\infty$} si 
\[
\forall \epsilon>0 \quad \exists B>0 \quad \forall x\in I \quad x>B \implies \vert f(x)-\ell\vert <\epsilon
\]
\end{mydefinition}

\myfigure{1}{
\tikzinput{fig_fonctions5}
}

\end{frame}



\begin{frame}

\begin{mydefinition}
On dit que \defi{$f$ a pour limite $+\infty$ en $+\infty$} si 
\[
\forall A>0 \quad \exists B>0 \quad \forall x\in I \quad x>B \implies  f(x)>A
\]
\end{mydefinition}


\bigskip
\pause

\begin{exemple}
\begin{itemize}
\item $\displaystyle\lim_{x\to +\infty} x^n = +\infty$ \quad \pause et \quad 
$\displaystyle\lim_{x\to -\infty} x^n = 
\begin{cases}
+\infty \text{ si $n$ est pair}\\
-\infty \text{ si $n$ est impair}
\end{cases}$

\pause
\bigskip

\item $\displaystyle\lim_{x\to +\infty} \left(\frac{1}{x^n}\right) = 0$ \quad et 
\quad $\displaystyle\lim_{x\to -\infty} \left(\frac{1}{x^n}\right) =0$
\end{itemize}
\end{exemple}

\end{frame}

\begin{frame}

\begin{exemple}
$P(x)=a_nx^n+a_{n-1}x^{n-1}+\cdots+a_1x+a_0$ 
$Q(x)=b_mx^m+b_{m-1}x^{m-1}+\cdots+b_1x+b_0$  \ avec $a_n, \ b_m>0$
\pause

\[\lim_{x\to+\infty} \frac{P(x)}{Q(x)} = \pause
\begin{cases}
  +\infty & \text{ si } n > m \\
\pause
  \frac{a_n}{b_m} & \text{ si } n = m \\
\pause
  0  & \text{ si } n < m \\
\end{cases}\]
\end{exemple}
 
\pause 
 
 \medskip
 
$$\frac{P(x)}{Q(x)} = \dfrac{x^n\left(a_n+\frac{a_{n-1}}{x}+\cdots\right)}{x^m\left(b_m+\frac{b_{m-1}}{x}+\cdots\right)}$$

 
\end{frame}


%---------------------------------------------------------------
\section{Propriétés}

\begin{frame}
\begin{proposition}
Si une fonction admet une limite, alors cette limite est unique
\end{proposition}

\end{frame}


\begin{frame}

Soient deux fonctions $f$ et $g$ et $x_0\in\Rr$ ou $x_0=\pm\infty$
 
\pause 

\begin{proposition}
Si $\displaystyle\lim_{x_0} f=\ell\in\Rr$ et $\displaystyle\lim_{x_0} g=\ell'\in\Rr$,  alors 
\pause
\begin{itemize}
  \item $\displaystyle\lim_{x_0} (\lambda\cdot f)=\lambda\cdot \ell$ \quad pour tout $\lambda\in\Rr$ 
   
\pause 
  \item $\displaystyle\lim_{x_0} (f+g) = \ell+\ell'$ 
  
\pause   
  \item $\displaystyle\lim_{x_0} (f\times g) = \ell\times \ell'$ 
   
\pause 
  \item si $\ell\neq 0$, alors $\displaystyle\lim_{x_0} \frac1f = \frac1\ell$
\end{itemize}
 
\pause 
De plus, si $\displaystyle\lim_{x_0} f=+\infty$ (ou $-\infty$) alors $\displaystyle\lim_{x_0} \frac1f = 0$
\end{proposition}

\end{frame}



\begin{frame}


\begin{proposition}
Si \quad $\displaystyle\lim_{x_0} f=\ell$  \quad et  \quad $\displaystyle\lim_\ell g=\ell'$  \quad alors  \quad 
$\displaystyle\lim_{x_0} g\circ f=\ell'$
\end{proposition}
 
\pause 

\begin{exemple}
Soit $u$ une fonction telle que $u(x) \to 2$  lorsque $x \to x_0 \in \Rr$ \pause

Posons $f(x) = \sqrt{1+\frac{1}{u(x)^2}+\ln u(x)}$  \pause

Si elle existe, quelle est la limite de $f$ en $x_0$ ?

\pause
\begin{itemize}
  \item lorsque $x \to x_0$ : $u(x) \to 2 \implies u(x)^2 \to 4 \pause \implies  \frac{1}{u(x)^2} \to \frac14$

\pause  
  \item comme $u(x) \to 2>0$ alors $u(x)>0$ dans un voisinage de $x_0$, 
  \pause donc $\ln u(x)$ est bien définie dans ce voisinage et $\ln u(x) \to \ln 2$
 
\pause 
  \item Donc  $1+\frac{1}{u(x)^2}+\ln u(x) \to 1+\frac 14 + \ln 2>0$
  
  \pause donc $f(x)$ est bien définie dans un voisinage de $x_0$
  
\pause
  \item Par composition
  $\lim_{x\to x_0} f(x) = \sqrt{1+\frac14 + \ln 2}$
\end{itemize}
\end{exemple}
  
\end{frame}


\begin{frame}

 \evidence{Formes indéterminées}
\[
 \text{si } \quad f \to +\infty \quad\text{ et }\quad g \to -\infty \quad \text{ alors } f+g \to \ \  \color{myred}{?}
 \] 

 \pause 
 
 \bigskip
 Formes indéterminées : étude au cas par cas
 
 $$+\infty-\infty  \qquad \pause
 0\times \infty \qquad
 \dfrac\infty\infty
  \qquad \dfrac00   \qquad 
1^\infty  \qquad   \infty^0$$
\end{frame}



\begin{frame}

\begin{proposition}
\begin{itemize}
  \item Si $f\leq g$ et si $\displaystyle\lim_{x_0} f=\ell\in\Rr$ et $\displaystyle\lim_{x_0} g=\ell'\in\Rr$, alors $\ell\leq \ell'$
  
\pause
  \item Si $f\leq g$ et si $\displaystyle\lim_{x_0} f=+\infty$, alors $\displaystyle\lim_{x_0} g=+\infty$
  
\pause
  \item Théorème des gendarmes
  \mybox{Si \ $f\leq g\leq h$ \ et si \ 
  $\displaystyle\lim_{x_0} f=\displaystyle\lim_{x_0} h=\ell\in\Rr$ \  alors \  
  $\displaystyle\lim_{x_0} g=\ell$}
\end{itemize}
\end{proposition}

\pause

\myfigure{0.95}{
\tikzinput{fig_fonctionsA12}
}
  
\end{frame}



%---------------------------------------------------------------
\section{Une preuve}

\begin{frame}

\vspace*{1ex}
\centerline{Si \quad $\displaystyle\lim_{x_0} f=\ell > 0$ \quad  alors \quad $\displaystyle\lim_{x_0} \frac1f = \frac1\ell$}
 
\pause 

\begin{proof} 

\begin{enumerate}
\item Montrons que $\frac 1 f$ est bien définie et bornée autour de $x_0$
  
\pause 

  \begin{itemize}
    \item Par hypothèse 
$\forall \epsilon'>0 \quad \exists \delta>0  \quad x\in ]x_0-\delta,x_0+\delta [
\implies \ell-\epsilon' < f(x) <\ell+\epsilon'$

\pause  

    \item Pour $\epsilon'<\ell/2$, \pause l'intervalle $J = ]x_0-\delta,x_0+\delta [$ \pause est tel que 
    $f(x) > \frac{\ell}{2}>0$ \pause et donc 
$\forall x\in J \quad 0< \frac{1}{f(x)} < M \ (=\frac{2}{\ell})$
  \end{itemize}


\pause 

\item Montrons que $\displaystyle\lim_{x_0} \frac1f = \frac1\ell$ \pause 


  \begin{itemize}
    \item 
$\displaystyle \forall x\in J \quad  \left\vert \frac{1}{f(x)} - \frac1\ell  \right\vert 
= \frac{\left\vert \ell - f(x) \right\vert }{f(x)\ell} \pause < \frac{M}{\ell}\left\vert \ell - f(x) \right\vert  $
 
\pause 
\smallskip

    \item Soit $\epsilon>0$. \pause On choisit $\epsilon'=\frac{\ell \epsilon}{M}$\,, 
    \pause alors il existe $\delta>0$ tel que
    \vspace*{-1ex}
\[x\in ]x_0-\delta,x_0+\delta [
\implies \left\vert \frac{1}{f(x)} - \frac1\ell  \right\vert \pause < \frac{M}{\ell}\left\vert \ell - f(x) \right\vert 
\pause <  \frac{M}{\ell}\epsilon' \pause = \epsilon 
\]
\vspace*{-5ex}
  \end{itemize}
\end{enumerate}
\end{proof}
 
\end{frame}


%%%%%%%%%%%%%%%%%%%%%%%%%%%%%%%%%%%%%%%%%%%%%%%%%%%%%%%%%%%%%%%%
\section{Mini-exercices}

\begin{frame}

\begin{miniexercice}
\begin{enumerate}
  \item Déterminer, si elle existe, la limite de $\frac{2x^2-x-2}{3x^2+2x+2}$ en $0$. 
  Et en $+\infty$ ?
  
  \item Déterminer, si elles existent, les limites de $\sin\left(\frac1x\right)$ et  $\frac{\cos x}{\sqrt{x}}$ en $+\infty$. 
  
  \item En utilisant la définition de la limite (avec des $\epsilon$), 
  montrer que $\lim_{x\to2} (3x+1) = 7$.
  
  \item Montrer que si $f$ admet une limite finie en $x_0$ alors il existe $\delta>0$ tel que
  $f$ soit bornée sur $]x_0-\delta,x_0+\delta[$.
  
  \item Déterminer, si elle existe,   $\lim_{x\to0} \frac{\sqrt{1+x}-\sqrt{1+x^2}}{x}$. 
  
  Et $\lim_{x\to2} \frac{x^2-4}{x^2-3x+2}$ ?
\end{enumerate}
\end{miniexercice}

\end{frame}

\end{document}