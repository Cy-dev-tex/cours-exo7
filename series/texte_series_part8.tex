
%%%%%%%%%%%%%%%%%% PREAMBULE %%%%%%%%%%%%%%%%%%


\documentclass[12pt]{article}

\usepackage{amsfonts,amsmath,amssymb,amsthm}
\usepackage[utf8]{inputenc}
\usepackage[T1]{fontenc}
\usepackage[francais]{babel}


% packages
\usepackage{amsfonts,amsmath,amssymb,amsthm}
\usepackage[utf8]{inputenc}
\usepackage[T1]{fontenc}
%\usepackage{lmodern}

\usepackage[francais]{babel}
\usepackage{fancybox}
\usepackage{graphicx}

\usepackage{float}

%\usepackage[usenames, x11names]{xcolor}
\usepackage{tikz}
\usepackage{datetime}

\usepackage{mathptmx}
%\usepackage{fouriernc}
%\usepackage{newcent}
\usepackage[mathcal,mathbf]{euler}

%\usepackage{palatino}
%\usepackage{newcent}


% Commande spéciale prompteur

%\usepackage{mathptmx}
%\usepackage[mathcal,mathbf]{euler}
%\usepackage{mathpple,multido}

\usepackage[a4paper]{geometry}
\geometry{top=2cm, bottom=2cm, left=1cm, right=1cm, marginparsep=1cm}

\newcommand{\change}{{\color{red}\rule{\textwidth}{1mm}\\}}

\newcounter{mydiapo}

\newcommand{\diapo}{\newpage
\hfill {\normalsize  Diapo \themydiapo \quad \texttt{[\jobname]}} \\
\stepcounter{mydiapo}}


%%%%%%% COULEURS %%%%%%%%%%

% Pour blanc sur noir :
%\pagecolor[rgb]{0.5,0.5,0.5}
% \pagecolor[rgb]{0,0,0}
% \color[rgb]{1,1,1}



%\DeclareFixedFont{\myfont}{U}{cmss}{bx}{n}{18pt}
\newcommand{\debuttexte}{
%%%%%%%%%%%%% FONTES %%%%%%%%%%%%%
\renewcommand{\baselinestretch}{1.5}
\usefont{U}{cmss}{bx}{n}
\bfseries

% Taille normale : commenter le reste !
%Taille Arnaud
%\fontsize{19}{19}\selectfont

% Taille Barbara
%\fontsize{21}{22}\selectfont

%Taille François
%\fontsize{25}{30}\selectfont

%Taille Pascal
%\fontsize{25}{30}\selectfont

%Taille Laura
%\fontsize{30}{35}\selectfont


%\myfont
%\usefont{U}{cmss}{bx}{n}

%\Huge
%\addtolength{\parskip}{\baselineskip}
}


% \usepackage{hyperref}
% \hypersetup{colorlinks=true, linkcolor=blue, urlcolor=blue,
% pdftitle={Exo7 - Exercices de mathématiques}, pdfauthor={Exo7}}


%section
% \usepackage{sectsty}
% \allsectionsfont{\bf}
%\sectionfont{\color{Tomato3}\upshape\selectfont}
%\subsectionfont{\color{Tomato4}\upshape\selectfont}

%----- Ensembles : entiers, reels, complexes -----
\newcommand{\Nn}{\mathbb{N}} \newcommand{\N}{\mathbb{N}}
\newcommand{\Zz}{\mathbb{Z}} \newcommand{\Z}{\mathbb{Z}}
\newcommand{\Qq}{\mathbb{Q}} \newcommand{\Q}{\mathbb{Q}}
\newcommand{\Rr}{\mathbb{R}} \newcommand{\R}{\mathbb{R}}
\newcommand{\Cc}{\mathbb{C}} 
\newcommand{\Kk}{\mathbb{K}} \newcommand{\K}{\mathbb{K}}

%----- Modifications de symboles -----
\renewcommand{\epsilon}{\varepsilon}
\renewcommand{\Re}{\mathop{\text{Re}}\nolimits}
\renewcommand{\Im}{\mathop{\text{Im}}\nolimits}
%\newcommand{\llbracket}{\left[\kern-0.15em\left[}
%\newcommand{\rrbracket}{\right]\kern-0.15em\right]}

\renewcommand{\ge}{\geqslant}
\renewcommand{\geq}{\geqslant}
\renewcommand{\le}{\leqslant}
\renewcommand{\leq}{\leqslant}

%----- Fonctions usuelles -----
\newcommand{\ch}{\mathop{\mathrm{ch}}\nolimits}
\newcommand{\sh}{\mathop{\mathrm{sh}}\nolimits}
\renewcommand{\tanh}{\mathop{\mathrm{th}}\nolimits}
\newcommand{\cotan}{\mathop{\mathrm{cotan}}\nolimits}
\newcommand{\Arcsin}{\mathop{\mathrm{Arcsin}}\nolimits}
\newcommand{\Arccos}{\mathop{\mathrm{Arccos}}\nolimits}
\newcommand{\Arctan}{\mathop{\mathrm{Arctan}}\nolimits}
\newcommand{\Argsh}{\mathop{\mathrm{Argsh}}\nolimits}
\newcommand{\Argch}{\mathop{\mathrm{Argch}}\nolimits}
\newcommand{\Argth}{\mathop{\mathrm{Argth}}\nolimits}
\newcommand{\pgcd}{\mathop{\mathrm{pgcd}}\nolimits} 

\newcommand{\Card}{\mathop{\text{Card}}\nolimits}
\newcommand{\Ker}{\mathop{\text{Ker}}\nolimits}
\newcommand{\id}{\mathop{\text{id}}\nolimits}
\newcommand{\ii}{\mathrm{i}}
\newcommand{\dd}{\mathrm{d}}
\newcommand{\Vect}{\mathop{\text{Vect}}\nolimits}
\newcommand{\Mat}{\mathop{\mathrm{Mat}}\nolimits}
\newcommand{\rg}{\mathop{\text{rg}}\nolimits}
\newcommand{\tr}{\mathop{\text{tr}}\nolimits}
\newcommand{\ppcm}{\mathop{\text{ppcm}}\nolimits}

%----- Structure des exercices ------

\newtheoremstyle{styleexo}% name
{2ex}% Space above
{3ex}% Space below
{}% Body font
{}% Indent amount 1
{\bfseries} % Theorem head font
{}% Punctuation after theorem head
{\newline}% Space after theorem head 2
{}% Theorem head spec (can be left empty, meaning ‘normal’)

%\theoremstyle{styleexo}
\newtheorem{exo}{Exercice}
\newtheorem{ind}{Indications}
\newtheorem{cor}{Correction}


\newcommand{\exercice}[1]{} \newcommand{\finexercice}{}
%\newcommand{\exercice}[1]{{\tiny\texttt{#1}}\vspace{-2ex}} % pour afficher le numero absolu, l'auteur...
\newcommand{\enonce}{\begin{exo}} \newcommand{\finenonce}{\end{exo}}
\newcommand{\indication}{\begin{ind}} \newcommand{\finindication}{\end{ind}}
\newcommand{\correction}{\begin{cor}} \newcommand{\fincorrection}{\end{cor}}

\newcommand{\noindication}{\stepcounter{ind}}
\newcommand{\nocorrection}{\stepcounter{cor}}

\newcommand{\fiche}[1]{} \newcommand{\finfiche}{}
\newcommand{\titre}[1]{\centerline{\large \bf #1}}
\newcommand{\addcommand}[1]{}
\newcommand{\video}[1]{}

% Marge
\newcommand{\mymargin}[1]{\marginpar{{\small #1}}}



%----- Presentation ------
\setlength{\parindent}{0cm}

%\newcommand{\ExoSept}{\href{http://exo7.emath.fr}{\textbf{\textsf{Exo7}}}}

\definecolor{myred}{rgb}{0.93,0.26,0}
\definecolor{myorange}{rgb}{0.97,0.58,0}
\definecolor{myyellow}{rgb}{1,0.86,0}

\newcommand{\LogoExoSept}[1]{  % input : echelle
{\usefont{U}{cmss}{bx}{n}
\begin{tikzpicture}[scale=0.1*#1,transform shape]
  \fill[color=myorange] (0,0)--(4,0)--(4,-4)--(0,-4)--cycle;
  \fill[color=myred] (0,0)--(0,3)--(-3,3)--(-3,0)--cycle;
  \fill[color=myyellow] (4,0)--(7,4)--(3,7)--(0,3)--cycle;
  \node[scale=5] at (3.5,3.5) {Exo7};
\end{tikzpicture}}
}



\theoremstyle{definition}
%\newtheorem{proposition}{Proposition}
%\newtheorem{exemple}{Exemple}
%\newtheorem{theoreme}{Théorème}
\newtheorem{lemme}{Lemme}
\newtheorem{corollaire}{Corollaire}
%\newtheorem*{remarque*}{Remarque}
%\newtheorem*{miniexercice}{Mini-exercices}
%\newtheorem{definition}{Définition}




%definition d'un terme
\newcommand{\defi}[1]{{\color{myorange}\textbf{\emph{#1}}}}
\newcommand{\evidence}[1]{{\color{blue}\textbf{\emph{#1}}}}



 %----- Commandes divers ------

\newcommand{\codeinline}[1]{\texttt{#1}}

%%%%%%%%%%%%%%%%%%%%%%%%%%%%%%%%%%%%%%%%%%%%%%%%%%%%%%%%%%%%%
%%%%%%%%%%%%%%%%%%%%%%%%%%%%%%%%%%%%%%%%%%%%%%%%%%%%%%%%%%%%%


\begin{document}

\debuttexte


%%%%%%%%%%%%%%%%%%%%%%%%%%%%%%%%%%%%%%%%%%%%%%%%%%%%%%%%%%%
\diapo

Nous terminons le chapitre sur les séries par une leçon consacrée à la sommation d'Abel. 
Cette leçon peut être passée lors d'une première approche sur les séries.

\change
\change
Nous commencerons par énoncer et démontrer le théorème de sommation d'Abel,

\change
puis nous l'appliquerons au cas des séries de Fourier.

%%%%%%%%%%%%%%%%%%%%%%%%%%%%%%%%%%%%%%%%%%%%%%%%%%%%%%%%%%%
\diapo

Le théorème de sommation d'Abel s'applique à certaines séries
convergentes mais qui ne sont pas absolument convergentes.
C'est un théorème qui s'applique aux séries de la forme $\sum a_kb_k$
et qui est plus fort que le critère de Leibniz pour les séries alternées. Mais il est aussi plus difficile à mettre en \oe uvre.

\change
Théorème.

Soient $(a_k)_{k\ge0}$ et $(b_k)_{k\ge0}$ deux suites telles que :

\change
1) La suite $(a_k)_{k\ge0}$ est une suite décroissante de
réels positifs qui tend vers $0$.

\change
2) Les sommes partielles de la série $\sum b_k$ sont bornées :
$$\exists M\quad\forall n\in \Nn\qquad
\big|b_0+\cdots+b_n\big|\le M.$$

\change
Alors la série $\sum_{k\ge0} a_kb_k$ converge.


%%%%%%%%%%%%%%%%%%%%%%%%%%%%%%%%%%%%%%%%%%%%%%%%%%%%%%%%%%%
\diapo
Le critère de Leibniz concernant les séries alternées est un cas particulier de ce théorème :

\change
en effet, si $(a_k)$ est une suite positive, décroissante, qui tend vers $0$, 

\change
et si $b_k$ est définie par $b_k=(-1)^k$ alors $\big|\sum_{k=0}^n b_k\big|\le 1$.


\change
alors par le théorème de sommation d'Abel on retrouve que la série $\sum a_k b_k$ converge.

%%%%%%%%%%%%%%%%%%%%%%%%%%%%%%%%%%%%%%%%%%%%%%%%%%%%%%%%%%%
\diapo

[grand B]

Démontrons le théorème de sommation d'Abel.

L'idée de la démonstration est d'effectuer un changement dans la
sommation, qui s'apparente à une intégration par parties. 

\change
Pour tout $n\ge0$, posons $B_n=b_0+\cdots+b_n$. Par hypothèse, la suite
$(B_n)$ est bornée. 

\change
Nous écrivons les sommes partielles de la série $\sum a_k b_k$ sous la forme suivante.

Par définition
$$
S_n = a_0b_0+a_1b_1+\cdots+a_{n-1}b_{n-1}+a_nb_n
$$

\change
ce qui est égal à 
$$
 a_0B_0+a_1(B_1-B_0)+\cdots +a_{n-1}(B_{n-1}-B_{n-2}) + a_n(B_n-B_{n-1})\\
$$

\change
Regroupons ainsi les termes :
$$
B_0(a_0-a_1)+B_1(a_1-a_2)+\cdots+B_{n-1}(a_{n-1}-a_n)+B_na_n .
$$

\change
Comme $(B_n)$ est bornée, et $a_n$ tend vers $0$, le dernier terme $B_na_n$
tend vers $0$. 

\change
Intéressons nous à présent aux premiers termes. Nous allons montrer que la série $\sum B_k(a_k-a_{k+1})$ est absolument convergente. 

\change
En effet, comme  la suite $(a_k)$ est une suite de réels positifs,
décroissante, et $|B_k|$ est borné par $M$, 

\change
on peut écrire
$$
\big|B_k(a_k-a_{k+1})\big| = \big|B_k\big|(a_k-a_{k+1})\le M(a_k-a_{k+1})
$$

\change
Or
$$
M(a_0-a_1)+\cdots+M(a_n-a_{n+1}) = M(a_0-a_{n+1})
$$

\change
qui tend vers $Ma_0$ puisque $(a_k)$ tend vers $0$.

\change
La série $\sum M(a_k-a_{k+1})$ converge, donc la série
$\sum\big|B_k(a_k-a_{k+1})\big|$ aussi, par le théorème de comparaison. 

\change
La série $\sum B_k(a_k-a_{k+1})$ converge, donc la suite $(S_n)$ est convergente, ce qui prouve que la série $\sum a_k b_k$ converge.

%%%%%%%%%%%%%%%%%%%%%%%%%%%%%%%%%%%%%%%%%%%%%%%%%%%%%%%%%%%
\diapo

Le cas d'application le plus fréquent est celui où
$b_k=e^{\ii k\theta}$.

\change
Corollaire.

Soit $\theta$ un réel, tel que $\theta \neq 2n\pi$ (pour tout $n \in\Zz$). 

\change
Soit $(a_k)$ une suite de réels positifs, décroissante,
tendant vers $0$. 

\change
Alors les séries de Fourier :
$$
\sum a_k e^{\ii k\theta} \qquad 
\sum a_k\cos(k\theta) \qquad 
\sum a_k\sin(k \theta)
$$
convergent.


%%%%%%%%%%%%%%%%%%%%%%%%%%%%%%%%%%%%%%%%%%%%%%%%%%%%%%%%%%%
\diapo

Démontrons le résultat précédent.

On souhaite appliquer le théorème de sommation d'Abel avec $b_k=e^{\ii k\theta}$, mais nous devons
d'abord vérifier que les sommes partielles de la suite $(e^{\ii k\theta})$ sont
bornées. 

\change
Or $e^{\ii k\theta}=(e^{\ii \theta})^k$, et par hypothèse 
$e^{\ii \theta}$ est différent de $1$. 

\change
On a donc la somme d'une suite géométrique :
$$
\big|1+e^{\ii \theta}+\cdots+e^{\ii k\theta}\big| =
 \left|\frac{1-e^{\ii (k+1)\theta}}{1-e^{\ii \theta}}\right|
 $$
 
\change
qui est inférieur ou égal à 
$$
\left|\frac{2}{1-e^{\ii \theta}}\right|\;.$$

\change
D'où le résultat. 

\change
Comme $\sum a_k e^{\ii k\theta} = \sum a_k\cos(k\theta) + \ii \sum a_k\sin(k \theta)$,

\change
les séries $\sum a_k\cos(k\theta)$ et $\sum a_k\sin(k \theta)$ sont convergentes en tant que partie réelle et partie imagniaire
d'une série complexe convergente.

%%%%%%%%%%%%%%%%%%%%%%%%%%%%%%%%%%%%%%%%%%%%%%%%%%%%%%%%%%%
\diapo

Entraînez-vous avec ces exercices pour vérifier que vous avez bien compris le cours.

\end{document}
