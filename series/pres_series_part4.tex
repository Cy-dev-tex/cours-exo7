
%%%%%%%%%%%%%%%%%% PREAMBULE %%%%%%%%%%%%%%%%%%

\documentclass[aspectratio=169,utf8]{beamer}
%\documentclass[aspectratio=169,handout]{beamer}

\usetheme{Boadilla}
%\usecolortheme{seahorse}
\usecolortheme[RGB={245,66,24}]{structure}
\useoutertheme{infolines}

% packages
\usepackage{amsfonts,amsmath,amssymb,amsthm}
\usepackage[utf8]{inputenc}
\usepackage[T1]{fontenc}
\usepackage{lmodern}

\usepackage[francais]{babel}
\usepackage{fancybox}
\usepackage{graphicx}

\usepackage{float}
\usepackage{xfrac}

%\usepackage[usenames, x11names]{xcolor}
\usepackage{tikz}
\usepackage{pgfplots}
\usepackage{datetime}



%-----  Package unités -----
\usepackage{siunitx}
\sisetup{locale = FR,detect-all,per-mode = symbol}

%\usepackage{mathptmx}
%\usepackage{fouriernc}
%\usepackage{newcent}
%\usepackage[mathcal,mathbf]{euler}

%\usepackage{palatino}
%\usepackage{newcent}
% \usepackage[mathcal,mathbf]{euler}



% \usepackage{hyperref}
% \hypersetup{colorlinks=true, linkcolor=blue, urlcolor=blue,
% pdftitle={Exo7 - Exercices de mathématiques}, pdfauthor={Exo7}}


%section
% \usepackage{sectsty}
% \allsectionsfont{\bf}
%\sectionfont{\color{Tomato3}\upshape\selectfont}
%\subsectionfont{\color{Tomato4}\upshape\selectfont}

%----- Ensembles : entiers, reels, complexes -----
\newcommand{\Nn}{\mathbb{N}} \newcommand{\N}{\mathbb{N}}
\newcommand{\Zz}{\mathbb{Z}} \newcommand{\Z}{\mathbb{Z}}
\newcommand{\Qq}{\mathbb{Q}} \newcommand{\Q}{\mathbb{Q}}
\newcommand{\Rr}{\mathbb{R}} \newcommand{\R}{\mathbb{R}}
\newcommand{\Cc}{\mathbb{C}} 
\newcommand{\Kk}{\mathbb{K}} \newcommand{\K}{\mathbb{K}}

%----- Modifications de symboles -----
\renewcommand{\epsilon}{\varepsilon}
\renewcommand{\Re}{\mathop{\text{Re}}\nolimits}
\renewcommand{\Im}{\mathop{\text{Im}}\nolimits}
%\newcommand{\llbracket}{\left[\kern-0.15em\left[}
%\newcommand{\rrbracket}{\right]\kern-0.15em\right]}

\renewcommand{\ge}{\geqslant}
\renewcommand{\geq}{\geqslant}
\renewcommand{\le}{\leqslant}
\renewcommand{\leq}{\leqslant}
\renewcommand{\epsilon}{\varepsilon}

%----- Fonctions usuelles -----
\newcommand{\ch}{\mathop{\text{ch}}\nolimits}
\newcommand{\sh}{\mathop{\text{sh}}\nolimits}
\renewcommand{\tanh}{\mathop{\text{th}}\nolimits}
\newcommand{\cotan}{\mathop{\text{cotan}}\nolimits}
\newcommand{\Arcsin}{\mathop{\text{arcsin}}\nolimits}
\newcommand{\Arccos}{\mathop{\text{arccos}}\nolimits}
\newcommand{\Arctan}{\mathop{\text{arctan}}\nolimits}
\newcommand{\Argsh}{\mathop{\text{argsh}}\nolimits}
\newcommand{\Argch}{\mathop{\text{argch}}\nolimits}
\newcommand{\Argth}{\mathop{\text{argth}}\nolimits}
\newcommand{\pgcd}{\mathop{\text{pgcd}}\nolimits} 


%----- Commandes divers ------
\newcommand{\ii}{\mathrm{i}}
\newcommand{\dd}{\text{d}}
\newcommand{\id}{\mathop{\text{id}}\nolimits}
\newcommand{\Ker}{\mathop{\text{Ker}}\nolimits}
\newcommand{\Card}{\mathop{\text{Card}}\nolimits}
\newcommand{\Vect}{\mathop{\text{Vect}}\nolimits}
\newcommand{\Mat}{\mathop{\text{Mat}}\nolimits}
\newcommand{\rg}{\mathop{\text{rg}}\nolimits}
\newcommand{\tr}{\mathop{\text{tr}}\nolimits}


%----- Structure des exercices ------

\newtheoremstyle{styleexo}% name
{2ex}% Space above
{3ex}% Space below
{}% Body font
{}% Indent amount 1
{\bfseries} % Theorem head font
{}% Punctuation after theorem head
{\newline}% Space after theorem head 2
{}% Theorem head spec (can be left empty, meaning ‘normal’)

%\theoremstyle{styleexo}
\newtheorem{exo}{Exercice}
\newtheorem{ind}{Indications}
\newtheorem{cor}{Correction}


\newcommand{\exercice}[1]{} \newcommand{\finexercice}{}
%\newcommand{\exercice}[1]{{\tiny\texttt{#1}}\vspace{-2ex}} % pour afficher le numero absolu, l'auteur...
\newcommand{\enonce}{\begin{exo}} \newcommand{\finenonce}{\end{exo}}
\newcommand{\indication}{\begin{ind}} \newcommand{\finindication}{\end{ind}}
\newcommand{\correction}{\begin{cor}} \newcommand{\fincorrection}{\end{cor}}

\newcommand{\noindication}{\stepcounter{ind}}
\newcommand{\nocorrection}{\stepcounter{cor}}

\newcommand{\fiche}[1]{} \newcommand{\finfiche}{}
\newcommand{\titre}[1]{\centerline{\large \bf #1}}
\newcommand{\addcommand}[1]{}
\newcommand{\video}[1]{}

% Marge
\newcommand{\mymargin}[1]{\marginpar{{\small #1}}}

\def\noqed{\renewcommand{\qedsymbol}{}}


%----- Presentation ------
\setlength{\parindent}{0cm}

%\newcommand{\ExoSept}{\href{http://exo7.emath.fr}{\textbf{\textsf{Exo7}}}}

\definecolor{myred}{rgb}{0.93,0.26,0}
\definecolor{myorange}{rgb}{0.97,0.58,0}
\definecolor{myyellow}{rgb}{1,0.86,0}

\newcommand{\LogoExoSept}[1]{  % input : echelle
{\usefont{U}{cmss}{bx}{n}
\begin{tikzpicture}[scale=0.1*#1,transform shape]
  \fill[color=myorange] (0,0)--(4,0)--(4,-4)--(0,-4)--cycle;
  \fill[color=myred] (0,0)--(0,3)--(-3,3)--(-3,0)--cycle;
  \fill[color=myyellow] (4,0)--(7,4)--(3,7)--(0,3)--cycle;
  \node[scale=5] at (3.5,3.5) {Exo7};
\end{tikzpicture}}
}


\newcommand{\debutmontitre}{
  \author{} \date{} 
  \thispagestyle{empty}
  \hspace*{-10ex}
  \begin{minipage}{\textwidth}
    \titlepage  
  \vspace*{-2.5cm}
  \begin{center}
    \LogoExoSept{2.5}
  \end{center}
  \end{minipage}

  \vspace*{-0cm}
  
  % Astuce pour que le background ne soit pas discrétisé lors de la conversion pdf -> png
\begin{tikzpicture}
        \fill[opacity=0,green!60!black] (0,0)--++(0,0)--++(0,0)--++(0,0)--cycle; 
\end{tikzpicture}

% toc S'affiche trop tot :
% \tableofcontents[hideallsubsections, pausesections]
}

\newcommand{\finmontitre}{
  \end{frame}
  \setcounter{framenumber}{0}
} % ne marche pas pour une raison obscure

%----- Commandes supplementaires ------

% \usepackage[landscape]{geometry}
% \geometry{top=1cm, bottom=3cm, left=2cm, right=10cm, marginparsep=1cm
% }
% \usepackage[a4paper]{geometry}
% \geometry{top=2cm, bottom=2cm, left=2cm, right=2cm, marginparsep=1cm
% }

%\usepackage{standalone}


% New command Arnaud -- november 2011
\setbeamersize{text margin left=24ex}
% si vous modifier cette valeur il faut aussi
% modifier le decalage du titre pour compenser
% (ex : ici =+10ex, titre =-5ex

\theoremstyle{definition}
%\newtheorem{proposition}{Proposition}
%\newtheorem{exemple}{Exemple}
%\newtheorem{theoreme}{Théorème}
%\newtheorem{lemme}{Lemme}
%\newtheorem{corollaire}{Corollaire}
%\newtheorem*{remarque*}{Remarque}
%\newtheorem*{miniexercice}{Mini-exercices}
%\newtheorem{definition}{Définition}

% Commande tikz
\usetikzlibrary{calc}
\usetikzlibrary{patterns,arrows}
\usetikzlibrary{matrix}
\usetikzlibrary{fadings} 

%definition d'un terme
\newcommand{\defi}[1]{{\color{myorange}\textbf{\emph{#1}}}}
\newcommand{\evidence}[1]{{\color{blue}\textbf{\emph{#1}}}}
\newcommand{\assertion}[1]{\emph{\og#1\fg}}  % pour chapitre logique
%\renewcommand{\contentsname}{Sommaire}
\renewcommand{\contentsname}{}
\setcounter{tocdepth}{2}



%------ Figures ------

\def\myscale{1} % par défaut 
\newcommand{\myfigure}[2]{  % entrée : echelle, fichier figure
\def\myscale{#1}
\begin{center}
\footnotesize
{#2}
\end{center}}


%------ Encadrement ------

\usepackage{fancybox}


\newcommand{\mybox}[1]{
\setlength{\fboxsep}{7pt}
\begin{center}
\shadowbox{#1}
\end{center}}

\newcommand{\myboxinline}[1]{
\setlength{\fboxsep}{5pt}
\raisebox{-10pt}{
\shadowbox{#1}
}
}

%--------------- Commande beamer---------------
\newcommand{\beameronly}[1]{#1} % permet de mettre des pause dans beamer pas dans poly


\setbeamertemplate{navigation symbols}{}
\setbeamertemplate{footline}  % tiré du fichier beamerouterinfolines.sty
{
  \leavevmode%
  \hbox{%
  \begin{beamercolorbox}[wd=.333333\paperwidth,ht=2.25ex,dp=1ex,center]{author in head/foot}%
    % \usebeamerfont{author in head/foot}\insertshortauthor%~~(\insertshortinstitute)
    \usebeamerfont{section in head/foot}{\bf\insertshorttitle}
  \end{beamercolorbox}%
  \begin{beamercolorbox}[wd=.333333\paperwidth,ht=2.25ex,dp=1ex,center]{title in head/foot}%
    \usebeamerfont{section in head/foot}{\bf\insertsectionhead}
  \end{beamercolorbox}%
  \begin{beamercolorbox}[wd=.333333\paperwidth,ht=2.25ex,dp=1ex,right]{date in head/foot}%
    % \usebeamerfont{date in head/foot}\insertshortdate{}\hspace*{2em}
    \insertframenumber{} / \inserttotalframenumber\hspace*{2ex} 
  \end{beamercolorbox}}%
  \vskip0pt%
}


\definecolor{mygrey}{rgb}{0.5,0.5,0.5}
\setlength{\parindent}{0cm}
%\DeclareTextFontCommand{\helvetica}{\fontfamily{phv}\selectfont}

% background beamer
\definecolor{couleurhaut}{rgb}{0.85,0.9,1}  % creme
\definecolor{couleurmilieu}{rgb}{1,1,1}  % vert pale
\definecolor{couleurbas}{rgb}{0.85,0.9,1}  % blanc
\setbeamertemplate{background canvas}[vertical shading]%
[top=couleurhaut,middle=couleurmilieu,midpoint=0.4,bottom=couleurbas] 
%[top=fondtitre!05,bottom=fondtitre!60]



\makeatletter
\setbeamertemplate{theorem begin}
{%
  \begin{\inserttheoremblockenv}
  {%
    \inserttheoremheadfont
    \inserttheoremname
    \inserttheoremnumber
    \ifx\inserttheoremaddition\@empty\else\ (\inserttheoremaddition)\fi%
    \inserttheorempunctuation
  }%
}
\setbeamertemplate{theorem end}{\end{\inserttheoremblockenv}}

\newenvironment{theoreme}[1][]{%
   \setbeamercolor{block title}{fg=structure,bg=structure!40}
   \setbeamercolor{block body}{fg=black,bg=structure!10}
   \begin{block}{{\bf Th\'eor\`eme }#1}
}{%
   \end{block}%
}


\newenvironment{proposition}[1][]{%
   \setbeamercolor{block title}{fg=structure,bg=structure!40}
   \setbeamercolor{block body}{fg=black,bg=structure!10}
   \begin{block}{{\bf Proposition }#1}
}{%
   \end{block}%
}

\newenvironment{corollaire}[1][]{%
   \setbeamercolor{block title}{fg=structure,bg=structure!40}
   \setbeamercolor{block body}{fg=black,bg=structure!10}
   \begin{block}{{\bf Corollaire }#1}
}{%
   \end{block}%
}

\newenvironment{mydefinition}[1][]{%
   \setbeamercolor{block title}{fg=structure,bg=structure!40}
   \setbeamercolor{block body}{fg=black,bg=structure!10}
   \begin{block}{{\bf Définition} #1}
}{%
   \end{block}%
}

\newenvironment{lemme}[0]{%
   \setbeamercolor{block title}{fg=structure,bg=structure!40}
   \setbeamercolor{block body}{fg=black,bg=structure!10}
   \begin{block}{\bf Lemme}
}{%
   \end{block}%
}

\newenvironment{remarque}[1][]{%
   \setbeamercolor{block title}{fg=black,bg=structure!20}
   \setbeamercolor{block body}{fg=black,bg=structure!5}
   \begin{block}{Remarque #1}
}{%
   \end{block}%
}


\newenvironment{exemple}[1][]{%
   \setbeamercolor{block title}{fg=black,bg=structure!20}
   \setbeamercolor{block body}{fg=black,bg=structure!5}
   \begin{block}{{\bf Exemple }#1}
}{%
   \end{block}%
}


\newenvironment{miniexercice}[0]{%
   \setbeamercolor{block title}{fg=structure,bg=structure!20}
   \setbeamercolor{block body}{fg=black,bg=structure!5}
   \begin{block}{Mini-exercices}
}{%
   \end{block}%
}


\newenvironment{tp}[0]{%
   \setbeamercolor{block title}{fg=structure,bg=structure!40}
   \setbeamercolor{block body}{fg=black,bg=structure!10}
   \begin{block}{\bf Travaux pratiques}
}{%
   \end{block}%
}
\newenvironment{exercicecours}[1][]{%
   \setbeamercolor{block title}{fg=structure,bg=structure!40}
   \setbeamercolor{block body}{fg=black,bg=structure!10}
   \begin{block}{{\bf Exercice }#1}
}{%
   \end{block}%
}
\newenvironment{algo}[1][]{%
   \setbeamercolor{block title}{fg=structure,bg=structure!40}
   \setbeamercolor{block body}{fg=black,bg=structure!10}
   \begin{block}{{\bf Algorithme}\hfill{\color{gray}\texttt{#1}}}
}{%
   \end{block}%
}


\setbeamertemplate{proof begin}{
   \setbeamercolor{block title}{fg=black,bg=structure!20}
   \setbeamercolor{block body}{fg=black,bg=structure!5}
   \begin{block}{{\footnotesize Démonstration}}
   \footnotesize
   \smallskip}
\setbeamertemplate{proof end}{%
   \end{block}}
\setbeamertemplate{qed symbol}{\openbox}


\makeatother
% Couleur à définir

   
%%%%%%%%%%%%%%%%%%%%%%%%%%%%%%%%%%%%%%%%%%%%%%%%%%%%%%%%%%%%%
%%%%%%%%%%%%%%%%%%%%%%%%%%%%%%%%%%%%%%%%%%%%%%%%%%%%%%%%%%%%%


\begin{document}


\title{{\bf Séries}}
\subtitle{Séries absolument convergentes -- Règle de d'Alembert}

\begin{frame}
  
  \debutmontitre

  \pause

{\footnotesize
\hfill
\setbeamercovered{transparent=50}
\begin{minipage}{0.6\textwidth}
  \begin{itemize}
    \item<3-> Séries absolument convergentes
    \item<4-> Règle du quotient de d'Alembert
    \item<5-> Règle des racines de Cauchy
%     \item<6-> D'Alembert vs Cauchy
%     \item<7-> Règle de Raabe-Duhamel
  \end{itemize}
\end{minipage}
}

\end{frame}

\setcounter{framenumber}{0}



%%%%%%%%%%%%%%%%%%%%%%%%%%%%%%%%%%%%%%%%%%%%%%%%%%%%%%%%%%%%%%%%
\section{Séries absolument convergentes}

\begin{frame}

\begin{mydefinition}
On dit qu'une série $\displaystyle\sum_{k\ge0} u_k$ est 
\defi{absolument convergente} si la série $\displaystyle\sum_{k\ge0} |u_k|$ 
est convergente  
\end{mydefinition}

\pause
\begin{exemple}
\begin{enumerate}
  \item La série $\sum_{k\ge1} \frac{\cos k}{k^2}$ est absolument convergente
  
  \pause
  En effet $\left\vert\frac{\cos k}{k^2}\right\vert \le  \frac{1}{k^2}$ et
  $\sum_{k\ge1} \frac{1}{k^2}$ converge
  
  \item\pause La série harmonique alternée $\sum_{k=0}^{+\infty} \frac{(-1)^k}{k+1}$ 
  n'est pas absolument convergente\pause , car la série $\sum_{k\ge0} \left\vert\frac{(-1)^k}{k+1}\right\vert = \sum_{k\ge0} \frac{1}{k+1}$ diverge
\end{enumerate}
\end{exemple}

\pause
Une série qui est convergente mais pas absolument convergente s'appelle une série \defi{semi-convergente}
\end{frame}


\begin{frame}

\begin{theoreme}
Toute série absolument convergente est convergente
\end{theoreme}

\pause
\begin{proof}
Soit $\sum u_k$ une série absolument convergente
\begin{itemize}
\item\pause La série $\sum |u_k|$ est convergente

\pause
Alors la suite des restes $(R'_n)$ tend vers $0$, avec $R'_n = \displaystyle\sum_{k=n+1}^{+\infty} |u_k|$

\pause
En particulier $(R'_n)$ est une suite de Cauchy

\item\pause Soit $\epsilon>0$ fixé. \pause Il existe $n_0 \in \Nn$ tel que 
pour tout $n \ge n_0$ et tout $p \ge 0$ 
$$|u_n|+|u_{n+1}|+\cdots+|u_{n+p}| < \epsilon$$
\pause
Et donc
$$\big|u_n+u_{n+1}+\cdots+u_{n+p}\big| \le |u_n|+|u_{n+1}|+\cdots+|u_{n+p}| < \epsilon$$

\item\pause D'après le critère de Cauchy $\sum u_k$ est donc convergente
\qedhere
\end{itemize}
\end{proof}
\end{frame}



%%%%%%%%%%%%%%%%%%%%%%%%%%%%%%%%%%%%%%%%%%%%%%%%%%%%%%%%%%%%%%%%
\section{Règle du quotient de d'Alembert}

\begin{frame}

\begin{theoreme}[Règle du quotient de d'Alembert]
\pause
Soit $\sum u_k$ une série de terme général non nul
\begin{enumerate}
\item\pause S'il existe une constante $0 < q < 1$ et un entier $k_0$ tels que pour tout
$k \ge k_0$
$$
\left|\frac{u_{k+1}}{u_k}\right| \le q <1,\quad \text{ alors }\quad\sum u_k\quad \text{converge}
$$
\pause
La série est même absolument convergente
\item\pause S'il existe un entier $k_0$ tel que pour tout $k \ge k_0$
$$
\left|\frac{u_{k+1}}{u_k}\right| \ge 1,\quad \text{ alors }\quad\sum u_k \quad\text{diverge}
$$
\end{enumerate}
\end{theoreme}
\end{frame}

\begin{frame}

\begin{corollaire}[Règle du quotient de d'Alembert]
\pause
Soit $\sum u_k$ une série à terme strictement positif telle que 
$\frac{u_{k+1}}{u_k}$ converge vers $\ell$
\medskip
\begin{enumerate}
\item\pause Si $\ell<1$ alors $\sum u_k$ converge
\medskip
\item\pause Si $\ell>1$ alors $\sum u_k$ diverge
\medskip
\item\pause Si $\ell=1$ on ne peut pas conclure en général
\end{enumerate}
\end{corollaire}

\end{frame}

\begin{frame}
\begin{proof}[th]
Rappel : la série géométrique $\sum q^k$ converge si $|q|<1$

\begin{enumerate}
\item\pause Si $\left|\frac{u_{k+1}}{u_k}\right| \le q< 1$

\begin{itemize}
\item\pause Alors $|u_{k_0+1}| \le |u_{k_0}| q,$ \pause \  puis \  $|u_{k_0+2}| \le |u_{k_0}| q^2, \ldots$ 
\item\pause Par récurrence : pour tout $k\ge k_0$ 
$$|u_k| \le |u_{k_0}| q^{-k_0} \cdot q^k = c \cdot q^k$$
où $c$ est une constante
\item\pause Comme $0 < q < 1$, la série $\sum q^k$ converge

\pause
alors la série $\sum |u_k|$ converge par le théorème de comparaison
\end{itemize}

\item\pause Si $\left|\frac{u_{k+1}}{u_k}\right| \ge 1$
\begin{itemize}
\item\pause la suite $(|u_k|)$ est croissante\pause : elle ne
peut donc pas tendre vers $0$

\pause
donc la série $\sum |u_k|$ diverge
\end{itemize}  
\end{enumerate}
\end{proof}
\end{frame}

\begin{frame}

\begin{exemple}
\begin{enumerate}  
  \item\pause Pour tout $x \in \Rr$ fixé, la \defi{série exponentielle}
$\displaystyle\sum_{k=0}^{+\infty} \frac{x^k}{k!}$ converge

\begin{itemize}
\item\pause Pour $u_k = \frac{x^k}{k!}$ on a 
\pause
\vspace*{-1ex}
$$
\left|\frac{u_{k+1}}{u_k}\right|
= \frac{\left|\frac{x^{k+1}}{(k+1)!}\right|}{\left|\frac{x^k}{k!}\right|}
=\frac{|x|}{k+1} \to 0 
\quad \text{lorsque } k \to +\infty
$$
\pause
La limite étant $\ell = 0 < 1$ alors la série est absolument convergente

\item\pause Par définition la somme est $\exp(x)$ : 
\vspace*{-1ex}
$$\exp(x) = \sum_{k=0}^{+\infty} \frac{x^k}{k!}$$
\end{itemize}  
\vspace*{-1ex}
\item\pause $\sum_{k\ge0} \frac{k!}{1\cdot 3\,\cdots\,(2k-1)}$ converge\pause , car $\frac{u_{k+1}}{u_k}=\frac{k+1}{2k+1}$ tend vers $\frac{1}{2}<1$
  
\item\pause $\sum_{k\ge0} \frac{(2k)!}{(k!)^2}$ diverge\pause , car $\frac{u_{k+1}}{u_k}=\frac{(2k+1)(2k+2)}{(k+1)^2}$ tend vers $4>1$
\end{enumerate}
\end{exemple}

\end{frame}


\begin{frame}

\begin{remarque}
\begin{itemize}
  \item\pause Le théorème ne peut pas s'appliquer si certains $u_k$ sont nuls
  \item\pause Le théorème ne permet pas toujours de conclure.
  
L'hypothèse est $\left|\frac{u_{k+1}}{u_k}\right| \le q <1$, ce qui est plus fort que $\left|\frac{u_{k+1}}{u_k}\right| <1$
  
  \item\pause De même le corollaire ne permet pas de conclure lorsque $\frac{u_{k+1}}{u_k} \to 1$
  
  \pause
  Par exemple pour $\sum u_k= \sum \frac{1}{k}$ et $\sum v_k = \sum \frac{1}{k^2}$
\begin{itemize}
\item\pause  Nous avons $\frac{u_{k+1}}{u_k} = \frac{k}{k+1} \to 1$\pause ,  de même $\frac{v_{k+1}}{v_k} = \frac{k^2}{(k+1)^2 } \to 1$
\item\pause Cependant la série $\sum \frac{1}{k}$ diverge 
  alors que $\sum \frac{1}{k^2}$ converge
\end{itemize}
\end{itemize}  
\end{remarque}
\end{frame}

\begin{frame}

\begin{exemple}
Trouver tous les $z\in \Cc$ tels que la série 
$\displaystyle\sum_{k\ge0} \binom{k}{3} z^k$ soit absolument convergente

\begin{itemize}
\item \pause Soit $u_k=\binom{k}{3} z^k$. \pause Alors, pour $z\neq0$,
$$\frac{|u_{k+1}|}{|u_k|}=\frac{\binom{k+1}{3}|z|^{k+1}}{\binom{k}{3}|z|^{k}}  \pause =
\frac{\frac{(k+1)k(k-1)}{3!}}{\frac{k(k-1)(k-2)}{3!}}|z|
 \pause =\frac{k+1}{k-2}|z|
  \pause \longrightarrow |z| $$ 
lorsque $k\to+\infty$

\item \pause Si $|z|<1$ alors pour $k$ assez grand $\frac{|u_{k+1}|}{|u_k|} < q <1$

 \pause
donc la série $\sum u_k$ est absolument convergente

\item \pause Si $|z|\ge 1$ alors $\frac{|u_{k+1}|}{|u_k|}=\frac{k+1}{k-2}|z|  \pause \ge \frac{k+1}{k-2}> 1$
pour tout $k$

 \pause
donc la série $\sum u_k$  diverge
\end{itemize}  
\end{exemple}
\end{frame}

%%%%%%%%%%%%%%%%%%%%%%%%%%%%%%%%%%%%%%%%%%%%%%%%%%%%%%%%%%%%%%%%
\section{Règle des racines de Cauchy}

\begin{frame}
\begin{theoreme}[Règle des racines de Cauchy]
\pause
Soit $\sum u_k$ une série de nombres réels ou complexes
\begin{enumerate}
\item\pause S'il existe une constante $0<q<1$ et un entier $k_0$ tels que pour tout
$k \ge k_0$
$$
\sqrt[k]{|u_k|} \le q <1,\quad \text{ alors }\quad\sum u_k\quad \text{converge}
$$
\pause
La série est même absolument convergente
\item\pause S'il existe un entier $k_0$ tel que pour tout $k \ge k_0$
$$
\sqrt[k]{|u_k|} \ge 1,\quad \text{ alors }\quad\sum u_k \quad\text{diverge}
$$
\end{enumerate}
\end{theoreme}
\end{frame}


\begin{frame}

\begin{corollaire}[Règle des racines de Cauchy]
\pause
Soit $\sum u_k$ une série à termes positifs telle que 
$\sqrt[k]{u_k}$ converge vers $\ell$
\medskip
\begin{enumerate}
\item\pause Si $\ell<1$ alors $\sum u_k$ converge
\medskip
\item\pause Si $\ell>1$ alors $\sum u_k$ diverge
\medskip
\item\pause Si $\ell=1$ on ne peut pas conclure en général
\end{enumerate}
\end{corollaire}

\bigskip

\pause
Dans la pratique, il faut savoir bien manipuler les racines $k$-ème : 
$$\sqrt[k]{u_k} = (u_k)^{\frac1k} = \exp\left(\tfrac1k \ln u_k\right)$$
\end{frame}



\begin{frame}
\begin{proof}
\begin{enumerate}
\item\pause Si $\sqrt[k]{|u_k|} \le q<1$
\begin{itemize}
\item\pause Alors $|u_k| \le q^k$
\item\pause Comme $0<q<1$, la série $\sum q^k$ converge

\pause
donc la série converge par le théorème de comparaison
\end{itemize}  
\item\pause Si $\sqrt[k]{|u_k|} \ge 1$
\begin{itemize}
\item\pause Alors $|u_k| \ge 1$ : le terme général ne tend pas vers $0$

\pause
donc la série diverge
\end{itemize}  
\item\pause Pour le dernier point du corollaire :

\pause
on pose $u_k=\frac{1}{k}$, $v_k=\frac{1}{k^2}$
\begin{itemize}
\item\pause On a $\sqrt[k]{u_k}\to 1$ de même que $\sqrt[k]{v_k}\to 1$
\item\pause Mais $\sum u_k$ diverge alors que $\sum v_k$ converge
\end{itemize}  
\end{enumerate}
\end{proof}
\end{frame}


\begin{frame}
\begin{exemple}
\begin{enumerate}
  \item Par exemple
$$\sum \left(\frac{2k+1}{3k+4}\right)^k\quad \text{ converge}$$
\pause
car $\sqrt[k]{u_k} = \frac{2k+1}{3k+4}$ tend vers $\frac{2}{3}<1$
  
  
  \item\pause Par contre  
  $$\sum \frac{2^k}{k^\alpha}\quad \text{ diverge}$$
  quel que soit $\alpha >0$. \pause
  En effet,
  $$\sqrt[k]{u_k} = \frac{\sqrt[k]{2^k}}{\big(\sqrt[k]{k}\big)^\alpha}
  = \frac{2}{\big(k^\frac{1}{k}\big)^\alpha}
  = \frac{2}{\big(\exp(\frac{1}{k}\ln k)\big)^\alpha} \to 2>1$$
\end{enumerate}
\end{exemple}
\end{frame}

\begin{frame}
\begin{exemple}
Déterminer tous les $z\in \Cc$ tels que la série  
$\sum_{k\ge1} \left( 1+\frac{1}{k}\right)^{k^2} z^k$ soit absolument convergente
\medskip

\pause
Notons $u_k =\left( 1+\frac{1}{k}\right)^{k^2} z^k$. \pause
On a $$\sqrt[k]{|u_k|}=\left( 1+\frac{1}{k}\right)^k |z| \to e|z|$$

\pause
Cette limite vérifie $e|z| < 1$ si et seulement si $|z|<\frac{1}{e}$

\begin{itemize}
  \item\pause Si $|z|<\frac{1}{e}$ alors la série $\sum u_k$ est absolument 
convergente
  
  \item\pause Si $|z|>\frac{1}{e}$, on a pour $k$ assez grand
  $\sqrt[k]{|u_k|}>1$, donc la série $\sum u_k$ diverge
  
  \item\pause Si $|z|=\frac{1}{e}$ il faut étudier le terme général à la main
\end{itemize}
\end{exemple}
\end{frame}


\begin{frame}
\begin{exemple}  
 $u_k =\left( 1+\frac{1}{k}\right)^{k^2} z^k$
\begin{itemize}
  \item Si $|z|=\frac{1}{e}$ \pause on obtient
  $|u_k|=\left(  1+\frac{1}{k}\right)^{k^2}\left(\frac{1}{e}\right)^k$
 \pause 
  Donc
  \begin{eqnarray*}
  \ln|u_k| 
  & = & k^2\ln\left(1+\tfrac{1}{k}\right)+k\ln \tfrac{1}{e} \\ \pause
  & = & k\left[k\ln (1+\tfrac{1}{k})-1\right] \\ \pause
  & = & k \left[ k\left(\tfrac{1}{k}-\tfrac{1}{2}\left( \tfrac{1}{k}\right)^2+
o\big( \tfrac{1}{k^2}\big) \right)-1\right] \\ \pause
  & = & k\left[1-\tfrac{1}{2}\tfrac{1}{k}+o\big( \tfrac{1}{k}\big) -1\right] \\
  & = & -\tfrac{1}{2} +o(1) \\ \pause
  & \to & -\tfrac{1}{2}
  \end{eqnarray*}
  \pause
Donc $|u_k| \to e^{-\frac{1}{2}}\neq 0$. Ainsi $\sum |u_k|$ diverge
\end{itemize}

\end{exemple}
\end{frame}

% %%%%%%%%%%%%%%%%%%%%%%%%%%%%%%%%%%%%%%%%%%%%%%%%%%%%%%%%%%%%%%%%
% \section{D'Alembert vs Cauchy}
% 
% \begin{frame}
% \pause
% \begin{proposition}
%  \pause
% Soit $(u_k)$ une suite à termes strictement positifs  \pause
% $$
% \text{Si } \quad \lim_{k\to+\infty} \frac{u_{k+1}}{u_k} = \ell
% \qquad\text{alors}\qquad
% \lim_{k\to+\infty} \sqrt[k]{u_k} = \ell
% $$
% \end{proposition} 
% 
% \pause
% \begin{exemple}
% \pause
% Définissons la suite $u_k$ par
% $
% u_k = \left\{
% \begin{array}{ll}
% \dfrac{2^n}{3^n}&\text{ si } k=2n\\[2ex]
% \dfrac{2^n}{3^{n+1}}&\text{ si } k=2n+1
% \end{array}
% \right.
% $
% \begin{itemize}
% \item\pause $\frac{u_{k+1}}{u_k}$ vaut $\frac{1}{3}$ si $k$ est pair, $2$ si $k$ est impair
% 
% \pause
% Donc la règle du quotient de d'Alembert ne s'applique pas
% \item\pause Pourtant $\sqrt[k]{u_k}$ converge vers $\sqrt{\frac{2}{3}}<1$
% 
% \pause
% Donc la règle des racines de Cauchy s'applique et $\sum u_k$ converge
% \end{itemize}
% \end{exemple}
% \end{frame}
% 
% 
% 
% %%%%%%%%%%%%%%%%%%%%%%%%%%%%%%%%%%%%%%%%%%%%%%%%%%%%%%%%%%%%%%%%
% \section{Règle de Raabe-Duhamel}
% 
% \begin{frame}
% 
% Les règles de d'Alembert et de Cauchy ne s'appliquent pas aux séries de Riemann  
% $$\sum_{k\ge1} \frac{1}{k^\alpha}$$
% \pause
% car $\frac{k^\alpha}{(k+1)^\alpha}\to 1$ et $\sqrt[k]{u_k} \to 1$
% 
% \pause
% \begin{theoreme}[Règle de Raabe-Duhamel]
% Soit $(u_k)$ une suite de nombres réels (ou complexes) non nuls
% \begin{enumerate}
%   \item\pause Si $\forall k\geq k_0$ on a 
%   $\left|\frac{u_{k+1}}{u_k}\right|\le 1-\frac{\beta}{k}$, avec $\beta>1$, \pause
% alors la série $\sum u_k$ est absolument convergente
%   \item\pause Si $\forall k\geq k_0$ on a 
%   $\left|\frac{u_{k+1}}{u_k}\right|\ge 1-\frac{1}{k}$, \pause alors
%  la série $\sum u_k$ n'est  pas absolument convergente
% \end{enumerate}
% \end{theoreme}
% 
% \pause
% \textbf{Attention !}
% Il existe des séries convergentes, 
% quoique $\left|\frac{u_{k+1}}{u_k}\right|\ge 1-\frac{1}{k}$
% 
% \pause
% En effet, pour $u_k=(-1)^k \frac{1}{k}$ alors $\tfrac{|u_{k+1}|}{|u_k|}= \tfrac{k}{k+1}= 1-\tfrac{1}{k+1}\ge 1-\tfrac{1}{k}$
% 
% \end{frame}
% 
% 
% \begin{frame}
% 
% \begin{proposition}[Séries de Riemann]
% Soit $\alpha>0$. Alors la série $\displaystyle\sum_{k\ge1} \frac{1}{k^\alpha}$
% converge si et seulement si $\alpha>1$
% \end{proposition}
% 
% \end{frame}



%%%%%%%%%%%%%%%%%%%%%%%%%%%%%%%%%%%%%%%%%%%%%%%%%%%%%%%%%%%%%%%%
\section{Mini-exercices}

\begin{frame}
\begin{miniexercice}
\begin{enumerate}

  \item Ces séries sont-elles convergentes ? Absolument convergentes ?
  
  $\displaystyle\sum_{k\geq 2} \tfrac{(-1)^{k^3}e^{\ii k}}{k^2+k}
  \qquad
  \sum_{k\geq 1} \tfrac{(-1)^k}{\sqrt{k(k+1)}}
  \qquad
  \sum_{k\geq 1} \tfrac{\sqrt k}{(-1)^k\ln k}$
  
  \item \'Etudier les séries dont voici le terme général, par la règle du quotient de d'Alembert 
  ou des racines de Cauchy :
  
  $\frac{k^{100}}{k!} \qquad \frac{k!}{(2k)!} \qquad 
  \frac{\ln k}{2^k+1} \qquad \frac{2\cdot 4 \cdot 6 \cdots (2k)}{k^k}  \qquad  \big(\sin \tfrac1k\big)^k \qquad 
  \left(\frac{7k-2}{3k+1}\right)^k \qquad \frac{2^k}{e^k-1}$
  
  
  \item Appliquer la règle du quotient de d'Alembert pour $u_k = \frac{k!}{k^k}$
  
  En déduire la limite de $\sqrt[k]{u_k}$ lorsque $k$ tend vers $+\infty$
  
  \item \'Etudier les séries dont voici le terme général en 
  fonction du paramètre $\alpha > 0$ :
  
  $\frac{k}{k^{\alpha}+1} \qquad \frac{\ln k}{k^\alpha} \qquad \sqrt{k}\alpha^k \qquad 
  \frac{\alpha^k}{k^2} \qquad \ln (1 + k^\alpha) 
  \qquad \frac{2\cdot 4 \cdot 6 \cdots (2k)}{3\cdot 5 \cdot 7 \cdots (2k+1)}
  $
\end{enumerate}
\end{miniexercice}
\end{frame}

\end{document}