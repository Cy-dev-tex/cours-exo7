\documentclass[class=report,crop=false]{standalone}
\usepackage[screen]{../exo7book}

\begin{document}

%====================================================================
\chapitre{Séries}
%====================================================================

%\insertvideo{}{partie 1. }


% Introduction

Dans ce chapitre nous allons nous intéresser à des sommes ayant une infinité de termes.
Par exemple que peut bien valoir la somme infinie suivante :
\[1+\frac{1}{2} +\frac{1}{4} +\frac{1}{8}+\frac{1}{16} +\cdots \ \ = \ \ ? \]


\myfigure{1.3}{
\tikzinput{fig_series01} 
}

Cette question a été
popularisée sous le nom du \evidence{paradoxe de Zénon}. 
On tire une flèche à $2$~mètres d'une cible. 
Elle met un certain laps de temps pour parcourir
la moitié de la distance, à savoir un mètre. Puis il lui faut encore du 
temps pour parcourir la moitié de la distance restante, et de nouveau un 
certain temps pour la moitié de la distance encore restante. 
On ajoute ainsi une infinité de durées non nulles, et Zénon en conclut 
que la flèche n'atteint jamais sa cible ! Zénon ne concevait pas qu'une infinité 
de distances finies puisse être parcourue en un temps fini.
Et pourtant nous allons voir dans ce chapitre que la somme d'une infinité de termes
peut être une valeur finie.



%%%%%%%%%%%%%%%%%%%%%%%%%%%%%%%%%%%%%%%%%%%%%%%%%%%%%%%%%%%%%%%%
\section{Définitions -- Série géométrique}



%---------------------------------------------------------------
\subsection{Définitions}

\begin{definition}
Soit $(u_k)_{k \ge 0}$ une suite de nombres réels (ou de nombres complexes).
On pose
$$S_n=u_0+u_1+u_2+\cdots+ u_n=\sum_{k=0}^n u_k.$$

La suite $(S_n)_{n \ge 0}$ s'appelle la \defi{série} de terme général 
$u_k$.

Cette série est notée par la somme infinie $\displaystyle \sum_{k \ge 0} u_k $.
La suite $(S_n)$ s'appelle aussi la \defi{suite des sommes partielles}.
\end{definition}

\begin{exemple}
Fixons $q\in \Cc$. Définissons la suite $(u_k)_{k \ge 0}$ par
$u_k = q^k$ ; c'est une suite géométrique.
La \defi{série géométrique} $\displaystyle \sum_{k \ge 0} q^k$ est la suite des sommes partielles :
$$S_0 = 1 \qquad S_1 = 1 + q \qquad S_2 = 1+q+q^2 \qquad \ldots \qquad S_n = 1+q+q^2+\cdots+q^n \quad \ldots$$
\end{exemple}


\begin{definition}
Si la suite $(S_n)_{n \ge 0}$ admet une limite finie dans $\Rr$ (ou dans $\Cc$),
on note
$$S = \sum_{k=0}^{+\infty} u_k =\lim_{n\to+\infty} S_n.$$

On appelle alors $S= \sum_{k=0}^{+\infty} u_k$ la \defi{somme} de la série $\sum_{k \ge 0} u_k$,
et on dit que la série est \defi{convergente}.
Sinon, on dit qu'elle est \defi{divergente}. 
\end{definition}



\textbf{Notations.}
On peut noter une série de différentes façons, et bien sûr avec différents symboles pour l'indice : 
$$\sum_{i = 0}^{+\infty} u_i \qquad \sum_{n \in \Nn} u_n \qquad {\textstyle \sum_{k \ge 0} u_k}\qquad \sum u_k.$$

Pour notre part, on fera la distinction entre une série quelconque $\displaystyle \sum_{k \ge 0} u_k$,
et on réservera la notation $\displaystyle \sum_{k=0}^{+\infty} u_k$ à une série convergente ou à sa somme.




%---------------------------------------------------------------
\subsection{Série géométrique}


\begin{proposition}
Soit $q\in \Cc$. La série géométrique $\sum_{k \ge 0} q^k$ est convergente 
si et seulement si $|q|<1$. On a alors 
\mybox{$\displaystyle \sum_{k=0}^{+\infty} q^k =1+q+q^2+q^3+\cdots= \frac{1}{1-q}$}
\end{proposition}

\begin{proof}

Considérons 
$$S_n=1+q+q^2+q^3+\cdots+q^n.$$

\begin{itemize}
  \item \'Ecartons tout de suite le cas $q=1$, pour lequel $S_n = n+1$.
Dans ce cas $S_n \to +\infty$, et la série diverge.
  
  \item Soit $q \neq 1$ et multiplions $S_n$ par $1-q$ :
$$(1-q)S_n=(1+q+q^2+q^3+\cdots+q^n)-(q+q^2+q^3+\cdots+q^{n+1})=1-q^{n+1}$$
Donc
\mybox{$\displaystyle S_n  = \frac{1-q^{n+1}}{1-q}$}

Si $|q|<1$, alors $q^n \to 0$, donc $q^{n+1} \to 0$ et ainsi $S_n \to \frac{1}{1-q}$.
Dans ce cas la série $\sum_{k \ge 0} q^k$ converge.

Si $|q| \ge 1$, alors la suite $(q^n)$ n'a pas de limite finie
(elle peut tendre vers $+\infty$, par exemple si $q=2$ ;
ou bien être divergente, par exemple si $q=-1$).
Donc si $|q| \ge 1$, $(S_n)$ n'a pas de limite finie, donc la série 
$\sum_{k \ge 0} q^k$ diverge.
\end{itemize}
\end{proof}


\begin{exemple}
\begin{enumerate}
  \item Série géométrique de raison $q=\frac 12$ : \quad 
  $\displaystyle \sum_{k=0}^{+\infty} \frac{1}{2^k} = \frac{1}{1-\frac12} = 2$.
  Cela résout le paradoxe de Zénon : la flèche arrive bien jusqu'au mur !

  
  \item Série géométrique de raison $q=\frac 13$, avec premier terme $\frac{1}{3^3}$. 
  On se ramène à la série géométrique commençant à $k=0$ en ajoutant et retranchant les premiers termes :
  $\displaystyle \sum_{k=3}^{+\infty} \frac{1}{3^k} = \sum_{k=0}^{+\infty} \frac{1}{3^k} \  -  1 - \frac 13-\frac 1{3^2} 
  = \frac{1}{1-\frac13} - \frac{13}{9}= \frac32-\frac{13}{9} = \frac{1}{18}$.
  
  \item Le fait de calculer la somme d'une série à partir de $k=0$ est
purement conventionnel. On peut toujours effectuer un changement
d'indice pour se ramener à une somme à partir de $0$. Une autre façon pour calculer la même série 
$\displaystyle \sum_{k=3}^{+\infty} \frac{1}{3^k}$ que précédemment
est de faire le changement d'indice $n=k-3$ (et donc $k=n+3$):
$$\sum_{k=3}^{+\infty} \frac{1}{3^k} 
= \sum_{n=0}^{+\infty} \frac{1}{3^{n+3}}
= \sum_{n=0}^{+\infty} \frac{1}{3^3}\frac{1}{3^n}
= \frac{1}{3^3}\sum_{n=0}^{+\infty} \frac{1}{3^n}
= \frac{1}{27}\frac{1}{1-\frac13}
= \frac{1}{18}$$


  \item $\displaystyle \sum_{k=0} ^{+\infty} (-1)^k \left(\frac{1}{2}\right)^{2k}
  =\sum_{k=0} ^{+\infty} \left(-\frac{1}{4}\right)^k=\frac{1}{1-\frac{-1}{4}}=\frac{4}{5}.$
 
\end{enumerate}
\end{exemple}

%---------------------------------------------------------------
\subsection{Séries convergentes}

La convergence d'une série ne dépend pas de ses premiers termes : 
changer un nombre fini de termes d'une
série ne change pas sa nature, convergente ou divergente. 
Par contre, si elle est convergente, sa somme est évidemment modifiée.


Une façon pratique d'étudier la convergence d'une série est d'étudier son reste :
le \defi{reste d'ordre~$n$} d'une série convergente $\sum_{k = 0}^{+\infty} u_k$ est :
$$R_n = u_{n+1}+u_{n+2}+\cdots = \sum_{k=n+1}^{+\infty} u_k$$

\begin{proposition}
Si une série est convergente, alors 
$S=S_n+R_n$ (pour tout $n\ge0$) et $\lim_{n\to+\infty} R_n=0$.
\end{proposition}

\begin{proof}
\begin{itemize}
  \item $S = \sum_{k=0}^{+\infty} u_k =\sum_{k=0}^{n} u_k+ \sum_{k=n+1}^{+\infty} u_k=S_n+R_n$.
  \item Donc $R_n = S-S_n \to S-S =0$ lorsque $n\to+\infty$.
\end{itemize} 
\end{proof}



%---------------------------------------------------------------
\subsection{Suites et séries}


Il n'y a pas de différence entre l'étude des suites et des séries. 
On passe de l'une à l'autre très facilement.

Tout d'abord rappelons qu'à une série $\sum_{k \ge 0} u_k$, on associe 
la somme partielle $S_n=\sum_{k=0}^n u_k$ et que par définition
la série est convergente si la suite $(S_n)_{n\ge0}$ converge.



Réciproquement si on veut étudier une suite $(a_k)_{k\ge0}$ on peut utiliser le résultat suivant :
\begin{proposition}
Une \defi{somme télescopique} est une série de la forme
$$\sum_{k\ge0} (a_{k+1}-a_k).$$

Cette série est convergente si et seulement si $\ell := \lim_{k\to+\infty} a_k$ existe et dans ce cas
on a :
$$\sum_{k=0}^{+\infty} (a_{k+1}-a_k) = \ell - a_0.$$
\end{proposition}

\begin{proof}
\begin{eqnarray*}
S_n
&=& \sum_{k=0}^n (a_{k+1}-a_k)\\
&=& (a_1-a_0)+(a_2-a_1)+(a_3-a_2) +\cdots +(a_{n+1}-a_n)\\
&=& -a_0 +a_1-a_1+a_2-a_2+ \cdots +a_n-a_n+a_{n+1}\\
&=& a_{n+1}-a_0 
\end{eqnarray*}
\end{proof}


Voici un exemple très important pour la suite.
\begin{exemple}
\label{ex:seriek2}
La série $$\sum_{k=0}^{+\infty} \frac{1}{(k+1)(k+2)}=\frac{1}{1\cdot 2}+\frac{1}{2\cdot 3}+
 \frac{1}{3\cdot 4}+\cdots$$
est convergente et a la valeur $1$. 
En effet, elle peut être écrite comme somme télescopique, et plus précisément la somme partielle vérifie :
$$S_n=\sum_{k=0}^n \frac{1}{(k+1)(k+2)}= \sum_{k=0}^n \left(\frac{1}{k+1}-\frac{1}{k+2}\right)
= 1 - \frac{1}{n+2} \to 1 \quad \text{ lorsque } n\to+\infty$$

Par changement d'indice, on a aussi que les séries 
$\sum_{k=1}^{+\infty} \frac{1}{k(k+1)}$ et 
$\sum_{k=2}^{+\infty} \frac{1}{k(k-1)}$ sont convergentes et de même somme $1$.
\end{exemple}


%---------------------------------------------------------------
\subsection{Le terme d'une série convergente tend vers $0$}


\begin{theoreme}
Si la série $\sum_{k\ge0} u_k$ converge, 
alors la suite des termes généraux $(u_k)_{k \ge 0}$ tend vers $0$.
\end{theoreme}

Le point clé est que l'on retrouve le terme général à partir des sommes partielles par la formule
$$u_n = S_n - S_{n-1}.$$

\begin{proof}
Pour tout $n \ge 0$, posons $S_n=\sum_{k=0}^{n} u_k$. Pour tout
$n \ge 1$, $u_n=S_n-S_{n-1}$. Si $\sum_{k\ge0} u_k$ converge, 
la suite $(S_n)_{n\ge0}$ converge vers la somme $S$ de la série. Il en
est de même de la suite $(S_{n-1})_{n \ge 1}$. Par linéarité de
la limite, la suite $(u_n)$ tend vers $S-S=0$.
\end{proof}

\bigskip

La contraposée de ce résultat est souvent utilisée : 
\mybox{Une série dont le terme général ne tend pas vers $0$ ne peut pas
converger.}

Par exemple les séries $\sum_{k \ge 1} (1+\frac{1}{k})$ et $\sum_{k \ge 1} k^2$ sont 
divergentes.

Plus intéressant, la série $\sum u_k$ de terme général 
$$
u_k = \left\{
\begin{array}{ll}
1&\text{ si } k=2^\ell \quad \text{ pour un certain } \ell \ge 0 \\
0&\text{ sinon }
\end{array}
\right.
$$
diverge. En effet, même si les termes valant $1$ sont très rares, il y en a quand
même une infinité !

%---------------------------------------------------------------
\subsection{Linéarité}

\begin{proposition}
Soient $\sum_{k=0}^{+\infty} a_k$ et $\sum_{k=0}^{+\infty} b_k$ 
deux séries convergentes de sommes respectives $A$ et $B$, et soient $\lambda, \mu \in \Rr$ (ou $\Cc$).
Alors la série $\sum_{k=0}^{+\infty} (\lambda a_k+\mu b_k)$ est convergente et de somme
 $\lambda A+\mu B$. On a donc
 $$\sum_{k=0}^{+\infty} (\lambda a_k+\mu b_k) = 
 \lambda \sum_{k=0}^{+\infty} a_k+ \mu \sum_{k=0}^{+\infty} b_k.$$

\end{proposition}
\begin{proof}
$A_n=\sum_{k=0}^n a_k\to A\in\Cc, B_n=\sum_{k=0}^n b_k\to B\in\Cc$. Donc
$ \sum_{k=0}^n (\lambda a_k+\mu b_k) 
= \lambda  \sum_{k=0}^n a_k + \mu \sum_{k=0}^n b_k 
= \lambda A_n + \mu B_n \to \lambda A +\mu B$.
\end{proof}


Par exemple :
$$
\sum_{k=0}^{+\infty} \left(\frac{1}{2^k}+\frac{5}{3^k}\right)
 = 
\sum_{k=0}^{+\infty} \frac{1}{2^k}+
5\sum_{k=0}^{+\infty} \frac{1}{3^k}
=
\frac{1}{1-\frac{1}{2}}+5\frac{1}{1-\frac{1}{3}} =
 2+5\frac{3}{2}=\frac{19}{2}\;.
$$

\bigskip

Comme application pour les séries à termes complexes, la convergence équivaut à
celle des parties réelle et imaginaire :
 
\begin{proposition}
\label{prop:reimseries}
Soit $(u_k)_{k\ge0}$ une suite de nombres complexes. Pour tout $k$, notons
$u_k = a_k + \ii b_k$, avec $a_k$ la partie réelle de $u_k$ et $b_k$ la partie imaginaire. 
La série $\sum u_k$ converge si et seulement si les deux séries 
$\sum a_k$ et $\sum b_k$ convergent. Si c'est le cas, on a :
$$
\sum_{k=0}^{+\infty} u_k = 
\sum_{k=0}^{+\infty} a_k + \ii \sum_{k=0}^{+\infty} b_k\;. 
$$
\end{proposition}

\begin{exemple}
Considérons par exemple la série géométrique $\sum_{k\ge0} r^k$, où
$r = \rho e^{\ii\theta}$ est un complexe de module $\rho<1$ et d'argument $\theta$.

Comme le module de $r$ est strictement inférieur à $1$, alors la série converge et 
$$\sum_{k=0}^{+\infty} r^k = \frac{1}{1-r}.$$

D'autre part, $r^k = \rho^k e^{\ii k\theta}$ par la formule de Moivre. 
Les parties réelle et imaginaire de $r^k$ sont
$$
a_k=\rho^k\cos(k\theta)
\quad\text{ et }\quad 
b_k=\rho^k\sin(k\theta)\;.
$$
On déduit de la proposition précédente que :
$$
\sum_{k=0}^{+\infty} a_k = \Re \left(\sum_{k=0}^{+\infty} r^k\right) = \Re \left(\frac{1}{1-r}\right) 
\quad\text{ et }\quad 
\sum_{k=0}^{+\infty} b_k = \Im \left(\sum_{k=0}^{+\infty} r^k\right)  = \Im \left(\frac{1}{1-r}\right)\;. 
$$
Le calcul donne :
$$
\sum_{k=0}^{+\infty} \rho^k\cos(k\theta) = 
\frac{1-\rho\cos\theta}{1+\rho^2-2\rho\cos\theta} 
\quad\text{ et }\quad 
\sum_{k=0}^{+\infty} \rho^k\sin(k\theta) = 
\frac{\rho\sin\theta}{1+\rho^2-2\rho\cos\theta}
\;. 
$$  
\end{exemple}

%---------------------------------------------------------------
\subsection{Sommes de séries}


Pour l'instant, il n'y a pas beaucoup de séries dont vous connaissez 
la somme, à part les séries géométriques. Il faudra attendre d'autres chapitres
et d'autres techniques pour calculer des sommes de séries. Dans ce chapitre
on s'intéressera essentiellement à savoir si une série converge ou diverge.

\bigskip

Voici cependant une exception !

\begin{exemple}
Soit $q \in \Cc$ tel que $|q|<1$.
Que vaut la somme
$$\sum_{k=0}^{+\infty} k q^k \quad \text{?}$$ 

Admettons un moment que cette série converge et notons 
$S = \sum_{k=0}^{+\infty} k q^k$.

\'Ecrivons : 
\begin{align*}
S 
&= \sum_{k=0}^{+\infty} kq^k = \sum_{k=1}^{+\infty} kq^k = q\sum_{k=1}^{+\infty} kq^{k-1} \\
&= q\sum_{k=1}^{+\infty} q^{k-1} + q\sum_{k=1}^{+\infty}(k-1)q^{k-1}\\
&= q\sum_{k=1}^{+\infty} q^{k-1} + q\sum_{k'=0}^{+\infty} k'q^{k'} \qquad \text{ en posant } k'=k-1\\
&= q\sum_{k=1}^{+\infty} q^{k-1} + q\cdot S\\
\end{align*}
En résolvant cette équation en $S$, on trouve que 
$$(1-q)S = q\sum_{k=1}^{+\infty} q^{k-1}\;.$$
Cette dernière série est une série géométrique de raison $q$ avec $|q|<1$ donc converge.
Cela justifie la convergence de $S$.

Ainsi $$(1-q)S = q \cdot \frac{1}{1-q}\;.$$
Conclusion :
$$S = \sum_{k=0}^{+\infty} k q^k = \frac{q}{(1-q)^2}\;.$$ 
\end{exemple}




%---------------------------------------------------------------
\subsection{Critère de Cauchy}

\evidence{Attention !}
Il existe des séries $\sum_{k\ge0} u_k$ telles que $\lim_{k\to+\infty} u_k=0$,
mais $\sum_{k\ge0} u_k$ diverge. L'exemple le plus classique est
la \defi{série harmonique} :
\mybox{La série \quad $\displaystyle \sum_{k \ge 1} \frac{1}{k} = 1 + \frac12+\frac13+\frac14+\cdots$ \quad diverge}

Plus précisément, on a $\lim_{n\to+\infty} S_n=+\infty$. Cependant
on a $u_k = \frac{1}{k} \to 0$ (lorsque $k \to +\infty$).


\bigskip

Pour montrer que la série diverge nous allons utiliser le critère de Cauchy.



\textbf{Rappel.}
Une suite $(s_n)$ de nombres réels (ou complexes) converge 
si et seulement si elle est une suite de Cauchy, c'est-à-dire :
$$\forall \epsilon>0\quad \exists n_0\in\Nn \quad \forall m,n \ge n_0 \qquad |s_n-s_m|<\epsilon$$

Pour les séries cela nous donne : 
\begin{theoreme}[Critère de Cauchy]
\label{th:cauchyserie}
Une série  $\displaystyle\sum_{k=0}^{+\infty} u_k$ converge si et seulement si  
$$\forall \epsilon>0 \quad \exists n_0\in\Nn \quad \forall m,n\ge n_0 \qquad \big|u_{n}+\cdots +u_m\big| < \epsilon \; .$$
\end{theoreme}
On le formule aussi de la façon suivante :
$$ \forall \epsilon>0 \quad \exists n_0\in\Nn \quad
\forall m,n \ge n_0 \qquad  \left| \sum_{k=n}^m u_k\right| <\epsilon$$
ou encore
$$ \forall \epsilon>0 \quad \exists n_0\in\Nn \quad
\forall n\ge n_0 \quad \forall p\in\Nn 
\qquad  \big|u_{n}+\cdots +u_{n+p}\big|<\epsilon$$



\begin{proof}
La preuve est simplement de dire que la suite $(S_n)$ des sommes partielles converge si
et seulement si c'est une suite de Cauchy. Ensuite il suffit de remarquer que 
$$\big|S_m-S_{n-1}\big|= \big|u_{n}+\cdots +u_m\big|.$$
\end{proof}



Revenons à la série harmonique $\sum_{k \ge 1} \frac{1}{k}$.
La somme partielle est $S_n = \sum_{k=1}^{n} \frac{1}{k}$.
Calculons la différence de deux sommes partielles, afin de conserver 
les termes entre $n+1$ (qui joue le rôle de $n$)
et $2n$ (qui joue le rôle de $m$) :
$$
S_{2n}-S_{n} = \frac{1}{n+1}+\cdots+\frac{1}{2n}\ge
\frac{n}{2n}=\frac{1}{2} 
$$
La suite des sommes partielles n'est pas de Cauchy (car $\frac12$ 
n'est pas inférieur à $\epsilon = \frac14$ par exemple), 
donc la série ne converge pas.


Si on souhaite terminer la démonstration sans utiliser directement le critère de Cauchy alors on raisonne par l'absurde. Supposons que $S_n \to \ell \in \Rr$ (lorsque $n \to +\infty$).
Alors on a aussi $S_{2n} \to \ell$ (lorsque $n \to +\infty$) et donc 
$S_{2n}-S_{n}  \to \ell - \ell = 0$. Ce qui entre en contradiction avec l'inégalité
$S_{2n}-S_{n} \ge \frac{1}{2}$.

\bigskip


On termine par une étude plus poussée de la série harmonique.

\begin{proposition}
Pour la série harmonique $\displaystyle \sum_{k \ge 1} \frac{1}{k}$  
et sa somme partielle
$\displaystyle S_n = \sum_{k = 1}^n \frac{1}{k}$, on a 
$$\lim_{n\to+\infty} S_n=+\infty.$$
\end{proposition}

\begin{proof}
Soit $M>0$. On choisit $m\in \Nn$ tel que $m\ge 2M$. 
Alors pour $n\ge 2^m$ on a:
\begin{eqnarray*}
S_n 
  & = & 1+\frac{1}{2}+\frac{1}{3}+\cdots+\frac{1}{2^m}+\cdots+ \frac{1}{n} \\
  &\ge & 1+\frac{1}{2}+\frac{1}{3}+\cdots+ \frac{1}{2^m} \\
  & = & 1 + \frac{1}{2}+ \left(\frac{1}{3}+\frac{1}{4}\right) 
  +\left(\frac{1}{5}+\frac{1}{6}+\frac{1}{7}+ \frac{1}{8}\right) +\left(\frac{1}{9}+\cdots +\frac{1}{16}\right)+  \cdots
+\left(\frac{1}{2^{m-1}+1}+\cdots + \frac{1}{2^m}\right) \\
 & \ge & 1+\frac{1}{2}+2 \frac{1}{4}+ 4\;\frac{1}{8}+ 8 \frac{1}{16}+\cdots +
2^{m-1} \frac{1}{2^m} \\
&=& 1+m \frac{1}{2}\ge M  \\
\end{eqnarray*}

L'astuce consiste à regrouper les termes. Entre chaque parenthèses il 
y a successivement $2,4,8,...$ termes jusqu'à 
$$2^m-(2^{m-1}+1)+1= 
2^m-2^{m-1}=2^{m-1} \quad \text{ termes.}$$

Ainsi pour tout $M>0$ il existe $n_0 \ge 0$ tel que, pour tout $n \ge n_0$, on ait $S_n \ge M$ ; ainsi
$(S_n)$ tend vers $+\infty$. Cela reprouve bien sûr que la série harmonique diverge.
\end{proof}


%---------------------------------------------------------------
%\subsection{Mini-exercices}

\begin{miniexercices}
\begin{enumerate}
  
  \item Calculer les sommes partielles $S_n$ de la série dont le terme général est 
  $\frac{1}{4^k}$, commençant à $k=1$. Cette série est-elle convergente ? 
  Si c'est possible, calculer la somme $S$ et les restes $R_n$.
  
  \item Mêmes questions avec $\sum_{k\ge0} (-1)^k$, $\sum_{k\ge0} 3^k$, 
  $\sum_{k\ge1} \frac{1}{10^k}$, $\sum_{k\ge2} \exp(-k)$.
  
 \item Pourquoi les séries suivantes sont-elles divergentes ?
  $\sum_{k\ge1} \left( \frac1k + (-1)^k \right)$ ; $\sum_{k\ge 0} \frac{k}{k+1}$ ; 
  $\sum_{k\ge1} \frac{1}{2k}$ ;
  $\sum _{k\ge1} k \cos(k)$ ; $\sum_{k\ge1} \exp(\frac1k)$.
    
   
  \item Calculer les sommes partielles de la série 
  $\sum_{k\ge1} \ln\left( 1-\frac{1}{k+1} \right)$. 
  Cette série est-elle convergente ?
  
  \item Montrer que $\displaystyle \sum_{k=0}^{+\infty} k^2q^k = \frac{q^2+q}{(1-q)^3}$.

\end{enumerate}
\end{miniexercices}



%%%%%%%%%%%%%%%%%%%%%%%%%%%%%%%%%%%%%%%%%%%%%%%%%%%%%%%%%%%%%%%%
\section{Séries à termes positifs}

Les séries à termes positifs ou nuls se comportent comme les suites croissantes
et sont donc plus faciles à étudier.


%---------------------------------------------------------------
\subsection{Convergence par les sommes partielles}

\textbf{Rappels.}
Soit $(s_n)_{n\ge0}$ une suite croissante de nombres réels.
\begin{itemize}
  \item Si la suite est majorée, alors la suite $(s_n)$ converge, c'est-à-dire
  qu'elle admet une limite finie.
  \item Sinon la suite $(s_n)$ tend vers $+\infty$.
\end{itemize}

Appliquons ceci aux séries $\sum u_k$ à \defi{termes positifs}, 
c'est-à-dire $u_k\ge 0$ pour tout $k$.
Dans ce cas la suite $(S_n)$ des sommes partielles, définie par 
$S_n = \sum_{k=0}^n u_k$, est une suite croissante.
En effet 
$$S_{n}-S_{n-1} = u_n \ge 0.$$

Par les rappels sur les suites, nous avons donc :
\begin{proposition}
\label{somme}
Une série à termes positifs est une série convergente si et seulement si 
la suite des sommes partielles est majorée.
Autrement dit, si et seulement s'il existe $M>0$
tel que, pour tout $n\ge 0$, $S_n \le M$.
\end{proposition}

De plus, dans le cas de convergence, la somme de la série
$S$ vérifie bien sûr $\lim S_n = S$, 
mais aussi $S_n \le S$, pour tout $n$.
Les deux situations convergence/divergence sont possibles :
$\sum_{k\ge0} q^k$ converge si $0<q<1$, et diverge si $q \ge 1$.



%---------------------------------------------------------------
\subsection{Théorème de comparaison}

Quelle est la méthode générale pour trouver la nature d'une série à termes positifs ?
On la compare avec des séries classiques simples au moyen du théorème de comparaison suivant.

\begin{theoreme}[Théorème de comparaison]
\label{th:comparaisonseries}
Soient $\sum u_k$ et $\sum v_k$ deux séries à termes positifs ou nuls. On
suppose qu'il existe $k_0\ge 0$ tel que, pour tout $k\ge k_0$, 
$u_k \le v_k$.
\begin{itemize}
\item Si $\sum v_k$ converge alors $\sum u_k$ converge.
\item Si $\sum u_k$ diverge alors $\sum v_k$ diverge.
\end{itemize}
\end{theoreme}

\begin{proof}
Comme nous l'avons observé, la convergence ne dépend pas des
premiers termes. Sans perte de généralité on peut donc supposer $k_0=0$.
Notons $S_n=u_0+\cdots+u_n$ et $S'_n = v_0+\cdots+v_n$. 
Les suites $(S_n)$ et $(S'_n)$ sont croissantes, et de plus, pour tout $n \ge 0$,
$S_n\le S'_n$. Si la série $\sum v_k$ converge, alors la suite
$(S'_n)$ converge. Soit $S'$ sa limite. La suite $(S_n)$ est croissante
et majorée par $S'$, donc elle converge, et ainsi la série $\sum u_k$
converge aussi. Inversement, si la série $\sum u_k$ diverge, alors
la suite $(S_n)$ tend vers $+\infty$, et il en est de même pour la
suite $(S'_n)$ et ainsi la série $\sum v_k$ diverge. 
\end{proof}


%---------------------------------------------------------------
\subsection{Exemples}

\begin{exemple}
Nous avons déjà vu dans l'exemple \ref{ex:seriek2} que la
série
$$
\sum_{k=0}^{+\infty} \frac{1}{(k+1)(k+2)}
\quad\text{ converge.}
$$
Nous allons en déduire que
$$
\sum_{k=1}^{+\infty} \frac{1}{k^2}
\quad\text{ converge.}
$$
En effet, on a :
$$
\lim_{k\to+\infty}
\frac{\frac{1}{2k^2}}{\frac{1}{(k+1)(k+2)}}=\frac{1}{2}. 
$$
En particulier, il existe $k_0$ tel que pour $k \ge k_0$ :
$$
\frac{1}{2k^2} \le \frac{1}{(k+1)(k+2)}
$$
En fait c'est vrai pour $k \ge 4$, mais il est inutile de calculer une
valeur précise de $k_0$. On en déduit que la série de terme
général $\frac{1}{2k^2}$ converge, d'où le résultat par
linéarité.  
\end{exemple}


\begin{exemple}
Voici un exemple fondamental, la \defi{série exponentielle}.
\mybox{$\displaystyle\text{La série} \quad \sum_{k\ge 0} \frac{1}{k!} \quad \text{converge.}$}
Notons que $0!=1$ et que pour $k\ge 1$, $k!=1\cdot 2\cdot 3\dots \cdot k$.

En effet 
$\frac{1}{k!}\le \frac{1}{k(k-1)}$ pour $k \ge 2$,
mais $\sum_{k \ge 2}\frac{1}{k(k-1)} = \sum_{k\ge0} \frac{1}{(k+1)(k+2)}$ (par changement d'indice)
est une série convergente.
Donc la série exponentielle $\sum_{k\ge 0} \frac{1}{k!}$ converge.

En fait, par définition, la somme $\sum_{k=0}^{+\infty} \frac{1}{k!}$ 
vaut le nombre d'Euler $e = \exp(1)$.
\end{exemple}


\begin{exemple}
Inversement, nous avons vu que la série $\sum_{k\ge1} \frac{1}{k}$
diverge. On en déduit facilement que les séries 
$\sum_{k\ge1} \frac{\ln(k)}{k}$ et $\sum_{k\ge1} \frac{1}{\sqrt{k}}$ divergent également.  
\end{exemple}




Terminons avec une application intéressante : le développement décimal d'un réel.
\begin{exemple}
Soit $(a_k)_{k\ge 1}$ une suite d'entiers tous compris entre $0$ et $9$. La série
$$
\sum_{k=1}^{+\infty} \frac{a_k}{10^k}
\quad\text{converge.}
$$
En effet, son terme général $u_k=\frac{a_k}{10^k}$ est majoré par $\frac{9}{10^k}$. 
Mais la série géométrique $\sum \frac{1}{10^k}$ converge, car $\frac{1}{10}<1$. La série 
$\sum \frac{9}{10^k}$ converge aussi par linéarité, d'où le résultat.

Une telle somme $\sum_{k=1}^{+\infty} \frac{a_k}{10^k}$ est une écriture décimale d'un réel
$x$, avec ici $0 \le x \le 1$.

Par exemple, si $a_k = 3$ pour tout $k$ : 
$$\sum_{k=1}^{+\infty} \frac{3}{10^k} 
= \frac{3}{10}+\frac{3}{100}+\frac{3}{1000}+\cdots
= 0,3+0,03+0,003+\cdots = 0,333\ldots = \frac13$$
On retrouve bien sûr le même résultat à l'aide de la série géométrique :
$$\sum_{k=1}^{+\infty} \frac{3}{10^k} 
= \frac{3}{10} \sum_{k=0}^{+\infty} \frac{1}{10^k}
=  \frac{3}{10} \cdot \frac{1}{1-\frac{1}{10}}
=  \frac{3}{10} \cdot \frac{10}{9}
= \frac13$$
\end{exemple}



%---------------------------------------------------------------
\subsection{Théorème des équivalents}

Nous allons améliorer le théorème de comparaison avec la notion de suites équivalentes.


Soient $(u_k)$ et $(v_k)$ deux suites \evidence{strictement positives}. Alors les suites
$(u_k)$ et $(v_k)$ sont \defi{équivalentes} si 
$$\lim_{k\to+\infty} \frac{u_k}{v_k}=1.$$
On note alors $$u_k \sim v_k.$$


\begin{theoreme}[Théorème des équivalents]
\label{th:equivalentseries}
Soient $(u_k)$ et $(v_k)$ deux suites à termes strictement positifs.
Si $u_k \sim v_k$ alors les séries $\sum u_k$ et $\sum v_k$ sont de même nature.
\end{theoreme}

Autrement dit, si les suites sont équivalentes alors elles sont soit toutes les deux convergentes,
soit toutes les deux divergentes. Bien sûr, en cas de convergence, il n'y a aucune 
raison que les sommes soient égales. Enfin, si les suites sont toutes les deux strictement négatives, la
conclusion reste valable.

Revenons sur un exemple qui montre que ce théorème est très pratique :
les suites $\frac{1}{k^2}$ et $\frac{1}{(k+1)(k+2)}=\frac{1}{k^2+3k+2}$
sont équivalentes. Comme la série $\sum \frac{1}{(k+1)(k+2)}$ converge 
(exemple \ref{ex:seriek2}), alors
cela implique que $\sum \frac{1}{k^2}$ converge.

\begin{proof}
Par hypothèse, pour tout $\epsilon>0$, il existe $k_0$ tel que, pour
tout $k \ge k_0$,
$$\left|\frac{u_k}{v_k} -1\right| < \epsilon,$$
ou autrement dit 
$$(1-\epsilon)v_k < u_k <(1+\epsilon) v_k.$$

Fixons un $\epsilon <1$.
Si $\sum u_k$ converge, alors par le théorème \ref{th:comparaisonseries} de comparaison, 
$\sum(1-\epsilon) v_k$ converge, donc $\sum v_k$ également. 
Réciproquement, si $\sum u_k$ diverge, alors
$\sum (1+\epsilon)v_k$ diverge, et $\sum v_k$ aussi.
\end{proof}


%---------------------------------------------------------------
\subsection{Exemples}

\begin{exemple}
Les deux séries 
$$
\sum \frac{k^2+3k+1}{k^4+2k^3+4}
\qquad \text{ et } \qquad 
\sum \frac{k +\ln(k)}{k^3}
\quad \text{ convergent.}
$$
Dans les deux cas, le terme général est équivalent à $\frac{1}{k^2}$, 
et nous savons que la série $\sum \frac{1}{k^2}$ converge.  
\end{exemple}

\begin{exemple}
Par contre
$$
\sum \frac{k^2+3k+1}{k^3+2k^2+4}
\qquad \text{ et } \qquad 
\sum \frac{k +\ln(k)}{k^2}
\;\mbox{ divergent.}
$$
Dans les deux cas, le terme général est équivalent à
$\frac{1}{k}$, et nous avons vu que la série $\sum \frac{1}{k}$
diverge.  
\end{exemple}

Voyons un exemple plus sophistiqué.
\begin{exemple}
Est-ce que la série 
$$\sum_{k \ge 1}  \ln \big(\tanh k\big) \quad \text{converge ?}$$

La méthode est de chercher un équivalent simple du terme général.

\begin{itemize}
  \item Remarquons tout d'abord que, pour $k >0$, $0<\tanh k<1$.
  
  \item Puis évaluons $\tanh k$ :
  $$\tanh k = \frac{\sh k}{\ch k} = \frac{e^k-e^{-k}}{e^k+e^{-k}}
  = 1+  \frac{-2e^{-k}}{e^k+e^{-k}} = 1+\frac{-2e^{-2k}}{1+e^{-2k}}$$

  
  \item Comme $\lim_{x\to 0} \frac{\ln(1+x)}{x}=1$, alors, si
  $u_k \to 0$, $\ln(1+u_k) \sim u_k$. Ainsi 
  $$\ln(\tanh k) = \ln\left( 1+\frac{-2e^{-2k}}{1+e^{-2k}} \right)
  \sim  \frac{-2e^{-2k}}{1+e^{-2k}} \sim -2e^{-2k}$$
  
  
  \item La série $\sum e^{-2k}= \sum (e^{-2})^k$ converge car c'est une série géométrique de raison
  $\frac{1}{e^2} < 1$.
  
  \item Les suites $\ln(\tanh k)$ et $-2e^{-2k}$ sont deux suites strictement négatives et on a vu que 
  $\ln(\tanh k) \sim -2e^{-2k}$. Par le théorème \ref{th:equivalentseries} des équivalents, comme la série
  $\sum -2e^{-2k}$ converge, alors la série $\sum\ln(\tanh k)$ converge également. (Si vous préférez, vous pouvez 
  appliquer le théorème aux suites strictement positives $-\ln(\tanh k)$ et $2e^{-2k}$.)
 
\end{itemize}

\end{exemple}



%---------------------------------------------------------------
%\subsection{Mini-exercices}


\begin{miniexercices}
\begin{enumerate}

  \item Montrer que la série 
  $\displaystyle \sum_{k=1}^{+\infty} \frac{(\ln k)^\alpha}{k^3}$ converge, 
  quel que soit $\alpha \in \Rr$.
  
  \item Montrer que, si la série à termes positifs $\sum_{k=0}^{+\infty} u_k$ converge, 
  alors la série $\sum_{k=0}^{+\infty} u_k^2$ converge aussi.

  \item Soient  $\sum_{k\ge0} u_k$ et $\sum_{k\ge0} v_k$ deux séries 
  vérifiant $u_k>0$, $v_k>0$ et pour tout $k\ge0$ :
  $0<m \le \frac{u_k}{v_k} \le M$.
  Montrer que les deux séries sont de même nature.

  \item Par comparaison ou recherche d'équivalent, déterminer la nature de la série
  $\sum_{k \ge 1} \frac{\ln k}{k}$.
  Même question avec les séries de terme général 
  $\sin \left(\frac{1}{(k-1)(k+1)}\right)$ ;
  $\sqrt{1+\frac{1}{k^2}}-\sqrt{1-\frac{1}{k^2}}$ ; 
  $\ln\left(\sqrt{1-\frac{1}{\sqrt[3]{k^2}}}\right)$.
  
  
  \item \'Ecrire la série associée au développement décimal $0,99999\ldots$
  Notons $S$ la somme de cette série. Calculer la série correspondant à $10 \cdot S$. Simplifier
  $10 \cdot S - S$. En déduire $S$. Retrouver cette valeur $S$ 
  à l'aide d'une série géométrique.
  
  
  \item Justifier que la série $\sum_{k=2}^{+\infty} \frac{1}{k^2-1}$ est convergente.
  Décomposer $\frac{1}{k^2-1}$ en éléments simples. 
  Déterminer une expression des sommes partielles $S_n$.
  En déduire que $\sum_{k=2}^{+\infty} \frac{1}{k^2-1} = \frac34$.

  \item Nous admettons ici que $\sum_{k=0}^{+\infty}\frac{1}{k!} = e$.
  Sans calculs, déterminer les sommes :
  $$\sum_{k=0}^{+\infty}\frac{k}{k!} \qquad 
  \sum_{k=0}^{+\infty}\frac{k(k-1)}{k!} \qquad
  \sum_{k=0}^{+\infty}\frac{k^2}{k!}$$
\end{enumerate}
\end{miniexercices}


%%%%%%%%%%%%%%%%%%%%%%%%%%%%%%%%%%%%%%%%%%%%%%%%%%%%%%%%%%%%%%%%
\section{Séries alternées}

Il existe un autre type de série facile à étudier : les séries alternées.
Ce sont celles où le signe du terme général change à chaque rang.

%---------------------------------------------------------------
\subsection{Critère de Leibniz}


Soit $(u_k)_{k\ge0}$ une suite qui vérifie $u_k \ge 0$. La série $\sum_{k \ge 0} (-1)^k u_k$ 
s'appelle une \defi{série alternée}.


On a le critère de convergence suivant, extrêmement facile à vérifier :
\begin{theoreme}[Critère de Leibniz]
\label{th:seriealternee}
Supposons que $(u_k)_{k\ge0}$ soit une suite qui vérifie :
\begin{enumerate}
  \item $u_k  \ge 0$ pour tout $k \ge 0$,
  \item la suite $(u_k)$ est une suite décroissante,
  \item et $\lim_{k\to+\infty} u_k=0$.
\end{enumerate}
Alors la série alternée $\displaystyle \sum_{k=0}^{+\infty} (-1)^k u_k$ converge.
\end{theoreme}

\begin{proof}
Nous allons nous ramener à deux suites adjacentes.

\begin{itemize}
  \item La suite $(S_{2n+1})$ est croissante car 
  $S_{2n+1}-S_{2n-1}=u_{2n}-u_{2n+1}\ge 0$.
  
  \item La suite $(S_{2n})$ est décroissante car
  $S_{2n}-S_{2n-2}= u_{2n}-u_{2n-1}\le 0$.
  
  \item $S_{2n} \ge S_{2n+1}$ car 
  $S_{2n+1} - S_{2n} = -u_{2n+1} \le 0$.
  
  \item Enfin $S_{2n+1} - S_{2n}$ tend vers $0$
  car $S_{2n+1} - S_{2n} = -u_{2n+1} \to 0$
  (lorsque $n\to+\infty$).
\end{itemize}
En conséquence $(S_{2n+1})$ et $(S_{2n})$ convergent
et en plus convergent vers la même limite $S$.
On conclut que $(S_n)$ converge vers $S$.

En plus on a montré que $S_{2n+1} \le S \le S_{2n}$ pour tout $n$.

Enfin on a aussi 
 $$0\ge R_{2n}= S-S_{2n} \ge S_{2n+1}-S_{2n}=-u_{2n+1}$$ et
 $$0\le R_{2n+1}= S-S_{2n+1} \le  S_{2n+2} -S_{2n+1}= u_{2n+2}.$$
 
 Ainsi, quelle que soit la parité de $n$, on a 
 $|R_n|=|S-S_n|\le  u_{n+1}$.
 \end{proof}
 

\begin{exemple}
La \defi{série harmonique alternée} 
$$\sum_{k=0}^{+\infty} (-1)^{k} \frac{1}{k+1} = 1-\frac{1}{2}+\frac{1}{3}-\frac{1}{4} + \cdots $$
converge.
En effet, en posant $u_k = \frac{1}{k+1}$, alors
\begin{enumerate}
  \item $u_k\ge0$,
  \item $(u_k)$ est une suite décroissante,
  \item la suite $(u_k)$ tend vers $0$.
\end{enumerate}
Par le critère de Leibniz (théorème \ref{th:seriealternee}), la série alternée
$\sum_{k=0}^{+\infty} (-1)^{k} \frac{1}{k+1}$ converge.
\end{exemple}

%---------------------------------------------------------------
\subsection{Reste}

Non seulement le critère de Leibniz prouve la convergence de la série 
$\sum_{k=0}^{+\infty} (-1)^k u_k$, mais la preuve nous fournit deux résultats importants supplémentaires :
un encadrement de la somme et une majoration du reste.

\begin{corollaire}
Soit une série alternée $\displaystyle \sum_{k=0}^{+\infty} (-1)^k u_k$ vérifiant 
les hypothèses du théorème \ref{th:seriealternee}.
Soit $S$ la somme de cette série et soit $(S_n)$ la suite des sommes partielles.
\begin{enumerate}
  \item La somme $S$ vérifie les encadrements :
  $$ S_1\le S_3\le S_5\le \cdots \le S_{2n+1} \le \cdots \le S 
  \le  \cdots\le S_{2n} \le \cdots\le S_4\le S_2\le S_0.$$
  \item En plus, si $\displaystyle R_n=S-S_n =\sum_{k=n+1}^{+\infty} (-1)^k u_k$ est le reste d'ordre $n$, alors on  a
$$\big|R_n\big|\le u_{n+1}.$$  
\end{enumerate}
\end{corollaire}

Pour une série alternée, la vitesse de convergence est donc
dictée par la décroissance vers $0$ de la suite $(u_k)$. Celle-ci
peut être assez lente.

\begin{exemple}
Par exemple, on a vu que la série harmonique alternée 
$\sum_{k=0}^{+\infty} \frac{(-1)^{k}}{k+1}$ converge ;
notons $S$ sa somme.
Les sommes partielles sont
$S_0 = 1$, 
$S_1 = 1-\frac{1}{2}$,
$S_2 = 1-\frac{1}{2}+\frac{1}{3}$,
$S_3 = 1-\frac{1}{2}+\frac{1}{3}-\frac{1}{4}$,
$S_4 = 1-\frac{1}{2}+\frac{1}{3}-\frac{1}{4}+\frac{1}{5}$,\ldots
L'encadrement du corollaire s'écrit 
$$1-\frac{1}{2} \le 1-\frac{1}{2}+\frac{1}{3}-\frac{1}{4} \le \cdots \le S_{2n+1} \le \cdots \le S 
  \le  \cdots\le S_{2n} \le \cdots 
  \le 1-\frac{1}{2}+\frac{1}{3}-\frac{1}{4}+\frac{1}{5}
  \le 1-\frac{1}{2}+\frac{1}{3} \le 1$$
On en déduit 
$$S_3 = \frac{35}{60} \simeq 0,58333\ldots \le S \le S_4 = \frac{47}{60} \simeq 0,78333\ldots$$

  
Si on pousse les calculs plus loin, alors pour $n=200$
on obtient 
$$S_{201} \simeq 0,69067\ldots \le S \le S_{200} \simeq 0,69562\ldots$$
Ce qui nous donne les deux premières décimales de $S \simeq 0,69\ldots$

En plus nous avons une majoration de l'erreur commise, en utilisant l'inégalité $|R_n|\le u_{n+1}$.
On trouve que l'erreur commise en approchant $S$ par $S_{200}$ est :
$|S-S_{200}| = |R_{200}| \le u_{201} = \frac{1}{202} < 5 \cdot 10^{-3}$.

En fait, vous verrez plus tard que $S=\ln 2 \simeq 0,69314\ldots$
\end{exemple}




%---------------------------------------------------------------
\subsection{Contre-exemple}

Terminons par deux mises en garde : 

\begin{enumerate}
  \item On ne peut pas laisser tomber la condition 
  de décroissance de la suite $(u_k)$ dans le critère
de Leibniz.

  \item Il n'est pas possible de remplacer
$u_k$ par un équivalent à l'infini dans le théorème
\ref{th:seriealternee}, car la décroissance n'est pas conservée par
équivalence.   
\end{enumerate}




\begin{exemple}
Voici deux séries alternées :
$$
\sum_{k \ge 2}  \frac{(-1)^k}{\sqrt{k}}
\quad \text{ converge,}
\qquad\qquad
\sum_{k \ge 2}  \frac{(-1)^k}{\sqrt{k}+(-1)^k}
\quad \text{ diverge.}
$$ 
Le critère de Leibniz (théorème \ref{th:seriealternee}) s'applique à la première :
la suite $u_k = \frac{1}{\sqrt k}$ est une suite positive, décroissante, 
qui tend vers $0$. Conséquence, la série alternée  
$\sum_{k \ge 2}  \frac{(-1)^k}{\sqrt{k}}$ converge.


Par contre le critère de Leibniz ne s'applique pas à la seconde, 
car si la suite $v_k = \frac{1}{\sqrt{k}+(-1)^k}$ est bien
positive (pour $k \ge 2$) et tend vers $0$, elle n'est pas
décroissante. 

Cependant, on a bien :
$$
v_k = \frac{1}{\sqrt{k}+(-1)^k} \ \sim \ \frac{1}{\sqrt{k}} = u_k
$$


Pour montrer que $\sum_{k \ge 2} \frac{(-1)^k}{\sqrt{k}+(-1)^k}$ diverge,
calculons la différence : 
$$
(-1)^ku_k-(-1)^kv_k 
=  \frac{(-1)^k}{\sqrt{k}}-\frac{(-1)^k}{\sqrt{k}+(-1)^k} 
= (-1)^k\frac{\sqrt{k}+(-1)^k-\sqrt{k}}{k+(-1)^k\sqrt{k}} 
=  \frac{1}{k+(-1)^k\sqrt{k}} 
\sim  \frac{1}{k} 
$$


Ainsi la série de terme général $w_k = (-1)^ku_k-(-1)^kv_k$
diverge, car son terme général est équivalent 
à celui de la série harmonique $\sum \frac{1}{k}$
qui diverge.

Supposons maintenant par l'absurde que la série
$\sum_{k \ge 2}(-1)^k v_k$ soit convergente.
On sait aussi que la série $\sum_{k \ge 2} (-1)^k u_k$
est convergente. Donc par linéarité la série
$\sum_{k \ge 2} w_k = \sum_{k \ge 2} (-1)^k u_k - \sum_{k \ge 2} (-1)^k v_k$
serait convergente. Ce qui est une contradiction.


Conclusion : la série $\sum_{k \ge 2} \frac{(-1)^k}{\sqrt{k}+(-1)^k}$ diverge.  
\end{exemple}

%---------------------------------------------------------------
%\subsection{Mini-exercices}

\begin{miniexercices}
\begin{enumerate}
  \item Est-ce que le critère de Leibniz s'applique aux séries suivantes ?
  
  \centerline{$\displaystyle
  \sum_{k\ge 2} \frac{(-1)^k}{\sqrt k + \ln k} \qquad
  \sum_{k\ge 2} \frac{(-1)^k}{\frac{k+1}{k}} \qquad
  \sum_{k\ge2} \frac{1}{(-1)^{k+1}(\sqrt k - \ln k)}
  $}
  
  \centerline{$\displaystyle
  \sum_{k\ge 2} (-1)^k\big(\ln(k+1)-\ln(k)\big) \qquad
  \sum_{k\ge2} \frac{1}{\sqrt k + (-1)^{k}\ln k} \qquad
  \sum_{k\ge 2} \frac{(-1)^k}{3k + (-1)^k}$}
  

  \item \`A partir de quel rang la somme partielle
  $S_n$ de la série $\sum_{k=1}^{+\infty} \frac{(-1)^k}{k^2}$ 
  est-elle une approximation à $0,1$ près de sa somme $S$ ?
  Et à $0,001$ près ?
  \`A l'aide d'une calculatrice ou d'un ordinateur, déterminer deux décimales
  exactes après la virgule de $S$.
  Mêmes questions avec  $\frac{(-1)^k}{2^k}$ ; $\frac{(-1)^k}{\sqrt{k}}$ ;
  $\frac{(-1)^k}{k!}$.
  
\end{enumerate}
\end{miniexercices}



%%%%%%%%%%%%%%%%%%%%%%%%%%%%%%%%%%%%%%%%%%%%%%%%%%%%%%%%%%%%%%%%
\section{Séries absolument convergentes -- Règle de d'Alembert}

%---------------------------------------------------------------
\subsection{Séries absolument convergentes}


\begin{definition}
On dit qu'une série $\sum_{k\ge0} u_k$ de nombres réels (ou complexes) est 
\defi{absolument convergente} si la série $\sum_{k\ge0} |u_k|$ 
est convergente.  
\end{definition}



\begin{exemple}
\begin{enumerate}
  \item Par exemple la série $\sum_{k\ge1} \frac{\cos k}{k^2}$ est absolument convergente.
  Car pour $u_k =  \frac{\cos k}{k^2}$ on a $|u_k| \le  \frac{1}{k^2}$. Comme la série
  $\sum_{k\ge1} \frac{1}{k^2}$ converge alors $\sum_{k\ge1} |u_k|$ converge aussi.
  
  \item La série harmonique alternée $\sum_{k=0}^{+\infty} \frac{(-1)^k}{k+1}$ 
  n'est pas absolument convergente. Car pour $v_k = \frac{(-1)^k}{k+1}$,
  la série $\sum_{k\ge0} |v_k| = \sum_{k\ge0} \frac{1}{k+1}$ diverge.
\end{enumerate}
\end{exemple}

Une série, telle que la série harmonique alternée, qui est convergente, 
mais pas absolument convergente, s'appelle une série \defi{semi-convergente}.

\medskip

\^Etre absolument convergent est plus fort qu'être convergent :
\begin{theoreme}
Toute série absolument convergente est convergente.
\end{theoreme}

\begin{proof}
Utilisons le critère de Cauchy.  Soit $\sum u_k$ une série absolument convergente.
La série $\sum |u_k|$ est convergente, donc la suite des restes $(R'_n)$ avec 
$R'_n = \sum_{k=n+1}^{+\infty} |u_k|$ est une suite 
qui tend vers $0$, donc en particulier c'est une suite de Cauchy.
Soit $\epsilon>0$ fixé. Il existe donc $n_0 \in \Nn$ tel que 
pour tout $n \ge n_0$ et pour tout $p \ge 0$ :
$$|u_n|+|u_{n+1}|+\cdots+|u_{n+p}| < \epsilon.$$
Par suite, pour $n \ge n_0$ et  $p \ge 0$ on a :
$$\big|u_n+u_{n+1}+\cdots+u_{n+p}\big| \le |u_n|+|u_{n+1}|+\cdots+|u_{n+p}| < \epsilon.$$
Donc, d'après le critère de Cauchy 
(théorème \ref{th:cauchyserie}), $\sum u_k$ est convergente.
\end{proof}


%---------------------------------------------------------------
\subsection{Règle du quotient de d'Alembert}

La règle du quotient de d'Alembert est un moyen efficace de montrer 
si une série de nombres réels ou complexes converge ou pas.

\begin{theoreme}[Règle du quotient de d'Alembert]
\label{th:regledalembert}
Soit $\sum u_k$ une série dont les termes généraux sont des nombres réels (ou complexes) non nuls.
\begin{enumerate}
\item S'il existe une constante $0<q<1$ et un entier $k_0$ tels que, pour tout
$k \ge k_0$,  
$$
\left|\frac{u_{k+1}}{u_k}\right| \le q <1,\quad \text{ alors }\quad\sum u_k\quad \text{converge.}
$$
La série est même absolument convergente.
\item S'il existe un entier $k_0$ tel que, pour tout $k \ge k_0$, 
$$
\left|\frac{u_{k+1}}{u_k}\right| \ge 1,\quad \text{ alors }\quad\sum u_k \quad\text{diverge.}
$$
\end{enumerate}
\end{theoreme}

Le plus souvent, la situation que l'on étudie est lorsque 
la suite $\frac{u_{k+1}}{u_k}$ converge ; la position
de la limite par rapport à $1$ détermine alors la nature de la série.

Voici une application directe et la plus utilisée, 
pour les séries de nombres réels, strictement positifs :
\begin{corollaire}[Règle du quotient de d'Alembert]
Soit $\sum u_k$ une série à termes strictement positifs, telle que 
$\frac{u_{k+1}}{u_k}$ converge vers $\ell$.
\begin{enumerate}
\item Si $\ell<1$ alors $\sum u_k$ converge.
\item Si $\ell>1$ alors $\sum u_k$ diverge.
\item Si $\ell=1$ on ne peut pas conclure en général.
\end{enumerate}
\end{corollaire}


\begin{proof}
Rappelons tout d'abord que la série géométrique $\sum q^k$
converge si $|q|<1$, diverge sinon.

Dans le premier cas du théorème, l'hypothèse $\left|\frac{u_{k+1}}{u_k}\right| \le q$
implique $|u_{k_0+1}| \le |u_{k_0}| q$, puis $|u_{k_0+2}| \le |u_{k_0}| q^2$.
On vérifie par récurrence que, pour tout $k\ge k_0$ :
$$|u_k| \le |u_{k_0}| q^{-k_0} \cdot q^k = c \cdot q^k,$$
où $c$ est une constante.
Comme $0 < q < 1$, alors la série $\sum q^k$ converge, d'où le résultat
par le théorème~\ref{th:comparaisonseries} de comparaison :
la série $\sum |u_k|$ converge.

\medskip

Si $\left|\frac{u_{k+1}}{u_k}\right| \ge 1$, la suite $(|u_k|)$ est croissante : elle ne
peut donc pas tendre vers $0$ et la série diverge.  
\end{proof}

\begin{exemple}
\begin{enumerate}  
  \item Pour tout $x \in \Rr$ fixé, la \defi{série exponentielle}
$$\sum_{k=0}^{+\infty} \frac{x^k}{k!}\quad \text{ converge.}$$

En effet pour $u_k = \frac{x^k}{k!}$ on a 
$$\left|\frac{u_{k+1}}{u_k}\right|
= \frac{\left|\frac{x^{k+1}}{(k+1)!}\right|}{\left|\frac{x^k}{k!}\right|}
=\frac{|x|}{k+1} \to 0 
\quad \text{lorsque } k \to +\infty.$$
La limite étant $\ell = 0 < 1$ alors par la règle du quotient de d'Alembert,
la série est absolument convergente, donc convergente.
Par définition la somme est $\exp(x)$ : 
$$\exp(x) = \sum_{k=0}^{+\infty} \frac{x^k}{k!}.$$


  \item $\sum_{k\ge0} \frac{k!}{1\cdot 3\,\cdots\,(2k-1)}$ converge,
  car $\frac{u_{k+1}}{u_k}=\frac{k+1}{2k+1}$ tend vers $\frac{1}{2}<1$.
  
  \item $\sum_{k\ge0} \frac{(2k)!}{(k!)^2}$ diverge, car 
  $\frac{u_{k+1}}{u_k}=\frac{(2k+1)(2k+2)}{(k+1)^2}$ tend vers $4>1$.
\end{enumerate}
\end{exemple}


\begin{remarque*}
\begin{itemize}
  \item Le théorème ne peut s'appliquer si certains $u_k$ sont nuls,
  contrairement à la règle des racines de Cauchy que l'on verra après.
  
  \item Notez bien que le théorème ne permet pas toujours de conclure.
  Faites aussi bien attention que l'hypothèse est 
  $\left|\frac{u_{k+1}}{u_k}\right| \le q <1$, ce qui est plus fort que
  $\left|\frac{u_{k+1}}{u_k}\right| <1$.
  
  \item De même le corollaire ne permet pas de conclure lorsque $\frac{u_{k+1}}{u_k} \to 1$.
  Par exemple pour les séries $\sum u_k= \sum \frac{1}{k}$ 
  et $\sum v_k = \sum \frac{1}{k^2}$ 
  nous avons $\frac{u_{k+1}}{u_k} = \frac{k}{k+1} \to 1$,
  de même que $\frac{v_{k+1}}{v_k} = \frac{k^2}{(k+1)^2 } \to 1$.
  Cependant la série $\sum \frac{1}{k}$ diverge 
  alors que $\sum \frac{1}{k^2}$ converge.
\end{itemize}
\end{remarque*}

Terminons par un exemple plus compliqué.

\begin{exemple}
Trouver tous les $z\in \Cc$ tels que la série 
$\sum_{k\ge0} \binom{k}{3} z^k$ soit absolument convergente.
\medskip

Soit $u_k=\binom{k}{3} z^k$. Alors, pour $z\neq0$,
$$\frac{|u_{k+1}|}{|u_k|}=\frac{\binom{k+1}{3}|z|^{k+1}}{\binom{k}{3}|z|^{k}}=
\frac{\frac{(k+1)k(k-1)}{3!}}{\frac{k(k-1)(k-2)}{3!}}|z|
=\frac{k+1}{k-2}|z|\to |z| \qquad \text{lorsque } k\to+\infty.$$ 


Si $|z|<1$ alors pour $k$ assez grand $\frac{|u_{k+1}|}{|u_k|} < q <1$
donc la série $\sum u_k$ est absolument convergente.

Si $|z|\ge 1$ alors $\frac{|u_{k+1}|}{|u_k|}=\frac{k+1}{k-2}|z|\ge \frac{k+1}{k-2}> 1$
pour tout $k$. Donc la série $\sum u_k$  diverge.

\end{exemple}


%---------------------------------------------------------------
\subsection{Règle des racines de Cauchy}

\begin{theoreme}[Règle des racines de Cauchy]
\label{th:reglecauchy}
Soit $\sum u_k$ une série de nombres réels ou complexes.
\begin{enumerate}
\item S'il existe une constante $0<q<1$ et un entier $k_0$ tels que, pour tout
$k \ge k_0$,  
$$
\sqrt[k]{|u_k|} \le q <1,\quad \text{ alors }\quad\sum u_k\quad \text{converge.}
$$
La série est même absolument convergente.
\item S'il existe un entier $k_0$ tel que, pour tout $k \ge k_0$, 
$$
\sqrt[k]{|u_k|} \ge 1,\quad \text{ alors }\quad\sum u_k \quad\text{diverge.}
$$
\end{enumerate}
\end{theoreme}



Le plus souvent vous l'appliquerez avec un terme général strictement positif.
\begin{corollaire}[Règle des racines de Cauchy]
Soit $\sum u_k$ une série à termes positifs, telle que 
$\sqrt[k]{u_k}$ converge vers $\ell$.
\begin{enumerate}
\item Si $\ell<1$ alors $\sum u_k$ converge.
\item Si $\ell>1$ alors $\sum u_k$ diverge.
\item Si $\ell=1$ on ne peut pas conclure en général.
\end{enumerate}
\end{corollaire}

Dans la pratique, il faut savoir bien manipuler les racines $k$-ème : 
$$\sqrt[k]{u_k} = (u_k)^{\frac1k} = \exp\left(\tfrac1k \ln u_k\right)$$.

\begin{proof}
Rappelons que la nature de la série ne dépend pas de ses premiers
termes. Dans le premier cas du théorème, $\sqrt[k]{|u_k|} \le q$
implique  $|u_k| \le q^k$. Comme $0<q<1$, alors la série $\sum q^k$ converge, d'où le résultat
par le théorème \ref{th:comparaisonseries} de comparaison.  

Dans le second cas, $\sqrt[k]{|u_k|} \ge 1$, donc $|u_k| \ge 1$.
Le terme général ne tend pas vers $0$, donc la série diverge.  

Enfin pour le dernier point du corollaire, on pose
$u_k=\frac{1}{k}$, $v_k=\frac{1}{k^2}$.
On a $\sqrt[k]{u_k}\to 1$ de même que $\sqrt[k]{v_k}\to 1$.
Mais $\sum u_k$ diverge alors que $\sum v_k$ converge.
\end{proof}

\begin{exemple}
\begin{enumerate}
  \item Par exemple, 
$$\sum \left(\frac{2k+1}{3k+4}\right)^k\quad \text{ converge,}$$
car $\sqrt[k]{u_k} = \frac{2k+1}{3k+4}$ tend vers $\frac{2}{3}<1$.
  
  
  \item Par contre  
  $$\sum \frac{2^k}{k^\alpha}\quad \text{ diverge,}$$
  quel que soit $\alpha >0$.
  En effet,
  $$\sqrt[k]{u_k} = \frac{\sqrt[k]{2^k}}{\big(\sqrt[k]{k}\big)^\alpha}
  = \frac{2}{\big(k^\frac{1}{k}\big)^\alpha}
  = \frac{2}{\big(\exp(\frac{1}{k}\ln k)\big)^\alpha} \to 2>1.$$
\end{enumerate}
\end{exemple}


\begin{exemple}
Déterminer tous les $z\in \Cc$ tels que la série  
$\sum_{k\ge1} \left( 1+\frac{1}{k}\right)^{k^2} z^k$ soit absolument convergente.
\medskip

Notons $u_k =\left( 1+\frac{1}{k}\right)^{k^2} z^k$.
On a $$\sqrt[k]{|u_k|}=\left( 1+\frac{1}{k}\right)^k |z| \to e|z|.$$

Cette limite vérifie $e|z| < 1$ si et seulement si $|z|<\frac{1}{e}$.

\begin{itemize}
  \item Si $|z|<\frac{1}{e}$ alors la série $\sum u_k$ est absolument 
convergente.
  
  \item Si $|z|>\frac{1}{e}$, on a pour $k$ assez grand
  $\sqrt[k]{|u_k|}>1$, donc la série $\sum u_k$ diverge.
  
  \item Si $|z|=\frac{1}{e}$ la règle des racines de Cauchy ne permet pas de conclure.
  On étudie le terme général à la main.
  On obtient:
  $$|u_k|=\left(  1+\frac{1}{k}\right)^{k^2}\left(\frac{1}{e}\right)^k$$
  Donc
  \begin{eqnarray*}
  \ln|u_k| 
  & = & k^2\ln\left(1+\tfrac{1}{k}\right)+k\ln \tfrac{1}{e} \\
  & = & k\left[k\ln (1+\tfrac{1}{k})-1\right] \\
  & = & k \left[ k\left(\tfrac{1}{k}-\tfrac{1}{2}\left( \tfrac{1}{k}\right)^2+
o\big( \tfrac{1}{k^2}\big) \right)-1\right] \\
  & = & k\left[1-\tfrac{1}{2}\tfrac{1}{k}+o\big( \tfrac{1}{k}\big) -1\right] \\
  & = & -\tfrac{1}{2} +o(1) \\
  & \to & -\tfrac{1}{2}\\
  \end{eqnarray*}

Donc $|u_k| \to e^{-\frac{1}{2}}\neq 0$. Ainsi $\sum |u_k|$ diverge. 
\end{itemize}

\end{exemple}




%---------------------------------------------------------------
\subsection{D'Alembert vs Cauchy}

Cette section peut être passée lors d'une première lecture.

\medskip

Nous allons comparer la règle du quotient de d'Alembert avec la règle des racines de Cauchy.
Nous allons voir que la règle des racines de Cauchy est plus puissante que la règle du quotient de d'Alembert.
Cependant dans la pratique la règle du quotient de d'Alembert reste la plus utilisée.


\begin{proposition}
\label{prop:cauchydalembert}
Soit $(u_k)$ une suite à termes strictement positifs.
$$
\text{Si } \quad \lim_{k\to+\infty} \frac{u_{k+1}}{u_k} = \ell
\qquad\text{alors}\qquad
\lim_{k\to+\infty} \sqrt[k]{u_k} = \ell\;.
$$
\end{proposition} 
Autrement dit, si on peut appliquer la règle du quotient de d'Alembert, alors on peut aussi appliquer 
la règle des racines de Cauchy.

\begin{proof}
Pour tout $\epsilon>0$, il existe $k_0$ tel que, pour tout $k\ge k_0$,
$$\ell-\epsilon < \frac{u_{k+1}}{u_k} < \ell+\epsilon\;.$$
Par récurrence, on en déduit :
$$u_{k_0}(\ell-\epsilon)^{k-k_0} \le u_k \le u_{k_0} (\ell+\epsilon)^{k-k_0}\;.$$
Or :
$$\lim_{k\to+\infty}\sqrt[k]{u_{k_0}(\ell-\epsilon)^{k-k_0}}
= \ell-\epsilon
\qquad\text{et}\qquad
\lim_{k\to+\infty}\sqrt[k]{u_{k_0}(\ell+\epsilon)^{k-k_0}}
= \ell+\epsilon\;.$$
Donc il existe $k_1>k_0$ tel que, pour $k>k_1$,
$$\ell-2\epsilon < \sqrt[k]{u_k} < \ell + 2\epsilon\;,$$
d'où le résultat.
\end{proof}

\bigskip
 
Terminons par un exemple où la règle des racines de Cauchy permet de conclure, 
mais pas la règle du quotient de d'Alembert.
\begin{exemple}
Définissons la suite $u_k$ par :
$$
u_k = \left\{
\begin{array}{ll}
\frac{2^n}{3^n}&\text{ si } k=2n\\[2ex]
\frac{2^n}{3^{n+1}}&\text{ si } k=2n+1
\end{array}
\right.
$$
Le rapport $\frac{u_{k+1}}{u_k}$ vaut $\frac{1}{3}$ si $k$ est pair,
$2$ si $k$ est impair. La règle du quotient de d'Alembert ne s'applique donc pas. 
Pourtant, $\sqrt[k]{u_k}$ converge vers $\sqrt{\frac{2}{3}}<1$,
donc la règle des racines de Cauchy s'applique et la série $\sum u_k$ converge.  
\end{exemple}



%---------------------------------------------------------------
\subsection{Règle de Raabe-Duhamel}

Cette section peut être passée lors d'une première lecture.

\medskip


La règle du quotient de d'Alembert et la règle des racines de Cauchy 
ne s'appliquent pas aux séries de Riemann  
$$\sum_{k\ge1} \frac{1}{k^\alpha}$$
car $\frac{k^\alpha}{(k+1)^\alpha}\to 1$
et $\sqrt[k]{u_k} \to 1$.

Il nous faut raffiner la règle de d'Alembert
pour pouvoir conclure. Cependant nous reviendrons 
sur la convergence des séries de Riemann par d'autres techniques.



\begin{theoreme}[Règle de Raabe-Duhamel]
Soit $(u_k)$ une suite de nombres réels (ou complexes) non nuls.
\begin{enumerate}
  \item Si $\forall k\geq k_0$ on a 
  $\left|\frac{u_{k+1}}{u_k}\right|\le 1-\frac{\beta}{k}$, avec $\beta>1$,
alors la série $\sum u_k$ est absolument convergente.
  \item Si $\forall k\geq k_0$ on a 
  $\left|\frac{u_{k+1}}{u_k}\right|\ge 1-\frac{1}{k}$, alors
 la série $\sum u_k$ n'est  pas absolument convergente.
\end{enumerate}
\end{theoreme}

\textbf{Attention !}
Il existe des séries convergentes, 
quoique $\left|\frac{u_{k+1}}{u_k}\right|\ge 1-\frac{1}{k}$.
Par le deuxième point une telle série ne peut pas être 
absolument convergente.

En effet, prenons $u_k=(-1)^k \frac{1}{k}$. Alors :
$$\frac{|u_{k+1}|}{|u_k|}= \frac{k}{k+1}= 1-\frac{1}{k+1}\ge 1-\frac{1}{k}.$$

\begin{proof}
\begin{enumerate}
  \item L'hypothèse implique $k|u_{k+1}| \le k|u_k| - \beta |u_k|$ (pour tout $k \ge k_0$).
  
Ainsi 
$$(\beta-1)|u_k| \le (k-1)|u_k|-k|u_{k+1}|.$$
Comme $\beta>1$ alors l'inégalité ci-dessus implique
$(k-1)|u_k|-k|u_{k+1}|>0$ et ainsi $(k-1)|u_k| > k|u_{k+1}|$.
La suite $(k |u_{k+1}|)_{k\ge k_0}$ est décroissante et minorée par $0$ ;
cette suite admet donc une limite.
Ainsi la série télescopique $\sum \big[(k-1)|u_k|-k|u_{k+1}|\big]$ converge. 
Comme 
$$(\beta-1)|u_k| \le (k-1)|u_k|-k|u_{k+1}|,$$
la série  $\sum(\beta-1)|u_k|$ converge et donc aussi $\sum |u_k|$.
  
  \item L'hypothèse implique $k|u_{k+1}| \ge (k-1)|u_k| >0$ (pour tout $k \ge k_0$).
Donc la suite $(k|u_{k+1}|)_{k \ge k_0}$ est croissante, ainsi
$k|u_{k+1}|\ge \epsilon>0$. Donc pour tout $k \ge k_0$, on a
$|u_{k+1}| \ge \frac{\epsilon}{k}$. 
Donc $\sum|u_k|$ diverge, car $\sum\frac{1}{k}$ diverge.
\end{enumerate}

\end{proof}


Nous pouvons maintenant savoir quelles sont les séries de Riemann
qui convergent.

\begin{proposition}[Séries de Riemann]
Soit $\alpha>0$. Alors la série $\sum_{k\ge1} \frac{1}{k^\alpha}$
converge si et seulement si $\alpha>1$.
\end{proposition}

\begin{proof}
Supposons $\alpha > 1$. 
Définissons $\Delta(k) = \frac{k^\alpha}{(k+1)^\alpha}$.
Montrons qu'il existe $\beta>1$  et $k_0$ tels que 
$$\Delta(k) \le 1-\frac{\beta}{k}  
\qquad \forall k \ge k_0.$$

Choisissons $\beta$ quelconque vérifiant $1 < \beta < \alpha$.
Considérons la fonction $f(x)= \frac{1}{(1+x)^\alpha} +\beta x.$

La fonction $f$ est $\mathcal{C}^\infty$ sur $[0,\infty[$ et 
$f(0)=1$.
Comme $f'(0)=\beta-\alpha<0$, on voit que $f$ est décroissante 
sur $[0, x_0]$ pour un certain $x_0$ avec  $0<x_0<1$.  
Ainsi $f(x)\le 1$ sur $[0,x_0]$ ce qui entraîne que
$\Delta(k)+\frac{\beta}{k}=f(\frac{1}{k}) \le 1$ pour $k \ge k_0$
avec $k_0$ entier tel que $\frac{1}{k_0} \le x_0$. Donc
$\Delta(k)\le 1-\frac{\beta}{k}$ et on peut appliquer la règle de Raabe-Duhamel pour 
déduire que $\sum \frac1{k^{\alpha}}$ converge.

\medskip

Si $0 < \alpha \le 1$, alors $\frac{1}{k} \le \frac{1}{k^\alpha}$.
Or la série $\sum \frac{1}{k}$ diverge donc la série 
$\sum\frac{1}{k^\alpha}$ diverge aussi.
\end{proof}


%---------------------------------------------------------------
%\subsection{Mini-exercices}

\begin{miniexercices}
\begin{enumerate}

  \item Est-ce que les séries suivantes sont convergentes ?
  Absolument convergentes ?
  
  $$\sum_{k\geq 2} \frac{(-1)^{k^3}e^{\ii k}}{k^2+k}
  \qquad
  \sum_{k\geq 1} \frac{(-1)^k}{\sqrt{k(k+1)}}
  \qquad
  \sum_{k\geq 1} \frac{\sqrt k}{(-1)^k\ln k}$$
  
  \item \'Etudier les séries dont voici le terme général, par la règle du quotient de d'Alembert 
  ou des racines de Cauchy :
  $$\frac{k^{100}}{k!} \qquad \frac{k!}{(2k)!} \qquad 
  \frac{\ln k}{2^k+1} \qquad \frac{2\cdot 4 \cdot 6 \cdots (2k)}{k^k}$$
  $$\big(\sin \tfrac1k\big)^k \qquad 
  \left(\frac{7k-2}{3k+1}\right)^k \qquad \frac{2^k}{e^k-1}$$
  
  
  \item Appliquer la règle du quotient de d'Alembert pour $u_k = \frac{k!}{k^k}$.
  En déduire la limite de $\sqrt[k]{u_k}$ lorsque $k$ tend vers $+\infty$.
  
  \item \'Etudier les séries dont voici le terme général en 
  fonction du paramètre $\alpha > 0$ :
  $$\frac{k}{k^{\alpha}+1} \qquad \frac{\ln k}{k^\alpha} \qquad \sqrt{k}\alpha^k \qquad 
  \frac{\alpha^k}{k^2} \qquad \ln (1 + k^\alpha) 
  \qquad \frac{2\cdot 4 \cdot 6 \cdots (2k)}{3\cdot 5 \cdot 7 \cdots (2k+1)}
  $$
\end{enumerate}
\end{miniexercices}



%%%%%%%%%%%%%%%%%%%%%%%%%%%%%%%%%%%%%%%%%%%%%%%%%%%%%%%%%%%%%%%%
\section{Comparaison série/intégrale}


Cette section fait la jonction entre les séries et les intégrales impropres.
C'est un lien essentiel entre deux objets mathématiques qui sont au final assez proches.
Pour cette partie il faut connaître les intégrales impropres $\int_0^{+\infty} f(t) \;\dd t$.


%---------------------------------------------------------------
\subsection{Théorème de comparaison série/intégrale}

\begin{theoreme}
\label{th:serieintegrale}
Soit $f : [0,+\infty[ \to [0,+\infty[$ une fonction décroissante.
Alors la série $\sum_{k \ge 0} f(k)$ (dont le terme général est $u_k = f(k)$) 
et l'intégrale impropre $\int_0^{+\infty} f(t) \; \dd t$ sont de même nature.
\end{theoreme}

<<~De même nature~>> signifie que la série et l'intégrale du théorème sont soit
convergentes en même temps, soit divergentes en même temps.

\textbf{Attention !} Il est important que $f$ soit positive et 
décroissante.

%---------------------------------------------------------------
\subsection{Preuve}


Le plus simple est de bien comprendre le dessin et de refaire la démonstration
chaque fois que l'on en a besoin.


\begin{proof}
Soit $k\in\Nn$. Comme $f$ est décroissante, pour $k\le t \le k+1$, 
on a $f(k+1)\le f(t)\le f(k)$ (attention à l'ordre). En intégrant 
sur l'intervalle $[k,k+1]$ de longueur $1$, on obtient:

\begin{minipage}{0.39\textwidth}
$$f(k+1)\le \int_k^{k+1} f(t) \; \dd t \le f(k)$$  
\end{minipage}
\begin{minipage}{0.49\textwidth}
\myfigure{1.5}{
\tikzinput{fig_series02} 
}  
\end{minipage}

Sur le dessin cette inégalité signifie que l'aire sous la courbe, 
entre les abscisses $k$ et $k+1$, est comprise entre l'aire du rectangle
vert de hauteur $f(k+1)$ et de base $1$ et l'aire du rectangle bleu de hauteur
$f(k)$ et de même base $1$.



On somme ces inégalités pour $k$ variant de $0$ à $n-1$ :
$$\sum_{k=0}^{n-1} f(k+1) \le \sum_{k=0}^{n-1} \int_{k}^{k+1} f(t) \; \dd t
\le \sum_{k=0}^{n-1} f(k).$$
Soit :
$$u_1+\cdots+u_{n} \le \int_0^{n} f(t)\; \dd t \le u_0+\cdots+u_{n-1}.$$

La série $\sum u_k$ converge et a pour somme $S$ 
si et seulement si la suite des sommes
partielles converge vers $S$.  
Si c'est le cas $\int_0^{n} f(t)\; \dd t$ est
majorée par $S$, et comme $\int_0^x f(t)\; \dd t$ est une fonction croissante de $x$
(par positivité de $f$), l'intégrale converge. Réciproquement, si l'intégrale
converge, alors $\int_0^{n} f(t)\;\dd t$ est majorée, la suite des
sommes partielles aussi, et la série converge.
\end{proof}


%---------------------------------------------------------------
\subsection{Séries de Riemann}


Le théorème de comparaison (théorème \ref{th:comparaisonseries}) et
le théorème des équivalents (théorème \ref{th:equivalentseries}) 
permettent de ramener l'étude des séries à termes positifs à un catalogue de séries 
dont la convergence est connue. Dans ce catalogue, on trouve les séries de Riemann 
et les séries de Bertrand.


Commençons par les \defi{séries de Riemann} $\sum_{k\ge 1} \frac{1}{k^\alpha}$, pour $\alpha>0$
un réel.

\begin{proposition}
\mybox{
Si \quad $\alpha >1$ \quad alors 
\quad $\displaystyle  \sum_{k = 1}^{+\infty} \frac{1}{k^\alpha}$ \quad converge}  
\mybox{
Si \quad $0 <\alpha \le1$ \quad alors 
\quad $\displaystyle  \sum_{k \ge 1} \frac{1}{k^\alpha}$ \quad diverge}
\end{proposition}

\begin{proof}
Dans le théorème \ref{th:serieintegrale}, rien n'oblige à démarrer de $0$ : pour $m\in\Nn$,
la série $\sum_{k \ge m} f(k)$ 
et l'intégrale impropre $\int_m^{+\infty} f(t) \;\dd t$ sont de même nature.

Nous l'appliquons à $f : [1,+\infty[ \to [0,+\infty[$ définie par 
$f(t)=\frac{1}{t^\alpha}$. Pour $\alpha >0$, c'est une fonction décroissante et positive.
On peut appliquer le théorème \ref{th:serieintegrale}.

On sait que :
$$
\int_1^{x} \frac{1}{t^\alpha}\;\dd t =
\left\{\begin{array}{ll}
\displaystyle{\frac{1}{1-\alpha}(x^{1-\alpha}-1)}&\mbox{si }\alpha\neq
    1\\[1.5ex]
\ln(x) &\mbox{si }\alpha=1
\end{array}\right. 
$$

Pour $\alpha > 1$, $\int_1^{+\infty} \frac{1}{t^\alpha} \; \dd t $ 
est convergente, donc la série $\sum_{k = 1}^{+\infty} \frac{1}{k^\alpha}$ converge.

Pour $0<\alpha\le 1$, $\int_1^{+\infty} \frac{1}{t^\alpha} \; \dd t $ 
est divergente, donc la série $\sum_{k \ge 1} \frac{1}{k^\alpha}$ diverge.
\end{proof}


%---------------------------------------------------------------
\subsection{Séries de Bertrand}


Une famille de séries plus sophistiquées sont les \defi{séries de Bertrand} : 
$\sum_{k\ge2} \frac{1}{k^\alpha(\ln k)^\beta}$
où $\alpha > 0$ et $\beta \in \Rr$.

\begin{proposition}
Soit la série de Bertrand 
$$\sum_{k\ge2} \frac{1}{k^\alpha(\ln k)^\beta}.$$
\mybox{Si \quad $\alpha>1$ \quad  alors elle converge. \qquad\qquad Si \quad $0<\alpha<1$ \quad  alors elle diverge.}

\mybox{Si \quad $\alpha=1$ \quad et \quad $\begin{cases} \beta>1 & \text{alors elle converge.}\\   
\beta\le 1 & \text{alors elle diverge.}
\end{cases}$}
\end{proposition}


\begin{proof}
La démonstration est la même que pour les séries de Riemann.
Par exemple pour le cas $\alpha=1$ : 
$$
\int_2^{x} \frac{1}{t(\ln t)^{\beta}}\;\dd t =
\left\{\begin{array}{ll}
\displaystyle{\frac{1}{1-\beta}\left((\ln x)^{1-\beta}-(\ln 2)^{1-\beta}\right)}&\text{si }\beta\neq
    1\\[1.5ex]
\ln(\ln x)-\ln(\ln 2) &\text{si }\beta=1
\end{array}\right. 
$$
\end{proof}


%---------------------------------------------------------------
\subsection{Applications}


Nous retrouvons en particulier le fait que :
\begin{enumerate}
  \item $\sum \frac{1}{k^2}$ converge (prendre $\alpha=2$),
  
  \item alors que $\sum \frac{1}{k}$ diverge (prendre $\alpha=1$). 
\end{enumerate}



\bigskip

Terminons avec deux exemples d'utilisation des équivalents 
avec les séries de Riemann et de Bertrand.

\begin{exemple}
\begin{enumerate}
  \item La série 
  $$\sum_{k \ge 1} \ln\left(1+\frac{1}{\sqrt{k^3}}\right)$$ 
  est-elle convergente ?
  
  Comme
  $$\ln\left(1+\frac{1}{\sqrt{k^3}}\right) \quad \sim \quad \frac{1}{\sqrt{k^3}}$$
  et que la série de Riemann $\sum \frac{1}{\sqrt{k^3}}=\sum \frac{1}{k^{\frac32}}$ 
  converge (car $\frac32>1$) alors par le théorème des équivalents la série 
  $\sum_{k=1}^{+\infty} \ln\left(1+\frac{1}{\sqrt{k^3}}\right)$
  converge également.
  
  \item La série
$$\sum_{k \ge 1} 
\frac{1-\cos\left(\frac{1}{k\sqrt{\ln k}}\right)}{\sin\left(\frac{1}{k}\right)}$$
  est-elle convergente ?
  
On cherche un équivalent du terme général (qui est positif) :  
$$\frac{1-\cos\left(\frac{1}{k\sqrt{\ln k}}\right)}
{\sin\left(\frac{1}{k}\right)} \quad \sim \quad \frac{1}{2k\ln k}$$
Or la série de Bertrand $\sum \frac{1}{k\ln k}$ diverge, donc
notre série diverge aussi.
\end{enumerate}

\end{exemple}





%---------------------------------------------------------------
%\subsection{Mini-exercices}

\begin{miniexercices}
\begin{enumerate}
  \item Notons $H_n = \sum_{k=1}^{n} \frac 1k$ la somme partielle de la série harmonique.
  Et soit $f : [1,+\infty[\to[0,+\infty[$ définie par $f(t)=\frac1t$.
  \begin{enumerate}
    \item Donner un encadrement simple de $\int_k^{k+1} f(t) \;\dd t$.
    \item Faire la somme de ces inégalités pour $k$ variant de $1$ à $n-1$, puis $k$ variant de $1$
  à $n$, pour obtenir :
 $$\ln(n+1) \le H_n \le1 + \ln n$$
    \item En déduire $H_n \sim \ln n$.
    \item La série harmonique converge-t-elle ?
   \end{enumerate}
  
  \item Reprendre le schéma d'étude précédent pour montrer que, pour la série de Riemann 
  et $0\le \alpha <1$,
  $$\sum_{k=1}^{n} \frac{1}{k^\alpha} \sim \frac{n^{1-\alpha}}{1-\alpha}.$$
  
  \item Reprendre le schéma d'étude précédent, mais cette fois pour le reste
  $R_n = \sum_{k=n+1}^{\infty} \frac{1}{k^2}$, afin de montrer que
  $$R_n \sim \frac{1}{n}.$$
  Calculer $R_{100}$. Quelle approximation cela fournit-il de la somme de la série ?
  
  \item \'Etudier la convergence des séries suivantes en fonction des paramètres $\alpha>0$ et $\beta \in \Rr$ :
  $$\sum \sqrt{k^\alpha+1}-\sqrt{k^\alpha} \qquad
  \sum \sin\left(\frac{k^\alpha}{\ln k}\right) \qquad
  \sum \ln \left(1+\frac{1}{k(\ln k)^\beta}\right)$$
  
  
\end{enumerate}
\end{miniexercices}




%%%%%%%%%%%%%%%%%%%%%%%%%%%%%%%%%%%%%%%%%%%%%%%%%%%%%%%%%%%%%%%%
\section{Produits de deux séries}

Cette section consacrée au produit de deux séries 
peut être passée lors d'une première lecture.

%---------------------------------------------------------------
\subsection{Motivation}

Pour un produit de sommes, il y a plusieurs façons d'ordonner les termes 
une fois le produit développé. Dans le cas d'une somme finie l'ordre des termes n'a pas d'importance, 
mais dans le cas d'une série c'est essentiel.
On choisit de regrouper les termes en fonction des indices, de la façon suivante :

$$\big(a_0+a_1\big)\big(b_0+b_1\big)=
\underbrace{a_0b_0}_{\text{somme des indices}=0}
+\underbrace{a_0b_1 + a_1b_0}_{\text{somme des indices}=1}
+\underbrace{a_1b_1}_{\text{somme des indices}=2}$$
\begin{align*}
\big(a_0+a_1+a_2\big)\big(b_0+b_1+b_2\big) & =
\underbrace{a_0b_0}_{\text{somme des indices}=0}
+\underbrace{a_0b_1 + a_1b_0}_{\text{somme des indices}=1} \\
& +\underbrace{a_0b_2+a_1b_1+a_2b_0}_{\text{somme des indices}=2}
+\underbrace{a_1b_2+a_2b_1}_{\text{somme des indices}=3}
+\underbrace{a_2b_2}_{\text{somme des indices}=4}  
\end{align*}




Plus généralement, voici différentes façons d'écrire un produit de deux sommes :
$$\left(\sum_{i=0}^n a_i\right)\; \left(\sum_{j=0}^n b_j\right)= 
\sum_{i=0}^n \sum_{j=0}^n a_ib_j
= \sum_{0 \le k \le 2n} \sum_{i+j=k} a_ib_j
= \sum_{0 \le k \le 2n} \sum_{0 \le i \le k} a_ib_{k-i}.$$
Les deux dernières formes correspondent à notre décomposition en fonction de la somme des indices.



%---------------------------------------------------------------
\subsection{Le produit de Cauchy}


\begin{definition}
Soient $\sum_{i \ge 0} a_i$ et $\sum_{j \ge 0} b_j$ deux séries.
On appelle \defi{série produit} ou \defi{produit de Cauchy}
la série $\sum_{k \ge 0} c_k$
o\`u \mybox{$\displaystyle c_k=\sum_{i=0}^k a_i b_{k-i}$} 
\end{definition}

Une autre façon d'écrire le coefficient $c_k$ est :
\mybox{$\displaystyle c_k=\sum_{i+j=k} a_i b_j$}

\begin{theoreme}
Si les séries $\sum_{i=0}^{+\infty} a_i$ et $\sum_{j=0}^{+\infty} b_j$ de nombres réels (ou complexes)
sont absolument convergentes, alors la série produit 
$$\sum_{k=0}^{+\infty} c_k = \sum_{k=0}^{+\infty} \left(\sum_{i=0}^k a_ib_{k-i}\right)$$
est absolument convergente et l'on a:
$$\sum_{k=0}^{+\infty} c_k = \left(\sum_{i=0}^{+\infty} a_i\right)\ \times\ \left(\sum_{j=0}^{+\infty} b_j\right).$$
\end{theoreme}

\begin{proof}
\textbf{Notations.}

\begin{itemize}
  \item $S_n=a_0+\dots+a_n$, $S_n\to S$,
  
  \item $T_n=b_0+\dots +b_n$, $T_n\to T$,
  
  \item $P_n= c_0+\dots+c_n$.
\end{itemize}

On doit montrer que $P_n \to S\cdot T$.

\medskip
\textbf{Premier cas.} $a_k\ge 0, b_k \ge 0$ ($\forall k$). 

Dans ce cas $c_k\ge 0$ et on a 
$$P_n \le S_n \cdot T_n\le S \cdot T.$$ 

La suite $(P_n)$ est croissante et majorée, donc convergente : $P_n\to P$.

Or on a aussi 
$$P_n \le S_n \cdot T_n \le P_{2n}.$$

\myfigure{1}{
\tikzinput{fig_series03} 
}
Le dessin représente le point correspondant aux indices $(i,j)$.
Le triangle rouge représente  $P_n$ (avec le regroupement des termes correspondant aux diagonales),
le carré vert correspond au produit $S_n \cdot T_n$,
le triangle bleu représente  $P_{2n}$. Le fait que le carré soit compris entre les deux triangles
traduit la double inégalité $P_n \le S_n \cdot T_n \le P_{2n}$.

\medskip

Donc en faisant $n\to+\infty$, on a: $P\le S \cdot T\le P$. Donc $P_n\to S \cdot T$.

\medskip
\textbf{Second cas.} $a_k\in\Cc, b_k \in \Cc$ ($\forall k$).

On pose :
\begin{itemize}
  \item $S_n'=|a_0|+\dots+|a_n|$, $S_n'\to S'$,
  
  \item $T_n'=|b_0|+\dots+|b_n|$, $T_n'\to T'$,
  
  \item $P_n'= c_0'+\dots+ c_n'$ o\`u $c_k'=\sum_{i=0}^k|a_ib_{k-i}|$.
\end{itemize}


D'après le premier cas, $P_n'\to P'$ avec $P'=S' \cdot T'$. Ainsi

$$|S_n \cdot T_n - P_n|= \bigl|\sum_{\stackrel{0\le i,j\le n}{i+j>n}} a_ib_j\bigr|\le
\sum_{\stackrel{0\le i,j\le n}{i+j>n}} |a_ib_j| = S_n' \cdot T_n'-P_n' \to S' \cdot T'-P'=0.
$$

Ainsi $P_n=S_n \cdot T_n-(S_n \cdot T_n-P_n)\to S \cdot T-0=S \cdot T$.

Donc la série $\sum c_k$ est convergente et sa somme est $S \cdot T$. 
De plus, $|c_k|\le c_k'$. La convergence de $\sum c_k'$ implique 
donc la convergence absolue de $\sum c_k$.
\end{proof}

%---------------------------------------------------------------
\subsection{Exemple}

\begin{exemple}
Soit $\sum_{i=0}^{+\infty} a_i$ une série absolument convergente
et soit $\sum_{j=0}^{+\infty} b_j$ la série définie par $b_j = \frac{1}{2^j}$.
La série $\sum b_j$ est absolument convergente.

Notons
$$c_k = \sum_{i=0}^k a_ib_{k-i} = \sum_{i=0}^k a_i \times \frac{1}{2^{k-i}}.$$

Alors la série $\sum c_k$ converge absolument et
$$\sum_{k=0}^{+\infty} c_k 
= \left(\sum_{i=0}^{+\infty} a_i\right)\ \times\ \left(\sum_{j=0}^{+\infty} b_j\right)
= 2 \sum_{i=0}^{+\infty} a_i.$$
\end{exemple}

%---------------------------------------------------------------
\subsection{Contre-exemple}

Si les séries $\sum a_i$ et $\sum b_j$ ne sont pas absolument 
convergentes, mais seulement convergentes,
alors la série de Cauchy peut être divergente.

\begin{exemple}
Soient $a_i=b_i=\frac{(-1)^{i}}{\sqrt {i+1}}, i\ge 0$.
Alors $\sum a_i$ et $\sum b_j$ sont convergentes par le critère de Leibniz, 
mais ne sont pas absolument convergentes. On a
$$c_k = \sum_{i=0}^k a_i b_{k-i} 
= \sum_{i=0}^k \frac{(-1)^{i}}{\sqrt {i+1}} \frac{(-1)^{k-i}}{\sqrt{k-i+1}} 
= (-1)^{k} \sum_{i=0}^k \frac{1}{\sqrt{(i+1)(k-i+1)}}$$

Or, pour $x\in\Rr$, $(x+1)(k-x+1)=-x^2+kx+(k+1) \le \frac{(k+2)^2}{4}$ (valeur au sommet de la parabole). 
D'o\`u $\sqrt{(i+1)(k-i+1)} \le \frac{(k+2)}{2}$.
Ainsi
$$|c_k|=\sum_{i=0}^k\frac{1}{\sqrt{(i+1)(k-i+1)}} \ge \sum_{i=0}^k \frac{2}{k+2} = \frac{2(k+1)}{k+2} \to 2.$$
Donc le terme général $c_k$ ne peut pas tendre $0$, donc la série $\sum c_k$ diverge.
\end{exemple}


%---------------------------------------------------------------
%\subsection{Mini-exercices}

\begin{miniexercices}
\begin{enumerate}
  \item Trouver une expression simple du terme général de la série produit
  $$\sum_{i=0}^{+\infty} \frac{1}{3^i} \times \sum_{j=0}^{+\infty} \frac{1}{3^j}.$$
  Calculer la somme de cette série produit.
  \item On admet ici que, pour $x\in \Rr$, la série $\sum_{k=0}^{+\infty} \frac{x^k}{k!}$ converge et vaut
  $\exp(x)$.
  Que vaut la série produit associée à $\exp(a) \times \exp(b)$ ?
  (Vous utiliserez la formule du binôme de Newton.)
\end{enumerate}
\end{miniexercices}

%%%%%%%%%%%%%%%%%%%%%%%%%%%%%%%%%%%%%%%%%%%%%%%%%%%%%%%%%%%%%%%%
\section{Permutation des termes}


Cette section consacrée à la permutation de termes
peut être passée lors d'une première lecture.

\begin{theoreme}
Soit $\sum_{k=0}^{+\infty} u_k$ une série absolument convergente et soit $S$ sa somme.
Soit $\sigma : \Nn \to \Nn$ une bijection de l'ensemble des indices.
Alors la série $\displaystyle \sum_{k=0}^{+\infty} u_{\sigma(k)}$ converge
et $$\sum_{k=0}^{+\infty} u_{\sigma(k)} = S.$$
\end{theoreme}

Remarque : la condition de convergence absolue est indispensable.
Il se trouve que, pour une série convergente, mais pas absolument convergente, 
on peut permuter les termes pour obtenir n'importe quelle valeur !

Comme exemple de permutation, on peut réordonner les termes $u_0,u_1,u_2,u_3,\ldots$ 
en prenant deux termes de rang pair puis un terme de rang impair, ce qui donne :
$$u_0, u_2, u_1, u_4, u_6, u_3, u_8, u_{10}, u_5,\ldots$$
Par contre il \emph{n'est pas autorisé} de regrouper tous les termes pairs d'abord et 
les termes impairs ensuite :
$$u_0,u_2, u_4,\ldots, u_{2k},\ldots,u_1,u_3,\ldots, u_{2k+1},\ldots$$


\begin{proof}
Par hypothèse $\sum_{k=0}^{+\infty} |u_k|$ converge. 
D'après le critère de Cauchy,
$$\forall \epsilon>0\quad \exists n_0\in\Nn \qquad \sum_{n=n_0+1}^{+\infty} |u_k|<\epsilon.$$
Soit $S=\sum_{k=0}^{+\infty} u_k$. Fixons $\epsilon>0$. 
Choisissons $k_0\in\Nn$ tel que 
$\big\{0, 1,2,\dots,n_0\big\} \subset \big\{\sigma(0),\sigma(1),\ldots, \sigma(k_0)\big\}$.  
Pour $n\geq k_0$ on a :
$$\left|S-\sum_{k=0}^n u_{\sigma(k)}\right| \le
\left|S-\sum_{k=0}^{n_0} u_k\right|+\left|\sum_{k=0}^{n_0} u_k- 
\sum_{k=0}^n u_{\sigma(k)}\right|$$
Pour le premier terme on a 
$$\left|S-\sum_{k=0}^{n_0} u_k\right| = \left| \sum_{k=n_0+1}^{+\infty} u_k\right| 
\le \sum_{k=n_0+1}^{+\infty} |u_k|
\le \epsilon.$$

Pour le second terme :
$$\left|\sum_{k=0}^n u_{\sigma(k)}-\sum_{k=0}^{n_0} u_k\right|
= \left|\sum_{k \in \{\sigma(0),\ldots,\sigma(n)\}\setminus\{0,\ldots,n_0\}} u_k \right|
\le \sum_{k \in \{\sigma(0),\ldots,\sigma(n)\}\setminus\{0,\ldots,n_0\}} |u_k| 
\le \sum_{k > n_0} |u_k| 
= \sum_{k=n_0+1}^{+\infty} |u_k|
\le \epsilon.$$
Ce qui prouve $\left|S-\sum_{k=0}^n u_{\sigma(k)}\right| \le 2\epsilon$ et donne le résultat.
\end{proof}



%---------------------------------------------------------------
%\subsection{Mini-exercices}


\begin{miniexercices}
Le but de cet exercice est de comprendre que si la série n'est pas absolument convergente,
des phénomènes étranges apparaissent.
Souvenez-vous que la série harmonique alternée converge :
$$\sum_{k=0}^{+\infty} (-1)^{k} \frac{1}{k+1} 
= 1-\frac{1}{2}+\frac{1}{3}-\frac{1}{4} +\cdots $$
Notons $S$ sa somme. (En fait $S = \ln 2$.)

Si on regroupe les termes de cette série par paquets de $3$, et si l'on simplifie, alors
on trouve la moitié de la somme !


\begin{align*}
& \left(1-\frac{1}{2}-\frac{1}{4} \right)+
  \left(\frac{1}{3}-\frac{1}{6}-\frac{1}{8} \right)+
  \left(\frac{1}{5}-\frac{1}{10}-\frac{1}{12} \right)+\cdots+ 
  \left(\frac{1}{2k-1}-\frac{1}{4k-2}-\frac{1}{4k} \right)+\cdots   \\
= &  \left(\frac{1}{2}-\frac{1}{4} \right)+
  \left(\frac{1}{6}-\frac{1}{8} \right)+
  \left(\frac{1}{10}-\frac{1}{12} \right)+\cdots+ 
  \left(\frac{1}{4k-2}-\frac{1}{4k} \right)+\cdots   \\ 
= & \frac12  \left( 1-\frac{1}{2}+\frac{1}{3}-\frac{1}{4} + \cdots \right) \\
= & \frac12 S
\end{align*}

Surprenant, non ?

\end{miniexercices}


%%%%%%%%%%%%%%%%%%%%%%%%%%%%%%%%%%%%%%%%%%%%%%%%%%%%%%%%%%%%%%%%
\section{Sommation d'Abel}

Cette section consacrée à la sommation d'Abel
peut être passée lors d'une première lecture.

%---------------------------------------------------------------
\subsection{Théorème de sommation d'Abel}


Le théorème de sommation d'Abel s'applique à certaines séries
convergentes mais qui ne sont pas absolument convergentes.
C'est un théorème qui s'applique aux séries de la forme $\sum a_kb_k$
et est plus fort que le critère de Leibniz pour les séries alternées,
mais il est aussi plus difficile à mettre en \oe uvre.

\begin{theoreme}[Théorème de sommation d'Abel]
\label{th:abelserie}
Soient $(a_k)_{k\ge0}$ et $(b_k)_{k\ge0}$ deux suites telles que :
\begin{enumerate}
\item La suite $(a_k)_{k\ge0}$ est une suite décroissante de
réels positifs qui tend vers $0$.
\item Les sommes partielles de la suite $(b_k)_{k\ge0}$ sont bornées :
$$\exists M\quad\forall n\in \Nn\qquad
\big|b_0+\cdots+b_n\big|\le M.$$
Alors la série $\sum_{k\ge0} a_kb_k$ converge.
\end{enumerate}
\end{theoreme}

Le critère de Leibniz concernant les séries alternées est un cas spécial :
en effet, si $b_k=(-1)^k$ alors $\big|\sum_{k=0}^n b_k\big|\le 1$.
Donc si $(a_k)$ est une suite positive, décroissante, qui tend vers $0$, alors 
$\sum a_k b_k$ converge.


\begin{proof}
L'idée de la démonstration est d'effectuer un changement dans la
sommation, qui s'apparente à une intégration par parties. Pour
tout $n\ge0$, posons $B_n=b_0+\cdots+b_n$. Par hypothèse, la suite
$(B_n)$ est bornée. Nous écrivons les sommes partielles de la
série $\sum a_k b_k$ sous la forme suivante :
$$
\begin{array}{rcl}
S_n & = & a_0b_0+a_1b_1+\cdots+a_{n-1}b_{n-1}+a_nb_n \\
&=& a_0B_0+a_1(B_1-B_0)+\cdots +a_{n-1}(B_{n-1}-B_{n-2}) + a_n(B_n-B_{n-1})\\
&=& B_0(a_0-a_1)+B_1(a_1-a_2)+\cdots+B_{n-1}(a_{n-1}-a_n)+B_na_n\;.
\end{array}
$$
Comme $(B_n)$ est bornée, et $a_n$ tend vers $0$, le dernier terme $B_na_n$
tend vers $0$. Nous allons montrer que la série $\sum B_k(a_k-a_{k+1})$ 
est absolument convergente. En effet,
$$
\big|B_k(a_k-a_{k+1})\big| = \big|B_k\big|(a_k-a_{k+1})\le M(a_k-a_{k+1})\;,
$$
car la suite $(a_k)$ est une suite de réels positifs,
décroissante, et $|B_k|$ est borné par $M$. Or
$$
M(a_0-a_1)+\cdots+M(a_n-a_{n+1}) = M(a_0-a_{n+1})\;,
$$
qui tend vers $Ma_0$ puisque $(a_k)$ tend vers $0$.
La série $\sum M(a_k-a_{k+1})$ converge, donc la série
$\sum\big|B_k(a_k-a_{k+1})\big|$ aussi, par le théorème 
\ref{th:comparaisonseries} de comparaison. Donc la série $\sum B_k(a_k-a_{k+1})$
est convergente, donc la suite $(S_n)$ est convergente, ce qui prouve que la série 
$\sum a_k b_k$ converge.
\end{proof}


%---------------------------------------------------------------
\subsection{Séries de Fourier}

Le cas d'application le plus fréquent est celui où
$b_k=e^{\ii k\theta}$.

\begin{corollaire}
Soit $\theta$ un réel, tel que $\theta \neq 2n\pi$ (pour tout $n \in\Zz$). 
Soit $(a_k)$ une suite de réels positifs, décroissante,
tendant vers $0$. Alors les \defi{séries de Fourier} :
\mybox{$\displaystyle 
\sum a_k e^{\ii k\theta} \qquad 
\sum a_k\cos(k\theta) \qquad 
\sum a_k\sin(k \theta)
\quad \text{ convergent}
$} 
\end{corollaire}




\begin{proof}
Pour appliquer le théorème de sommation d'Abel (théorème \ref{th:abelserie}) avec 
$b_k=e^{\ii k\theta}$, nous devons
vérifier que les sommes partielles de la suite $(e^{\ii k\theta})$ sont
bornées. Or $e^{\ii k\theta}=(e^{\ii \theta})^k$, et par hypothèse 
$e^{\ii \theta}$ est différent de $1$. On
a donc la somme d'une suite géométrique :
$$
\big|1+e^{\ii \theta}+\cdots+e^{\ii k\theta}\big| =
 \left|\frac{1-e^{\ii (k+1)\theta}}{1-e^{\ii \theta}}\right|\le
\left|\frac{2}{1-e^{\ii \theta}}\right|\;.$$
D'où le résultat. 

Comme $\sum a_k e^{\ii k\theta} = \sum a_k\cos(k\theta) + \ii \sum a_k\sin(k \theta)$,
la convergence des séries
$\sum a_k\cos(k\theta)$ et $\sum a_k\sin(k \theta)$ est une
conséquence directe de la proposition \ref{prop:reimseries}.
\end{proof}

%---------------------------------------------------------------
%\subsection{Mini-exercices}


\begin{miniexercices}
\begin{enumerate}
  \item Justifier que les sommes $\sum_{k=0}^n (-1)^k$, $\sum_{k=0}^n \cos(k\theta)$
  et $\sum_{k=0}^n \sin(k\theta)$ sont bornées, pour $\theta \neq 0 \pmod{2\pi}$.
 
  \item Montrer que les séries suivantes convergent par le critère de sommation d'Abel :
  $$\sum \frac{(-1)^k \cos k}{k} \qquad \sum \frac{\sqrt{k+1}}{k} \sin (k\theta)
  \qquad \frac{1}{e^{\ii k \theta}\ln k}$$
  pour $\theta \neq 0 \pmod{2\pi}$.
\end{enumerate}
\end{miniexercices}






\auteurs{
\begin{itemize}
  \item[$\bullet$] D'après un cours de Raymond Mortini, de l'université de Lorraine,

  \item[$\bullet$] et un cours de Luc Rozoy et Bernard Ycart de l'université de Grenoble
  pour le site \texttt{\href{http://ljk.imag.fr/membres/Bernard.Ycart/mel/}{M\at ths en Ligne}}.
 
  \item[$\bullet$] mixé, révisé par Arnaud Bodin. Relu par Stéphanie Bodin et Vianney Combet.
\end{itemize}
}

\finchapitre
\end{document}

