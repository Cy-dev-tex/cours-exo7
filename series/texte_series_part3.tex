
%%%%%%%%%%%%%%%%%% PREAMBULE %%%%%%%%%%%%%%%%%%


\documentclass[12pt]{article}

\usepackage{amsfonts,amsmath,amssymb,amsthm}
\usepackage[utf8]{inputenc}
\usepackage[T1]{fontenc}
\usepackage[francais]{babel}


% packages
\usepackage{amsfonts,amsmath,amssymb,amsthm}
\usepackage[utf8]{inputenc}
\usepackage[T1]{fontenc}
%\usepackage{lmodern}

\usepackage[francais]{babel}
\usepackage{fancybox}
\usepackage{graphicx}

\usepackage{float}

%\usepackage[usenames, x11names]{xcolor}
\usepackage{tikz}
\usepackage{datetime}

\usepackage{mathptmx}
%\usepackage{fouriernc}
%\usepackage{newcent}
\usepackage[mathcal,mathbf]{euler}

%\usepackage{palatino}
%\usepackage{newcent}


% Commande spéciale prompteur

%\usepackage{mathptmx}
%\usepackage[mathcal,mathbf]{euler}
%\usepackage{mathpple,multido}

\usepackage[a4paper]{geometry}
\geometry{top=2cm, bottom=2cm, left=1cm, right=1cm, marginparsep=1cm}

\newcommand{\change}{{\color{red}\rule{\textwidth}{1mm}\\}}

\newcounter{mydiapo}

\newcommand{\diapo}{\newpage
\hfill {\normalsize  Diapo \themydiapo \quad \texttt{[\jobname]}} \\
\stepcounter{mydiapo}}


%%%%%%% COULEURS %%%%%%%%%%

% Pour blanc sur noir :
%\pagecolor[rgb]{0.5,0.5,0.5}
% \pagecolor[rgb]{0,0,0}
% \color[rgb]{1,1,1}



%\DeclareFixedFont{\myfont}{U}{cmss}{bx}{n}{18pt}
\newcommand{\debuttexte}{
%%%%%%%%%%%%% FONTES %%%%%%%%%%%%%
\renewcommand{\baselinestretch}{1.5}
\usefont{U}{cmss}{bx}{n}
\bfseries

% Taille normale : commenter le reste !
%Taille Arnaud
%\fontsize{19}{19}\selectfont

% Taille Barbara
%\fontsize{21}{22}\selectfont

%Taille François
%\fontsize{25}{30}\selectfont

%Taille Pascal
%\fontsize{25}{30}\selectfont

%Taille Laura
%\fontsize{30}{35}\selectfont


%\myfont
%\usefont{U}{cmss}{bx}{n}

%\Huge
%\addtolength{\parskip}{\baselineskip}
}


% \usepackage{hyperref}
% \hypersetup{colorlinks=true, linkcolor=blue, urlcolor=blue,
% pdftitle={Exo7 - Exercices de mathématiques}, pdfauthor={Exo7}}


%section
% \usepackage{sectsty}
% \allsectionsfont{\bf}
%\sectionfont{\color{Tomato3}\upshape\selectfont}
%\subsectionfont{\color{Tomato4}\upshape\selectfont}

%----- Ensembles : entiers, reels, complexes -----
\newcommand{\Nn}{\mathbb{N}} \newcommand{\N}{\mathbb{N}}
\newcommand{\Zz}{\mathbb{Z}} \newcommand{\Z}{\mathbb{Z}}
\newcommand{\Qq}{\mathbb{Q}} \newcommand{\Q}{\mathbb{Q}}
\newcommand{\Rr}{\mathbb{R}} \newcommand{\R}{\mathbb{R}}
\newcommand{\Cc}{\mathbb{C}} 
\newcommand{\Kk}{\mathbb{K}} \newcommand{\K}{\mathbb{K}}

%----- Modifications de symboles -----
\renewcommand{\epsilon}{\varepsilon}
\renewcommand{\Re}{\mathop{\text{Re}}\nolimits}
\renewcommand{\Im}{\mathop{\text{Im}}\nolimits}
%\newcommand{\llbracket}{\left[\kern-0.15em\left[}
%\newcommand{\rrbracket}{\right]\kern-0.15em\right]}

\renewcommand{\ge}{\geqslant}
\renewcommand{\geq}{\geqslant}
\renewcommand{\le}{\leqslant}
\renewcommand{\leq}{\leqslant}

%----- Fonctions usuelles -----
\newcommand{\ch}{\mathop{\mathrm{ch}}\nolimits}
\newcommand{\sh}{\mathop{\mathrm{sh}}\nolimits}
\renewcommand{\tanh}{\mathop{\mathrm{th}}\nolimits}
\newcommand{\cotan}{\mathop{\mathrm{cotan}}\nolimits}
\newcommand{\Arcsin}{\mathop{\mathrm{Arcsin}}\nolimits}
\newcommand{\Arccos}{\mathop{\mathrm{Arccos}}\nolimits}
\newcommand{\Arctan}{\mathop{\mathrm{Arctan}}\nolimits}
\newcommand{\Argsh}{\mathop{\mathrm{Argsh}}\nolimits}
\newcommand{\Argch}{\mathop{\mathrm{Argch}}\nolimits}
\newcommand{\Argth}{\mathop{\mathrm{Argth}}\nolimits}
\newcommand{\pgcd}{\mathop{\mathrm{pgcd}}\nolimits} 

\newcommand{\Card}{\mathop{\text{Card}}\nolimits}
\newcommand{\Ker}{\mathop{\text{Ker}}\nolimits}
\newcommand{\id}{\mathop{\text{id}}\nolimits}
\newcommand{\ii}{\mathrm{i}}
\newcommand{\dd}{\mathrm{d}}
\newcommand{\Vect}{\mathop{\text{Vect}}\nolimits}
\newcommand{\Mat}{\mathop{\mathrm{Mat}}\nolimits}
\newcommand{\rg}{\mathop{\text{rg}}\nolimits}
\newcommand{\tr}{\mathop{\text{tr}}\nolimits}
\newcommand{\ppcm}{\mathop{\text{ppcm}}\nolimits}

%----- Structure des exercices ------

\newtheoremstyle{styleexo}% name
{2ex}% Space above
{3ex}% Space below
{}% Body font
{}% Indent amount 1
{\bfseries} % Theorem head font
{}% Punctuation after theorem head
{\newline}% Space after theorem head 2
{}% Theorem head spec (can be left empty, meaning ‘normal’)

%\theoremstyle{styleexo}
\newtheorem{exo}{Exercice}
\newtheorem{ind}{Indications}
\newtheorem{cor}{Correction}


\newcommand{\exercice}[1]{} \newcommand{\finexercice}{}
%\newcommand{\exercice}[1]{{\tiny\texttt{#1}}\vspace{-2ex}} % pour afficher le numero absolu, l'auteur...
\newcommand{\enonce}{\begin{exo}} \newcommand{\finenonce}{\end{exo}}
\newcommand{\indication}{\begin{ind}} \newcommand{\finindication}{\end{ind}}
\newcommand{\correction}{\begin{cor}} \newcommand{\fincorrection}{\end{cor}}

\newcommand{\noindication}{\stepcounter{ind}}
\newcommand{\nocorrection}{\stepcounter{cor}}

\newcommand{\fiche}[1]{} \newcommand{\finfiche}{}
\newcommand{\titre}[1]{\centerline{\large \bf #1}}
\newcommand{\addcommand}[1]{}
\newcommand{\video}[1]{}

% Marge
\newcommand{\mymargin}[1]{\marginpar{{\small #1}}}



%----- Presentation ------
\setlength{\parindent}{0cm}

%\newcommand{\ExoSept}{\href{http://exo7.emath.fr}{\textbf{\textsf{Exo7}}}}

\definecolor{myred}{rgb}{0.93,0.26,0}
\definecolor{myorange}{rgb}{0.97,0.58,0}
\definecolor{myyellow}{rgb}{1,0.86,0}

\newcommand{\LogoExoSept}[1]{  % input : echelle
{\usefont{U}{cmss}{bx}{n}
\begin{tikzpicture}[scale=0.1*#1,transform shape]
  \fill[color=myorange] (0,0)--(4,0)--(4,-4)--(0,-4)--cycle;
  \fill[color=myred] (0,0)--(0,3)--(-3,3)--(-3,0)--cycle;
  \fill[color=myyellow] (4,0)--(7,4)--(3,7)--(0,3)--cycle;
  \node[scale=5] at (3.5,3.5) {Exo7};
\end{tikzpicture}}
}



\theoremstyle{definition}
%\newtheorem{proposition}{Proposition}
%\newtheorem{exemple}{Exemple}
%\newtheorem{theoreme}{Théorème}
\newtheorem{lemme}{Lemme}
\newtheorem{corollaire}{Corollaire}
%\newtheorem*{remarque*}{Remarque}
%\newtheorem*{miniexercice}{Mini-exercices}
%\newtheorem{definition}{Définition}




%definition d'un terme
\newcommand{\defi}[1]{{\color{myorange}\textbf{\emph{#1}}}}
\newcommand{\evidence}[1]{{\color{blue}\textbf{\emph{#1}}}}



 %----- Commandes divers ------

\newcommand{\codeinline}[1]{\texttt{#1}}

%%%%%%%%%%%%%%%%%%%%%%%%%%%%%%%%%%%%%%%%%%%%%%%%%%%%%%%%%%%%%
%%%%%%%%%%%%%%%%%%%%%%%%%%%%%%%%%%%%%%%%%%%%%%%%%%%%%%%%%%%%%


\begin{document}

\debuttexte


%%%%%%%%%%%%%%%%%%%%%%%%%%%%%%%%%%%%%%%%%%%%%%%%%%%%%%%%%%%
\diapo

Poursuivons l'étude des séries par un autre type de série facile à étudier : les séries alternées.
Ce sont celles où le signe du terme général change à chaque rang.

\change
\change
Le résultat principal concernant les suites alternées est le critère de Leibniz.

\change
Nous en verrons ensuite un corollaire utile portant sur le reste de ces séries.

\change
Enfin, l'étude de deux contre-exemples nous permettra de vérifier que les hypothèses du critère de Leibniz sont indispensables.

%%%%%%%%%%%%%%%%%%%%%%%%%%%%%%%%%%%%%%%%%%%%%%%%%%%%%%%%%%%
\diapo

Commençons par définir ce que sont les séries alternées.

Soit $(u_k)_{k\ge0}$ une suite à termes positifs.

\change
La série $\sum_{k \ge 0} (-1)^k u_k$ s'appelle une \defi{série alternée}.

Comme les $u_k$ sont $\ge0$ alors le signe de $(-1)^k u_k$ est alternativement $+$,$-$$+$$-$...


\change
Pour ce type de séries, on dispose d'un critère de convergence, extrêmement facile à vérifier : le critère de Leibniz.

\change
Considérons une suite $(u_k)_{k\ge0}$ qui vérifie trois conditions.

\change
Tout d'abord, $u_k  \ge 0$ pour tout $k$.

\change
La suite $(u_k)$ est une suite décroissante.

\change
Enfin $\lim_{k\to+\infty} u_k=0$.

\change
Alors la série alternée $\displaystyle \sum_{k=0}^{+\infty} (-1)^k u_k$ converge.

%%%%%%%%%%%%%%%%%%%%%%%%%%%%%%%%%%%%%%%%%%%%%%%%%%%%%%%%%%%
\diapo

Pour montrer que la suite $(S_n)$ des sommes partielles converge, nous allons montrer que les deux suites extraites $(S_{2n+1})$ et $(S_{2n})$ sont adjacentes. En effet :

\change
La suite $(S_{2n+1})$ est croissante 

\change
puisque $S_{2n+1}-S_{2n-1}=u_{2n}-u_{2n+1}$ quantité qui est positive par décroissance de la suite $(u_n)$.
  
\change
De même, la suite $(S_{2n})$ est décroissante

\change
vu que  $S_{2n}-S_{2n-2}= u_{2n}-u_{2n-1}\le 0$.
  
\change 
$S_{2n} \ge S_{2n+1}$ 

\change
car $S_{2n+1} - S_{2n} = -u_{2n+1} $ et car la suite $(u_n)$ est positive.
  
\change
Enfin $S_{2n+1} - S_{2n}$ tend vers $0$
  
\change
car $S_{2n+1} - S_{2n} = -u_{2n+1} \to 0$ (lorsque $n\to+\infty$).

\change
En conséquence $(S_{2n+1})$ et $(S_{2n})$ sont des suites adjacentes 
et donc convergent
et en plus convergent vers la même limite $S$.

\change
On conclut que $(S_n)$ converge vers $S$.

\change
Mais on a montré plus ! En effet, on a également que $S_{2n+1} \le S \le S_{2n}$ pour tout $n$.

\change
Et donc le reste de rang pair, $R_{2n}$, qui vaut par définition $S-S_{2n}$, ...

\change
est négatif.

\change
Il est également supérieur ou égal à $ S_{2n+1}-S_{2n}$ 

\change
qui vaut $-u_{2n+1}$.

\change
De même, le reste de rang impair est compris entre $0$ et $u_{2n+2}.$

\change
Ainsi, quelle que soit la parité de $n$, on a $|R_n|=|S-S_n|\le  u_{n+1}$.

%%%%%%%%%%%%%%%%%%%%%%%%%%%%%%%%%%%%%%%%%%%%%%%%%%%%%%%%%%%
\diapo

Appliquons le critère de Leibniz sur un exemple, et montrons que la série harmonique alternée
$$\sum_{k=0}^{+\infty} (-1)^{k} \frac{1}{k+1} = 1-\frac{1}{2}+\frac{1}{3}-\frac{1}{4} + \cdots $$
converge.

\change
En effet, en posant $u_k = \frac{1}{k+1}$, on a

\change
premièrement : $u_k\ge0$,

\change
deuxièmement : $(u_k)$ est une suite décroissante,

\change
troisièmement : la suite $(u_k)$ tend vers $0$.

\change
Par le critère de Leibniz, la série alternée $\sum_{k=0}^{+\infty} (-1)^{k} \frac{1}{k+1}$ converge.

%%%%%%%%%%%%%%%%%%%%%%%%%%%%%%%%%%%%%%%%%%%%%%%%%%%%%%%%%%%
\diapo

Non seulement le critère de Leibniz prouve la convergence de la série 
$\sum_{k=0}^{+\infty} (-1)^k u_k$, mais sa démonstration nous fournit deux résultats importants supplémentaires : un encadrement de la somme et une majoration du reste.

Soit une série alternée $\displaystyle \sum_{k=0}^{+\infty} (-1)^k u_k$ vérifiant 
les hypothèses du critère de Leibniz.

\change
Soit $S$ la somme de cette série et soit $(S_n)$ la suite des sommes partielles.

\change
La somme $S$ vérifie les encadrements :
  $$ S_1\le S_3\le S_5\le \cdots \le S_{2n+1} \le \cdots \le S 
  \le  \cdots\le S_{2n} \le \cdots\le S_4\le S_2\le S_0.$$
  
\change
De plus, pour le reste d'ordre $n$, $\displaystyle R_n=S-S_n =\sum_{k=n+1}^{+\infty} (-1)^k u_k$, on  a
$$\big|R_n\big|\le u_{n+1}.$$  

\change
Pour une série alternée, la vitesse de convergence est donc
dictée par la décroissance vers $0$ de la suite $(u_k)$. Celle-ci
peut être assez lente.

%%%%%%%%%%%%%%%%%%%%%%%%%%%%%%%%%%%%%%%%%%%%%%%%%%%%%%%%%%%
\diapo

Par exemple, on a vu que la série harmonique alternée 
$\sum_{k=0}^{+\infty} \frac{(-1)^{k}}{k+1}$ converge. Notons $S$ sa somme.

\change
Les sommes partielles sont
$S_0 = 1$, 

\change
$S_1 = 1-\frac{1}{2}$,

\change
$S_2 = 1-\frac{1}{2}+\frac{1}{3}$,

\change
$S_3 = 1-\frac{1}{2}+\frac{1}{3}-\frac{1}{4}$,
$S_4 = 1-\frac{1}{2}+\frac{1}{3}-\frac{1}{4}+\frac{1}{5}$,\ldots

\change
L'encadrement du corollaire s'écrit 
$$1-\frac{1}{2} \le 1-\frac{1}{2}+\frac{1}{3}-\frac{1}{4} \le \cdots \le S_{2n+1} \le \cdots \le S 
  \le  \cdots\le S_{2n} \le \cdots 
  \le 1-\frac{1}{2}+\frac{1}{3}-\frac{1}{4}+\frac{1}{5}
  \le 1-\frac{1}{2}+\frac{1}{3} \le 1$$
  
\change
On en déduit que $S$ est compris entre 

$S_3 = \dfrac{35}{60} \simeq 0,58333$

et $S_4 = \dfrac{47}{60} \simeq 0,78333\ldots$

Poursuivons cet exemple pour obtenir un encadrement plus fin de $S$.


%%%%%%%%%%%%%%%%%%%%%%%%%%%%%%%%%%%%%%%%%%%%%%%%%%%%%%%%%%%
\diapo

Considérons toujours la série harmonique alternée.

\change  
Si on pousse les calculs plus loin, alors pour $n=200$ on obtient que $S$ est compris entre 

$S_{201} \simeq 0,69067$

et $S_{200} \simeq 0,69562\ldots$

\change  
Ce qui nous donne les deux premières décimales de $S \simeq 0,69\ldots$

\change  
De plus nous avons une majoration de l'erreur commise, en utilisant le deuxième résultat du corollaire, c'est-à-dire l'inégalité $|R_n|\le u_{n+1}$.
On trouve que l'erreur commise en approchant $S$ par $S_{200}$ est :

\change  
$|S-S_{200}| = |R_{200}| \le u_{201} = \frac{1}{202} < 5 \cdot 10^{-3}$.

En fait, vous verrez plus tard que $S=\ln 2 \simeq 0,69314\ldots$

%%%%%%%%%%%%%%%%%%%%%%%%%%%%%%%%%%%%%%%%%%%%%%%%%%%%%%%%%%%
\diapo

Terminons par deux mises en garde : dans le critère
de Leibniz

\change
1) on ne peut pas laisser tomber la condition de décroissance de la suite $(u_k)$ .

\change 
2) il n'est pas possible de remplacer $u_k$ par un équivalent à l'infini, car la décroissance n'est pas conservée par équivalence.   

Nous allons voir des contre-exemples qui démontrent ces deux affirmations.

%%%%%%%%%%%%%%%%%%%%%%%%%%%%%%%%%%%%%%%%%%%%%%%%%%%%%%%%%%%
\diapo
Voici deux séries alternées. Nous allons montrer que 
$$
\sum_{k \ge 2}  \frac{(-1)^k}{\sqrt{k}}
\quad \text{ converge,}$$
alors que
$$
\sum_{k \ge 2}  \frac{(-1)^k}{\sqrt{k}+(-1)^k}
\quad \text{ diverge.}
$$ 

\change
Le critère de Leibniz s'applique à la première de ces séries :

\change
la suite $u_k = \frac{1}{\sqrt k}$ est une suite 

\change
positive, 

\change
décroissante, 

\change
qui tend vers $0$. 

\change
Conséquence, on a bien montré que la série alternée  $\sum_{k \ge 2}  \frac{(-1)^k}{\sqrt{k}}$ converge.

\change
Par contre le critère de Leibniz ne s'applique pas à la seconde, 

\change
car si la suite $v_k = \frac{1}{\sqrt{k}+(-1)^k}$ est bien
positive (pour $k \ge 2$) et tend vers $0$, 

\change
elle n'est pas décroissante. 

\change
Cependant, on a bien que les suites $u_k$ et $v_k$ sont équivalentes.


%%%%%%%%%%%%%%%%%%%%%%%%%%%%%%%%%%%%%%%%%%%%%%%%%%%%%%%%%%%
\diapo
Montrons à présent que $\sum_{k \ge 2} \frac{(-1)^k}{\sqrt{k}+(-1)^k}$ diverge.

\change
Pour cela, calculons la différence $w_k = (-1)^ku_k-(-1)^kv_k$. 

\change
Par définition,  

$w_k= \dfrac{(-1)^k}{\sqrt{k}}-\dfrac{(-1)^k}{\sqrt{k}+(-1)^k}$

\change
en réduisant au même dénominateur, on trouve

$ (-1)^k\dfrac{\sqrt{k}+(-1)^k-\sqrt{k}}{k+(-1)^k\sqrt{k}}  $

\change
ce qui se simplifie en 

$ \dfrac{1}{k+(-1)^k\sqrt{k}} $

\change
la suite $w_k $ est donc équivalent à $ \frac{1}{k} $, qui est le terme général de la suite harmonique

\change
Ainsi la série de terme général $w_k = (-1)^ku_k-(-1)^kv_k$
diverge, car la série harmonique $\sum \frac{1}{k}$ diverge.

\change
On sait que la série $\sum_{k \ge 2} (-1)^k u_k$
est convergente.

\change
Supposons maintenant par l'absurde que la série
$\sum_{k \ge 2}(-1)^k v_k$ soit également convergente.

\change
Alors par linéarité la série
$\sum_{k \ge 2} w_k = \sum_{k \ge 2} (-1)^k u_k - \sum_{k \ge 2} (-1)^k v_k$
serait convergente, comme somme de deux séries convergente.

\change
Ce qui est une contradiction.

\change
Conclusion : la série $\sum \frac{(-1)^k}{\sqrt{k}+(-1)^k}$ diverge.  

%%%%%%%%%%%%%%%%%%%%%%%%%%%%%%%%%%%%%%%%%%%%%%%%%%%%%%%%%%%
\diapo

Voici des exercices qui vous permettront de vous familiariser avec les séries alternées.

\end{document}
