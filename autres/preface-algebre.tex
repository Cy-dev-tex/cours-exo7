
\pagestyle{empty}\thispagestyle{empty}
\vspace*{\fill}
\begin{center}
\fontsize{52}{52}\selectfont
\textsc{ALGÈBRE}
\end{center}
\vfill
\begin{center}
\huge
\textsc{Cours de mathématiques}\\ 
\textsc{Première année}
\end{center}

\begin{center}
\LogoExoSept{3}
\end{center}
\clearemptydoublepage
\thispagestyle{empty}

\vspace*{\fill}
\section*{À la découverte de l'algèbre}


La première année d'études supérieures pose les bases des mathématiques. 
Pourquoi se lancer dans une telle expédition ? Déjà parce que les mathématiques vous offriront 
un langage unique pour accéder à une multitude de domaines scientifiques. Mais aussi 
parce qu'il s'agit d'un domaine passionnant ! Nous vous proposons de partir à la découverte 
des maths, de leur logique et de leur beauté. 


Dans vos bagages, des objets que vous connaissez déjà : les entiers, les fonctions... 
Ces notions en apparence simples et intuitives seront abordées ici avec un souci de rigueur, 
en adoptant un langage précis et en présentant les preuves. Vous découvrirez ensuite de 
nouvelles théories (les espaces vectoriels, les équations différentielles,...).

\medskip

Ce tome est consacré à l'algèbre et se divise en deux parties. 
La première partie débute par la logique et les ensembles, qui sont des 
fondamentaux en mathématiques. Ensuite vous étudierez des ensembles particuliers : 
les nombres complexes, les entiers ainsi que les polynômes. Cette partie se termine 
par l'étude d'une première structure algébrique, avec la notion de groupe.

La seconde partie est entièrement consacrée à l'algèbre linéaire. C'est un domaine 
totalement nouveau pour vous et très riche, qui recouvre la notion de matrice et 
d'espace vectoriel. Ces concepts, à la fois profonds et utiles, demandent du temps 
et du travail pour être bien compris.

\medskip

Les efforts que vous devrez fournir sont importants : tout d'abord comprendre le cours, 
ensuite connaître par c\oe ur les définitions, les théorèmes, les propositions... 
sans oublier de travailler les exemples et les démonstrations, qui permettent de 
bien assimiler les notions nouvelles et les mécanismes de raisonnement. Enfin, vous 
devrez passer autant de temps à pratiquer les mathématiques : il est indispensable 
de résoudre activement par vous-même des exercices, sans regarder les solutions.
Pour vous aider, vous trouverez sur le site Exo7 toutes les vidéos correspondant 
à ce cours, ainsi que des exercices corrigés.

Au bout du chemin, le plaisir de découvrir de nouveaux univers, 
de chercher à résoudre des problèmes... et d'y parvenir. Bonne route !

\vspace*{\fill}

\newpage
\addtocontents{toc}{\protect\setcounter{tocdepth}{1}}
\tableofcontents

