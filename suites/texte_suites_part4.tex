
%%%%%%%%%%%%%%%%%% PREAMBULE %%%%%%%%%%%%%%%%%%


\documentclass[12pt]{article}

\usepackage{amsfonts,amsmath,amssymb,amsthm}
\usepackage[utf8]{inputenc}
\usepackage[T1]{fontenc}
\usepackage[francais]{babel}


% packages
\usepackage{amsfonts,amsmath,amssymb,amsthm}
\usepackage[utf8]{inputenc}
\usepackage[T1]{fontenc}
%\usepackage{lmodern}

\usepackage[francais]{babel}
\usepackage{fancybox}
\usepackage{graphicx}

\usepackage{float}

%\usepackage[usenames, x11names]{xcolor}
\usepackage{tikz}
\usepackage{datetime}

\usepackage{mathptmx}
%\usepackage{fouriernc}
%\usepackage{newcent}
\usepackage[mathcal,mathbf]{euler}

%\usepackage{palatino}
%\usepackage{newcent}


% Commande spéciale prompteur

%\usepackage{mathptmx}
%\usepackage[mathcal,mathbf]{euler}
%\usepackage{mathpple,multido}

\usepackage[a4paper]{geometry}
\geometry{top=2cm, bottom=2cm, left=1cm, right=1cm, marginparsep=1cm}

\newcommand{\change}{{\color{red}\rule{\textwidth}{1mm}\\}}

\newcounter{mydiapo}

\newcommand{\diapo}{\newpage
\hfill {\normalsize  Diapo \themydiapo \quad \texttt{[\jobname]}} \\
\stepcounter{mydiapo}}


%%%%%%% COULEURS %%%%%%%%%%

% Pour blanc sur noir :
%\pagecolor[rgb]{0.5,0.5,0.5}
% \pagecolor[rgb]{0,0,0}
% \color[rgb]{1,1,1}



%\DeclareFixedFont{\myfont}{U}{cmss}{bx}{n}{18pt}
\newcommand{\debuttexte}{
%%%%%%%%%%%%% FONTES %%%%%%%%%%%%%
\renewcommand{\baselinestretch}{1.5}
\usefont{U}{cmss}{bx}{n}
\bfseries

% Taille normale : commenter le reste !
%Taille Arnaud
%\fontsize{19}{19}\selectfont

% Taille Barbara
%\fontsize{21}{22}\selectfont

%Taille François
%\fontsize{25}{30}\selectfont

%Taille Pascal
%\fontsize{25}{30}\selectfont

%Taille Laura
%\fontsize{30}{35}\selectfont


%\myfont
%\usefont{U}{cmss}{bx}{n}

%\Huge
%\addtolength{\parskip}{\baselineskip}
}


% \usepackage{hyperref}
% \hypersetup{colorlinks=true, linkcolor=blue, urlcolor=blue,
% pdftitle={Exo7 - Exercices de mathématiques}, pdfauthor={Exo7}}


%section
% \usepackage{sectsty}
% \allsectionsfont{\bf}
%\sectionfont{\color{Tomato3}\upshape\selectfont}
%\subsectionfont{\color{Tomato4}\upshape\selectfont}

%----- Ensembles : entiers, reels, complexes -----
\newcommand{\Nn}{\mathbb{N}} \newcommand{\N}{\mathbb{N}}
\newcommand{\Zz}{\mathbb{Z}} \newcommand{\Z}{\mathbb{Z}}
\newcommand{\Qq}{\mathbb{Q}} \newcommand{\Q}{\mathbb{Q}}
\newcommand{\Rr}{\mathbb{R}} \newcommand{\R}{\mathbb{R}}
\newcommand{\Cc}{\mathbb{C}} 
\newcommand{\Kk}{\mathbb{K}} \newcommand{\K}{\mathbb{K}}

%----- Modifications de symboles -----
\renewcommand{\epsilon}{\varepsilon}
\renewcommand{\Re}{\mathop{\text{Re}}\nolimits}
\renewcommand{\Im}{\mathop{\text{Im}}\nolimits}
%\newcommand{\llbracket}{\left[\kern-0.15em\left[}
%\newcommand{\rrbracket}{\right]\kern-0.15em\right]}

\renewcommand{\ge}{\geqslant}
\renewcommand{\geq}{\geqslant}
\renewcommand{\le}{\leqslant}
\renewcommand{\leq}{\leqslant}

%----- Fonctions usuelles -----
\newcommand{\ch}{\mathop{\mathrm{ch}}\nolimits}
\newcommand{\sh}{\mathop{\mathrm{sh}}\nolimits}
\renewcommand{\tanh}{\mathop{\mathrm{th}}\nolimits}
\newcommand{\cotan}{\mathop{\mathrm{cotan}}\nolimits}
\newcommand{\Arcsin}{\mathop{\mathrm{Arcsin}}\nolimits}
\newcommand{\Arccos}{\mathop{\mathrm{Arccos}}\nolimits}
\newcommand{\Arctan}{\mathop{\mathrm{Arctan}}\nolimits}
\newcommand{\Argsh}{\mathop{\mathrm{Argsh}}\nolimits}
\newcommand{\Argch}{\mathop{\mathrm{Argch}}\nolimits}
\newcommand{\Argth}{\mathop{\mathrm{Argth}}\nolimits}
\newcommand{\pgcd}{\mathop{\mathrm{pgcd}}\nolimits} 

\newcommand{\Card}{\mathop{\text{Card}}\nolimits}
\newcommand{\Ker}{\mathop{\text{Ker}}\nolimits}
\newcommand{\id}{\mathop{\text{id}}\nolimits}
\newcommand{\ii}{\mathrm{i}}
\newcommand{\dd}{\mathrm{d}}
\newcommand{\Vect}{\mathop{\text{Vect}}\nolimits}
\newcommand{\Mat}{\mathop{\mathrm{Mat}}\nolimits}
\newcommand{\rg}{\mathop{\text{rg}}\nolimits}
\newcommand{\tr}{\mathop{\text{tr}}\nolimits}
\newcommand{\ppcm}{\mathop{\text{ppcm}}\nolimits}

%----- Structure des exercices ------

\newtheoremstyle{styleexo}% name
{2ex}% Space above
{3ex}% Space below
{}% Body font
{}% Indent amount 1
{\bfseries} % Theorem head font
{}% Punctuation after theorem head
{\newline}% Space after theorem head 2
{}% Theorem head spec (can be left empty, meaning ‘normal’)

%\theoremstyle{styleexo}
\newtheorem{exo}{Exercice}
\newtheorem{ind}{Indications}
\newtheorem{cor}{Correction}


\newcommand{\exercice}[1]{} \newcommand{\finexercice}{}
%\newcommand{\exercice}[1]{{\tiny\texttt{#1}}\vspace{-2ex}} % pour afficher le numero absolu, l'auteur...
\newcommand{\enonce}{\begin{exo}} \newcommand{\finenonce}{\end{exo}}
\newcommand{\indication}{\begin{ind}} \newcommand{\finindication}{\end{ind}}
\newcommand{\correction}{\begin{cor}} \newcommand{\fincorrection}{\end{cor}}

\newcommand{\noindication}{\stepcounter{ind}}
\newcommand{\nocorrection}{\stepcounter{cor}}

\newcommand{\fiche}[1]{} \newcommand{\finfiche}{}
\newcommand{\titre}[1]{\centerline{\large \bf #1}}
\newcommand{\addcommand}[1]{}
\newcommand{\video}[1]{}

% Marge
\newcommand{\mymargin}[1]{\marginpar{{\small #1}}}



%----- Presentation ------
\setlength{\parindent}{0cm}

%\newcommand{\ExoSept}{\href{http://exo7.emath.fr}{\textbf{\textsf{Exo7}}}}

\definecolor{myred}{rgb}{0.93,0.26,0}
\definecolor{myorange}{rgb}{0.97,0.58,0}
\definecolor{myyellow}{rgb}{1,0.86,0}

\newcommand{\LogoExoSept}[1]{  % input : echelle
{\usefont{U}{cmss}{bx}{n}
\begin{tikzpicture}[scale=0.1*#1,transform shape]
  \fill[color=myorange] (0,0)--(4,0)--(4,-4)--(0,-4)--cycle;
  \fill[color=myred] (0,0)--(0,3)--(-3,3)--(-3,0)--cycle;
  \fill[color=myyellow] (4,0)--(7,4)--(3,7)--(0,3)--cycle;
  \node[scale=5] at (3.5,3.5) {Exo7};
\end{tikzpicture}}
}



\theoremstyle{definition}
%\newtheorem{proposition}{Proposition}
%\newtheorem{exemple}{Exemple}
%\newtheorem{theoreme}{Théorème}
\newtheorem{lemme}{Lemme}
\newtheorem{corollaire}{Corollaire}
%\newtheorem*{remarque*}{Remarque}
%\newtheorem*{miniexercice}{Mini-exercices}
%\newtheorem{definition}{Définition}




%definition d'un terme
\newcommand{\defi}[1]{{\color{myorange}\textbf{\emph{#1}}}}
\newcommand{\evidence}[1]{{\color{blue}\textbf{\emph{#1}}}}



 %----- Commandes divers ------

\newcommand{\codeinline}[1]{\texttt{#1}}

%%%%%%%%%%%%%%%%%%%%%%%%%%%%%%%%%%%%%%%%%%%%%%%%%%%%%%%%%%%%%
%%%%%%%%%%%%%%%%%%%%%%%%%%%%%%%%%%%%%%%%%%%%%%%%%%%%%%%%%%%%%



\begin{document}

\debuttexte

%%%%%%%%%%%%%%%%%%%%%%%%%%%%%%%%%%%%%%%%%%%%%%%%%%%%%%%%%%
\diapo

\change

Dans cette partie du chapitre sur les suites, nous nous intéressons aux théorèmes 
de convergence, c'est-à-dire ceux qui permettent d'affirmer qu'une suite converge.

\change

Nous commencerons par nous intéresser aux suites croissantes et majorées,

\change

puis nous étudierons en détail deux exemples,

\change

avant d'énoncer le théorème sur les suites adjacentes.


\change


On conclut par un résultat très important : le théorème de Bolzano-Weierstrass.


%%%%%%%%%%%%%%%%%%%%%%%%%%%%%%%%%%%%%%%%%%%%%%%%%%%%%%%%%%
 \diapo

Le premier résultat est simple à énoncer : 
toute suite croissante et majorée est convergente
C-à-d admet une limite finie.

\change

On a aussi aisément le résultat symétrique : 
toute suite décroissante et minorée converge.

\change

Pour une suite croissante qui ne serait pas majorée alors elle tend vers $+\infty$.

\change

Passons à la démonstration du théorème. On note $A$ 
l'ensemble des valeurs de la suite, c'est un sous-ensemble de $\Rr$. 

\change

Par hypothèse la suite $(u_n)_{n\in \Nn}$ 
est majorée par un réel $M$, donc l'ensemble $A$ est majoré par $M$.

\change

De plus $A$ est clairement non vide, 
puisqu'il contient tous les termes de la suite.

Donc d'après un théorème du chapitre sur les réels, 
l'ensemble $A$ admet une borne supérieure finie, que l'on va noter $\ell=\sup A$. 

C'est un nombre réel.

\change

Montrons qu'en fait $\lim_{n\to +\infty} u_n=\ell$,

\change

[petit $n$, grand $N$]

On fixe un $\epsilon >0$. 


Par définition de la borne supérieure, il existe un élément $u_N$ de $A$ tel que
 $\ell-\epsilon < u_N \leq \ell$. 
 

Mais comme la suite est croissante, il s'ensuit que pour $n\geq N$ on a 
$\ell-\epsilon < u_N \leq u_n \leq \ell$, 
 
et donc $\lvert u_n-\ell \rvert \leq \epsilon$, ce qui signifie exactement que $u_n \to \ell$
et conclut la preuve.


%%%%%%%%%%%%%%%%%%%%%%%%%%%%%%%%%%%%%%%%%%%%%%%%%%%%%%%%%%
 \diapo

Le premier exemple est la suite dont le terme général est :

$$ u_n =1+\frac{1}{2^2} +\frac{1}{3^2}+\cdots+\frac{1}{n^2} $$

%$u_1 = 1$, $u_2=1+\frac{1}{2^2}$, $u_3=1+\frac{1}{2^2} +\frac{1}{3^2}$,  \qquad etc.

\change

Il est clair que la suite $(u_n)_{n\geq 1}$ est croissante : 
car le calcul de la différence entre deux termes consécutifs est
$$u_{n+1}-u_n= \frac{1}{(n+1)^2}>0$$ 

\change

De plus, par récurrence nous allons prouver que pour tout  
$n\geq 1$ on a $$u_n\leq 2 - \frac{1}{n} $$

\change

Pour $n=1$, on a $u_1=1\leq 2 - \frac{1}{1}$.

\change

Fixons $n\geq 1$ et supposons $u_n\leq 2 - \frac{1}{n}$. 
Alors $u_{n+1}=u_n+ \frac{1}{(n+1)^2}\leq  2 - \frac{1}{n}+ \frac{1}{(n+1)^2}$
par notre hypothèse de récurrence.

Pour continuer on peut par exemple écrire 
$\frac{1}{(n+1)^2}\leq \frac{1}{n(n+1)}=\frac{1}{n}-\frac{1}{n+1}$, 

donc en utilisant cette inégalité on trouve
qu'au rang suivant $u_{n+1}\leq 2-\frac{1}{n+1}$, .

\change

Ce qui achève la récurrence.
Donc pour tout $n$, $u_n\leq 2 - \frac{1}{n} $ et donc en particulier $u_n \le 2$.

\change

On peut appliquer le théorème que nous venons de voir : 
la suite $(u_n)_{n\geq 1}$ est croissante et majorée par $2$, donc  elle converge.

\change

Cette limite est notée $\zeta(2)$, pour l'instant on ne connaît pas la valeur de
cette limite mais plus tard vous prouverez que $\zeta(2)=\frac{\pi^2}{6}$.


%%%%%%%%%%%%%%%%%%%%%%%%%%%%%%%%%%%%%%%%%%%%%%%%%%%%%%%%%%
 \diapo


 Le deuxième exemple est la suite harmonique, qui est définie de manière analogue : 
 son terme général est la somme des inverses des 
 $n$ premiers entiers naturels :
 $$ u_n =1+\frac{1}{2} +\frac{1}{3}+\cdots+\frac{1}{n} $$

 \change

Nous allons montrer que $u_n \to +\infty$.

\change

La suite $(u_n)_{n\geq 1}$ est croissante :  
on calcule de nouveau la différence entre deux termes consécutifs $u_{n+1}-u_n= \frac{1}{n+1}>0$. 

\change

On va par contre montrer que la suite n'est pas majorée, 
pour cela on va calculer une minoration de certains termes.


\change


Estimons la différence entre
le terme au rang ${2^p}$ et celui au rang ${2^{p-1}}$.

\change

Lorsque l'on calcule la différence
$u_{2^p}-u_{2^{p-1}}$ les premiers termes se télescopent et il reste
cette somme des inverses, 

\change 

Il reste exactement $2^p-2^{p-1}$
soit $2^{p-1}$ termes 

\change

et chaque terme est supérieur à $\frac{1}{2^p}$, 

\change 

ce qui conduit à la minoration
$2^{p-1}\times \frac{1}{2^p}$ 


\change

qui vaut exactement $\frac{1}{2}$.

 
\change

À présent $u_{2^p}-1 = u_{2^p}-u_1$ est la somme télescopique 
$$(u_2 - u_1)+ (u_4 - u_2)+(u_8-u_4) \cdots +  (u_{2^p}-u_{2^{p-1}})$$ 
C'est la somme de $p$ termes tous supérieurs à $\frac{1}{2}$, 
donc cette somme est minorée par $\frac{p}{2}$.

\change

Comme $p$ est un entier arbitrairement grand, 
on en déduit que la suite  $(u_n)_{n\geq 1}$ est croissante 
et non bornée, donc elle tend vers $+\infty$.   

Conclusion : la suite harmonique tend vers $+\infty$.

%%%%%%%%%%%%%%%%%%%%%%%%%%%%%%%%%%%%%%%%%%%%%%%%%%%%%%%%%%
\diapo

Passons à présent aux suites adjacentes.


Deux suites $(u_n)_{n\in \Nn}$ et $(v_n)_{n\in \Nn}$ sont \defi{adjacentes} si
elles vérifient les trois points suivants. 

\begin{enumerate}
  \item la suite $(u_n)_{n\in \Nn}$ est croissante et la suite $(v_n)_{n\in \Nn}$ est décroissante,
  \item pour tout $n$, $u_n$ est majoré par  $v_n$,
  \item La limite de la suite différence $v_n-u_n$ est nulle.
\end{enumerate}

\change

Les suites adjacentes sont intéressantes car elles convergent :

Théorème : si $(u_n)_{n\in \Nn}$ et $(v_n)_{n\in \Nn}$ 
sont deux suites adjacentes alors elles sont toutes les deux convergentes 
et tendent vers la même limite.

\change

Voici la situation :

$u_n$ est croissante, $u_0 \le u_1 \le u_2 \le \cdots \le u_n $

$v_n$ est décroissante $v_n\le \cdots \le v_2 \le v_1 \le v_0$

et $u_n$ doit toujours être inférieure à $v_n$.


\change

La preuve du théorème est simple :

 \begin{itemize}
    \item La suite $(u_n)_{n\in \Nn}$ est croissante et majorée par $v_0$, 
    donc elle converge vers une limite que l'on note $\ell$.

\change 
    
\item De même la suite $(v_n)_{n\in \Nn}$ est décroissante et minorée par $u_0$, 
    donc elle converge vers une limite notée cette fois $\ell'$.

\change

    \item De plus  $\ell'-\ell$ est la limite de la suite 
    $v_n -u_n$, donc vaut $0$ par hypothèse, d'où l'égalité des limites $\ell=\ell'$.
  \end{itemize}

%%%%%%%%%%%%%%%%%%%%%%%%%%%%%%%%%%%%%%%%%%%%%%%%%%%%%%%%%%
\diapo

Reprenons l'exemple de $\zeta(2)$. 
Soit $(u_n)$ la suite définie 
$$u_n =  1+\frac{1}{2^2} +\frac{1}{3^2}+\cdots+\frac{1}{n^2}$$

\change

et soit $v_n$ la suite définie par $$ v_n = u_n + \frac2{n+1} .$$

\change

Montrons que $(u_n)$ et $(v_n)$ sont deux suites adjacentes :

\change

Tout d'abord $(u_n)$ est une suite croissante car $u_{n+1}-u_n = \frac{1}{(n+1)^2} > 0$.

\change

$(v_n)$ est une suite décroissante :
    
    
Mais le calcul est plus compliqué :
$v_{n+1}-v_n = \frac{1}{(n+1)^2} + \frac{2}{n+2} - \frac{2}{n+1} $

\change

On réduit au même dénominateur 

\change

Et l'expression se simplifie en 
$= \frac{-n}{(n+2)(n+1)^2}$

qui est clairement négatif.


La suite $(u_n)$ est croissante, 

la suite $(v_n)$ est décroissante.


\change


La différence entre les deux suites est
$v_n-u_n = \frac{2}{n+1} >0$, 

\change

donc $u_n$ est inférieure à $v_n$ pour tout $n$.

\change

 Enfin comme $v_n-u_n = \frac{2}{n+1}$ 
 
\change

 alors $ (v_n-u_n) \to 0$.

 
 \change
 
 On vient de montrer que les suites $(u_n)$ et $(v_n)$ sont deux suites adjacentes, 
 
 \change
 
 Par le théorème sur les suites adjacentes ces deux suite
 convergent vers une même limite finie $\ell$. 

\change

Nous avons même un peu plus car nous avons obtenu un 
encadrement de la limite : $u_n \le \ell \le v_n$ pour tout $n$.

\change

Par exemple si $n=3$
nous obtenons l'encadrement suivant [[show]]


\change

Et numériquement cela donne ces inégalités pour la limite  [[show]]

Bien sûr pour avoir un encadrement plus précis il 
faut prendre de plus grandes valeurs de $n$.
 
 
 
%%%%%%%%%%%%%%%%%%%%%%%%%%%%%%%%%%%%%%%%%%%%%%%%%%%%%%%%%%
 \diapo

 Nous introduisons à présent la dernière notion importante de ce chapitre, 
 celle de sous-suite ou suite extraite.

 
 De manière informelle, extraire une sous-suite, 
 
 c'est à partir d'une suite $(u_n)$, ici représentée par des croix bleues,
 
ne retenir que certains termes, ici entourés en rouge, afin 
 obtenir une nouvelle suite.


 Plus rigoureusement, une \defi{suite extraite} de la suite
 $(u_n)_{n\in \Nn}$ est une suite qui s'écrit 
$(u_{\phi(n)})_{n\in \Nn}$, où $\phi$ est une 
application strictement croissante de $\Nn$ dans $\Nn$.

 
%%%%%%%%%%%%%%%%%%%%%%%%%%%%%%%%%%%%%%%%%%%%%%%%%%%%%%%%%%
 \diapo
 
Revoici la définition de suite extraite 

\change

et regardons des exemples tiré de la suite $(u_n)$ 
de terme général $(-1)^n$. 

\change

Si on ne retient que les termes d'indices pairs, c-à-d si on pose
$\phi(n)=2n$ on trouve la sous-suite de terme $u_{\phi(n)} = (-1) ^{2n}$ 

\change

donc c'est une suite
constante égale à $1$.

\change

Voici la représentation, on extrait le terme d'indice $\phi(0)$ donc $u_0$ qui vaut $1$,
puis le terme d'indice $\phi(1)$ donc $u_2$ qui vaut aussi $1$, etc.

En fait à partir de la suite, $(u_n)$ on ne retient qu'un terme sur $2$.

\change

Si on ne retient que les termes d'indices multiples de $3$, 
c-à-d cette fois on pose $\psi(n)=3n$

alors $(-1) ^{3n}= (-1)^n$ 

\change

\change

et on retrouve la suite de départ.
La suite qui vaut d'abord $+1$, puis $-1$, puis $+1$,...

%%%%%%%%%%%%%%%%%%%%%%%%%%%%%%%%%%%%%%%%%%%%%%%%%%%%%%%%%%
 \diapo

On a le résultat suivant : toute suite extraite d'une suite 
convergente converge vers la même limite.

C-à-d : Si $u_n \to \ell$ alors toutes les suites extraites $u_{\phi(n)}$ 
tendent aussi vers $\ell$.

\change

La notion de suite extraite est utile en particulier 
pour montrer qu'une suite diverge. 

Si une suite admet une sous-suite qui diverge alors la suite initiale diverge.

C'est aussi le cas si une suite admet deux sous-suites 
convergeant vers deux limites distinctes : elle est alors divergente.

\change

On peut appliquer ce corollaire à la suite $(-1)^n$.


La sous-suite des termes d'indices pairs est la suite constante égale à $1$ (donc converge vers $1$)

alors que celle des termes d'indices impairs est la suite constante égale à $-1$ (donc converge vers $-1$),

Les deux  sous-suites ont des limites distinctes donc la suite diverge.


%%%%%%%%%%%%%%%%%%%%%%%%%%%%%%%%%%%%%%%%%%%%%%%%%%%%%%%%%%
 \diapo

 Nous terminons ce chapitre par un résultat important, 
 le théorème de Bolzano-Weierstrass.


 Celui-ci affirme que toute suite bornée admet une sous-suite convergente.

 
 
 \change
 
 Reprenons notre exemple de la suite $u_n = (-1)^n$.
 
C'est bien une suite bornée et nous avons
bien des sous suites convergentes,

par exemple la suite des termes de rang pair, ou celle des termes de rang impairs
(qui sont même des suites constantes).

\change

Autre exemple avec la suite définie par $v_n = \cos n$.

C'est encore une suite bornée, mais il n'est pas facile de montrer explicitement qu'il existe une sous-suite
qui converge. 

Le théorème de Bolzano-Weierstrass affirme pourtant qu'une telle
sous-suite existe !

C'est un résultat d'existence mais qui n'indique pas comment expliciter une sous-suite convergente.
% 
%  Voici la preuve, qui repose sur un procédé par dichotomie. 
%  
%  \change
% 
%  Comme la suite est bornée, l'ensemble des valeurs de la suite est contenue dans un intervalle $[a,b]$. 
% 
% \change
% 
% On initialise la construction en posant $a_0=a$, $b_0=b$, $\phi(0)=0$.
% 
% 
% \change
% 
% Au moins un des deux intervalles obtenu en coupant $[a,b]$ en deux sous-intervalles égaux contient $u_n$ pour une infinité d'indices $n$. 
% 
% \change
% 
% On note $[a_1,b_1]$ un tel  intervalle, par hypothèse il existe $\phi(1)$ un entier $\phi(1)>\phi(0)$  tel que $u_{\phi(1)}\in [a_1,b_1]$. 
% 
% \change
% 
% Par récurrence,  on construit pour tout entier naturel $n$ un intervalle $[a_n,b_n]$, 
% de longueur $\frac{b-a}{2^n}$,
% 
% \change
% 
% et un entier $\phi(n)>\phi(n-1)$ tel que $u_{\phi(n)}\in [a_n,b_n]$. 
% 
% \change
% 
% Par construction la suite $(a_n)_{n\in \Nn}$ est croissante et la suite 
% $(b_n)_{n\in \Nn}$ est décroissante. 
% 
% \change
% 
% De plus $\lim_{n\to +\infty} (b_n-a_n)=\lim_{n\to +\infty} \frac{b-a}{2^n}=0$, 
% 
% \change
% 
% donc les suites $(a_n)_{n\in \Nn}$ et $(b_n)_{n\in \Nn}$ sont adjacentes et convergent vers une même limite $\ell$.
% 
% \change
% 
% On peut conclure, en appliquant le théorème \og{} des gendarmes\fg{}, que $\lim_{n\to +\infty} u_{\phi(n)}=\ell$.




%%%%%%%%%%%%%%%%%%%%%%%%%%%%%%%%%%%%%%%%%%%%%%%%%%%%%%%%%%
 \diapo


Voici quelques exercices pour vous aider à assimiler le cours.


\end{document}
