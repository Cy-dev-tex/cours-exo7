
%%%%%%%%%%%%%%%%%% PREAMBULE %%%%%%%%%%%%%%%%%%


\documentclass[12pt]{article}

\usepackage{amsfonts,amsmath,amssymb,amsthm}
\usepackage[utf8]{inputenc}
\usepackage[T1]{fontenc}
\usepackage[francais]{babel}


% packages
\usepackage{amsfonts,amsmath,amssymb,amsthm}
\usepackage[utf8]{inputenc}
\usepackage[T1]{fontenc}
%\usepackage{lmodern}

\usepackage[francais]{babel}
\usepackage{fancybox}
\usepackage{graphicx}

\usepackage{float}

%\usepackage[usenames, x11names]{xcolor}
\usepackage{tikz}
\usepackage{datetime}

\usepackage{mathptmx}
%\usepackage{fouriernc}
%\usepackage{newcent}
\usepackage[mathcal,mathbf]{euler}

%\usepackage{palatino}
%\usepackage{newcent}


% Commande spéciale prompteur

%\usepackage{mathptmx}
%\usepackage[mathcal,mathbf]{euler}
%\usepackage{mathpple,multido}

\usepackage[a4paper]{geometry}
\geometry{top=2cm, bottom=2cm, left=1cm, right=1cm, marginparsep=1cm}

\newcommand{\change}{{\color{red}\rule{\textwidth}{1mm}\\}}

\newcounter{mydiapo}

\newcommand{\diapo}{\newpage
\hfill {\normalsize  Diapo \themydiapo \quad \texttt{[\jobname]}} \\
\stepcounter{mydiapo}}


%%%%%%% COULEURS %%%%%%%%%%

% Pour blanc sur noir :
%\pagecolor[rgb]{0.5,0.5,0.5}
% \pagecolor[rgb]{0,0,0}
% \color[rgb]{1,1,1}



%\DeclareFixedFont{\myfont}{U}{cmss}{bx}{n}{18pt}
\newcommand{\debuttexte}{
%%%%%%%%%%%%% FONTES %%%%%%%%%%%%%
\renewcommand{\baselinestretch}{1.5}
\usefont{U}{cmss}{bx}{n}
\bfseries

% Taille normale : commenter le reste !
%Taille Arnaud
%\fontsize{19}{19}\selectfont

% Taille Barbara
%\fontsize{21}{22}\selectfont

%Taille François
%\fontsize{25}{30}\selectfont

%Taille Pascal
%\fontsize{25}{30}\selectfont

%Taille Laura
%\fontsize{30}{35}\selectfont


%\myfont
%\usefont{U}{cmss}{bx}{n}

%\Huge
%\addtolength{\parskip}{\baselineskip}
}


% \usepackage{hyperref}
% \hypersetup{colorlinks=true, linkcolor=blue, urlcolor=blue,
% pdftitle={Exo7 - Exercices de mathématiques}, pdfauthor={Exo7}}


%section
% \usepackage{sectsty}
% \allsectionsfont{\bf}
%\sectionfont{\color{Tomato3}\upshape\selectfont}
%\subsectionfont{\color{Tomato4}\upshape\selectfont}

%----- Ensembles : entiers, reels, complexes -----
\newcommand{\Nn}{\mathbb{N}} \newcommand{\N}{\mathbb{N}}
\newcommand{\Zz}{\mathbb{Z}} \newcommand{\Z}{\mathbb{Z}}
\newcommand{\Qq}{\mathbb{Q}} \newcommand{\Q}{\mathbb{Q}}
\newcommand{\Rr}{\mathbb{R}} \newcommand{\R}{\mathbb{R}}
\newcommand{\Cc}{\mathbb{C}} 
\newcommand{\Kk}{\mathbb{K}} \newcommand{\K}{\mathbb{K}}

%----- Modifications de symboles -----
\renewcommand{\epsilon}{\varepsilon}
\renewcommand{\Re}{\mathop{\text{Re}}\nolimits}
\renewcommand{\Im}{\mathop{\text{Im}}\nolimits}
%\newcommand{\llbracket}{\left[\kern-0.15em\left[}
%\newcommand{\rrbracket}{\right]\kern-0.15em\right]}

\renewcommand{\ge}{\geqslant}
\renewcommand{\geq}{\geqslant}
\renewcommand{\le}{\leqslant}
\renewcommand{\leq}{\leqslant}

%----- Fonctions usuelles -----
\newcommand{\ch}{\mathop{\mathrm{ch}}\nolimits}
\newcommand{\sh}{\mathop{\mathrm{sh}}\nolimits}
\renewcommand{\tanh}{\mathop{\mathrm{th}}\nolimits}
\newcommand{\cotan}{\mathop{\mathrm{cotan}}\nolimits}
\newcommand{\Arcsin}{\mathop{\mathrm{Arcsin}}\nolimits}
\newcommand{\Arccos}{\mathop{\mathrm{Arccos}}\nolimits}
\newcommand{\Arctan}{\mathop{\mathrm{Arctan}}\nolimits}
\newcommand{\Argsh}{\mathop{\mathrm{Argsh}}\nolimits}
\newcommand{\Argch}{\mathop{\mathrm{Argch}}\nolimits}
\newcommand{\Argth}{\mathop{\mathrm{Argth}}\nolimits}
\newcommand{\pgcd}{\mathop{\mathrm{pgcd}}\nolimits} 

\newcommand{\Card}{\mathop{\text{Card}}\nolimits}
\newcommand{\Ker}{\mathop{\text{Ker}}\nolimits}
\newcommand{\id}{\mathop{\text{id}}\nolimits}
\newcommand{\ii}{\mathrm{i}}
\newcommand{\dd}{\mathrm{d}}
\newcommand{\Vect}{\mathop{\text{Vect}}\nolimits}
\newcommand{\Mat}{\mathop{\mathrm{Mat}}\nolimits}
\newcommand{\rg}{\mathop{\text{rg}}\nolimits}
\newcommand{\tr}{\mathop{\text{tr}}\nolimits}
\newcommand{\ppcm}{\mathop{\text{ppcm}}\nolimits}

%----- Structure des exercices ------

\newtheoremstyle{styleexo}% name
{2ex}% Space above
{3ex}% Space below
{}% Body font
{}% Indent amount 1
{\bfseries} % Theorem head font
{}% Punctuation after theorem head
{\newline}% Space after theorem head 2
{}% Theorem head spec (can be left empty, meaning ‘normal’)

%\theoremstyle{styleexo}
\newtheorem{exo}{Exercice}
\newtheorem{ind}{Indications}
\newtheorem{cor}{Correction}


\newcommand{\exercice}[1]{} \newcommand{\finexercice}{}
%\newcommand{\exercice}[1]{{\tiny\texttt{#1}}\vspace{-2ex}} % pour afficher le numero absolu, l'auteur...
\newcommand{\enonce}{\begin{exo}} \newcommand{\finenonce}{\end{exo}}
\newcommand{\indication}{\begin{ind}} \newcommand{\finindication}{\end{ind}}
\newcommand{\correction}{\begin{cor}} \newcommand{\fincorrection}{\end{cor}}

\newcommand{\noindication}{\stepcounter{ind}}
\newcommand{\nocorrection}{\stepcounter{cor}}

\newcommand{\fiche}[1]{} \newcommand{\finfiche}{}
\newcommand{\titre}[1]{\centerline{\large \bf #1}}
\newcommand{\addcommand}[1]{}
\newcommand{\video}[1]{}

% Marge
\newcommand{\mymargin}[1]{\marginpar{{\small #1}}}



%----- Presentation ------
\setlength{\parindent}{0cm}

%\newcommand{\ExoSept}{\href{http://exo7.emath.fr}{\textbf{\textsf{Exo7}}}}

\definecolor{myred}{rgb}{0.93,0.26,0}
\definecolor{myorange}{rgb}{0.97,0.58,0}
\definecolor{myyellow}{rgb}{1,0.86,0}

\newcommand{\LogoExoSept}[1]{  % input : echelle
{\usefont{U}{cmss}{bx}{n}
\begin{tikzpicture}[scale=0.1*#1,transform shape]
  \fill[color=myorange] (0,0)--(4,0)--(4,-4)--(0,-4)--cycle;
  \fill[color=myred] (0,0)--(0,3)--(-3,3)--(-3,0)--cycle;
  \fill[color=myyellow] (4,0)--(7,4)--(3,7)--(0,3)--cycle;
  \node[scale=5] at (3.5,3.5) {Exo7};
\end{tikzpicture}}
}



\theoremstyle{definition}
%\newtheorem{proposition}{Proposition}
%\newtheorem{exemple}{Exemple}
%\newtheorem{theoreme}{Théorème}
\newtheorem{lemme}{Lemme}
\newtheorem{corollaire}{Corollaire}
%\newtheorem*{remarque*}{Remarque}
%\newtheorem*{miniexercice}{Mini-exercices}
%\newtheorem{definition}{Définition}




%definition d'un terme
\newcommand{\defi}[1]{{\color{myorange}\textbf{\emph{#1}}}}
\newcommand{\evidence}[1]{{\color{blue}\textbf{\emph{#1}}}}



 %----- Commandes divers ------

\newcommand{\codeinline}[1]{\texttt{#1}}

%%%%%%%%%%%%%%%%%%%%%%%%%%%%%%%%%%%%%%%%%%%%%%%%%%%%%%%%%%%%%
%%%%%%%%%%%%%%%%%%%%%%%%%%%%%%%%%%%%%%%%%%%%%%%%%%%%%%%%%%%%%



\begin{document}

\debuttexte

%%%%%%%%%%%%%%%%%%%%%%%%%%%%%%%%%%%%%%%%%%%%%%%%%%%%%%%%%%
\diapo

\change

\change

Dans cette partie sur les suites, nous donnons la définition 
précise de la limite (finie ou infinie) pour une suite de réels,

\change

après quoi nous discuterons des propriétés des limites.

\change

Après avoir donné quelques preuves, 

\change 

nous conclurons en précisant les liens entre limites et inégalités.

%%%%%%%%%%%%%%%%%%%%%%%%%%%%%%%%%%%%%%%%%%%%%%%%%%%%%%%%%%
% \diapo
% 
% Commençons par un exemple concret de limite de suite.
% 
% \change
% 
% On achète un abonnement de train qui permet de payer les billets à 
% $50\%$ du prix normal soit $10$ euros au lieu de $20$ euros. 
% 
% \change
% 
% L'abonnement vaut lui-même $50$ euros.
% 
%  \change 
% 
%  Pour calculer la réduction réelle, on considère deux suites,
%  
%  tout d'abord une suite  $u_n$ correspondant au prix payés au bout de $n$ d'achats sans réduction.
%  
%  $ u_n=20n$
%  
%  \change
% 
%  et $v_n$ calculant le prix payé au bout de $n$ d'achats avec réduction en incluant le prix de l'abonnement.
%  
%   $v_n=10n+50$
% 
%  \change 
% 
%  Le pourcentage de la réduction réelle entre ces deux options : 
% 
%  est 
%  $1-\frac{v_n}{u_n}$
%  
%  \change
%  
%  ce qui se réécrit  $=\frac{u_n-v_n}{u_n}$
%  
%  \change
%  
%  on remplace par les valeurs des termes pour obtenir $\frac{10n-50}{20n}$
%  
%  \change
%  
%  Ce qui donne  $0,5 -\frac{5}{2n}$
%  
%  \change
%  
%  Lorsque $n$ devient grand la réduction s'approche de $0,5$.
% 
% 
%  La vraie réduction \emph{tend} vers $50\%$ pour un nombre infini d'achats, 
%  tout en lui restant toujours inférieure. 
% 
%  
%  

%%%%%%%%%%%%%%%%%%%%%%%%%%%%%%%%%%%%%%%%%%%%%%%%%%%%%%%%%%
\diapo

Voici la définition mathématique de limite.

\change

[petit n, grand N]

La suite $(u_n)$ a pour \defi{limite} $\ell\in \Rr$ si : 
pour tout $\epsilon >0$, il existe un entier 
$N$ tel que dès que l'entier $n$ est $\geq N$ alors $\lvert u_n-\ell\rvert\leq\epsilon$. 

\change

voici la phrase mathématique :

pour tout $\epsilon >0$, il existe un entier naturel $N$ 
tel que si $n\geq N$ alors $\lvert u_n-\ell\rvert\leq\epsilon$

Autrement dit : $u_n$ est aussi proche que l'on veut de $\ell$, à partir d'un certain rang.

\change

Si elle existe on note "limite de $u_n$ égale $\ell$ lorsque $n$ tend vers $+\infty$"

ou aussi "$u_n$ tend vers $\ell$ lorsque $n$ tend vers $+\infty$".

\change

L'inégalité $|u_n-\ell| \le \epsilon$ équivaut à l'encadrement 
$\ell-\epsilon \le u_n \le \ell +\epsilon$

%%%%%%%%%%%%%%%%%%%%%%%%%%%%%%%%%%%%%%%%%%%%%%%%%%%%%%%%%%
\diapo

Voici une figure pour vous aidez à comprendre et retenir la définition
de $u_n\to \ell$, lorsque $n\to +\infty$.

\change


La suite est représentée en bleue.



Tout choix de $\epsilon$, qui moralement est très petit,
définie une bande horizontale, entre ces deux trais rouge, qui correspond aux ordonnées compris
entre $\ell-\epsilon$ et $\ell+\epsilon$.

Nous devons alors trouver un rang grand $N$, tel que pour tout les indices petit $n$
supérieurs à ce grand $N$, alors $u$ indice petit $n$ est dans la bande, c-à-d compris entre 
$\ell-\epsilon$ et $\ell+\epsilon$.

Sur cet exemple, j'ai choisi cette valeur pour grand $N$, 
et effectivement toutes les valeurs suivantes sont dans la bande.
C'est le cas en particulier pour cette valeur de $u_n$.

Bien sûr si on me donne $\epsilon$ plus petit, la bande sera plus fine ; 
alors ce grand $N$ ne convient plus et je dois en trouver un beaucoup plus grand.

Si pour tout les epsilon je peux trouver un tel grand $N$, alors par définition
$u_n \to \ell$.


%%%%%%%%%%%%%%%%%%%%%%%%%%%%%%%%%%%%%%%%%%%%%%%%%%%%%%%%%%
\diapo

La suite $(u_n)_{n\in \Nn}$ \defi{tend vers $+\infty$} si :
pour tout $A >0$, il existe un entier naturel $N$ tel que si $n\geq N$ alors $u_n\geq A$ . 

Autrement dit : $u_n$ est aussi grand que l'on veut, à partir d'un certain rang.

\change

Sur ce graphique, pour chaque $A$, qui moralement est très grand,
on doit trouver un rang grand $N$ tel que, pour des indices plus grands 
que $N$ alors les termes sont supérieurs à $A$.

\change

On obtient la définition de $(u_n)_{n\in \Nn}$ 
\defi{tend vers $-\infty$} en inversant le signe et le sens 
de l'inégalité pour la constante $A$, 
ce qui revient à dire pour faire bref que la suite $(-u_n)_{n\in \Nn}$ tend vers $+\infty$.


%%%%%%%%%%%%%%%%%%%%%%%%%%%%%%%%%%%%%%%%%%%%%%%%%%%%%%%%%%
\diapo

Enfin, une suite $(u_n)_{n\in \Nn}$ est dite \defi{convergente} 
si elle admet une limite \evidence{finie}.

Sinon on dit que la suite est \defi{divergente} :

c'est-à-dire soit la suite tend vers $\pm \infty$, soit elle n'admet pas de limite.

\change

Voici une proposition importante :

"Si une suite est convergente, sa limite est unique"

Il y a donc \emph{unicité} de la limite d'une suite.


Autrement dit, si une limite existe, elle est unique.
et on parle de *\evidence{la}* limite.
% 
% 
% \change
% 
%  On raisonne par l'absurde : soit donc $(u_n)_{n\in \Nn}$ une suite convergente ayant deux limites $\ell\neq \ell'$. 
%  
% \change
% 
% On choisit un nombre $\epsilon >0$ strictement plus petit que la moitié de la distance entre les deux limites $l$ et $l'$. 
% 
% \change
% 
% Comme $\lim_{n\to +\infty}u_n=\ell$, il existe un entier naturel $N_1$ tel que $n\geq N_1$ implique que la distance de $ u_n$ à $\ell$ soit plus petite que $\epsilon$. 
% 
% \change
% 
% De même comme $\lim_{n\to +\infty}u_n=\ell'$, il existe $N_2$ tel que $n\geq N_2$ implique que la distance de $ u_n$ à $\ell'$ soit plus petite que $\epsilon$. 
% 
% 
% \change 
% 
% Alors pour $N$ assez grand, par exemple pour $N=\max(N_1,N_2)$, on a que les distances de  $ u_n$ à $l$ et à $l'$ sont plus petites que $\epsilon$. 
% 
% \change
% 
% 
% 
% Alors l'inégalité triangulaire implique que la distance de $l$ à $l'$ est plus petite que $2\epsilon$.
% 
% \change
% 
% Ceci contredit notre choix de $\epsilon$ et conclut la démonstration.

%%%%%%%%%%%%%%%%%%%%%%%%%%%%%%%%%%%%%%%%%%%%%%%%%%%%%%%%%%
 \diapo

 On rassemble ici quelques opérations usuelles sur les limites.

 On commence par les limites \emph{finies}.
 
 Si une suite $(u_n)$ tend vers un réel $\ell$ alors
 la suite $(\lambda u_n)$, où chaque terme est multiplié par 
 le réel $\lambda$ a pour limite $\lambda \ell$.

 \change 
 
 Si on a une suite qui tend vers $\ell$ et une autre qui tend vers $\ell'$ alors
 la somme tend vers $\ell+\ell'$.
 
 Autrement dit : "La limite d'une somme est la somme des limites".
 
 
\change

C'est pareil pour le produit :
"La limite d'un produit est le produit des limites".

Si $u_n \to \ell$ et $v_n \to \ell'$ alors la suite dont le terme général
est le produit $u_n \times v_n$ tend vers $\ell \times \ell'$.

\change

C'est deux propriétés sont utiles continuellement.

\change

Soit une suite $(u_n)$ qui a pour limite $\ell$ un réel *non nul*
alors 

 * premièrement  à partir d'un certain rang la suite $u_n$ ne s'annule pas,
 
 * deuxièmement la suite des termes $1/u_n$ tend vers $1/\ell$.
 
 \change
 
Nous utilisons ces propriétés, le plus souvent sans nous en rendre compte.

Par exemple si $u_n \to \ell$ avec $\ell \neq \pm 1$, alors 
$$u_n(1-3u_n)-\frac{1}{u_n^2-1} \to \ell(1-3\ell) - \frac{1}{\ell^2-1}$$



%%%%%%%%%%%%%%%%%%%%%%%%%%%%%%%%%%%%%%%%%%%%%%%%%%%%%%%%%%
\diapo

On passe maintenant aux limites \emph{infinies}.

Si $u_n$ tend vers $+\infty$ alors la suite des $1/u_n$ tend vers $0$.


\change 

Réciproquement si $u_n$ tend vers $0$ et que les $u_n$ sont strictement positifs alors 
$1/u_n$ tend vers $+\infty$.

\change 

Une somme de suites dont l'une tend vers $+\infty$ et l'autre 
est minorée tend vers $+\infty$.

\change

La même propriété a lieu avec le produit si l'une des suites 
tend vers $+\infty$ et l'autre est minorée par un nombre $>0$.

%%%%%%%%%%%%%%%%%%%%%%%%%%%%%%%%%%%%%%%%%%%%%%%%%%%%%%%%%%
 \diapo

 On va maintenant prouver complètement l'affirmation 
 "la limite d'un produit est le produit des limites".

 On commence par la proposition suivante, qui affirme qu'une suite convergente est nécessairement \emph{bornée}.

 \change

 Soit donc $(u_n)$ une suite qui converge, on note $\ell$ sa limite.
 
 \change
 
 Pour montrer cette proposition, on applique la définition de limite finie 
 en prenant une précision arbitraire, par exemple $\epsilon=1$. 

 Voici une représentation graphique.
 
 
 \change 

 Donc à partir d'une certain rang $N$, la suite $u_n$ est à distance au plus $1$ de la limite $\ell$ :
 c-à-d $|u_n-\ell| \le 1$.

 \change

 Nous allons majorer $u_n$ en valeur absolue :
 
 \change
 
 Tout d'abord on réécrit $u_n$ comme étant $\ell + (u_n-\ell)$.
 
 \change
 
 
 En appliquant l'inégalité triangulaire, on majore la valeur absolue de $u_n$ par celle de $\ell$ plus 
 celle de $u_n-\ell$.

 \change
 
 Mais $|u_n-\ell| \le 1$ donc on obtient que $|u_n| \le |\ell|+1$.
 
 On vient de prouver que la suite est bornée à partir d'un certain rang $N$.

\change

Mais comme les premiers termes de la suite 
$u_0, u_1,\ldots,u_{N-1}$ sont en nombre fini, 
ils forment un ensemble également borné.

On pose $M$ le maximum des valeurs absolues des premiers termes, 
et de la valeur absolue de $\ell \quad +1$, obtenue pour les autres.

\change


On obtient un majorant pour $|u_n|$ quelque soit $n$.

Ce qui prouve exactement que la suite est bornée.

%%%%%%%%%%%%%%%%%%%%%%%%%%%%%%%%%%%%%%%%%%%%%%%%%%%%%%%%%%
 \diapo

Dans une deuxième étape, on prouve une nouvelle proposition : 
le produit d'une suite bornée et d'une suite qui tend vers $0$ est une suite qui tend vers $0$.

\change

Comme application directe, on peut prendre par exemple la suite $u_n=\cos(n)$ 
et la suite $v_n=\frac{1}{\sqrt{n}}$ :
alors la suite $\frac{\cos n}{\sqrt n}$ tend vers $0$ (lorsque $n \to +\infty$).

\change
Donnons la preuve de la proposition.
Comme la suite $(u_n)_{n\in \Nn}$ est bornée, on peut donc trouver par définition 
un réel $M>0$ tel que pour tout $n$ on ait $\lvert u_n \rvert\leq M$. 

\change

On nous donne un $\epsilon >0$. On doit montrer que pour $n$ assez grand, on a  $\lvert u_nv_n \rvert<\epsilon$.

\change 

On applique la  définition de limite à la suite $(v_n)_{n\in \Nn}$ 

\change

pour  la précision $\epsilon'=\frac{\epsilon}{M}$.  

\change

Il existe donc un entier naturel $N$ tel que 
$n\geq N$ implique $$ \lvert v_n  \rvert\leq  \epsilon'$$ 


\change

On en déduit que pour  $n\geq N$ on a :
\[ \lvert u_nv_n \rvert= \lvert u_n \rvert \lvert v_n \rvert \leq M\times \epsilon'=\epsilon .\]

\change

Pour tout $\epsilon$ on a trouvé un rang $N$ tel que pour $n\ge N$ alors $|u_n v_n| \le \epsilon$.

On a bien montré que $\lim_{n\to +\infty}\left(u_n\times v_n\right)=0$.


%%%%%%%%%%%%%%%%%%%%%%%%%%%%%%%%%%%%%%%%%%%%%%%%%%%%%%%%%%
 \diapo

On peut à présent prouver le résultat : la limite d'un produit de suites convergentes existe et est le produit des limites.
%de chacun des termes du produit.

\change

On commence par écrire l'égalité 
  \[
    u_nv_n-\ell\ell'=(u_n-\ell)v_n+\ell(v_n-\ell')
  \]

  \change

  La suite $\ell(v_n-\ell')$ tend vers $0$ car $\ell$ est une constante et $v_n\to \ell'$. 

  \change

  D'après la proposition précédente, c'est aussi le cas pour la suite $(u_n-\ell)v_n$, 

  \change
   
   Tout d'abord $u_n \to \ell$ donc $u_n-\ell \to 0$
   
   \change
   
 la suite $(v_n)$ est bornée, car $(v_n)$ est convergente
 
 \change
 
 donc par la proposition de la page précédente $(u_n-\ell)v_n \to 0$.

\change

Ainsi chacun de ces deux termes tendent tend vers $0$.

On en déduit $u_nv_n-\ell\ell' \to 0$, 
ce qui n'est rien d'autre que le résultat voulu :  $\lim_{n\to +\infty} u_nv_n=\ell\ell'$.


%%%%%%%%%%%%%%%%%%%%%%%%%%%%%%%%%%%%%%%%%%%%%%%%%%%%%%%%%%
 \diapo

Lorsqu'on est dans une situation où les règles précédentes 
concernant les opérations sur les limites ne s'appliquent pas, 
on parle de \emph{forme indéterminée}.

\change

Une forme indéterminée des plus courantes est \og{}$+\infty-\infty$\fg{} 


Cela signifie que si $u_n \to + \infty$ et $v_n \to + \infty$
alors on ne peut rien dire de général sur la limite de la différence 
$(u_n-v_n)$

Voici trois exemples où $u_n \to + \infty$ et $v_n \to + \infty$
mais avec des comportement pour $u_n-v_n$ différents.

\change

Premier exemple $u_n = e^n$  et $v_n=\ln(n)$ qui tendent toutes deux vers $+\infty$
et ici la différence $u_n-v_n$ tend vers $+\infty$.

\change

Deuxième exemple $u_n = n$  et $v_n=n^2$ qui tendent encore toutes deux vers $+\infty$
mais cette fois $u_n-v_n$ tend vers $-\infty$.

\change

Enfin pour ce troisième exemple $u_n = n+1/n$  et $v_n=n$
la différence vaut $1/n$ et tend donc vers $0$.

\change

Autre situation fréquente : une forme indéterminée du type \og{} $0\times \infty$\fg{}

\change

Voici des exemples qui montrent que tout type de résultat 
est possible.

\change

Voici encore d'autres formes indéterminées possibles.

Lorsque l'on est en présence d'une forme indéterminée, 
une étude plus approfondie au cas par cas 
s'impose pour déterminer la limite.



%%%%%%%%%%%%%%%%%%%%%%%%%%%%%%%%%%%%%%%%%%%%%%%%%%%%%%%%%%
\diapo

On conclut cette partie du chapitre sur les suites par une proposition reliant limites et inégalités.

Si deux suites convergentes sont dans un certain ordre, alors leurs limites sont dans le même ordre.

Autrement dit si on a $u_n \le v_n$ pour tout $n$ et que les deux suites admettent une limite alors
$\lim u_n \le \lim v_n$.

\change

Une variante existe dans le cas 
où $u_n$ tend vers $+\infty$ et que $v_n$ est plus grand que $u_n$ pour tout $n$ alors
la suite $(v_n)$ tend aussi vers $+\infty$.


%%%%%%%%%%%%%%%%%%%%%%%%%%%%%%%%%%%%%%%%%%%%%%%%%%%%%%%%%%
\diapo

Enfin le théorème des  \og{}gendarmes  \fg{} 


\change

[image]

affirme que si une suite est coincée 
entre deux autres tendant vers une *même* limite, alors la suite du milieu tend elle-même vers la limite commune.

Plus précisément : nous avons trois suites $(u_n)$, $(v_n)$, $(w_n)$
telles que $v_n$ soit comprise entre $u_n$ et $w_n$ (ceci pour tout $n$)

On suppose de plus que les deux suites encadrantes, $u_n$ et $w_n$ tendent vers une même limite $\ell$.

Alors la conclusion est double :

tout d'abord $v_n$ est une suite convergente (ce que l'on ne savait pas au départ)

et en plus sa limite est aussi $\ell$.


\change
Il est intuitif de deviner quelle est la limite de la suite $u_n =2 +\frac{(-1)^n}{1+n+n^2}$.

A l'aide du théorème des gendarmes il est facile de prouver que l'intuition est juste.

%%%%%%%%%%%%%%%%%%%%%%%%%%%%%%%%%%%%%%%%%%%%%%%%%%%%%%%%%%
\diapo

 Vous pouvez à présent vous exercer avec les énoncés suivants.



\end{document}
