
%%%%%%%%%%%%%%%%%% PREAMBULE %%%%%%%%%%%%%%%%%%

\documentclass[aspectratio=169,utf8]{beamer}
%\documentclass[aspectratio=169,handout]{beamer}

\usetheme{Boadilla}
%\usecolortheme{seahorse}
\usecolortheme[RGB={245,66,24}]{structure}
\useoutertheme{infolines}

% packages
\usepackage{amsfonts,amsmath,amssymb,amsthm}
\usepackage[utf8]{inputenc}
\usepackage[T1]{fontenc}
\usepackage{lmodern}

\usepackage[francais]{babel}
\usepackage{fancybox}
\usepackage{graphicx}

\usepackage{float}
\usepackage{xfrac}

%\usepackage[usenames, x11names]{xcolor}
\usepackage{tikz}
\usepackage{pgfplots}
\usepackage{datetime}



%-----  Package unités -----
\usepackage{siunitx}
\sisetup{locale = FR,detect-all,per-mode = symbol}

%\usepackage{mathptmx}
%\usepackage{fouriernc}
%\usepackage{newcent}
%\usepackage[mathcal,mathbf]{euler}

%\usepackage{palatino}
%\usepackage{newcent}
% \usepackage[mathcal,mathbf]{euler}



% \usepackage{hyperref}
% \hypersetup{colorlinks=true, linkcolor=blue, urlcolor=blue,
% pdftitle={Exo7 - Exercices de mathématiques}, pdfauthor={Exo7}}


%section
% \usepackage{sectsty}
% \allsectionsfont{\bf}
%\sectionfont{\color{Tomato3}\upshape\selectfont}
%\subsectionfont{\color{Tomato4}\upshape\selectfont}

%----- Ensembles : entiers, reels, complexes -----
\newcommand{\Nn}{\mathbb{N}} \newcommand{\N}{\mathbb{N}}
\newcommand{\Zz}{\mathbb{Z}} \newcommand{\Z}{\mathbb{Z}}
\newcommand{\Qq}{\mathbb{Q}} \newcommand{\Q}{\mathbb{Q}}
\newcommand{\Rr}{\mathbb{R}} \newcommand{\R}{\mathbb{R}}
\newcommand{\Cc}{\mathbb{C}} 
\newcommand{\Kk}{\mathbb{K}} \newcommand{\K}{\mathbb{K}}

%----- Modifications de symboles -----
\renewcommand{\epsilon}{\varepsilon}
\renewcommand{\Re}{\mathop{\text{Re}}\nolimits}
\renewcommand{\Im}{\mathop{\text{Im}}\nolimits}
%\newcommand{\llbracket}{\left[\kern-0.15em\left[}
%\newcommand{\rrbracket}{\right]\kern-0.15em\right]}

\renewcommand{\ge}{\geqslant}
\renewcommand{\geq}{\geqslant}
\renewcommand{\le}{\leqslant}
\renewcommand{\leq}{\leqslant}
\renewcommand{\epsilon}{\varepsilon}

%----- Fonctions usuelles -----
\newcommand{\ch}{\mathop{\text{ch}}\nolimits}
\newcommand{\sh}{\mathop{\text{sh}}\nolimits}
\renewcommand{\tanh}{\mathop{\text{th}}\nolimits}
\newcommand{\cotan}{\mathop{\text{cotan}}\nolimits}
\newcommand{\Arcsin}{\mathop{\text{arcsin}}\nolimits}
\newcommand{\Arccos}{\mathop{\text{arccos}}\nolimits}
\newcommand{\Arctan}{\mathop{\text{arctan}}\nolimits}
\newcommand{\Argsh}{\mathop{\text{argsh}}\nolimits}
\newcommand{\Argch}{\mathop{\text{argch}}\nolimits}
\newcommand{\Argth}{\mathop{\text{argth}}\nolimits}
\newcommand{\pgcd}{\mathop{\text{pgcd}}\nolimits} 


%----- Commandes divers ------
\newcommand{\ii}{\mathrm{i}}
\newcommand{\dd}{\text{d}}
\newcommand{\id}{\mathop{\text{id}}\nolimits}
\newcommand{\Ker}{\mathop{\text{Ker}}\nolimits}
\newcommand{\Card}{\mathop{\text{Card}}\nolimits}
\newcommand{\Vect}{\mathop{\text{Vect}}\nolimits}
\newcommand{\Mat}{\mathop{\text{Mat}}\nolimits}
\newcommand{\rg}{\mathop{\text{rg}}\nolimits}
\newcommand{\tr}{\mathop{\text{tr}}\nolimits}


%----- Structure des exercices ------

\newtheoremstyle{styleexo}% name
{2ex}% Space above
{3ex}% Space below
{}% Body font
{}% Indent amount 1
{\bfseries} % Theorem head font
{}% Punctuation after theorem head
{\newline}% Space after theorem head 2
{}% Theorem head spec (can be left empty, meaning ‘normal’)

%\theoremstyle{styleexo}
\newtheorem{exo}{Exercice}
\newtheorem{ind}{Indications}
\newtheorem{cor}{Correction}


\newcommand{\exercice}[1]{} \newcommand{\finexercice}{}
%\newcommand{\exercice}[1]{{\tiny\texttt{#1}}\vspace{-2ex}} % pour afficher le numero absolu, l'auteur...
\newcommand{\enonce}{\begin{exo}} \newcommand{\finenonce}{\end{exo}}
\newcommand{\indication}{\begin{ind}} \newcommand{\finindication}{\end{ind}}
\newcommand{\correction}{\begin{cor}} \newcommand{\fincorrection}{\end{cor}}

\newcommand{\noindication}{\stepcounter{ind}}
\newcommand{\nocorrection}{\stepcounter{cor}}

\newcommand{\fiche}[1]{} \newcommand{\finfiche}{}
\newcommand{\titre}[1]{\centerline{\large \bf #1}}
\newcommand{\addcommand}[1]{}
\newcommand{\video}[1]{}

% Marge
\newcommand{\mymargin}[1]{\marginpar{{\small #1}}}

\def\noqed{\renewcommand{\qedsymbol}{}}


%----- Presentation ------
\setlength{\parindent}{0cm}

%\newcommand{\ExoSept}{\href{http://exo7.emath.fr}{\textbf{\textsf{Exo7}}}}

\definecolor{myred}{rgb}{0.93,0.26,0}
\definecolor{myorange}{rgb}{0.97,0.58,0}
\definecolor{myyellow}{rgb}{1,0.86,0}

\newcommand{\LogoExoSept}[1]{  % input : echelle
{\usefont{U}{cmss}{bx}{n}
\begin{tikzpicture}[scale=0.1*#1,transform shape]
  \fill[color=myorange] (0,0)--(4,0)--(4,-4)--(0,-4)--cycle;
  \fill[color=myred] (0,0)--(0,3)--(-3,3)--(-3,0)--cycle;
  \fill[color=myyellow] (4,0)--(7,4)--(3,7)--(0,3)--cycle;
  \node[scale=5] at (3.5,3.5) {Exo7};
\end{tikzpicture}}
}


\newcommand{\debutmontitre}{
  \author{} \date{} 
  \thispagestyle{empty}
  \hspace*{-10ex}
  \begin{minipage}{\textwidth}
    \titlepage  
  \vspace*{-2.5cm}
  \begin{center}
    \LogoExoSept{2.5}
  \end{center}
  \end{minipage}

  \vspace*{-0cm}
  
  % Astuce pour que le background ne soit pas discrétisé lors de la conversion pdf -> png
\begin{tikzpicture}
        \fill[opacity=0,green!60!black] (0,0)--++(0,0)--++(0,0)--++(0,0)--cycle; 
\end{tikzpicture}

% toc S'affiche trop tot :
% \tableofcontents[hideallsubsections, pausesections]
}

\newcommand{\finmontitre}{
  \end{frame}
  \setcounter{framenumber}{0}
} % ne marche pas pour une raison obscure

%----- Commandes supplementaires ------

% \usepackage[landscape]{geometry}
% \geometry{top=1cm, bottom=3cm, left=2cm, right=10cm, marginparsep=1cm
% }
% \usepackage[a4paper]{geometry}
% \geometry{top=2cm, bottom=2cm, left=2cm, right=2cm, marginparsep=1cm
% }

%\usepackage{standalone}


% New command Arnaud -- november 2011
\setbeamersize{text margin left=24ex}
% si vous modifier cette valeur il faut aussi
% modifier le decalage du titre pour compenser
% (ex : ici =+10ex, titre =-5ex

\theoremstyle{definition}
%\newtheorem{proposition}{Proposition}
%\newtheorem{exemple}{Exemple}
%\newtheorem{theoreme}{Théorème}
%\newtheorem{lemme}{Lemme}
%\newtheorem{corollaire}{Corollaire}
%\newtheorem*{remarque*}{Remarque}
%\newtheorem*{miniexercice}{Mini-exercices}
%\newtheorem{definition}{Définition}

% Commande tikz
\usetikzlibrary{calc}
\usetikzlibrary{patterns,arrows}
\usetikzlibrary{matrix}
\usetikzlibrary{fadings} 

%definition d'un terme
\newcommand{\defi}[1]{{\color{myorange}\textbf{\emph{#1}}}}
\newcommand{\evidence}[1]{{\color{blue}\textbf{\emph{#1}}}}
\newcommand{\assertion}[1]{\emph{\og#1\fg}}  % pour chapitre logique
%\renewcommand{\contentsname}{Sommaire}
\renewcommand{\contentsname}{}
\setcounter{tocdepth}{2}



%------ Figures ------

\def\myscale{1} % par défaut 
\newcommand{\myfigure}[2]{  % entrée : echelle, fichier figure
\def\myscale{#1}
\begin{center}
\footnotesize
{#2}
\end{center}}


%------ Encadrement ------

\usepackage{fancybox}


\newcommand{\mybox}[1]{
\setlength{\fboxsep}{7pt}
\begin{center}
\shadowbox{#1}
\end{center}}

\newcommand{\myboxinline}[1]{
\setlength{\fboxsep}{5pt}
\raisebox{-10pt}{
\shadowbox{#1}
}
}

%--------------- Commande beamer---------------
\newcommand{\beameronly}[1]{#1} % permet de mettre des pause dans beamer pas dans poly


\setbeamertemplate{navigation symbols}{}
\setbeamertemplate{footline}  % tiré du fichier beamerouterinfolines.sty
{
  \leavevmode%
  \hbox{%
  \begin{beamercolorbox}[wd=.333333\paperwidth,ht=2.25ex,dp=1ex,center]{author in head/foot}%
    % \usebeamerfont{author in head/foot}\insertshortauthor%~~(\insertshortinstitute)
    \usebeamerfont{section in head/foot}{\bf\insertshorttitle}
  \end{beamercolorbox}%
  \begin{beamercolorbox}[wd=.333333\paperwidth,ht=2.25ex,dp=1ex,center]{title in head/foot}%
    \usebeamerfont{section in head/foot}{\bf\insertsectionhead}
  \end{beamercolorbox}%
  \begin{beamercolorbox}[wd=.333333\paperwidth,ht=2.25ex,dp=1ex,right]{date in head/foot}%
    % \usebeamerfont{date in head/foot}\insertshortdate{}\hspace*{2em}
    \insertframenumber{} / \inserttotalframenumber\hspace*{2ex} 
  \end{beamercolorbox}}%
  \vskip0pt%
}


\definecolor{mygrey}{rgb}{0.5,0.5,0.5}
\setlength{\parindent}{0cm}
%\DeclareTextFontCommand{\helvetica}{\fontfamily{phv}\selectfont}

% background beamer
\definecolor{couleurhaut}{rgb}{0.85,0.9,1}  % creme
\definecolor{couleurmilieu}{rgb}{1,1,1}  % vert pale
\definecolor{couleurbas}{rgb}{0.85,0.9,1}  % blanc
\setbeamertemplate{background canvas}[vertical shading]%
[top=couleurhaut,middle=couleurmilieu,midpoint=0.4,bottom=couleurbas] 
%[top=fondtitre!05,bottom=fondtitre!60]



\makeatletter
\setbeamertemplate{theorem begin}
{%
  \begin{\inserttheoremblockenv}
  {%
    \inserttheoremheadfont
    \inserttheoremname
    \inserttheoremnumber
    \ifx\inserttheoremaddition\@empty\else\ (\inserttheoremaddition)\fi%
    \inserttheorempunctuation
  }%
}
\setbeamertemplate{theorem end}{\end{\inserttheoremblockenv}}

\newenvironment{theoreme}[1][]{%
   \setbeamercolor{block title}{fg=structure,bg=structure!40}
   \setbeamercolor{block body}{fg=black,bg=structure!10}
   \begin{block}{{\bf Th\'eor\`eme }#1}
}{%
   \end{block}%
}


\newenvironment{proposition}[1][]{%
   \setbeamercolor{block title}{fg=structure,bg=structure!40}
   \setbeamercolor{block body}{fg=black,bg=structure!10}
   \begin{block}{{\bf Proposition }#1}
}{%
   \end{block}%
}

\newenvironment{corollaire}[1][]{%
   \setbeamercolor{block title}{fg=structure,bg=structure!40}
   \setbeamercolor{block body}{fg=black,bg=structure!10}
   \begin{block}{{\bf Corollaire }#1}
}{%
   \end{block}%
}

\newenvironment{mydefinition}[1][]{%
   \setbeamercolor{block title}{fg=structure,bg=structure!40}
   \setbeamercolor{block body}{fg=black,bg=structure!10}
   \begin{block}{{\bf Définition} #1}
}{%
   \end{block}%
}

\newenvironment{lemme}[0]{%
   \setbeamercolor{block title}{fg=structure,bg=structure!40}
   \setbeamercolor{block body}{fg=black,bg=structure!10}
   \begin{block}{\bf Lemme}
}{%
   \end{block}%
}

\newenvironment{remarque}[1][]{%
   \setbeamercolor{block title}{fg=black,bg=structure!20}
   \setbeamercolor{block body}{fg=black,bg=structure!5}
   \begin{block}{Remarque #1}
}{%
   \end{block}%
}


\newenvironment{exemple}[1][]{%
   \setbeamercolor{block title}{fg=black,bg=structure!20}
   \setbeamercolor{block body}{fg=black,bg=structure!5}
   \begin{block}{{\bf Exemple }#1}
}{%
   \end{block}%
}


\newenvironment{miniexercice}[0]{%
   \setbeamercolor{block title}{fg=structure,bg=structure!20}
   \setbeamercolor{block body}{fg=black,bg=structure!5}
   \begin{block}{Mini-exercices}
}{%
   \end{block}%
}


\newenvironment{tp}[0]{%
   \setbeamercolor{block title}{fg=structure,bg=structure!40}
   \setbeamercolor{block body}{fg=black,bg=structure!10}
   \begin{block}{\bf Travaux pratiques}
}{%
   \end{block}%
}
\newenvironment{exercicecours}[1][]{%
   \setbeamercolor{block title}{fg=structure,bg=structure!40}
   \setbeamercolor{block body}{fg=black,bg=structure!10}
   \begin{block}{{\bf Exercice }#1}
}{%
   \end{block}%
}
\newenvironment{algo}[1][]{%
   \setbeamercolor{block title}{fg=structure,bg=structure!40}
   \setbeamercolor{block body}{fg=black,bg=structure!10}
   \begin{block}{{\bf Algorithme}\hfill{\color{gray}\texttt{#1}}}
}{%
   \end{block}%
}


\setbeamertemplate{proof begin}{
   \setbeamercolor{block title}{fg=black,bg=structure!20}
   \setbeamercolor{block body}{fg=black,bg=structure!5}
   \begin{block}{{\footnotesize Démonstration}}
   \footnotesize
   \smallskip}
\setbeamertemplate{proof end}{%
   \end{block}}
\setbeamertemplate{qed symbol}{\openbox}


\makeatother
\usecolortheme[RGB={34,139,34}]{structure}

%%%%%%%%%%%%%%%%%%%%%%%%%%%%%%%%%%%%%%%%%%%%%%%%%%%%%%%%%%%%%
%%%%%%%%%%%%%%%%%%%%%%%%%%%%%%%%%%%%%%%%%%%%%%%%%%%%%%%%%%%%%

\begin{document}




\title{{\bf Suites}}
\subtitle{Limites}


\begin{frame}
 \setcounter{framenumber}{0}
 
  \debutmontitre

  \pause

{\footnotesize
\hfill
\setbeamercovered{transparent=50}
\begin{minipage}{0.6\textwidth}
  \begin{itemize}
    \item<3-> Limite finie, limite infinie
    \item<4-> Propriétés des limites
    \item<5-> Des preuves !
    \item<6-> Limites et inégalités
  \end{itemize}
\end{minipage}
}
\vspace*{1cm}
\end{frame}
% 
% \section{Introduction}
% \begin{frame}
% 
% Pour un trajet au prix normal de $20$ euros, on achète un abonnement de train 
% à $50$ euros et on obtient chaque billet au prix réduit de $10$ euros
% 
% \pause
% \bigskip
% 
% 
%   $$\left\{\begin{array}{ccr}
%   u_n&=20n    & \qquad \text{tarif plein} \\
%   \pause
%   v_n&=10n+50 & \qquad \text{tarif réduit}
% \end{array}\right.$$
% 
% \pause
% \bigskip
% 
% La réduction est donc, en pourcentage :
% $$1-\frac{v_n}{u_n}=
% \pause
% \frac{u_n-v_n}{u_n}=
% \pause
% \frac{10n-50}{20n}=
% \pause 
% 0,5 -\frac{5}{2n}
% \pause
% \xrightarrow[n\to +\infty]{}0,5$$
%   
% 
% \pause
% 
% [[ Figure 4]]
%  
%  \end{frame}


\section{Limite finie, limite infinie}

\begin{frame}

\begin{mydefinition}
\label{def_lim}
  La suite $(u_n)_{n\in \Nn}$ a pour \defi{limite} $\ell\in \Rr$ si : 
pour tout $\epsilon >0$, il existe un entier $N$ tel que si $n\geq N$ alors 
$\lvert u_n-\ell\rvert\leq\epsilon$
\pause
\mybox{$\forall \epsilon >0 \quad \exists N \in \Nn \quad 
    \forall n \in \Nn \qquad 
    \left( n\geq N \implies \lvert u_n-\ell\rvert\leq\epsilon \right)$}
\end{mydefinition}

\pause
\bigskip
 \begin{itemize}
    \item On note \quad $\displaystyle \lim_{n\to +\infty}u_n=\ell$ \quad ou \quad $u_n\xrightarrow[n\to +\infty]{} \ell$
\pause
\medskip
    \item $|u_n-\ell| \le \epsilon \iff \ell-\epsilon \le u_n \le \ell +\epsilon$
 \end{itemize}
\end{frame}

\begin{frame}

\mybox{$\forall \epsilon >0 \quad \exists N \in \Nn \quad 
    \forall n \in \Nn \qquad 
    \left( n\geq N \implies \lvert u_n-\ell\rvert\leq\epsilon \right)$}

\pause    
    
\myfigure{1.2}{\tikzinput{fig_suites05}}


\end{frame}


\begin{frame}
\begin{mydefinition}  
 La suite $(u_n)_{n\in \Nn}$ \defi{tend vers $+\infty$} si :
\mybox{$\forall A >0 \quad \exists N \in \Nn \quad 
    \forall n \in \Nn \qquad \left( n\geq N \implies u_n\geq A \right)$}


    
\uncover<3->{La suite $(u_n)_{n\in \Nn}$ \defi{tend vers $-\infty$} 
si la suite $(-u_n)_{n\in \Nn}$ tend vers $+\infty$}
\end{mydefinition}



\pause

\uncover<2->{
\myfigure{0.9}{\tikzinput{fig_suites05bis}}
}

\end{frame}



\begin{frame}
\begin{mydefinition}
Une suite $(u_n)_{n\in \Nn}$ est \defi{convergente} si elle admet une limite \evidence{finie}

Elle est \defi{divergente} sinon (y compris donc lorsqu'elle tend vers $\pm \infty$)
\end{mydefinition}

\pause
\bigskip

\begin{proposition}
  Si une suite est convergente, sa limite est unique
\end{proposition}

\end{frame}


\section{Propriétés des limites}

\begin{frame}

\begin{proposition}[Opérations sur les limites]
\label{prop:suitelimite}
%Soient $(u_n)_{n\in \Nn}$ et $(v_n)_{n\in \Nn}$ deux suites convergentes.
\begin{enumerate}
\setlength{\itemsep}{9pt} 
 \item Si $\displaystyle \lim_{n\to +\infty}u_n=\ell$, où $\ell\in \Rr$ et $\lambda \in \Rr$ 
 on a $\displaystyle \lim_{n\to +\infty}\lambda u_n=\lambda \ell$
 
 \pause
 
  \item Si $\displaystyle \lim_{n\to +\infty}u_n=\ell$ et $\displaystyle \lim_{n\to +\infty}v_n=\ell'$, où $\ell,\ell'\in \Rr$, alors
\mybox{$\displaystyle  \lim_{n\to +\infty}\left(u_n+v_n\right)=\ell+\ell' \pause
\;\; \text{ et } \; \lim_{n\to +\infty}\left(u_n\times v_n\right)=\ell\times \ell'$\pause}
    % \begin{align*}
      % \lim_{n\to +\infty}\left(u_n+v_n\right)&=\ell+\ell'&\\
      % \lim_{n\to +\infty}\left(u_n\times v_n\right)&=\ell\times \ell'
    % \end{align*}
    
\pause    
    
  \item Si $\displaystyle \lim_{n\to +\infty}u_n=\ell$ où $\ell\in \Rr^*$ 
alors $u_n\neq 0 $ pour $n$ assez grand et \\
\hfil \hfil $\displaystyle \lim_{n\to +\infty}\frac{1}{u_n}=\frac{1}{\ell}$
\end{enumerate}
\end{proposition}

\medskip
\pause

Exemple : Si $u_n \to \ell$ avec $\ell \neq \pm 1$, alors 
\vspace*{-2ex}
$$u_n(1-3u_n)-\frac{1}{u_n^2-1} \xrightarrow[n\to +\infty]{} \ell(1-3\ell) - \frac{1}{\ell^2-1}$$

\end{frame}


\begin{frame}
\begin{proposition}[Opérations sur les limites infinies]
\label{prop:suiteinfty}
\small
  \begin{enumerate}
    \setlength{\itemsep}{9pt} 
    \item Si $\displaystyle \lim_{n\to +\infty}u_n=+\infty$ alors $\displaystyle \lim_{n\to +\infty}\frac{1}{u_n}=0$
    \pause
    \item Si $\displaystyle \lim_{n\to +\infty}u_n= 0$ et $u_n>0$ pour $n$ assez grand 
    alors $\displaystyle \lim_{n\to +\infty}\frac{1}{u_n}=+\infty$
    \pause    
    
    \item Si $\displaystyle \lim_{n\to +\infty}u_n=+\infty$ et $(v_n)_{n\in \Nn}$ est minorée alors 
    $\displaystyle \lim_{n\to +\infty}\left(u_n+v_n\right)=+\infty$
    \pause    
    \item Si  $\displaystyle \lim_{n\to +\infty}u_n=+\infty$ et $(v_n)_{n\in \Nn}$ est minorée par un nombre $\lambda>0$ 
alors \\ \hfil  \hfil $\displaystyle \lim_{n\to +\infty}\left(u_n\times v_n\right)=+\infty$
    
  \end{enumerate}
\end{proposition}

\end{frame}

\section{Des preuves !}

\begin{frame}

\begin{proposition}
Toute suite convergente est bornée
\end{proposition}
\pause
\begin{proof}
\begin{itemize}
  \item Soit $(u_n)$ une suite convergeant vers $\ell \in \Rr$
\pause  
  \item Définition de limite avec $\epsilon=1$ : 
  \uncover<4->{ il existe $N$ t.q. pour $n\geq N$ on a 
$\lvert u_n-\ell \rvert \leq 1$}
\uncover<5->{  
  \item Pour $n\geq N$ on a 
\[\lvert u_n \rvert 
\uncover<6->{=  \lvert \ell+(u_n-\ell) \rvert }
\uncover<7->{\leq \lvert \ell \rvert + \lvert u_n-\ell  \rvert }
\uncover<8->{\leq \lvert \ell \rvert +1 \] }
}
\vspace*{-3ex}
\uncover<9->{  
  \item On pose $M= \max(\lvert u_0 \rvert,\lvert u_1\rvert, \cdots , \lvert u_{N-1}\rvert , \lvert \ell \rvert +1)$
  }
\uncover<10->{    
  \item $\forall n \in \Nn \quad \lvert u_n \rvert \leq M$
  }
\end{itemize}
\vspace*{-8ex}
\uncover<3->{ \myfigure{0.6}{\hspace*{8em}\tikzinput{fig_suites09}}}
\vspace*{-5ex}
\end{proof}
\end{frame}

\begin{frame}
\begin{proposition}
\label{prop:suitebornezero}
Si la suite $(u_n)_{n\in \Nn}$ est bornée et $\lim_{n\to +\infty}v_n=0$ 
alors \\
\hfil \hfil $\lim_{n\to +\infty}\left(u_n\times v_n\right)=0$
\end{proposition}
\pause
Exemple : $u_n=\cos n$ et $v_n=\frac{1}{\sqrt{n}}$, alors $u_nv_n \to 0$

\pause
\begin{proof}
\begin{itemize}
  \item $(u_n)$ est bornée, il existe $M>0$ tel que  $\lvert u_n \rvert\leq M$
  
\pause  
  \item Fixons $\epsilon >0$
\pause
  \begin{itemize}
    \item Définition de limite à la suite $(v_n)_{n\in \Nn}$
\pause     
    \item Pour $\epsilon'=\frac{\epsilon}{M}$
\pause     
    \item Il existe  $N$ tel que $n\geq N$ implique $ \lvert v_n  \rvert\leq  \epsilon'$
  \end{itemize}
    
  
\pause 

  \item Pour $n\geq N$ on a :
\[ \lvert u_nv_n \rvert= \lvert u_n \rvert \lvert v_n \rvert \leq M\times \epsilon'=
M\times\frac{\epsilon}{M} =\epsilon\]
  
\pause  
  \item Ce qui montre $u_n\times v_n \to 0$ \qedhere
\end{itemize}

\end{proof}

\end{frame}

\begin{frame}
%\centerline{\og{}\emph{Si \quad $\lim u_n = \ell$ \quad et \quad $\lim v_n = \ell'$ \quad alors \quad $\lim u_n v_n = \ell \ell'$.}\fg{}}

\begin{proposition}
	Si \quad $\lim u_n = \ell$ \quad et \quad $\lim v_n = \ell'$ \quad alors \quad $\lim u_n v_n = \ell \ell'$
\end{proposition}
\pause
\begin{proof}

\begin{itemize}
  \item $u_nv_n-\ell\ell'=(u_n-\ell)v_n+\ell(v_n-\ell')$
\pause  
  \item $\ell(v_n-\ell') \to 0$
\pause  
  \item $(u_n-\ell)v_n \to 0$ 
\pause      
  \begin{itemize} 
    \item $u_n-\ell \to 0$
\pause      
    \item $v_n \to \ell'$ donc $(v_n)$ est bornée
\pause      
    \item par la proposition précédente : $(u_n-\ell)v_n \to 0$
  \end{itemize}
\pause  
  \item Conclusion : $u_nv_n-\ell\ell' \to 0$
\end{itemize}


\end{proof}

\end{frame}

\section{Formes indéterminées} 

\begin{frame}

\hfill\evidence{Formes indéterminées}

\pause  

  \begin{enumerate}
    \item \og{}$+\infty-\infty$\fg{} 
    \pause 
\vspace*{-2ex}  
   $$  \begin{array}{rl}
\displaystyle   \lim_{n\to +\infty}\left(e^n-\ln n\right)&=+\infty\\[3mm]
 \pause                
\displaystyle         \lim_{n\to +\infty}\left(n-n^2\right)&=-\infty \\[4mm]       
\pause           
\displaystyle         \lim_{n\to +\infty}\left(\left(n+\frac{1}{n}\right)-n\right)&=0
      \end{array}$$
\pause      
\vspace*{-2ex}        
    \item \og{} $0\times \infty$\fg{} 
\pause
    $$  \begin{array}{rc}
\displaystyle        \lim_{n\to +\infty} \frac{1}{\ln n} \times e^n &=+\infty\\ [4mm]
\displaystyle        \lim_{n\to +\infty} \frac{1}{n} \times \ln n &=0 \\[4mm] 
\displaystyle        \lim_{n\to +\infty} \frac{1}{n} \times (n+1) &=1
      \end{array}$$

      \medskip

\pause                         
    \item \og{} $\frac{\infty}{\infty}$\fg{}, \og{} $\frac{0}{0}$\fg{}, \og{} $1^\infty$\fg{}, ...
  \end{enumerate}
%\end{exemple}

\end{frame}

\section{Limite et inégalités}

\begin{frame}
	\begin{proposition}
\label{prop:lim_ineg}
  \begin{enumerate}
    \item Soient $(u_n)_{n\in \Nn}$ et $(v_n)_{n\in \Nn}$ deux suites convergentes 
telles que : $\forall n \in \Nn$, $u_n\leq v_n$. Alors
      \[\lim_{n\to +\infty} u_n \leq \lim_{n\to +\infty} v_n\]
      
\bigskip
\pause

    \item  Soient $(u_n)_{n\in \Nn}$ et $(v_n)_{n\in \Nn}$ deux suites telles que 
$\lim_{n\to +\infty} u_n=+\infty$ et  $\forall n \in \Nn$, $v_n \geq u_n$. 
Alors $$\lim_{n\to +\infty} v_n=+\infty$$
  \end{enumerate}
\end{proposition}
\end{frame}

\begin{frame}

\begin{theoreme}[des  \og{}gendarmes\fg{}]   
Si $(u_n)_{n\in \Nn}$, $(v_n)_{n\in \Nn}$ et $(w_n)_{n\in \Nn}$ sont trois suites telles que pour tout $n$ :
\[ u_n\leq v_n\leq w_n \qquad \text{ et } \qquad \lim_{n\to +\infty}u_n=\ell=\lim_{n\to +\infty}w_n\]
alors \ la suite $(v_n)_{n\in \Nn}$ est convergente \ \ et \ \ $\lim_{n\to +\infty} v_n=\ell$  
  
\end{theoreme}

\pause

\myfigure{1}{\tikzinput{fig_suites06}}  
  
\pause


Exemple : Trouver la limite de $\displaystyle u_n =2 +\frac{(-1)^n}{1+n+n^2}$


\end{frame}



%%%%%%%%%%%%%%%%%%%%%%%%%%%%%%%%%%%%%%%%%%%%%%%%%%%%%%%%%%%%%%%
\section{Mini-exercices}

 \begin{frame}
 \begin{miniexercice}
	 \begin{enumerate}
  \item Soit $(u_n)_{n\in \Nn}$ la suite définie par $u_n = \frac{2n+1}{n+2}$. 
En utilisant la définition de la limite montrer que $\lim_{n\to+\infty} u_n = 2$. 
Trouver explicitement un rang à partir duquel $1,999 \le u_n \le 2,001$. 

    \item  Déterminer la limite $\ell$ de la suite $(u_n)_{n\in \Nn}$ de terme général : 
$\frac{n+\cos n}{n-\sin n}$ et trouver un entier $N$ 
tel que si $n\geq N$, on ait $|u_n-\ell|\leq 10^{-2}$. 

  \item La suite $(u_n)_{n\in \Nn}$ de terme général $(-1)^ne^{n}$ admet-elle une limite ? 
Et la suite de terme général $\frac{1}{u_n}$ ? 

  \item  Déterminer la limite de la suite $(u_n)_{n\geq 1}$ de terme général $\sqrt{n+1}-\sqrt{n}$.  
   Idem avec $v_n = \frac{\cos n}{\sin n + \ln n}$. Idem avec $w_n=\frac{n!}{n^n}$.
  
%  \item \ [[???]] 
  
  %La suite $(u_n)_{n\geq 1}$ de terme général $n^{-2+(-1)^n}$ admet-elle une limite ?
%   \item  Déterminer la limite de la suite $(u_n)_{n\geq 1}$ de terme général 
% $\frac{E(\pi)+E(2\pi)+\cdots+E(n\pi)}{n^2}$. 
  \end{enumerate}

 \end{miniexercice}
 \end{frame}



\end{document}
