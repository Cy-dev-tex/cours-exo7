
%%%%%%%%%%%%%%%%%% PREAMBULE %%%%%%%%%%%%%%%%%%


\documentclass[12pt]{article}

\usepackage{amsfonts,amsmath,amssymb,amsthm}
\usepackage[utf8]{inputenc}
\usepackage[T1]{fontenc}
\usepackage[francais]{babel}


% packages
\usepackage{amsfonts,amsmath,amssymb,amsthm}
\usepackage[utf8]{inputenc}
\usepackage[T1]{fontenc}
%\usepackage{lmodern}

\usepackage[francais]{babel}
\usepackage{fancybox}
\usepackage{graphicx}

\usepackage{float}

%\usepackage[usenames, x11names]{xcolor}
\usepackage{tikz}
\usepackage{datetime}

\usepackage{mathptmx}
%\usepackage{fouriernc}
%\usepackage{newcent}
\usepackage[mathcal,mathbf]{euler}

%\usepackage{palatino}
%\usepackage{newcent}


% Commande spéciale prompteur

%\usepackage{mathptmx}
%\usepackage[mathcal,mathbf]{euler}
%\usepackage{mathpple,multido}

\usepackage[a4paper]{geometry}
\geometry{top=2cm, bottom=2cm, left=1cm, right=1cm, marginparsep=1cm}

\newcommand{\change}{{\color{red}\rule{\textwidth}{1mm}\\}}

\newcounter{mydiapo}

\newcommand{\diapo}{\newpage
\hfill {\normalsize  Diapo \themydiapo \quad \texttt{[\jobname]}} \\
\stepcounter{mydiapo}}


%%%%%%% COULEURS %%%%%%%%%%

% Pour blanc sur noir :
%\pagecolor[rgb]{0.5,0.5,0.5}
% \pagecolor[rgb]{0,0,0}
% \color[rgb]{1,1,1}



%\DeclareFixedFont{\myfont}{U}{cmss}{bx}{n}{18pt}
\newcommand{\debuttexte}{
%%%%%%%%%%%%% FONTES %%%%%%%%%%%%%
\renewcommand{\baselinestretch}{1.5}
\usefont{U}{cmss}{bx}{n}
\bfseries

% Taille normale : commenter le reste !
%Taille Arnaud
%\fontsize{19}{19}\selectfont

% Taille Barbara
%\fontsize{21}{22}\selectfont

%Taille François
%\fontsize{25}{30}\selectfont

%Taille Pascal
%\fontsize{25}{30}\selectfont

%Taille Laura
%\fontsize{30}{35}\selectfont


%\myfont
%\usefont{U}{cmss}{bx}{n}

%\Huge
%\addtolength{\parskip}{\baselineskip}
}


% \usepackage{hyperref}
% \hypersetup{colorlinks=true, linkcolor=blue, urlcolor=blue,
% pdftitle={Exo7 - Exercices de mathématiques}, pdfauthor={Exo7}}


%section
% \usepackage{sectsty}
% \allsectionsfont{\bf}
%\sectionfont{\color{Tomato3}\upshape\selectfont}
%\subsectionfont{\color{Tomato4}\upshape\selectfont}

%----- Ensembles : entiers, reels, complexes -----
\newcommand{\Nn}{\mathbb{N}} \newcommand{\N}{\mathbb{N}}
\newcommand{\Zz}{\mathbb{Z}} \newcommand{\Z}{\mathbb{Z}}
\newcommand{\Qq}{\mathbb{Q}} \newcommand{\Q}{\mathbb{Q}}
\newcommand{\Rr}{\mathbb{R}} \newcommand{\R}{\mathbb{R}}
\newcommand{\Cc}{\mathbb{C}} 
\newcommand{\Kk}{\mathbb{K}} \newcommand{\K}{\mathbb{K}}

%----- Modifications de symboles -----
\renewcommand{\epsilon}{\varepsilon}
\renewcommand{\Re}{\mathop{\text{Re}}\nolimits}
\renewcommand{\Im}{\mathop{\text{Im}}\nolimits}
%\newcommand{\llbracket}{\left[\kern-0.15em\left[}
%\newcommand{\rrbracket}{\right]\kern-0.15em\right]}

\renewcommand{\ge}{\geqslant}
\renewcommand{\geq}{\geqslant}
\renewcommand{\le}{\leqslant}
\renewcommand{\leq}{\leqslant}

%----- Fonctions usuelles -----
\newcommand{\ch}{\mathop{\mathrm{ch}}\nolimits}
\newcommand{\sh}{\mathop{\mathrm{sh}}\nolimits}
\renewcommand{\tanh}{\mathop{\mathrm{th}}\nolimits}
\newcommand{\cotan}{\mathop{\mathrm{cotan}}\nolimits}
\newcommand{\Arcsin}{\mathop{\mathrm{Arcsin}}\nolimits}
\newcommand{\Arccos}{\mathop{\mathrm{Arccos}}\nolimits}
\newcommand{\Arctan}{\mathop{\mathrm{Arctan}}\nolimits}
\newcommand{\Argsh}{\mathop{\mathrm{Argsh}}\nolimits}
\newcommand{\Argch}{\mathop{\mathrm{Argch}}\nolimits}
\newcommand{\Argth}{\mathop{\mathrm{Argth}}\nolimits}
\newcommand{\pgcd}{\mathop{\mathrm{pgcd}}\nolimits} 

\newcommand{\Card}{\mathop{\text{Card}}\nolimits}
\newcommand{\Ker}{\mathop{\text{Ker}}\nolimits}
\newcommand{\id}{\mathop{\text{id}}\nolimits}
\newcommand{\ii}{\mathrm{i}}
\newcommand{\dd}{\mathrm{d}}
\newcommand{\Vect}{\mathop{\text{Vect}}\nolimits}
\newcommand{\Mat}{\mathop{\mathrm{Mat}}\nolimits}
\newcommand{\rg}{\mathop{\text{rg}}\nolimits}
\newcommand{\tr}{\mathop{\text{tr}}\nolimits}
\newcommand{\ppcm}{\mathop{\text{ppcm}}\nolimits}

%----- Structure des exercices ------

\newtheoremstyle{styleexo}% name
{2ex}% Space above
{3ex}% Space below
{}% Body font
{}% Indent amount 1
{\bfseries} % Theorem head font
{}% Punctuation after theorem head
{\newline}% Space after theorem head 2
{}% Theorem head spec (can be left empty, meaning ‘normal’)

%\theoremstyle{styleexo}
\newtheorem{exo}{Exercice}
\newtheorem{ind}{Indications}
\newtheorem{cor}{Correction}


\newcommand{\exercice}[1]{} \newcommand{\finexercice}{}
%\newcommand{\exercice}[1]{{\tiny\texttt{#1}}\vspace{-2ex}} % pour afficher le numero absolu, l'auteur...
\newcommand{\enonce}{\begin{exo}} \newcommand{\finenonce}{\end{exo}}
\newcommand{\indication}{\begin{ind}} \newcommand{\finindication}{\end{ind}}
\newcommand{\correction}{\begin{cor}} \newcommand{\fincorrection}{\end{cor}}

\newcommand{\noindication}{\stepcounter{ind}}
\newcommand{\nocorrection}{\stepcounter{cor}}

\newcommand{\fiche}[1]{} \newcommand{\finfiche}{}
\newcommand{\titre}[1]{\centerline{\large \bf #1}}
\newcommand{\addcommand}[1]{}
\newcommand{\video}[1]{}

% Marge
\newcommand{\mymargin}[1]{\marginpar{{\small #1}}}



%----- Presentation ------
\setlength{\parindent}{0cm}

%\newcommand{\ExoSept}{\href{http://exo7.emath.fr}{\textbf{\textsf{Exo7}}}}

\definecolor{myred}{rgb}{0.93,0.26,0}
\definecolor{myorange}{rgb}{0.97,0.58,0}
\definecolor{myyellow}{rgb}{1,0.86,0}

\newcommand{\LogoExoSept}[1]{  % input : echelle
{\usefont{U}{cmss}{bx}{n}
\begin{tikzpicture}[scale=0.1*#1,transform shape]
  \fill[color=myorange] (0,0)--(4,0)--(4,-4)--(0,-4)--cycle;
  \fill[color=myred] (0,0)--(0,3)--(-3,3)--(-3,0)--cycle;
  \fill[color=myyellow] (4,0)--(7,4)--(3,7)--(0,3)--cycle;
  \node[scale=5] at (3.5,3.5) {Exo7};
\end{tikzpicture}}
}



\theoremstyle{definition}
%\newtheorem{proposition}{Proposition}
%\newtheorem{exemple}{Exemple}
%\newtheorem{theoreme}{Théorème}
\newtheorem{lemme}{Lemme}
\newtheorem{corollaire}{Corollaire}
%\newtheorem*{remarque*}{Remarque}
%\newtheorem*{miniexercice}{Mini-exercices}
%\newtheorem{definition}{Définition}




%definition d'un terme
\newcommand{\defi}[1]{{\color{myorange}\textbf{\emph{#1}}}}
\newcommand{\evidence}[1]{{\color{blue}\textbf{\emph{#1}}}}



 %----- Commandes divers ------

\newcommand{\codeinline}[1]{\texttt{#1}}

%%%%%%%%%%%%%%%%%%%%%%%%%%%%%%%%%%%%%%%%%%%%%%%%%%%%%%%%%%%%%
%%%%%%%%%%%%%%%%%%%%%%%%%%%%%%%%%%%%%%%%%%%%%%%%%%%%%%%%%%%%%



\begin{document}

\debuttexte


%%%%%%%%%%%%%%%%%%%%%%%%%%%%%%%%%%%%%%%%%%%%%%%%%%%%%%%%%%%
\diapo

\change

\change

Cette partie est consacrée aux suites définies par une relation de récurrence
à l'aide d'une fonction

Après des résultats généraux,

\change

On consacre l'essentiel de notre travail au cas où la fonction est croissante,

\change

Puis au cas où la fonction est décroissante.



%%%%%%%%%%%%%%%%%%%%%%%%%%%%%%%%%%%%%%%%%%%%%%%%%%%%%%%%%%
\diapo

Pour définir une suite récurrente il faut 

\change

une fonction $f : \Rr \to \Rr$,

\change

son premier terme $u_0$

\change

et une relation permettant de calculer les termes de proche en proche.

\change

La suite s'écrit ainsi :
le terme initial $u_0$ 

\change

$u_1 = f(u_0),$

\change

$u_2 = f(u_1) =f(f(u_0))$

\change

$u_3 = f(u_2)=f(f(u_1)) =f(f(f(u_0))),\ldots$

\change

Que l'on résume ainsi :
terme initial et relation de récurrence.

\change



Soit la fonction définie par $f(x)=1+\sqrt{x}$. 

\change

Fixons $u_0= 2$ 

\change

et définissons pour $n\ge0$ :
$u_{n+1}=f(u_n)  $. C'est-à-dire $u_{n+1}=1+\sqrt{u_n}$.

\change

Alors les premiers termes de la suite sont :

$2$,

\change

$f(2)=1+\sqrt{2}$, 

\change

$f(f(2))= 1+\sqrt{1+\sqrt{2}}$

\change

Puis $1+\sqrt{1+\sqrt{1+\sqrt{2}}}$

\change 

etc...

Comme vous le voyez le comportement peut très vite devenir complexe.


%%%%%%%%%%%%%%%%%%%%%%%%%%%%%%%%%%%%%%%%%%%%%%%%%%%%%%%%%%%
\diapo

Voici un résultat essentiel concernant la limite si elle existe.


Si $f$ est une fonction continue et la suite récurrente $(u_n)$ converge vers un réel $\ell$, alors
$\ell$ est une solution de l'équation :
$f(\ell)=\ell$  

\change

Une valeur $\ell$, vérifiant $f(\ell)=\ell$ s'appelle un \defi{point fixe} de la fonction $f$.

Graphiquement pour trouver les points fixes :
on trace le graphe de $f$ ainsi que la bissectrice d'équation $(y=x)$.

On obtient ici $3$ points d'intersections,

et les points fixes sont les abscisses de ces points d'intersection.


%%%%%%%%%%%%%%%%%%%%%%%%%%%%%%%%%%%%%%%%%%%%%%%%%%%%%%%%%%
\diapo

Revenons sur cette proposition :

si la limite existe alors c'est un point fixe de $f$.

Ainsi, si on arrive à montrer que la limite existe alors cette proposition permet 
de calculer des candidats à être cette limite.

La preuve est très simple et mérite d'être refaite à chaque fois. 

\change


Lorsque $n\to +\infty$, $u_n\to \ell$ et donc aussi $u_{n+1} \to \ell$
car $(u_{n+1})$ est la même suite que $(u_n)$ mais avec un décalage d'un rang.


\change

Comme $u_n\to \ell$ et que la fonction $f$ est continue alors 
la suite $(f(u_n)) \to f(\ell)$.

\change

La relation de récurrence $u_{n+1} = f(u_n)$ devient à la limite (lorsque $n\to+\infty$) : $\ell=f(\ell)$.

Ainsi $\ell$ est un point fixe.

Nous allons maintenant étudier en détail deux cas particuliers fondamentaux : lorsque la fonction est croissante, 
puis lorsque la fonction est décroissante.



%%%%%%%%%%%%%%%%%%%%%%%%%%%%%%%%%%%%%%%%%%%%%%%%%%%%%%%%%%%
\diapo

Commençons par remarquer que pour une fonction croissante, le comportement de la suite 
$(u_n)$ définie par récurrence est assez simple.

Si $f : [a,b] \to [a,b]$ est une fonction continue et croissante, 
alors quelque soit $u_0 \in [a,b]$, la suite récurrente $(u_n)$ est
monotone (c-à-d croissante ou bien décroissante) et converge vers $\ell \in [a,b]$ qui vérifie comme on l'a vu 
$f(\ell)=\ell$.

On sait donc que pour le cas d'une fonction croissante la suite converge vers un point fixe.

\change

Il y a cependant  une hypothèse importante qui est un peu cachée : $f$ va de l'intervalle $[a,b]$ dans
lui-même.

\change

Dans la pratique, pour appliquer cette proposition, il faut commencer par trouver comme ici un intervalle fermé et borné
$[a,b]$ et vérifier que $f([a,b]) \subset [a,b]$.

\change

On peut savoir comment varie la suite :

Si $u_1\ge u_0$ alors comme $f$ est croissante $u_2 = f(u_1)\ge f(u_0)=u_1$,
Partant de $u_2\ge u_1$ on en déduit $u_3 \ge u_2$,...

La suite  est alors croissante.
 
 
\change

 Si $u_1 \le u_0$ alors on montre de la même façon que la suite est décroissante.
 
 \change
 

La preuve de cette proposition est une conséquence des résultats précédents. 
Par exemple si $u_1\ge u_0$ alors la suite $(u_n)$ est croissante, elle est majorée par 
l'extrémité $b$, donc elle converge vers un réel $\ell$. 

\change

Par la proposition de la page précédente, alors $f(\ell)=\ell$.


Si $u_1 \le u_0$, alors $(u_n)$ est une décroissante et minorée par l'extrémité $a$, et la conclusion est la même.


%%%%%%%%%%%%%%%%%%%%%%%%%%%%%%%%%%%%%%%%%%%%%%%%%%%%%%%%%%
\diapo

Nous allons étudier un exemple en détails.

$f$ est la fonction définie par $f(x)=\frac14(x^2-1)(x-2)+x$ et $u_0$ le terme initial est un élément de $[0,2]$.

La suite $(u_n)$ est définie par récurrence : $u_{n+1}=f(u_n)$.

\change

La première chose à faire est d'étudier la fonction et tracer son graphe.

\change

Ici $f$ est une fonction continue sur $\Rr$.

\change

Elle est même dérivable sur $\Rr$ et on vérifie que sa dérivée est strictement positive.

\change

Ce qui fait que $f$ est strictement croissante.

\change

N'oubliez pas que l'on doit trouver un intervalle fermé et borné $[a,b]$ tel que 
$f([a,b]) \subset [a,b]$.

Comme $f(0)=\frac12$ et $f(2)=2$ et que la fonction est croissante alors $f([0,2]) \subset [0,2]$.

On restreint donc notre fonction à l'intervalle $[0,2]$ au départ et à l'arrivée.


%%%%%%%%%%%%%%%%%%%%%%%%%%%%%%%%%%%%%%%%%%%%%%%%%%%%%%%%%%%
\diapo


Voici comment tracer la suite : on trace le graphe de $f$ et la bissectrice $(y=x)$.
On part d'une valeur $u_0$ (en rouge) sur l'axe des abscisses, la valeur $u_1=f(u_0)$ se lit
sur l'axe des ordonnées, mais on reporte la valeur de $u_1$ sur l'axe des 
abscisses par symétrie par rapport à la bissectrice.
On recommence : $u_2=f(u_1)$ se lit sur l'axe des ordonnées et on le reporte sur l'axe des abscisses, etc.
On obtient ainsi une sorte d'escalier, et graphiquement on conjecture que la suite est croissante et tend vers $1$.
Si on part d'une autre valeur initiale $u_0'$ (en vert), c'est le même principe, mais cette fois on obtient
un escalier qui descend.



Le graphe de $f$ joue un rôle très important, il faut le tracer même si on ne le demande pas explicitement. 
Il permet de se faire une idée très précise du comportement de la suite :
Est-elle croissante ? Est-elle positive ? Semble-t-elle converger ? Vers quelle limite ? 
Ces indications sont essentielles pour savoir ce qu'il faut montrer lors de l'étude de la suite.

%%%%%%%%%%%%%%%%%%%%%%%%%%%%%%%%%%%%%%%%%%%%%%%%%%%%%%%%%%
\diapo


L'étape suivante est de calculer les points fixes.

\change

Il s'agit donc de trouver les valeurs $x$ qui vérifient l'équation $(f(x)=x)$.
autrement dit $(f(x)-x=0)$, mais 

\change

comme $f(x)-x=\frac14 (x^2-1)(x-2)$

\change
alors les points fixes sont les $\{-1,1,2\}$.

\change

La limite de $(u_n)$, si elle existe est donc à chercher parmi ces $3$ valeurs.

\change

Débarrassons nous d'un cas trivial :
si $u_0=1$ ou bien si $u_0=2$.
  
  
\change

  Alors $u_1=f(u_0)=u_0$ et par récurrence la suite $(u_n)$ est constante (et converge 
  donc vers $u_0$).

%%%%%%%%%%%%%%%%%%%%%%%%%%%%%%%%%%%%%%%%%%%%%%%%%%%%%%%%%%%
\diapo

Le deuxième cas à étudier c'est lorsque $0 \le u_0 <1$. 

\change

Comme $f([0,1])\subset [0,1]$, la fonction $f$ se restreint en une fonction
    $f : [0,1] \to [0,1]$.

    \change
    
De plus sur $[0,1]$, $f(x)-x\ge0$. Cela se déduit de l'étude de $f$ ou 
    directement de l'expression de $f(x)-x$ vu auparavant.
    
\change
    
Pour le terme initial $u_0 \in [0,1[$, alors $u_1 = f(u_0) \ge u_0$ d'après le point précédent. 

Et maintenant comme $f$ est croissante, par récurrence, la suite $(u_n)$ est croissante.

\change

La suite $(u_n)$ est croissante et majorée par l'extrémité $1$ de l'intervalle, 
donc elle converge. J'appelle $\ell$ sa limite.
 
\change

D'une part $\ell$ doit être un point fixe de $f$ : $f(\ell)=\ell$. Donc $\ell \in \{-1,1,2\}$.

\change

D'autre part la suite $(u_n)$ étant croissante avec $u_0\ge0$ et majorée par $1$, donc $\ell \in [0,1]$.

\change

Il n'y a qu'une seule possibilité !

Conclusion : si $0\le u_0<1$ alors $(u_n)$ converge vers $\ell = 1$. 


\change

\change

Le troisième cas, pour lequel $1 < u_0 < 2$ 
se traite de manière tout à fait similaire.

\change

  La fonction $f$ se restreint en $f : [1,2] \to [1,2]$. Sur l'intervalle $[1,2]$, $f$ est croissante
  mais cette fois $f(x) \le x$. 
  
\change

Donc $u_1 \le u_0$, et la suite $(u_n)$ est décroissante.
  La suite $(u_n)$ étant minorée par $1$, elle converge.

\change

Il n'y a encore qu'une possibilité : $(u_n)$ converge vers $\ell=1$.
  
  
%%%%%%%%%%%%%%%%%%%%%%%%%%%%%%%%%%%%%%%%%%%%%%%%%%%%%%%%%%%
\diapo

Passons au cas où la fonction $f$ est décroissante,
qui est un peu plus compliqué.

On part de $f : [a,b] \to [a,b]$ une fonction continue et décroissante.
Soit $u_0 \in [a,b]$ et la suite récurrente $(u_n)$ définie par $u_{n+1}=f(u_n)$.

Alors :

(1) La sous-suite $(u_{2n})$ des termes de rang pair converge vers une limite $\ell$  vérifiant $f\circ f(\ell)=\ell$.

(2) La sous-suite $(u_{2n+1})$ des termes de rang impair converge vers une limite $\ell'$ vérifiant $f\circ f(\ell')=\ell'$.


Il se peut (ou pas !) que $\ell=\ell'$.

La preuve se déduit du cas croissant.
La fonction $f$ étant décroissante, la fonction $f\circ f$ est croissante.
Et on applique les résultats précédent à la fonction $f\circ f$ et 
à la sous-suite des termes de rang pair, puis à la sous-suite des termes de rang impair.


%%%%%%%%%%%%%%%%%%%%%%%%%%%%%%%%%%%%%%%%%%%%%%%%%%%%%%%%%%%
\diapo

Voici un exemple pour la cas décroissant.

La fonction est $f(x)=1+\frac1x$

le terme initial est une valeur strictement positive.

Et la suite est définie par récurrence : $u_{n+1} = 1 + \frac{1}{u_n}$.

\change

On commence par une étude de la fonction $f$

La fonction $f : ]0,+\infty[\to ]0,+\infty[$ est 
  une fonction continue et strictement décroissante.
  

  \change
  
  On veut ensuite calculer les  points fixes de $f\circ f$.
  
  On commence par calculer $f\circ f$ :
  
  $$f\circ f(x)= f\big( f(x)\big)= f\big(1+\frac1x\big)= 1+ \frac{1}{1+\frac1x}=1+\frac{x}{x+1} = \frac{2x+1}{x+1}$$
  
  
  \change
  
  Donc 
  $$f\circ f(x)=x \iff  \frac{2x+1}{x+1} = x \iff x^2-x-1=0 \iff x \in 
\left\{\frac{1-\sqrt{5}}{2},\frac{1+\sqrt{5}}{2}\right\}$$
  
  \change
  
  Comme la suite est positive, la limite doit être positive, et le seul point fixe à considérer est $\ell=\frac{1+\sqrt{5}}{2}$.
  
  Attention ! Il y a un unique point fixe, mais on ne peut pas conclure 
  à ce stade car $f$ est définie sur $]0,+\infty[$ qui n'est pas un intervalle fermé bornée.
   
  


%%%%%%%%%%%%%%%%%%%%%%%%%%%%%%%%%%%%%%%%%%%%%%%%%%%%%%%%%%%
\diapo

Le principe pour tracer la suite est le même qu'auparavant : on place $u_0$,
on trace $u_1=f(u_0)$ sur l'axe des ordonnées et on le reporte par symétrie sur l'axe des abscisses,...
On obtient ainsi une sorte d'escargot, et graphiquement on conjecture que la suite converge
vers le point fixe de $f$. En plus on note que la suite des termes de rang pair semble une suite croissante,
alors que la suite des termes de rang impair semble décroissante.



%%%%%%%%%%%%%%%%%%%%%%%%%%%%%%%%%%%%%%%%%%%%%%%%%%%%%%%%%%%
\diapo

Pour l'étude de la convergence on distingue deux cas.

 Premier cas $0 < u_0 \le \ell = \frac{1+\sqrt{5}}{2}$.
 
 \change
 
  Comme $f$ est décroissante  $u_1 = f(u_0) \ge f(\ell)=\ell$ ; et par une étude de $f\circ f(x)-x$, on obtient que  :  
  $u_2 = f\circ f(u_0) \ge u_0$.

  \change
  
  Comme $u_2 \ge u_0$ et $f\circ f$ est croissante, la suite $(u_{2n})$ des termes de rangs pairs, est croissante. 

  \change
   De même $u_3  \le u_1$, donc la suite $(u_{2n+1})$ est décroissante. 
   
   \change
   
  De plus comme  $u_0 \le u_1$, en appliquant $f\circ f$, on obtient que $u_{2n} \le u_{2n+1}$.
  
  \change
  
  La situation est donc la suivante :
  $$u_0 \le u_2 \le \cdots \le u_{2n} \le \cdots \le u_{2n+1} \le \cdots \le u_3 \le u_1$$
  
  \change
  
  La suite $(u_{2n})$ est croissante et majorée par $u_1$, donc elle converge. 
  
  \change
  
  Sa limite ne peut
  être que l'unique point fixe de $f\circ f$: $\ell = \frac{1+\sqrt{5}}{2}$.
  
  \change
  
   La suite $(u_{2n+1})$ est décroissante et minorée par $u_0$, donc elle converge aussi vers 
   $\ell = \frac{1+\sqrt{5}}{2}$.
   

\change
   
   On en conclut que la suite $(u_{n})$ converge vers $\ell = \frac{1+\sqrt{5}}{2}$.
   
\change

\change

Le deuxième cas, c'est lorsque $u_0 \ge \ell = \frac{1+\sqrt{5}}{2}$.

\change  

  On montre de la même façon que $(u_{2n})$ est décroissante et converge vers $\ell$,
  et que $(u_{2n+1})$ est croissante et converge aussi vers $\ell$.
  

Dans tous les cas $(u_n)$ tend vers $\frac{1+\sqrt{5}}{2}$

%%%%%%%%%%%%%%%%%%%%%%%%%%%%%%%%%%%%%%%%%%%%%%%%%%%%%%%%%%%
\diapo

Les suites récurrentes jouent un rôle essentiel en mathématiques et sont très utiles pour les
applications, par exemple pour les calculs approchés de solution d'équation.


\end{document}
