\documentclass[class=report,crop=false]{standalone}
\usepackage[screen]{../exo7book}

\begin{document}

%====================================================================
\chapitre{Les suites}
%====================================================================

\insertvideo{eKWRb_wLczo}{partie 1. Premières définitions}

\insertvideo{253AEiNBvGw}{partie 2. Limite}

\insertvideo{tvbsvRGI_38}{partie 3. Exemples remarquables}

\insertvideo{0W5KVpj769E}{partie 4. Théorèmes de convergence}

\insertvideo{hqPxTPEqDXw}{partie 5. Suites récurrentes}

\insertfiche{fic00010.pdf}{Suites}


%%%%%%%%%%%%%%%%%%%%%%%%%%%%%%%%%%%%%%%%%%%%%%%%%%%%%%%%%%%%%%%%
\section*{Introduction}

L'étude des suites numériques a pour objet la compréhension de l'évolution de séquences de nombres (réels, complexes ...).
Ceci permet de modéliser de nombreux phénomènes de la vie quotidienne. Supposons par exemple que l'on place une somme
$S$ à un taux annuel de $10\%$. Si $S_n$ représente la somme que l'on obtiendra après $n$ années, on a
$$S_0=S \quad    S_1=S\times 1,1\quad \ldots \quad   S_n=S\times (1,1)^n \;\;.$$

Au bout de $n=10$ ans, on possédera donc $S_{10}=S\times (1,1)^{10}\thickapprox S\times 2,59$ :
la somme de départ avec les intérêts cumulés.



%%%%%%%%%%%%%%%%%%%%%%%%%%%%%%%%%%%%%%%%%%%%%%%%%%%%%%%%%%%%%%%%
\section{Définitions}

%---------------------------------------------------------------
\subsection{Définition d'une suite}
\begin{definition}
\sauteligne
    \begin{itemize}
      \item Une \defi{suite}\index{suite} est une application $u:\Nn\to \Rr$.
      \item Pour $n\in \Nn$, on note $u(n)$ par $u_n$ et on l'appelle $n$-ème{} \defi{terme} ou \defi{terme général} de la suite.
    \end{itemize}
\end{definition}

La suite est notée $u$, ou plus souvent $(u_n)_{n\in \Nn}$ ou simplement $(u_n)$.
Il arrive fréquemment que l'on considère des suites
définies à partir d'un certain entier naturel $n_0$ plus grand que $0$, on note alors $(u_n)_{n\geq n_0}$.
\begin{exemple}
\sauteligne
    \begin{itemize}
      \item $(\sqrt{n})_{n\geq 0}$ est la suite de termes : $0$, $1$, $\sqrt2$, $\sqrt3$,\ldots
      \item $((-1)^n)_{n\geq 0}$ est la suite qui alterne $+1$, $-1$, $+1$, $-1$,\ldots
      \item La suite $(S_n)_{n\geq 0}$ de l'introduction définie par $S_n=S\times (1,1)^n$,
      \item $(F_n)_{n\geq 0}$ définie par $F_0=1$, $F_1=1$ et la relation $F_{n+2}=F_{n+1}+F_n$
 pour $n\in \Nn$ (suite de Fibonacci). Les premiers termes sont $1$, $1$, $2$, $3$, $5$, $8$, $13$, \ldots
 Chaque terme est la somme des deux précédents.
      \item $\left(\frac{1}{n^2}\right)_{n\geq 1}$. Les premiers termes sont
      $1$, $\frac14$, $\frac19$, $\frac{1}{16}$, \ldots
    \end{itemize}
\end{exemple}
%---------------------------------------------------------------
\subsection{Suite majorée, minorée, bornée}

\begin{definition}
Soit $(u_n)_{n\in \Nn}$ une suite.
\begin{itemize}
  \item $(u_n)_{n\in \Nn}$ est \defi{majorée}\index{suite!majoree@majorée} si \quad $\exists M \in \Rr \quad \forall n\in \Nn \quad u_n \leq M$.
  \item $(u_n)_{n\in \Nn}$ est \defi{minorée}\index{suite!minoree@minorée} si \quad $\exists m \in \Rr \quad  \forall n\in \Nn \quad u_n \geq m$.
  \item $(u_n)_{n\in \Nn}$ est \defi{bornée}\index{suite!bornee@bornée} si elle est majorée et minorée, ce qui revient à dire :
$$\exists M \in \Rr \quad  \forall n\in \Nn \quad |u_n| \leq M.$$
\end{itemize}
\end{definition}

\myfigure{1}{
\tikzinput{fig_suites01a}
\qquad
\tikzinput{fig_suites01b}
}

%---------------------------------------------------------------
\subsection{Suite croissante, décroissante}
\begin{definition}
Soit $(u_n)_{n\in \Nn}$ une suite.
\begin{itemize}
  \item $(u_n)_{n\in \Nn}$ est \defi{croissante}\index{suite!croissante} si \quad $\forall n\in \Nn \quad u_{n+1} \ge u_n $.
  \item $(u_n)_{n\in \Nn}$ est \defi{strictement croissante} si \quad $\forall n\in \Nn \quad u_{n+1} > u_n$.
  \item $(u_n)_{n\in \Nn}$ est \defi{décroissante}\index{suite!decroissante@décroissante} si \quad $\forall n\in \Nn \quad u_{n+1} \le u_n$.
  \item $(u_n)_{n\in \Nn}$ est \defi{strictement décroissante} si \quad $\forall n\in \Nn \quad u_{n+1} < u_n$.
    \item $(u_n)_{n\in \Nn}$ est \defi{monotone}\index{suite!monotone} si elle est croissante ou décroissante.
      \item $(u_n)_{n\in \Nn}$ est \defi{strictement monotone}
si elle est strictement croissante ou strictement décroissante.
  \end{itemize}
\end{definition}


Voici un exemple d'une suite croissante (mais pas strictement croissante) :
\myfigure{0.9}{
\tikzinput{fig_suites02}
}

\begin{remarque*}
\sauteligne
    \begin{itemize}
      \item $(u_n)_{n\in \Nn}$ est croissante si et seulement si $\forall n\in \Nn \quad  u_{n+1}-u_n \geq 0$.
      \item Si $(u_n)_{n\in \Nn}$ est une suite à termes strictement positifs, elle est
croissante si et seulement si $\forall n\in \Nn \quad \frac{u_{n+1}}{u_n} \geq 1$.
    \end{itemize}
\end{remarque*}
\begin{exemple}
\sauteligne
    \begin{itemize}
      \item La suite $(S_n)_{n\geq 0}$ de l'introduction est strictement croissante car $S_{n+1}/S_n=1,1>1$.
      \item La suite $(u_n)_{n\geq 1}$ définie par $u_n=(-1)^n/n$ pour $n\geq 1$,
n'est ni croissante ni décroissante. Elle est majorée par $1/2$ (borne atteinte en $n=2$), minorée par $-1$ (borne atteinte en $n=1$).

\myfigure{1}{
\tikzinput{fig_suites03}
}

      \item La suite $\left(\frac1n\right)_{n\geq1}$ est une suite strictement décroissante. Elle est majorée par $1$
      (borne atteinte pour $n=1$), elle est minorée par $0$ mais cette valeur n'est jamais atteinte.
    \end{itemize}
\end{exemple}



%---------------------------------------------------------------
%\subsection{Mini-exercices}

\begin{miniexercices}
\sauteligne
\begin{enumerate}
  \item La suite $\big(\frac{n}{n+1}\big)_{n\in \Nn}$ est-elle monotone ? Est-elle bornée ?
%  \item La suite $((1+\frac{1}{n})^n)_{n \ge 1}$ est-elle bornée ? [[Arnaud : comment on fait ??]]
  \item La suite $\big(\frac{n\sin(n!)}{1+n^2}\big)_{n\in \Nn}$ est-elle bornée ?
  \item Réécrire les phrases suivantes en une phrase mathématique.
  \'Ecrire ensuite la négation mathématique de chacune des phrases.
  (a) La suite $(u_n)_{n\in\Nn}$ est majorée par $7$.
  (b) La suite $(u_n)_{n\in\Nn}$ est constante.
  (c) La suite $(u_n)_{n\in\Nn}$ est strictement positive à partir d'un certain rang.
  (d) $(u_n)_{n\in\Nn}$ n'est pas strictement croissante.
  \item Est-il vrai qu'une suite croissante est minorée ? Majorée ?
  \item Soit $x>0$ un réel. Montrer que la suite $\left(\frac{x^n}{n!}\right)_{n\in \Nn}$
  est décroissante à partir d'un certain rang.
\end{enumerate}
\end{miniexercices}


%%%%%%%%%%%%%%%%%%%%%%%%%%%%%%%%%%%%%%%%%%%%%%%%%%%%%%%%%%%%%%%%
\section{Limites}

\subsection{Introduction}
Pour un trajet au prix normal de 20 euros on achète une carte d'abonnement de train à $50$ euros
et on obtient chaque billet à $10$ euros.
La publicité affirme \og 50\% de réduction \fg{}. Qu'en pensez-vous ?

Pour modéliser la situation en termes de suites, on pose pour un entier $n\geq 1$ :
\begin{align*}
  u_n&=20n \\
  v_n&=10n+50
\end{align*}
$u_n$ est le prix payé au bout de $n$ achats au tarif plein, et $v_n$ celui au tarif réduit,
y compris le prix de l'abonnement. La réduction est donc, en pourcentage :
$$1-\frac{v_n}{u_n}=\frac{u_n-v_n}{u_n}=\frac{10n-50}{20n}=0,5 -\frac{5}{2n}\xrightarrow[n\to +\infty]{}0,5 $$


Il faut donc une infinité de trajets pour arriver à 50\% de réduction !

\myfigure{1}{
\tikzinput{fig_suites04}
}

%---------------------------------------------------------------
\subsection{Limite finie, limite infinie}

Soit $(u_n)_{n\in \Nn}$ une suite.

\begin{definition}
\label{def_lim}
  La suite $(u_n)_{n\in \Nn}$ a pour \defi{limite}\index{suite!limite}\index{limite} $\ell\in \Rr$ si :
pour tout $\epsilon >0$, il existe un entier naturel $N$ tel que si $n\geq N$ alors
$\lvert u_n-\ell\rvert\leq\epsilon$ :
\mybox{$\forall \epsilon >0 \quad \exists N \in \Nn \quad
    \forall n \in \Nn \qquad
    \left( n\geq N \implies \lvert u_n-\ell\rvert\leq\epsilon \right)$}
\end{definition}

On dit aussi que la suite $(u_n)_{n\in \Nn}$ \defi{tend vers $\ell$}.
Autrement dit : $u_n$ est proche d'aussi près que l'on veut de $\ell$, à partir d'un certain rang.

\myfigure{1}{
\tikzinput{fig_suites05}
}

\begin{definition}
\sauteligne
\begin{enumerate}
  \item   La suite $(u_n)_{n\in \Nn}$ \defi{tend vers $+\infty$} si :
\[ \forall A >0 \quad \exists N \in \Nn \quad
    \forall n \in \Nn \qquad \left( n\geq N \implies u_n\geq A \right)\]
\item La suite $(u_n)_{n\in \Nn}$ \defi{tend vers $-\infty$} si :
\[ \forall A >0 \quad \exists N \in \Nn \quad
    \forall n \in \Nn \qquad \left( n\geq N \implies u_n\leq -A \right)\]
\end{enumerate}
\end{definition}


\begin{remarque*}
\sauteligne
  \begin{enumerate}
    \item On note $\lim_{n\to +\infty}u_n=\ell$ ou parfois $u_n\xrightarrow[n\to +\infty]{} \ell$,
et de même pour une limite $\pm \infty$.

    \item $\lim_{n\to +\infty}u_n=-\infty\iff \lim_{n\to +\infty}-u_n=+\infty$.

    \item On raccourcit souvent la phrase logique en :
    $$\forall \epsilon >0 \quad \exists N \in \Nn  \qquad
    \left( n\geq N \implies \lvert u_n-\ell\rvert\leq\epsilon \right).$$
    Noter que $N$ dépend de $\epsilon$ et qu'on ne peut pas échanger l'ordre du \og pour tout \fg{}
    et du \og il existe \fg{}.



    \item L'inégalité $|u_n-\ell| \le \epsilon$ signifie $\ell-\epsilon \le u_n \le \ell +\epsilon$.
    On aurait aussi pu définir la limite par la phrase : $\forall \epsilon >0 \quad \exists N \in \Nn  \quad
    \left( n\geq N \implies \lvert u_n-\ell\rvert < \epsilon \right)$, où l'on a remplacé la dernière inégalité
    large par une inégalité stricte.

  \end{enumerate}
\end{remarque*}

\begin{definition}
  Une suite $(u_n)_{n\in \Nn}$ est \defi{convergente}\index{suite!convergente}\index{convergence} si elle admet une limite \evidence{finie}.
Elle est \defi{divergente}\index{suite!divergente}\index{divergence} sinon (c'est-à-dire soit la suite tend vers $\pm \infty$,
soit elle n'admet pas de limite).
\end{definition}



On va pouvoir parler de \evidence{la} limite, si elle existe, car il y a unicité de la limite :

\begin{proposition}
  Si une suite est convergente, sa limite est unique.
\end{proposition}

\begin{proof}
  On procède par l'absurde. Soit $(u_n)_{n\in \Nn}$ une suite convergente ayant
deux limites $\ell\neq \ell'$. Choisissons $\epsilon >0$ tel que $\epsilon <\frac{\lvert \ell-\ell'\rvert}{2}$.

Comme $\lim_{n\to +\infty}u_n=\ell$, il existe $N_1$ tel que $n\geq N_1$ implique $\lvert u_n-\ell\rvert<\epsilon$.

De même $\lim_{n\to +\infty}u_n=\ell'$, il existe $N_2$ tel que $n\geq N_2$ implique $\lvert u_n-\ell'\rvert<\epsilon$.

Notons $N=\max(N_1,N_2)$, on a alors pour ce $N$ :

$$\lvert u_N-\ell\rvert<\epsilon \quad \text{ et } \quad \lvert u_N-\ell'\rvert<\epsilon$$

Donc $\lvert \ell-\ell'\rvert =\lvert \ell-u_N+u_N-\ell'\rvert\leq \lvert \ell-u_N\rvert+\lvert u_N-\ell'\rvert$
d'après l'inégalité triangulaire. On en tire
$\lvert \ell-\ell'\rvert \leq \epsilon + \epsilon =2\epsilon<\lvert \ell-\ell'\rvert$.
On vient d'aboutir à l'inégalité $\lvert \ell-\ell'\rvert < \lvert \ell-\ell'\rvert$ qui est impossible.
Bilan : notre hypothèse de départ est fausse et donc $\ell=\ell'$.
\end{proof}

%---------------------------------------------------------------
\subsection{Propriétés des limites}

\begin{proposition}
\sauteligne
  \begin{enumerate}
    \item $\lim_{n\to +\infty}u_n=\ell\iff \lim_{n\to +\infty}(u_n-\ell)=0 \iff \lim_{n\to +\infty}\lvert u_n-\ell\rvert =0$,
    \item $\lim_{n\to +\infty}u_n=\ell\implies \lim_{n\to +\infty}\lvert u_n\rvert=\lvert \ell\rvert$.
  \end{enumerate}
\end{proposition}

\begin{proof}
  Cela résulte directement de la définition.
\end{proof}

\begin{proposition}[Opérations sur les limites]
\label{prop:suitelimite}
Soient $(u_n)_{n\in \Nn}$ et $(v_n)_{n\in \Nn}$ deux suites convergentes.
\begin{enumerate}
  \item Si $\lim_{n\to +\infty}u_n=\ell$, où $\ell\in \Rr$, alors pour $\lambda \in \Rr$ on a $\lim_{n\to +\infty}\lambda u_n=\lambda \ell$.
  \item Si $\lim_{n\to +\infty}u_n=\ell$ et $\lim_{n\to +\infty}v_n=\ell'$, où $\ell,\ell'\in \Rr$, alors
    \begin{align*}
      \lim_{n\to +\infty}\left(u_n+v_n\right)&=\ell+\ell'&\\
      \lim_{n\to +\infty}\left(u_n\times v_n\right)&=\ell\times \ell'
    \end{align*}
  \item Si $\lim_{n\to +\infty}u_n=\ell$ où $\ell\in \Rr^*= \Rr\backslash \left\{0\right\}$
alors $u_n\neq 0 $ pour $n$ assez grand et $\lim_{n\to +\infty}\frac{1}{u_n}=\frac{1}{\ell}$.
\end{enumerate}
\end{proposition}

Nous ferons la preuve dans la section suivante.

Nous utilisons continuellement ces propriétés, le plus souvent sans nous en rendre compte.
\begin{exemple}
Si $u_n \to \ell$ avec $\ell \neq \pm 1$, alors
$$u_n(1-3u_n)-\frac{1}{u_n^2-1} \xrightarrow[n\to +\infty]{} \ell(1-3\ell) - \frac{1}{\ell^2-1}.$$
\end{exemple}


\begin{proposition}[Opérations sur les limites infinies]
\label{prop:suiteinfty}
  Soient $(u_n)_{n\in \Nn}$ et $(v_n)_{n\in \Nn}$ deux suites telles que $\lim_{n\to +\infty}v_n=+\infty$.
  \begin{enumerate}
    \item  $\lim_{n\to +\infty}\frac{1}{v_n}=0$
    \item Si $(u_n)_{n\in \Nn}$ est minorée alors  $\lim_{n\to +\infty}\left(u_n+v_n\right)=+\infty$.
    \item Si $(u_n)_{n\in \Nn}$ est minorée par un nombre $\lambda>0$
alors $\lim_{n\to +\infty}\left(u_n\times v_n\right)=+\infty$.
    \item Si $\lim_{n\to +\infty}u_n= 0$ et $u_n>0$ pour $n$ assez
grand alors $\lim_{n\to +\infty}\frac{1}{u_n}=+\infty$.
  \end{enumerate}
\end{proposition}


\begin{exemple}
La suite $(\sqrt{n})$ tend vers $+\infty$, donc la suite $(\frac1{\sqrt{n}})$ tend vers $0$.
\end{exemple}



%-----------------------------------------------------
\subsection{Des preuves !}

Nous n'allons pas tout prouver mais seulement quelques résultats importants.
Les autres se démontrent de manière tout à fait semblable.

Commençons par prouver un résultat assez facile (le premier point de la proposition \ref{prop:suiteinfty}) :

\medskip

\centerline{\og{}\emph{Si \quad $\lim u_n = +\infty$ \quad alors \quad $\lim \frac{1}{u_n} = 0$.}\fg{}}

\begin{proof}
Fixons $\epsilon>0$. Comme $\lim_{n\to +\infty}u_n= +\infty$,
il existe un entier naturel $N$ tel que $n\ge N$ implique $u_n\ge \frac{1}{\epsilon}$.
On obtient alors $0 \le \frac{1}{u_n} \le \epsilon$ pour $n\geq N$.
On a donc montré que $\lim_{n\to +\infty}\frac{1}{u_n}= 0$.
\end{proof}

Afin de prouver que la limite d'un produit est le produit des limites
nous aurons besoin d'un peu de travail.

\begin{proposition}
Toute suite convergente est bornée.
\end{proposition}

\begin{proof}
Soit $(u_n)_{n\in \Nn}$  une suite convergeant vers le réel $\ell$. En appliquant
la définition de limite (définition \ref{def_lim}) avec $\epsilon=1$,
on obtient qu'il existe un entier naturel $N$ tel que pour $n\geq N$ on ait
$\lvert u_n-\ell \rvert \leq 1$, et donc pour $n\geq N$ on a
\[\lvert u_n \rvert =\lvert \ell+(u_n-\ell) \rvert \leq \lvert \ell \rvert + \lvert u_n-\ell  \rvert \leq \lvert \ell \rvert +1 .\]

\myfigure{1}{
\tikzinput{fig_suites09}
}

Donc si on pose
\[ M= \max(\lvert u_0 \rvert,\lvert u_1\rvert, \cdots , \lvert u_{N-1}\rvert , \lvert \ell \rvert +1) \]
on a alors $\forall n \in \Nn\;\; \lvert u_n \rvert \leq M$.
\end{proof}




\begin{proposition}
\label{prop:suitebornezero}
Si la suite $(u_n)_{n\in \Nn}$ est bornée et $\lim_{n\to +\infty}v_n=0$ alors $\lim_{n\to +\infty}\left(u_n\times v_n\right)=0$.
\end{proposition}

\begin{exemple}
 Si $(u_n)_{n\geq1}$ est la suite donnée par $u_n=\cos(n)$ et
$(v_n)_{n\geq1}$ est celle donnée par $v_n=\frac{1}{\sqrt{n}}$, alors $\lim_{n\to +\infty}\left(u_nv_n\right)=0$.
\end{exemple}

\begin{proof}
La suite $(u_n)_{n\in \Nn}$ est bornée, on peut donc trouver un réel $M>0$ tel que pour tout
entier naturel $n$ on ait $\lvert u_n \rvert\leq M$. Fixons $\epsilon >0$. On applique la
définition de limite (définition \ref{def_lim}) à la suite $(v_n)_{n\in \Nn}$ pour
$\epsilon'=\frac{\epsilon}{M}$.  Il existe donc un entier naturel $N$ tel que
$n\geq N$ implique $ \lvert v_n  \rvert\leq  \epsilon'$. Mais alors pour $n\geq N$ on a :
\[ \lvert u_nv_n \rvert= \lvert u_n \rvert \lvert v_n \rvert \leq M\times \epsilon'=\epsilon .\]
On a bien montré que $\lim_{n\to +\infty}\left(u_n\times v_n\right)=0$.
\end{proof}

\bigskip

Prouvons maintenant la formule concernant le produit de deux limites (voir proposition \ref{prop:suitelimite}).

\medskip

\centerline{\og \emph{Si \quad $\lim u_n = \ell$ \quad et \quad $\lim v_n = \ell'$ 
\quad alors \quad $\lim u_n v_n = \ell \ell'$.} \fg{}}

\begin{proof}[Démonstration de la formule concernant le produit de deux limites]
Le principe est d'écrire :
  \[
    u_nv_n-\ell\ell'=(u_n-\ell)v_n+\ell(v_n-\ell')
  \]
D'après la proposition \ref{prop:suitebornezero}, la suite de terme général $\ell(v_n-\ell')$ tend vers $0$.
Par la même proposition il en est
de même de la suite de terme général $(u_n-\ell)v_n$, car la suite convergente $(v_n)_{n\in \Nn}$
est bornée. On conclut que $\lim_{n\to +\infty} (u_nv_n-\ell\ell')=0$, ce qui équivaut à $\lim_{n\to +\infty} u_nv_n=\ell\ell'$.
\end{proof}



%-----------------------------------------------------
\subsection{Formes indéterminées}
Dans certaines situations, on ne peut rien dire à priori sur la limite,
il faut faire une étude au cas par cas.

\begin{exemple}
\sauteligne
  \begin{enumerate}
    \item \og $+\infty-\infty$ \fg{} Cela signifie que si $u_n \to + \infty$ et $v_n \to - \infty$
    il faut faire faire l'étude en fonction de chaque suite pour déterminer $\lim (u_n+v_n)$ comme le prouve les exemples suivants.
      \begin{align*}
        \lim_{n\to +\infty}\left(e^n-\ln(n)\right)&=+\infty\\
                                \lim_{n\to +\infty}\left(n-n^2\right)&=-\infty \\
                                \lim_{n\to +\infty}\left(\left(n+\frac{1}{n}\right)-n\right)&=0
      \end{align*}
    \item \og $0\times \infty$ \fg{}
             \begin{align*}
               \lim_{n\to +\infty} \frac{1}{\ln n} \times e^n &=+\infty\\
                                \lim_{n\to +\infty} \frac{1}{n} \times \ln n
                                &=0 \\
                                \lim_{n\to +\infty} \frac{1}{n} \times (n+1)
                                &=1
                        \end{align*}
    \item \og $\frac{\infty}{\infty}$\fg{}, \og $\frac{0}{0}$ \fg{}, \og $1^\infty$\fg{}, ...
  \end{enumerate}
\end{exemple}


%---------------------------------------------------------------
\subsection{Limite et inégalités}

\begin{proposition}
\label{prop:lim_ineg}
\sauteligne
  \begin{enumerate}
    \item Soient $(u_n)_{n\in \Nn}$ et $(v_n)_{n\in \Nn}$ deux suites convergentes
telles que : $\forall n \in \Nn$, $u_n\leq v_n$. Alors
      \[\lim_{n\to +\infty} u_n \leq \lim_{n\to +\infty} v_n\]
    \item  Soient $(u_n)_{n\in \Nn}$ et $(v_n)_{n\in \Nn}$ deux suites telles que
$\lim_{n\to +\infty} u_n=+\infty$ et  $\forall n \in \Nn$, $v_n \geq u_n$.
Alors $\lim_{n\to +\infty} v_n=+\infty$.
    \item Théorème des  \og gendarmes \fg{}\index{theoreme@théorème!des gendarmes} : si $(u_n)_{n\in \Nn}$,
$(v_n)_{n\in \Nn}$ et $(w_n)_{n\in \Nn}$ sont trois suites telles que
      \[ \forall n \in \Nn \quad u_n\leq v_n\leq w_n \]
et $\lim_{n\to +\infty}u_n=\ell=\lim_{n\to +\infty}w_n$, alors la suite $(v_n)_{n\in \Nn}$
est convergente et $\lim_{n\to +\infty} v_n=\ell$.
  \end{enumerate}
\myfigure{1}{
\tikzinput{fig_suites06}
}
\end{proposition}

\begin{remarque*}
\sauteligne
  \begin{enumerate}
    \item Soit $(u_n)_{n\in \Nn}$ une suite convergente telle que :
$\forall n \in \Nn$, $u_n\geq 0$. Alors $\lim_{n\to +\infty} u_n\geq 0$.
    \item Attention, si $(u_n)_{n\in \Nn}$ est une suite convergente telle que :
$\forall n \in \Nn$, $u_n > 0$, on ne peut affirmer que la limite est strictement positive mais
seulement que $\lim_{n\to +\infty} u_n \ge 0$.
Par exemple la suite $(u_n)_{n\in \Nn}$ donnée par $u_n=\frac{1}{n+1} $ est à termes strictement positifs, mais converge vers zéro.


  \end{enumerate}
\end{remarque*}

\begin{proof}[Démonstration de la proposition \ref{prop:lim_ineg}]
~
  \begin{enumerate}
    \item En posant $w_n=v_n-u_n$, on se ramène à montrer que si une
suite $(w_n)_{n\in \Nn}$ vérifie $\forall n \in \Nn$, $w_n\geq 0$ et converge,
alors $\lim_{n\to +\infty} w_n\geq 0$. On procède par l'absurde en supposant
que $\ell=\lim_{n\to +\infty} w_n <0$. En prenant $\epsilon=\lvert\frac{\ell}{2}  \rvert $
dans la définition de limite (définition \ref{def_lim}), on obtient qu'il existe
un entier naturel $N$ tel que $n\geq N$ implique $ \lvert w_n - \ell  \rvert <  \epsilon=-\frac{\ell}{2}$.
En particulier on a pour $n\geq N$ que $w_n< \ell-\frac{\ell}{2}=\frac{\ell}{2}<0$, une contradiction.

\myfigure{0.9}{
\tikzinput{fig_suites07}
}
    \item Laissé en exercice.

    \item En soustrayant la suite $(u_n)_{n\in \Nn}$, on se ramène à
montrer l'énoncé suivant : si  $(u_n)_{n\in \Nn}$ et $(v_n)_{n\in \Nn}$ sont deux
suites telles que : $\forall n \in \Nn$, $0\leq u_n\leq v_n$ et $\lim_{n\to +\infty} v_n= 0$,
alors $(u_n)$ converge et $\lim_{n\to +\infty} u_n= 0$. Soit $\epsilon >0$ et $N$ un entier naturel
tel que $n\geq N$ implique $\lvert v_n \rvert <\epsilon$. Comme
$\lvert u_n \rvert=u_n\leq v_n=\lvert v_n \rvert$, on a donc : $n\geq N$ implique
$\lvert u_n \rvert <\epsilon$. On a bien montré que $\lim_{n\to +\infty} u_n= 0$.

  \end{enumerate}
\end{proof}

\begin{exemple}[Exemple d'application du théorème des \og gendarmes \fg{}]
Trouver la limite de la suite $(u_n)_{n\in \Nn}$ de terme général :
\[u_n =2 +\frac{(-1)^n}{1+n+n^2}\]
\end{exemple}



%---------------------------------------------------------------
%\subsection{}

%---------------------------------------------------------------
%\subsection{Mini-exercices}

\begin{miniexercices}
\sauteligne
\begin{enumerate}
  \item Soit $(u_n)_{n\in \Nn}$ la suite définie par $u_n = \frac{2n+1}{n+2}$.
En utilisant la définition de la limite montrer que $\lim_{n\to+\infty} u_n = 2$.
Trouver explicitement un rang à partir duquel $1,999 \le u_n \le 2,001$.

    \item  Déterminer la limite $\ell$ de la suite $(u_n)_{n\in \Nn^*}$ de terme général :
$\frac{n+\cos n}{n-\sin n}$ et trouver un entier $N$
tel que si $n\geq N$, on ait $|u_n-\ell|\leq 10^{-2}$.

  \item La suite $(u_n)_{n\in \Nn}$ de terme général $(-1)^ne^{n}$ admet-elle une limite ?
Et la suite de terme général $\frac{1}{u_n}$ ?

  \item  Déterminer la limite de la suite $(u_n)_{n\geq 1}$ de terme général $\sqrt{n+1}-\sqrt{n}$.
   Idem avec $v_n = \frac{\cos n}{\sin n + \ln n}$. Idem avec $w_n=\frac{n!}{n^n}$.


  %La suite $(u_n)_{n\geq 1}$ de terme général $n^{-2+(-1)^n}$ admet-elle une limite ?
%   \item  Déterminer la limite de la suite $(u_n)_{n\geq 1}$ de terme général
% $\frac{E(\pi)+E(2\pi)+\cdots+E(n\pi)}{n^2}$.
\end{enumerate}
\end{miniexercices}

%%%%%%%%%%%%%%%%%%%%%%%%%%%%%%%%%%%%%%%%%%%%%%%%%%%%%%%%%%%%%%%%
\section{Exemples remarquables}

%---------------------------------------------------------------
\subsection{Suite géométrique}

\begin{proposition}[Suite géométrique]
\index{suite!geometrique@géométrique}
~
\label{prop:suitegeo}
On fixe un réel $a$. Soit $(u_n)_{n\in \Nn}$ la suite de terme général : $u_n=a^n$.
\begin{enumerate}
  \item Si $a=1$, on a pour tout $n\in \Nn$ : $u_n=1$.
  \item Si $a>1$, alors $\lim_{n\to +\infty} u_n= +\infty$.
  \item Si $-1<a<1$, alors $\lim_{n\to +\infty} u_n= 0$.
  \item Si $a\le-1$, la suite $(u_n)_{n\in \Nn}$ diverge.
\end{enumerate}
\end{proposition}

\begin{proof}~
  \begin{enumerate}
    \item est évident.
    \item Écrivons $a=1+b$ avec $b>0$. Alors le binôme de Newton s'écrit
$a^n=(1+b)^n=1+nb+\binom{n}{2}b^2+\cdots+\binom{n}{k}b^k+\cdots+b^n$. Tous les termes
sont positifs, donc pour tout entier naturel $n$ on a : $a^n\geq 1+nb$. Or
$\lim_{n\to +\infty}(1+nb)=+\infty$ car $b>0$. On en déduit que $\lim_{n\to +\infty} a^n=+\infty$.
    \item Si $a=0$, le résultat est clair. Sinon, on pose $b=\lvert \frac{1}{a}
\rvert$. Alors $b>1$ et d'après le point précédent $\lim_{n\to +\infty} b^n=+\infty$.
Comme pour tout entier naturel $n$ on a : $\lvert a \rvert ^n=\frac{1}{b^n}$, on en
déduit que $\lim_{n\to +\infty} \lvert a \rvert ^n=0$, et donc aussi $\lim_{n\to +\infty} a  ^n=0$.
    \item Supposons par l'absurde que la suite $(u_n)_{n\in \Nn}$
converge vers le réel $\ell$. De $a^2\geq 1$, on déduit que pour tout entier naturel $n$,
on a $a^{2n}\geq 1$. En passant à la limite, il vient $\ell\geq 1$. Comme de plus pour tout
entier naturel $n$ on a $a^{2n+1}\leq a \leq -1$, il vient en passant de nouveau à la
limite $\ell\leq -1$. Mais comme on a déjà $\ell\geq 1$, on obtient une contradiction, et donc $(u_n)$ ne converge pas.
  \end{enumerate}
\end{proof}



%---------------------------------------------------------------
\subsection{Série géométrique}

\begin{proposition}[Série géométrique]
\index{serie geometrique@série géométrique}
  Soit $a$ un réel, $a\neq 1$. En notant $\sum_{k=0}^na^k=1+a+a^2+\cdots+a^n$, on a :
  \mybox{$\displaystyle \sum_{k=0}^na^k=\frac{1-a^{n+1}}{1-a} $}

\end{proposition}

\begin{proof}
En multipliant par $1-a$ on fait apparaître une somme télescopique
(presque tous les termes s'annulent) :
  \[(1-a)\big(1+a+a^2+\cdots+a^n \big)=
  \big(1+a+a^2+\cdots+a^n \big) - \big(a+a^2+\cdots+a^{n+1} \big)=1-a^{n+1} .\]
\end{proof}

\begin{remarque*}
Si $a\in ]-1,1[$ et $(u_n)_{n\in \Nn}$ est la suite de terme général :
$u_n= \sum_{k=0}^na^k$, alors $\lim_{n\to +\infty} u_n= \frac{1}{1-a}$.
De manière plus frappante, on peut écrire :
\[1+a+a^2+a^3 +\cdots = \frac{1}{1-a}\]

Enfin, ces formules sont aussi valables si $a\in \Cc\setminus\{1\}$.
Si $a=1$, alors $1+a+a^2+\cdots+a^n=n+1$.
\end{remarque*}

\begin{exemple}
L'exemple précédent avec $a=\frac{1}{2}$ donne

  \[1+\frac{1}{2} +\frac{1}{4} +\frac{1}{8} +\cdots = 2 .\]

Cette formule était difficilement concevable avant l’avènement du calcul infinitésimal et a été
popularisée sous le nom du \evidence{paradoxe de Zénon}. On tire une flèche à $2$ mètres d'une cible.
Elle met un certain laps de temps pour parcourir
la moitié de la distance, à savoir un mètre. Puis il lui faut encore du
temps pour parcourir la moitié de la distance restante, et de nouveau un
certain temps pour la moitié de la distance encore restante.
On ajoute ainsi une infinité de durées non nulles, et Zénon en conclut
que la flèche n'atteint jamais sa cible !

L'explication est bien donnée par l'égalité ci-dessus :
la somme d'une infinité de termes peut bien être une valeur finie !!
Par exemple si la flèche va à une vitesse de $1$\,m/s, alors
elle parcoure la première moitié en $1$\,s, le moitié de la distance restante en $\frac12$\,s, etc.
Elle parcoure bien toute la distance en $1+\frac{1}{2} +\frac{1}{4} +\frac{1}{8} +\cdots = 2$ secondes !

\myfigure{1.3}{
\tikzinput{fig_suites08}
}
\end{exemple}


%---------------------------------------------------------------
\subsection{Suites telles que $\left|\frac{u_{n+1}}{u_n}\right|<\ell<1$ }

\begin{theoreme}
Soit  $(u_n)_{n\in \Nn}$ une suite de réels non nuls. On suppose qu'il existe un réel $\ell$
%tel que $0< l <1$ et
tel que pour tout entier naturel $n$ (ou seulement à partir d'un certain rang) on ait :
\[ \left | \frac{u_{n+1}}{u_n}\right |  <\ell<1.\]
Alors $\lim_{n\to +\infty} u_n= 0$.

\end{theoreme}

\begin{proof}
  On suppose que la propriété $\left|\frac{u_{n+1}}{u_n}\right|<\ell<1$ est vraie
pour tout entier naturel $n$ (la preuve dans le cas où cette propriété n'est vraie
qu'à partir d'un certain rang n'est pas très différente).
  On écrit
  \[ \frac{u_n}{u_0}=\frac{u_1}{u_0} \times\frac{u_2}{u_1} \times\frac{u_3}{u_2} \times\cdots\times \frac{u_n}{u_{n-1}}\]
ce dont on déduit
  \[ \left\lvert\frac{u_n}{u_0}  \right\rvert <\ell\times \ell\times \ell \times \cdots \times \times \ell=\ell^n\]
et donc $\lvert u_n \rvert <\lvert u_0 \rvert \ell^n$. Comme $\ell<1$, on a $\lim_{n\to +\infty} \ell^n= 0$.
On conclut que $\lim_{n\to +\infty} u_n= 0$.
\end{proof}

\begin{corollaire}
Soit  $(u_n)_{n\in \Nn}$ une suite de réels non nuls.
\mybox{Si $\lim_{n\to +\infty}\frac{u_{n+1}}{u_n}=  0$, alors $\lim_{n\to +\infty} u_n= 0$.}
\end{corollaire}

\begin{exemple}
Soit $a\in \Rr$. Alors $\lim_{n\to +\infty} \frac{a^n}{n!} =0$.
\end{exemple}

\begin{proof}
  Si $a=0$, le résultat est évident. Supposons $a\neq 0$, et posons $u_n= \frac{a^n}{n!}$. Alors
  \[\frac{u_{n+1}}{u_n}= \frac{a^{n+1}}{(n+1)!}\cdot \frac{n!}{a^n}=\frac{a}{n+1} .\]

  Pour conclure, on peut ou bien directement utiliser le corollaire : comme $\lim \frac{u_{n+1}}{u_n} = 0$
  (car $a$ est fixe), on a $\lim u_n = 0$. Ou bien, comme $\frac{u_{n+1}}{u_n} = \frac{a}{n+1}$, on
  déduit par le théorème que pour $n\geq N>2 \lvert a \rvert $ on a :

  \[\left | \frac{u_{n+1}}{u_n}\right | =\frac{\lvert a \rvert }{n+1}\leq
  \frac{\lvert a \rvert }{N+1}<\frac{\lvert a \rvert  }{N}< \frac{1}{2} =\ell\]
et donc $\lim_{n\to +\infty} u_n =0$.
\end{proof}

\begin{remarque*}
\sauteligne
  \begin{enumerate}
    \item Avec les notations du théorème, si on a pour tout entier naturel $n$
à partir d'un certain rang : $\left|\frac{u_{n+1}}{u_n}\right|>\ell>1$, alors la suite $(u_n)_{n\in \Nn}$ diverge. En effet, il suffit d'appliquer le théorème à la suite de terme général $ \frac{1}{\lvert u_n \rvert }$ pour voir que  $\lim_{n\to +\infty} \lvert u_n \rvert =+\infty$.
    \item Toujours avec les notations du théorème, si $\ell=1$ on ne peut rien dire.
  \end{enumerate}
\end{remarque*}

\begin{exemple}
Pour un nombre réel $a$, $a>0$, calculer $\lim_{n\to +\infty} \sqrt[n]{a}$.

On va montrer que $\lim_{n\to +\infty} \sqrt[n]{a}=1$. Si $a=1$, c'est clair.
Supposons $a>1$. Écrivons $a=1+h$, avec $h>0$. Comme
\[\left(1+\frac{h}{n}\right)^n\geq 1+n\frac{h}{n}=1+h=a\]
(voir la preuve de la proposition \ref{prop:suitegeo}) on a en appliquant la fonction racine $n$-ème, $\sqrt[n]{\cdot}$ :
\[ 1+\frac{h}{n} \geq \sqrt[n]{a} \geq 1 .\]
On peut conclure grâce au théorème \og des gendarmes \fg{} que $\lim_{n\to +\infty} \sqrt[n]{a}=1$.
Enfin, si $a<1$, on applique le cas précédent à $b= \frac{1}{a}>1$.
\end{exemple}






%---------------------------------------------------------------
\subsection{Approximation des réels par des décimaux}

\begin{proposition}
\label{prop:ecdecim}
  Soit $a\in \Rr$. Posons
  \[u_n = \frac{E(10^na)}{10^n}. \]
  Alors $u_n$ est une approximation décimale de $a$ à $10^{-n}$ près,
  en particulier $\lim_{n\to +\infty} u_n=a$.
\end{proposition}

\begin{exemple} $\pi= 3,14159265\ldots$

 $$ \begin{array}{l}
    u_0  = \frac{E(10^0\pi)}{10^0}  = E(\pi)  = 3\\
    u_1  = \frac{E(10^1\pi)}{10^1}  = \frac{E(31,415\ldots)}{10}  = 3,1\\
    u_2  = \frac{E(10^2\pi)}{10^2}  = \frac{E(314,15\ldots)}{100} = 3,14\\
    u_3  = 3,141
  \end{array}$$
\end{exemple}

\begin{proof}
  D'après la définition de la partie entière, on a
  \[E(10^na)\leq 10^na < E(10^na)+1\]
donc
\[ u_n \leq a < u_n+\frac{1}{10^n} \]
ou encore
\[ 0 \leq a -u_n< \frac{1}{10^n} .\]
Or la suite de terme général $\frac{1}{10^n}$ est une suite géométrique de raison
$\frac{1}{10}$, donc elle tend vers $0$. On en déduit que $\lim_{n\to +\infty} u_n=a$.

\end{proof}

\begin{exercicecours}
  Montrer que la suite $(u_n)_{n\in \Nn}$ de la proposition \ref{prop:ecdecim} est croissante.
\end{exercicecours}

\begin{remarque*}
\sauteligne
  \begin{enumerate}
    \item Les $u_n$ sont des nombres décimaux, en particulier ce sont des nombres rationnels.
    \item Ceci fournit une démonstration de la densité de $\Qq$ dans $\Rr$.
Pour $\epsilon >0$, et $I=]a-\epsilon,a+\epsilon[$, alors pour $n$ assez grand, $u_n\in I\cap \Qq$.
  \end{enumerate}
\end{remarque*}


%---------------------------------------------------------------
%\subsection{Mini-exercices}

\begin{miniexercices}
\sauteligne
\begin{enumerate}
  \item Déterminer la limite de la suite $(u_n)_{n\in \Nn}$ de terme général $5^n-4^n$.

  \item Soit $v_n=1+a+a^2+\cdots + a^n$. Pour quelle valeur de $a \in \Rr$ la suite
  $(v_n)_{n\ge 1}$ a pour limite $3$ (lorsque $n \to + \infty$) ?

  \item Calculer la limite de $\frac{1+2+2^2+\cdots + 2^n}{2^n}$.

  \item Montrer que la somme des racines $n$-èmes de l'unité est nulle.

  \item Montrer que si $\sin(\frac \theta 2)\neq 0$ alors
  $\frac{1}{2}+\cos(\theta )+\cos(2\theta )+\cdots+\cos(n\theta )
  =\frac{\sin\left( (n+ \frac{1}{2})\theta \right)}{2\sin(\frac{\theta }{2})}$
  (penser à $e^{\ii\theta}$).

  \item Soit $(u_n)_{n\geq 2}$ la suite de terme général
  % $\frac{n!}{n^n}$. [[Arnaud : trop dur !]]
  $u_n= \ln(1+\frac12)\times\ln(1+\frac 13)\times\cdots\times\ln(1+\frac1n)$.
Déterminer la limite de $\frac{u_{n+1}}{u_n}$. Que peut-on en déduire ?
  \item Déterminer la limite de $\frac{\pi^n}{1\times 3 \times 5 \times \cdots \times (2n+1)}$
  (où $\pi = 3,14\ldots$).

   \item Soit $a$ un réel. Montrer que pour tout $\epsilon>0$ il existe un couple
   $(m,n)\in \Zz\times \Nn$ (et même une infinité)
tel que $\left|a-\frac{m}{2^n}\right|\leq \epsilon$.
\end{enumerate}
\end{miniexercices}

%%%%%%%%%%%%%%%%%%%%%%%%%%%%%%%%%%%%%%%%%%%%%%%%%%%%%%%%%%%%%%%%
\section{Théorème de convergence}

%---------------------------------------------------------------
\subsection{Toute suite convergente est bornée}


Revenons sur une propriété importante que nous avons déjà démontrée dans la section sur les limites.
\begin{proposition}
Toute suite convergente est bornée.
\end{proposition}

La réciproque est fausse mais nous allons ajouter une
hypothèse supplémentaire pour obtenir des résultats.


%---------------------------------------------------------------
\subsection{Suite monotone}

\begin{theoreme}
\label{thm:suite_croiss_maj}
\sauteligne
\mybox{Toute suite croissante et majorée est convergente.}
\end{theoreme}

\begin{remarque*}
  Et aussi :
  \begin{itemize}
    \item Toute suite décroissante et minorée est convergente.
    \item Une suite croissante et qui n'est pas majorée tend vers $+\infty$.
    \item Une suite décroissante et qui n'est pas minorée tend vers $-\infty$.
  \end{itemize}
\end{remarque*}

\begin{proof}[Démonstration du théorème \ref{thm:suite_croiss_maj}]
Notons $A=\left\{ u_n| n\in \Nn\right\} \subset \Rr$. Comme la suite $(u_n)_{n\in \Nn}$
est majorée, disons par le réel $M$, l'ensemble $A$ est majoré par $M$, et de plus
 il est non vide. Donc d'après le théorème $\Rr4$ du chapitre sur les réels,
l'ensemble $A$ admet une borne supérieure : notons $\ell=\sup A$. Montrons que
$\lim_{n\to +\infty} u_n=\ell$. Soit $\epsilon >0$. Par la caractérisation
de la borne supérieure, il existe un élément $u_N$ de $A$ tel que
 $\ell-\epsilon < u_N \leq \ell$. Mais alors pour $n\geq N$ on a $\ell-\epsilon < u_N \leq u_n \leq \ell$,
et donc $\lvert u_n-\ell \rvert \leq \epsilon$.
\end{proof}



%---------------------------------------------------------------
\subsection{Deux exemples}


\subsubsection{La limite $\zeta(2)$}
\label{ssub:zeta}

Soit $(u_n)_{n\geq 1}$ la suite de terme général :
\[ u_n =1+\frac{1}{2^2} +\frac{1}{3^2}+\cdots+\frac{1}{n^2} .\]

\begin{itemize}
  \item  La suite $(u_n)_{n\geq 1}$ est croissante : en effet $u_{n+1}-u_n= \frac{1}{(n+1)^2}>0$.
  \item Montrons par récurrence que pour tout entier naturel $n\geq 1$ on a $u_n\leq 2 - \frac{1}{n} $.
  \begin{itemize}
  \item Pour $n=1$, on a $u_1=1\leq 1=2 - \frac{1}{1}$.
  \item Fixons $n\geq 1$ pour lequel on suppose $u_n\leq 2 - \frac{1}{n}$.
Alors $u_{n+1}=u_n+ \frac{1}{(n+1)^2}\leq  2 - \frac{1}{n}+ \frac{1}{(n+1)^2}$.
Or $\frac{1}{(n+1)^2}\leq \frac{1}{n(n+1)}=\frac{1}{n}-\frac{1}{n+1}$, donc $u_{n+1}\leq 2-\frac{1}{n+1}$, ce qui achève la récurrence.
\end{itemize}
\item Donc la suite $(u_n)_{n\geq 1}$ est croissante et majorée par $2$ : elle converge.
\end{itemize}

On note $\zeta(2)$ cette limite, vous montrerez plus tard qu'en fait $\zeta(2)=\frac{\pi^2}{6}$.



\subsubsection{Suite harmonique}

C'est la suite $(u_n)_{n\geq 1}$ de terme général :
\[ u_n =1+\frac{1}{2} +\frac{1}{3}+\cdots+\frac{1}{n} .\]
Calculons $\lim_{n\to +\infty} u_n$.

\begin{itemize}
  \item La suite $(u_n)_{n\geq 1}$ est croissante : en effet $u_{n+1}-u_n= \frac{1}{n+1}>0$.
  \item Minoration de $u_{2^p}-u_{2^{p-1}}$. On a
$u_2-u_1=1+\frac{1}{2}-1=\frac{1}{2}$ ; $u_4-u_2=\frac{1}{3}+\frac{1}{4}> \frac{1}{4}+\frac{1}{4}=\frac{1}{2}$, et en général :
    \[ u_{2^p}-u_{2^{p-1}}=\underbrace{\frac{1}{2^{p-1}+1}+ \frac{1}{2^{p-1}+2}+\cdots+\frac{1}{2^p}}_{2^{p-1}=2^p-2^{p-1}
\textrm{ termes }\geq \frac{1}{2^p}}> 2^{p-1}\times \frac{1}{2^p}=\frac{1}{2}\]
  \item  $\lim_{n\to +\infty} u_n=+\infty$. En effet
    \[  u_{2^p}-1 = u_{2^p}-u_1= (u_2 - u_1)+ (u_4 - u_2)+ \cdots +  (u_{2^p}-u_{2^{p-1}}) \geq \frac{p}{2} \]
donc la suite  $(u_n)_{n\geq 1}$ est croissante mais n'est pas bornée, donc elle tend vers $+\infty$.

\end{itemize}


%---------------------------------------------------------------
\subsection{Suites adjacentes}

\begin{definition}
Les suites $(u_n)_{n\in \Nn}$ et $(v_n)_{n\in \Nn}$ sont dites \defi{adjacentes}\index{suite!adjacente} si
\begin{enumerate}
  \item $(u_n)_{n\in \Nn}$ est croissante et $(v_n)_{n\in \Nn}$ est décroissante,
  \item pour tout $n\geq 0$, on a $u_n\leq v_n$,
  \item $\lim_{n\to +\infty} (v_n -u_n) = 0$.
\end{enumerate}
\end{definition}

\begin{theoreme}
\sauteligne
  \mybox{Si les suites $(u_n)_{n\in \Nn}$ et $(v_n)_{n\in \Nn}$ sont adjacentes, \\
  elles convergent vers la même limite.}
\end{theoreme}

Il y a donc deux résultats dans ce théorème, la convergence de $(u_n)$ et
$(v_n)$ et en plus l'égalité des limites.
Les termes de la suites sont ordonnées ainsi :
$$u_0 \le u_1 \le u_2 \le \cdots \le u_n \le \cdots \cdots \le v_n\le \cdots \le v_2 \le v_1 \le v_0$$

\begin{proof}~
  \begin{itemize}
    \item La suite $(u_n)_{n\in \Nn}$ est croissante et majorée par $v_0$,
    donc elle converge vers une limite $\ell$.
    \item La suite $(v_n)_{n\in \Nn}$ est décroissante et minorée par $u_0$,
    donc elle converge vers une limite $\ell'$.
    \item Donc $\ell'-\ell=\lim_{n\to +\infty} (v_n -u_n) = 0$, d'où $\ell'=\ell$.
  \end{itemize}
\end{proof}


\begin{exemple}
Reprenons l'exemple de $\zeta(2)$. Soient $(u_n)$ et $(v_n)$ les deux suites définies pour $n \ge 1$ par
$$u_n = \sum_{k=1}^n \frac 1 {k^2} = 1+\frac{1}{2^2} +\frac{1}{3^2}+\cdots+\frac{1}{n^2} \quad \text{ et } \quad v_n = u_n + \frac2{n+1} .$$
Montrons que $(u_n)$ et $(v_n)$ sont deux suites adjacentes :
\begin{enumerate}
  \item
  \begin{enumerate}
    \item $(u_n)$ est croissante car $u_{n+1}-u_n = \frac{1}{(n+1)^2} > 0$.
    \item $(v_n)$ est décroissante : \\
    $v_{n+1}-v_n = \frac{1}{(n+1)^2} + \frac{2}{n+2} - \frac{2}{n+1}
    = \frac{n+2+2(n+1)^2-2(n+1)(n+2)}{(n+2)(n+1)^2} = \frac{-n}{(n+2)(n+1)^2}< 0$
  \end{enumerate}

  \item Pour tout $n\ge 1$ : $v_n-u_n = \frac{2}{n+1} >0$, donc $u_n \le v_n$.

  \item Enfin comme $v_n-u_n = \frac{2}{n+1}$ alors $\lim (v_n-u_n) = 0$.
\end{enumerate}

Les suites $(u_n)$ et $(v_n)$ sont deux suites adjacentes, elles convergent donc vers une même limite
finie $\ell$. Nous avons en plus l'encadrement $u_n \le \ell \le v_n$ pour tout $n\ge 1$.
Ceci fournit des approximations de la limite :
par exemple pour $n=3$, $1+\frac{1}{4} +\frac{1}{9} \le \ell \le 1+\frac{1}{4} +\frac{1}{9} + \frac{1}{2}$
donc $1,3611\ldots \le \ell \le 1,8611\ldots$
\end{exemple}


\begin{exercicecours}
  Soit $(u_n)_{n\geq 1}$ la suite de terme général :
\[ u_n =1+\frac{1}{2^3} +\frac{1}{3^3}+\cdots+\frac{1}{n^3} .\]
Montrer que la suite $(u_n)_{n\geq 1}$ converge (on pourra considérer la suite
$(v_n)_{n\geq 1}$ de terme général $v_n=u_n+  \frac{1}{n^2}$).
\end{exercicecours}

\begin{remarque*}
  On note $\zeta(3)$ cette limite. On l'appelle aussi constante d'Apéry qui a prouvé en 1978 que $\zeta(3)\notin \Qq$.
\end{remarque*}

\subsection{Théorème de Bolzano-Weierstrass}

\begin{definition}
  Soit $(u_n)_{n\in \Nn}$ une suite. Une \defi{suite extraite}\index{suite!extraite} ou
\defi{sous-suite}\index{sous-suite} de  $(u_n)_{n\in \Nn}$ est une suite de la forme
$(u_{\phi(n)})_{n\in \Nn}$, où $\phi : \Nn \to \Nn$ est une application strictement croissante.
\end{definition}


\myfigure{1}{
\tikzinput{fig_suites10}
}


\begin{exemple}
Soit la suite $(u_n)_{n\in \Nn}$ de terme général $u_n=(-1)^n$.
\begin{itemize}
  \item Si on considère $\phi : \Nn \to \Nn$ donnée par $\phi(n)=2n$,
alors la suite extraite correspondante a pour terme général $u_{\phi(n)}=(-1)^{2n}=1$,
donc la suite $(u_{\phi(n)})_{n\in \Nn}$ est constante égale à $1$.
  \item Si on considère $\psi : \Nn \to \Nn$ donnée par $\psi(n)=3n$, alors la suite
extraite correspondante a pour terme général $u_{\psi(n)}=(-1)^{3n}=\big( (-1)^3\big)^n = (-1)^n$.
La suite $(u_{\psi(n)})_{n\in \Nn}$ est donc égale à $(u_n)_{n\in \Nn}$.

\end{itemize}
\myfigure{0.85}{
\tikzinput{fig_suites11}\quad
\tikzinput{fig_suites12}
}
\end{exemple}

\begin{proposition}
  Soit $(u_n)_{n\in \Nn}$ une suite. Si $\lim_{n\to +\infty}u_n=\ell$, alors
pour toute suite extraite $(u_{\phi(n)})_{n\in \Nn}$ on a $\lim_{n\to +\infty} u_{\phi(n)}=\ell$.
\end{proposition}

\begin{proof}
  Soit $\epsilon >0$. D'après la définition de limite (définition \ref{def_lim}),
il existe un entier naturel $N$ tel que $n\geq N$ implique $\lvert u_n-\ell \rvert <\epsilon$.
Comme l'application $\phi$ est strictement croissante, on montre facilement par récurrence
que pour tout $n$, on a $\phi(n)\geq n$. Ceci implique en particulier que si $n\geq N$,
alors aussi $\phi(n)\geq N$, et donc $\lvert u_{\phi(n)}-\ell \rvert <\epsilon$.
Donc la définition de limite (définition \ref{def_lim}) s'applique aussi à la suite extraite.

\end{proof}

\begin{corollaire}
  Soit $(u_n)_{n\in \Nn}$ une suite. Si elle admet une sous-suite divergente,
ou bien si elle admet deux sous-suites convergeant vers des limites distinctes, alors elle diverge.
\end{corollaire}

\begin{exemple}
Soit la suite $(u_n)_{n\in \Nn}$ de terme général $u_n=(-1)^n$. Alors
$(u_{2n})_{n\in \Nn}$ converge vers $1$, et $(u_{2n+1})_{n\in \Nn}$ converge vers $-1$
(en fait ces deux sous-suites sont constantes). On en déduit que la suite $(u_n)_{n\in \Nn}$ diverge.
\end{exemple}


\begin{exercicecours}
  Soit $(u_n)_{n\in \Nn}$ une suite. On suppose que les deux sous-suites
$(u_{2n})_{n\in \Nn}$ et $(u_{2n+1})_{n\in \Nn}$ convergent vers la même limite $\ell$.
Montrer que $(u_n)_{n\in \Nn}$ converge également vers $\ell$.
\end{exercicecours}

Terminons par un résultat théorique très important.
\begin{theoreme}[Théorème de Bolzano-Weierstrass]
\index{theoreme@théorème!de Bolzano-Weierstrass}
\label{thm:Bolzano_Weierstrass}
  Toute suite bornée admet une sous-suite convergente.
\end{theoreme}

\begin{exemple}
\sauteligne
  \begin{enumerate}
    \item On considère la suite $(u_n)_{n\in \Nn}$ de terme général
$u_n=(-1)^n$. Alors on peut considérer les deux sous-suites $(u_{2n})_{n\in \Nn}$
et $(u_{2n+1})_{n\in \Nn}$.
    \item On considère la suite $(v_n)_{n\in \Nn}$ de terme général $v_n=\cos n$.
Le théorème affirme qu'il existe une sous-suite convergente, mais il est moins facile de l'expliciter.
  \end{enumerate}
\end{exemple}

\begin{proof}[Démonstration du théorème \ref{thm:Bolzano_Weierstrass}]
On procède par dichotomie. L'ensemble des valeurs de la suite est par hypothèse
contenu dans un intervalle $[a,b]$. Posons $a_0=a$, $b_0=b$, $\phi(0)=0$.
Au moins l'un des deux intervalles $\left[a_0, \frac{a_0+b_0}{2}\right]$ ou
$\left[\frac{a_0+b_0}{2},b_0 \right]$ contient $u_n$ pour une infinité d'indices $n$.
On note $[a_1,b_1]$ un tel intervalle, et on note $\phi(1)$ un entier $\phi(1)>\phi(0)$
tel que $u_{\phi(1)}\in [a_1,b_1]$.

\myfigure{1}{
\tikzinput{fig_suites13}
}

En itérant cette construction,
on construit pour tout entier naturel $n$ un intervalle $[a_n,b_n]$,
de longueur $\frac{b-a}{2^n}$, et un entier $\phi(n)>\phi(n-1)$ tel que $u_{\phi(n)}\in [a_n,b_n]$.
Notons que par construction la suite $(a_n)_{n\in \Nn}$ est croissante et la suite
$(b_n)_{n\in \Nn}$ est décroissante.

% \myfigure{1}{
% \tikzinput{fig_suites14}
% }

Comme de plus $\lim_{n\to +\infty} (b_n-a_n)=
 \lim_{n\to +\infty} \frac{b-a}{2^n}=0$, les suites $(a_n)_{n\in \Nn}$ et
$(b_n)_{n\in \Nn}$ sont adjacentes et donc convergent vers une même limite $\ell$.
On peut appliquer le théorème \og des gendarmes \fg{} pour conclure que
$\lim_{n\to +\infty} u_{\phi(n)}=\ell$.
\end{proof}




%---------------------------------------------------------------
%\subsection{Mini-exercices}

\begin{miniexercices}
\sauteligne
\begin{enumerate}
  \item Soit $(u_n)_{n\in \Nn}$ la suite définie par $u_0=1$ et pour $n\geq 1$,
$u_n=\sqrt{2+u_{n-1}}$. Montrer que cette suite est croissante et majorée par $2$.
Que peut-on en conclure ?

  \item Soit $(u_n)_{n \ge 2}$ la suite définie par
  $u_n =\frac{\ln 4}{\ln 5}\times \frac{\ln 6}{\ln 7}\times\frac{\ln 8}{\ln 9}\times
  \cdots \times\frac{\ln (2n)}{\ln (2n+1)}.$  \'Etudier la croissance de la suite.
  Montrer que la suite $(u_n)$ converge.

  \item Soit $N\geq 1$ un entier et $(u_n)_{n\in \Nn}$ la suite de terme général
$u_n=\cos(\frac{n\pi}{N})$. Montrer que la suite diverge.

  \item Montrer que les suites de terme général $u_n=\sum_{k=1}^n \frac{1}{k!}$ et
$v_n=u_n+\frac{1}{n \cdot (n!)}$ sont adjacentes. Que peut-on en déduire ?

   \item Soit $(u_n)_{n\geq 1}$ la suite de terme général $\sum_{k=1}^{n}\frac{(-1)^{k+1}}{k}$.
On considère les deux suites extraites de terme général $v_n=u_{2n}$ et $w_n= u_{2n+1}$.
Montrer que les deux suites $(v_n)_{n\geq 1}$ et $(w_n)_{n\geq 1}$ sont adjacentes.
En déduire que la suite $(u_n)_{n\geq 1}$ converge.

   \item Montrer qu'une suite bornée et divergente admet deux sous-suites convergeant
   vers des valeurs distinctes.
\end{enumerate}
\end{miniexercices}


%%%%%%%%%%%%%%%%%%%%%%%%%%%%%%%%%%%%%%%%%%%%%%%%%%%%%%%%%%%%%%%%
\section{Suites récurrentes}

%Une catégorie essentielle de suites sont les suites récurrentes définies par une fonction.
Les suites récurrentes définies par une fonction forment une catégorie essentielle de suites.


Ce 
%chapitre 
paragraphe est l'aboutissement de notre étude %sur les 
des suites, mais sa lecture nécessite aussi la maîtrise préalable de l'étude de fonctions (voir \og Limites et fonctions continues \fg).

%---------------------------------------------------------------
\subsection{Suite récurrente définie par une fonction}

Soit $f : \Rr \to \Rr$ une fonction. Une \defi{suite récurrente}\index{suite!recurrente@récurrente} 
est définie par son premier terme et une relation permettant de calculer les termes de proche en proche :
$$u_0 \in \Rr \quad \text{ et } \quad u_{n+1} = f(u_n) \ \ \text{ pour } n \ge 0.$$
Une suite récurrente est donc définie par deux données : un terme initial $u_0$, et une
relation de récurrence $u_{n+1}=f(u_n)$. La suite s'écrit ainsi :
$$u_0, \quad u_1 = f(u_0), \quad u_2 = f(u_1) =f(f(u_0)), \quad u_3 = f(u_2) =f(f(f(u_0))),\ldots$$

Le comportement d'une telle suite peut très vite devenir complexe.
\begin{exemple}
Soit $f(x)=1+\sqrt{x}$. Fixons $u_0= 2$ et définissons pour $n\ge0$ :
$u_{n+1}=f(u_n)  $. C'est-à-dire $u_{n+1}=1+\sqrt{u_n}$.
Alors les premiers termes de la suite sont :
$$2, \quad 1+\sqrt{2}, \quad 1+\sqrt{1+\sqrt{2}},  \quad 1+\sqrt{1+\sqrt{1+\sqrt{2}}},  \quad 1+\sqrt{1+\sqrt{1+\sqrt{1+\sqrt{2}}}},\ldots$$
\end{exemple}


Une suite récurrente donnée n'est pas forcément convergente. Lorsqu'elle admet une limite, l'ensemble des valeurs possibles est restreint par le résultat essentiel suivant.

\begin{proposition}
\label{prop:fll}
Si $f$ est une fonction continue et la suite récurrente $(u_n)$ converge vers $\ell$, alors
$\ell$ est une solution de l'équation :
\mybox{$f(\ell)=\ell$}
\end{proposition}

Si on arrive à montrer que la limite existe, cette proposition affirme qu'elle est à chercher parmi les solutions de l'équation $f(\ell)=\ell$.
\myfigure{1}{
\tikzinput{fig_suites15}
}
Une valeur $\ell$, vérifiant $f(\ell)=\ell$ est un \defi{point fixe}\index{point fixe} de $f$.
La preuve est très simple et %mérite d'être refaite à chaque fois.
utilise essentiellement la continuité de la fonction $f$ :
\begin{proof}
Lorsque $n\to +\infty$, $u_n\to \ell$ et donc aussi $u_{n+1} \to \ell$.
Comme $u_n\to \ell$ et que $f$ est continue alors
la suite $(f(u_n)) \to f(\ell)$.
La relation $u_{n+1} = f(u_n)$ devient à la limite (lorsque $n\to+\infty$) : $\ell=f(\ell)$.
\end{proof}



Nous allons étudier en détail deux cas particuliers, celui ou la fonction est croissante, puis celui ou la fonction est décroissante. %fondamentaux : lorsque la fonction est croissante,puis lorsque la fonction est décroissante.

%---------------------------------------------------------------
\subsection{Cas d'une fonction croissante}

Commençons par remarquer que pour une fonction croissante, le comportement de la suite
$(u_n)$ définie par récurrence est assez simple :
\begin{itemize}
  \item Si $u_1\ge u_0$ alors $(u_n)$ est croissante.
  \item Si $u_1 \le u_0$ alors $(u_n)$ est décroissante.
\end{itemize}

La preuve est %une simple 
facile par récurrence :
par exemple si $u_1\ge u_0$, alors comme $f$ est croissante
on a $u_2 = f(u_1)\ge f(u_0)=u_1$. Partant de $u_2\ge u_1$ on en déduit $u_3 \ge u_2$,...

\bigskip

Voici le résultat principal :
\begin{proposition}
\label{prop:fcroissante}
Si $f : [a,b] \to [a,b]$ une fonction continue et \evidence{croissante},
alors quelque soit $u_0 \in [a,b]$, la suite récurrente $(u_n)$ est
monotone et converge vers $\ell \in [a,b]$ vérifiant \myboxinline{$f(\ell)=\ell$}.
\end{proposition}

Il y a une hypothèse importante qui est un peu cachée : $f$ va de l'intervalle $[a,b]$ dans
lui-même. Dans la pratique, pour appliquer cette proposition, il faut commencer par choisir $[a,b]$
et vérifier que $f([a,b]) \subset [a,b]$.

\myfigure{1}{
\tikzinput{fig_suites16}
}


\begin{proof}
La preuve est une conséquence des résultats précédents.
Par exemple si $u_1\ge u_0$ alors la suite $(u_n)$ est croissante, comme par ailleurs elle est majorée par
$b$, %donc 
elle converge vers un réel $\ell$. Par la proposition \ref{prop:fll}, %alors
on a $f(\ell)=\ell$.
Si $u_1 \le u_0$, %alors
$(u_n)$ est une décroissante et minorée par $a$, et la conclusion est la même.
\end{proof}

\begin{exemple}
Soit $f : \Rr \to \Rr$ définie par $f(x)=\frac14(x^2-1)(x-2)+x$ et $u_0 \in [0,2]$.
\'Etudions la suite $(u_n)$ définie par récurrence : $u_{n+1}=f(u_n)$ (pour tout $n\ge0$).

\begin{enumerate}

  \item \'Etude de $f$
  \begin{enumerate}
    \item $f$ est continue sur $\Rr$.
    \item $f$ est dérivable sur $\Rr$ et $f'(x)>0$.
    \item Sur l'intervalle $[0,2]$, $f$ est strictement croissante.
    \item Et comme $f(0)=\frac12$ et $f(2)=2$ alors $f([0,2]) \subset [0,2]$.
  \end{enumerate}

  \item Graphe de $f$

\myfigure{1}{
\tikzinput{fig_suites17}
}


Voici comment tracer la suite : on trace le graphe de $f$ et la bissectrice $(y=x)$.
On part d'une valeur $u_0$ \couleurnb{(en rouge)}{} sur l'axe des abscisses, la valeur $u_1=f(u_0)$ se lit
sur l'axe des ordonnées, mais on reporte la valeur de $u_1$ sur l'axe des
abscisses par symétrie par rapport à la bissectrice.
On recommence : $u_2=f(u_1)$ se lit sur l'axe des ordonnées et on le reporte sur l'axe des abscisses, etc.
On obtient ainsi une sorte d'escalier, et graphiquement on conjecture que la suite est croissante et tend vers $1$.
Si on part d'une autre valeur initiale $u_0'$ \couleurnb{(en vert)}{}, c'est le même principe, mais cette fois on obtient
un escalier qui descend.





  \item Calcul des points fixes.

Cherchons les valeurs $x$ qui vérifient $(f(x)=x)$, autrement dit $(f(x)-x=0)$, mais
\begin{equation}
\label{eq:fxmoinsx}
f(x)-x=\frac14 (x^2-1)(x-2)
\end{equation}

Donc les points fixes sont les $\{-1,1,2\}$.
La limite de $(u_n)$ est donc à chercher parmi ces $3$ valeurs.


  \item Premier cas : $u_0=1$ ou $u_0=2$.

  Alors $u_1=f(u_0)=u_0$ et par récurrence la suite $(u_n)$ est constante (et converge
  donc vers $u_0$).

  \item Deuxième cas : $0 \le u_0 <1$.
  \begin{itemize}
    \item Comme $f([0,1])\subset [0,1]$, la fonction $f$ se restreint sur l'intervalle $[0,1]$ en une fonction
    $f : [0,1] \to [0,1]$.

    \item De plus sur $[0,1]$, $f(x)-x\ge0$. Cela se déduit de l'étude de $f$ ou
    directement de l'expression (\ref{eq:fxmoinsx}).

    \item Pour $u_0 \in [0,1[$, $u_1 = f(u_0) \ge u_0$ d'après le point précédent. Comme $f$ est croissante, par récurrence, comme on l'a vu, la suite $(u_n)$ est croissante.

    \item La suite $(u_n)$ est croissante et majorée par $1$, donc elle converge. Notons $\ell$ sa limite.

    \item D'une part $\ell$ doit être un point fixe de $f$ : $f(\ell)=\ell$. Donc $\ell \in \{-1,1,2\}$.

    \item D'autre part la suite $(u_n)$ étant croissante avec $u_0\ge0$ et majorée par $1$, donc $\ell \in [0,1]$.

    \item Conclusion : si $0\le u_0<1$ alors $(u_n)$ converge vers $\ell = 1$.
  \end{itemize}

  \item Troisième cas : $1 < u_0 < 2$.

  La fonction $f$ se restreint en $f : [1,2] \to [1,2]$. Sur l'intervalle $[1,2]$, $f$ est croissante
  mais cette fois $f(x) \le x$. Donc $u_1 \le u_0$, et la suite $(u_n)$ est décroissante.
  La suite $(u_n)$ étant minorée par $1$, elle converge. Si on note $\ell$ sa limite alors
  d'une part $f(\ell)=\ell$, donc $\ell \in \{-1,1,2\}$,
  et d'autre part $\ell \in [1,2[$. Conclusion : $(u_n)$ converge vers $\ell=1$.

\end{enumerate}


\end{exemple}



Le graphe de $f$ joue un rôle très important, il faut le tracer même si on ne le demande pas explicitement.
Il permet de se faire une idée très précise du comportement de la suite :
Est-elle croissante ? Est-elle positive ? Semble-t-elle converger ? Vers quelle limite ?
Ces indications sont essentielles pour savoir ce qu'il faut montrer lors de l'étude de la suite.




%---------------------------------------------------------------
\subsection{Cas d'une fonction décroissante}



\begin{proposition}
Soit $f : [a,b] \to [a,b]$ une fonction continue et \evidence{décroissante}.
Soit $u_0 \in [a,b]$ et la suite récurrente $(u_n)$ définie par $u_{n+1}=f(u_n)$. Alors :
\begin{itemize}
  \item La sous-suite $(u_{2n})$ converge vers une limite $\ell$  vérifiant $f\circ f(\ell)=\ell$.
  \item La sous-suite $(u_{2n+1})$ converge vers une limite $\ell'$ vérifiant $f\circ f(\ell')=\ell'$.
\end{itemize}
\end{proposition}


Il se peut (ou pas !) que $\ell=\ell'$.

\begin{proof}
La preuve se déduit du cas croissant.
La fonction $f$ étant décroissante, la fonction $f\circ f$ est croissante.
Et on applique la proposition \ref{prop:fcroissante} à la fonction $f\circ f$ et
à la sous-suite $(u_{2n})$ définie par récurrence $u_2 = f\circ f(u_0)$, $u_4 = f\circ f(u_2)$,\ldots

De même en partant de $u_1$ et $u_3 = f \circ f(u_1)$,\ldots
\end{proof}

\begin{exemple}
$$f(x)=1+\frac1x,\qquad u_0 >0, \qquad u_{n+1} = f(u_n)= 1 + \frac{1}{u_n}$$


\begin{enumerate}
  \item \'Etude de $f$. La fonction $f : ]0,+\infty[\to ]0,+\infty[$ est
  une fonction continue et strictement décroissante.

  \item Graphe de $f$.

\myfigure{1}{
\tikzinput{fig_suites18}
}

Le principe pour tracer la suite est le même qu'auparavant : on place $u_0$,
on trace $u_1=f(u_0)$ sur l'axe des ordonnées et on le reporte par symétrie sur l'axe des abscisses,...
On obtient ainsi une sorte d'escargot, et graphiquement on conjecture que la suite converge
vers le point fixe de $f$. En plus on note que la suite des termes de rang pair semble une suite croissante,
alors que la suite des termes de rang impair semble décroissante.

  \item Points fixes de $f\circ f$.

  $$f\circ f(x)= f\big( f(x)\big)= f\big(1+\frac1x\big)= 1+ \frac{1}{1+\frac1x}=1+\frac{x}{x+1} = \frac{2x+1}{x+1}$$
  Donc
  $$f\circ f(x)=x \iff  \frac{2x+1}{x+1} = x \iff x^2-x-1=0 \iff x \in
\left\{\frac{1-\sqrt{5}}{2},\frac{1+\sqrt{5}}{2}\right\}$$

  Comme la limite doit être positive, le seul point fixe à considérer est $\ell=\frac{1+\sqrt{5}}{2}$.

  Attention ! Il y a un unique point fixe, mais on ne peut pas conclure à ce stade car $f$ est définie sur $]0,+\infty[$ qui n'est pas un intervalle compact.

  \item Premier cas $0 < u_0 \le \ell = \frac{1+\sqrt{5}}{2}$.

  Alors,  $u_1 = f(u_0) \ge f(\ell)=\ell$ ; et par une étude de $f\circ f(x)-x$, on obtient que  :  $u_2 = f\circ f(u_0) \ge u_0$ ; $u_1  \ge f\circ f(u_1)=u_3$.

  Comme $u_2 \ge u_0$ et $f\circ f$ est croissante, la suite $(u_{2n})$ est croissante.
  De même $u_3  \le u_1$, donc la suite $(u_{2n+1})$ est décroissante.
  De plus comme  $u_0 \le u_1$, en appliquant $f$ un nombre \emph{pair} de fois, on obtient que $u_{2n} \le u_{2n+1}$.
  La situation est donc la suivante :
  $$u_0 \le u_2 \le \cdots \le u_{2n} \le \cdots \le u_{2n+1} \le \cdots \le u_3 \le u_1$$

  La suite $(u_{2n})$ est croissante et majorée par $u_1$, donc elle converge. Sa limite ne peut
  être que l'unique point fixe de $f\circ f$: $\ell = \frac{1+\sqrt{5}}{2}$.

   La suite $(u_{2n+1})$ est décroissante et minorée par $u_0$, donc elle converge aussi vers
   $\ell = \frac{1+\sqrt{5}}{2}$.

   On en conclut que la suite $(u_{n})$ converge vers $\ell = \frac{1+\sqrt{5}}{2}$.

  \item Deuxième cas $u_0 \ge \ell = \frac{1+\sqrt{5}}{2}$.

  On montre de la même façon que $(u_{2n})$ est décroissante et converge vers $\frac{1+\sqrt{5}}{2}$,
  et que $(u_{2n+1})$ est croissante et converge aussi vers $\frac{1+\sqrt{5}}{2}$.


\end{enumerate}

\end{exemple}



%---------------------------------------------------------------
%\subsection{Mini-exercices}

\begin{miniexercices}
\sauteligne
\begin{enumerate}

  \item Soit $f(x)=\frac19x^3+1$, $u_0=0$ et pour $n\ge0$ :
  $u_{n+1}=f(u_n)$. \'Etudier en détail la suite $(u_n)$ :
(a) montrer que $u_n\ge0$ ;
(b) étudier et tracer le graphe de $g$ ;
(c) tracer les premiers termes de $(u_n)$ ;
(d) montrer que $(u_n)$ est croissante ;
(e) étudier la fonction $g(x)=f(x)-x$ ;
(f) montrer que $f$ admet deux points fixes sur $\Rr_+$, $0 < \ell < \ell'$ ;
(g) montrer que $f([0,\ell]) \subset [0,\ell]$ ;
(h) en déduire que $(u_n)$ converge vers $\ell$.


  \item Soit $f(x)=1+\sqrt{x}$, $u_0= 2$ et pour $n\ge0$ :
  $u_{n+1}=f(u_n)$. \'Etudier en détail la suite $(u_n)$.


  \item Soit $(u_n)_{n\in\Nn}$ la suite définie par :
$u_0 \in[0,1]$ et $u_{n+1} = u_n - u_n^2$. \'Etudier en détail la suite $(u_n)$.

  \item \'Etudier la suite définie par $u_0=4$ et $u_{n+1}=\frac{4}{u_n+2}$.

\end{enumerate}
\end{miniexercices}


\auteurs{


Auteurs : Arnaud Bodin, Niels Borne, Laura Desideri

Dessins : Benjamin Boutin
}

\finchapitre
\end{document}


