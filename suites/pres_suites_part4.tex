\input{./preamb-pres-suites.tex}



%%%%%%%%%%%%%%%%%%%%%%%%%%%%%%%%%%%%%%%%%%%%%%%%%%%%%%%%%%%%%
%%%%%%%%%%%%%%%%%%%%%%%%%%%%%%%%%%%%%%%%%%%%%%%%%%%%%%%%%%%%%



\begin{document}




\title{{\bf Suites}}
\subtitle{Théorèmes de convergence}


\begin{frame}
  
  \debutmontitre

  \pause

{\footnotesize
\hfill
\setbeamercovered{transparent=50}
\begin{minipage}{0.6\textwidth}
  \begin{itemize}
    \item<3-> Suite monotone
    \item<4-> Deux exemples
    \item<5-> Suites adjacentes
    \item<6-> Théorème de Bolzano-Weierstrass
  \end{itemize}
\end{minipage}
}

\end{frame}

\setcounter{framenumber}{0}

%%%%%%%%%%%%%%%%%%%%%%%%%%%%%%%%%%%%%%%%%%%%%%%%%%%%%%%%%%%%%%%
\section{Suite monotone}

\begin{frame}
\begin{theoreme}
%\vspace*{-2ex}
  \mybox{Toute suite croissante et majorée est convergente}
\end{theoreme}
\pause

  \begin{itemize}
    \item Toute suite décroissante et minorée est convergente
\pause    
    \item Une suite croissante et qui n'est pas majorée tend vers $+\infty$
  \end{itemize}

\pause
\begin{proof}
\begin{itemize}
  \item Notons $A=\left\{ u_n| n\in \Nn\right\} \subset \Rr$
\pause 
  \item Si $M$ majore la suite $(u_n)$ alors $A$ est majoré par $M$
\pause  
  \item Notons $\ell=\sup A$
\pause  
  \item Montrons que $\lim_{n\to +\infty} u_n=\ell$
\pause    
  \begin{itemize}
    \item Soit $\epsilon >0$
    \item Il existe $u_N \in A$ tel que  $\ell-\epsilon < u_N \leq \ell$
    \item Pour $n\geq N$ on a $\ell-\epsilon < u_N \leq u_n \leq \ell$
    \item Donc $\lvert u_n-\ell \rvert \leq \epsilon$ \qedhere
  \end{itemize}

\end{itemize}

\end{proof}

\end{frame}


\section{Deux exemples}

\begin{frame} 
\defi{Zêta(2)} 
\mybox{$ u_n =1+\frac{1}{2^2} +\frac{1}{3^2}+\cdots+\frac{1}{n^2}$}

\pause
\begin{itemize}
  \item  $(u_n)$ est croissante : $u_{n+1}-u_n= \frac{1}{(n+1)^2}>0$
\pause
  \item Montrons par récurrence que $u_n\leq 2 - \frac{1}{n}$
\pause  
  \begin{itemize}
    \item $u_1=1\leq 2 - \frac{1}{1}$
\pause    
    \item Fixons $n\geq 1$ pour lequel on suppose $u_n\leq 2 - \frac{1}{n}$
    \begin{itemize}
      \item $u_{n+1}=u_n+ \frac{1}{(n+1)^2}\leq  2 - \frac{1}{n}+ \frac{1}{(n+1)^2}$
      \item Or $\frac{1}{(n+1)^2}\leq \frac{1}{n(n+1)}=\frac{1}{n}-\frac{1}{n+1}$
      \item Donc $u_{n+1}\leq 2-\frac{1}{n+1}$
    \end{itemize}
\pause    
    \item Ce qui achève la récurrence 
  \end{itemize}

  \pause  
  \item $(u_n)$ est croissante et majorée par $2$ : elle converge

\pause 
  \item Cette limite est  $\zeta(2)=\frac{\pi^2}{6}$
\end{itemize}
\end{frame}

\begin{frame}
\defi{Suite harmonique} 
\mybox{$ u_n =1+\frac{1}{2} +\frac{1}{3}+\cdots+\frac{1}{n}$}

\bigskip
\pause


Calculons $\lim_{n\to +\infty} u_n$
\pause
\begin{itemize}
  \item La suite $(u_n)$ est croissante : $u_{n+1}-u_n= \frac{1}{n+1}>0$
\pause

  \item Minoration de $u_{2^p}-u_{2^{p-1}}$ :
  \pause
\[ u_{2^p}-u_{2^{p-1}}
\pause
=\underbrace{\frac{1}{2^{p-1}+1}+ \frac{1}{2^{p-1}+2}+\cdots+\frac{1}{2^p}}_{\pause 2^{p-1}
=2^p-2^{p-1}
\text{ termes }
\pause
\geq \frac{1}{2^p}}
\pause
> 2^{p-1}\times \frac{1}{2^p}
\pause
=\frac{1}{2}\]

\vspace*{-2ex}
\pause

  \item $ u_{2^p}-1 = u_{2^p}-u_1= (u_2 - u_1)+ (u_4 - u_2)+ \cdots +  (u_{2^p}-u_{2^{p-1}}) \geq \frac{p}{2} $


\pause

 \item $\lim_{n\to +\infty} u_n=+\infty$
\end{itemize}

\end{frame}

\section{Suites adjacentes}


\begin{frame}
\begin{mydefinition}
Deux suites $(u_n)_{n\in \Nn}$ et $(v_n)_{n\in \Nn}$ sont dites \defi{adjacentes} si
\begin{enumerate}
  \item $(u_n)_{n\in \Nn}$ est croissante et $(v_n)_{n\in \Nn}$ est décroissante
  \item pour tout $n\geq 0$, on a $u_n\leq v_n$
  \item $\lim_{n\to +\infty} (v_n -u_n) = 0$
\end{enumerate}
\end{mydefinition}

\pause
\medskip

\begin{theoreme}
Si $(u_n)$ et $(v_n)$ sont adjacentes, 
  elles convergent et ont la même limite
\end{theoreme}

\pause

$$u_0 \le u_1 \le u_2 \le \cdots \le u_n \le \cdots \cdots \le v_n\le \cdots \le v_2 \le v_1 \le v_0$$

\pause

\begin{proof}
  \begin{itemize}
    \item $(u_n)$ est croissante et majorée par $v_0$ donc elle converge vers une limite $\ell$
\pause
    \item $(v_n)$ est décroissante et minorée par $u_0$, donc elle converge vers une limite $\ell'$
\pause
    \item Comme $\lim_{n\to +\infty} (v_n -u_n) = 0$, d'où $\ell'-\ell=0$ \qedhere
  \end{itemize}
\end{proof}
	
\end{frame}


\begin{frame}


\defi{Zêta(2)} 
\vspace*{-1ex}
$$u_n = \sum_{k=1}^n \frac 1 {k^2} = 1+\frac{1}{2^2} +\frac{1}{3^2}+\cdots+\frac{1}{n^2}
\pause 
\quad \text{ et } \quad v_n = u_n + \frac2{n+1}$$

\pause
\medskip
$(u_n)$ et $(v_n)$ sont deux suites adjacentes :
\medskip
\pause
\begin{enumerate}
  \item 
  \begin{itemize}
    \item $(u_n)$ est croissante car $u_{n+1}-u_n = \frac{1}{(n+1)^2} > 0$
\pause    
    \item $(v_n)$ est décroissante :
    $v_{n+1}-v_n = \frac{1}{(n+1)^2} + \frac{2}{n+2} - \frac{2}{n+1} 
\pause  = \frac{n+2+2(n+1)^2-2(n+1)(n+2)}{(n+2)(n+1)^2} 
\pause  = \frac{-n}{(n+2)(n+1)^2}< 0$
  \end{itemize}
\pause   
  \item Pour tout $n\ge 1$ : $v_n-u_n = \frac{2}{n+1} >0$, \pause donc $u_n \le v_n$
\pause 
  \item $v_n-u_n = \frac{2}{n+1}$  \pause donc $\lim (v_n-u_n) = 0$
\end{enumerate}

\pause  
\bigskip

Conclusion
\begin{itemize}
  \item Les suites $(u_n)$ et $(v_n)$ sont adjacentes
\pause    
  \item Elles convergent vers une même limite $\ell$
\pause    
  \item Encadrement $u_n \le \ell \le v_n$
\pause    
  \item Exemple $n=3$ \quad $1+\frac{1}{4} +\frac{1}{9} \le \ell \le 1+\frac{1}{4} +\frac{1}{9} + \frac{1}{2}$
\pause   
  \item $1,3611\ldots \le \ell \le 1,8611\ldots$
\end{itemize}

\end{frame}

\section{Théorème de Bolzano-Weierstrass}

\begin{frame}
\begin{mydefinition}
Soit $(u_n)_{n\in \Nn}$ une suite. Une \defi{suite extraite} ou  
\defi{sous-suite} de $(u_n)_{n\in \Nn}$ est une suite de la forme 
$(u_{\phi(n)})_{n\in \Nn}$, où $\phi : \Nn \to \Nn$ est une application strictement croissante
\end{mydefinition}



\myfigure{1.1}{\tikzinput{fig_suites10}}  
\end{frame}


\begin{frame}

\begin{mydefinition}
Une \defi{suite extraite} de $(u_n)_{n\in \Nn}$ est une suite de la forme 
$(u_{\phi(n)})_{n\in \Nn}$
\end{mydefinition}

\pause

Exemple : $(u_n)_{n\in \Nn}$ la suite de terme général $u_n=(-1)^n$

\pause

\begin{itemize}
  \item $\phi : \Nn \to \Nn$, $\phi(n)=2n$, $u_{\phi(n)}=(-1)^{2n}=1$
 
\pause

La suite extraite $(u_{\phi(n)})_{n\in \Nn}$ est constante égale à $1$

\pause


\myfigure{0.85}{\tikzinput{fig_suites11} \quad 
\uncover<8->{\tikzinput{fig_suites12}} }

\pause
  \item $\psi : \Nn \to \Nn$, $\psi(n)=3n$, $u_{\psi(n)}=(-1)^{3n}=\big( (-1)^3\big)^n = (-1)^n$
 
\pause

La suite $(u_{\psi(n)})_{n\in \Nn}$ est donc égale à $(u_n)_{n\in \Nn}$


\end{itemize}




\end{frame}


\begin{frame}

\begin{proposition}
Soit $(u_n)_{n\in \Nn}$ une suite. Si $\lim_{n\to +\infty}u_n=\ell$, alors 
pour toute suite extraite $(u_{\phi(n)})_{n\in \Nn}$ 
on a $\lim_{n\to +\infty} u_{\phi(n)}=\ell$
\end{proposition}

\pause
\bigskip

\evidence{Critère de divergence.} Si une suite $(u_n)_{n\in \Nn}$ admet une sous-suite divergente, 
ou bien si elle admet deux sous-suites convergeant vers des 
limites distinctes, alors elle diverge


\pause
\bigskip

Exemple :
\begin{itemize}
  \item Soit la suite $(u_n)_{n\in \Nn}$ définie par $u_n=(-1)^n$
  \item Alors $(u_{2n})_{n\in \Nn}$ converge vers $1$
  \item Et $(u_{2n+1})_{n\in \Nn}$ converge vers $-1$
  \item Conclusion : la suite $(u_n)_{n\in \Nn}$ diverge
\end{itemize}
	
\end{frame}


\begin{frame}
\begin{theoreme}[de Bolzano-Weierstrass]
\label{thm:Bolzano_Weierstrass}
  Toute suite bornée admet une sous-suite convergente
\end{theoreme}

\pause
\bigskip
\bigskip

Exemples :
  \begin{enumerate}
    \item Si $(u_n)_{n\in \Nn}$ est définie par 
    $u_n=(-1)^n$ on a deux sous-suites $(u_{2n})_{n\in \Nn}$ 
et $(u_{2n+1})_{n\in \Nn}$ convergente

\pause
\medskip

    \item Si $(v_n)_{n\in \Nn}$ a pour terme général $v_n=\cos n$, 
le théorème affirme qu'il existe une sous-suite convergente, 
mais il est moins facile de l'expliciter
  \end{enumerate}

\end{frame}



%%%%%%%%%%%%%%%%%%%%%%%%%%%%%%%%%%%%%%%%%%%%%%%%%%%%%%%%%%%%%%%
 \section*{Mini-exercices}

\begin{frame}
 \begin{miniexercice}	
\begin{enumerate}
  \item Soit $(u_n)_{n\in \Nn}$ la suite définie par $u_0=1$ et pour $n\geq 1$, 
$u_n=\sqrt{2+u_{n-1}}$. Montrer que cette suite est croissante et majorée par $2$. 
Que peut-on en conclure ?

  \item Soit $(u_n)_{n \ge 2}$ la suite définie par 
  $u_n =\frac{\ln 4}{\ln 5}\times \frac{\ln 6}{\ln 7}\times\frac{\ln 8}{\ln 9}\times 
  \cdots \times\frac{\ln (2n)}{\ln (2n+1)}.$  \'Etudier la croissance de la suite.
  Montrer que la suite $(u_n)$ converge.
  
  \item Soit $N\geq 1$ un entier et $(u_n)_{n\in \Nn}$ la suite de terme général 
$u_n=\cos(\frac{n\pi}{N})$. Montrer que la suite diverge. 
  
  \item Montrer que les suites de terme général $u_n=\sum_{k=1}^n \frac{1}{k!}$ et 
$v_n=u_n+\frac{1}{n \cdot (n!)}$ sont adjacentes. Que peut-on en déduire ? 

   \item Soit $(u_n)_{n\geq 1}$ la suite de terme général $\sum_{k=1}^{n}\frac{(-1)^{k+1}}{k}$. 
On considère les deux suites extraites de terme général $v_n=u_{2n}$ et $w_n= u_{2n+1}$. 
Montrer que les deux suites $(v_n)_{n\geq 1}$ et $(w_n)_{n\geq 1}$ sont adjacentes. 
En déduire que la suite $(u_n)_{n\geq 1}$ converge.

 %  \item Montrer qu'une suite bornée et divergente admet deux sous-suites convergeant 
  % vers des valeurs distinctes. 
\end{enumerate}

 \end{miniexercice}

 \end{frame}


\end{document}
