
%%%%%%%%%%%%%%%%%% PREAMBULE %%%%%%%%%%%%%%%%%%

\documentclass[aspectratio=169,utf8]{beamer}
%\documentclass[aspectratio=169,handout]{beamer}

\usetheme{Boadilla}
%\usecolortheme{seahorse}
\usecolortheme[RGB={245,66,24}]{structure}
\useoutertheme{infolines}

% packages
\usepackage{amsfonts,amsmath,amssymb,amsthm}
\usepackage[utf8]{inputenc}
\usepackage[T1]{fontenc}
\usepackage{lmodern}

\usepackage[francais]{babel}
\usepackage{fancybox}
\usepackage{graphicx}

\usepackage{float}
\usepackage{xfrac}

%\usepackage[usenames, x11names]{xcolor}
\usepackage{tikz}
\usepackage{pgfplots}
\usepackage{datetime}



%-----  Package unités -----
\usepackage{siunitx}
\sisetup{locale = FR,detect-all,per-mode = symbol}

%\usepackage{mathptmx}
%\usepackage{fouriernc}
%\usepackage{newcent}
%\usepackage[mathcal,mathbf]{euler}

%\usepackage{palatino}
%\usepackage{newcent}
% \usepackage[mathcal,mathbf]{euler}



% \usepackage{hyperref}
% \hypersetup{colorlinks=true, linkcolor=blue, urlcolor=blue,
% pdftitle={Exo7 - Exercices de mathématiques}, pdfauthor={Exo7}}


%section
% \usepackage{sectsty}
% \allsectionsfont{\bf}
%\sectionfont{\color{Tomato3}\upshape\selectfont}
%\subsectionfont{\color{Tomato4}\upshape\selectfont}

%----- Ensembles : entiers, reels, complexes -----
\newcommand{\Nn}{\mathbb{N}} \newcommand{\N}{\mathbb{N}}
\newcommand{\Zz}{\mathbb{Z}} \newcommand{\Z}{\mathbb{Z}}
\newcommand{\Qq}{\mathbb{Q}} \newcommand{\Q}{\mathbb{Q}}
\newcommand{\Rr}{\mathbb{R}} \newcommand{\R}{\mathbb{R}}
\newcommand{\Cc}{\mathbb{C}} 
\newcommand{\Kk}{\mathbb{K}} \newcommand{\K}{\mathbb{K}}

%----- Modifications de symboles -----
\renewcommand{\epsilon}{\varepsilon}
\renewcommand{\Re}{\mathop{\text{Re}}\nolimits}
\renewcommand{\Im}{\mathop{\text{Im}}\nolimits}
%\newcommand{\llbracket}{\left[\kern-0.15em\left[}
%\newcommand{\rrbracket}{\right]\kern-0.15em\right]}

\renewcommand{\ge}{\geqslant}
\renewcommand{\geq}{\geqslant}
\renewcommand{\le}{\leqslant}
\renewcommand{\leq}{\leqslant}
\renewcommand{\epsilon}{\varepsilon}

%----- Fonctions usuelles -----
\newcommand{\ch}{\mathop{\text{ch}}\nolimits}
\newcommand{\sh}{\mathop{\text{sh}}\nolimits}
\renewcommand{\tanh}{\mathop{\text{th}}\nolimits}
\newcommand{\cotan}{\mathop{\text{cotan}}\nolimits}
\newcommand{\Arcsin}{\mathop{\text{arcsin}}\nolimits}
\newcommand{\Arccos}{\mathop{\text{arccos}}\nolimits}
\newcommand{\Arctan}{\mathop{\text{arctan}}\nolimits}
\newcommand{\Argsh}{\mathop{\text{argsh}}\nolimits}
\newcommand{\Argch}{\mathop{\text{argch}}\nolimits}
\newcommand{\Argth}{\mathop{\text{argth}}\nolimits}
\newcommand{\pgcd}{\mathop{\text{pgcd}}\nolimits} 


%----- Commandes divers ------
\newcommand{\ii}{\mathrm{i}}
\newcommand{\dd}{\text{d}}
\newcommand{\id}{\mathop{\text{id}}\nolimits}
\newcommand{\Ker}{\mathop{\text{Ker}}\nolimits}
\newcommand{\Card}{\mathop{\text{Card}}\nolimits}
\newcommand{\Vect}{\mathop{\text{Vect}}\nolimits}
\newcommand{\Mat}{\mathop{\text{Mat}}\nolimits}
\newcommand{\rg}{\mathop{\text{rg}}\nolimits}
\newcommand{\tr}{\mathop{\text{tr}}\nolimits}


%----- Structure des exercices ------

\newtheoremstyle{styleexo}% name
{2ex}% Space above
{3ex}% Space below
{}% Body font
{}% Indent amount 1
{\bfseries} % Theorem head font
{}% Punctuation after theorem head
{\newline}% Space after theorem head 2
{}% Theorem head spec (can be left empty, meaning ‘normal’)

%\theoremstyle{styleexo}
\newtheorem{exo}{Exercice}
\newtheorem{ind}{Indications}
\newtheorem{cor}{Correction}


\newcommand{\exercice}[1]{} \newcommand{\finexercice}{}
%\newcommand{\exercice}[1]{{\tiny\texttt{#1}}\vspace{-2ex}} % pour afficher le numero absolu, l'auteur...
\newcommand{\enonce}{\begin{exo}} \newcommand{\finenonce}{\end{exo}}
\newcommand{\indication}{\begin{ind}} \newcommand{\finindication}{\end{ind}}
\newcommand{\correction}{\begin{cor}} \newcommand{\fincorrection}{\end{cor}}

\newcommand{\noindication}{\stepcounter{ind}}
\newcommand{\nocorrection}{\stepcounter{cor}}

\newcommand{\fiche}[1]{} \newcommand{\finfiche}{}
\newcommand{\titre}[1]{\centerline{\large \bf #1}}
\newcommand{\addcommand}[1]{}
\newcommand{\video}[1]{}

% Marge
\newcommand{\mymargin}[1]{\marginpar{{\small #1}}}

\def\noqed{\renewcommand{\qedsymbol}{}}


%----- Presentation ------
\setlength{\parindent}{0cm}

%\newcommand{\ExoSept}{\href{http://exo7.emath.fr}{\textbf{\textsf{Exo7}}}}

\definecolor{myred}{rgb}{0.93,0.26,0}
\definecolor{myorange}{rgb}{0.97,0.58,0}
\definecolor{myyellow}{rgb}{1,0.86,0}

\newcommand{\LogoExoSept}[1]{  % input : echelle
{\usefont{U}{cmss}{bx}{n}
\begin{tikzpicture}[scale=0.1*#1,transform shape]
  \fill[color=myorange] (0,0)--(4,0)--(4,-4)--(0,-4)--cycle;
  \fill[color=myred] (0,0)--(0,3)--(-3,3)--(-3,0)--cycle;
  \fill[color=myyellow] (4,0)--(7,4)--(3,7)--(0,3)--cycle;
  \node[scale=5] at (3.5,3.5) {Exo7};
\end{tikzpicture}}
}


\newcommand{\debutmontitre}{
  \author{} \date{} 
  \thispagestyle{empty}
  \hspace*{-10ex}
  \begin{minipage}{\textwidth}
    \titlepage  
  \vspace*{-2.5cm}
  \begin{center}
    \LogoExoSept{2.5}
  \end{center}
  \end{minipage}

  \vspace*{-0cm}
  
  % Astuce pour que le background ne soit pas discrétisé lors de la conversion pdf -> png
\begin{tikzpicture}
        \fill[opacity=0,green!60!black] (0,0)--++(0,0)--++(0,0)--++(0,0)--cycle; 
\end{tikzpicture}

% toc S'affiche trop tot :
% \tableofcontents[hideallsubsections, pausesections]
}

\newcommand{\finmontitre}{
  \end{frame}
  \setcounter{framenumber}{0}
} % ne marche pas pour une raison obscure

%----- Commandes supplementaires ------

% \usepackage[landscape]{geometry}
% \geometry{top=1cm, bottom=3cm, left=2cm, right=10cm, marginparsep=1cm
% }
% \usepackage[a4paper]{geometry}
% \geometry{top=2cm, bottom=2cm, left=2cm, right=2cm, marginparsep=1cm
% }

%\usepackage{standalone}


% New command Arnaud -- november 2011
\setbeamersize{text margin left=24ex}
% si vous modifier cette valeur il faut aussi
% modifier le decalage du titre pour compenser
% (ex : ici =+10ex, titre =-5ex

\theoremstyle{definition}
%\newtheorem{proposition}{Proposition}
%\newtheorem{exemple}{Exemple}
%\newtheorem{theoreme}{Théorème}
%\newtheorem{lemme}{Lemme}
%\newtheorem{corollaire}{Corollaire}
%\newtheorem*{remarque*}{Remarque}
%\newtheorem*{miniexercice}{Mini-exercices}
%\newtheorem{definition}{Définition}

% Commande tikz
\usetikzlibrary{calc}
\usetikzlibrary{patterns,arrows}
\usetikzlibrary{matrix}
\usetikzlibrary{fadings} 

%definition d'un terme
\newcommand{\defi}[1]{{\color{myorange}\textbf{\emph{#1}}}}
\newcommand{\evidence}[1]{{\color{blue}\textbf{\emph{#1}}}}
\newcommand{\assertion}[1]{\emph{\og#1\fg}}  % pour chapitre logique
%\renewcommand{\contentsname}{Sommaire}
\renewcommand{\contentsname}{}
\setcounter{tocdepth}{2}



%------ Figures ------

\def\myscale{1} % par défaut 
\newcommand{\myfigure}[2]{  % entrée : echelle, fichier figure
\def\myscale{#1}
\begin{center}
\footnotesize
{#2}
\end{center}}


%------ Encadrement ------

\usepackage{fancybox}


\newcommand{\mybox}[1]{
\setlength{\fboxsep}{7pt}
\begin{center}
\shadowbox{#1}
\end{center}}

\newcommand{\myboxinline}[1]{
\setlength{\fboxsep}{5pt}
\raisebox{-10pt}{
\shadowbox{#1}
}
}

%--------------- Commande beamer---------------
\newcommand{\beameronly}[1]{#1} % permet de mettre des pause dans beamer pas dans poly


\setbeamertemplate{navigation symbols}{}
\setbeamertemplate{footline}  % tiré du fichier beamerouterinfolines.sty
{
  \leavevmode%
  \hbox{%
  \begin{beamercolorbox}[wd=.333333\paperwidth,ht=2.25ex,dp=1ex,center]{author in head/foot}%
    % \usebeamerfont{author in head/foot}\insertshortauthor%~~(\insertshortinstitute)
    \usebeamerfont{section in head/foot}{\bf\insertshorttitle}
  \end{beamercolorbox}%
  \begin{beamercolorbox}[wd=.333333\paperwidth,ht=2.25ex,dp=1ex,center]{title in head/foot}%
    \usebeamerfont{section in head/foot}{\bf\insertsectionhead}
  \end{beamercolorbox}%
  \begin{beamercolorbox}[wd=.333333\paperwidth,ht=2.25ex,dp=1ex,right]{date in head/foot}%
    % \usebeamerfont{date in head/foot}\insertshortdate{}\hspace*{2em}
    \insertframenumber{} / \inserttotalframenumber\hspace*{2ex} 
  \end{beamercolorbox}}%
  \vskip0pt%
}


\definecolor{mygrey}{rgb}{0.5,0.5,0.5}
\setlength{\parindent}{0cm}
%\DeclareTextFontCommand{\helvetica}{\fontfamily{phv}\selectfont}

% background beamer
\definecolor{couleurhaut}{rgb}{0.85,0.9,1}  % creme
\definecolor{couleurmilieu}{rgb}{1,1,1}  % vert pale
\definecolor{couleurbas}{rgb}{0.85,0.9,1}  % blanc
\setbeamertemplate{background canvas}[vertical shading]%
[top=couleurhaut,middle=couleurmilieu,midpoint=0.4,bottom=couleurbas] 
%[top=fondtitre!05,bottom=fondtitre!60]



\makeatletter
\setbeamertemplate{theorem begin}
{%
  \begin{\inserttheoremblockenv}
  {%
    \inserttheoremheadfont
    \inserttheoremname
    \inserttheoremnumber
    \ifx\inserttheoremaddition\@empty\else\ (\inserttheoremaddition)\fi%
    \inserttheorempunctuation
  }%
}
\setbeamertemplate{theorem end}{\end{\inserttheoremblockenv}}

\newenvironment{theoreme}[1][]{%
   \setbeamercolor{block title}{fg=structure,bg=structure!40}
   \setbeamercolor{block body}{fg=black,bg=structure!10}
   \begin{block}{{\bf Th\'eor\`eme }#1}
}{%
   \end{block}%
}


\newenvironment{proposition}[1][]{%
   \setbeamercolor{block title}{fg=structure,bg=structure!40}
   \setbeamercolor{block body}{fg=black,bg=structure!10}
   \begin{block}{{\bf Proposition }#1}
}{%
   \end{block}%
}

\newenvironment{corollaire}[1][]{%
   \setbeamercolor{block title}{fg=structure,bg=structure!40}
   \setbeamercolor{block body}{fg=black,bg=structure!10}
   \begin{block}{{\bf Corollaire }#1}
}{%
   \end{block}%
}

\newenvironment{mydefinition}[1][]{%
   \setbeamercolor{block title}{fg=structure,bg=structure!40}
   \setbeamercolor{block body}{fg=black,bg=structure!10}
   \begin{block}{{\bf Définition} #1}
}{%
   \end{block}%
}

\newenvironment{lemme}[0]{%
   \setbeamercolor{block title}{fg=structure,bg=structure!40}
   \setbeamercolor{block body}{fg=black,bg=structure!10}
   \begin{block}{\bf Lemme}
}{%
   \end{block}%
}

\newenvironment{remarque}[1][]{%
   \setbeamercolor{block title}{fg=black,bg=structure!20}
   \setbeamercolor{block body}{fg=black,bg=structure!5}
   \begin{block}{Remarque #1}
}{%
   \end{block}%
}


\newenvironment{exemple}[1][]{%
   \setbeamercolor{block title}{fg=black,bg=structure!20}
   \setbeamercolor{block body}{fg=black,bg=structure!5}
   \begin{block}{{\bf Exemple }#1}
}{%
   \end{block}%
}


\newenvironment{miniexercice}[0]{%
   \setbeamercolor{block title}{fg=structure,bg=structure!20}
   \setbeamercolor{block body}{fg=black,bg=structure!5}
   \begin{block}{Mini-exercices}
}{%
   \end{block}%
}


\newenvironment{tp}[0]{%
   \setbeamercolor{block title}{fg=structure,bg=structure!40}
   \setbeamercolor{block body}{fg=black,bg=structure!10}
   \begin{block}{\bf Travaux pratiques}
}{%
   \end{block}%
}
\newenvironment{exercicecours}[1][]{%
   \setbeamercolor{block title}{fg=structure,bg=structure!40}
   \setbeamercolor{block body}{fg=black,bg=structure!10}
   \begin{block}{{\bf Exercice }#1}
}{%
   \end{block}%
}
\newenvironment{algo}[1][]{%
   \setbeamercolor{block title}{fg=structure,bg=structure!40}
   \setbeamercolor{block body}{fg=black,bg=structure!10}
   \begin{block}{{\bf Algorithme}\hfill{\color{gray}\texttt{#1}}}
}{%
   \end{block}%
}


\setbeamertemplate{proof begin}{
   \setbeamercolor{block title}{fg=black,bg=structure!20}
   \setbeamercolor{block body}{fg=black,bg=structure!5}
   \begin{block}{{\footnotesize Démonstration}}
   \footnotesize
   \smallskip}
\setbeamertemplate{proof end}{%
   \end{block}}
\setbeamertemplate{qed symbol}{\openbox}


\makeatother
\usecolortheme[RGB={34,139,34}]{structure}

%%%%%%%%%%%%%%%%%%%%%%%%%%%%%%%%%%%%%%%%%%%%%%%%%%%%%%%%%%%%%
%%%%%%%%%%%%%%%%%%%%%%%%%%%%%%%%%%%%%%%%%%%%%%%%%%%%%%%%%%%%%

\begin{document}




\title{{\bf Suites}}
\subtitle{Exemples remarquables}


\begin{frame}
  
  \debutmontitre

  \pause

{\footnotesize
\hfill
\setbeamercovered{transparent=50}
\begin{minipage}{0.6\textwidth}
  \begin{itemize}
    \item<3-> Suite géométrique
    \item<4-> Série géométrique
    \item<5-> Suites telles que $\left|\frac{u_{n+1}}{u_n}\right|<\ell<1$
    \item<6-> Approximation des réels par des décimaux
  \end{itemize}
\end{minipage}
}
\vspace*{1cm}
\end{frame}

\setcounter{framenumber}{0}

\section{Suite géométrique}

\begin{frame}
\begin{proposition}[Suite géométrique]
\label{prop:suitegeo}
On fixe un réel $a$. Soit $(u_n)_{n\in \Nn}$ la suite de terme général : $u_n=a^n$
\begin{enumerate}      
  \setlength{\itemsep}{7pt} 
\pause
  \item Si $a=1$, on a pour tout $n\in \Nn$ : $u_n=1$
\pause
  \item Si $a>1$, alors $\lim_{n\to +\infty} u_n= +\infty$
\pause  
  \item Si $-1<a<1$, alors $\lim_{n\to +\infty} u_n= 0$
\pause  
  \item Si $a\le-1$, la suite $(u_n)_{n\in \Nn}$ diverge
\end{enumerate}
\end{proposition}
\pause
\begin{proof}
  \begin{enumerate}
  \setcounter{enumi}{1}
    \item 
    
    \begin{itemize}
      \item Écrivons $a=1+b$ avec $b>0$
\pause      
      \item Alors $a^n=(1+b)^n=1+nb+\binom{n}{2}b^2+\cdots+\binom{n}{k}b^k+\cdots+b^n$
\pause      
      \item Donc on a : $a^n\geq 1+nb$
\pause      
      \item Or $\lim_{n\to +\infty}(1+nb)=+\infty$ car $b>0$
 \pause     
      \item D'où $\lim_{n\to +\infty} a^n=+\infty$ \qedhere
    \end{itemize}
\end{enumerate}
\end{proof}



\end{frame}

\section{Série géométrique}

\begin{frame}
\begin{proposition}[Série géométrique]
  Soit $a\neq 1$ 
  \vspace*{-3ex}
  \mybox{$\displaystyle \sum_{k=0}^na^k=\frac{1-a^{n+1}}{1-a} $}  
\end{proposition}
\pause
$$1+a+a^2+\cdots+a^n=\frac{1-a^{n+1}}{1-a}$$
\pause
\begin{proof}
  \[(1-a)\big(1+a+a^2+\cdots+a^n \big) 
  \pause =   \big(1+a+a^2+\cdots+a^n \big) - \big(a+a^2+\cdots+a^{n+1} \big)
  \pause = 1-a^{n+1} \]
\end{proof}
\end{frame}


\begin{frame}
\begin{remarque}
\begin{itemize}  \setlength{\itemsep}{3pt} 
  \item Soit $a\in ]-1,1[$ et la suite de terme $\displaystyle u_n= \sum_{k=0}^na^k
  \pause = \frac{1-a^{n+1}}{1-a}$
\pause\vspace*{-3pt}
  \item $\displaystyle\lim_{n\to +\infty} u_n= \frac{1}{1-a}$
\pause  
  \item On pourrait écrire $\displaystyle1+a+a^2+a^3 +\cdots = \frac{1}{1-a}$
\pause  
  \item Formule $\displaystyle\sum_{k=0}^na^k=\frac{1-a^{n+1}}{1-a} $ aussi valable si $a\in \Cc\setminus\{1\}$
\pause  
  \item Si $a=1$, alors $1+a+a^2+\cdots+a^n=n+1$
\end{itemize}
\end{remarque}
\pause


\[a=\tfrac{1}{2} \quad \qquad {\color{red}1} + {\color{red}\frac{1}{2}} +
{\color{red}\frac{1}{4}} + {\color{red}\frac{1}{8}} +\cdots = {\color{blue}2} \]  
\vspace*{-2ex}
\myfigure{1.5}{\tikzinput{fig_suites08bis}}    

\end{frame}

\section{Suites telles que $\left|{u_{n+1}}/{u_n}\right|<\ell<1$ }

\begin{frame}
	\begin{theoreme}
Soit  $(u_n)_{n\in \Nn}$ une suite de réels non nuls. On suppose qu'il existe $\ell$ 
%tel que $0< l <1$ et 
tel que pour tout $n$  on ait :
\[ \left | \frac{u_{n+1}}{u_n}\right |  <\ell<1
\qquad \qquad \text{alors} \qquad \lim_{n\to +\infty} u_n= 0
\]
\end{theoreme}

\pause
\medskip
Conclusion valable si $\left |\frac{u_{n+1}}{u_n}\right |  <\ell<1$ seulement à partir d'un certain rang

\medskip
\pause
\begin{proof}
  \[ \frac{u_n}{u_0}=\frac{u_1}{u_0} \times\frac{u_2}{u_1} \times\frac{u_3}{u_2} 
  \times\cdots\times \frac{u_n}{u_{n-1}}\] 
\pause
  \[\text{donc} \quad \left\lvert\frac{u_n}{u_0}  \right\rvert 
  <\ell\times \ell\times \ell \times \cdots \times \times \ell=\ell^n\] 
\pause
  \[\text{donc} \quad \lvert u_n \rvert <\lvert u_0 \rvert \ell^n \]
\pause
\hfill Comme $\ell<1$ on a $\lim_{n\to +\infty} \ell^n= 0$
ainsi $\lim_{n\to +\infty} u_n= 0$
\end{proof}

\end{frame}

\begin{frame}

\begin{corollaire}
\mybox{Si $\displaystyle\lim_{n\to +\infty}\frac{u_{n+1}}{u_n}=  0$ \quad  alors \quad $\displaystyle\lim_{n\to +\infty} u_n= 0$}
\end{corollaire}
\pause
\begin{exemple}
Soit $a\in \Rr$. Alors $\displaystyle \lim_{n\to +\infty} \frac{a^n}{n!} =0$
\end{exemple}
\pause
\begin{proof}
\begin{itemize}
  \item Si $a=0$, évident
\pause
  \item Supposons $a\neq 0$ et posons $u_n= \frac{a^n}{n!}$
\pause
  \item $\displaystyle\frac{u_{n+1}}{u_n}\pause= \frac{a^{n+1}}{(n+1)!}\cdot \frac{n!}{a^n}\pause=\frac{a}{n+1}$
\pause
  \item $\lim \frac{u_{n+1}}{u_n} = 0$ 
 \pause 
  \item $\lim u_n =0$ \qedhere
  
\end{itemize}
\end{proof}

\end{frame}

\section{Approximation des réels par des décimaux}

\begin{frame}
\begin{proposition}
\label{prop:ecdecim}
  Soit $a\in \Rr$. Posons
  \[u_n = \frac{E(10^na)}{10^n} \]
  
   \hfil \hfil Alors $u_n$ est une approximation décimale de $a$ à $10^{-n}$ près, 
  
  
  \hfil \hfil en particulier $\lim_{n\to +\infty} u_n=a$
\end{proposition}
\pause

\begin{proof}

  \[E(10^na)\leq 10^na < E(10^na)+1\]
\pause  
donc
\[ u_n \leq a < u_n+\frac{1}{10^n} \]
\pause
donc
\[ 0 \leq a -u_n< \frac{1}{10^n} \]
\pause
Conclusion :  $\lim_{n\to +\infty} u_n=a$
\end{proof}

\end{frame}


\begin{frame}
\begin{exemple} 
\hfil\hfil$\pi= 3,14159265\ldots$
\pause
\renewcommand{\arraystretch}{1.8}
 $$ \begin{array}{l}    
    u_0  = \frac{E(10^0\pi)}{10^0}  = E(\pi) = 3\\
\pause
    u_1  = \frac{E(10^1\pi)}{10^1}  = \frac{E(31,415\ldots)}{10}  = 3,1\\
\pause    
    u_2  = \frac{E(10^2\pi)}{10^2}  = \frac{E(314,15\ldots)}{100} = 3,14\\
\pause    
    u_3  = 3,141
  \end{array}$$
\end{exemple}
\pause


  \begin{enumerate}
    \item Les $u_n$ sont des décimaux, en particulier ce sont des rationnels
\pause
\medskip
    \item $\Qq$ est dense dans $\Rr$ : \\
Pour $\epsilon >0$, et $I=]a-\epsilon,a+\epsilon[$, pour $n$ assez grand, $u_n\in I\cap \Qq$
  \end{enumerate}


\end{frame}



%%%%%%%%%%%%%%%%%%%%%%%%%%%%%%%%%%%%%%%%%%%%%%%%%%%%%%%%%%%%%%%
\section{Mini-exercices}

\begin{frame}
 \begin{miniexercice}
 
 \begin{enumerate}
  \item Déterminer la limite de la suite $(u_n)_{n\in \Nn}$ de terme général $5^n-4^n$.
  
  \item Soit $v_n=1+a+a^2+\cdots + a^n$. Pour quelle valeur de $a \in \Rr$ la suite 
  $(v_n)_{n\ge 1}$ a pour limite $3$ (lorsque $n \to + \infty$) ?
     
  \item Calculer la limite de $\frac{1+2+2^2+\cdots + 2^n}{2^n}$.
  
  \item Montrer que la somme des racines $n$-ièmes de l'unité est nulle.
  
  \item Montrer que si $\sin(\frac \theta 2)\neq 0$ alors  
  $\frac{1}{2}+\cos(\theta )+\cos(2\theta )+\cdots+\cos(n\theta )
  =\frac{\sin\left( (n+ \frac{1}{2})\theta \right)}{2\sin(\frac{\theta }{2})}$ 
  (penser à $e^{\ii\theta}$).

  \item Soit $(u_n)_{n\geq 2}$ la suite de terme général 
  % $\frac{n!}{n^n}$. [[Arnaud : trop dur !]]
  $u_n= \ln(1+\frac12)\times\ln(1+\frac 13)\times\cdots\times\ln(1+\frac1n)$.
Déterminer la limite de $\frac{u_{n+1}}{u_n}$. Que peut-on en déduire ? 
  \item Déterminer la limite de $\frac{\pi^n}{1\times 3 \times 5 \times \cdots \times (2n+1)}$ 
  (où $\pi = 3,14\ldots$).
  
   \item Soit $a$ un réel. Montrer que pour tout $\epsilon>0$ il existe un couple 
   $(m,n)\in \Zz\times \Nn$ (et même une infinité) 
tel que $\left|a-\frac{m}{2^n}\right|\leq \epsilon$.
\end{enumerate}

 \end{miniexercice}
 \end{frame}



\end{document}
