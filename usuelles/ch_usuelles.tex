\documentclass[class=report,crop=false]{standalone}
\usepackage[screen]{../exo7book}

\begin{document}

%====================================================================
\chapitre{Fonctions usuelles}
%====================================================================

\insertvideo{l8ZaQUjM5h8}{partie 1. Logarithme et exponentielle}

\insertvideo{lGfC0R_FGaM}{partie 2. Fonctions circulaires inverses}

\insertvideo{tsN8Wn8j-1U}{partie 3. Fonctions hyperboliques et hyperboliques inverses}

\insertfiche{fic00014.pdf}{Fonctions circulaires et hyperboliques inverses}

\bigskip

Vous connaissez déjà des fonctions classiques : 
$\exp$, $\ln$, $\cos$, $\sin$, $\tan$.
Dans ce chapitre il s'agit d'ajouter à notre catalogue de nouvelles fonctions :
$\ch$, $\sh$, $\tanh$, $\arccos$, $\arcsin$, $\arctan$, $\Argch$, $\Argsh$, 
$\Argth$.

Ces fonctions apparaissent naturellement dans la résolution de problèmes simples,
en particulier issus de la physique.
Par exemple lorsqu'un fil est suspendu entre deux poteaux (ou un collier tenu entre deux mains)
alors la courbe dessinée est une \defi{chaînette} dont l'équation fait intervenir le cosinus hyperbolique
et un paramètre $a$ (qui dépend de la longueur du fil et de l'écartement des poteaux) :
$$y = a\ch \left( \frac x a \right)$$

\myfigure{0.6}{
\tikzinput{fig_usuelles01}
}


%%%%%%%%%%%%%%%%%%%%%%%%%%%%%%%%%%%%%%%%%%%%%%%%%%%%%%%%%%%%%%%%
\section{Logarithme et exponentielle}

%---------------------------------------------------------------
\subsection{Logarithme}
\index{logarithme}


\begin{proposition}
Il existe une unique fonction, notée $\ln : ]0,+\infty[ \to \Rr$ telle que :
$$\ln'(x) = \frac 1x \quad (\text{pour tout } x>0) \qquad \text { et } \qquad \ln(1) = 0.$$
De plus cette fonction vérifie (pour tout $a,b >0$) :
\begin{enumerate}
  \item $\ln (a \times b) = \ln a + \ln b$,
  \item $\ln(\frac 1 a) = - \ln a$,
  \item $\ln(a^n) = n \ln a$, (pour tout $n \in \Nn$)
  \item $\ln$ est une fonction continue, strictement croissante et définit une bijection de
$]0,+\infty[$ sur $\Rr$,
  \item $\lim_{x\to0}\frac{\ln(1+x)}{x} = 1$,
  \item la fonction $\ln$ est concave et $\ln x \le x-1$ (pour tout $x>0$).
\end{enumerate}
\end{proposition}

\myfigure{1}{
\tikzinput{fig_usuelles02}
}

\begin{remarque*}
$\ln x$ s'appelle le \defi{logarithme naturel} ou aussi \defi{logarithme  néperien}\index{logarithme!neperien@néperien}.
Il est caractérisé par $\ln(e)=1$.
On définit le \defi{logarithme en base $a$}\index{logarithme!en base quelconque} par
$$\log_a(x) = \frac{\ln(x)}{\ln(a)}$$
De sorte que $\log_a(a)=1$.

Pour $a=10$ on obtient le \defi{logarithme décimal}\index{logarithme!decimal@décimal} $\log_{10}$ qui vérifie
$\log_{10} (10)=1$ (et donc $\log_{10}(10^n)=n$).
Dans la pratique on utilise l'équivalence : 
\mybox{$x=10^y \iff y = \log_{10}(x)$}
En informatique intervient aussi le logarithme en base $2$ : $\log_2(2^n)=n$.
\end{remarque*}

\begin{proof}
L'existence et l'unicité viennent de la théorie de l'intégrale :
  $\ln(x) = \int_1^x \frac1t \, dt$.
Passons aux propriétés.
\begin{enumerate}
  \item Posons $f(x)=\ln(xy)-\ln(x)$ où $y>0$ est fixé.
  Alors $f'(x)=y\ln'(xy)-\ln'(x)=\frac{y}{xy}-\frac1x = 0$.
  Donc $x \mapsto f(x)$ a une dérivée nulle,
  donc est constante et vaut $f(1)=\ln(y)-\ln(1)=\ln(y)$. Donc
  $\ln(xy)-\ln(x)=\ln(y)$.

  \item D'une part $\ln (a \times \frac 1a) = \ln a + \ln \frac 1a$, mais d'autre part
  $\ln (a \times \frac 1a) = \ln(1) = 0$. Donc $\ln a + \ln \frac 1a=0$.

  \item Similaire ou récurrence.

  \item $\ln$ est dérivable donc continue, $\ln'(x)=\frac 1x >0$
  donc la fonction est strictement croissante.
  Comme $\ln(2)>\ln(1)=0$ alors $\ln(2^n)=n\ln(2) \to +\infty$ (lorsque $n\to+\infty$).
  Donc $\lim_{x\to+\infty} \ln x = +\infty$. De $\ln x= -\ln \frac{1}{x} $
   on déduit $\lim_{x\to 0} \ln x = -\infty$.
  Par le théorème sur les fonctions continues et strictement croissantes,
  $\ln : ]0,+\infty[ \to \Rr$ est une bijection.

  \item $\lim_{x\to0}\frac{\ln(1+x)}{x}$ est la dérivée de $\ln$ au point $x_0=1$,
  donc cette limite existe et vaut $\ln'(1)=1$.

  \item $\ln'(x)=\frac1x$ est décroissante, donc la fonction $\ln$ est concave.
  Posons $f(x) =  x-1 - \ln x$ ; $f'(x)= 1 -\frac1x$. Par une étude de fonction $f$ atteint son minimum en $x_0=1$.
  Donc $f(x)\ge f(1)=0$. Donc $\ln x \le x-1$.

\end{enumerate}


\end{proof}



%---------------------------------------------------------------
\subsection{Exponentielle}

\begin{definition}
La bijection réciproque de $\ln : ]0,+\infty[ \to \Rr$ s'appelle la
fonction \defi{exponentielle}\index{exponentielle}, notée $\exp : \Rr \to   ]0,+\infty[$.
\end{definition}


\myfigure{1}{
\tikzinput{fig_usuelles03}
}


Pour $x \in \Rr$ on note aussi $e^x$ pour $\exp x$.

\begin{proposition}
La fonction exponentielle vérifie les propriétés suivantes :
\begin{itemize}
  \item \myboxinline{$\exp(\ln x) = x$ pour tout $x >0$} et
  \myboxinline{$\ln(\exp x) = x$ pour tout $x \in \Rr$}

  \item $\exp(a+b) = \exp(a) \times \exp(b)$

  \item $\exp(nx) = (\exp x)^n$

  \item $\exp : \Rr \to   ]0,+ \infty[$ est une fonction continue, strictement croissante vérifiant
$\lim_{x\to-\infty} \exp x = 0$ et $\lim_{x\to +\infty} \exp = +\infty$.

  \item La fonction exponentielle est dérivable et $\exp' x = \exp x$, pour tout $x\in \Rr$. 
  Elle est convexe et $\exp x \ge 1+x$.
\end{itemize}

\end{proposition}

\begin{remarque*}
La fonction exponentielle est l'unique fonction qui vérifie
$\exp'(x) = \exp(x)$ (pour tout $x\in \Rr$)
et $\exp(1) = e$. Où $e \simeq 2,718\ldots$ est le nombre qui vérifie $\ln e = 1$.
\end{remarque*}



\begin{proof}
Ce sont les propriétés du logarithme retranscrites pour sa bijection réciproque.

Par exemple pour la dérivée : on part de l'égalité $\ln(\exp x)=x$ que l'on dérive.
Cela donne $\exp'(x) \times \ln'(\exp x) = 1$ donc $\exp'(x) \times \frac{1}{\exp x} = 1$
et ainsi $\exp'(x) = \exp x$.
\end{proof}


%---------------------------------------------------------------
\subsection{Puissance et comparaison}

Par définition, pour $a>0$ et $b\in \Rr$,
\mybox{$\displaystyle a^b = \exp\big(b \ln a\big)$}

\begin{remarque*}
\sauteligne
\begin{itemize}
  \item $\sqrt a = a^\frac12 = \exp\big( \frac12 \ln a\big)$
  \item $\sqrt[n] a = a^\frac1n = \exp\big( \frac1n \ln a\big)$ (la \defi{racine $n$-ème}\index{racine@racine $n$-ème} de $a$)
  \item On note aussi $\exp x$ par $e^x$ ce qui se justifie par le calcul :
$e^x =  \exp\big(x \ln e \big) = \exp(x)$.
  \item Les fonctions $x \mapsto a^x$ s'appellent aussi des fonctions exponentielles et
se ramènent systématiquement à la fonction exponentielle classique
par l'égalité $a^x = \exp(x \ln a)$. Il ne faut surtout pas les confondre avec les fonctions puissances
$x \mapsto x^a$.
\end{itemize}
\end{remarque*}


\begin{proposition}
Soit $x,y >0$ et $a,b\in \Rr$.
\begin{itemize}
  \item \myboxinline{$x^{a+b} = x^ax^b$}
  \item \myboxinline{$x^{-a}=\frac{1}{x^a}$}
  \item \myboxinline{$(xy)^a  = x^ay^a$}
  \item \myboxinline{$(x^a)^b = x^{ab}$}
  \item \myboxinline{$\ln(x^a) = a \ln x$}  
\end{itemize}

\end{proposition}


Comparons les fonctions $\ln x$, $\exp x$ avec $x$ :
\begin{proposition}
$$\lim_{x\to +\infty} \frac{\ln x}{x} = 0 \qquad \text{ et }  \qquad
\lim_{x\to +\infty} \frac{\exp x}{x} = +\infty.$$
\end{proposition}


\myfigure{0.9}{
\tikzinput{fig_usuelles04}
}


\begin{proof}~
\begin{enumerate}
  \item On a vu $\ln x \le x-1$ (pour tout $x>0$).
  Donc $\ln x \le x$ donc $\frac{\ln \sqrt x}{\sqrt x} \le 1$. Cela donne
  $$0 \le \frac{\ln x}{x} = \frac{\ln \left(\sqrt{x}^2\right)}{x} = 2 \frac{\ln \sqrt x}{x}
  = 2 \frac{\ln \sqrt x}{\sqrt x} \frac{1}{\sqrt x} \le \frac{2}{\sqrt x}$$
  Cette double inégalité entraîne $\lim_{x\to +\infty} \frac{\ln x}{x} = 0$.

  \item On a vu $\exp x \ge 1+x$ (pour tout $x\in\Rr$).
  Donc $\exp x \to +\infty$ (lorsque $x\to+\infty$).
  $$\frac{x}{\exp x} = \frac{\ln(\exp x)}{\exp x} = \frac{\ln u}{u}$$
  Lorsque $x\to +\infty$ alors $u=\exp x \to +\infty$ et donc par le premier point
  $\frac{\ln u}{u} \to 0$. Donc $\frac{x}{\exp x} \to 0$ et reste positive, ainsi
  $\lim_{x\to +\infty} \frac{\exp x}{x} = +\infty$.
\end{enumerate}
\end{proof}



%---------------------------------------------------------------
%\subsection{Mini-exercices}

\begin{miniexercices}
\sauteligne
\begin{enumerate}
  \item Montrer que $\ln(1+e^x) = x + \ln(1+e^{-x})$, pour tout $x \in \Rr$.

  \item \'Etudier la fonction $f(x)=\ln(x^2+1)-\ln(x)-1$. Tracer son graphe. Résoudre l'équation
  $(f(x)=0)$. Idem avec $g(x)=\frac{1+\ln x}{x}$. Idem avec $h(x)=x^x$.

  \item Expliquer comment $\log_{10}$ permet de calculer le nombre de chiffres d'un entier $n$.

  \item Montrer $\ln(1+x) \ge x-\frac{x^2}{2}$ pour $x\ge0$ (faire une étude de fonction). Idem avec
  $e^x \ge 1+x+\frac{x^2}{2}$ pour tout $x \ge 0$.

  \item Calculer la limite de la suite définie par $u_n=\left(1+\frac1n\right)^n$ lorsque $n\to +\infty$.
  Idem avec $v_n=\left(\frac1n\right)^n$ et $w_n=n^{\frac1n}$.
\end{enumerate}
\end{miniexercices}






%%%%%%%%%%%%%%%%%%%%%%%%%%%%%%%%%%%%%%%%%%%%%%%%%%%%%%%%%%%%%%%%
\section{Fonctions circulaires inverses}

%---------------------------------------------------------------
\subsection{Arccosinus}

Considérons la fonction cosinus\index{cosinus} $\cos : \Rr \to [-1,1]$, $x \mapsto \cos x$.
Pour obtenir une bijection à partir de cette fonction, il faut considérer la restriction
de cosinus à l'intervalle $[0,\pi]$. Sur cet intervalle la fonction cosinus est continue
et strictement décroissante, donc la restriction
$$\cos_| : [0,\pi] \to [-1,1]$$
est une bijection.
Sa bijection réciproque est la fonction \defi{arccosinus}\index{arccosinus} :
$$\arccos : [-1,1] \to [0,\pi]$$

\myfigure{1}{
\tikzinput{fig_usuelles05}
}
\myfigure{1}{
\tikzinput{fig_usuelles06}
}


On a donc, par définition de la bijection réciproque :
\mybox{$
\begin{array}{cl}
\displaystyle \cos\big(\arccos(x)\big) = x &  \forall x \in [-1,1]\\
\displaystyle \arccos\big(\cos(x)\big) = x &  \forall x \in [0,\pi] \\
\end{array}
$}

Autrement dit :
\mybox{$\text{Si } \quad x\in [0,\pi] \qquad \cos(x)=y \iff x = \arccos y$}

Terminons avec la dérivée de $\arccos$ :
\mybox{$\displaystyle \arccos'(x) = \frac{-1}{\sqrt{1-x^2}} \qquad \forall x \in ]-1,1[$}

\begin{proof}
On démarre de l'égalité $\cos(\arccos x) = x$ que l'on dérive :
\begin{align*}
          & \cos(\arccos x) = x \\
 \implies &  -\arccos'(x) \times  \sin(\arccos x) = 1 \\
 \implies &  \arccos'(x) = \frac{-1}{\sin(\arccos x)} \\
 \implies &  \arccos'(x) = \frac{-1}{\sqrt{1-\cos^2(\arccos x)}} \qquad (*) \\
 \implies &  \arccos'(x) = \frac{-1}{\sqrt{1-x^2}} \\
\end{align*}

Le point crucial $(*)$ se justifie ainsi :
on démarre de l'égalité $\cos^2 y + \sin^2 y = 1$,  en substituant $y= \arccos x$ on obtient $\cos^2(\arccos x)+ \sin^2(\arccos x) = 1$
donc $x^2+ \sin^2(\arccos x) = 1$. On en déduit : $\sin(\arccos x) = + \sqrt{1-x^2}$ (avec le signe $+$ car $\arccos x \in [0,\pi]$, et donc on a $\sin(\arccos x)\geq 0$).
\end{proof}


%---------------------------------------------------------------
\subsection{Arcsinus}

La restriction
$$\sin_| : [-\tfrac\pi2,+\tfrac\pi2] \to [-1,1]$$
est une bijection\index{sinus}.
Sa bijection réciproque est la fonction \defi{arcsinus}\index{arcsinus} :
$$\arcsin : [-1,1] \to [-\tfrac\pi2,+\tfrac\pi2]$$

\myfigure{1}{
\tikzinput{fig_usuelles07}
\tikzinput{fig_usuelles08}
}


\mybox{$
\begin{array}{cl}
\displaystyle \sin\big(\arcsin(x)\big) = x &  \forall x \in [-1,1]\\
\displaystyle \arcsin\big(\sin(x)\big) = x &  \forall x \in [-\frac\pi2,+\frac\pi2] \\
\end{array}
$}

\mybox{$\text{Si } \quad  x\in [-\frac\pi2,+\frac\pi2] \qquad \sin(x)=y \iff x = \arcsin y$}


\mybox{$\displaystyle \arcsin'(x) = \frac{1}{\sqrt{1-x^2}} \qquad \forall x \in ]-1,1[$}


%---------------------------------------------------------------
\subsection{Arctangente}

La restriction
$$\tan_| : ]-\tfrac\pi2,+\tfrac\pi2[ \to \Rr$$
est une bijection\index{tangente}.
Sa bijection réciproque est la fonction \defi{arctangente}\index{arctangente} :
$$\arctan : \Rr \to ]-\tfrac\pi2,+\tfrac\pi2[$$


\myfigure{0.6}{
\tikzinput{fig_usuelles09} \
\tikzinput{fig_usuelles10}
}


\mybox{$
\begin{array}{cl}
\displaystyle \tan\big(\arctan(x)\big) = x &  \forall x \in \Rr\\
\displaystyle \arctan\big(\tan(x)\big) = x &  \forall x \in ]-\frac\pi2,+\frac\pi2[ \\
\end{array}
$}

\mybox{$\text{Si } \quad x\in ]-\frac\pi2,+\frac\pi2[ \qquad \tan(x)=y \iff x = \arctan y$}


\mybox{$\displaystyle \arctan'(x) = \frac{1}{1+x^2} \qquad \forall x \in \Rr$}


%---------------------------------------------------------------
%\subsection{Mini-exercices}

\begin{miniexercices}
\sauteligne
\begin{enumerate}
  \item Calculer les valeurs de $\arccos$ et $\arcsin$ en
  $0$, $1$, $\frac12$, $\frac{\sqrt2}{2}$, $\frac{\sqrt3}{2}$. Idem pour
  $\arctan$ en $0$, $1$, $\sqrt3$ et $\frac{1}{\sqrt3}$.

  \item Calculer $\arccos(\cos \frac{7\pi}{3})$. Idem avec $\arcsin(\sin \frac{7\pi}{3})$
  et $\arctan(\tan \frac{7\pi}{3})$ (attention aux intervalles !)

  \item Calculer $\cos(\arctan x)$, $\cos(\arcsin x)$, $\tan(\arcsin x)$.

  \item Calculer la dérivée de $f(x)=\arctan\left(\frac{x}{\sqrt{1-x^2}}\right)$.
  En déduire que $f(x)= \arcsin x$, pour tout $x\in]-1,1[$.

  \item Montrer que $\arccos x + \arcsin x = \frac\pi2$, pour tout $x \in[-1,1]$.
\end{enumerate}
\end{miniexercices}

%%%%%%%%%%%%%%%%%%%%%%%%%%%%%%%%%%%%%%%%%%%%%%%%%%%%%%%%%%%%%%%%
\section{Fonctions hyperboliques et hyperboliques inverses}



%---------------------------------------------------------------
\subsection{Cosinus hyperbolique et son inverse}

Pour $x\in \Rr$, le \defi{cosinus hyperbolique}\index{cosinus hyperbolique} est :
\mybox{$\displaystyle \ch x = \frac{e^x+e^{-x}}{2}$}

La restriction $\ch_| : [0,+\infty[ \to [1,+\infty[$ est une bijection.
Sa bijection réciproque est $\Argch : [1,+\infty[ \to [0,+\infty[$.


\myfigure{0.9}{
\tikzinput{fig_usuelles11}
\tikzinput{fig_usuelles12}}



%---------------------------------------------------------------
\subsection{Sinus hyperbolique et son inverse}

Pour $x\in \Rr$, le \defi{sinus hyperbolique}\index{sinus hyperbolique} est :
\mybox{$\displaystyle \sh x = \frac{e^x-e^{-x}}{2}$}

$\sh : \Rr \to \Rr$ est une fonction continue, dérivable, strictement
croissante vérifiant $\lim_{x\to -\infty} \sh x = -\infty$
et $\lim_{x\to +\infty} \sh x = +\infty$, c'est donc une bijection.
Sa bijection réciproque est $\Argsh : \Rr \to \Rr$.

\begin{proposition}
\sauteligne
\begin{itemize}
  \item $\ch^2 x - \sh^2 x =1$
  \item $\ch' x = \sh x$, $\sh'x = \ch x$
  \item $\Argsh : \Rr \to \Rr$ est strictement croissante et continue.
  \item $\Argsh$ est dérivable  et $\Argsh'x=\frac{1}{\sqrt{x^2+1}}$.
  \item $\Argsh x = \ln\big(x+ \sqrt{x^2+1}\big)$
\end{itemize}
\end{proposition}




\begin{proof}~
\begin{itemize}
  \item $\ch^2 x- \sh^2 x = \frac14 \big[ (e^x+e^{-x})^2 - (e^x-e^{-x})^2 \big] =
 \frac14 \big[ (e^{2x}+2+e^{-2x}) - (e^{2x}-2+e^{-2x})  \big] = 1$.

  \item $\frac{d}{dx}(\ch x) = \frac{d}{dx} \frac{e^x+e^{-x}}{2} = \frac{e^x-e^{-x}}{2}  = \sh x$.
Idem pour la dérivée de $\sh x$.

  \item Car c'est la réciproque de $\sh$.

  \item Comme la fonction $x \mapsto \sh'x$  ne s'annule pas sur $\Rr$
alors la fonction $\Argsh$ est dérivable sur $\Rr$.
On calcule la dérivée par dérivation de l'égalité $\sh(\Argsh x) = x$ :
$$\Argsh' x = \frac{1}{\ch(\Argsh x)} = \frac{1}{\sqrt{\sh^2(\Argsh x)+1}}
= \frac{1}{\sqrt{x^2+1}}$$

  \item Notons $f(x)=\ln\big(x+ \sqrt{x^2+1}\big)$ alors
   $$f'(x) = \frac{1+\frac{x}{\sqrt{x^2+1}}}{x+ \sqrt{x^2+1}} = \frac{1}{\sqrt{x^2+1}} = \Argsh' x$$
  Comme de plus $f(0)=\ln(1)=0$ et $\Argsh 0 = 0$ (car $\sh 0 = 0$), on en déduit que pour tout
  $x\in \Rr$, $f(x) = \Argsh x$.
 \end{itemize}


\end{proof}



%---------------------------------------------------------------
\subsection{Tangente hyperbolique et son inverse}


Par définition la \defi{tangente hyperbolique}\index{tangente hyperbolique} est :
\mybox{$\displaystyle \tanh x = \frac{\sh x}{\ch x}$}

La fonction $\tanh : \Rr \to ]-1,1[$ est une bijection, on note
$\Argth : ]-1,1[\to\Rr$ sa bijection réciproque.


\myfigure{0.9}{
\tikzinput{fig_usuelles13}
\tikzinput{fig_usuelles14}
}


%---------------------------------------------------------------
\subsection{Trigonométrie hyperbolique}

$$\ch^2 x - \sh^2 x = 1$$
\begin{align*}
\ch(a+b)&=\ch a\cdot\ch b + \sh a\cdot\sh b\\
\ch(2a) &=\ch^2a+\sh^2a = 2\,\ch^2a-1 = 1+2\,\sh^2a\\  \\
\sh(a+b)&=\sh a\cdot\ch b  +  \sh b\cdot\ch a\\
\sh(2a) &= 2\,\sh a\cdot\ch a\\ \\
\tanh (a+b)&=\frac{\tanh a + \tanh b}{1+\tanh a\cdot\tanh b}\\
\end{align*}
\begin{align*}
\ch'x&= \sh x\\
\sh'x&=\ch x\\
\tanh' x &= 1-\tanh^2x=\frac{1}{\ch^2x}\\
\end{align*}
\begin{align*}
\Argch'x&=\frac{1}{\sqrt{x^2-1}} \quad (x>1)\\
\Argsh'x&=\frac{1}{\sqrt{x^2+1}} \\
\Argth'x&=\frac{1}{1-x^2} \quad (|x|<1)\\
\end{align*}
\begin{align*}
\Argch x&= \ln\big(x+ \sqrt{x^2-1}\big) \quad (x\ge1)\\
\Argsh x&= \ln\big(x+ \sqrt{x^2+1}\big)  \quad (x \in \Rr) \\
\Argth x&= \frac12\ln\left(\frac{1+x}{1-x}\right) \quad (-1<x<1) \\
\end{align*}


%---------------------------------------------------------------
%\subsection{Mini-exercices}

\begin{miniexercices}
\sauteligne
\begin{enumerate}
  \item Dessiner les courbes paramétrées $t \mapsto (\cos t, \sin t)$ et $t \mapsto (\ch t, \sh t)$.
Pourquoi $\cos$ et $\sin$ s'appellent des fonctions trigonométriques \emph{circulaires} alors que
$\ch$ et $\sh$ sont des fonctions trigonométriques \emph{hyperboliques} ?
  \item Prouver par le calcul la formule $\ch(a+b)=\ldots$ En utilisant que $\cos x = \frac{e^{\ii x}+e^{-\ii x}}{2}$
retrouver la formule pour $\cos(a+b)$.
  \item Résoudre l'équation $\sh x = 3$.
  \item Montrer que $\frac{\sh(2x)}{1+\ch(2x)} = \tanh x$.
  \item Calculer les dérivées des fonctions définies par :
  $\tanh(1+x^2)$, $\ln(\ch x)$, $\Argch(\exp x)$, $\Argth(\cos x)$.
\end{enumerate}
\end{miniexercices}


\bigskip
\bigskip

\auteurs{
Arnaud Bodin, 
Niels Borne, 
Laura Desideri
}

\finchapitre
\end{document}
