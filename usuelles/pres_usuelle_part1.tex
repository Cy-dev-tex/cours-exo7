
%%%%%%%%%%%%%%%%%% PREAMBULE %%%%%%%%%%%%%%%%%%

\documentclass[aspectratio=169,utf8]{beamer}
%\documentclass[aspectratio=169,handout]{beamer}

\usetheme{Boadilla}
%\usecolortheme{seahorse}
\usecolortheme[RGB={245,66,24}]{structure}
\useoutertheme{infolines}

% packages
\usepackage{amsfonts,amsmath,amssymb,amsthm}
\usepackage[utf8]{inputenc}
\usepackage[T1]{fontenc}
\usepackage{lmodern}

\usepackage[francais]{babel}
\usepackage{fancybox}
\usepackage{graphicx}

\usepackage{float}
\usepackage{xfrac}

%\usepackage[usenames, x11names]{xcolor}
\usepackage{tikz}
\usepackage{pgfplots}
\usepackage{datetime}



%-----  Package unités -----
\usepackage{siunitx}
\sisetup{locale = FR,detect-all,per-mode = symbol}

%\usepackage{mathptmx}
%\usepackage{fouriernc}
%\usepackage{newcent}
%\usepackage[mathcal,mathbf]{euler}

%\usepackage{palatino}
%\usepackage{newcent}
% \usepackage[mathcal,mathbf]{euler}



% \usepackage{hyperref}
% \hypersetup{colorlinks=true, linkcolor=blue, urlcolor=blue,
% pdftitle={Exo7 - Exercices de mathématiques}, pdfauthor={Exo7}}


%section
% \usepackage{sectsty}
% \allsectionsfont{\bf}
%\sectionfont{\color{Tomato3}\upshape\selectfont}
%\subsectionfont{\color{Tomato4}\upshape\selectfont}

%----- Ensembles : entiers, reels, complexes -----
\newcommand{\Nn}{\mathbb{N}} \newcommand{\N}{\mathbb{N}}
\newcommand{\Zz}{\mathbb{Z}} \newcommand{\Z}{\mathbb{Z}}
\newcommand{\Qq}{\mathbb{Q}} \newcommand{\Q}{\mathbb{Q}}
\newcommand{\Rr}{\mathbb{R}} \newcommand{\R}{\mathbb{R}}
\newcommand{\Cc}{\mathbb{C}} 
\newcommand{\Kk}{\mathbb{K}} \newcommand{\K}{\mathbb{K}}

%----- Modifications de symboles -----
\renewcommand{\epsilon}{\varepsilon}
\renewcommand{\Re}{\mathop{\text{Re}}\nolimits}
\renewcommand{\Im}{\mathop{\text{Im}}\nolimits}
%\newcommand{\llbracket}{\left[\kern-0.15em\left[}
%\newcommand{\rrbracket}{\right]\kern-0.15em\right]}

\renewcommand{\ge}{\geqslant}
\renewcommand{\geq}{\geqslant}
\renewcommand{\le}{\leqslant}
\renewcommand{\leq}{\leqslant}
\renewcommand{\epsilon}{\varepsilon}

%----- Fonctions usuelles -----
\newcommand{\ch}{\mathop{\text{ch}}\nolimits}
\newcommand{\sh}{\mathop{\text{sh}}\nolimits}
\renewcommand{\tanh}{\mathop{\text{th}}\nolimits}
\newcommand{\cotan}{\mathop{\text{cotan}}\nolimits}
\newcommand{\Arcsin}{\mathop{\text{arcsin}}\nolimits}
\newcommand{\Arccos}{\mathop{\text{arccos}}\nolimits}
\newcommand{\Arctan}{\mathop{\text{arctan}}\nolimits}
\newcommand{\Argsh}{\mathop{\text{argsh}}\nolimits}
\newcommand{\Argch}{\mathop{\text{argch}}\nolimits}
\newcommand{\Argth}{\mathop{\text{argth}}\nolimits}
\newcommand{\pgcd}{\mathop{\text{pgcd}}\nolimits} 


%----- Commandes divers ------
\newcommand{\ii}{\mathrm{i}}
\newcommand{\dd}{\text{d}}
\newcommand{\id}{\mathop{\text{id}}\nolimits}
\newcommand{\Ker}{\mathop{\text{Ker}}\nolimits}
\newcommand{\Card}{\mathop{\text{Card}}\nolimits}
\newcommand{\Vect}{\mathop{\text{Vect}}\nolimits}
\newcommand{\Mat}{\mathop{\text{Mat}}\nolimits}
\newcommand{\rg}{\mathop{\text{rg}}\nolimits}
\newcommand{\tr}{\mathop{\text{tr}}\nolimits}


%----- Structure des exercices ------

\newtheoremstyle{styleexo}% name
{2ex}% Space above
{3ex}% Space below
{}% Body font
{}% Indent amount 1
{\bfseries} % Theorem head font
{}% Punctuation after theorem head
{\newline}% Space after theorem head 2
{}% Theorem head spec (can be left empty, meaning ‘normal’)

%\theoremstyle{styleexo}
\newtheorem{exo}{Exercice}
\newtheorem{ind}{Indications}
\newtheorem{cor}{Correction}


\newcommand{\exercice}[1]{} \newcommand{\finexercice}{}
%\newcommand{\exercice}[1]{{\tiny\texttt{#1}}\vspace{-2ex}} % pour afficher le numero absolu, l'auteur...
\newcommand{\enonce}{\begin{exo}} \newcommand{\finenonce}{\end{exo}}
\newcommand{\indication}{\begin{ind}} \newcommand{\finindication}{\end{ind}}
\newcommand{\correction}{\begin{cor}} \newcommand{\fincorrection}{\end{cor}}

\newcommand{\noindication}{\stepcounter{ind}}
\newcommand{\nocorrection}{\stepcounter{cor}}

\newcommand{\fiche}[1]{} \newcommand{\finfiche}{}
\newcommand{\titre}[1]{\centerline{\large \bf #1}}
\newcommand{\addcommand}[1]{}
\newcommand{\video}[1]{}

% Marge
\newcommand{\mymargin}[1]{\marginpar{{\small #1}}}

\def\noqed{\renewcommand{\qedsymbol}{}}


%----- Presentation ------
\setlength{\parindent}{0cm}

%\newcommand{\ExoSept}{\href{http://exo7.emath.fr}{\textbf{\textsf{Exo7}}}}

\definecolor{myred}{rgb}{0.93,0.26,0}
\definecolor{myorange}{rgb}{0.97,0.58,0}
\definecolor{myyellow}{rgb}{1,0.86,0}

\newcommand{\LogoExoSept}[1]{  % input : echelle
{\usefont{U}{cmss}{bx}{n}
\begin{tikzpicture}[scale=0.1*#1,transform shape]
  \fill[color=myorange] (0,0)--(4,0)--(4,-4)--(0,-4)--cycle;
  \fill[color=myred] (0,0)--(0,3)--(-3,3)--(-3,0)--cycle;
  \fill[color=myyellow] (4,0)--(7,4)--(3,7)--(0,3)--cycle;
  \node[scale=5] at (3.5,3.5) {Exo7};
\end{tikzpicture}}
}


\newcommand{\debutmontitre}{
  \author{} \date{} 
  \thispagestyle{empty}
  \hspace*{-10ex}
  \begin{minipage}{\textwidth}
    \titlepage  
  \vspace*{-2.5cm}
  \begin{center}
    \LogoExoSept{2.5}
  \end{center}
  \end{minipage}

  \vspace*{-0cm}
  
  % Astuce pour que le background ne soit pas discrétisé lors de la conversion pdf -> png
\begin{tikzpicture}
        \fill[opacity=0,green!60!black] (0,0)--++(0,0)--++(0,0)--++(0,0)--cycle; 
\end{tikzpicture}

% toc S'affiche trop tot :
% \tableofcontents[hideallsubsections, pausesections]
}

\newcommand{\finmontitre}{
  \end{frame}
  \setcounter{framenumber}{0}
} % ne marche pas pour une raison obscure

%----- Commandes supplementaires ------

% \usepackage[landscape]{geometry}
% \geometry{top=1cm, bottom=3cm, left=2cm, right=10cm, marginparsep=1cm
% }
% \usepackage[a4paper]{geometry}
% \geometry{top=2cm, bottom=2cm, left=2cm, right=2cm, marginparsep=1cm
% }

%\usepackage{standalone}


% New command Arnaud -- november 2011
\setbeamersize{text margin left=24ex}
% si vous modifier cette valeur il faut aussi
% modifier le decalage du titre pour compenser
% (ex : ici =+10ex, titre =-5ex

\theoremstyle{definition}
%\newtheorem{proposition}{Proposition}
%\newtheorem{exemple}{Exemple}
%\newtheorem{theoreme}{Théorème}
%\newtheorem{lemme}{Lemme}
%\newtheorem{corollaire}{Corollaire}
%\newtheorem*{remarque*}{Remarque}
%\newtheorem*{miniexercice}{Mini-exercices}
%\newtheorem{definition}{Définition}

% Commande tikz
\usetikzlibrary{calc}
\usetikzlibrary{patterns,arrows}
\usetikzlibrary{matrix}
\usetikzlibrary{fadings} 

%definition d'un terme
\newcommand{\defi}[1]{{\color{myorange}\textbf{\emph{#1}}}}
\newcommand{\evidence}[1]{{\color{blue}\textbf{\emph{#1}}}}
\newcommand{\assertion}[1]{\emph{\og#1\fg}}  % pour chapitre logique
%\renewcommand{\contentsname}{Sommaire}
\renewcommand{\contentsname}{}
\setcounter{tocdepth}{2}



%------ Figures ------

\def\myscale{1} % par défaut 
\newcommand{\myfigure}[2]{  % entrée : echelle, fichier figure
\def\myscale{#1}
\begin{center}
\footnotesize
{#2}
\end{center}}


%------ Encadrement ------

\usepackage{fancybox}


\newcommand{\mybox}[1]{
\setlength{\fboxsep}{7pt}
\begin{center}
\shadowbox{#1}
\end{center}}

\newcommand{\myboxinline}[1]{
\setlength{\fboxsep}{5pt}
\raisebox{-10pt}{
\shadowbox{#1}
}
}

%--------------- Commande beamer---------------
\newcommand{\beameronly}[1]{#1} % permet de mettre des pause dans beamer pas dans poly


\setbeamertemplate{navigation symbols}{}
\setbeamertemplate{footline}  % tiré du fichier beamerouterinfolines.sty
{
  \leavevmode%
  \hbox{%
  \begin{beamercolorbox}[wd=.333333\paperwidth,ht=2.25ex,dp=1ex,center]{author in head/foot}%
    % \usebeamerfont{author in head/foot}\insertshortauthor%~~(\insertshortinstitute)
    \usebeamerfont{section in head/foot}{\bf\insertshorttitle}
  \end{beamercolorbox}%
  \begin{beamercolorbox}[wd=.333333\paperwidth,ht=2.25ex,dp=1ex,center]{title in head/foot}%
    \usebeamerfont{section in head/foot}{\bf\insertsectionhead}
  \end{beamercolorbox}%
  \begin{beamercolorbox}[wd=.333333\paperwidth,ht=2.25ex,dp=1ex,right]{date in head/foot}%
    % \usebeamerfont{date in head/foot}\insertshortdate{}\hspace*{2em}
    \insertframenumber{} / \inserttotalframenumber\hspace*{2ex} 
  \end{beamercolorbox}}%
  \vskip0pt%
}


\definecolor{mygrey}{rgb}{0.5,0.5,0.5}
\setlength{\parindent}{0cm}
%\DeclareTextFontCommand{\helvetica}{\fontfamily{phv}\selectfont}

% background beamer
\definecolor{couleurhaut}{rgb}{0.85,0.9,1}  % creme
\definecolor{couleurmilieu}{rgb}{1,1,1}  % vert pale
\definecolor{couleurbas}{rgb}{0.85,0.9,1}  % blanc
\setbeamertemplate{background canvas}[vertical shading]%
[top=couleurhaut,middle=couleurmilieu,midpoint=0.4,bottom=couleurbas] 
%[top=fondtitre!05,bottom=fondtitre!60]



\makeatletter
\setbeamertemplate{theorem begin}
{%
  \begin{\inserttheoremblockenv}
  {%
    \inserttheoremheadfont
    \inserttheoremname
    \inserttheoremnumber
    \ifx\inserttheoremaddition\@empty\else\ (\inserttheoremaddition)\fi%
    \inserttheorempunctuation
  }%
}
\setbeamertemplate{theorem end}{\end{\inserttheoremblockenv}}

\newenvironment{theoreme}[1][]{%
   \setbeamercolor{block title}{fg=structure,bg=structure!40}
   \setbeamercolor{block body}{fg=black,bg=structure!10}
   \begin{block}{{\bf Th\'eor\`eme }#1}
}{%
   \end{block}%
}


\newenvironment{proposition}[1][]{%
   \setbeamercolor{block title}{fg=structure,bg=structure!40}
   \setbeamercolor{block body}{fg=black,bg=structure!10}
   \begin{block}{{\bf Proposition }#1}
}{%
   \end{block}%
}

\newenvironment{corollaire}[1][]{%
   \setbeamercolor{block title}{fg=structure,bg=structure!40}
   \setbeamercolor{block body}{fg=black,bg=structure!10}
   \begin{block}{{\bf Corollaire }#1}
}{%
   \end{block}%
}

\newenvironment{mydefinition}[1][]{%
   \setbeamercolor{block title}{fg=structure,bg=structure!40}
   \setbeamercolor{block body}{fg=black,bg=structure!10}
   \begin{block}{{\bf Définition} #1}
}{%
   \end{block}%
}

\newenvironment{lemme}[0]{%
   \setbeamercolor{block title}{fg=structure,bg=structure!40}
   \setbeamercolor{block body}{fg=black,bg=structure!10}
   \begin{block}{\bf Lemme}
}{%
   \end{block}%
}

\newenvironment{remarque}[1][]{%
   \setbeamercolor{block title}{fg=black,bg=structure!20}
   \setbeamercolor{block body}{fg=black,bg=structure!5}
   \begin{block}{Remarque #1}
}{%
   \end{block}%
}


\newenvironment{exemple}[1][]{%
   \setbeamercolor{block title}{fg=black,bg=structure!20}
   \setbeamercolor{block body}{fg=black,bg=structure!5}
   \begin{block}{{\bf Exemple }#1}
}{%
   \end{block}%
}


\newenvironment{miniexercice}[0]{%
   \setbeamercolor{block title}{fg=structure,bg=structure!20}
   \setbeamercolor{block body}{fg=black,bg=structure!5}
   \begin{block}{Mini-exercices}
}{%
   \end{block}%
}


\newenvironment{tp}[0]{%
   \setbeamercolor{block title}{fg=structure,bg=structure!40}
   \setbeamercolor{block body}{fg=black,bg=structure!10}
   \begin{block}{\bf Travaux pratiques}
}{%
   \end{block}%
}
\newenvironment{exercicecours}[1][]{%
   \setbeamercolor{block title}{fg=structure,bg=structure!40}
   \setbeamercolor{block body}{fg=black,bg=structure!10}
   \begin{block}{{\bf Exercice }#1}
}{%
   \end{block}%
}
\newenvironment{algo}[1][]{%
   \setbeamercolor{block title}{fg=structure,bg=structure!40}
   \setbeamercolor{block body}{fg=black,bg=structure!10}
   \begin{block}{{\bf Algorithme}\hfill{\color{gray}\texttt{#1}}}
}{%
   \end{block}%
}


\setbeamertemplate{proof begin}{
   \setbeamercolor{block title}{fg=black,bg=structure!20}
   \setbeamercolor{block body}{fg=black,bg=structure!5}
   \begin{block}{{\footnotesize Démonstration}}
   \footnotesize
   \smallskip}
\setbeamertemplate{proof end}{%
   \end{block}}
\setbeamertemplate{qed symbol}{\openbox}


\makeatother
\usecolortheme[RGB={153,51,0}]{structure}

%%%%%%%%%%%%%%%%%%%%%%%%%%%%%%%%%%%%%%%%%%%%%%%%%%%%%%%%%%%%%
%%%%%%%%%%%%%%%%%%%%%%%%%%%%%%%%%%%%%%%%%%%%%%%%%%%%%%%%%%%%%

\begin{document}

\title{{\bf Fonctions usuelles}}
\subtitle{Logarithme et exponentielle}




\begin{frame}
  
  \debutmontitre

  \pause

{\footnotesize
\hfill
\setbeamercovered{transparent=50}
\begin{minipage}{0.6\textwidth}
  \begin{itemize}
    \item<3-> Logarithme
    \item<4-> Exponentielle
    \item<5-> Puissance et comparaison
%    \item<6-> 
  \end{itemize}
\end{minipage}
}

\end{frame}
\setcounter{framenumber}{0}



%%%%%%%%%%%%%%%%%%%%%%%%%%%%%%%%%%%%%%%%%%%%%%%%%%%%%%%%%%%%%%%
\section{Logarithme}

\begin{frame}


\begin{proposition}
Il existe une unique fonction, le \defi{logarithme} $\ln : ]0,+\infty[ \to \Rr$ tel que :
$$\ln'(x) = \frac 1x \quad (\text{pour tout } x>0) \qquad \text { et } \qquad \ln(1) = 0$$
\end{proposition}

\myfigure{0.9}{
\tikzinput{fig_usuelles02}
}
\end{frame}


\begin{frame}
\begin{proposition}
\begin{enumerate}
  \item $\ln (a \times b) = \ln a + \ln b$
\uncover<2-> {  \item  $\ln(\frac 1 a) = - \ln a$}
\uncover<3-> {  \item $\ln(a^n) = n \ln a$}
\uncover<4-> {   \item $\ln$ est continue, strictement croissante et définit une bijection de
$]0,+\infty[$ sur $\Rr$ }
\uncover<5-> {  \item $\lim_{x\to0}\frac{\ln(1+x)}{x} = 1$}
\uncover<6-> {  \item la fonction $\ln$ est concave et $\ln x \le x-1$ (pour tout $x>0$)}
\end{enumerate}
\end{proposition} 

\myfigure{0.5}{
\tikzinput{fig_usuelles02}
}

\end{frame}


\begin{frame}

\begin{itemize}
  \item $\ln x$ s'appelle le \defi{logarithme naturel} ou \defi{logarithme  néperien} \\
\pause
  Il est caractérisé par $\ln(e)=1$
\pause  
  \item On définit le \defi{logarithme en base $a$} par 
\mybox{$\displaystyle \log_a(x) = \frac{\ln(x)}{\ln(a)}$}
\pause  
De sorte que $\log_a(a)=1$
\pause  
  \item Pour $a=10$, le \defi{logarithme décimal} $\log_{10}$ 
$$\log_{10} (10)=1 \quad \text{ et } \quad  \log_{10}(10^n)=n$$
\pause  
\vspace*{-3ex}
\mybox{$x=10^y \iff y = \log_{10}(x)$}
\pause    
  \item Le logarithme en base $2$ : $\log_2(2^n)=n$
  
\end{itemize}

\end{frame}

\begin{frame}
\begin{proof}  
\begin{enumerate}
\setcounter{enumi}{-1}
  \item 

L'existence et l'unicité viennent de la théorie de l'intégrale :  
  $\ln(x) = \int_1^x \frac1t \, dt$
  
\pause


  \item Montrons $\ln(xy)=\ln(x)+\ln(y)$
  \begin{itemize}
  
\pause  
  
    \item Posons $f(x)=\ln(xy)-\ln(x)$ où $y>0$ est fixé
  
\pause    
    
    \item Alors $f'(x)=y\ln'(xy)-\ln'(x)=\frac{y}{xy}-\frac1x = 0$
   
\pause   
    \item $x \mapsto f(x)$ a une dérivée nulle, 
  donc est constante 
  
\pause  
  
    \item Ainsi $f(x)=f(1) =\ln(y)-\ln(1)=\ln(y)$
  
\pause    
    
    \item Donc $\ln(xy)-\ln(x)=\ln(y)$
  \end{itemize}
\pause  
  
\setcounter{enumi}{5}  

 
  \item Montrons $\ln x \le x-1$
  
\pause  
    \begin{itemize}
      \item $f(x) =  x-1 - \ln x$
  
\pause      
      \item $f'(x)= 1 -\frac1x$
  
\pause      
      \item \'Etude de fonction : $f$ atteint son minimum en $x_0=1$
   
\pause     
      \item Donc $f(x) \ge f(1)=0$ et ainsi $\ln x \le x-1$
    \end{itemize}
\qedhere
  
\end{enumerate}

  
\end{proof}  
\end{frame}



%%%%%%%%%%%%%%%%%%%%%%%%%%%%%%%%%%%%%%%%%%%%%%%%%%%%%%%%%%%%%%%
\section{Exponentielle}

\begin{frame}

\begin{mydefinition}
La bijection réciproque de $\ln : ]0,+\infty[ \to \Rr$ est l'\defi{exponentielle}  
$$\exp : \Rr \to   ]0,+\infty[$$
\end{mydefinition}

On note aussi $e^x$ pour $\exp x$

\myfigure{0.9}{
\tikzinput{fig_usuelles03}
}
\end{frame}


\begin{frame}

\begin{proposition}

\begin{itemize}
  \item \myboxinline{$\exp(\ln x) = x$ \quad $\forall x >0$} 
\uncover<2->{  et \myboxinline{$\ln(\exp x) = x$ \quad $\forall x \in \Rr$} }

\uncover<3->{   \item $\exp(a+b) = \exp(a) \times \exp(b)$}

\uncover<4->{   \item $(\exp x)^n = \exp(nx)$}

\uncover<5->{   \item $\exp : \Rr \to   ]0,+ \infty[$ est continue, strictement croissante}

\uncover<6->{   \item L'exponentielle est dérivable \quad et \quad $\exp' x = \exp x$}

\uncover<7->{   \item Elle est convexe \quad et \quad $\exp x \ge 1+x$ }
\end{itemize}
\end{proposition}
\vspace*{-1ex}
\myfigure{0.5}{
\tikzinput{fig_usuelles03}
}

\end{frame}



\begin{frame}

\begin{proof}
$$\exp'(x) = \exp x$$\pause
$$\ln(\exp x)=x 
\pause \quad \implies\quad  \exp'(x) \times \ln'(\exp x) = 1 
\pause \quad \implies\quad  \exp'(x) \times \frac{1}{\exp x} = 1$$
\pause$$\implies \exp'(x) = \exp x$$
\end{proof}

\medskip

\begin{minipage}{0.6\textwidth}
\begin{itemize}
\uncover<6->{    \item La fonction exponentielle est l'unique fonction qui vérifie 
$\exp'(x) = \exp(x)$ (pour tout $x\in \Rr$) et $\exp(1) = e$ }
  
\uncover<7->{   \item $e \simeq 2,718\ldots$ est le nombre qui vérifie $\ln e = 1$ }
\end{itemize}  
\end{minipage}
\begin{minipage}{0.29\textwidth}
\uncover<6->{  \myfigure{0.5}{
\tikzinput{fig_usuelles03}
}  }
\end{minipage}



\end{frame}



%%%%%%%%%%%%%%%%%%%%%%%%%%%%%%%%%%%%%%%%%%%%%%%%%%%%%%%%%%%%%%%
\section{Puissance et comparaison}

\begin{frame}


Par définition, pour $a>0$ et $b\in \Rr$,
\mybox{$\displaystyle a^b = \exp\big(b \ln a\big)$}

\pause

\begin{remarque}
\begin{itemize}\setlength{\itemsep}{7pt} 
  \item $\sqrt a = a^\frac12 = \exp\big( \frac12 \ln a\big)$
\pause
  \item $\sqrt[n] a = a^\frac1n = \exp\big( \frac1n \ln a\big)$ : la \defi{racine $n$-ième} de $a$
\pause  
  \item Notation $e^x = \exp x$ se justifie par : $e^x =  \exp\big(x \ln e \big) = \exp(x)$
\pause
  \item 
  \begin{itemize}
    \item Les fonctions $x \mapsto a^x$ s'appellent aussi des fonctions exponentielles
\pause    
    \item On se ramène à l'exponentielle par $a^x = \exp(x \ln a)$
\pause    
    \item Ne pas les confondre avec les fonctions puissances $x \mapsto x^a$
  \end{itemize}
\end{itemize}
\end{remarque}

\end{frame}


\begin{frame}

\begin{proposition}
$$\lim_{x\to +\infty} \frac{\ln x}{x} = 0 
\pause \qquad \text{ et }  \qquad 
\lim_{x\to +\infty} \frac{\exp x}{x} = +\infty$$
\end{proposition}

\pause

\myfigure{0.65}{
\tikzinput{fig_usuelles04}
}
\end{frame}


\begin{frame}


$$\lim_{x\to +\infty} \frac{\ln x}{x} = 0$$
\pause
\begin{proof}
\begin{itemize}
  \item On a vu $\ln x \le x-1$ donc $\ln x \le x$
\pause  
  \item Donc $\frac{\ln \sqrt x}{\sqrt x} \le 1$
\pause  
  \item 
  $$0 \le \frac{\ln x}{x} = \frac{\ln \left(\sqrt{x}^2\right)}{x} = 2 \frac{\ln \sqrt x}{x} 
  = 2 \frac{\ln \sqrt x}{\sqrt x} \frac{1}{\sqrt x} \le \frac{2}{\sqrt x}$$
\pause  
  \item Ainsi $\displaystyle \lim_{x\to +\infty} \frac{\ln x}{x} = 0$
\end{itemize}
\end{proof}
\end{frame}


%%%%%%%%%%%%%%%%%%%%%%%%%%%%%%%%%%%%%%%%%%%%%%%%%%%%%%%%%%%%%%%
\section*{Mini-exercices}


\begin{frame}
\begin{miniexercice}
\begin{enumerate}
  \item Montrer que $\ln(1+e^x) = x + \ln(1+e^{-x})$, pour tout $x \in \Rr$.
  
  \item \'Etudier la fonction $f(x)=\ln(x^2+1)-\ln(x)-1$. Tracer son graphe. Résoudre l'équation 
  $(f(x)=0)$. Idem avec $g(x)=\frac{1+\ln x}{x}$. Idem avec $h(x)=x^x$.
  
  \item Expliquer comment $\log_{10}$ permet de calculer le nombre de chiffres d'un entier $n$.
  
  \item Montrer $\ln(1+x) \ge x-\frac{x^2}{2}$ pour $x\ge0$ (faire une étude de fonction). Idem avec
  $e^x \ge 1+x+\frac{x^2}{2}$ pour tout $x \ge 0$.
  
  \item Calculer la limite de la suite définie par $u_n=\left(1+\frac1n\right)^n$ lorsque $n\to +\infty$. 
  Idem avec $v_n=\left(\frac1n\right)^n$ et $w_n=n^{\frac1n}$.
\end{enumerate}
\end{miniexercice}


\end{frame}

\end{document}
