
%%%%%%%%%%%%%%%%%% PREAMBULE %%%%%%%%%%%%%%%%%%


\documentclass[12pt]{article}

\usepackage{amsfonts,amsmath,amssymb,amsthm}
\usepackage[utf8]{inputenc}
\usepackage[T1]{fontenc}
\usepackage[francais]{babel}


% packages
\usepackage{amsfonts,amsmath,amssymb,amsthm}
\usepackage[utf8]{inputenc}
\usepackage[T1]{fontenc}
%\usepackage{lmodern}

\usepackage[francais]{babel}
\usepackage{fancybox}
\usepackage{graphicx}

\usepackage{float}

%\usepackage[usenames, x11names]{xcolor}
\usepackage{tikz}
\usepackage{datetime}

\usepackage{mathptmx}
%\usepackage{fouriernc}
%\usepackage{newcent}
\usepackage[mathcal,mathbf]{euler}

%\usepackage{palatino}
%\usepackage{newcent}


% Commande spéciale prompteur

%\usepackage{mathptmx}
%\usepackage[mathcal,mathbf]{euler}
%\usepackage{mathpple,multido}

\usepackage[a4paper]{geometry}
\geometry{top=2cm, bottom=2cm, left=1cm, right=1cm, marginparsep=1cm}

\newcommand{\change}{{\color{red}\rule{\textwidth}{1mm}\\}}

\newcounter{mydiapo}

\newcommand{\diapo}{\newpage
\hfill {\normalsize  Diapo \themydiapo \quad \texttt{[\jobname]}} \\
\stepcounter{mydiapo}}


%%%%%%% COULEURS %%%%%%%%%%

% Pour blanc sur noir :
%\pagecolor[rgb]{0.5,0.5,0.5}
% \pagecolor[rgb]{0,0,0}
% \color[rgb]{1,1,1}



%\DeclareFixedFont{\myfont}{U}{cmss}{bx}{n}{18pt}
\newcommand{\debuttexte}{
%%%%%%%%%%%%% FONTES %%%%%%%%%%%%%
\renewcommand{\baselinestretch}{1.5}
\usefont{U}{cmss}{bx}{n}
\bfseries

% Taille normale : commenter le reste !
%Taille Arnaud
%\fontsize{19}{19}\selectfont

% Taille Barbara
%\fontsize{21}{22}\selectfont

%Taille François
%\fontsize{25}{30}\selectfont

%Taille Pascal
%\fontsize{25}{30}\selectfont

%Taille Laura
%\fontsize{30}{35}\selectfont


%\myfont
%\usefont{U}{cmss}{bx}{n}

%\Huge
%\addtolength{\parskip}{\baselineskip}
}


% \usepackage{hyperref}
% \hypersetup{colorlinks=true, linkcolor=blue, urlcolor=blue,
% pdftitle={Exo7 - Exercices de mathématiques}, pdfauthor={Exo7}}


%section
% \usepackage{sectsty}
% \allsectionsfont{\bf}
%\sectionfont{\color{Tomato3}\upshape\selectfont}
%\subsectionfont{\color{Tomato4}\upshape\selectfont}

%----- Ensembles : entiers, reels, complexes -----
\newcommand{\Nn}{\mathbb{N}} \newcommand{\N}{\mathbb{N}}
\newcommand{\Zz}{\mathbb{Z}} \newcommand{\Z}{\mathbb{Z}}
\newcommand{\Qq}{\mathbb{Q}} \newcommand{\Q}{\mathbb{Q}}
\newcommand{\Rr}{\mathbb{R}} \newcommand{\R}{\mathbb{R}}
\newcommand{\Cc}{\mathbb{C}} 
\newcommand{\Kk}{\mathbb{K}} \newcommand{\K}{\mathbb{K}}

%----- Modifications de symboles -----
\renewcommand{\epsilon}{\varepsilon}
\renewcommand{\Re}{\mathop{\text{Re}}\nolimits}
\renewcommand{\Im}{\mathop{\text{Im}}\nolimits}
%\newcommand{\llbracket}{\left[\kern-0.15em\left[}
%\newcommand{\rrbracket}{\right]\kern-0.15em\right]}

\renewcommand{\ge}{\geqslant}
\renewcommand{\geq}{\geqslant}
\renewcommand{\le}{\leqslant}
\renewcommand{\leq}{\leqslant}

%----- Fonctions usuelles -----
\newcommand{\ch}{\mathop{\mathrm{ch}}\nolimits}
\newcommand{\sh}{\mathop{\mathrm{sh}}\nolimits}
\renewcommand{\tanh}{\mathop{\mathrm{th}}\nolimits}
\newcommand{\cotan}{\mathop{\mathrm{cotan}}\nolimits}
\newcommand{\Arcsin}{\mathop{\mathrm{Arcsin}}\nolimits}
\newcommand{\Arccos}{\mathop{\mathrm{Arccos}}\nolimits}
\newcommand{\Arctan}{\mathop{\mathrm{Arctan}}\nolimits}
\newcommand{\Argsh}{\mathop{\mathrm{Argsh}}\nolimits}
\newcommand{\Argch}{\mathop{\mathrm{Argch}}\nolimits}
\newcommand{\Argth}{\mathop{\mathrm{Argth}}\nolimits}
\newcommand{\pgcd}{\mathop{\mathrm{pgcd}}\nolimits} 

\newcommand{\Card}{\mathop{\text{Card}}\nolimits}
\newcommand{\Ker}{\mathop{\text{Ker}}\nolimits}
\newcommand{\id}{\mathop{\text{id}}\nolimits}
\newcommand{\ii}{\mathrm{i}}
\newcommand{\dd}{\mathrm{d}}
\newcommand{\Vect}{\mathop{\text{Vect}}\nolimits}
\newcommand{\Mat}{\mathop{\mathrm{Mat}}\nolimits}
\newcommand{\rg}{\mathop{\text{rg}}\nolimits}
\newcommand{\tr}{\mathop{\text{tr}}\nolimits}
\newcommand{\ppcm}{\mathop{\text{ppcm}}\nolimits}

%----- Structure des exercices ------

\newtheoremstyle{styleexo}% name
{2ex}% Space above
{3ex}% Space below
{}% Body font
{}% Indent amount 1
{\bfseries} % Theorem head font
{}% Punctuation after theorem head
{\newline}% Space after theorem head 2
{}% Theorem head spec (can be left empty, meaning ‘normal’)

%\theoremstyle{styleexo}
\newtheorem{exo}{Exercice}
\newtheorem{ind}{Indications}
\newtheorem{cor}{Correction}


\newcommand{\exercice}[1]{} \newcommand{\finexercice}{}
%\newcommand{\exercice}[1]{{\tiny\texttt{#1}}\vspace{-2ex}} % pour afficher le numero absolu, l'auteur...
\newcommand{\enonce}{\begin{exo}} \newcommand{\finenonce}{\end{exo}}
\newcommand{\indication}{\begin{ind}} \newcommand{\finindication}{\end{ind}}
\newcommand{\correction}{\begin{cor}} \newcommand{\fincorrection}{\end{cor}}

\newcommand{\noindication}{\stepcounter{ind}}
\newcommand{\nocorrection}{\stepcounter{cor}}

\newcommand{\fiche}[1]{} \newcommand{\finfiche}{}
\newcommand{\titre}[1]{\centerline{\large \bf #1}}
\newcommand{\addcommand}[1]{}
\newcommand{\video}[1]{}

% Marge
\newcommand{\mymargin}[1]{\marginpar{{\small #1}}}



%----- Presentation ------
\setlength{\parindent}{0cm}

%\newcommand{\ExoSept}{\href{http://exo7.emath.fr}{\textbf{\textsf{Exo7}}}}

\definecolor{myred}{rgb}{0.93,0.26,0}
\definecolor{myorange}{rgb}{0.97,0.58,0}
\definecolor{myyellow}{rgb}{1,0.86,0}

\newcommand{\LogoExoSept}[1]{  % input : echelle
{\usefont{U}{cmss}{bx}{n}
\begin{tikzpicture}[scale=0.1*#1,transform shape]
  \fill[color=myorange] (0,0)--(4,0)--(4,-4)--(0,-4)--cycle;
  \fill[color=myred] (0,0)--(0,3)--(-3,3)--(-3,0)--cycle;
  \fill[color=myyellow] (4,0)--(7,4)--(3,7)--(0,3)--cycle;
  \node[scale=5] at (3.5,3.5) {Exo7};
\end{tikzpicture}}
}



\theoremstyle{definition}
%\newtheorem{proposition}{Proposition}
%\newtheorem{exemple}{Exemple}
%\newtheorem{theoreme}{Théorème}
\newtheorem{lemme}{Lemme}
\newtheorem{corollaire}{Corollaire}
%\newtheorem*{remarque*}{Remarque}
%\newtheorem*{miniexercice}{Mini-exercices}
%\newtheorem{definition}{Définition}




%definition d'un terme
\newcommand{\defi}[1]{{\color{myorange}\textbf{\emph{#1}}}}
\newcommand{\evidence}[1]{{\color{blue}\textbf{\emph{#1}}}}



 %----- Commandes divers ------

\newcommand{\codeinline}[1]{\texttt{#1}}

%%%%%%%%%%%%%%%%%%%%%%%%%%%%%%%%%%%%%%%%%%%%%%%%%%%%%%%%%%%%%
%%%%%%%%%%%%%%%%%%%%%%%%%%%%%%%%%%%%%%%%%%%%%%%%%%%%%%%%%%%%%



\begin{document}

\debuttexte

%%%%%%%%%%%%%%%%%%%%%%%%%%%%%%%%%%%%%%%%%%%%%%%%%%%%%%%%%%
\diapo

\change

Nous consacrons cette partie sur les fonctions usuelles 

\change

tout d'abord à la fonction logarithme, et ses variantes.

\change

Puis à la fonction exponentielle et ses variantes.

\change

Puis nous comparerons le comportement de ces fonctions.


%%%%%%%%%%%%%%%%%%%%%%%%%%%%%%%%%%%%%%%%%%%%%%%%%%%%%%%%%%
\diapo

Il existe une unique fonction, le logarithme naturel,

que l'on note $\ln$ telle que :

$$\ln'(x) = \frac 1x \quad (\text{pour tout } x>0) \qquad \text { et } \qquad \ln(1) = 0.$$

[[Sur le graphe]]


La fonction logarithme tend vers $-\infty$ quand $x$ tend vers $0$.

Elle tend vers $+\infty$ très lentement lorsque $x\to +\infty$.

Elle s'annule en $x=1$.

Et on appelle $e$ le réel tel que $ln(e)=1$.



%%%%%%%%%%%%%%%%%%%%%%%%%%%%%%%%%%%%%%%%%%%%%%%%%%%%%%%%%%
\diapo

Voici les propriétés les plus remarquables de la fonction logarithme :

tout d'abord pour n'importe quel $a,b>0$
on a : $\ln (a \times b) = \ln a + \ln b$,

C'est l'une des raisons d'être du logarithme :

"Le logarithme d'un produit est la somme des logarithmes."


\change

On en déduit facilement que
$\ln(\frac 1 a) = - \ln a$.


\change

Et aussi pour tout entier $n$ que
$\ln(a^n) = n \ln a$,


\change

Maintenant on s'intéresse à la fonction qui à $x>0$ associe $\ln(x)$.

Cette fonction logarithme est une fonction continue, 

elle est strictement croissante et en plus définit une bijection de
l'intervalle $]0,+\infty[$ (ouvert en $0$) sur $\Rr$,

% La limite lorsque $x \to 0$ est $-\infty$ et la limite lorsque  $x\to +\infty$
% est $+\infty$.

\change

Une autre limite importante est la limite 
 de $\frac{\ln(1+x)}{x}$ qui vaut  $1$ lorsque $x\to0$.
 
 
\change

Enfin la fonction logarithme est une fonction concave 

ce qui géométriquement signifie que le segment qui joint deux point du graphe de la fonction reste entièrement *sous* le graphe.

Enfin par une étude de fonction, on prouve que pour tout $x>0$ :
 $\ln x \le x-1$.






%%%%%%%%%%%%%%%%%%%%%%%%%%%%%%%%%%%%%%%%%%%%%%%%%%%%%%%%%%
\diapo

$\ln x$ s'appelle le \defi{logarithme naturel} ou aussi \defi{logarithme  néperien} de $x$

\change

Ce logarithme est caractérisé par l'égalité $\ln(e)=1$.

\change

On définit d'autres fonctions logarithmes.

Plus précisément on appelle le \defi{logarithme en base $a$} de $x$
le réel défini par 
$$\log_a(x) = \frac{\ln(x)}{\ln(a)}$$

\change


Ce logarithme est défini de sorte que $\log_a(a)=1$.

\change

Le plus connu est  le \defi{logarithme décimal},

pour le quel la base est $a=10$ 

il vérifie $\log_{10} (10)=1$ 

et plus généralement $\log_{10}(10^n)=n$.


Dans la pratique on utilise l'équivalence : $x=10^y \iff y = \log_{10}(x)$.

\change

En informatique, dans l'écriture des nombres binaires, intervient aussi fréquemment 
le logarithme en base $2$ qui vérifie lui que $\log_2(2^n)=n$.


%%%%%%%%%%%%%%%%%%%%%%%%%%%%%%%%%%%%%%%%%%%%%%%%%%%%%%%%%%
\diapo


Passons à quelques démonstration sur la fonction $\ln$.

L'existence et l'unicité viennent de la théorie de l'intégration :  

par définition $\ln(x) = \int_1^x \frac1t \, dt$.

En tant que primitive c'est donc une fonction continue, dérivable et bien sûr 
sa dérivée est $1/x$.


  
\change

Passons aux démonstrations de quelques propriétés.

Tout d'abord montrons que $\ln(x \times y)=\ln x + \ln y$.

\change

Pour cela on introduit une fonction auxiliaire
$f(x)=\ln(xy)-\ln(x)$, où $y$ est ici un réel fixé. 

\change

Calculons la dérivée de $f$ :

   $f'(x)=y\ln'(xy)-\ln'(x)=$ et donc par définition du $\ln$ cela vaut $\frac{y}{xy}-\frac1x$ ce qui fait $0$.
   
\change


  Donc $f$ a une dérivée nulle,  ce qui implique que $f$ est une fonction constante 
 
\change

Ainsi, par exemple, pour tout $x$, $f(x)$ est égal à $f(1)$
 
 Mais $f(1) = \ln(y)-\ln(1)=\ln(y)$. 
 
 
 \change 
 
 Conclusion $f(x)=\ln y$, ce qui entraine bien   $\ln(xy)=\ln(x)+\ln(y)$.
 
 
\change

Montrons aussi que pour tout $x>0$ : $\ln x \le x-1$

\change

On étudie cette fois la fonction définie par 
$f(x)=x-1-\ln(x)$. 

\change

Sa dérivée est $f'(x)= 1 -\frac1x$.

La dérivée est négative avant $1$ et positive après.

\change

Donc la fonction $f$ est décroissante, puis croissante et 
atteint son minimum en $x_0=1$.

\change

  Donc $f(x)\ge f(1)$ qui vaut $0$. Ce qui implique l'inégalité $\ln x \le x-1$.
  
%%%%%%%%%%%%%%%%%%%%%%%%%%%%%%%%%%%%%%%%%%%%%%%%%%%%%%%%%%
\diapo


Par définition la bijection réciproque du logarithme s'appelle la 
fonction \defi{exponentielle}, c'est une fonction qui est définie pour tout réel et à valeurs dans l'intervalle $]0,+\infty[$. 

On note $exp(x)$ ou $e^x$ pour l'exponentielle évaluée en $x$.

Voici le graphe de l'exponentielle,

la fonction tend vers $0$ lorsque $x\to -\infty$

et croît très rapidement vers $+\infty$ lorsque $x\to+\infty$.

Comme $\ln(1)=0$ alors $\exp(0)=1$.

Et comme $\ln(e)=1$ alors $\exp(1)=e$.


%%%%%%%%%%%%%%%%%%%%%%%%%%%%%%%%%%%%%%%%%%%%%%%%%%%%%%%%%%
\diapo

Comme l'exponentielle est par définition la bijection réciproque du logarithme alors

d'une part 
$\exp(\ln x) = x$, ceci pour tout $x >0$,

\change

et d'autre part $\ln(\exp x) = x$, ceci pour tout $x$ réel.

\change

Comme l'exponentielle est la bijection réciproque du logarithme, on transcrit les propriétés du log
en des propriétés pour l'exponentielle.


Tout d'abord $\exp(a+b) = \exp(a) \times \exp(b)$

\change

Ensuite $(\exp x)^n$ vaut $\exp(nx)$.

\change

L'exponentielle $\exp$ est une fonction continue et strictement croissante

\change

 La fonction exponentielle est aussi dérivable et $\exp' x = \exp x$, pour tout réel
 
c'est-à-dire sa dérivée est elle-même 

\change

C'est une fonction convexe 


donc géométriquement le segment qui joint deux points du graphe de la fonction reste entièrement *au-dessus* du graphe.

Enfin $\exp x \ge 1+x$.

Ce sont toutes des propriétés fondamentales qu'il faut connaître.

%%%%%%%%%%%%%%%%%%%%%%%%%%%%%%%%%%%%%%%%%%%%%%%%%%%%%%%%%%
\diapo

Encore une fois les propriétés précédentes sont celles du logarithme retranscrites pour sa bijection réciproque.

Par exemple vérifions  que la dérivée de l'exponentielle est elle-même :

\change

on part de l'égalité $\ln(\exp x)=x$ que l'on dérive.

\change

Il s'agit donc de dériver une composition et cela donne 
$\exp'(x) \times \ln'(\exp x) = 1$ 

\change

donc $\exp'(x) \times \frac{1}{\exp x} = 1$

\change

et ainsi $\exp'(x) = \exp x$.


\change

On caractérise la fonction exponentielle de la façon suivante :

"La fonction exponentielle est l'unique fonction
telle que sa dérivée est elle même, c-à-d: $\exp'(x) = \exp(x)$ 

*et* telle que $\exp(1) = e$. "

\change


Où $e$ est le nombre d'Euler qui vérifie $\ln e = 1$.

$e$ vaut environ $2,718$




%%%%%%%%%%%%%%%%%%%%%%%%%%%%%%%%%%%%%%%%%%%%%%%%%%%%%%%%%%
\diapo

Pour $a$ un réel strictement positif et pour tout réel $b$,

on pose par définition :
$\displaystyle a^b = \exp\big(b \ln a\big)$


\change

Voici quelques exemples :

$\sqrt a = a^\frac12 = \exp\big( \frac12 \ln a\big)$


\change

$\sqrt[n] a = a^\frac1n = \exp\big( \frac1n \ln a\big)$

\change

On note aussi $\exp x$ par $e^x$ ce qui se justifie par le calcul : 
$e^x =  \exp\big(x \ln e \big) = \exp(x)$.


\change

Les fonctions $x \mapsto a^x$ s'appellent aussi des fonctions exponentielles 

\change

elles se ramènent *systématiquement* à la fonction exponentielle classique 
par l'égalité $a^x = \exp(x \ln a)$. 


\change

Attention ! Il ne faut surtout pas les confondre avec les fonctions puissances 
$x \mapsto x^a$.


%%%%%%%%%%%%%%%%%%%%%%%%%%%%%%%%%%%%%%%%%%%%%%%%%%%%%%%%%%
\diapo

Comparons les fonctions $\ln x$, $\exp x$ avec $x$ :

Tout d'abord 
$$\lim_{x\to +\infty} \frac{\ln x}{x} = 0$$


\change

et d'autre part 

$$\lim_{x\to +\infty} \frac{\exp x}{x} = +\infty.$$

\change


Sur les graphes cela se traduit ainsi :

la fonction $\ln$ tend vers $+ \infty$ mais beaucoup lentement
que la fonction $x$.

L'exponentielle croît aussi vers $+\infty$ mais beaucoup plus rapidement que $x$

On en profite pour dessiner les graphes de quelques fonctions puissances.

Par exemple $x \mapsto x^a$, avec $a <1$, comme $\sqrt x$ ou $\sqrt[3]x$.

Cette fonction croît moins vite que $x$ mais plus vite que $\ln x$.


Si $a > 1$, par exemple $x^2$ ou $x^3$,

alors cette fonction croît plus vite que $x$ mais moins vite que $\exp x$.



%%%%%%%%%%%%%%%%%%%%%%%%%%%%%%%%%%%%%%%%%%%%%%%%%%%%%%%%%%
\diapo

Nous allons prouver que la limite de $\frac{\ln x}{x}$
lorsque $x\to +\infty$ est nulle.

\change

On a vu précédemment que pour tout $x>0$ : $\ln x \le x-1$. 

En particulier cela entraîne que $\ln x \le x$


\change

En appliquant cette inégalité à $\sqrt x$ au lieu de $x$ on obtient 

$\frac{\ln \sqrt x}{\sqrt x} \le 1$

\change

Partons de $\frac{\ln x}{x}$

on réécrit le numérateur sous la forme $\ln \left(\sqrt{x}^2\right)$

Par la propriété $\ln u^2 = 2\ln u$ on obtient
$2 \frac{\ln \sqrt x}{x} $

On écrit cette fois le dénominateur sous la forme 
$\sqrt x \times \sqrt x$.

On utilise maintenant la majoration obtenue juste au-dessus :
$\frac{\ln \sqrt x}{\sqrt x} \le 1$

ce qui conduit à l'encadrement 

$0 \le \frac{\ln x}{x} \le \frac{2}{\sqrt x}$

\change

Bien sûr $\frac{2}{\sqrt x}$ tend vers $0$,

ainsi lorsque $x\to +\infty$ on a $\frac{\ln x}{x} \to 0$.


%%%%%%%%%%%%%%%%%%%%%%%%%%%%%%%%%%%%%%%%%%%%%%%%%%%%%%%%%%
\diapo

Les fonctions logarithmes et exponentielles sont omniprésentes
en mathématiques, aussi cela vaut le coup de bien s'entraîner !

\end{document}
