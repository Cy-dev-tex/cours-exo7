
%%%%%%%%%%%%%%%%%% PREAMBULE %%%%%%%%%%%%%%%%%%

\documentclass[aspectratio=169,utf8]{beamer}
%\documentclass[aspectratio=169,handout]{beamer}

\usetheme{Boadilla}
%\usecolortheme{seahorse}
\usecolortheme[RGB={245,66,24}]{structure}
\useoutertheme{infolines}

% packages
\usepackage{amsfonts,amsmath,amssymb,amsthm}
\usepackage[utf8]{inputenc}
\usepackage[T1]{fontenc}
\usepackage{lmodern}

\usepackage[francais]{babel}
\usepackage{fancybox}
\usepackage{graphicx}

\usepackage{float}
\usepackage{xfrac}

%\usepackage[usenames, x11names]{xcolor}
\usepackage{tikz}
\usepackage{pgfplots}
\usepackage{datetime}



%-----  Package unités -----
\usepackage{siunitx}
\sisetup{locale = FR,detect-all,per-mode = symbol}

%\usepackage{mathptmx}
%\usepackage{fouriernc}
%\usepackage{newcent}
%\usepackage[mathcal,mathbf]{euler}

%\usepackage{palatino}
%\usepackage{newcent}
% \usepackage[mathcal,mathbf]{euler}



% \usepackage{hyperref}
% \hypersetup{colorlinks=true, linkcolor=blue, urlcolor=blue,
% pdftitle={Exo7 - Exercices de mathématiques}, pdfauthor={Exo7}}


%section
% \usepackage{sectsty}
% \allsectionsfont{\bf}
%\sectionfont{\color{Tomato3}\upshape\selectfont}
%\subsectionfont{\color{Tomato4}\upshape\selectfont}

%----- Ensembles : entiers, reels, complexes -----
\newcommand{\Nn}{\mathbb{N}} \newcommand{\N}{\mathbb{N}}
\newcommand{\Zz}{\mathbb{Z}} \newcommand{\Z}{\mathbb{Z}}
\newcommand{\Qq}{\mathbb{Q}} \newcommand{\Q}{\mathbb{Q}}
\newcommand{\Rr}{\mathbb{R}} \newcommand{\R}{\mathbb{R}}
\newcommand{\Cc}{\mathbb{C}} 
\newcommand{\Kk}{\mathbb{K}} \newcommand{\K}{\mathbb{K}}

%----- Modifications de symboles -----
\renewcommand{\epsilon}{\varepsilon}
\renewcommand{\Re}{\mathop{\text{Re}}\nolimits}
\renewcommand{\Im}{\mathop{\text{Im}}\nolimits}
%\newcommand{\llbracket}{\left[\kern-0.15em\left[}
%\newcommand{\rrbracket}{\right]\kern-0.15em\right]}

\renewcommand{\ge}{\geqslant}
\renewcommand{\geq}{\geqslant}
\renewcommand{\le}{\leqslant}
\renewcommand{\leq}{\leqslant}
\renewcommand{\epsilon}{\varepsilon}

%----- Fonctions usuelles -----
\newcommand{\ch}{\mathop{\text{ch}}\nolimits}
\newcommand{\sh}{\mathop{\text{sh}}\nolimits}
\renewcommand{\tanh}{\mathop{\text{th}}\nolimits}
\newcommand{\cotan}{\mathop{\text{cotan}}\nolimits}
\newcommand{\Arcsin}{\mathop{\text{arcsin}}\nolimits}
\newcommand{\Arccos}{\mathop{\text{arccos}}\nolimits}
\newcommand{\Arctan}{\mathop{\text{arctan}}\nolimits}
\newcommand{\Argsh}{\mathop{\text{argsh}}\nolimits}
\newcommand{\Argch}{\mathop{\text{argch}}\nolimits}
\newcommand{\Argth}{\mathop{\text{argth}}\nolimits}
\newcommand{\pgcd}{\mathop{\text{pgcd}}\nolimits} 


%----- Commandes divers ------
\newcommand{\ii}{\mathrm{i}}
\newcommand{\dd}{\text{d}}
\newcommand{\id}{\mathop{\text{id}}\nolimits}
\newcommand{\Ker}{\mathop{\text{Ker}}\nolimits}
\newcommand{\Card}{\mathop{\text{Card}}\nolimits}
\newcommand{\Vect}{\mathop{\text{Vect}}\nolimits}
\newcommand{\Mat}{\mathop{\text{Mat}}\nolimits}
\newcommand{\rg}{\mathop{\text{rg}}\nolimits}
\newcommand{\tr}{\mathop{\text{tr}}\nolimits}


%----- Structure des exercices ------

\newtheoremstyle{styleexo}% name
{2ex}% Space above
{3ex}% Space below
{}% Body font
{}% Indent amount 1
{\bfseries} % Theorem head font
{}% Punctuation after theorem head
{\newline}% Space after theorem head 2
{}% Theorem head spec (can be left empty, meaning ‘normal’)

%\theoremstyle{styleexo}
\newtheorem{exo}{Exercice}
\newtheorem{ind}{Indications}
\newtheorem{cor}{Correction}


\newcommand{\exercice}[1]{} \newcommand{\finexercice}{}
%\newcommand{\exercice}[1]{{\tiny\texttt{#1}}\vspace{-2ex}} % pour afficher le numero absolu, l'auteur...
\newcommand{\enonce}{\begin{exo}} \newcommand{\finenonce}{\end{exo}}
\newcommand{\indication}{\begin{ind}} \newcommand{\finindication}{\end{ind}}
\newcommand{\correction}{\begin{cor}} \newcommand{\fincorrection}{\end{cor}}

\newcommand{\noindication}{\stepcounter{ind}}
\newcommand{\nocorrection}{\stepcounter{cor}}

\newcommand{\fiche}[1]{} \newcommand{\finfiche}{}
\newcommand{\titre}[1]{\centerline{\large \bf #1}}
\newcommand{\addcommand}[1]{}
\newcommand{\video}[1]{}

% Marge
\newcommand{\mymargin}[1]{\marginpar{{\small #1}}}

\def\noqed{\renewcommand{\qedsymbol}{}}


%----- Presentation ------
\setlength{\parindent}{0cm}

%\newcommand{\ExoSept}{\href{http://exo7.emath.fr}{\textbf{\textsf{Exo7}}}}

\definecolor{myred}{rgb}{0.93,0.26,0}
\definecolor{myorange}{rgb}{0.97,0.58,0}
\definecolor{myyellow}{rgb}{1,0.86,0}

\newcommand{\LogoExoSept}[1]{  % input : echelle
{\usefont{U}{cmss}{bx}{n}
\begin{tikzpicture}[scale=0.1*#1,transform shape]
  \fill[color=myorange] (0,0)--(4,0)--(4,-4)--(0,-4)--cycle;
  \fill[color=myred] (0,0)--(0,3)--(-3,3)--(-3,0)--cycle;
  \fill[color=myyellow] (4,0)--(7,4)--(3,7)--(0,3)--cycle;
  \node[scale=5] at (3.5,3.5) {Exo7};
\end{tikzpicture}}
}


\newcommand{\debutmontitre}{
  \author{} \date{} 
  \thispagestyle{empty}
  \hspace*{-10ex}
  \begin{minipage}{\textwidth}
    \titlepage  
  \vspace*{-2.5cm}
  \begin{center}
    \LogoExoSept{2.5}
  \end{center}
  \end{minipage}

  \vspace*{-0cm}
  
  % Astuce pour que le background ne soit pas discrétisé lors de la conversion pdf -> png
\begin{tikzpicture}
        \fill[opacity=0,green!60!black] (0,0)--++(0,0)--++(0,0)--++(0,0)--cycle; 
\end{tikzpicture}

% toc S'affiche trop tot :
% \tableofcontents[hideallsubsections, pausesections]
}

\newcommand{\finmontitre}{
  \end{frame}
  \setcounter{framenumber}{0}
} % ne marche pas pour une raison obscure

%----- Commandes supplementaires ------

% \usepackage[landscape]{geometry}
% \geometry{top=1cm, bottom=3cm, left=2cm, right=10cm, marginparsep=1cm
% }
% \usepackage[a4paper]{geometry}
% \geometry{top=2cm, bottom=2cm, left=2cm, right=2cm, marginparsep=1cm
% }

%\usepackage{standalone}


% New command Arnaud -- november 2011
\setbeamersize{text margin left=24ex}
% si vous modifier cette valeur il faut aussi
% modifier le decalage du titre pour compenser
% (ex : ici =+10ex, titre =-5ex

\theoremstyle{definition}
%\newtheorem{proposition}{Proposition}
%\newtheorem{exemple}{Exemple}
%\newtheorem{theoreme}{Théorème}
%\newtheorem{lemme}{Lemme}
%\newtheorem{corollaire}{Corollaire}
%\newtheorem*{remarque*}{Remarque}
%\newtheorem*{miniexercice}{Mini-exercices}
%\newtheorem{definition}{Définition}

% Commande tikz
\usetikzlibrary{calc}
\usetikzlibrary{patterns,arrows}
\usetikzlibrary{matrix}
\usetikzlibrary{fadings} 

%definition d'un terme
\newcommand{\defi}[1]{{\color{myorange}\textbf{\emph{#1}}}}
\newcommand{\evidence}[1]{{\color{blue}\textbf{\emph{#1}}}}
\newcommand{\assertion}[1]{\emph{\og#1\fg}}  % pour chapitre logique
%\renewcommand{\contentsname}{Sommaire}
\renewcommand{\contentsname}{}
\setcounter{tocdepth}{2}



%------ Figures ------

\def\myscale{1} % par défaut 
\newcommand{\myfigure}[2]{  % entrée : echelle, fichier figure
\def\myscale{#1}
\begin{center}
\footnotesize
{#2}
\end{center}}


%------ Encadrement ------

\usepackage{fancybox}


\newcommand{\mybox}[1]{
\setlength{\fboxsep}{7pt}
\begin{center}
\shadowbox{#1}
\end{center}}

\newcommand{\myboxinline}[1]{
\setlength{\fboxsep}{5pt}
\raisebox{-10pt}{
\shadowbox{#1}
}
}

%--------------- Commande beamer---------------
\newcommand{\beameronly}[1]{#1} % permet de mettre des pause dans beamer pas dans poly


\setbeamertemplate{navigation symbols}{}
\setbeamertemplate{footline}  % tiré du fichier beamerouterinfolines.sty
{
  \leavevmode%
  \hbox{%
  \begin{beamercolorbox}[wd=.333333\paperwidth,ht=2.25ex,dp=1ex,center]{author in head/foot}%
    % \usebeamerfont{author in head/foot}\insertshortauthor%~~(\insertshortinstitute)
    \usebeamerfont{section in head/foot}{\bf\insertshorttitle}
  \end{beamercolorbox}%
  \begin{beamercolorbox}[wd=.333333\paperwidth,ht=2.25ex,dp=1ex,center]{title in head/foot}%
    \usebeamerfont{section in head/foot}{\bf\insertsectionhead}
  \end{beamercolorbox}%
  \begin{beamercolorbox}[wd=.333333\paperwidth,ht=2.25ex,dp=1ex,right]{date in head/foot}%
    % \usebeamerfont{date in head/foot}\insertshortdate{}\hspace*{2em}
    \insertframenumber{} / \inserttotalframenumber\hspace*{2ex} 
  \end{beamercolorbox}}%
  \vskip0pt%
}


\definecolor{mygrey}{rgb}{0.5,0.5,0.5}
\setlength{\parindent}{0cm}
%\DeclareTextFontCommand{\helvetica}{\fontfamily{phv}\selectfont}

% background beamer
\definecolor{couleurhaut}{rgb}{0.85,0.9,1}  % creme
\definecolor{couleurmilieu}{rgb}{1,1,1}  % vert pale
\definecolor{couleurbas}{rgb}{0.85,0.9,1}  % blanc
\setbeamertemplate{background canvas}[vertical shading]%
[top=couleurhaut,middle=couleurmilieu,midpoint=0.4,bottom=couleurbas] 
%[top=fondtitre!05,bottom=fondtitre!60]



\makeatletter
\setbeamertemplate{theorem begin}
{%
  \begin{\inserttheoremblockenv}
  {%
    \inserttheoremheadfont
    \inserttheoremname
    \inserttheoremnumber
    \ifx\inserttheoremaddition\@empty\else\ (\inserttheoremaddition)\fi%
    \inserttheorempunctuation
  }%
}
\setbeamertemplate{theorem end}{\end{\inserttheoremblockenv}}

\newenvironment{theoreme}[1][]{%
   \setbeamercolor{block title}{fg=structure,bg=structure!40}
   \setbeamercolor{block body}{fg=black,bg=structure!10}
   \begin{block}{{\bf Th\'eor\`eme }#1}
}{%
   \end{block}%
}


\newenvironment{proposition}[1][]{%
   \setbeamercolor{block title}{fg=structure,bg=structure!40}
   \setbeamercolor{block body}{fg=black,bg=structure!10}
   \begin{block}{{\bf Proposition }#1}
}{%
   \end{block}%
}

\newenvironment{corollaire}[1][]{%
   \setbeamercolor{block title}{fg=structure,bg=structure!40}
   \setbeamercolor{block body}{fg=black,bg=structure!10}
   \begin{block}{{\bf Corollaire }#1}
}{%
   \end{block}%
}

\newenvironment{mydefinition}[1][]{%
   \setbeamercolor{block title}{fg=structure,bg=structure!40}
   \setbeamercolor{block body}{fg=black,bg=structure!10}
   \begin{block}{{\bf Définition} #1}
}{%
   \end{block}%
}

\newenvironment{lemme}[0]{%
   \setbeamercolor{block title}{fg=structure,bg=structure!40}
   \setbeamercolor{block body}{fg=black,bg=structure!10}
   \begin{block}{\bf Lemme}
}{%
   \end{block}%
}

\newenvironment{remarque}[1][]{%
   \setbeamercolor{block title}{fg=black,bg=structure!20}
   \setbeamercolor{block body}{fg=black,bg=structure!5}
   \begin{block}{Remarque #1}
}{%
   \end{block}%
}


\newenvironment{exemple}[1][]{%
   \setbeamercolor{block title}{fg=black,bg=structure!20}
   \setbeamercolor{block body}{fg=black,bg=structure!5}
   \begin{block}{{\bf Exemple }#1}
}{%
   \end{block}%
}


\newenvironment{miniexercice}[0]{%
   \setbeamercolor{block title}{fg=structure,bg=structure!20}
   \setbeamercolor{block body}{fg=black,bg=structure!5}
   \begin{block}{Mini-exercices}
}{%
   \end{block}%
}


\newenvironment{tp}[0]{%
   \setbeamercolor{block title}{fg=structure,bg=structure!40}
   \setbeamercolor{block body}{fg=black,bg=structure!10}
   \begin{block}{\bf Travaux pratiques}
}{%
   \end{block}%
}
\newenvironment{exercicecours}[1][]{%
   \setbeamercolor{block title}{fg=structure,bg=structure!40}
   \setbeamercolor{block body}{fg=black,bg=structure!10}
   \begin{block}{{\bf Exercice }#1}
}{%
   \end{block}%
}
\newenvironment{algo}[1][]{%
   \setbeamercolor{block title}{fg=structure,bg=structure!40}
   \setbeamercolor{block body}{fg=black,bg=structure!10}
   \begin{block}{{\bf Algorithme}\hfill{\color{gray}\texttt{#1}}}
}{%
   \end{block}%
}


\setbeamertemplate{proof begin}{
   \setbeamercolor{block title}{fg=black,bg=structure!20}
   \setbeamercolor{block body}{fg=black,bg=structure!5}
   \begin{block}{{\footnotesize Démonstration}}
   \footnotesize
   \smallskip}
\setbeamertemplate{proof end}{%
   \end{block}}
\setbeamertemplate{qed symbol}{\openbox}


\makeatother
\usecolortheme[RGB={0,45,179}]{structure}

% Commande spécifique à ce chapitre
\newcounter{saveenumi}

%%%%%%%%%%%%%%%%%%%%%%%%%%%%%%%%%%%%%%%%%%%%%%%%%%%%%%%%%%%%%
%%%%%%%%%%%%%%%%%%%%%%%%%%%%%%%%%%%%%%%%%%%%%%%%%%%%%%%%%%%%%



\begin{document}



\title{{\bf Intégrales}}
\subtitle{L'intégrale de Riemann}

\begin{frame}
  
  \debutmontitre

  \pause

{\footnotesize
\hfill
\setbeamercovered{transparent=50}
\begin{minipage}{0.6\textwidth}
  \begin{itemize}
    \item<3-> Intégrale d'une fonction en escalier
    \item<4-> Fonction intégrable
    \item<5-> Premières propriétés
    \item<6-> Les fonctions continues sont intégrables
    \item<7-> Une preuve
  \end{itemize}
\end{minipage}
}

\end{frame}

\setcounter{framenumber}{0}


%%%%%%%%%%%%%%%%%%%%%%%%%%%%%%%%%%%%%%%%%%%%%%%%%%%%%%%%%%%%%%%%


\section*{Motivation}


\begin{frame}

\begin{minipage}{0.6\textwidth}
\begin{itemize}
\uncover<1->{  \item $f(x)=e^x$ }

\uncover<2->{   \item Calculer l'aire $\mathcal{A}$ }

\uncover<3->{    \item Rectangles inférieurs : base $\big[\frac{i-1}{n},\frac{i}{n}\big]$,
hauteur : $f\big(\frac{i-1}{n}\big)=e^{(i-1)/n}$ }

\uncover<4->{    \item $\sum_{i=1}^{n} \frac1n \times e^{\frac{i-1}{n}}
\uncover<5->{  = \frac{1}{n} \sum_{i=1}^{n} \big(e^{\frac 1n}\big)^{i-1}}
\uncover<6->{= \frac{1}{n} \frac{1-\big(e^{\frac 1n}\big)^{n}}{1-e^{\frac 1n}}}
\uncover<7->{= \frac{\frac{1}{n}}{e^{\frac 1n}-1}\big(e-1\big) }
\uncover<8->{\xrightarrow[n\to+\infty]{} e-1$ }
}

\uncover<9->{  \item Rectangles supérieurs $\sum_{i=1}^{n} \frac{e^{\frac{i}{n}}}{n} \to e-1$ }

\uncover<10->{   \item Subdivision de plus en plus petites ($n\to +\infty$) }

\uncover<11->{   \item $\mathcal{A} = e-1$ }
\end{itemize} 
\end{minipage}
\begin{minipage}{0.39\textwidth}
\myfigure{0.9}{
\only<1,2>{\tikzinput{fig_int01a}}
\only<3-8>{\tikzinput{fig_int01b1}} 
\only<9>{\tikzinput{fig_int01b2}} 
\only<10->{\tikzinput{fig_int01cpres}}
}  
\end{minipage}
 

\end{frame}




%---------------------------------------------------------------
\section*{Intégrale d'une fonction en escalier}


\begin{frame}


\myfigure{0.7}{
\tikzinput{fig_int02} 
} 

\pause

\begin{itemize}
  \item On remplace les rectangles par les \evidence{fonctions en escalier}

\pause

  \item Si la limite des aires en-dessous égale la limite des aires au-dessus on appelle cette limite commune 
\evidence{l'intégrale} de $f$

\pause

  \item  $\int_a^b f(x) \; dx$

\pause

  \item Notion de fonction \evidence{intégrable}

\pause

  \item Les fonctions continues sont intégrables
\end{itemize}




\end{frame}



\begin{frame}

\vfil
\hspace*{-2em}
\begin{minipage}{0.39\textwidth}
\uncover<1->{\myfigure{0.75}{\tikzinput{fig_int03}}}
\uncover<3->{\myfigure{0.75}{\tikzinput{fig_int04}}}  
\end{minipage}
\hfill
\begin{minipage}{0.6\textwidth}
\begin{mydefinition}
\begin{enumerate}
\uncover<1->{  
\item Une \defi{subdivision} de $[a,b]$ $\mathcal{S}=(x_0,x_1,\ldots,x_n)$
telle que \\
\hfil\hfil $a=x_0< x_1<\ldots< x_n=b$
}

\uncover<2->{ 
\item $f : [a,b] \to \Rr$ est une \defi{fonction en escalier}
s'il existe  une subdivision $(x_0,x_1,\ldots,x_n)$ et des réels
$c_1,\ldots,c_n$ tels que pour tout $i\in \{1,\ldots,n\}$ on ait
$\forall x \in ]x_{i-1},x_i[ \quad f(x)=c_i$
}

\uncover<4->{
\item L'\defi{intégrale} d'une fonction en escalier : 
\myboxinline{$\int_a^b f(x) \; dx = \sum_{i=1}^n c_i(x_i-x_{i-1})$}  
}
\end{enumerate}
\end{mydefinition}  
\end{minipage}

\end{frame}





%---------------------------------------------------------------
\section*{Fonction intégrable}


\begin{frame}

\begin{itemize}
  \item $f : [a,b] \to \Rr$ est \defi{bornée} s'il existe $M\ge0$ tel que : \\ \hfil\hfil $\forall x \in [a,b] \quad -M \le f(x) \le M$

\pause
  \item $f \le g  \quad \iff \quad \forall x \in [a,b] \quad f(x) \le g(x)$
\end{itemize}


\pause
Soit $f : [a,b] \to \Rr$ est une fonction bornée quelconque
\begin{itemize}

\pause
  \item $I^-(f) = \sup \left\{ \int_a^b \phi(x) \; dx \mid \phi \text{ en escalier et } \phi \le f \right\}$

\pause
  \item $I^+(f) = \inf \left\{ \int_a^b \phi(x) \; dx \mid \phi \text{ en escalier et } \phi \ge f \right\}$

\pause
  \item Proposition : $I^-(f) \le I^+(f)$
\end{itemize}

 

\pause

\begin{mydefinition}
Une fonction bornée $f :[a,b] \to \Rr$ est dite \defi{intégrable} (\defi{au sens de Riemann})
si $I^-(f) = I^+(f)$ 

\pause

Ce nombre s'appelle alors \defi{l'intégrale de Riemann} de $f$ sur $[a,b]$

\hfil\hfil\myboxinline{$\displaystyle\quad\int_a^b f(x)\; dx\quad$}
\end{mydefinition}
\end{frame}

\begin{frame}
\myfigure{1}{
\tikzinput{fig_int05pres} 
}   
\end{frame}



\begin{frame}

\begin{exemple}
\begin{itemize}
\item Les fonctions en escalier sont intégrables ! 

\pause

\item Les fonctions continues sont intégrables 

\pause

\item Il existe des fonctions non intégrables 

\pause

\begin{minipage}{0.39\textwidth}
$$\begin{array}{rcl} 
f : [0,1] & \to & \Rr    \\
 x & \mapsto & \begin{cases}
               1 & \text{ si } x \in \Qq \\
               0 & \text{ sinon }    
             \end{cases}
  \end{array}$$  
\end{minipage}
\pause
\begin{minipage}{0.39\textwidth}
\myfigure{0.8}{
\tikzinput{fig_int06}
}  
\end{minipage}

\end{itemize}
\end{exemple}
\end{frame}


\begin{frame}

\begin{exemple}

\begin{minipage}{0.49\textwidth}
Soit $f:[0,1] \to \Rr$, $f(x)=x^2$  

\uncover<2->{Est-elle intégrable ?}

\uncover<3->{Que vaut $\int_0^1f(x) \; dx$ ?}
\end{minipage}
\hfill
\begin{minipage}{0.44\textwidth}
\myfigure{0.75}{
\only<1-3>{\tikzinput{fig_int11presbis}}
\only<4->{\tikzinput{fig_int11pres}}
}  
\end{minipage}



\pause
\pause 
\pause

\vspace*{-10mm}

\begin{itemize}
  \item Subdivision régulière de $[0,1]$ suivante 
$\mathcal{S}=\big(0,\frac1n,\frac2n,\ldots,\frac in,\ldots, \frac{n-1}{n},1\big)$
\pause

  \item Sur $\big[\frac{i-1}{n},\frac in\big]$ $\big(\tfrac{i-1}{n}\big)^2 \le x^2 \le \big(\tfrac in\big)^2$
\pause

  \item Fonction en escalier $\phi^-$ définie par $\phi^-(x) = \frac{(i-1)^2}{n^2}$ si $x \in \big[\frac{i-1}{n},\frac in\big[$
\pause

  \item Fonction en escalier $\phi^+$ définie par  $\phi^+(x) = \frac{i^2}{n^2}$ si $x \in \big[\frac{i-1}{n},\frac in\big[$
\end{itemize}
\end{exemple}

\end{frame}


\begin{frame}

\begin{exemple}[$\int_0^1 x^2 \; dx$]

\hfill
\begin{minipage}{0.40\textwidth}
\myfigure{0.75}{
\tikzinput{fig_int11prester}
}  
\end{minipage}



\pause
\pause
\pause
\pause
\vspace*{-30mm}
\begin{itemize}
  \item $\phi^- \le f \le \phi^+$
\pause

  \item 
  \begin{itemize}
     \item $\int_0^1 \phi^+(x)\; dx = \sum_{i=1}^n  \frac{i^2}{n^2} \left(\frac in - \frac{i-1}{n}\right) $

\qquad $= \sum_{i=1}^n  \frac{i^2}{n^2} \frac 1n  = \frac{1}{n^3} \sum_{i=1}^n i^2$
\pause

     \item $\sum_{i=1}^n i^2=\tfrac{n(n + 1)(2n + 1)}{6}$
\pause

     \item $\int_0^1 \phi^+(x)\; dx = \frac{n(n + 1)(2n + 1)}{6n^3} =  \frac{(n + 1)(2n + 1)}{6n^2}$
  \end{itemize}
\pause

  \item $\int_0^1 \phi^-(x)\; dx = \sum_{i=1}^n  \frac{(i-1)^2}{n^2} \frac 1n 
=   \frac{(n - 1)(2n - 1)}{6n^2}$
\pause

  \item {\footnotesize $\frac{(n - 1)(2n - 1)}{6n^2}=\int_0^1 \phi^-(x)\; dx \le I^-(f) 
\pause
\le I^+(f) \le \int_0^1 \phi^+(x)\; dx =\frac{(n + 1)(2n + 1)}{6n^2}$}

\pause

  \item $n \to +\infty$ $\implies$ $I^-(f)= I^+(f)=\frac13$
\pause

  \item $f$ est intégrable et $\int_0^1 x^2\; dx = \frac13$
\end{itemize}

\end{exemple}
\end{frame}





%---------------------------------------------------------------
\section*{Premières propriétés}


\begin{frame}
\begin{proposition}
\label{prop:intprop}
\begin{enumerate}
  \item Si $f : [a,b] \to \Rr$ est intégrable et si l'on change les valeurs de $f$
en un nombre fini de points de $[a,b]$ alors la fonction $f$ est 
toujours intégrable et la valeur de $\int_a^b f(x)\; dx$ ne change pas

\pause

  \item Si $f  : [a,b] \to \Rr$ est intégrable alors la restriction de $f$
à tout intervalle $[a',b'] \subset [a,b]$ est encore intégrable
\end{enumerate}
\end{proposition}
\end{frame}




%---------------------------------------------------------------
\section*{Les fonctions continues sont intégrables}


\begin{frame}

\begin{theoreme}
\label{th:continueintegrable}
Si $f : [a,b] \to \Rr$ est continue alors $f$ est intégrable
\end{theoreme}

\pause
\medskip

$f : [a,b] \to \Rr$ est \defi{continue par morceaux} s'il existe 
une subdivision $(x_0,\ldots,x_n)$ telle que $f_{|]x_{i-1},x_i[}$ soit continue pour tout 
$i \in \{ 1,\ldots,n \}$


\myfigure{0.6}{
\tikzinput{fig_int12}
}
\pause
\vspace*{-5mm}
\begin{corollaire}
Les fonctions continues par morceaux sont intégrables
\end{corollaire}

\end{frame}




%---------------------------------------------------------------
\section*{Une preuve}


\begin{frame}
$f$ est de \defi{classe $\mathcal{C}^1$} si $f$ est continue, dérivable et $f'$ est continue

\pause

\begin{theoreme}[(version faible)]
\label{th:c1integrable}
Si $f : [a,b] \to \Rr$ est de classe $\mathcal{C}^1$ alors $f$ est intégrable 
\end{theoreme}

\pause

\begin{proof}
\begin{itemize}
  \item Il existe $M\ge 0$ tel que pour tout $x \in [a,b]$ on ait $|f'(x)|\le M$
\pause
  \item Inégalité des accroissements finis : $\forall x,y \in [a,b] \quad |f(x)-f(y)| \le M |x-y|$
\pause
  \item Fixons $\epsilon>0$ et une subdivision $(x_0,x_1,\ldots,x_n)$ telle que $0 < x_i-x_{i-1} \le \epsilon$
\pause
  \item Fonction en escalier $\phi^-$ : 

pour  $x\in[x_{i-1},x_i[$ on pose 

$c_i = \phi^-(x)= \inf_{t\in [x_{i-1},x_i[} f(t)$

\uncover<8->{
  \item  $\phi^+$ : $d_i = \phi^+(x)= \sup_{t\in [x_{i-1},x_i[} f(t)$
}
\end{itemize}

\vspace*{-22mm}
\hfill
\begin{minipage}{0.4\textwidth}
\uncover<7->{
\myfigure{0.85}{
\tikzinput{fig_int13} 
}   
}
\end{minipage}
\vspace*{-5mm}

\end{proof}

\end{frame}



\begin{frame}
\begin{proof}

\hfill
\begin{minipage}{0.4\textwidth}
\myfigure{0.85}{
\tikzinput{fig_int13} 
}  
\end{minipage}

\vspace*{-20mm}
\begin{itemize}
 \item $\phi^-\le f \le \phi^+$
\pause
  \item $\int_a^b \phi^-(x)\;dx \le I^-(f) \le I^+(f) \le \int_a^b \phi^+(x)\;dx$
\pause
  \item $f$ continue sur l'intervalle $[x_{i - 1}, x_i]$ 

donc il existe $a_i,b_i \in [x_{i -1},x_i]$ 

tels que $f(a_i)=c_i$ et $f(b_i)=d_i$
\pause
  \item $d_i-c_i = f(b_i) - f(a_i) \le M |b_i - a_i| \le M (x_i-x_{i-1}) \le M\epsilon$
\pause
  \item  $\int_a^b \phi^+(x)\;dx - \int_a^b \phi^-(x)\;dx \le \sum_{i=1}^n M\epsilon(x_i-x_{i-1})=M\epsilon(b-a)$
\pause
   \item $0 \le I^+(f) - I^-(f) \le M\epsilon(b-a)$
\pause
  \item lorsque $\epsilon \to 0$ : $I^+(f) = I^-(f)$
\pause
  \item  $f$ est intégrable 
\end{itemize}
\end{proof}
\end{frame}





%%%%%%%%%%%%%%%%%%%%%%%%%%%%%%%%%%%%%%%%%%%%%%%%%%%%%%%%%%%%%%%%
\section*{Mini-exercices}


\begin{frame}
\begin{miniexercice}
\begin{enumerate}
  \item Soit $f : [1,4] \to \Rr$ définie par $f(x)=1$ si $x\in[1,2[$, $f(x)=3$ si $x\in [2,3[$
et $f(x)=-1$ si $x \in [3,4]$. Calculer $\int_1^2 f(x) \;dx$, $\int_1^3 f(x)\;dx$, $\int_1^4 f(x)\;dx$,
$\int_1^{\frac32} f(x) \;dx$, $\int_{\frac32}^{\frac72} f(x) \;dx$.
  \item Montrer que $\int_0^1 x \; dx =1$ (prendre une subdivision régulière 
et utiliser $\sum_{i=1}^n i = \frac{n(n+1)}{2}$).
  \item Montrer que si $f$ est une fonction intégrable et \emph{paire} sur l'intervalle $[-a,a]$ alors
$\int_{-a}^a f(x)\;dx = 2 \int_0^a f(x)\; dx$ (on prendra une subdivision symétrique par rapport à l'origine).
  \item Montrer que si $f$ est une fonction intégrable et \emph{impaire} sur l'intervalle $[-a,a]$ alors
  $\int_{-a}^a f(x)\;dx = 0$.
  \item Montrer que tout fonction monotone est intégrable en s'inspirant des preuves du cours.
\end{enumerate}
\end{miniexercice}
\end{frame}


\end{document}