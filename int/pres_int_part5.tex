
%%%%%%%%%%%%%%%%%% PREAMBULE %%%%%%%%%%%%%%%%%%

\documentclass[aspectratio=169,utf8]{beamer}
%\documentclass[aspectratio=169,handout]{beamer}

\usetheme{Boadilla}
%\usecolortheme{seahorse}
\usecolortheme[RGB={245,66,24}]{structure}
\useoutertheme{infolines}

% packages
\usepackage{amsfonts,amsmath,amssymb,amsthm}
\usepackage[utf8]{inputenc}
\usepackage[T1]{fontenc}
\usepackage{lmodern}

\usepackage[francais]{babel}
\usepackage{fancybox}
\usepackage{graphicx}

\usepackage{float}
\usepackage{xfrac}

%\usepackage[usenames, x11names]{xcolor}
\usepackage{tikz}
\usepackage{pgfplots}
\usepackage{datetime}



%-----  Package unités -----
\usepackage{siunitx}
\sisetup{locale = FR,detect-all,per-mode = symbol}

%\usepackage{mathptmx}
%\usepackage{fouriernc}
%\usepackage{newcent}
%\usepackage[mathcal,mathbf]{euler}

%\usepackage{palatino}
%\usepackage{newcent}
% \usepackage[mathcal,mathbf]{euler}



% \usepackage{hyperref}
% \hypersetup{colorlinks=true, linkcolor=blue, urlcolor=blue,
% pdftitle={Exo7 - Exercices de mathématiques}, pdfauthor={Exo7}}


%section
% \usepackage{sectsty}
% \allsectionsfont{\bf}
%\sectionfont{\color{Tomato3}\upshape\selectfont}
%\subsectionfont{\color{Tomato4}\upshape\selectfont}

%----- Ensembles : entiers, reels, complexes -----
\newcommand{\Nn}{\mathbb{N}} \newcommand{\N}{\mathbb{N}}
\newcommand{\Zz}{\mathbb{Z}} \newcommand{\Z}{\mathbb{Z}}
\newcommand{\Qq}{\mathbb{Q}} \newcommand{\Q}{\mathbb{Q}}
\newcommand{\Rr}{\mathbb{R}} \newcommand{\R}{\mathbb{R}}
\newcommand{\Cc}{\mathbb{C}} 
\newcommand{\Kk}{\mathbb{K}} \newcommand{\K}{\mathbb{K}}

%----- Modifications de symboles -----
\renewcommand{\epsilon}{\varepsilon}
\renewcommand{\Re}{\mathop{\text{Re}}\nolimits}
\renewcommand{\Im}{\mathop{\text{Im}}\nolimits}
%\newcommand{\llbracket}{\left[\kern-0.15em\left[}
%\newcommand{\rrbracket}{\right]\kern-0.15em\right]}

\renewcommand{\ge}{\geqslant}
\renewcommand{\geq}{\geqslant}
\renewcommand{\le}{\leqslant}
\renewcommand{\leq}{\leqslant}
\renewcommand{\epsilon}{\varepsilon}

%----- Fonctions usuelles -----
\newcommand{\ch}{\mathop{\text{ch}}\nolimits}
\newcommand{\sh}{\mathop{\text{sh}}\nolimits}
\renewcommand{\tanh}{\mathop{\text{th}}\nolimits}
\newcommand{\cotan}{\mathop{\text{cotan}}\nolimits}
\newcommand{\Arcsin}{\mathop{\text{arcsin}}\nolimits}
\newcommand{\Arccos}{\mathop{\text{arccos}}\nolimits}
\newcommand{\Arctan}{\mathop{\text{arctan}}\nolimits}
\newcommand{\Argsh}{\mathop{\text{argsh}}\nolimits}
\newcommand{\Argch}{\mathop{\text{argch}}\nolimits}
\newcommand{\Argth}{\mathop{\text{argth}}\nolimits}
\newcommand{\pgcd}{\mathop{\text{pgcd}}\nolimits} 


%----- Commandes divers ------
\newcommand{\ii}{\mathrm{i}}
\newcommand{\dd}{\text{d}}
\newcommand{\id}{\mathop{\text{id}}\nolimits}
\newcommand{\Ker}{\mathop{\text{Ker}}\nolimits}
\newcommand{\Card}{\mathop{\text{Card}}\nolimits}
\newcommand{\Vect}{\mathop{\text{Vect}}\nolimits}
\newcommand{\Mat}{\mathop{\text{Mat}}\nolimits}
\newcommand{\rg}{\mathop{\text{rg}}\nolimits}
\newcommand{\tr}{\mathop{\text{tr}}\nolimits}


%----- Structure des exercices ------

\newtheoremstyle{styleexo}% name
{2ex}% Space above
{3ex}% Space below
{}% Body font
{}% Indent amount 1
{\bfseries} % Theorem head font
{}% Punctuation after theorem head
{\newline}% Space after theorem head 2
{}% Theorem head spec (can be left empty, meaning ‘normal’)

%\theoremstyle{styleexo}
\newtheorem{exo}{Exercice}
\newtheorem{ind}{Indications}
\newtheorem{cor}{Correction}


\newcommand{\exercice}[1]{} \newcommand{\finexercice}{}
%\newcommand{\exercice}[1]{{\tiny\texttt{#1}}\vspace{-2ex}} % pour afficher le numero absolu, l'auteur...
\newcommand{\enonce}{\begin{exo}} \newcommand{\finenonce}{\end{exo}}
\newcommand{\indication}{\begin{ind}} \newcommand{\finindication}{\end{ind}}
\newcommand{\correction}{\begin{cor}} \newcommand{\fincorrection}{\end{cor}}

\newcommand{\noindication}{\stepcounter{ind}}
\newcommand{\nocorrection}{\stepcounter{cor}}

\newcommand{\fiche}[1]{} \newcommand{\finfiche}{}
\newcommand{\titre}[1]{\centerline{\large \bf #1}}
\newcommand{\addcommand}[1]{}
\newcommand{\video}[1]{}

% Marge
\newcommand{\mymargin}[1]{\marginpar{{\small #1}}}

\def\noqed{\renewcommand{\qedsymbol}{}}


%----- Presentation ------
\setlength{\parindent}{0cm}

%\newcommand{\ExoSept}{\href{http://exo7.emath.fr}{\textbf{\textsf{Exo7}}}}

\definecolor{myred}{rgb}{0.93,0.26,0}
\definecolor{myorange}{rgb}{0.97,0.58,0}
\definecolor{myyellow}{rgb}{1,0.86,0}

\newcommand{\LogoExoSept}[1]{  % input : echelle
{\usefont{U}{cmss}{bx}{n}
\begin{tikzpicture}[scale=0.1*#1,transform shape]
  \fill[color=myorange] (0,0)--(4,0)--(4,-4)--(0,-4)--cycle;
  \fill[color=myred] (0,0)--(0,3)--(-3,3)--(-3,0)--cycle;
  \fill[color=myyellow] (4,0)--(7,4)--(3,7)--(0,3)--cycle;
  \node[scale=5] at (3.5,3.5) {Exo7};
\end{tikzpicture}}
}


\newcommand{\debutmontitre}{
  \author{} \date{} 
  \thispagestyle{empty}
  \hspace*{-10ex}
  \begin{minipage}{\textwidth}
    \titlepage  
  \vspace*{-2.5cm}
  \begin{center}
    \LogoExoSept{2.5}
  \end{center}
  \end{minipage}

  \vspace*{-0cm}
  
  % Astuce pour que le background ne soit pas discrétisé lors de la conversion pdf -> png
\begin{tikzpicture}
        \fill[opacity=0,green!60!black] (0,0)--++(0,0)--++(0,0)--++(0,0)--cycle; 
\end{tikzpicture}

% toc S'affiche trop tot :
% \tableofcontents[hideallsubsections, pausesections]
}

\newcommand{\finmontitre}{
  \end{frame}
  \setcounter{framenumber}{0}
} % ne marche pas pour une raison obscure

%----- Commandes supplementaires ------

% \usepackage[landscape]{geometry}
% \geometry{top=1cm, bottom=3cm, left=2cm, right=10cm, marginparsep=1cm
% }
% \usepackage[a4paper]{geometry}
% \geometry{top=2cm, bottom=2cm, left=2cm, right=2cm, marginparsep=1cm
% }

%\usepackage{standalone}


% New command Arnaud -- november 2011
\setbeamersize{text margin left=24ex}
% si vous modifier cette valeur il faut aussi
% modifier le decalage du titre pour compenser
% (ex : ici =+10ex, titre =-5ex

\theoremstyle{definition}
%\newtheorem{proposition}{Proposition}
%\newtheorem{exemple}{Exemple}
%\newtheorem{theoreme}{Théorème}
%\newtheorem{lemme}{Lemme}
%\newtheorem{corollaire}{Corollaire}
%\newtheorem*{remarque*}{Remarque}
%\newtheorem*{miniexercice}{Mini-exercices}
%\newtheorem{definition}{Définition}

% Commande tikz
\usetikzlibrary{calc}
\usetikzlibrary{patterns,arrows}
\usetikzlibrary{matrix}
\usetikzlibrary{fadings} 

%definition d'un terme
\newcommand{\defi}[1]{{\color{myorange}\textbf{\emph{#1}}}}
\newcommand{\evidence}[1]{{\color{blue}\textbf{\emph{#1}}}}
\newcommand{\assertion}[1]{\emph{\og#1\fg}}  % pour chapitre logique
%\renewcommand{\contentsname}{Sommaire}
\renewcommand{\contentsname}{}
\setcounter{tocdepth}{2}



%------ Figures ------

\def\myscale{1} % par défaut 
\newcommand{\myfigure}[2]{  % entrée : echelle, fichier figure
\def\myscale{#1}
\begin{center}
\footnotesize
{#2}
\end{center}}


%------ Encadrement ------

\usepackage{fancybox}


\newcommand{\mybox}[1]{
\setlength{\fboxsep}{7pt}
\begin{center}
\shadowbox{#1}
\end{center}}

\newcommand{\myboxinline}[1]{
\setlength{\fboxsep}{5pt}
\raisebox{-10pt}{
\shadowbox{#1}
}
}

%--------------- Commande beamer---------------
\newcommand{\beameronly}[1]{#1} % permet de mettre des pause dans beamer pas dans poly


\setbeamertemplate{navigation symbols}{}
\setbeamertemplate{footline}  % tiré du fichier beamerouterinfolines.sty
{
  \leavevmode%
  \hbox{%
  \begin{beamercolorbox}[wd=.333333\paperwidth,ht=2.25ex,dp=1ex,center]{author in head/foot}%
    % \usebeamerfont{author in head/foot}\insertshortauthor%~~(\insertshortinstitute)
    \usebeamerfont{section in head/foot}{\bf\insertshorttitle}
  \end{beamercolorbox}%
  \begin{beamercolorbox}[wd=.333333\paperwidth,ht=2.25ex,dp=1ex,center]{title in head/foot}%
    \usebeamerfont{section in head/foot}{\bf\insertsectionhead}
  \end{beamercolorbox}%
  \begin{beamercolorbox}[wd=.333333\paperwidth,ht=2.25ex,dp=1ex,right]{date in head/foot}%
    % \usebeamerfont{date in head/foot}\insertshortdate{}\hspace*{2em}
    \insertframenumber{} / \inserttotalframenumber\hspace*{2ex} 
  \end{beamercolorbox}}%
  \vskip0pt%
}


\definecolor{mygrey}{rgb}{0.5,0.5,0.5}
\setlength{\parindent}{0cm}
%\DeclareTextFontCommand{\helvetica}{\fontfamily{phv}\selectfont}

% background beamer
\definecolor{couleurhaut}{rgb}{0.85,0.9,1}  % creme
\definecolor{couleurmilieu}{rgb}{1,1,1}  % vert pale
\definecolor{couleurbas}{rgb}{0.85,0.9,1}  % blanc
\setbeamertemplate{background canvas}[vertical shading]%
[top=couleurhaut,middle=couleurmilieu,midpoint=0.4,bottom=couleurbas] 
%[top=fondtitre!05,bottom=fondtitre!60]



\makeatletter
\setbeamertemplate{theorem begin}
{%
  \begin{\inserttheoremblockenv}
  {%
    \inserttheoremheadfont
    \inserttheoremname
    \inserttheoremnumber
    \ifx\inserttheoremaddition\@empty\else\ (\inserttheoremaddition)\fi%
    \inserttheorempunctuation
  }%
}
\setbeamertemplate{theorem end}{\end{\inserttheoremblockenv}}

\newenvironment{theoreme}[1][]{%
   \setbeamercolor{block title}{fg=structure,bg=structure!40}
   \setbeamercolor{block body}{fg=black,bg=structure!10}
   \begin{block}{{\bf Th\'eor\`eme }#1}
}{%
   \end{block}%
}


\newenvironment{proposition}[1][]{%
   \setbeamercolor{block title}{fg=structure,bg=structure!40}
   \setbeamercolor{block body}{fg=black,bg=structure!10}
   \begin{block}{{\bf Proposition }#1}
}{%
   \end{block}%
}

\newenvironment{corollaire}[1][]{%
   \setbeamercolor{block title}{fg=structure,bg=structure!40}
   \setbeamercolor{block body}{fg=black,bg=structure!10}
   \begin{block}{{\bf Corollaire }#1}
}{%
   \end{block}%
}

\newenvironment{mydefinition}[1][]{%
   \setbeamercolor{block title}{fg=structure,bg=structure!40}
   \setbeamercolor{block body}{fg=black,bg=structure!10}
   \begin{block}{{\bf Définition} #1}
}{%
   \end{block}%
}

\newenvironment{lemme}[0]{%
   \setbeamercolor{block title}{fg=structure,bg=structure!40}
   \setbeamercolor{block body}{fg=black,bg=structure!10}
   \begin{block}{\bf Lemme}
}{%
   \end{block}%
}

\newenvironment{remarque}[1][]{%
   \setbeamercolor{block title}{fg=black,bg=structure!20}
   \setbeamercolor{block body}{fg=black,bg=structure!5}
   \begin{block}{Remarque #1}
}{%
   \end{block}%
}


\newenvironment{exemple}[1][]{%
   \setbeamercolor{block title}{fg=black,bg=structure!20}
   \setbeamercolor{block body}{fg=black,bg=structure!5}
   \begin{block}{{\bf Exemple }#1}
}{%
   \end{block}%
}


\newenvironment{miniexercice}[0]{%
   \setbeamercolor{block title}{fg=structure,bg=structure!20}
   \setbeamercolor{block body}{fg=black,bg=structure!5}
   \begin{block}{Mini-exercices}
}{%
   \end{block}%
}


\newenvironment{tp}[0]{%
   \setbeamercolor{block title}{fg=structure,bg=structure!40}
   \setbeamercolor{block body}{fg=black,bg=structure!10}
   \begin{block}{\bf Travaux pratiques}
}{%
   \end{block}%
}
\newenvironment{exercicecours}[1][]{%
   \setbeamercolor{block title}{fg=structure,bg=structure!40}
   \setbeamercolor{block body}{fg=black,bg=structure!10}
   \begin{block}{{\bf Exercice }#1}
}{%
   \end{block}%
}
\newenvironment{algo}[1][]{%
   \setbeamercolor{block title}{fg=structure,bg=structure!40}
   \setbeamercolor{block body}{fg=black,bg=structure!10}
   \begin{block}{{\bf Algorithme}\hfill{\color{gray}\texttt{#1}}}
}{%
   \end{block}%
}


\setbeamertemplate{proof begin}{
   \setbeamercolor{block title}{fg=black,bg=structure!20}
   \setbeamercolor{block body}{fg=black,bg=structure!5}
   \begin{block}{{\footnotesize Démonstration}}
   \footnotesize
   \smallskip}
\setbeamertemplate{proof end}{%
   \end{block}}
\setbeamertemplate{qed symbol}{\openbox}


\makeatother
\usecolortheme[RGB={0,45,179}]{structure}

% Commande spécifique à ce chapitre
\newcounter{saveenumi}

%%%%%%%%%%%%%%%%%%%%%%%%%%%%%%%%%%%%%%%%%%%%%%%%%%%%%%%%%%%%%
%%%%%%%%%%%%%%%%%%%%%%%%%%%%%%%%%%%%%%%%%%%%%%%%%%%%%%%%%%%%%


\begin{document}



\title{{\bf Intégrales}}
\subtitle{Intégration des fractions rationnelles}

\begin{frame}
  
  \debutmontitre

  \pause

{\footnotesize
\hfill
\setbeamercovered{transparent=50}
\begin{minipage}{0.6\textwidth}
  \begin{itemize}
    \item<3-> Trois situations de base
    \item<4-> Intégration des éléments simples
    \item<5-> Intégration des fonctions trigonométriques
  \end{itemize}
\end{minipage}
}

\end{frame}

\setcounter{framenumber}{0}


%%%%%%%%%%%%%%%%%%%%%%%%%%%%%%%%%%%%%%%%%%%%%%%%%%%%%%%%%%%%%%%%


\section*{Motivation}


\begin{frame}

\begin{itemize}
  \item $f(x)=a_0+a_1x+a_2x^2+\cdots+ a_n x^n$

\pause

 $\int f(x)\; dx = a_0x+a_1\frac{x^2}{2}+a_2\frac{x^3}{3}+\cdots+a_n\frac{x^{n+1}}{n+1}+c$

\pause

  \item $f(t)=\sqrt{a^2\cos^2 t+ b^2 \sin^2 t}$

\pause

 $\int_0^{2\pi} f(t) \; dt = ?$
\pause
  \item Longueur de l'ellipse $(a\cos t, b\sin t)$

\myfigure{1.2}{
\tikzinput{fig_int09} 
} 

\pause
  \item Fonctions que l'on saura intégrer : les fractions rationnelles

\end{itemize}

\end{frame}




%---------------------------------------------------------------
\section*{Trois situations de base}


\begin{frame}

$$f(x)=\frac{\alpha x + \beta}{a x^2+b x+c}$$
\medskip

\pause

\textbf{Premier cas.} $a x^2+b x+c$ a deux racines réelles distinctes $x_1,x_ 2\in \Rr$

\pause

$$f(x)=\frac{\alpha x + \beta}{a(x - x_1)(x - x_2)}
\pause
=\frac{A}{x - x_1}+\frac{B}{x -x_2}$$
\pause
$$\int f(x)\;dx = A \ln|x - x_1|+B\ln|x -x_2|+c$$ 
\pause
\medskip

\textbf{Deuxième cas.} $a x^2+b x+c$ a une racine double $x_0 \in \Rr$

\pause

$$f(x)=\frac{\alpha x + \beta}{a(x -x_0)^2}
\pause
=\frac{A}{(x - x_0)^2}+\frac{B}{x - x_0}$$
\pause
$$\int f(x)\;dx = -\frac{A}{x - x_0} + B\ln|x - x_0|+c$$ 

\end{frame}




\begin{frame}

\textbf{Troisième cas.} Le dénominateur $a x^2+b x+c$ n'a pas de racine réelle
\pause
\begin{exemple}
$$f(x)=\frac{x+1}{2x^2+x+1}$$

\pause

\begin{enumerate}
  \item Faire apparaître une fraction du type $\frac{u'}{u}$ 
\pause
$$f(x) = \frac{(4x+1)\frac14-\frac14+1}{2x^2+x+1} 
\pause
= \tfrac14 \cdot \frac{4x+1}{2x^2+x+1} + \tfrac34 \cdot \frac{1}{2x^2+x+1}$$

\pause

  \item On intègre en $\ln|u|$
\pause
$$\int \frac{4x+1}{2x^2+x+1} \; dx \pause= \int \frac{u'(x)}{u(x)} \; dx \pause = \ln \big| 2x^2+x+1 \big|+c$$

\setcounter{saveenumi}{\theenumi}

\end{enumerate}
\end{exemple}
\end{frame}


\begin{frame}

\begin{exemple}
\hfil $f(x) = \frac{x+1}{2x^2+x+1}
= \frac14 \cdot \frac{4x+1}{2x^2+x+1} + \frac34 \cdot \frac{1}{2x^2+x+1}$

\pause
\medskip

\begin{enumerate}
\setcounter{enumi}{\thesaveenumi}
  \item \'Ecrire $\frac{1}{2x^2+x+1}$ sous la forme $\frac{1}{u^2+1}$ \pause (dont une primitive est $\arctan u$)

\pause
\medskip
\hfil $\frac{1}{2x^2+x+1} \pause=  \frac{1}{2(x+\frac 14)^2-\frac18+1}
\pause=\frac{1}{2(x+\frac 14)^2+\frac78}
\pause= \frac87 \cdot \frac{1}{\big(\frac{4}{\sqrt7}(x+\frac 14)\big)^2+1}$

\medskip
\pause

On pose  $u= \frac{4}{\sqrt7}(x+\frac 14)$ \pause(et donc $du = \frac{4}{\sqrt7} dx$) 
\pause

\hfil $\int \frac{dx}{2x^2+x+1} 
\pause= \int \frac87 \frac{dx}{\big(\frac{4}{\sqrt7}(x+\frac 14)\big)^2+1}
\pause= \frac 87 \int \frac{du}{u^2+1} \cdot \frac{\sqrt7}{4}$

\pause
\hfil $\quad = \frac{2}{\sqrt7}\arctan u+ c 
\pause= \frac{2}{\sqrt7}\arctan \left(\frac{4}{\sqrt7}\big(x+\frac 14\big)\right) + c$

\medskip
\pause

  \item $\int f(x)\; dx \pause= \frac14\ln \big(2x^2+x+1\big) 
+\pause \frac{3}{2\sqrt7}\arctan \left(\frac{4}{\sqrt7}\big(x+\frac 14\big)\right)+c$

\end{enumerate}
\end{exemple}
\end{frame}





%---------------------------------------------------------------
\section*{Intégration des éléments simples}


\begin{frame}
Une fraction rationnelle $\frac{P(x)}{Q(x)}$ \pause se décompose comme somme 
d'un polynôme \pause et d'éléments simples \pause $\frac{\gamma}{(x - x_0)^k}$ \pause ou $\frac{\alpha x+\beta}{(a x^2+b x+c)^k}$

\pause
\begin{enumerate}
  \item Intégration de l'élément simple $\frac{\gamma}{(x - x_0)^k}$
\pause
  \begin{itemize}
     \item Si $k=1$ alors $\int \frac{\gamma \; dx}{x - x_0} = \gamma \ln|x - x_0|+c$
\pause
     \item Si $k\ge 2$, $\int \frac{\gamma \; dx}{(x - x_0)^k} = \gamma \int (x - x_0)^{-k} \; dx
= \frac{\gamma}{-k+1}(x - x_0)^{-k+1}+c$
  \end{itemize}

\pause
\medskip 

  \item Intégration de l'élément simple 

\hfil $\frac{\alpha x+\beta}{(a x^2+b x+c)^k} 
\pause = \gamma \frac{2a x+b}{(a x^2+b x+c)^k} + \delta \frac{1}{(a x^2+b x+c)^k}$

\pause
  \begin{itemize}
     \item $\int \frac{2a x+b}{(a x^2+b x+c)^k} \; dx \pause= \int \frac{u'(x)}{u(x)^k} \; dx \pause= \frac{1}{-k+1}u(x)^{-k+1}+c
\pause= \frac{1}{-k+1}(a x^2+b x+c)^{-k+1}+c$
 
 \pause   
     \item Si $k=1$, calcul de $\int \frac{1}{a x^2+b x+c}\; dx$.  \pause Du type $\int \frac{du}{u^2+1}=\arctan u + c$

\pause
     \item Si $k\ge 2$, calcul de $\int \frac{1}{(a x^2+b x+c)^k} \; dx$. \pause Du type $I_k =\int \frac{du}{(u^2+1)^k}$.
\pause

Une IPP permet de passer de $I_k$ à $I_{k-1}$
  \end{itemize}
\end{enumerate}
\end{frame}


\begin{frame}

\begin{exemple}[Calcul de $I_2 = \int \frac{du}{(u^2+1)^2}$]

\pause

Intégration par parties à partir de $I_1=\int \frac{du}{u^2+1}$


\pause 
$f=\frac{1}{u^2+1}$ et $g'=1$ 
\pause 
(avec $f'=-\frac{2u}{(u^2+1)^2}$ et $g=u$) 

\pause

$$\begin{array}{rcl}
I_1 
  & = & \int \frac{du}{u^2+1} =\pause \left[ \frac{u}{u^2+1} \right] + \int \frac{2u^2 \; du}{(u^2+1)^2}
        \pause = \left[ \frac{u}{u^2+1} \right] + 2 \int  \frac{u^2+1 - 1}{(u^2+1)^2}du \\
\pause
 & = & \left[ \frac{u}{u^2+1} \right] + 2 \int \frac{du}{u^2+1} - 2 \int \frac{du}{(u^2+1)^2}
    \pause = \left[ \frac{u}{u^2+1} \right] + 2I_1 - 2I_2  
\end{array}
$$

\pause
D'où $\displaystyle I_2 = \tfrac12 I_1 + \tfrac12\frac{u}{u^2+1} + c$

\pause
\medskip

Mais $I_1=\arctan u$ donc 

\pause

$$I_2 = \int \frac{du}{(u^2+1)^2} =  \tfrac12 \arctan u + \tfrac12\frac{u}{u^2+1} + c$$ 
\end{exemple}
\end{frame}





%---------------------------------------------------------------
\section*{Intégration des fonctions trigonométriques}


\begin{frame}
$$\int P(\cos x,\sin x)\;dx \qquad\pause \int \frac{P(\cos x,\sin x)}{Q(\cos x, \sin x)}\;dx$$

\pause

\textbf{Les règles de Bioche} \pause \quad $\omega(x) = f(x)\;dx$

\pause

\centerline{$\omega(-x)=-f(-x)\;dx$ \pause et $\omega (\pi-x)=-f(\pi-x)\;dx$}

\pause
\medskip

\begin{itemize}
  \item Si $\omega(-x)=\omega(x)$ \pause alors poser $u=\cos x$
\pause
  \item Si $\omega(\pi-x)=\omega(x)$ alors poser $u=\sin x$
\pause
  \item Si $\omega(\pi + x)=\omega(x)$ alors poser $u=\tan x$
\end{itemize}

\pause
\medskip

\begin{exemple}[Calcul de la primitive $\int \frac{\cos x \; dx}{2-\cos^2 x}$]
\pause
$\omega(x)= \frac{\cos x \; dx}{2-\cos^2 x}$ 
\pause
\hfill
$\omega(\pi-x)=\frac{\cos(\pi-x) \; d(\pi-x)}{2-\cos^2 (\pi-x)} = \frac{(-\cos x) \; (-dx)}{2-\cos^2 x} 
= \omega(x)$ 

\pause
\medskip

Le changement de variable  $u = \sin x$
\pause
\hfil $du= \cos x \; dx$

\pause
\medskip
$
\int \frac{\cos x \; dx}{2-\cos^2 x}
\pause
= \int \frac{\cos x \; dx}{1+\sin^2 x}
\pause
= \int \frac{du}{1+u^2} 
\pause
= \big[ \arctan u \big] 
\pause
= \arctan (\sin x) + c
$
\end{exemple}
\end{frame}


\begin{frame}

\textbf{Le changement de variable $\displaystyle t=\tan \frac x2$}

\pause

\mybox{$\displaystyle
    \cos x = \frac {1-t^2}{1+t^2} \quad \pause \sin x = \frac{2t}{1+t^2} \quad \pause
\tan x = \frac{2t}{1-t^2} \quad\pause
\quad dx=\dfrac{2\;dt}{1+t^2}
\pause
$}
\pause
\begin{exemple}

\uncover<8->{Changement de variable $t=\tan \frac{x}{2}$} \uncover<9->{ : bijection de $[-\frac\pi2,0]$ vers $[-1,0]$}



$$\int_{-\frac\pi2}^0 \frac{dx}{1-\sin x} \pause\pause\pause
= \int_{-1}^0 \frac{\frac{2\;dt}{1+t^2}}{1-\frac{2t}{1+t^2}} \pause
= 2\int_{-1}^0 \frac{dt}{1+t^2-2t} $$
\pause
$$
= 2\int_{-1}^0 \frac{dt}{(1-t)^2} \pause
= 2 \left[\frac{1}{1-t}\right]_{-1}^0\pause
= 2\big(1-\frac12\big)\pause
= 1$$
 
\end{exemple}

\end{frame}



%%%%%%%%%%%%%%%%%%%%%%%%%%%%%%%%%%%%%%%%%%%%%%%%%%%%%%%%%%%%%%%%
\section*{Mini-exercices}


\begin{frame}
\begin{miniexercice}
\begin{enumerate}
  \item Calculer les primitives $\int \frac{4x+5}{x^2+x-2}\; dx$,
$\int \frac{6-x}{x^2-4x+4}\; dx$, $\int \frac{2x-4}{(x-2)^2+1}\; dx$, $\int \frac{1}{(x-2)^2+1}\; dx$.

  \item Calculer les primitives $I_k = \int \frac{dx}{(x-1)^k}$ pour tout $k \ge 1$.
Idem avec $J_k = \int \frac{x\; dx}{(x^2+1)^k}$.

  \item Calculer les intégrales suivantes :
$\int_0^1 \frac{dx}{x^2+x+1}$, $\int_0^1 \frac{x \; dx}{x^2+x+1}$, $\int_0^1\frac{x\; dx}{(x^2+x+1)^2}$,  
$\int_0^1\frac{dx}{(x^2+x+1)^2}$.

  \item Calculer les intégrales suivantes :
$\int_{-\frac\pi2}^{\frac\pi2} \sin^2 x \cos^3 x \; dx$,
$\int_{0}^{\frac\pi2} \cos^4 x \; dx$,
$\int_0^{2\pi} \frac{dx}{2+\sin x}$.
\end{enumerate}
\end{miniexercice}
\end{frame}


\end{document}