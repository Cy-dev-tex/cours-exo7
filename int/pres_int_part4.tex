
%%%%%%%%%%%%%%%%%% PREAMBULE %%%%%%%%%%%%%%%%%%

\documentclass[aspectratio=169,utf8]{beamer}
%\documentclass[aspectratio=169,handout]{beamer}

\usetheme{Boadilla}
%\usecolortheme{seahorse}
\usecolortheme[RGB={245,66,24}]{structure}
\useoutertheme{infolines}

% packages
\usepackage{amsfonts,amsmath,amssymb,amsthm}
\usepackage[utf8]{inputenc}
\usepackage[T1]{fontenc}
\usepackage{lmodern}

\usepackage[francais]{babel}
\usepackage{fancybox}
\usepackage{graphicx}

\usepackage{float}
\usepackage{xfrac}

%\usepackage[usenames, x11names]{xcolor}
\usepackage{tikz}
\usepackage{pgfplots}
\usepackage{datetime}



%-----  Package unités -----
\usepackage{siunitx}
\sisetup{locale = FR,detect-all,per-mode = symbol}

%\usepackage{mathptmx}
%\usepackage{fouriernc}
%\usepackage{newcent}
%\usepackage[mathcal,mathbf]{euler}

%\usepackage{palatino}
%\usepackage{newcent}
% \usepackage[mathcal,mathbf]{euler}



% \usepackage{hyperref}
% \hypersetup{colorlinks=true, linkcolor=blue, urlcolor=blue,
% pdftitle={Exo7 - Exercices de mathématiques}, pdfauthor={Exo7}}


%section
% \usepackage{sectsty}
% \allsectionsfont{\bf}
%\sectionfont{\color{Tomato3}\upshape\selectfont}
%\subsectionfont{\color{Tomato4}\upshape\selectfont}

%----- Ensembles : entiers, reels, complexes -----
\newcommand{\Nn}{\mathbb{N}} \newcommand{\N}{\mathbb{N}}
\newcommand{\Zz}{\mathbb{Z}} \newcommand{\Z}{\mathbb{Z}}
\newcommand{\Qq}{\mathbb{Q}} \newcommand{\Q}{\mathbb{Q}}
\newcommand{\Rr}{\mathbb{R}} \newcommand{\R}{\mathbb{R}}
\newcommand{\Cc}{\mathbb{C}} 
\newcommand{\Kk}{\mathbb{K}} \newcommand{\K}{\mathbb{K}}

%----- Modifications de symboles -----
\renewcommand{\epsilon}{\varepsilon}
\renewcommand{\Re}{\mathop{\text{Re}}\nolimits}
\renewcommand{\Im}{\mathop{\text{Im}}\nolimits}
%\newcommand{\llbracket}{\left[\kern-0.15em\left[}
%\newcommand{\rrbracket}{\right]\kern-0.15em\right]}

\renewcommand{\ge}{\geqslant}
\renewcommand{\geq}{\geqslant}
\renewcommand{\le}{\leqslant}
\renewcommand{\leq}{\leqslant}
\renewcommand{\epsilon}{\varepsilon}

%----- Fonctions usuelles -----
\newcommand{\ch}{\mathop{\text{ch}}\nolimits}
\newcommand{\sh}{\mathop{\text{sh}}\nolimits}
\renewcommand{\tanh}{\mathop{\text{th}}\nolimits}
\newcommand{\cotan}{\mathop{\text{cotan}}\nolimits}
\newcommand{\Arcsin}{\mathop{\text{arcsin}}\nolimits}
\newcommand{\Arccos}{\mathop{\text{arccos}}\nolimits}
\newcommand{\Arctan}{\mathop{\text{arctan}}\nolimits}
\newcommand{\Argsh}{\mathop{\text{argsh}}\nolimits}
\newcommand{\Argch}{\mathop{\text{argch}}\nolimits}
\newcommand{\Argth}{\mathop{\text{argth}}\nolimits}
\newcommand{\pgcd}{\mathop{\text{pgcd}}\nolimits} 


%----- Commandes divers ------
\newcommand{\ii}{\mathrm{i}}
\newcommand{\dd}{\text{d}}
\newcommand{\id}{\mathop{\text{id}}\nolimits}
\newcommand{\Ker}{\mathop{\text{Ker}}\nolimits}
\newcommand{\Card}{\mathop{\text{Card}}\nolimits}
\newcommand{\Vect}{\mathop{\text{Vect}}\nolimits}
\newcommand{\Mat}{\mathop{\text{Mat}}\nolimits}
\newcommand{\rg}{\mathop{\text{rg}}\nolimits}
\newcommand{\tr}{\mathop{\text{tr}}\nolimits}


%----- Structure des exercices ------

\newtheoremstyle{styleexo}% name
{2ex}% Space above
{3ex}% Space below
{}% Body font
{}% Indent amount 1
{\bfseries} % Theorem head font
{}% Punctuation after theorem head
{\newline}% Space after theorem head 2
{}% Theorem head spec (can be left empty, meaning ‘normal’)

%\theoremstyle{styleexo}
\newtheorem{exo}{Exercice}
\newtheorem{ind}{Indications}
\newtheorem{cor}{Correction}


\newcommand{\exercice}[1]{} \newcommand{\finexercice}{}
%\newcommand{\exercice}[1]{{\tiny\texttt{#1}}\vspace{-2ex}} % pour afficher le numero absolu, l'auteur...
\newcommand{\enonce}{\begin{exo}} \newcommand{\finenonce}{\end{exo}}
\newcommand{\indication}{\begin{ind}} \newcommand{\finindication}{\end{ind}}
\newcommand{\correction}{\begin{cor}} \newcommand{\fincorrection}{\end{cor}}

\newcommand{\noindication}{\stepcounter{ind}}
\newcommand{\nocorrection}{\stepcounter{cor}}

\newcommand{\fiche}[1]{} \newcommand{\finfiche}{}
\newcommand{\titre}[1]{\centerline{\large \bf #1}}
\newcommand{\addcommand}[1]{}
\newcommand{\video}[1]{}

% Marge
\newcommand{\mymargin}[1]{\marginpar{{\small #1}}}

\def\noqed{\renewcommand{\qedsymbol}{}}


%----- Presentation ------
\setlength{\parindent}{0cm}

%\newcommand{\ExoSept}{\href{http://exo7.emath.fr}{\textbf{\textsf{Exo7}}}}

\definecolor{myred}{rgb}{0.93,0.26,0}
\definecolor{myorange}{rgb}{0.97,0.58,0}
\definecolor{myyellow}{rgb}{1,0.86,0}

\newcommand{\LogoExoSept}[1]{  % input : echelle
{\usefont{U}{cmss}{bx}{n}
\begin{tikzpicture}[scale=0.1*#1,transform shape]
  \fill[color=myorange] (0,0)--(4,0)--(4,-4)--(0,-4)--cycle;
  \fill[color=myred] (0,0)--(0,3)--(-3,3)--(-3,0)--cycle;
  \fill[color=myyellow] (4,0)--(7,4)--(3,7)--(0,3)--cycle;
  \node[scale=5] at (3.5,3.5) {Exo7};
\end{tikzpicture}}
}


\newcommand{\debutmontitre}{
  \author{} \date{} 
  \thispagestyle{empty}
  \hspace*{-10ex}
  \begin{minipage}{\textwidth}
    \titlepage  
  \vspace*{-2.5cm}
  \begin{center}
    \LogoExoSept{2.5}
  \end{center}
  \end{minipage}

  \vspace*{-0cm}
  
  % Astuce pour que le background ne soit pas discrétisé lors de la conversion pdf -> png
\begin{tikzpicture}
        \fill[opacity=0,green!60!black] (0,0)--++(0,0)--++(0,0)--++(0,0)--cycle; 
\end{tikzpicture}

% toc S'affiche trop tot :
% \tableofcontents[hideallsubsections, pausesections]
}

\newcommand{\finmontitre}{
  \end{frame}
  \setcounter{framenumber}{0}
} % ne marche pas pour une raison obscure

%----- Commandes supplementaires ------

% \usepackage[landscape]{geometry}
% \geometry{top=1cm, bottom=3cm, left=2cm, right=10cm, marginparsep=1cm
% }
% \usepackage[a4paper]{geometry}
% \geometry{top=2cm, bottom=2cm, left=2cm, right=2cm, marginparsep=1cm
% }

%\usepackage{standalone}


% New command Arnaud -- november 2011
\setbeamersize{text margin left=24ex}
% si vous modifier cette valeur il faut aussi
% modifier le decalage du titre pour compenser
% (ex : ici =+10ex, titre =-5ex

\theoremstyle{definition}
%\newtheorem{proposition}{Proposition}
%\newtheorem{exemple}{Exemple}
%\newtheorem{theoreme}{Théorème}
%\newtheorem{lemme}{Lemme}
%\newtheorem{corollaire}{Corollaire}
%\newtheorem*{remarque*}{Remarque}
%\newtheorem*{miniexercice}{Mini-exercices}
%\newtheorem{definition}{Définition}

% Commande tikz
\usetikzlibrary{calc}
\usetikzlibrary{patterns,arrows}
\usetikzlibrary{matrix}
\usetikzlibrary{fadings} 

%definition d'un terme
\newcommand{\defi}[1]{{\color{myorange}\textbf{\emph{#1}}}}
\newcommand{\evidence}[1]{{\color{blue}\textbf{\emph{#1}}}}
\newcommand{\assertion}[1]{\emph{\og#1\fg}}  % pour chapitre logique
%\renewcommand{\contentsname}{Sommaire}
\renewcommand{\contentsname}{}
\setcounter{tocdepth}{2}



%------ Figures ------

\def\myscale{1} % par défaut 
\newcommand{\myfigure}[2]{  % entrée : echelle, fichier figure
\def\myscale{#1}
\begin{center}
\footnotesize
{#2}
\end{center}}


%------ Encadrement ------

\usepackage{fancybox}


\newcommand{\mybox}[1]{
\setlength{\fboxsep}{7pt}
\begin{center}
\shadowbox{#1}
\end{center}}

\newcommand{\myboxinline}[1]{
\setlength{\fboxsep}{5pt}
\raisebox{-10pt}{
\shadowbox{#1}
}
}

%--------------- Commande beamer---------------
\newcommand{\beameronly}[1]{#1} % permet de mettre des pause dans beamer pas dans poly


\setbeamertemplate{navigation symbols}{}
\setbeamertemplate{footline}  % tiré du fichier beamerouterinfolines.sty
{
  \leavevmode%
  \hbox{%
  \begin{beamercolorbox}[wd=.333333\paperwidth,ht=2.25ex,dp=1ex,center]{author in head/foot}%
    % \usebeamerfont{author in head/foot}\insertshortauthor%~~(\insertshortinstitute)
    \usebeamerfont{section in head/foot}{\bf\insertshorttitle}
  \end{beamercolorbox}%
  \begin{beamercolorbox}[wd=.333333\paperwidth,ht=2.25ex,dp=1ex,center]{title in head/foot}%
    \usebeamerfont{section in head/foot}{\bf\insertsectionhead}
  \end{beamercolorbox}%
  \begin{beamercolorbox}[wd=.333333\paperwidth,ht=2.25ex,dp=1ex,right]{date in head/foot}%
    % \usebeamerfont{date in head/foot}\insertshortdate{}\hspace*{2em}
    \insertframenumber{} / \inserttotalframenumber\hspace*{2ex} 
  \end{beamercolorbox}}%
  \vskip0pt%
}


\definecolor{mygrey}{rgb}{0.5,0.5,0.5}
\setlength{\parindent}{0cm}
%\DeclareTextFontCommand{\helvetica}{\fontfamily{phv}\selectfont}

% background beamer
\definecolor{couleurhaut}{rgb}{0.85,0.9,1}  % creme
\definecolor{couleurmilieu}{rgb}{1,1,1}  % vert pale
\definecolor{couleurbas}{rgb}{0.85,0.9,1}  % blanc
\setbeamertemplate{background canvas}[vertical shading]%
[top=couleurhaut,middle=couleurmilieu,midpoint=0.4,bottom=couleurbas] 
%[top=fondtitre!05,bottom=fondtitre!60]



\makeatletter
\setbeamertemplate{theorem begin}
{%
  \begin{\inserttheoremblockenv}
  {%
    \inserttheoremheadfont
    \inserttheoremname
    \inserttheoremnumber
    \ifx\inserttheoremaddition\@empty\else\ (\inserttheoremaddition)\fi%
    \inserttheorempunctuation
  }%
}
\setbeamertemplate{theorem end}{\end{\inserttheoremblockenv}}

\newenvironment{theoreme}[1][]{%
   \setbeamercolor{block title}{fg=structure,bg=structure!40}
   \setbeamercolor{block body}{fg=black,bg=structure!10}
   \begin{block}{{\bf Th\'eor\`eme }#1}
}{%
   \end{block}%
}


\newenvironment{proposition}[1][]{%
   \setbeamercolor{block title}{fg=structure,bg=structure!40}
   \setbeamercolor{block body}{fg=black,bg=structure!10}
   \begin{block}{{\bf Proposition }#1}
}{%
   \end{block}%
}

\newenvironment{corollaire}[1][]{%
   \setbeamercolor{block title}{fg=structure,bg=structure!40}
   \setbeamercolor{block body}{fg=black,bg=structure!10}
   \begin{block}{{\bf Corollaire }#1}
}{%
   \end{block}%
}

\newenvironment{mydefinition}[1][]{%
   \setbeamercolor{block title}{fg=structure,bg=structure!40}
   \setbeamercolor{block body}{fg=black,bg=structure!10}
   \begin{block}{{\bf Définition} #1}
}{%
   \end{block}%
}

\newenvironment{lemme}[0]{%
   \setbeamercolor{block title}{fg=structure,bg=structure!40}
   \setbeamercolor{block body}{fg=black,bg=structure!10}
   \begin{block}{\bf Lemme}
}{%
   \end{block}%
}

\newenvironment{remarque}[1][]{%
   \setbeamercolor{block title}{fg=black,bg=structure!20}
   \setbeamercolor{block body}{fg=black,bg=structure!5}
   \begin{block}{Remarque #1}
}{%
   \end{block}%
}


\newenvironment{exemple}[1][]{%
   \setbeamercolor{block title}{fg=black,bg=structure!20}
   \setbeamercolor{block body}{fg=black,bg=structure!5}
   \begin{block}{{\bf Exemple }#1}
}{%
   \end{block}%
}


\newenvironment{miniexercice}[0]{%
   \setbeamercolor{block title}{fg=structure,bg=structure!20}
   \setbeamercolor{block body}{fg=black,bg=structure!5}
   \begin{block}{Mini-exercices}
}{%
   \end{block}%
}


\newenvironment{tp}[0]{%
   \setbeamercolor{block title}{fg=structure,bg=structure!40}
   \setbeamercolor{block body}{fg=black,bg=structure!10}
   \begin{block}{\bf Travaux pratiques}
}{%
   \end{block}%
}
\newenvironment{exercicecours}[1][]{%
   \setbeamercolor{block title}{fg=structure,bg=structure!40}
   \setbeamercolor{block body}{fg=black,bg=structure!10}
   \begin{block}{{\bf Exercice }#1}
}{%
   \end{block}%
}
\newenvironment{algo}[1][]{%
   \setbeamercolor{block title}{fg=structure,bg=structure!40}
   \setbeamercolor{block body}{fg=black,bg=structure!10}
   \begin{block}{{\bf Algorithme}\hfill{\color{gray}\texttt{#1}}}
}{%
   \end{block}%
}


\setbeamertemplate{proof begin}{
   \setbeamercolor{block title}{fg=black,bg=structure!20}
   \setbeamercolor{block body}{fg=black,bg=structure!5}
   \begin{block}{{\footnotesize Démonstration}}
   \footnotesize
   \smallskip}
\setbeamertemplate{proof end}{%
   \end{block}}
\setbeamertemplate{qed symbol}{\openbox}


\makeatother
\usecolortheme[RGB={0,45,179}]{structure}

%%%%%%%%%%%%%%%%%%%%%%%%%%%%%%%%%%%%%%%%%%%%%%%%%%%%%%%%%%%%%
%%%%%%%%%%%%%%%%%%%%%%%%%%%%%%%%%%%%%%%%%%%%%%%%%%%%%%%%%%%%%



\begin{document}



\title{{\bf Intégrales}}
\subtitle{Intégration par parties -- Changement de variable}

\begin{frame}
  
  \debutmontitre

  \pause

{\footnotesize
\hfill
\setbeamercovered{transparent=50}
\begin{minipage}{0.6\textwidth}
  \begin{itemize}
    \item<3-> Intégration par parties
    \item<4-> Changement de variable
  \end{itemize}
\end{minipage}
}

\end{frame}

\setcounter{framenumber}{0}


%%%%%%%%%%%%%%%%%%%%%%%%%%%%%%%%%%%%%%%%%%%%%%%%%%%%%%%%%%%%%%%%



%---------------------------------------------------------------
\section*{Intégration par parties}


\begin{frame}
\begin{theoreme}
Soient $u$ et $v$ deux fonctions de classe $\mathcal{C}^1$ sur un intervalle $[a,b]$

\mybox{$\displaystyle\int_a^b u(x) \, v'(x)\;dx= \big[uv\big]_a^b - \int_a^b u'(x) \, v(x)\;dx$}
\end{theoreme} 

\pause

\textbf{Notation.}  $\big[F\big]_a^b=F(b)-F(a)$

\pause

$$\int u(x)v'(x)\;dx= \big[uv\big] - \int u'(x)v(x)\;dx$$

\pause

\textbf{Notation.}  $\big[F\big]$ désigne la fonction $F+c$

\pause

\begin{proof}
$(uv)'=u'v+uv'$  \pause donc $\int_a^b (u'v+uv')=\int_a^b (uv)' \pause =\big[uv\big]_a^b$

\pause

Ainsi $\int_a^b uv'= \big[uv\big]_a^b - \int_a^b u'v$
\end{proof}

\end{frame}



\begin{frame}
\begin{exemple}[Calcul de $\int_0^1 x e^x \; dx$]

\pause

\begin{minipage}{0.66\textwidth}
$$
\begin{array}{rcl@{\vrule depth 3ex height 1.5ex width 0mm \ }}
\displaystyle \int_0^1 x e^x \; dx & = & \uncover<3->{\displaystyle \int_0^1 u(x)v'(x)\;dx} \\ 
\uncover<7->{ & = &\displaystyle \big[u(x)v(x)\big]_0^1 - \int_0^1 u'(x)v(x)\;dx \\}
\uncover<8->{   & = &\displaystyle \big[x e^x\big]_0^1 \ \  - \  \int_0^1 1\cdot e^x \;dx \\}
\uncover<9->{   & = & \big(1\cdot e^1-0\cdot e^0\big) - \uncover<10->{\pause \big[e^x\big]_0^1} \\}
\uncover<11->{   & = & e -(e^1-e^0) \\}
\uncover<12->{   & = & 1 }
\end{array}
$$  
\end{minipage}
\begin{minipage}{0.32\textwidth}
$$\begin{array}{cc@{\vrule depth 3ex height 1.5ex width 0mm \ }}  
\uncover<4->{\textcolor{blue}{u =x}}       & \uncover<5->{\textcolor{orange}{u' = 1}} \\
\uncover<4->{\textcolor{blue}{v'=e^x}}     & \uncover<6->{\textcolor{orange}{v=e^x}} \\
\end{array}$$ 
\vspace*{3.5cm} 
\end{minipage}
\end{exemple}
\end{frame}



\begin{frame}
\begin{exemple}[Calcul de $\int_1^e x\ln x \; dx$]
\pause
\begin{minipage}{0.66\textwidth}
$$\begin{array}{rcl@{\vrule depth 3ex height 1.5ex width 0mm \ }}
\displaystyle\int_1^e \ln x \cdot  x\; dx 
\uncover<3->{ & =  & \displaystyle\int_1^e  uv'} \uncover<7->{ = \big[uv\big]_1^e - \int_1^e u'v} \\
\uncover<8->{ & = &\displaystyle \big[\ln x \cdot\tfrac{x^2}{2}\big]_1^e -\int_1^e \tfrac 1x \tfrac{x^2}{2} \; dx \\}
\uncover<9->{ & = &\displaystyle \big(\ln e \tfrac{e^2}{2} - \ln 1 \tfrac{1^2}{2} \big) - } \uncover<10->{\tfrac12 \int_1^e x  \; dx \\}
\uncover<11->{ & = & \tfrac{e^2}{2} -\frac12 \left[ \tfrac{x^2}{2} \right]_1^e \\}
\uncover<12->{ & = & \tfrac{e^2}{2} - \tfrac{e^2}{4} + \tfrac{1}{4} = \tfrac{e^2+1}{4}\\  }
\end{array}  $$
\end{minipage}
\begin{minipage}{0.32\textwidth}
$$\begin{array}{cc@{\vrule depth 3ex height 1.5ex width 0mm \ }}  
\uncover<4->{\textcolor{blue}{u = \ln x}} & \uncover<5->{\textcolor{orange}{u' = \frac1x}} \\
\uncover<4->{\textcolor{blue}{v'= x}}     & \uncover<6->{\textcolor{orange}{v= \frac{x^2}{2}}} \\
\end{array}$$ 
\vspace*{3.2cm} 
\end{minipage}
\end{exemple}

\end{frame}


\begin{frame}
\begin{exemple}[Calcul de $\int \arcsin x \; dx$]


\pause
$$\int 1\cdot \arcsin x \; dx  \uncover<5->{= \big[x\arcsin x\big] - \int \frac{x}{\sqrt{1-x^2}} \; dx}$$
$$\uncover<6->{=  \big[x\arcsin x\big] - \big[-\sqrt {1-x^2}\big]}  \uncover<7->{= x\arcsin x+ \sqrt {1-x^2}+c}$$
\hfil\hfil \uncover<3->{$\textcolor{blue}{u=\arcsin x}$, $\textcolor{blue}{v'=1}$ }
\uncover<4->{(donc $\textcolor{orange}{u'=\frac{1}{\sqrt{1-x^2}}}$ et $\textcolor{orange}{v=x}$)}
\end{exemple}

\pause\pause\pause\pause\pause\pause

\begin{exemple}[Calcul de $\int x^2e^x \; dx$]
\pause
\begin{itemize}
  \item $\displaystyle\int x^2e^x \; dx \uncover<11->{= \big[ x^2e^x \big] - 2\int x e^x \; dx }
\qquad \uncover<10->{\textcolor{blue}{u=x^2} \text{ et } \textcolor{blue}{v'=e^x}}$
\pause \pause \pause
  \item $\displaystyle\int x e^x \; dx \pause = \big[x e^x\big] - \int e^x \; dx \pause= (x-1)e^x+c$
\pause
  \item $\displaystyle\int x^2e^x \; dx = (x^2-2x+2) e^x + c$
\end{itemize}
\end{exemple}

\end{frame}




%---------------------------------------------------------------
\section*{Changement de variable}

\begin{frame}
\begin{theoreme}
Soit $f : I \to \Rr$ et $\varphi : J \to I$ une bijection $\mathcal{C}^1$

Pour tout $a,b\in J$ 
\mybox{$\displaystyle\int_{\varphi(a)}^{\varphi(b)} f(x) \; dx \ \  = \ \ \int_a^b f\big(\varphi(t)\big)\cdot\varphi'(t) \; dt$}
\end{theoreme} 

\vfill
\pause

\centerline{$x=\varphi(t) \pause \implies \frac{dx}{dt} = \varphi'(t) \pause \implies dx = \varphi'(t) \; dt$}
\pause
\medskip
\centerline{$\implies f(x) \; dx = f(\varphi(t)) \; \varphi'(t) \; dt$}
\pause
\medskip
\centerline{$\implies \int_{\varphi(a)}^{\varphi(b)} f(x) \; dx = \int_a^b f(\varphi(t)) \; \varphi'(t) \; dt$}

\end{frame}




\begin{frame}
\begin{exemple}
\[
F=\int \tan t \; dt \pause = \int \frac{\sin t}{\cos t} \; dt \;
\]
\pause
\begin{itemize}
  \item Forme $\frac{u'}{u}$ (avec $u=\cos t$ et $u'=-\sin t$), \pause primitive est $\ln|u|$

\pause

Donc $F = \int -\frac{u'}{u} = -\big[\ln |u| \big] = -\ln|u|+c = -\ln|\cos t|+c$


\pause

  \item Changement de variable $x = \varphi(t)= \cos t$, \pause $dx = -\sin t \; dt$
\pause
$$
F = \int \tan t \; dt \pause = \int -\frac{\varphi'(t)}{\varphi(t)} \; dt
$$
\pause
$$f(x)=\frac1x \pause \qquad F = - \int \varphi'(t) f(\varphi(t))\; dt \pause = -\int f(x) \; dx$$
\pause
 $$\qquad\qquad\qquad\qquad=-\int \frac1x \;dx \pause = -\ln|x|+c \; \pause = -\ln|\cos t| + c$$
\end{itemize}
\end{exemple}
\end{frame}



\begin{frame}
\begin{exemple}

\hfil\hfil Calcul de $\displaystyle \int_0^{1/2}\frac{x}{(1-x^2)^{3/2}} \;dx$

\pause

\begin{itemize}
  \item Changement de variable $u=\varphi(x) = 1-x^2$
\pause 
  \item Alors $du = \varphi'(x) \; dx = -2x \; dx$

\uncover<6->{  \item Pour \textcolor{blue}{$x=0$} on a \textcolor{blue}{$u=\varphi(0)=1$}} 

\uncover<9->{  \item Pour \textcolor{orange}{$x=\frac12$} on a \textcolor{orange}{$u=\varphi(\frac{1}{2})=\frac34$}}

\end{itemize}

\pause
$$\int_{\uncover<7->{\textcolor{blue}{0}}}^{\uncover<10->{\raise1ex\hbox{\scriptsize \textcolor{orange}{1/2}}}}\frac{x \; dx}{(1-x^2)^{3/2}} 
\pause= \int_{\uncover<8->{\textcolor{blue}{1}}}^{\uncover<11->{\raise1ex\hbox{\scriptsize \textcolor{orange}{3/4}}}} \frac{-\tfrac12 \; du}{u^{3/2}} 
\pause\pause\pause\pause\pause\pause\pause = -\tfrac12\int_1^{3/4} u^{-3/2}\;du$$
\pause 
$$
= -\tfrac12\big[-2u^{-1/2}\big]_1^{3/4}
\pause
=\big[\frac1{\sqrt{u}}\big]_1^{3/4} \pause = \frac1{\sqrt{\frac34}}-1\pause = \frac{2}{\sqrt3}-1$$
\end{exemple}
\end{frame}


\begin{frame}
\begin{exemple}
\hfil\hfil Calcul de $\displaystyle \int_0^{1/2}\frac{1}{(1-x^2)^{3/2}} \;dx$

\pause

\begin{itemize}
  \item Changement de variable $x=\varphi(t) = \sin t$ \pause \quad $1-x^2=\cos^2t$
\pause
  \item $dx = \cos t \; dt$
\pause
  \item $t=\arcsin x$ donc pour $x=0$ on a $t=\arcsin(0)=0$
\pause
  \item Pour $x=\frac12$ on a $t=\arcsin(\frac12)=\frac\pi6$
\end{itemize}

\pause

\[
\int_0^{1/2} \frac{dx}{(1-x^2)^{3/2}}
\pause
= \int_0^{\pi/6}\frac{\cos t \; dt}{(1-\sin^2 t)^{3/2}}
\pause
= \int_0^{\pi/6}\frac{\cos t \; dt}{(\cos^2 t)^{3/2}} 
\]
\pause
\[
= \int_0^{\pi/6}\frac{\cos t}{\cos^3 t} \;dt 
\pause
=  \int_0^{\pi/6}\frac{1}{\cos^2 t} \;dt 
\pause
= \big[\tan t\big]_0^{\pi/6}
\pause
=\frac{1}{\sqrt{3}} 
\]
\end{exemple}
\end{frame}






%%%%%%%%%%%%%%%%%%%%%%%%%%%%%%%%%%%%%%%%%%%%%%%%%%%%%%%%%%%%%%%%
\section*{Mini-exercices}


\begin{frame}
\begin{miniexercice}
\begin{enumerate}
  \item Calculer les intégrales à l'aide d'intégrations par parties: 
  $\int_0^{\pi/2} t \sin t \; dt$, $\int_0^{\pi/2} t^2 \sin t \; dt$, puis par récurrence
  $\int_0^{\pi/2} t^n \sin t \; dt$.

  \item Déterminer les primitives à l'aide d'intégrations par parties:
  $\int t \sh t \; dt$, $\int t^2 \sh t \; dt$, puis par récurrence
  $\int t^n \sh t \; dt$.

  \item Calculer les intégrales à l'aide de changements de variable:
  $\int_0^a \sqrt{a^2 - t^2} \; dt$ ;
  $\int_{-\pi}^\pi \sqrt{1+\cos t} \; dt$ (pour ce dernier poser deux changements de variables : 
  $u = \cos t$, puis $v=1-u$). 

  \item Déterminer les primitives suivantes à l'aide de changements de variable :
  $\int \tanh t \; dt$ où $\tanh t = \frac{\sh t}{\ch t}$,
  $\int e^{\sqrt{t}} \; dt$.
\end{enumerate}
\end{miniexercice}
\end{frame}


\end{document}