
%%%%%%%%%%%%%%%%%% PREAMBULE %%%%%%%%%%%%%%%%%%

\documentclass[aspectratio=169,utf8]{beamer}
%\documentclass[aspectratio=169,handout]{beamer}

\usetheme{Boadilla}
%\usecolortheme{seahorse}
\usecolortheme[RGB={245,66,24}]{structure}
\useoutertheme{infolines}

% packages
\usepackage{amsfonts,amsmath,amssymb,amsthm}
\usepackage[utf8]{inputenc}
\usepackage[T1]{fontenc}
\usepackage{lmodern}

\usepackage[francais]{babel}
\usepackage{fancybox}
\usepackage{graphicx}

\usepackage{float}
\usepackage{xfrac}

%\usepackage[usenames, x11names]{xcolor}
\usepackage{tikz}
\usepackage{pgfplots}
\usepackage{datetime}



%-----  Package unités -----
\usepackage{siunitx}
\sisetup{locale = FR,detect-all,per-mode = symbol}

%\usepackage{mathptmx}
%\usepackage{fouriernc}
%\usepackage{newcent}
%\usepackage[mathcal,mathbf]{euler}

%\usepackage{palatino}
%\usepackage{newcent}
% \usepackage[mathcal,mathbf]{euler}



% \usepackage{hyperref}
% \hypersetup{colorlinks=true, linkcolor=blue, urlcolor=blue,
% pdftitle={Exo7 - Exercices de mathématiques}, pdfauthor={Exo7}}


%section
% \usepackage{sectsty}
% \allsectionsfont{\bf}
%\sectionfont{\color{Tomato3}\upshape\selectfont}
%\subsectionfont{\color{Tomato4}\upshape\selectfont}

%----- Ensembles : entiers, reels, complexes -----
\newcommand{\Nn}{\mathbb{N}} \newcommand{\N}{\mathbb{N}}
\newcommand{\Zz}{\mathbb{Z}} \newcommand{\Z}{\mathbb{Z}}
\newcommand{\Qq}{\mathbb{Q}} \newcommand{\Q}{\mathbb{Q}}
\newcommand{\Rr}{\mathbb{R}} \newcommand{\R}{\mathbb{R}}
\newcommand{\Cc}{\mathbb{C}} 
\newcommand{\Kk}{\mathbb{K}} \newcommand{\K}{\mathbb{K}}

%----- Modifications de symboles -----
\renewcommand{\epsilon}{\varepsilon}
\renewcommand{\Re}{\mathop{\text{Re}}\nolimits}
\renewcommand{\Im}{\mathop{\text{Im}}\nolimits}
%\newcommand{\llbracket}{\left[\kern-0.15em\left[}
%\newcommand{\rrbracket}{\right]\kern-0.15em\right]}

\renewcommand{\ge}{\geqslant}
\renewcommand{\geq}{\geqslant}
\renewcommand{\le}{\leqslant}
\renewcommand{\leq}{\leqslant}
\renewcommand{\epsilon}{\varepsilon}

%----- Fonctions usuelles -----
\newcommand{\ch}{\mathop{\text{ch}}\nolimits}
\newcommand{\sh}{\mathop{\text{sh}}\nolimits}
\renewcommand{\tanh}{\mathop{\text{th}}\nolimits}
\newcommand{\cotan}{\mathop{\text{cotan}}\nolimits}
\newcommand{\Arcsin}{\mathop{\text{arcsin}}\nolimits}
\newcommand{\Arccos}{\mathop{\text{arccos}}\nolimits}
\newcommand{\Arctan}{\mathop{\text{arctan}}\nolimits}
\newcommand{\Argsh}{\mathop{\text{argsh}}\nolimits}
\newcommand{\Argch}{\mathop{\text{argch}}\nolimits}
\newcommand{\Argth}{\mathop{\text{argth}}\nolimits}
\newcommand{\pgcd}{\mathop{\text{pgcd}}\nolimits} 


%----- Commandes divers ------
\newcommand{\ii}{\mathrm{i}}
\newcommand{\dd}{\text{d}}
\newcommand{\id}{\mathop{\text{id}}\nolimits}
\newcommand{\Ker}{\mathop{\text{Ker}}\nolimits}
\newcommand{\Card}{\mathop{\text{Card}}\nolimits}
\newcommand{\Vect}{\mathop{\text{Vect}}\nolimits}
\newcommand{\Mat}{\mathop{\text{Mat}}\nolimits}
\newcommand{\rg}{\mathop{\text{rg}}\nolimits}
\newcommand{\tr}{\mathop{\text{tr}}\nolimits}


%----- Structure des exercices ------

\newtheoremstyle{styleexo}% name
{2ex}% Space above
{3ex}% Space below
{}% Body font
{}% Indent amount 1
{\bfseries} % Theorem head font
{}% Punctuation after theorem head
{\newline}% Space after theorem head 2
{}% Theorem head spec (can be left empty, meaning ‘normal’)

%\theoremstyle{styleexo}
\newtheorem{exo}{Exercice}
\newtheorem{ind}{Indications}
\newtheorem{cor}{Correction}


\newcommand{\exercice}[1]{} \newcommand{\finexercice}{}
%\newcommand{\exercice}[1]{{\tiny\texttt{#1}}\vspace{-2ex}} % pour afficher le numero absolu, l'auteur...
\newcommand{\enonce}{\begin{exo}} \newcommand{\finenonce}{\end{exo}}
\newcommand{\indication}{\begin{ind}} \newcommand{\finindication}{\end{ind}}
\newcommand{\correction}{\begin{cor}} \newcommand{\fincorrection}{\end{cor}}

\newcommand{\noindication}{\stepcounter{ind}}
\newcommand{\nocorrection}{\stepcounter{cor}}

\newcommand{\fiche}[1]{} \newcommand{\finfiche}{}
\newcommand{\titre}[1]{\centerline{\large \bf #1}}
\newcommand{\addcommand}[1]{}
\newcommand{\video}[1]{}

% Marge
\newcommand{\mymargin}[1]{\marginpar{{\small #1}}}

\def\noqed{\renewcommand{\qedsymbol}{}}


%----- Presentation ------
\setlength{\parindent}{0cm}

%\newcommand{\ExoSept}{\href{http://exo7.emath.fr}{\textbf{\textsf{Exo7}}}}

\definecolor{myred}{rgb}{0.93,0.26,0}
\definecolor{myorange}{rgb}{0.97,0.58,0}
\definecolor{myyellow}{rgb}{1,0.86,0}

\newcommand{\LogoExoSept}[1]{  % input : echelle
{\usefont{U}{cmss}{bx}{n}
\begin{tikzpicture}[scale=0.1*#1,transform shape]
  \fill[color=myorange] (0,0)--(4,0)--(4,-4)--(0,-4)--cycle;
  \fill[color=myred] (0,0)--(0,3)--(-3,3)--(-3,0)--cycle;
  \fill[color=myyellow] (4,0)--(7,4)--(3,7)--(0,3)--cycle;
  \node[scale=5] at (3.5,3.5) {Exo7};
\end{tikzpicture}}
}


\newcommand{\debutmontitre}{
  \author{} \date{} 
  \thispagestyle{empty}
  \hspace*{-10ex}
  \begin{minipage}{\textwidth}
    \titlepage  
  \vspace*{-2.5cm}
  \begin{center}
    \LogoExoSept{2.5}
  \end{center}
  \end{minipage}

  \vspace*{-0cm}
  
  % Astuce pour que le background ne soit pas discrétisé lors de la conversion pdf -> png
\begin{tikzpicture}
        \fill[opacity=0,green!60!black] (0,0)--++(0,0)--++(0,0)--++(0,0)--cycle; 
\end{tikzpicture}

% toc S'affiche trop tot :
% \tableofcontents[hideallsubsections, pausesections]
}

\newcommand{\finmontitre}{
  \end{frame}
  \setcounter{framenumber}{0}
} % ne marche pas pour une raison obscure

%----- Commandes supplementaires ------

% \usepackage[landscape]{geometry}
% \geometry{top=1cm, bottom=3cm, left=2cm, right=10cm, marginparsep=1cm
% }
% \usepackage[a4paper]{geometry}
% \geometry{top=2cm, bottom=2cm, left=2cm, right=2cm, marginparsep=1cm
% }

%\usepackage{standalone}


% New command Arnaud -- november 2011
\setbeamersize{text margin left=24ex}
% si vous modifier cette valeur il faut aussi
% modifier le decalage du titre pour compenser
% (ex : ici =+10ex, titre =-5ex

\theoremstyle{definition}
%\newtheorem{proposition}{Proposition}
%\newtheorem{exemple}{Exemple}
%\newtheorem{theoreme}{Théorème}
%\newtheorem{lemme}{Lemme}
%\newtheorem{corollaire}{Corollaire}
%\newtheorem*{remarque*}{Remarque}
%\newtheorem*{miniexercice}{Mini-exercices}
%\newtheorem{definition}{Définition}

% Commande tikz
\usetikzlibrary{calc}
\usetikzlibrary{patterns,arrows}
\usetikzlibrary{matrix}
\usetikzlibrary{fadings} 

%definition d'un terme
\newcommand{\defi}[1]{{\color{myorange}\textbf{\emph{#1}}}}
\newcommand{\evidence}[1]{{\color{blue}\textbf{\emph{#1}}}}
\newcommand{\assertion}[1]{\emph{\og#1\fg}}  % pour chapitre logique
%\renewcommand{\contentsname}{Sommaire}
\renewcommand{\contentsname}{}
\setcounter{tocdepth}{2}



%------ Figures ------

\def\myscale{1} % par défaut 
\newcommand{\myfigure}[2]{  % entrée : echelle, fichier figure
\def\myscale{#1}
\begin{center}
\footnotesize
{#2}
\end{center}}


%------ Encadrement ------

\usepackage{fancybox}


\newcommand{\mybox}[1]{
\setlength{\fboxsep}{7pt}
\begin{center}
\shadowbox{#1}
\end{center}}

\newcommand{\myboxinline}[1]{
\setlength{\fboxsep}{5pt}
\raisebox{-10pt}{
\shadowbox{#1}
}
}

%--------------- Commande beamer---------------
\newcommand{\beameronly}[1]{#1} % permet de mettre des pause dans beamer pas dans poly


\setbeamertemplate{navigation symbols}{}
\setbeamertemplate{footline}  % tiré du fichier beamerouterinfolines.sty
{
  \leavevmode%
  \hbox{%
  \begin{beamercolorbox}[wd=.333333\paperwidth,ht=2.25ex,dp=1ex,center]{author in head/foot}%
    % \usebeamerfont{author in head/foot}\insertshortauthor%~~(\insertshortinstitute)
    \usebeamerfont{section in head/foot}{\bf\insertshorttitle}
  \end{beamercolorbox}%
  \begin{beamercolorbox}[wd=.333333\paperwidth,ht=2.25ex,dp=1ex,center]{title in head/foot}%
    \usebeamerfont{section in head/foot}{\bf\insertsectionhead}
  \end{beamercolorbox}%
  \begin{beamercolorbox}[wd=.333333\paperwidth,ht=2.25ex,dp=1ex,right]{date in head/foot}%
    % \usebeamerfont{date in head/foot}\insertshortdate{}\hspace*{2em}
    \insertframenumber{} / \inserttotalframenumber\hspace*{2ex} 
  \end{beamercolorbox}}%
  \vskip0pt%
}


\definecolor{mygrey}{rgb}{0.5,0.5,0.5}
\setlength{\parindent}{0cm}
%\DeclareTextFontCommand{\helvetica}{\fontfamily{phv}\selectfont}

% background beamer
\definecolor{couleurhaut}{rgb}{0.85,0.9,1}  % creme
\definecolor{couleurmilieu}{rgb}{1,1,1}  % vert pale
\definecolor{couleurbas}{rgb}{0.85,0.9,1}  % blanc
\setbeamertemplate{background canvas}[vertical shading]%
[top=couleurhaut,middle=couleurmilieu,midpoint=0.4,bottom=couleurbas] 
%[top=fondtitre!05,bottom=fondtitre!60]



\makeatletter
\setbeamertemplate{theorem begin}
{%
  \begin{\inserttheoremblockenv}
  {%
    \inserttheoremheadfont
    \inserttheoremname
    \inserttheoremnumber
    \ifx\inserttheoremaddition\@empty\else\ (\inserttheoremaddition)\fi%
    \inserttheorempunctuation
  }%
}
\setbeamertemplate{theorem end}{\end{\inserttheoremblockenv}}

\newenvironment{theoreme}[1][]{%
   \setbeamercolor{block title}{fg=structure,bg=structure!40}
   \setbeamercolor{block body}{fg=black,bg=structure!10}
   \begin{block}{{\bf Th\'eor\`eme }#1}
}{%
   \end{block}%
}


\newenvironment{proposition}[1][]{%
   \setbeamercolor{block title}{fg=structure,bg=structure!40}
   \setbeamercolor{block body}{fg=black,bg=structure!10}
   \begin{block}{{\bf Proposition }#1}
}{%
   \end{block}%
}

\newenvironment{corollaire}[1][]{%
   \setbeamercolor{block title}{fg=structure,bg=structure!40}
   \setbeamercolor{block body}{fg=black,bg=structure!10}
   \begin{block}{{\bf Corollaire }#1}
}{%
   \end{block}%
}

\newenvironment{mydefinition}[1][]{%
   \setbeamercolor{block title}{fg=structure,bg=structure!40}
   \setbeamercolor{block body}{fg=black,bg=structure!10}
   \begin{block}{{\bf Définition} #1}
}{%
   \end{block}%
}

\newenvironment{lemme}[0]{%
   \setbeamercolor{block title}{fg=structure,bg=structure!40}
   \setbeamercolor{block body}{fg=black,bg=structure!10}
   \begin{block}{\bf Lemme}
}{%
   \end{block}%
}

\newenvironment{remarque}[1][]{%
   \setbeamercolor{block title}{fg=black,bg=structure!20}
   \setbeamercolor{block body}{fg=black,bg=structure!5}
   \begin{block}{Remarque #1}
}{%
   \end{block}%
}


\newenvironment{exemple}[1][]{%
   \setbeamercolor{block title}{fg=black,bg=structure!20}
   \setbeamercolor{block body}{fg=black,bg=structure!5}
   \begin{block}{{\bf Exemple }#1}
}{%
   \end{block}%
}


\newenvironment{miniexercice}[0]{%
   \setbeamercolor{block title}{fg=structure,bg=structure!20}
   \setbeamercolor{block body}{fg=black,bg=structure!5}
   \begin{block}{Mini-exercices}
}{%
   \end{block}%
}


\newenvironment{tp}[0]{%
   \setbeamercolor{block title}{fg=structure,bg=structure!40}
   \setbeamercolor{block body}{fg=black,bg=structure!10}
   \begin{block}{\bf Travaux pratiques}
}{%
   \end{block}%
}
\newenvironment{exercicecours}[1][]{%
   \setbeamercolor{block title}{fg=structure,bg=structure!40}
   \setbeamercolor{block body}{fg=black,bg=structure!10}
   \begin{block}{{\bf Exercice }#1}
}{%
   \end{block}%
}
\newenvironment{algo}[1][]{%
   \setbeamercolor{block title}{fg=structure,bg=structure!40}
   \setbeamercolor{block body}{fg=black,bg=structure!10}
   \begin{block}{{\bf Algorithme}\hfill{\color{gray}\texttt{#1}}}
}{%
   \end{block}%
}


\setbeamertemplate{proof begin}{
   \setbeamercolor{block title}{fg=black,bg=structure!20}
   \setbeamercolor{block body}{fg=black,bg=structure!5}
   \begin{block}{{\footnotesize Démonstration}}
   \footnotesize
   \smallskip}
\setbeamertemplate{proof end}{%
   \end{block}}
\setbeamertemplate{qed symbol}{\openbox}


\makeatother
\usecolortheme[RGB={204,0,0}]{structure}
   
%%%%%%%%%%%%%%%%%%%%%%%%%%%%%%%%%%%%%%%%%%%%%%%%%%%%%%%%%%%%%
%%%%%%%%%%%%%%%%%%%%%%%%%%%%%%%%%%%%%%%%%%%%%%%%%%%%%%%%%%%%%


\begin{document}


\title{{\bf Déterminants}}
\subtitle{Applications des déterminants}

\begin{frame}
  
  \debutmontitre

  \pause

{\footnotesize
\hfill
\setbeamercovered{transparent=50}
\begin{minipage}{0.6\textwidth}
  \begin{itemize}
    \item<3-> Méthode de Cramer
    \item<4-> Déterminant et base
    \item<5-> Mineurs d'une matrice
    \item<6-> Calcul du rang d'une matrice
    \item<7-> Rang d'une matrice transposée
  \end{itemize}
\end{minipage}
}

\end{frame}

\setcounter{framenumber}{0}


%%%%%%%%%%%%%%%%%%%%%%%%%%%%%%%%%%%%%%%%%%%%%%%%%%%%%%%%%%%%%%%%
\section{Méthode de Cramer}

\begin{frame}

\begin{itemize}
  \item Soit un système d'équations linéaires à $n$ équations et $n$ inconnues
\[
\left\{
\begin{array}{ccc}
a_{11} x_1 + a_{12} x_2 + \dots + a_{1n} x_{n} & = & b_1\\
a_{21} x_1 + a_{22} x_2 + \dots + a_{2n} x_n & = & b_2\\
\qquad\qquad \dots \qquad\qquad &  &\\
a_{n1} x_1 + a_{n2} x_2 + \dots + a_{nn} x_n & = & b_n
\end{array}
\right. 
\]
\item\pause Il peut s'écrire sous forme matricielle $AX=B$ où \pause
$$
\hspace{-14pt} A = \left(
\begin{array}{cccc}
a_{11} & a_{12} & \cdots & a_{1n}\\
a_{21} & a_{22} & \cdots & a_{2n}\\
\vdots & \vdots && \vdots\\
a_{n1} & a_{n2}& \cdots &a_{nn}
\end{array}\right) \pause \qquad X= \left( \begin{array}{c}x_1\\x_2 \\\vdots \\x_n \end{array} \right)  \qquad \pause B=\left( \begin{array}{c}b_1\\b_2 \\\vdots \\b_n \end{array} \right)
$$
\vspace{-4pt}

\item\pause Définissons la matrice $A_j \in M_{n}(\Kk)$ par 
\vspace{-4pt}
$$
A_j = \left(
\begin{array}{ccccccc}
a_{11} &  \dots & a_{1,j-1} & {\color{myred}b_1} & a_{1,j+1} & \dots & a_{1n}\\
a_{21} & \dots & a_{2,j-1}& {\color{myred}b_2} & a_{2,j+1}& \dots & a_{2n}\\
\vdots &  & \vdots & {\color{myred}\vdots} & \vdots& &\vdots\\
a_{n1} &\dots & a_{n,j-1}& {\color{myred}b_n}& a_{n,j+1}& \dots & a_{nn}
\end{array}\right)
$$
\end{itemize}

\end{frame}


\begin{frame}
\begin{theoreme}[Règle de Cramer] Soit un système de $n$ équations  à $n$ inconnues $$
AX = B
$$
\pause
Supposons que $\det A \neq 0$. \pause Alors l'unique solution $(x_1,x_2,\ldots,x_n)$ du système est donnée par
$$
x_1 = \frac{\det A_1}{\det A} \qquad x_2 = \frac{\det A_2}{\det A} \qquad \ldots \qquad x_n = \frac{\det A_n}{\det A}
$$
\end{theoreme}
\end{frame}

% \begin{frame}
% \vspace{-6pt}
% \[
% x_i = \frac{\det A_i}{\det A}
% \]
% 
% \vspace{-3pt}
% \begin{proof}
% \begin{itemize}
%  \item\pause $\det A \neq 0 \pause \implies A$ inversible \pause $\implies X = A^{-1} B$ est l'unique solution
%  \item\pause Or $A^{-1} = \frac{1}{\det A} C^T$ où $C$ est la comatrice. \pause Donc $X = \frac{1}{\det A} C^T B$
%  \item\pause En développant
% \[\hspace{-14pt}
% \begin{pmatrix}
% x_1\\
% \vdots\\
% x_n
% \end{pmatrix}
% = \frac{1}{\det A}
% \begin{pmatrix}
% C_{11} & \dots & C_{n1}\\
% \vdots && \vdots\\
% C_{1n} &\dots & C_{nn}  
% \end{pmatrix}
% \ 
% \begin{pmatrix}
% b_1\\
% \vdots\\
% b_n  
% \end{pmatrix}
% \pause
% =  \frac{1}{\det A}
% \begin{pmatrix}
% C_{11} b_1  + \dots + C_{n1} b_n\\
% \vdots\\
% C_{1n}b_1  + \dots + C_{nn} b_n  
% \end{pmatrix}
% \]
%  \item\pause C'est-à-dire 
% \[
% x_1  =  \frac{C_{11} b_1 + \dots + C_{n1} b_n}{\det A}
% \quad\ldots\quad
% x_i  =  \frac{C_{1i}b_1 + \dots + C_{ni} b_n}{\det A}
% \quad\ldots
% \]
%  \item\pause $b_1 C_{1i} + \dots + b_n C_{ni}$
% est le développement de $\det A_i$ par rapport à sa colonne $i$
% 
%  \item\pause Donc $x_i = \frac{\det A_i}{\det A}$
% \qedhere
% \end{itemize}
% \end{proof}
% 
% \end{frame}


\begin{frame}
\begin{exemple}
 \vspace{-5pt}\[\text{Résolvons le système  \qquad }\left\{ 
\begin{array}{ccccccc}
x_1 & && + &2x_3 & = &6\\
 -3x_1 &+ &4x_2 &+ &6x_3 & = & 30\\
 -x_1 &- &2x_2 &+ &3 x_3 & = & 8 \end{array}
\right. \]\vspace{-7pt}
\begin{itemize}
  \item\pause On a  
$
A = \left(
\begin{array}{rrc}
1 & 0 & 2\\
-3 & 4 & 6\\
-1 & -2 & 3
\end{array}\right)
\quad
B = \left(
\begin{array}{c} 6 \\ 30 \\ 8 \end{array}\right)
$
\pause
\[
\hspace{-23pt}
A_1 = \left(
\begin{array}{crc}
6 & 0 & 2\\
30 & 4 & 6\\
8 & -2 & 3
\end{array}\right) \pause
\  A_2 = \left(
\begin{array}{rcc}
1 & 6 & 2\\
-3 & 30 & 6\\
-1 & 8 & 3
\end{array}\right) \pause
\  A_3 = \left(
\begin{array}{rrc}
1 & 0 & 6\\
-3 & 4 & 30\\
-1 & -2 & 8
\end{array}\right)
\] \vspace{-7pt}
\item\pause $\det A  =  44 \qquad \pause \det A_1  =  -40 \qquad  \pause
\det A_2  =  72 \qquad \pause \det A_3  =  152$
\smallskip
\item\pause $\displaystyle
x_1 = \frac{\det A_1}{\det A} \pause = -\frac{10}{11}\qquad \pause
x_2 = \frac{\det A_2}{\det A} \pause = \frac{18}{11}\qquad \pause
x_3 = \frac{\det A_3}{\det A} \pause = \frac{38}{11} 
$
\end{itemize}
\end{exemple}
\end{frame}

%%%%%%%%%%%%%%%%%%%%%%%%%%%%%%%%%%%%%%%%%%%%%%%%%%%%%%%%%%%%%%%%
\section{Déterminant et base}

\begin{frame}

\begin{itemize}
  \item $E$ un $\Kk$-espace vectoriel de dimension $n$ et $\mathcal{B}$ une base de $E$
  \item\pause $v_1,v_2,\ldots,v_n$ vecteurs de $E$ \pause $\implies$ base ?
  \item\pause On définit $A \in M_n(\Kk)$ la matrice dont la $j$-ème colonne est formée des coordonnées du vecteur $v_j$ dans $\mathcal{B}$ 
\end{itemize}

\pause
\begin{theoreme}
Les vecteurs $(v_1,v_2,\ldots,v_n)$ forment une base de $E$ si et seulement si $\det A \neq 0$
\end{theoreme}

\end{frame}

\begin{frame}
\begin{proof} \pause
\[
\begin{array}{rcl}
(v_1,v_2,\ldots,v_n) \text{ est une base} 
&\iff& \rg(v_1,v_2,\ldots,v_n) = n \\
\pause&\iff& \rg A = n \\
\pause&\iff& A \text{ est inversible} \\
\pause&\iff& \det A \neq 0 
\end{array}
\]
\end{proof} 

\pause
\begin{corollaire}
Une famille de $n$ vecteurs de $\Rr^n$ 
$$\begin{pmatrix}a_{11}\\a_{21}\\\vdots\\a_{n1}\end{pmatrix}
\quad
\begin{pmatrix}a_{12}\\a_{22}\\\vdots\\a_{n2}\end{pmatrix}
\quad \cdots
\quad
\begin{pmatrix}a_{1n}\\a_{2n}\\\vdots\\a_{nn}\end{pmatrix}$$
\pause 
forme une base si et seulement si $\det \;(a_{ij}) \neq 0$
\end{corollaire}

\end{frame}


\begin{frame}
\begin{exemple}
Pour quelles valeurs de $a,b \in \Rr$ les vecteurs 
$$
\begin{pmatrix}0\\a\\b\end{pmatrix} \quad
\begin{pmatrix}a\\b\\0\end{pmatrix} \quad
\begin{pmatrix}b\\0\\a\end{pmatrix}$$
forment une base de $\Rr^3$ ?  

\begin{itemize}
  \item\pause  Il suffit de calculer le déterminant \quad 
$
\begin{vmatrix}
  0 & a & b\\a & b & 0\\b & 0 & a  
  \end{vmatrix}
\pause = - a^3 - b^3 
$
  \item\pause  Conclusion: 
\begin{itemize}
  \item\pause  Si $a^3 \neq - b^3$ alors les trois vecteurs forment une base de $\Rr^3$
  \item\pause  Si $a^3 = - b^3$ alors les trois vecteurs sont liés
\end{itemize}  
  \item\pause  Exercice: montrer que $a^3=-b^3$ si et seulement si $a=-b$
\end{itemize}
\end{exemple}
\end{frame}



%%%%%%%%%%%%%%%%%%%%%%%%%%%%%%%%%%%%%%%%%%%%%%%%%%%%%%%%%%%%%%%%
\section{Mineurs d'une matrice}

\begin{frame}
\begin{itemize}
  \item  Soit $A=(a_{ij}) \in M_{n,p}(\Kk)$ une matrice à $n$ lignes 
et $p$ colonnes
\item\pause Soit $k$ un entier inférieur à $n$ et à $p$
\end{itemize}

\pause
\begin{mydefinition}
On appelle \defi{mineur d'ordre $k$} le déterminant de toute matrice carrée de taille $k$ extraite  de $A$
\end{mydefinition}

\begin{itemize}
  \item\pause Une telle matrice est obtenue en supprimant $n-k$ lignes et $p-k$ colonnes de $A$
  \item\pause $A$ n'a pas besoin d'être une matrice carrée
\end{itemize}

\end{frame}


\begin{frame}
\begin{exemple}
Soit la matrice 
$$A = 
\begin{pmatrix}
1&2&3&4\\
1&0&1&7\\
0&1&6&5
\end{pmatrix}$$

\begin{itemize}
  \item\pause Un mineur d'ordre $1$ est un coefficient de $A$
    \item\pause Un mineur d'ordre $2$ est le déterminant d'une matrice $2\times 2$ extraite de $A$
\begin{itemize}
  \item\pause  Par exemple en supprimant $L_2$, $C_1$ et $C_3$ \pause on obtient $\begin{pmatrix}2&4\\1&5\end{pmatrix}$
  \item\pause Donc un des mineurs d'ordre $2$ de $A$ est $\begin{vmatrix}2&4\\1&5\end{vmatrix} = 6$
\end{itemize} 
\end{itemize}
\end{exemple}
\end{frame}

\begin{frame}
\begin{exemple} 
$$A = 
\begin{pmatrix}
1&2&3&4\\
1&0&1&7\\
0&1&6&5
\end{pmatrix}$$

\begin{itemize} 
  \item\pause Un mineur d'ordre $3$ est le déterminant d'une matrice $3\times 3$ extraite de $A$
\begin{itemize}  
  \item\pause Par exemple en supprimant $C_2$ \pause on obtient
  le mineur
  \[
  \begin{vmatrix}
1&3&4\\
1&1&7\\
0&6&5
\end{vmatrix} = -28 
\]
\end{itemize}
  
  \item\pause Il n'y a pas de mineur d'ordre $4$

\end{itemize}
\end{exemple}
\end{frame}

%%%%%%%%%%%%%%%%%%%%%%%%%%%%%%%%%%%%%%%%%%%%%%%%%%%%%%%%%%%%%%%%
\section{Calcul du rang d'une matrice}


\begin{frame}

\begin{mydefinition}
Le rang d'une matrice est la dimension de l'espace vectoriel engendré par les vecteurs colonnes
\end{mydefinition}

\bigskip
\pause
\begin{theoreme}
Le rang d'une matrice $A \in M_{n,p}(\Kk)$ est le plus grand entier $r$ 
tel qu'il existe un mineur d'ordre~$r$ extrait de $A$ non nul
\end{theoreme}


\end{frame}


\begin{frame}

\begin{exemple}  	
Soit $\alpha\in\Rr$. Calculons le rang de la matrice $A \in M_{3,4}(\Rr)$
$$A= \begin{pmatrix}
1 & 1& 2&1\cr
1&2&3&1\cr
1&1&\alpha & 1 \cr
\end{pmatrix}$$\vspace*{-3ex}
\begin{itemize}
  \item\pause $\rg A \neq4$, puisque les colonnes sont dans $\Rr^3$
  
  \item\pause Calculons le mineur d'ordre $3$ obtenu en supprimant $C_1$ dans $A$
\pause
$$
\left | \begin{matrix}
1 & 2 & 1 \cr
2 & 3& 1 \cr
1& \alpha & 1 \cr
\end{matrix} \right |
\pause=
\left |\begin{matrix}3&1 \cr
\alpha & 1 \cr
\end{matrix}\right |
-2 \left | \begin{matrix}
2&1 \cr
\alpha & 1 \cr
\end{matrix}\right |
+\left | \begin{matrix}
2&1 \cr
3 & 1 \cr
\end{matrix}\right | \pause = \alpha -2 $$\vspace*{-2ex}
\item\pause Si $\alpha \neq 2$, le rang de la matrice $A$ est $3$
\end{itemize}
\end{exemple}
\end{frame}


\begin{frame}  
\begin{exemple}  
\begin{itemize}
  \item Si $\alpha =2$ \qquad $A= \begin{pmatrix}
1 & 1& 2&1\cr
1&2&3&1\cr
1&1&2 & 1 \cr
\end{pmatrix}$%\vspace*{-3ex}
\begin{itemize}
  \item\pause on vérifie que les $4$ mineurs d'ordre $3$ de $A$ sont nuls
  \pause
  \[
  \begin{vmatrix} 1& 2&1 \\ 2&3&1 \\ 1& 2 & 1 \end{vmatrix} \pause
  \onslide<7->{=} \begin{vmatrix} 1 & 2&1 \\ 1&3&1 \\ 1& 2 & 1 \end{vmatrix} \pause
  \onslide<7->{=} \begin{vmatrix} 1 & 1& 1 \\ 1&2&1 \\ 1&1& 1 \end{vmatrix} \pause
  \onslide<7->{=} \begin{vmatrix} 1 & 1& 2 \\ 1&2&3 \\ 1&1& 2 \end{vmatrix} \pause
  = 0
  \] 
  \item\pause Donc  $\rg A \leq 2$
  \item\pause En supprimant $L_3$, $C_3$, $C_4$ dans $A$, \pause on obtient  un mineur d'ordre $2$
$\Bigg| \begin{matrix}
1 & 1 \cr
1 & 2 \cr
\end{matrix} \Bigg| \pause= 1 \neq 0$
\end{itemize}

\pause
Donc si $\alpha = 2$ , le rang de $A$ est $2$
\end{itemize}
\end{exemple}
\end{frame}



%%%%%%%%%%%%%%%%%%%%%%%%%%%%%%%%%%%%%%%%%%%%%%%%%%%%%%%%%%%%%%%%
\section{Rang d'une matrice transposée}

\begin{frame}
\begin{proposition}
Le rang de $A$ est égal au rang de sa transposée $A^T$
\end{proposition}


\end{frame}


%%%%%%%%%%%%%%%%%%%%%%%%%%%%%%%%%%%%%%%%%%%%%%%%%%%%%%%%%%%%%%%%
\section{Mini-exercices}

\begin{frame}
\begin{miniexercice}
\begin{enumerate}
  \item Résoudre ce système linéaire en fonction du paramètre $t\in\Rr$ :
  $\left\{\begin{array}{rcl}ty+z&=&1\\2x+ty&=&2\\-y+tz&=&3\end{array} \right.$

  \item Pour quelles valeurs de $a,b \in \Rr$ les vecteurs suivants 
  forment-ils une base de $\Rr^3$ ?
  $$\begin{pmatrix}a\\1\\b\end{pmatrix} \begin{pmatrix}2a\\1\\b\end{pmatrix}
  \begin{pmatrix}3a\\1\\-2b\end{pmatrix}$$

  \item Calculer le rang de $A$ selon les paramètres $a,b\in\Rr$
  $$A=\begin{pmatrix}1&2&b\\0&a&1\\1&0&2\\1&2&1\end{pmatrix}$$

\end{enumerate}
\end{miniexercice}
\end{frame}

\end{document}