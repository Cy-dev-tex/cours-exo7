
%%%%%%%%%%%%%%%%%% PREAMBULE %%%%%%%%%%%%%%%%%%

\documentclass[aspectratio=169,utf8]{beamer}
%\documentclass[aspectratio=169,handout]{beamer}

\usetheme{Boadilla}
%\usecolortheme{seahorse}
\usecolortheme[RGB={245,66,24}]{structure}
\useoutertheme{infolines}

% packages
\usepackage{amsfonts,amsmath,amssymb,amsthm}
\usepackage[utf8]{inputenc}
\usepackage[T1]{fontenc}
\usepackage{lmodern}

\usepackage[francais]{babel}
\usepackage{fancybox}
\usepackage{graphicx}

\usepackage{float}
\usepackage{xfrac}

%\usepackage[usenames, x11names]{xcolor}
\usepackage{tikz}
\usepackage{pgfplots}
\usepackage{datetime}



%-----  Package unités -----
\usepackage{siunitx}
\sisetup{locale = FR,detect-all,per-mode = symbol}

%\usepackage{mathptmx}
%\usepackage{fouriernc}
%\usepackage{newcent}
%\usepackage[mathcal,mathbf]{euler}

%\usepackage{palatino}
%\usepackage{newcent}
% \usepackage[mathcal,mathbf]{euler}



% \usepackage{hyperref}
% \hypersetup{colorlinks=true, linkcolor=blue, urlcolor=blue,
% pdftitle={Exo7 - Exercices de mathématiques}, pdfauthor={Exo7}}


%section
% \usepackage{sectsty}
% \allsectionsfont{\bf}
%\sectionfont{\color{Tomato3}\upshape\selectfont}
%\subsectionfont{\color{Tomato4}\upshape\selectfont}

%----- Ensembles : entiers, reels, complexes -----
\newcommand{\Nn}{\mathbb{N}} \newcommand{\N}{\mathbb{N}}
\newcommand{\Zz}{\mathbb{Z}} \newcommand{\Z}{\mathbb{Z}}
\newcommand{\Qq}{\mathbb{Q}} \newcommand{\Q}{\mathbb{Q}}
\newcommand{\Rr}{\mathbb{R}} \newcommand{\R}{\mathbb{R}}
\newcommand{\Cc}{\mathbb{C}} 
\newcommand{\Kk}{\mathbb{K}} \newcommand{\K}{\mathbb{K}}

%----- Modifications de symboles -----
\renewcommand{\epsilon}{\varepsilon}
\renewcommand{\Re}{\mathop{\text{Re}}\nolimits}
\renewcommand{\Im}{\mathop{\text{Im}}\nolimits}
%\newcommand{\llbracket}{\left[\kern-0.15em\left[}
%\newcommand{\rrbracket}{\right]\kern-0.15em\right]}

\renewcommand{\ge}{\geqslant}
\renewcommand{\geq}{\geqslant}
\renewcommand{\le}{\leqslant}
\renewcommand{\leq}{\leqslant}
\renewcommand{\epsilon}{\varepsilon}

%----- Fonctions usuelles -----
\newcommand{\ch}{\mathop{\text{ch}}\nolimits}
\newcommand{\sh}{\mathop{\text{sh}}\nolimits}
\renewcommand{\tanh}{\mathop{\text{th}}\nolimits}
\newcommand{\cotan}{\mathop{\text{cotan}}\nolimits}
\newcommand{\Arcsin}{\mathop{\text{arcsin}}\nolimits}
\newcommand{\Arccos}{\mathop{\text{arccos}}\nolimits}
\newcommand{\Arctan}{\mathop{\text{arctan}}\nolimits}
\newcommand{\Argsh}{\mathop{\text{argsh}}\nolimits}
\newcommand{\Argch}{\mathop{\text{argch}}\nolimits}
\newcommand{\Argth}{\mathop{\text{argth}}\nolimits}
\newcommand{\pgcd}{\mathop{\text{pgcd}}\nolimits} 


%----- Commandes divers ------
\newcommand{\ii}{\mathrm{i}}
\newcommand{\dd}{\text{d}}
\newcommand{\id}{\mathop{\text{id}}\nolimits}
\newcommand{\Ker}{\mathop{\text{Ker}}\nolimits}
\newcommand{\Card}{\mathop{\text{Card}}\nolimits}
\newcommand{\Vect}{\mathop{\text{Vect}}\nolimits}
\newcommand{\Mat}{\mathop{\text{Mat}}\nolimits}
\newcommand{\rg}{\mathop{\text{rg}}\nolimits}
\newcommand{\tr}{\mathop{\text{tr}}\nolimits}


%----- Structure des exercices ------

\newtheoremstyle{styleexo}% name
{2ex}% Space above
{3ex}% Space below
{}% Body font
{}% Indent amount 1
{\bfseries} % Theorem head font
{}% Punctuation after theorem head
{\newline}% Space after theorem head 2
{}% Theorem head spec (can be left empty, meaning ‘normal’)

%\theoremstyle{styleexo}
\newtheorem{exo}{Exercice}
\newtheorem{ind}{Indications}
\newtheorem{cor}{Correction}


\newcommand{\exercice}[1]{} \newcommand{\finexercice}{}
%\newcommand{\exercice}[1]{{\tiny\texttt{#1}}\vspace{-2ex}} % pour afficher le numero absolu, l'auteur...
\newcommand{\enonce}{\begin{exo}} \newcommand{\finenonce}{\end{exo}}
\newcommand{\indication}{\begin{ind}} \newcommand{\finindication}{\end{ind}}
\newcommand{\correction}{\begin{cor}} \newcommand{\fincorrection}{\end{cor}}

\newcommand{\noindication}{\stepcounter{ind}}
\newcommand{\nocorrection}{\stepcounter{cor}}

\newcommand{\fiche}[1]{} \newcommand{\finfiche}{}
\newcommand{\titre}[1]{\centerline{\large \bf #1}}
\newcommand{\addcommand}[1]{}
\newcommand{\video}[1]{}

% Marge
\newcommand{\mymargin}[1]{\marginpar{{\small #1}}}

\def\noqed{\renewcommand{\qedsymbol}{}}


%----- Presentation ------
\setlength{\parindent}{0cm}

%\newcommand{\ExoSept}{\href{http://exo7.emath.fr}{\textbf{\textsf{Exo7}}}}

\definecolor{myred}{rgb}{0.93,0.26,0}
\definecolor{myorange}{rgb}{0.97,0.58,0}
\definecolor{myyellow}{rgb}{1,0.86,0}

\newcommand{\LogoExoSept}[1]{  % input : echelle
{\usefont{U}{cmss}{bx}{n}
\begin{tikzpicture}[scale=0.1*#1,transform shape]
  \fill[color=myorange] (0,0)--(4,0)--(4,-4)--(0,-4)--cycle;
  \fill[color=myred] (0,0)--(0,3)--(-3,3)--(-3,0)--cycle;
  \fill[color=myyellow] (4,0)--(7,4)--(3,7)--(0,3)--cycle;
  \node[scale=5] at (3.5,3.5) {Exo7};
\end{tikzpicture}}
}


\newcommand{\debutmontitre}{
  \author{} \date{} 
  \thispagestyle{empty}
  \hspace*{-10ex}
  \begin{minipage}{\textwidth}
    \titlepage  
  \vspace*{-2.5cm}
  \begin{center}
    \LogoExoSept{2.5}
  \end{center}
  \end{minipage}

  \vspace*{-0cm}
  
  % Astuce pour que le background ne soit pas discrétisé lors de la conversion pdf -> png
\begin{tikzpicture}
        \fill[opacity=0,green!60!black] (0,0)--++(0,0)--++(0,0)--++(0,0)--cycle; 
\end{tikzpicture}

% toc S'affiche trop tot :
% \tableofcontents[hideallsubsections, pausesections]
}

\newcommand{\finmontitre}{
  \end{frame}
  \setcounter{framenumber}{0}
} % ne marche pas pour une raison obscure

%----- Commandes supplementaires ------

% \usepackage[landscape]{geometry}
% \geometry{top=1cm, bottom=3cm, left=2cm, right=10cm, marginparsep=1cm
% }
% \usepackage[a4paper]{geometry}
% \geometry{top=2cm, bottom=2cm, left=2cm, right=2cm, marginparsep=1cm
% }

%\usepackage{standalone}


% New command Arnaud -- november 2011
\setbeamersize{text margin left=24ex}
% si vous modifier cette valeur il faut aussi
% modifier le decalage du titre pour compenser
% (ex : ici =+10ex, titre =-5ex

\theoremstyle{definition}
%\newtheorem{proposition}{Proposition}
%\newtheorem{exemple}{Exemple}
%\newtheorem{theoreme}{Théorème}
%\newtheorem{lemme}{Lemme}
%\newtheorem{corollaire}{Corollaire}
%\newtheorem*{remarque*}{Remarque}
%\newtheorem*{miniexercice}{Mini-exercices}
%\newtheorem{definition}{Définition}

% Commande tikz
\usetikzlibrary{calc}
\usetikzlibrary{patterns,arrows}
\usetikzlibrary{matrix}
\usetikzlibrary{fadings} 

%definition d'un terme
\newcommand{\defi}[1]{{\color{myorange}\textbf{\emph{#1}}}}
\newcommand{\evidence}[1]{{\color{blue}\textbf{\emph{#1}}}}
\newcommand{\assertion}[1]{\emph{\og#1\fg}}  % pour chapitre logique
%\renewcommand{\contentsname}{Sommaire}
\renewcommand{\contentsname}{}
\setcounter{tocdepth}{2}



%------ Figures ------

\def\myscale{1} % par défaut 
\newcommand{\myfigure}[2]{  % entrée : echelle, fichier figure
\def\myscale{#1}
\begin{center}
\footnotesize
{#2}
\end{center}}


%------ Encadrement ------

\usepackage{fancybox}


\newcommand{\mybox}[1]{
\setlength{\fboxsep}{7pt}
\begin{center}
\shadowbox{#1}
\end{center}}

\newcommand{\myboxinline}[1]{
\setlength{\fboxsep}{5pt}
\raisebox{-10pt}{
\shadowbox{#1}
}
}

%--------------- Commande beamer---------------
\newcommand{\beameronly}[1]{#1} % permet de mettre des pause dans beamer pas dans poly


\setbeamertemplate{navigation symbols}{}
\setbeamertemplate{footline}  % tiré du fichier beamerouterinfolines.sty
{
  \leavevmode%
  \hbox{%
  \begin{beamercolorbox}[wd=.333333\paperwidth,ht=2.25ex,dp=1ex,center]{author in head/foot}%
    % \usebeamerfont{author in head/foot}\insertshortauthor%~~(\insertshortinstitute)
    \usebeamerfont{section in head/foot}{\bf\insertshorttitle}
  \end{beamercolorbox}%
  \begin{beamercolorbox}[wd=.333333\paperwidth,ht=2.25ex,dp=1ex,center]{title in head/foot}%
    \usebeamerfont{section in head/foot}{\bf\insertsectionhead}
  \end{beamercolorbox}%
  \begin{beamercolorbox}[wd=.333333\paperwidth,ht=2.25ex,dp=1ex,right]{date in head/foot}%
    % \usebeamerfont{date in head/foot}\insertshortdate{}\hspace*{2em}
    \insertframenumber{} / \inserttotalframenumber\hspace*{2ex} 
  \end{beamercolorbox}}%
  \vskip0pt%
}


\definecolor{mygrey}{rgb}{0.5,0.5,0.5}
\setlength{\parindent}{0cm}
%\DeclareTextFontCommand{\helvetica}{\fontfamily{phv}\selectfont}

% background beamer
\definecolor{couleurhaut}{rgb}{0.85,0.9,1}  % creme
\definecolor{couleurmilieu}{rgb}{1,1,1}  % vert pale
\definecolor{couleurbas}{rgb}{0.85,0.9,1}  % blanc
\setbeamertemplate{background canvas}[vertical shading]%
[top=couleurhaut,middle=couleurmilieu,midpoint=0.4,bottom=couleurbas] 
%[top=fondtitre!05,bottom=fondtitre!60]



\makeatletter
\setbeamertemplate{theorem begin}
{%
  \begin{\inserttheoremblockenv}
  {%
    \inserttheoremheadfont
    \inserttheoremname
    \inserttheoremnumber
    \ifx\inserttheoremaddition\@empty\else\ (\inserttheoremaddition)\fi%
    \inserttheorempunctuation
  }%
}
\setbeamertemplate{theorem end}{\end{\inserttheoremblockenv}}

\newenvironment{theoreme}[1][]{%
   \setbeamercolor{block title}{fg=structure,bg=structure!40}
   \setbeamercolor{block body}{fg=black,bg=structure!10}
   \begin{block}{{\bf Th\'eor\`eme }#1}
}{%
   \end{block}%
}


\newenvironment{proposition}[1][]{%
   \setbeamercolor{block title}{fg=structure,bg=structure!40}
   \setbeamercolor{block body}{fg=black,bg=structure!10}
   \begin{block}{{\bf Proposition }#1}
}{%
   \end{block}%
}

\newenvironment{corollaire}[1][]{%
   \setbeamercolor{block title}{fg=structure,bg=structure!40}
   \setbeamercolor{block body}{fg=black,bg=structure!10}
   \begin{block}{{\bf Corollaire }#1}
}{%
   \end{block}%
}

\newenvironment{mydefinition}[1][]{%
   \setbeamercolor{block title}{fg=structure,bg=structure!40}
   \setbeamercolor{block body}{fg=black,bg=structure!10}
   \begin{block}{{\bf Définition} #1}
}{%
   \end{block}%
}

\newenvironment{lemme}[0]{%
   \setbeamercolor{block title}{fg=structure,bg=structure!40}
   \setbeamercolor{block body}{fg=black,bg=structure!10}
   \begin{block}{\bf Lemme}
}{%
   \end{block}%
}

\newenvironment{remarque}[1][]{%
   \setbeamercolor{block title}{fg=black,bg=structure!20}
   \setbeamercolor{block body}{fg=black,bg=structure!5}
   \begin{block}{Remarque #1}
}{%
   \end{block}%
}


\newenvironment{exemple}[1][]{%
   \setbeamercolor{block title}{fg=black,bg=structure!20}
   \setbeamercolor{block body}{fg=black,bg=structure!5}
   \begin{block}{{\bf Exemple }#1}
}{%
   \end{block}%
}


\newenvironment{miniexercice}[0]{%
   \setbeamercolor{block title}{fg=structure,bg=structure!20}
   \setbeamercolor{block body}{fg=black,bg=structure!5}
   \begin{block}{Mini-exercices}
}{%
   \end{block}%
}


\newenvironment{tp}[0]{%
   \setbeamercolor{block title}{fg=structure,bg=structure!40}
   \setbeamercolor{block body}{fg=black,bg=structure!10}
   \begin{block}{\bf Travaux pratiques}
}{%
   \end{block}%
}
\newenvironment{exercicecours}[1][]{%
   \setbeamercolor{block title}{fg=structure,bg=structure!40}
   \setbeamercolor{block body}{fg=black,bg=structure!10}
   \begin{block}{{\bf Exercice }#1}
}{%
   \end{block}%
}
\newenvironment{algo}[1][]{%
   \setbeamercolor{block title}{fg=structure,bg=structure!40}
   \setbeamercolor{block body}{fg=black,bg=structure!10}
   \begin{block}{{\bf Algorithme}\hfill{\color{gray}\texttt{#1}}}
}{%
   \end{block}%
}


\setbeamertemplate{proof begin}{
   \setbeamercolor{block title}{fg=black,bg=structure!20}
   \setbeamercolor{block body}{fg=black,bg=structure!5}
   \begin{block}{{\footnotesize Démonstration}}
   \footnotesize
   \smallskip}
\setbeamertemplate{proof end}{%
   \end{block}}
\setbeamertemplate{qed symbol}{\openbox}


\makeatother
\usecolortheme[RGB={204,0,0}]{structure}

   
%%%%%%%%%%%%%%%%%%%%%%%%%%%%%%%%%%%%%%%%%%%%%%%%%%%%%%%%%%%%%
%%%%%%%%%%%%%%%%%%%%%%%%%%%%%%%%%%%%%%%%%%%%%%%%%%%%%%%%%%%%%


\begin{document}


\title{{\bf Déterminants}}
\subtitle{Définition du déterminant}

\begin{frame}
  
  \debutmontitre

  \pause

{\footnotesize
\hfill
\setbeamercovered{transparent=50}
\begin{minipage}{0.6\textwidth}
  \begin{itemize}
    \item<3-> Définition
    \item<4-> Premières propriétés
    \item<5-> Déterminant de matrices particulières
    \item<6-> Démonstration de l'existence du déterminant
  \end{itemize}
\end{minipage}
}

\end{frame}

\setcounter{framenumber}{0}


%%%%%%%%%%%%%%%%%%%%%%%%%%%%%%%%%%%%%%%%%%%%%%%%%%%%%%%%%%%%%%%%
\section{Définition et premières propriétés}

\begin{frame}
Le déterminant est une application qui à une matrice associe un scalaire
$$\det : M_n(\Kk) \longrightarrow \Kk$$

\pause

\begin{theoreme}
\label{th:def:determinant}
Il existe une unique application de $M_n(\Kk)$ dans $\Kk$, 
appelée \defi{déterminant}, telle que
\begin{itemize}
  \item[(i)]\pause le déterminant est linéaire par rapport à chaque
vecteur colonne, les autres étant fixés 
  \item[(ii)]\pause si $A$ a deux colonnes identiques, 
alors son déterminant est nul 
  \item[(iii)]\pause le déterminant de la matrice identité $I_n$ vaut $1$
\end{itemize}
\end{theoreme}
\pause
\begin{remarque}
\begin{itemize}
  \item Une application satisfaisant (i) est appelée \defi{forme multilinéaire}
  
  \item\pause Si elle satisfait (ii), on dit qu'elle est \defi{alternée}
  
  %\item Le déterminant est donc la seule forme multilinéaire alternée qui prend comme valeur $1$ sur la matrice $I_n$
\end{itemize} 
\end{remarque}

\end{frame}


\begin{frame}

\begin{itemize}
  \item On note le  déterminant d'une matrice $A = (a_{ij})$ par
$$
\det A \qquad \text{ ou } \qquad
\left|\begin{array}{cccc}
a_{11} & a_{12} & \cdots & a_{1n} \\
a_{21} & a_{22} & \cdots & a_{2n} \\
\vdots & \vdots &  & \vdots \\
a_{n1} & a_{n2} & \cdots & a_{nn}
\end{array}\right|$$

  \item \pause Si on note $C_{i}$ la $i$-ème colonne de $A$ alors
$$\det A=\left|\begin{matrix}
C_1&C_2&\cdots&C_n
\end{matrix}\right|
 = \det (C_1,C_2,\ldots,C_n) 
$$
\end{itemize}
\end{frame}

%%%%%%%%%%%%%%

\begin{frame}
\begin{itemize}
  \item  La propriété (i) s'écrit \pause
\begin{gather*} 
 \det (C_1,\ldots,\lambda C_j + \mu C'_j,\ldots, C_n) \\
\onslide<3->{= \lambda  \det (C_1,\ldots,C_j,\ldots, C_n)+ \mu \det (C_1,\ldots,C_j',\ldots, C_n)} 
\end{gather*}
\onslide<4-> c'est-à-dire
\begin{gather*}
\begin{vmatrix}
a_{11} & \cdots & \lambda a_{1j}+\mu a_{1j}' & \cdots & a_{1n} \\
\vdots &        &       \vdots                &        & \vdots\\
a_{i1} & \cdots & \lambda a_{ij}+\mu a_{ij}' & \cdots & a_{in} \\
\vdots &        &       \vdots                &        & \vdots\\
a_{n1} & \cdots & \lambda a_{nj}+\mu a_{nj}' & \cdots & a_{nn} \\
\end{vmatrix} \\
\onslide<5->{ =
\lambda \begin{vmatrix}
a_{11} & \cdots &  a_{1j} & \cdots & a_{1n} \\
\vdots &        &       \vdots                &        & \vdots\\
a_{i1} & \cdots & a_{ij} & \cdots & a_{in} \\
\vdots &        &       \vdots                &        & \vdots\\
a_{n1} & \cdots & a_{nj} & \cdots & a_{nn} \\
\end{vmatrix} 
 +
\mu \begin{vmatrix}
a_{11} & \cdots & a_{1j}' & \cdots & a_{1n} \\
\vdots &        &       \vdots                &        & \vdots\\
a_{i1} & \cdots & a_{ij}' & \cdots & a_{in} \\
\vdots &        &       \vdots                &        & \vdots\\
a_{n1} & \cdots & a_{nj}' & \cdots & a_{nn} \\
\end{vmatrix}}
\end{gather*}

\end{itemize}
\pause\pause\pause
\end{frame}

%%%%%%%%%%%%%%

\begin{frame}

\begin{exemple}
\begin{itemize}
  \item Comme la seconde colonne est un multiple de $5$ 
  $$\begin{vmatrix}
6&5&4\\7&-10&-3\\12&25&-1    
  \end{vmatrix}
\pause =
5\times\begin{vmatrix}
6&1&4\\7&-2&-3\\12&5&-1    
  \end{vmatrix} $$
 

  \item \pause Par linéarité sur la troisième colonne
$$
% \begin{vmatrix}
% 3&2&1\\7&-5&1\\9&2&6\\   
%   \end{vmatrix}
%   = 
  \begin{vmatrix}
3&2&4-3\\7&-5&3-2\\9&2&10-4\\   
  \end{vmatrix}
\pause =
\begin{vmatrix}
3&2&4\\7&-5&3\\9&2&10\\   
  \end{vmatrix}
  - \begin{vmatrix}
3&2&3\\7&-5&2\\9&2&4\\   
  \end{vmatrix}
$$
\end{itemize}
\end{exemple}

\end{frame}

%%%%%%%%%%%%%%%%%%%%%%%%%%%%%%%%%%%%%%%%%%%%%%%%%%%%%%%%%%%%%%%%
\section{Premières propriétés}

\begin{frame}

Nous connaissons déjà le déterminant de deux matrices:
\begin{itemize}
  \item \pause $\det 0_n = 0$ (par la propriété (ii))
  \item \pause $\det I_n =1$ (par la propriété (iii))
\end{itemize}

\pause

\begin{proposition}
Soit $A = \left( C_1, C_2, \ldots, C_n\right) \in M_n(\Kk)$

\pause

Soit $A'\in M_n(\Kk)$ obtenue par opération élémentaire sur les colonnes :
\begin{enumerate}
  \item\pause $C_i \leftarrow \lambda C_i$ avec $\lambda \neq 0$.  \pause Alors $\det A' = \lambda \det A$
    
  \item\pause $C_i \leftarrow C_i+\lambda C_j$ avec $\lambda \in \Kk$ (et $j\neq i$). \pause Alors $\det A' = \det A$
  
  \item\pause $C_i \leftrightarrow C_j$. \pause Alors \myboxinline{$\det A' = - \det A$}
\end{enumerate}
\end{proposition}

\pause

Plus généralement pour (2), $C_i \leftarrow C_i+\displaystyle\sum_{\substack{j=1 \\ j\neq i}}^n\lambda_j C_j$ conserve le déterminant

\end{frame}

%%%%%%%%%%%%%%

\begin{frame}

\begin{proof} 

\begin{enumerate}
  \setcounter{enumi}{1}
  \item \pause Soit $A=\begin{pmatrix}
C_1&\cdots&C_i&\cdots&C_j&\cdots&C_n
\end{pmatrix}$

\begin{itemize}
  
  \item\pause L'opération $C_i \leftarrow C_i+\lambda C_j$ transforme $A$ en 
    \vspace{-.2cm}
$$A'=\begin{pmatrix}
C_1&\cdots&\displaystyle C_i+\lambda C_j&\cdots&C_j&\cdots&C_n\end{pmatrix}$$
 
  \vspace{-.2cm}
  
  \item \pause Par linéarité par rapport à la colonne $i$ 
    \vspace{-.2cm}
$$\det A'=\det A+\lambda \det \begin{pmatrix} C_1&\cdots&C_j&\cdots&C_j&\cdots&C_n
\end{pmatrix}$$

  \vspace{-.2cm}

  \item \pause Or les colonnes $i$ et $j$ de la matrice 
  \vspace{-.2cm}
  $$\begin{pmatrix} C_1&\cdots&C_j&\cdots&C_j&\cdots&C_n\end{pmatrix}$$ 
  
    \vspace{-.2cm}

  sont identiques \pause donc son déterminant est nul 
  
  \item \pause $\det A'=\det A$ \qedhere
\end{itemize}
\end{enumerate}
\qedhere\end{proof} 
\pause



\begin{corollaire} 
Si une colonne de $A$ est combinaison linéaire 
des autres colonnes alors \\
\centerline{$\det A=0$}
\end{corollaire}
\end{frame}

%%%%%%%%%%%%%%%%%%%%%%%%%%%%%%%%%%%%%%%%%%%%%%%%%%%%%%%%%%%%%%%%
\section{Déterminant de matrices particulières}

\begin{frame}

\begin{proposition}
Le déterminant d'une matrice triangulaire supérieure (ou inférieure)
est égal au produit des termes diagonaux
\end{proposition}

\pause

Autrement dit, pour une matrice triangulaire $A = (a_{ij})$
$$\det A = \begin{vmatrix}
{\color{myred}a_{11}} & a_{12} &\dots&\dots&\dots & a_{1n}\\
0&{\color{myred}a_{22}}&\dots&\dots&\dots&a_{2n}\\
\vdots&\ddots&{\color{myred}\ddots}&&&\vdots\\
\vdots&&\ddots&{\color{myred}\ddots}&&\vdots\\
\vdots & &&\ddots&{\color{myred}\ddots}&\vdots\\
0&\dots&\dots&\dots&0&{\color{myred}a_{nn}}    
  \end{vmatrix}\pause  = a_{11}\cdot a_{22} \ \cdots\  a_{nn}
$$

\pause

\begin{corollaire}
Le déterminant d'une matrice diagonale est égal au produit des termes diagonaux
\end{corollaire}

\end{frame}


%%%%%%%%%%%%%%

\begin{frame}
\begin{proof}

\begin{itemize}
\item Soit 
$$A=\begin{pmatrix}
{a_{11}}&a_{12}&a_{13}&\cdots&a_{1n}\\
0&{a_{22}}&a_{23}&\cdots&a_{2n}\\
0&0&{a_{33}}&\cdots&a_{3n}\\
\vdots&\vdots&\vdots&\ddots&\vdots\\
0&0&0&\cdots&{a_{nn}}
\end{pmatrix}
$$

  \item\pause On utilise l'algorithme du pivot de Gauss sur les colonnes
\end{itemize} 

\onslide<3->{Par linéarité par rapport à la première colonne}
$$
\onslide<3->
\det A=a_{11}\left|\begin{matrix}
1&a_{12}&a_{13}&\cdots&a_{1n}\\
0&{a_{22}}&a_{23}&\cdots&a_{2n}\\
0&0&{a_{33}}&\cdots&a_{3n}\\
\vdots&\vdots&\vdots&\ddots&\vdots\\
0&0&0&\cdots&{a_{nn}}
\end{matrix}\right|
\onslide<5->{
=
a_{11}\left|\begin{matrix}
1&0&0&\cdots&0\\
0&{a_{22}}&a_{23}&\cdots&a_{2n}\\
0&0&{a_{33}}&\cdots&a_{3n}\\
\vdots&\vdots&\vdots&\ddots&\vdots\\
0&0&0&\cdots&{a_{nn}}
\end{matrix}\right| 
}$$

\begin{itemize}  
  \item<4-> Pour tout $j \geq 2$, $C_j \leftarrow C_j - a_{1j}C_1$
\end{itemize}  
\vspace*{-3ex}
\noqed
\end{proof} 
\end{frame}


%%%%%%%%%%%%%%

\begin{frame}
\begin{proof}
\begin{itemize}
  \item Par linéarité par rapport à la deuxième colonne 
\[
\det A =a_{11} \cdot a_{22}\left|\begin{matrix}
1&0&0&\cdots&0\\
0&1&a_{23}&\cdots&a_{2n}\\
0&0&{a_{33}}&\cdots&a_{3n}\\
\vdots&\vdots&\vdots&\ddots&\vdots\\
0&0&0&\cdots&{a_{nn}}
\end{matrix}\right|
\pause=a_{11} \cdot a_{22} \cdot a_{33}\cdots{a_{nn}}\left|\begin{matrix}
1&0&0&\cdots&0\\
0&1&0&\cdots&0\\
0&0&1&\cdots&0\\
\vdots&\vdots&\vdots&\ddots&\vdots\\
0&0&0&\cdots&1
\end{matrix}\right| 
\]

  \item\pause  $\det A = a_{11} \cdot a_{22} \cdot a_{33}\cdots{a_{nn}} \cdot \det I_n \pause = a_{11} \cdot a_{22} \cdot a_{33}\cdots{a_{nn}}$

\end{itemize}
\end{proof} 
\end{frame}



%%%%%%%%%%%%%%%%%%%%%%%%%%%%%%%%%%%%%%%%%%%%%%%%%%%%%%%%%%%%%%%%
\section{Démonstration de l'existence du déterminant}

\begin{frame}


\textbf{Notation.}
Pour $A \in M_n(\Kk)$ \pause on note $A_{i,j}$ la matrice obtenue en supprimant la $i$-ème ligne et la $j$-ème colonne de $A$ 
 \pause

\begin{theoreme}
Les formules suivantes définissent par récurrence pour $n\ge 1$, l'application déterminant
de $M_n(\Kk)$ dans $\Kk$ qui satisfait aux propriétés (i), (ii), (iii)
\smallskip

\begin{itemize}
  \item \pause \textbf{Déterminant d'une matrice $\mathbf{1\times 1}$.} 
  Si $A=(a)$, $\det A = a$
  
\smallskip  
  \item \pause \textbf{Formule de récurrence.} 
  Si $A=(a_{i,j}) \in M_n(\Kk)$, alors pour tout $i$
$$\det A = (-1)^{i+1}a_{i,1}\det A_{i,1} +\dots 
+ (-1)^{i+n}a_{i,n}\det A_{i,n}$$  
\end{itemize}

\end{theoreme}

\end{frame}


%%%%%%%%%%%%%%%%%%%%%%%%%%%%%%%%%%%%%%%%%%%%%%%%%%%%%%%%%%%%%%%%
\section{Mini-exercices}

\begin{frame}
\begin{miniexercice}
\begin{enumerate}
  \item En utilisant la linéarité du déterminant, calculer $\det \, (-I_n)$.
  
  \item Pour $A \in M_n(\Kk)$ calculer $\det (\lambda A)$
  en fonction de $\det A$.
  
  \item Montrer que le déterminant reste invariant par l'opération \\
  \centerline{$C_i \leftarrow C_i+\displaystyle\sum_{{\substack{j=1..n \\ j\neq i}}}\lambda_j C_j$}
  
  (on ajoute à une colonne une combinaison linéaire des autres colonnes).

\end{enumerate}
\end{miniexercice}
\end{frame}

\end{document}