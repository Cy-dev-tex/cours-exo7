
%%%%%%%%%%%%%%%%%% PREAMBULE %%%%%%%%%%%%%%%%%%


\documentclass[12pt]{article}

\usepackage{amsfonts,amsmath,amssymb,amsthm}
\usepackage[utf8]{inputenc}
\usepackage[T1]{fontenc}
\usepackage[francais]{babel}


% packages
\usepackage{amsfonts,amsmath,amssymb,amsthm}
\usepackage[utf8]{inputenc}
\usepackage[T1]{fontenc}
%\usepackage{lmodern}

\usepackage[francais]{babel}
\usepackage{fancybox}
\usepackage{graphicx}

\usepackage{float}

%\usepackage[usenames, x11names]{xcolor}
\usepackage{tikz}
\usepackage{datetime}

\usepackage{mathptmx}
%\usepackage{fouriernc}
%\usepackage{newcent}
\usepackage[mathcal,mathbf]{euler}

%\usepackage{palatino}
%\usepackage{newcent}


% Commande spéciale prompteur

%\usepackage{mathptmx}
%\usepackage[mathcal,mathbf]{euler}
%\usepackage{mathpple,multido}

\usepackage[a4paper]{geometry}
\geometry{top=2cm, bottom=2cm, left=1cm, right=1cm, marginparsep=1cm}

\newcommand{\change}{{\color{red}\rule{\textwidth}{1mm}\\}}

\newcounter{mydiapo}

\newcommand{\diapo}{\newpage
\hfill {\normalsize  Diapo \themydiapo \quad \texttt{[\jobname]}} \\
\stepcounter{mydiapo}}


%%%%%%% COULEURS %%%%%%%%%%

% Pour blanc sur noir :
%\pagecolor[rgb]{0.5,0.5,0.5}
% \pagecolor[rgb]{0,0,0}
% \color[rgb]{1,1,1}



%\DeclareFixedFont{\myfont}{U}{cmss}{bx}{n}{18pt}
\newcommand{\debuttexte}{
%%%%%%%%%%%%% FONTES %%%%%%%%%%%%%
\renewcommand{\baselinestretch}{1.5}
\usefont{U}{cmss}{bx}{n}
\bfseries

% Taille normale : commenter le reste !
%Taille Arnaud
%\fontsize{19}{19}\selectfont

% Taille Barbara
%\fontsize{21}{22}\selectfont

%Taille François
%\fontsize{25}{30}\selectfont

%Taille Pascal
%\fontsize{25}{30}\selectfont

%Taille Laura
%\fontsize{30}{35}\selectfont


%\myfont
%\usefont{U}{cmss}{bx}{n}

%\Huge
%\addtolength{\parskip}{\baselineskip}
}


% \usepackage{hyperref}
% \hypersetup{colorlinks=true, linkcolor=blue, urlcolor=blue,
% pdftitle={Exo7 - Exercices de mathématiques}, pdfauthor={Exo7}}


%section
% \usepackage{sectsty}
% \allsectionsfont{\bf}
%\sectionfont{\color{Tomato3}\upshape\selectfont}
%\subsectionfont{\color{Tomato4}\upshape\selectfont}

%----- Ensembles : entiers, reels, complexes -----
\newcommand{\Nn}{\mathbb{N}} \newcommand{\N}{\mathbb{N}}
\newcommand{\Zz}{\mathbb{Z}} \newcommand{\Z}{\mathbb{Z}}
\newcommand{\Qq}{\mathbb{Q}} \newcommand{\Q}{\mathbb{Q}}
\newcommand{\Rr}{\mathbb{R}} \newcommand{\R}{\mathbb{R}}
\newcommand{\Cc}{\mathbb{C}} 
\newcommand{\Kk}{\mathbb{K}} \newcommand{\K}{\mathbb{K}}

%----- Modifications de symboles -----
\renewcommand{\epsilon}{\varepsilon}
\renewcommand{\Re}{\mathop{\text{Re}}\nolimits}
\renewcommand{\Im}{\mathop{\text{Im}}\nolimits}
%\newcommand{\llbracket}{\left[\kern-0.15em\left[}
%\newcommand{\rrbracket}{\right]\kern-0.15em\right]}

\renewcommand{\ge}{\geqslant}
\renewcommand{\geq}{\geqslant}
\renewcommand{\le}{\leqslant}
\renewcommand{\leq}{\leqslant}

%----- Fonctions usuelles -----
\newcommand{\ch}{\mathop{\mathrm{ch}}\nolimits}
\newcommand{\sh}{\mathop{\mathrm{sh}}\nolimits}
\renewcommand{\tanh}{\mathop{\mathrm{th}}\nolimits}
\newcommand{\cotan}{\mathop{\mathrm{cotan}}\nolimits}
\newcommand{\Arcsin}{\mathop{\mathrm{Arcsin}}\nolimits}
\newcommand{\Arccos}{\mathop{\mathrm{Arccos}}\nolimits}
\newcommand{\Arctan}{\mathop{\mathrm{Arctan}}\nolimits}
\newcommand{\Argsh}{\mathop{\mathrm{Argsh}}\nolimits}
\newcommand{\Argch}{\mathop{\mathrm{Argch}}\nolimits}
\newcommand{\Argth}{\mathop{\mathrm{Argth}}\nolimits}
\newcommand{\pgcd}{\mathop{\mathrm{pgcd}}\nolimits} 

\newcommand{\Card}{\mathop{\text{Card}}\nolimits}
\newcommand{\Ker}{\mathop{\text{Ker}}\nolimits}
\newcommand{\id}{\mathop{\text{id}}\nolimits}
\newcommand{\ii}{\mathrm{i}}
\newcommand{\dd}{\mathrm{d}}
\newcommand{\Vect}{\mathop{\text{Vect}}\nolimits}
\newcommand{\Mat}{\mathop{\mathrm{Mat}}\nolimits}
\newcommand{\rg}{\mathop{\text{rg}}\nolimits}
\newcommand{\tr}{\mathop{\text{tr}}\nolimits}
\newcommand{\ppcm}{\mathop{\text{ppcm}}\nolimits}

%----- Structure des exercices ------

\newtheoremstyle{styleexo}% name
{2ex}% Space above
{3ex}% Space below
{}% Body font
{}% Indent amount 1
{\bfseries} % Theorem head font
{}% Punctuation after theorem head
{\newline}% Space after theorem head 2
{}% Theorem head spec (can be left empty, meaning ‘normal’)

%\theoremstyle{styleexo}
\newtheorem{exo}{Exercice}
\newtheorem{ind}{Indications}
\newtheorem{cor}{Correction}


\newcommand{\exercice}[1]{} \newcommand{\finexercice}{}
%\newcommand{\exercice}[1]{{\tiny\texttt{#1}}\vspace{-2ex}} % pour afficher le numero absolu, l'auteur...
\newcommand{\enonce}{\begin{exo}} \newcommand{\finenonce}{\end{exo}}
\newcommand{\indication}{\begin{ind}} \newcommand{\finindication}{\end{ind}}
\newcommand{\correction}{\begin{cor}} \newcommand{\fincorrection}{\end{cor}}

\newcommand{\noindication}{\stepcounter{ind}}
\newcommand{\nocorrection}{\stepcounter{cor}}

\newcommand{\fiche}[1]{} \newcommand{\finfiche}{}
\newcommand{\titre}[1]{\centerline{\large \bf #1}}
\newcommand{\addcommand}[1]{}
\newcommand{\video}[1]{}

% Marge
\newcommand{\mymargin}[1]{\marginpar{{\small #1}}}



%----- Presentation ------
\setlength{\parindent}{0cm}

%\newcommand{\ExoSept}{\href{http://exo7.emath.fr}{\textbf{\textsf{Exo7}}}}

\definecolor{myred}{rgb}{0.93,0.26,0}
\definecolor{myorange}{rgb}{0.97,0.58,0}
\definecolor{myyellow}{rgb}{1,0.86,0}

\newcommand{\LogoExoSept}[1]{  % input : echelle
{\usefont{U}{cmss}{bx}{n}
\begin{tikzpicture}[scale=0.1*#1,transform shape]
  \fill[color=myorange] (0,0)--(4,0)--(4,-4)--(0,-4)--cycle;
  \fill[color=myred] (0,0)--(0,3)--(-3,3)--(-3,0)--cycle;
  \fill[color=myyellow] (4,0)--(7,4)--(3,7)--(0,3)--cycle;
  \node[scale=5] at (3.5,3.5) {Exo7};
\end{tikzpicture}}
}



\theoremstyle{definition}
%\newtheorem{proposition}{Proposition}
%\newtheorem{exemple}{Exemple}
%\newtheorem{theoreme}{Théorème}
\newtheorem{lemme}{Lemme}
\newtheorem{corollaire}{Corollaire}
%\newtheorem*{remarque*}{Remarque}
%\newtheorem*{miniexercice}{Mini-exercices}
%\newtheorem{definition}{Définition}




%definition d'un terme
\newcommand{\defi}[1]{{\color{myorange}\textbf{\emph{#1}}}}
\newcommand{\evidence}[1]{{\color{blue}\textbf{\emph{#1}}}}



 %----- Commandes divers ------

\newcommand{\codeinline}[1]{\texttt{#1}}

%%%%%%%%%%%%%%%%%%%%%%%%%%%%%%%%%%%%%%%%%%%%%%%%%%%%%%%%%%%%%
%%%%%%%%%%%%%%%%%%%%%%%%%%%%%%%%%%%%%%%%%%%%%%%%%%%%%%%%%%%%%



\begin{document}

\debuttexte


%%%%%%%%%%%%%%%%%%%%%%%%%%%%%%%%%%%%%%%%%%%%%%%%%%%%%%%%%%%
\diapo

\change
Dans cette leçon, nous allons aborder systématiquement le calcul explicite de déterminants. 
\\
Une des techniques les plus utiles pour cela est le 
«dévelop\-pement par rapport à une ligne (ou une colonne)».

\change
Nous commencerons par définir  un outil indispensable : le cofacteur.

\change
Puis nous donnerons la formule de développement du déterminant suivant une ligne ou une colonne, qui est l'objet principal de cette leçon,

\change
que nous appliquerons aussitôt sur un exemple.

\change
Enfin, nous verrons comment les cofacteurs nous permettent également d'exprimer l'inverse d'une matrice.

%%%%%%%%%%%%%%%%%%%%%%%%%%%%%%%%%%%%%%%%%%%%%%%%%%%%%%%%%%%
\diapo
On commence par des définitions. Considérons $A$ une matrice carrée de taille $n$, 
et de  coefficients $a_{ij}$.

\change
On note $A_{ij}$ la matrice extraite de $A$, obtenue en effaçant la ligne~$i$ et la colonne $j$ de $A$.

\change
Le déterminant de cette matrice, c-à-d le nombre $\det A_{ij}$ est appelé un \defi{mineur d'ordre $n-1$} de la matrice $A$.

\change
Enfin, le nombre $C_{ij} = (-1)^{i+j}\det A_{ij}$ est par définition le \defi{cofacteur} de $A$ 
relatif au coefficient (petit) $a_{ij}$.

%%%%%%%%%%%%%%%%%%%%%%%%%%%%%%%%%%%%%%%%%%%%%%%%%%%%%%%%%%%
\diapo

Revenons un instant sur les définitions précédentes.

\change

[sur la figure]  La matrice $A_{ij}$ est obtenue en supprimant la ligne~$i$ [horizontale]
et la colonne $j$ [verticale] de $A$, c-à-d la ligne et la colonne qui se croisent 
au coefficient $a_{ij}$ de $A$.

\change
C'est donc cette matrice carrée, de taille $n-1$.

\change
Le cofacteur de $A$ relatif au coefficient $a_{ij}$ est égal au déterminant de cette matrice, multiplié par $(-1)^{i+j}$.

\change
C'est donc ce nombre.

\change
(...)
\newpage	
Pour déterminer si $C_{ij} = +$ ou $ -\det A_{ij}$, 

\change
on peut se souvenir que l'on associe des signes en suivant le schéma d'un échiquier:
$$  A = 
\begin{pmatrix}
+ & - & + & - &\dots\\
- & + & - & + &\dots \\
+ & - & + & - &\dots\\
\vdots & \vdots & \vdots & \vdots &          
\end{pmatrix}
$$
Donc par exemple $C_{11} = +\det A_{11}$, $C_{12} = - \det A_{12}$,  $C_{21} = - \det A_{21}$ etc.

%%%%%%%%%%%%%%%%%%%%%%%%%%%%%%%%%%%%%%%%%%%%%%%%%%%%%%%%%%%
\diapo
Considérons la matrice $ A $ suivante et calculons 
\\
$A_{11}, C_{11}, A_{32}, C_{32}$.

\change
Pour la matrice $A_{11}$, il s'agit d'effacer dans $A$ la première ligne et la première colonne.

\change
$A_{11}$ est  donc la matrice 
$
\begin{pmatrix}
2 & 1\\
1 & 1  
\end{pmatrix}.$

\change
Le cofacteur $C_{11}$ est le déterminant de cette matrice multiplié par $-1$ à la puissance $1+1$

\change
c'est-à-dire $1$ fois $1$, ce qui donne $+1$.

\change
Pour la matrice $A_{32}$, il s'agit d'effacer dans $A$ la troisième ligne et la deuxième colonne.

\change
Ce qui donne
$
\begin{pmatrix}
1 & 3\\
4 & 1  
\end{pmatrix}.$

\change
Le cofacteur correspondant est égal au produit de $-1$ à la puissance $3+2$ avec le déterminant de la matrice $A_{32}$.

\change
On obtient donc $-1$ multiplié par  $1\times 1 - 4 \times 3$, c-à-d par $-11$, (...)

\change
ce qui donne $C_{32} =11$.

%%%%%%%%%%%%%%%%%%%%%%%%%%%%%%%%%%%%%%%%%%%%%%%%%%%%%%%%%%%
\diapo

Passons à présent au résultat principal de cette leçon : le développement du déterminant suivant une ligne ou une colonne. Voici l'énoncé du théorème.

\change
Commençons par le développement par rapport à une ligne $i$. On peut exprimer le déterminant d'une matrice~$A$

\change
comme la somme pour $j$ variant de $1$ à $n$ de

$(-1)^{i+j} \; a_{ij} \; \det A_{ij}$.\\

Dans cette somme l'indice de sommation est l'indice de colonne $j$, l'indice de ligne $i$ étant fixé.

\change
Ceci s'exprime plus simplement avec les cofacteurs : c'est

$\sum_{j=1}^n \; a_{ij} \; C_{ij}$.

\change
De façon analogue, voici la formule de développement par rapport à la colonne $j$

$\det A = \sum_{i=1}^n \; (-1)^{i+j} \; a_{ij} \; \det A_{ij}$

\change
(...)
\newpage	
c-à-d $= \sum_{i=1}^n a_{ij}C_{ij}$.
Dans cette formule, l'indice de sommation est l'indice de ligne $i$, alors que l'indice de colonne $j$ est fixé.

On peut remarquer qu'étant donné que $\det A=\det A^T$, les deux formules sont équivalentes, c-à-d qu'elles se déduisent l'une de l'autre.

%%%%%%%%%%%%%%%%%%%%%%%%%%%%%%%%%%%%%%%%%%%%%%%%%%%%%%%%%%%
\diapo
Retrouvons la formule des déterminants $3\times 3$, déjà présentée 
dans la règle de Sarrus, en développant par rapport à la première ligne.

\change
Considérons un déterminant $3\times 3$ quelconque, et écrivons son développement par rapport à la première ligne : en appliquant la première des formules du théorème précédent avec $i=1$ on obtient

\change
$a_{11}C_{11}+a_{12}C_{12}+a_{13}C_{13}. $\\

Remplaçons à présent les cofacteurs par leur valeur. Rappelez-vous que les signes des cofacteurs sont distribués selon un échiquier : $C_{11}$ prend un $+$, $C_{12}$ un $-$, et $C_{13}$ de nouveau un $+$.

\change
$C_{11}$ est donc égal à 

$+\begin{vmatrix}
a_{22}&a_{23}\\
a_{32}&a_{33}
\end{vmatrix}$,

\change
et de même pour $C_{12}$ et $C_{13}$. Calculons maintenant ces trois déterminants de taille $2\times 2$.

\change
(...)
\newpage	
Le premier vaut $a_{22}a_{33}-a_{32}a_{23}$ 

\change
et les deux autres se calculent de la même manière.

\change
En réordonnant les termes, on obtient cette formule, qui n'est autre que la formule de Sarrus 
vue à la première leçon de ce chapitre.


%%%%%%%%%%%%%%%%%%%%%%%%%%%%%%%%%%%%%%%%%%%%%%%%%%%%%%%%%%%
\diapo

\'Etudions un exemple en détail. On souhaite calculer le déterminant de cette matrice $4\times 4$.

On commence par repérer où la matrice comporte des zéros.

\change
Et on choisit de développer par rapport à la seconde colonne, car c'est là qu'il y a le plus de zéros.

\change
On obtient que 

$\det A = 0 \times C_{12} + 2 \times  C_{22} + 3 \times  C_{32}+0 \times  C_{42}$.\\

Grâce aux zéros, on n'a que deux cofacteurs à calculer !

\change
Remplaçons les cofacteurs $C_{22}$ et $C_{32}$ par leur valeur, en n'oubliant pas leur signe.

\change
On recommence en développant chacun de ces deux déter\-minants $3\times 3$,

\change
le premier par rapport à la première colonne,

\change
(...)
\newpage	
ce qui donne $+4$ fois ce déterminant $-0$ fois celui-ci $+1$ fois celui-là.

\change
Et on développe le deuxième déterminant $3\times 3$ par rapport à la deuxième ligne

\change
ce qui donne ceci. Il ne reste plus que des déterminants $2\times 2$. En les développant, on obtient que :

\change
$\det A =2 \times \big(4\times 5 +1\times(-4)\big) -3 \times \big(-4\times7 +1\times11  \big) $

\change
c-à-d $ 83$.


%%%%%%%%%%%%%%%%%%%%%%%%%%%%%%%%%%%%%%%%%%%%%%%%%%%%%%%%%%%
\diapo

Le développement par rapport à une ligne permet de ramener 
le calcul d'un déterminant $n\times n$ à celui de $n$ déterminants 
$(n-1)\times(n-1)$. 

\change
Par récurrence descendante, on se ramène ainsi au calcul de $n!$ sous-déterminants, ce qui devient vite fastidieux.

\change
C'est pourquoi le développement par rapport à une ligne ou une
colonne  n'est utile pour calculer explicitement un déterminant que
si la matrice de départ a beaucoup de zéros.

\change 
On commence donc souvent par faire apparaître
un maximum de zéros par des opérations élémentaires sur les lignes et les colonnes
qui ne modifient pas le déterminant, avant de développer le déterminant suivant la ligne ou la colonne qui a le plus de zéros.


%%%%%%%%%%%%%%%%%%%%%%%%%%%%%%%%%%%%%%%%%%%%%%%%%%%%%%%%%%%
\diapo

Soit $A$ une matrice carrée de taille $n$. \\

On associe à $A$ la matrice $C$ des cofacteurs, appelée \defi{comatrice}.

\change
La matrice $C$, de coefficients $C_{ij}$, s'écrit donc ainsi. \\

La comatrice nous fournit un moyen d'exprimer l'inverse de $A$ lorsque celle-ci existe.

\change
Plus précisément, on a le théorème suivant. On considère une matrice inversible $A$ et sa comatrice $C$. 

\change
On a alors

$\displaystyle A^{-1} = \frac{1}{\det A} \, C^T$

%%%%%%%%%%%%%%%%%%%%%%%%%%%%%%%%%%%%%%%%%%%%%%%%%%%%%%%%%%%
\diapo
Calculons l'inverse, si elle existe, de la matrice $A$ suivante.

\change
Tout d'abord, le calcul donne que $\det A = 2$. 

\change
En particulier, ce déterminant étant non nul, la matrice $A$ est inversible.

\change
La comatrice $C$ 
s'obtient en calculant $9$ déterminants
$2\times 2$ (sans oublier les signes $+/-$ des cofacteurs). 

\change
On trouve :
$$C = \begin{pmatrix}
1 & 1 & -1\\
-1 & 1 & 1\\
1 & -1 & 1        
      \end{pmatrix}
$$


\change
Et donc l'inverse de $A$ vaut
$\frac{1}{\det A} \cdot C^T$

\change
c-à-d 

$
 = \frac{1}{2}
\begin{pmatrix}
1 & -1 & 1\\
1 & 1 & -1\\
-1 & 1 & 1  
\end{pmatrix}
$.

%%%%%%%%%%%%%%%%%%%%%%%%%%%%%%%%%%%%%%%%%%%%%%%%%%%%%%%%%%%
\diapo
Entraînez-vous sur ces exercices pour vérifier que vous avez bien compris le cours !


\end{document}
