
%%%%%%%%%%%%%%%%%% PREAMBULE %%%%%%%%%%%%%%%%%%

\documentclass[aspectratio=169,utf8]{beamer}
%\documentclass[aspectratio=169,handout]{beamer}

\usetheme{Boadilla}
%\usecolortheme{seahorse}
\usecolortheme[RGB={245,66,24}]{structure}
\useoutertheme{infolines}

% packages
\usepackage{amsfonts,amsmath,amssymb,amsthm}
\usepackage[utf8]{inputenc}
\usepackage[T1]{fontenc}
\usepackage{lmodern}

\usepackage[francais]{babel}
\usepackage{fancybox}
\usepackage{graphicx}

\usepackage{float}
\usepackage{xfrac}

%\usepackage[usenames, x11names]{xcolor}
\usepackage{tikz}
\usepackage{pgfplots}
\usepackage{datetime}



%-----  Package unités -----
\usepackage{siunitx}
\sisetup{locale = FR,detect-all,per-mode = symbol}

%\usepackage{mathptmx}
%\usepackage{fouriernc}
%\usepackage{newcent}
%\usepackage[mathcal,mathbf]{euler}

%\usepackage{palatino}
%\usepackage{newcent}
% \usepackage[mathcal,mathbf]{euler}



% \usepackage{hyperref}
% \hypersetup{colorlinks=true, linkcolor=blue, urlcolor=blue,
% pdftitle={Exo7 - Exercices de mathématiques}, pdfauthor={Exo7}}


%section
% \usepackage{sectsty}
% \allsectionsfont{\bf}
%\sectionfont{\color{Tomato3}\upshape\selectfont}
%\subsectionfont{\color{Tomato4}\upshape\selectfont}

%----- Ensembles : entiers, reels, complexes -----
\newcommand{\Nn}{\mathbb{N}} \newcommand{\N}{\mathbb{N}}
\newcommand{\Zz}{\mathbb{Z}} \newcommand{\Z}{\mathbb{Z}}
\newcommand{\Qq}{\mathbb{Q}} \newcommand{\Q}{\mathbb{Q}}
\newcommand{\Rr}{\mathbb{R}} \newcommand{\R}{\mathbb{R}}
\newcommand{\Cc}{\mathbb{C}} 
\newcommand{\Kk}{\mathbb{K}} \newcommand{\K}{\mathbb{K}}

%----- Modifications de symboles -----
\renewcommand{\epsilon}{\varepsilon}
\renewcommand{\Re}{\mathop{\text{Re}}\nolimits}
\renewcommand{\Im}{\mathop{\text{Im}}\nolimits}
%\newcommand{\llbracket}{\left[\kern-0.15em\left[}
%\newcommand{\rrbracket}{\right]\kern-0.15em\right]}

\renewcommand{\ge}{\geqslant}
\renewcommand{\geq}{\geqslant}
\renewcommand{\le}{\leqslant}
\renewcommand{\leq}{\leqslant}
\renewcommand{\epsilon}{\varepsilon}

%----- Fonctions usuelles -----
\newcommand{\ch}{\mathop{\text{ch}}\nolimits}
\newcommand{\sh}{\mathop{\text{sh}}\nolimits}
\renewcommand{\tanh}{\mathop{\text{th}}\nolimits}
\newcommand{\cotan}{\mathop{\text{cotan}}\nolimits}
\newcommand{\Arcsin}{\mathop{\text{arcsin}}\nolimits}
\newcommand{\Arccos}{\mathop{\text{arccos}}\nolimits}
\newcommand{\Arctan}{\mathop{\text{arctan}}\nolimits}
\newcommand{\Argsh}{\mathop{\text{argsh}}\nolimits}
\newcommand{\Argch}{\mathop{\text{argch}}\nolimits}
\newcommand{\Argth}{\mathop{\text{argth}}\nolimits}
\newcommand{\pgcd}{\mathop{\text{pgcd}}\nolimits} 


%----- Commandes divers ------
\newcommand{\ii}{\mathrm{i}}
\newcommand{\dd}{\text{d}}
\newcommand{\id}{\mathop{\text{id}}\nolimits}
\newcommand{\Ker}{\mathop{\text{Ker}}\nolimits}
\newcommand{\Card}{\mathop{\text{Card}}\nolimits}
\newcommand{\Vect}{\mathop{\text{Vect}}\nolimits}
\newcommand{\Mat}{\mathop{\text{Mat}}\nolimits}
\newcommand{\rg}{\mathop{\text{rg}}\nolimits}
\newcommand{\tr}{\mathop{\text{tr}}\nolimits}


%----- Structure des exercices ------

\newtheoremstyle{styleexo}% name
{2ex}% Space above
{3ex}% Space below
{}% Body font
{}% Indent amount 1
{\bfseries} % Theorem head font
{}% Punctuation after theorem head
{\newline}% Space after theorem head 2
{}% Theorem head spec (can be left empty, meaning ‘normal’)

%\theoremstyle{styleexo}
\newtheorem{exo}{Exercice}
\newtheorem{ind}{Indications}
\newtheorem{cor}{Correction}


\newcommand{\exercice}[1]{} \newcommand{\finexercice}{}
%\newcommand{\exercice}[1]{{\tiny\texttt{#1}}\vspace{-2ex}} % pour afficher le numero absolu, l'auteur...
\newcommand{\enonce}{\begin{exo}} \newcommand{\finenonce}{\end{exo}}
\newcommand{\indication}{\begin{ind}} \newcommand{\finindication}{\end{ind}}
\newcommand{\correction}{\begin{cor}} \newcommand{\fincorrection}{\end{cor}}

\newcommand{\noindication}{\stepcounter{ind}}
\newcommand{\nocorrection}{\stepcounter{cor}}

\newcommand{\fiche}[1]{} \newcommand{\finfiche}{}
\newcommand{\titre}[1]{\centerline{\large \bf #1}}
\newcommand{\addcommand}[1]{}
\newcommand{\video}[1]{}

% Marge
\newcommand{\mymargin}[1]{\marginpar{{\small #1}}}

\def\noqed{\renewcommand{\qedsymbol}{}}


%----- Presentation ------
\setlength{\parindent}{0cm}

%\newcommand{\ExoSept}{\href{http://exo7.emath.fr}{\textbf{\textsf{Exo7}}}}

\definecolor{myred}{rgb}{0.93,0.26,0}
\definecolor{myorange}{rgb}{0.97,0.58,0}
\definecolor{myyellow}{rgb}{1,0.86,0}

\newcommand{\LogoExoSept}[1]{  % input : echelle
{\usefont{U}{cmss}{bx}{n}
\begin{tikzpicture}[scale=0.1*#1,transform shape]
  \fill[color=myorange] (0,0)--(4,0)--(4,-4)--(0,-4)--cycle;
  \fill[color=myred] (0,0)--(0,3)--(-3,3)--(-3,0)--cycle;
  \fill[color=myyellow] (4,0)--(7,4)--(3,7)--(0,3)--cycle;
  \node[scale=5] at (3.5,3.5) {Exo7};
\end{tikzpicture}}
}


\newcommand{\debutmontitre}{
  \author{} \date{} 
  \thispagestyle{empty}
  \hspace*{-10ex}
  \begin{minipage}{\textwidth}
    \titlepage  
  \vspace*{-2.5cm}
  \begin{center}
    \LogoExoSept{2.5}
  \end{center}
  \end{minipage}

  \vspace*{-0cm}
  
  % Astuce pour que le background ne soit pas discrétisé lors de la conversion pdf -> png
\begin{tikzpicture}
        \fill[opacity=0,green!60!black] (0,0)--++(0,0)--++(0,0)--++(0,0)--cycle; 
\end{tikzpicture}

% toc S'affiche trop tot :
% \tableofcontents[hideallsubsections, pausesections]
}

\newcommand{\finmontitre}{
  \end{frame}
  \setcounter{framenumber}{0}
} % ne marche pas pour une raison obscure

%----- Commandes supplementaires ------

% \usepackage[landscape]{geometry}
% \geometry{top=1cm, bottom=3cm, left=2cm, right=10cm, marginparsep=1cm
% }
% \usepackage[a4paper]{geometry}
% \geometry{top=2cm, bottom=2cm, left=2cm, right=2cm, marginparsep=1cm
% }

%\usepackage{standalone}


% New command Arnaud -- november 2011
\setbeamersize{text margin left=24ex}
% si vous modifier cette valeur il faut aussi
% modifier le decalage du titre pour compenser
% (ex : ici =+10ex, titre =-5ex

\theoremstyle{definition}
%\newtheorem{proposition}{Proposition}
%\newtheorem{exemple}{Exemple}
%\newtheorem{theoreme}{Théorème}
%\newtheorem{lemme}{Lemme}
%\newtheorem{corollaire}{Corollaire}
%\newtheorem*{remarque*}{Remarque}
%\newtheorem*{miniexercice}{Mini-exercices}
%\newtheorem{definition}{Définition}

% Commande tikz
\usetikzlibrary{calc}
\usetikzlibrary{patterns,arrows}
\usetikzlibrary{matrix}
\usetikzlibrary{fadings} 

%definition d'un terme
\newcommand{\defi}[1]{{\color{myorange}\textbf{\emph{#1}}}}
\newcommand{\evidence}[1]{{\color{blue}\textbf{\emph{#1}}}}
\newcommand{\assertion}[1]{\emph{\og#1\fg}}  % pour chapitre logique
%\renewcommand{\contentsname}{Sommaire}
\renewcommand{\contentsname}{}
\setcounter{tocdepth}{2}



%------ Figures ------

\def\myscale{1} % par défaut 
\newcommand{\myfigure}[2]{  % entrée : echelle, fichier figure
\def\myscale{#1}
\begin{center}
\footnotesize
{#2}
\end{center}}


%------ Encadrement ------

\usepackage{fancybox}


\newcommand{\mybox}[1]{
\setlength{\fboxsep}{7pt}
\begin{center}
\shadowbox{#1}
\end{center}}

\newcommand{\myboxinline}[1]{
\setlength{\fboxsep}{5pt}
\raisebox{-10pt}{
\shadowbox{#1}
}
}

%--------------- Commande beamer---------------
\newcommand{\beameronly}[1]{#1} % permet de mettre des pause dans beamer pas dans poly


\setbeamertemplate{navigation symbols}{}
\setbeamertemplate{footline}  % tiré du fichier beamerouterinfolines.sty
{
  \leavevmode%
  \hbox{%
  \begin{beamercolorbox}[wd=.333333\paperwidth,ht=2.25ex,dp=1ex,center]{author in head/foot}%
    % \usebeamerfont{author in head/foot}\insertshortauthor%~~(\insertshortinstitute)
    \usebeamerfont{section in head/foot}{\bf\insertshorttitle}
  \end{beamercolorbox}%
  \begin{beamercolorbox}[wd=.333333\paperwidth,ht=2.25ex,dp=1ex,center]{title in head/foot}%
    \usebeamerfont{section in head/foot}{\bf\insertsectionhead}
  \end{beamercolorbox}%
  \begin{beamercolorbox}[wd=.333333\paperwidth,ht=2.25ex,dp=1ex,right]{date in head/foot}%
    % \usebeamerfont{date in head/foot}\insertshortdate{}\hspace*{2em}
    \insertframenumber{} / \inserttotalframenumber\hspace*{2ex} 
  \end{beamercolorbox}}%
  \vskip0pt%
}


\definecolor{mygrey}{rgb}{0.5,0.5,0.5}
\setlength{\parindent}{0cm}
%\DeclareTextFontCommand{\helvetica}{\fontfamily{phv}\selectfont}

% background beamer
\definecolor{couleurhaut}{rgb}{0.85,0.9,1}  % creme
\definecolor{couleurmilieu}{rgb}{1,1,1}  % vert pale
\definecolor{couleurbas}{rgb}{0.85,0.9,1}  % blanc
\setbeamertemplate{background canvas}[vertical shading]%
[top=couleurhaut,middle=couleurmilieu,midpoint=0.4,bottom=couleurbas] 
%[top=fondtitre!05,bottom=fondtitre!60]



\makeatletter
\setbeamertemplate{theorem begin}
{%
  \begin{\inserttheoremblockenv}
  {%
    \inserttheoremheadfont
    \inserttheoremname
    \inserttheoremnumber
    \ifx\inserttheoremaddition\@empty\else\ (\inserttheoremaddition)\fi%
    \inserttheorempunctuation
  }%
}
\setbeamertemplate{theorem end}{\end{\inserttheoremblockenv}}

\newenvironment{theoreme}[1][]{%
   \setbeamercolor{block title}{fg=structure,bg=structure!40}
   \setbeamercolor{block body}{fg=black,bg=structure!10}
   \begin{block}{{\bf Th\'eor\`eme }#1}
}{%
   \end{block}%
}


\newenvironment{proposition}[1][]{%
   \setbeamercolor{block title}{fg=structure,bg=structure!40}
   \setbeamercolor{block body}{fg=black,bg=structure!10}
   \begin{block}{{\bf Proposition }#1}
}{%
   \end{block}%
}

\newenvironment{corollaire}[1][]{%
   \setbeamercolor{block title}{fg=structure,bg=structure!40}
   \setbeamercolor{block body}{fg=black,bg=structure!10}
   \begin{block}{{\bf Corollaire }#1}
}{%
   \end{block}%
}

\newenvironment{mydefinition}[1][]{%
   \setbeamercolor{block title}{fg=structure,bg=structure!40}
   \setbeamercolor{block body}{fg=black,bg=structure!10}
   \begin{block}{{\bf Définition} #1}
}{%
   \end{block}%
}

\newenvironment{lemme}[0]{%
   \setbeamercolor{block title}{fg=structure,bg=structure!40}
   \setbeamercolor{block body}{fg=black,bg=structure!10}
   \begin{block}{\bf Lemme}
}{%
   \end{block}%
}

\newenvironment{remarque}[1][]{%
   \setbeamercolor{block title}{fg=black,bg=structure!20}
   \setbeamercolor{block body}{fg=black,bg=structure!5}
   \begin{block}{Remarque #1}
}{%
   \end{block}%
}


\newenvironment{exemple}[1][]{%
   \setbeamercolor{block title}{fg=black,bg=structure!20}
   \setbeamercolor{block body}{fg=black,bg=structure!5}
   \begin{block}{{\bf Exemple }#1}
}{%
   \end{block}%
}


\newenvironment{miniexercice}[0]{%
   \setbeamercolor{block title}{fg=structure,bg=structure!20}
   \setbeamercolor{block body}{fg=black,bg=structure!5}
   \begin{block}{Mini-exercices}
}{%
   \end{block}%
}


\newenvironment{tp}[0]{%
   \setbeamercolor{block title}{fg=structure,bg=structure!40}
   \setbeamercolor{block body}{fg=black,bg=structure!10}
   \begin{block}{\bf Travaux pratiques}
}{%
   \end{block}%
}
\newenvironment{exercicecours}[1][]{%
   \setbeamercolor{block title}{fg=structure,bg=structure!40}
   \setbeamercolor{block body}{fg=black,bg=structure!10}
   \begin{block}{{\bf Exercice }#1}
}{%
   \end{block}%
}
\newenvironment{algo}[1][]{%
   \setbeamercolor{block title}{fg=structure,bg=structure!40}
   \setbeamercolor{block body}{fg=black,bg=structure!10}
   \begin{block}{{\bf Algorithme}\hfill{\color{gray}\texttt{#1}}}
}{%
   \end{block}%
}


\setbeamertemplate{proof begin}{
   \setbeamercolor{block title}{fg=black,bg=structure!20}
   \setbeamercolor{block body}{fg=black,bg=structure!5}
   \begin{block}{{\footnotesize Démonstration}}
   \footnotesize
   \smallskip}
\setbeamertemplate{proof end}{%
   \end{block}}
\setbeamertemplate{qed symbol}{\openbox}


\makeatother
\usecolortheme[RGB={102,102,255}]{structure}

% Commande spécifique à ce chapitre
\newcommand{\construc}{\mathcal{C}}
\newcommand{\plan}{\mathcal{P}}
\newcommand{\cercle}{\mathcal{C}}
   
%%%%%%%%%%%%%%%%%%%%%%%%%%%%%%%%%%%%%%%%%%%%%%%%%%%%%%%%%%%%%
%%%%%%%%%%%%%%%%%%%%%%%%%%%%%%%%%%%%%%%%%%%%%%%%%%%%%%%%%%%%%


\begin{document}


\title{{\bf La règle et le compas}}
\subtitle{Corps et nombres constructibles}

\begin{frame}
  
  \debutmontitre

  \pause

{\footnotesize
\hfill
\setbeamercovered{transparent=50}
\begin{minipage}{0.6\textwidth}
  \begin{itemize}
    \item<3-> Nombre constructible et extensions quadratiques
    \item<4-> Corollaire
    \item<5-> Exemples
    \item<6-> Preuve
  \end{itemize}
\end{minipage}
}

\end{frame}

\setcounter{framenumber}{0}


%%%%%%%%%%%%%%%%%%%%%%%%%%%%%%%%%%%%%%%%%%%%%%%%%%%%%%%%%%%%%%%%
\section{Nombre constructible et extensions quadratiques}

\begin{frame}

\begin{theoreme}[de Wantzel]
\label{th:wantzel}
Un nombre réel $x$ est constructible si et seulement s'il existe 
des extensions quadratiques 
$$\Qq = K_0 \subset K_1 \subset \cdots \subset K_r$$
telles que $x \in K_r$
\end{theoreme}

\pause

\begin{itemize}
  \item Chacune des extensions est quadratique
\pause
  \item $[K_{i+1}:K_i]=2$
\pause
  \item $K_{i+1} = K_i(\sqrt{\delta_i})$ pour un certain $\delta_i \in K_i$ 
\end{itemize}

\pause
$$\Qq \subset \Qq(\sqrt{\delta_0}) \subset \Qq(\sqrt{\delta_0})(\sqrt{\delta_1}) \subset \cdots$$
\end{frame}


%%%%%%%%%%%%%%%%%%%%%%%%%%%%%%%%%%%%%%%%%%%%%%%%%%%%%%%%%%%%%%%%
\section{Corollaire}

\begin{frame}
  \begin{corollaire}
Tout nombre réel constructible est un nombre algébrique
dont le degré algébrique est de la forme $2^n$, $n\ge 0$ 
\end{corollaire}

\pause

\begin{proof}
Soit $x$ un nombre constructible
\pause
\begin{itemize}
  \item 
  \begin{itemize}
    \item Théorème de Wantzel :
il existe des extensions quadratiques 
$\Qq = K_0 \subset K_1 \subset \cdots \subset K_r$
telles que $x \in K_r$
\pause
    \item Donc $x$ appartient à une extension de $\Qq$ de degré fini
\pause
    \item Ainsi $x$ est un nombre algébrique
  \end{itemize}
\pause
  \item 
  \begin{itemize}
    \item On sait de plus que $[K_{i+1}:K_i]=2$
\pause
    \item Donc $[K_r:\Qq]=2^r$
\pause
    \item Comme $\Qq(x) \subset K_r$ : $[K_r:\Qq(x)] \times [\Qq(x):\Qq]=[K_r:\Qq]$
\pause    
    \item Donc $[\Qq(x):\Qq]$ divise $[K_r:\Qq]=2^r$
\pause    
    \item $[\Qq(x):\Qq]=2^n$ \qedhere
  \end{itemize}
 
\end{itemize}
\end{proof}

\end{frame}



%%%%%%%%%%%%%%%%%%%%%%%%%%%%%%%%%%%%%%%%%%%%%%%%%%%%%%%%%%%%%%%%
\section{Exemples}

\begin{frame}

\evidence{Intersection de deux droites}

\begin{minipage}{0.59\textwidth}
\begin{itemize}
  \uncover<3->{\item Deux droites $(AB)$, $(A'B')$}
  \uncover<4->{\item $A(0,1)$, $B(1,-1)$, $A'(0,-2)$, $B'(1,1)$}
  \uncover<5->{\item Coordonnées dans $K=\Qq$}
  \uncover<6->{\item \'Equations $2x+y=1$ et $3x-y=2$}
  \uncover<7->{\item $P = (AB) \cap (A'B')$}
  \uncover<8->{\item $P(\frac 35, -\frac 15)$}
  \uncover<9->{\item Coordonnées de $P$ sont dans $K=\Qq$}
  \uncover<10->{\item Pas d'extension de corps}
\end{itemize}
\end{minipage}
\begin{minipage}{0.39\textwidth}
\uncover<2->{\myfigure{0.7}{\hspace*{-0.5em}
\tikzinput{fig_compas46}
}}
\end{minipage}

\end{frame}


\begin{frame}

\evidence{Intersection d'une droite et d'un cercle}

\vspace*{-2ex}
\begin{minipage}{0.61\textwidth}
\vspace*{-3ex}
\begin{itemize}
  \uncover<2->{\item Droite passant par $A(0,1)$ et $B(1,-1)$}
  \uncover<3->{\item \'Equation $2x+y=1$}
  \uncover<4->{\item Cercle de centre $A'(2,1)$ passant par $B'(-1,1)$}
  \uncover<5->{\item \'Equation $(x-2)^2+(y-1)^2=9$}
\end{itemize}
\end{minipage}
\begin{minipage}{0.38\textwidth}
\vspace*{5ex}\myfigure{0.6}{\hspace*{-0.5em}
\tikzinput{fig_compas47}
}
\end{minipage}

\pause\pause\pause\pause\pause
\vspace*{-9ex}
\begin{itemize}[<+->]
  \item $P\left(\frac 1 5 \left(2- \sqrt{29}\right),\frac 1 5 \left(1+2\sqrt{29}\right) \right)$
  \item $P'\left(\frac 1 5 \left(2 + \sqrt{29}\right),\frac 1 5 \left(1-2\sqrt{29}\right) \right)$
  \item $\alpha+\beta\sqrt{\delta}$ \qquad $\delta = 29$ \qquad $\alpha,\beta \in \Qq$
  \item $K = \Qq(\sqrt{29})$
\end{itemize}
\end{frame}


\begin{frame}  
\evidence{Intersection de deux cercles}


\begin{minipage}{0.63\textwidth}
\begin{itemize}
  \setlength{\itemsep}{7pt}
  \uncover<2->{\item $\cercle((-1,0),2)$ : $(x+1)^2+y^2=4$}
  \uncover<3->{\item $\cercle((2,1),\sqrt{5})$ : $(x-2)^2+(y-1)^2=5$}
  \uncover<4->{\item $\left(\frac{1}{20}\left( 7 - \sqrt{79}\right),\frac{3}{20}\left( 3 + \sqrt{79}\right)\right)$}
  \uncover<5->{\item $\left(\frac{1}{20}\left( 7 + \sqrt{79}\right),\frac{3}{20}\left( 3 - \sqrt{79}\right)\right)$}
  \uncover<6->{\item $\alpha+\beta\sqrt{\delta}$ \qquad $\delta = 79$ \qquad $\alpha,\beta \in \Qq$}
  \uncover<7->{\item $K = \Qq(\sqrt{79})$  }
\end{itemize}
 
\end{minipage}
\begin{minipage}{0.34\textwidth}
\myfigure{0.7}{\hspace*{-2.5em}
\tikzinput{fig_compas48}
}    
\end{minipage}
\end{frame}



%%%%%%%%%%%%%%%%%%%%%%%%%%%%%%%%%%%%%%%%%%%%%%%%%%%%%%%%%%%%%%%%
\section{Preuve}


\begin{frame}

\begin{theoreme}[de Wantzel]
\label{th:wantzel}
Un nombre réel $x$ est constructible si et seulement s'il existe 
des extensions quadratiques 
$$\Qq = K_0 \subset K_1 \subset \cdots \subset K_r$$
telles que $x \in K_r$
\end{theoreme}

\pause
\bigskip

Sens <<\,facile\,>>
\pause
\begin{itemize}
  \item Si $x\ge 0$ constructible, alors $\sqrt{x}$ est constructible
  \pause
  \item Les nombres de $K_1 = \Qq(\sqrt{\delta_0})$ sont constructibles
  \pause
  \item Par récurrence tout élément de $K_1,K_2$, $K_3$,... est constructible
\end{itemize}

\end{frame}


\begin{frame}

\evidence{Preuve du théorème de Wantzel}

\pause

\begin{itemize}
  \item Construction par étapes des points constructibles  $\construc_0$, $\construc_1$, $\construc_2$,\ldots
 \pause 
  \item $\construc_{j+1}$ s'obtient à partir de $\construc_j$ en ajoutant 
  les intersections des droites et des cercles tracés à partir de $\construc_j$
\pause  
  \item Soit $K$ le plus petit corps contenant les coordonnées des points de $\construc_j$
\pause  
  \item Soit $P$ un point de $\construc_{j+1}$
\end{itemize}

\pause
\evidence{Premier cas.} \emph{$P$ est l'intersection de deux droites}

\pause
\begin{itemize}[<+->]

  \item $A(x_A,y_A)$ et $B(x_B,y_B)$ avec $x_A,y_A,x_B,y_B \in K$
  \item Droite $(AB)$ : $y = \frac{y_B-y_A}{x_B-x_A} (x-x_A) + y_A$ 
  \item Droite $(AB)$ : $ax+by=c$ avec $a,b,c \in K$
  \item Autre droite : $a'x+b'y=c'$ avec $a',b',c' \in K$
  \item $P$ point d'intersection : $\left(\frac{cb'-c'b}{ab'-a'b} , \frac{ac'-a'c}{ab'-a'b} \right)$
  \item $P$ a ses coordonnées dans $K$
  \item Pas besoin d'extension de $K$
\end{itemize}

\end{frame}


\begin{frame}
\evidence{Deuxième cas.} \emph{$P$ intersection d'une droite et d'un cercle} 

\pause
\begin{itemize}[<+->]
  \item Droite $ax+by=c$ avec $a,b,c \in K$
  \item Cercle $(x-x_0)^2+(y-y_0)^2=r^2$, $x_0,y_0,r^2 \in K$
  \item Deux points d'intersection $(x,y)$, $(x',y')$
  \item 
  $\begin{array}{rcl}
  \delta &=& -2\,x_{{0}}{a}^{3}by_{{0}}+2\,y_{{0}}{a}^{2}cb-{b}^{2}{y_{{0}}}^{2}{a}^{2}+{b}^
{2}{r}^{2}{a}^{2}\\
&&+2\,{a}^{3}x_{{0}}c-{a}^{4}{x_{{0}}}^{2}-{a}^{2}{c}^{
2}+{a}^{4}{r}^{2} \in K
\end{array}$
  \item $x =- \frac ba \frac{1}{a^2+b^2} \left(-x_{{0}}ab+y_{{0}}{a}^{2}+cb - \frac cb (a^2+b^2) \mathbf{+} \sqrt \delta \right)$
 
 $y = \frac{c-ax}{b}$
  
  \item $x' =- \frac ba \frac{1}{a^2+b^2} \left(-x_{{0}}ab+y_{{0}}{a}^{2}+cb - \frac cb (a^2+b^2) \mathbf{-} \sqrt \delta \right)$
  
  $y' = \frac{c-ax'}{b}$

  
  \item Coordonnées de la forme $\alpha+\beta\sqrt{\delta}$ avec $\alpha,\beta \in K$, $\delta \in K$
  \item Extension quadratique $K(\sqrt \delta)$
\end{itemize}
\end{frame}


\begin{frame}  
\evidence{Troisième cas.} \emph{$P$ appartient à l'intersection de deux cercles}

\pause
Coordonnées de l'intersection sont aussi de la forme $\alpha+\beta\sqrt\delta$ avec
$\delta \in K$ fixé et $\alpha,\beta\in K$

\bigskip
\pause

\evidence{Conclusion.}
\begin{itemize}
  \item Dans tous les cas, les coordonnées de $P$ sont dans une extension quadratique du corps $K$
 
  \item Une récurrence termine la preuve du théorème
\end{itemize}

\end{frame}



\end{document}
