
%%%%%%%%%%%%%%%%%% PREAMBULE %%%%%%%%%%%%%%%%%%

\documentclass[aspectratio=169,utf8]{beamer}
%\documentclass[aspectratio=169,handout]{beamer}

\usetheme{Boadilla}
%\usecolortheme{seahorse}
\usecolortheme[RGB={245,66,24}]{structure}
\useoutertheme{infolines}

% packages
\usepackage{amsfonts,amsmath,amssymb,amsthm}
\usepackage[utf8]{inputenc}
\usepackage[T1]{fontenc}
\usepackage{lmodern}

\usepackage[francais]{babel}
\usepackage{fancybox}
\usepackage{graphicx}

\usepackage{float}
\usepackage{xfrac}

%\usepackage[usenames, x11names]{xcolor}
\usepackage{tikz}
\usepackage{pgfplots}
\usepackage{datetime}



%-----  Package unités -----
\usepackage{siunitx}
\sisetup{locale = FR,detect-all,per-mode = symbol}

%\usepackage{mathptmx}
%\usepackage{fouriernc}
%\usepackage{newcent}
%\usepackage[mathcal,mathbf]{euler}

%\usepackage{palatino}
%\usepackage{newcent}
% \usepackage[mathcal,mathbf]{euler}



% \usepackage{hyperref}
% \hypersetup{colorlinks=true, linkcolor=blue, urlcolor=blue,
% pdftitle={Exo7 - Exercices de mathématiques}, pdfauthor={Exo7}}


%section
% \usepackage{sectsty}
% \allsectionsfont{\bf}
%\sectionfont{\color{Tomato3}\upshape\selectfont}
%\subsectionfont{\color{Tomato4}\upshape\selectfont}

%----- Ensembles : entiers, reels, complexes -----
\newcommand{\Nn}{\mathbb{N}} \newcommand{\N}{\mathbb{N}}
\newcommand{\Zz}{\mathbb{Z}} \newcommand{\Z}{\mathbb{Z}}
\newcommand{\Qq}{\mathbb{Q}} \newcommand{\Q}{\mathbb{Q}}
\newcommand{\Rr}{\mathbb{R}} \newcommand{\R}{\mathbb{R}}
\newcommand{\Cc}{\mathbb{C}} 
\newcommand{\Kk}{\mathbb{K}} \newcommand{\K}{\mathbb{K}}

%----- Modifications de symboles -----
\renewcommand{\epsilon}{\varepsilon}
\renewcommand{\Re}{\mathop{\text{Re}}\nolimits}
\renewcommand{\Im}{\mathop{\text{Im}}\nolimits}
%\newcommand{\llbracket}{\left[\kern-0.15em\left[}
%\newcommand{\rrbracket}{\right]\kern-0.15em\right]}

\renewcommand{\ge}{\geqslant}
\renewcommand{\geq}{\geqslant}
\renewcommand{\le}{\leqslant}
\renewcommand{\leq}{\leqslant}
\renewcommand{\epsilon}{\varepsilon}

%----- Fonctions usuelles -----
\newcommand{\ch}{\mathop{\text{ch}}\nolimits}
\newcommand{\sh}{\mathop{\text{sh}}\nolimits}
\renewcommand{\tanh}{\mathop{\text{th}}\nolimits}
\newcommand{\cotan}{\mathop{\text{cotan}}\nolimits}
\newcommand{\Arcsin}{\mathop{\text{arcsin}}\nolimits}
\newcommand{\Arccos}{\mathop{\text{arccos}}\nolimits}
\newcommand{\Arctan}{\mathop{\text{arctan}}\nolimits}
\newcommand{\Argsh}{\mathop{\text{argsh}}\nolimits}
\newcommand{\Argch}{\mathop{\text{argch}}\nolimits}
\newcommand{\Argth}{\mathop{\text{argth}}\nolimits}
\newcommand{\pgcd}{\mathop{\text{pgcd}}\nolimits} 


%----- Commandes divers ------
\newcommand{\ii}{\mathrm{i}}
\newcommand{\dd}{\text{d}}
\newcommand{\id}{\mathop{\text{id}}\nolimits}
\newcommand{\Ker}{\mathop{\text{Ker}}\nolimits}
\newcommand{\Card}{\mathop{\text{Card}}\nolimits}
\newcommand{\Vect}{\mathop{\text{Vect}}\nolimits}
\newcommand{\Mat}{\mathop{\text{Mat}}\nolimits}
\newcommand{\rg}{\mathop{\text{rg}}\nolimits}
\newcommand{\tr}{\mathop{\text{tr}}\nolimits}


%----- Structure des exercices ------

\newtheoremstyle{styleexo}% name
{2ex}% Space above
{3ex}% Space below
{}% Body font
{}% Indent amount 1
{\bfseries} % Theorem head font
{}% Punctuation after theorem head
{\newline}% Space after theorem head 2
{}% Theorem head spec (can be left empty, meaning ‘normal’)

%\theoremstyle{styleexo}
\newtheorem{exo}{Exercice}
\newtheorem{ind}{Indications}
\newtheorem{cor}{Correction}


\newcommand{\exercice}[1]{} \newcommand{\finexercice}{}
%\newcommand{\exercice}[1]{{\tiny\texttt{#1}}\vspace{-2ex}} % pour afficher le numero absolu, l'auteur...
\newcommand{\enonce}{\begin{exo}} \newcommand{\finenonce}{\end{exo}}
\newcommand{\indication}{\begin{ind}} \newcommand{\finindication}{\end{ind}}
\newcommand{\correction}{\begin{cor}} \newcommand{\fincorrection}{\end{cor}}

\newcommand{\noindication}{\stepcounter{ind}}
\newcommand{\nocorrection}{\stepcounter{cor}}

\newcommand{\fiche}[1]{} \newcommand{\finfiche}{}
\newcommand{\titre}[1]{\centerline{\large \bf #1}}
\newcommand{\addcommand}[1]{}
\newcommand{\video}[1]{}

% Marge
\newcommand{\mymargin}[1]{\marginpar{{\small #1}}}

\def\noqed{\renewcommand{\qedsymbol}{}}


%----- Presentation ------
\setlength{\parindent}{0cm}

%\newcommand{\ExoSept}{\href{http://exo7.emath.fr}{\textbf{\textsf{Exo7}}}}

\definecolor{myred}{rgb}{0.93,0.26,0}
\definecolor{myorange}{rgb}{0.97,0.58,0}
\definecolor{myyellow}{rgb}{1,0.86,0}

\newcommand{\LogoExoSept}[1]{  % input : echelle
{\usefont{U}{cmss}{bx}{n}
\begin{tikzpicture}[scale=0.1*#1,transform shape]
  \fill[color=myorange] (0,0)--(4,0)--(4,-4)--(0,-4)--cycle;
  \fill[color=myred] (0,0)--(0,3)--(-3,3)--(-3,0)--cycle;
  \fill[color=myyellow] (4,0)--(7,4)--(3,7)--(0,3)--cycle;
  \node[scale=5] at (3.5,3.5) {Exo7};
\end{tikzpicture}}
}


\newcommand{\debutmontitre}{
  \author{} \date{} 
  \thispagestyle{empty}
  \hspace*{-10ex}
  \begin{minipage}{\textwidth}
    \titlepage  
  \vspace*{-2.5cm}
  \begin{center}
    \LogoExoSept{2.5}
  \end{center}
  \end{minipage}

  \vspace*{-0cm}
  
  % Astuce pour que le background ne soit pas discrétisé lors de la conversion pdf -> png
\begin{tikzpicture}
        \fill[opacity=0,green!60!black] (0,0)--++(0,0)--++(0,0)--++(0,0)--cycle; 
\end{tikzpicture}

% toc S'affiche trop tot :
% \tableofcontents[hideallsubsections, pausesections]
}

\newcommand{\finmontitre}{
  \end{frame}
  \setcounter{framenumber}{0}
} % ne marche pas pour une raison obscure

%----- Commandes supplementaires ------

% \usepackage[landscape]{geometry}
% \geometry{top=1cm, bottom=3cm, left=2cm, right=10cm, marginparsep=1cm
% }
% \usepackage[a4paper]{geometry}
% \geometry{top=2cm, bottom=2cm, left=2cm, right=2cm, marginparsep=1cm
% }

%\usepackage{standalone}


% New command Arnaud -- november 2011
\setbeamersize{text margin left=24ex}
% si vous modifier cette valeur il faut aussi
% modifier le decalage du titre pour compenser
% (ex : ici =+10ex, titre =-5ex

\theoremstyle{definition}
%\newtheorem{proposition}{Proposition}
%\newtheorem{exemple}{Exemple}
%\newtheorem{theoreme}{Théorème}
%\newtheorem{lemme}{Lemme}
%\newtheorem{corollaire}{Corollaire}
%\newtheorem*{remarque*}{Remarque}
%\newtheorem*{miniexercice}{Mini-exercices}
%\newtheorem{definition}{Définition}

% Commande tikz
\usetikzlibrary{calc}
\usetikzlibrary{patterns,arrows}
\usetikzlibrary{matrix}
\usetikzlibrary{fadings} 

%definition d'un terme
\newcommand{\defi}[1]{{\color{myorange}\textbf{\emph{#1}}}}
\newcommand{\evidence}[1]{{\color{blue}\textbf{\emph{#1}}}}
\newcommand{\assertion}[1]{\emph{\og#1\fg}}  % pour chapitre logique
%\renewcommand{\contentsname}{Sommaire}
\renewcommand{\contentsname}{}
\setcounter{tocdepth}{2}



%------ Figures ------

\def\myscale{1} % par défaut 
\newcommand{\myfigure}[2]{  % entrée : echelle, fichier figure
\def\myscale{#1}
\begin{center}
\footnotesize
{#2}
\end{center}}


%------ Encadrement ------

\usepackage{fancybox}


\newcommand{\mybox}[1]{
\setlength{\fboxsep}{7pt}
\begin{center}
\shadowbox{#1}
\end{center}}

\newcommand{\myboxinline}[1]{
\setlength{\fboxsep}{5pt}
\raisebox{-10pt}{
\shadowbox{#1}
}
}

%--------------- Commande beamer---------------
\newcommand{\beameronly}[1]{#1} % permet de mettre des pause dans beamer pas dans poly


\setbeamertemplate{navigation symbols}{}
\setbeamertemplate{footline}  % tiré du fichier beamerouterinfolines.sty
{
  \leavevmode%
  \hbox{%
  \begin{beamercolorbox}[wd=.333333\paperwidth,ht=2.25ex,dp=1ex,center]{author in head/foot}%
    % \usebeamerfont{author in head/foot}\insertshortauthor%~~(\insertshortinstitute)
    \usebeamerfont{section in head/foot}{\bf\insertshorttitle}
  \end{beamercolorbox}%
  \begin{beamercolorbox}[wd=.333333\paperwidth,ht=2.25ex,dp=1ex,center]{title in head/foot}%
    \usebeamerfont{section in head/foot}{\bf\insertsectionhead}
  \end{beamercolorbox}%
  \begin{beamercolorbox}[wd=.333333\paperwidth,ht=2.25ex,dp=1ex,right]{date in head/foot}%
    % \usebeamerfont{date in head/foot}\insertshortdate{}\hspace*{2em}
    \insertframenumber{} / \inserttotalframenumber\hspace*{2ex} 
  \end{beamercolorbox}}%
  \vskip0pt%
}


\definecolor{mygrey}{rgb}{0.5,0.5,0.5}
\setlength{\parindent}{0cm}
%\DeclareTextFontCommand{\helvetica}{\fontfamily{phv}\selectfont}

% background beamer
\definecolor{couleurhaut}{rgb}{0.85,0.9,1}  % creme
\definecolor{couleurmilieu}{rgb}{1,1,1}  % vert pale
\definecolor{couleurbas}{rgb}{0.85,0.9,1}  % blanc
\setbeamertemplate{background canvas}[vertical shading]%
[top=couleurhaut,middle=couleurmilieu,midpoint=0.4,bottom=couleurbas] 
%[top=fondtitre!05,bottom=fondtitre!60]



\makeatletter
\setbeamertemplate{theorem begin}
{%
  \begin{\inserttheoremblockenv}
  {%
    \inserttheoremheadfont
    \inserttheoremname
    \inserttheoremnumber
    \ifx\inserttheoremaddition\@empty\else\ (\inserttheoremaddition)\fi%
    \inserttheorempunctuation
  }%
}
\setbeamertemplate{theorem end}{\end{\inserttheoremblockenv}}

\newenvironment{theoreme}[1][]{%
   \setbeamercolor{block title}{fg=structure,bg=structure!40}
   \setbeamercolor{block body}{fg=black,bg=structure!10}
   \begin{block}{{\bf Th\'eor\`eme }#1}
}{%
   \end{block}%
}


\newenvironment{proposition}[1][]{%
   \setbeamercolor{block title}{fg=structure,bg=structure!40}
   \setbeamercolor{block body}{fg=black,bg=structure!10}
   \begin{block}{{\bf Proposition }#1}
}{%
   \end{block}%
}

\newenvironment{corollaire}[1][]{%
   \setbeamercolor{block title}{fg=structure,bg=structure!40}
   \setbeamercolor{block body}{fg=black,bg=structure!10}
   \begin{block}{{\bf Corollaire }#1}
}{%
   \end{block}%
}

\newenvironment{mydefinition}[1][]{%
   \setbeamercolor{block title}{fg=structure,bg=structure!40}
   \setbeamercolor{block body}{fg=black,bg=structure!10}
   \begin{block}{{\bf Définition} #1}
}{%
   \end{block}%
}

\newenvironment{lemme}[0]{%
   \setbeamercolor{block title}{fg=structure,bg=structure!40}
   \setbeamercolor{block body}{fg=black,bg=structure!10}
   \begin{block}{\bf Lemme}
}{%
   \end{block}%
}

\newenvironment{remarque}[1][]{%
   \setbeamercolor{block title}{fg=black,bg=structure!20}
   \setbeamercolor{block body}{fg=black,bg=structure!5}
   \begin{block}{Remarque #1}
}{%
   \end{block}%
}


\newenvironment{exemple}[1][]{%
   \setbeamercolor{block title}{fg=black,bg=structure!20}
   \setbeamercolor{block body}{fg=black,bg=structure!5}
   \begin{block}{{\bf Exemple }#1}
}{%
   \end{block}%
}


\newenvironment{miniexercice}[0]{%
   \setbeamercolor{block title}{fg=structure,bg=structure!20}
   \setbeamercolor{block body}{fg=black,bg=structure!5}
   \begin{block}{Mini-exercices}
}{%
   \end{block}%
}


\newenvironment{tp}[0]{%
   \setbeamercolor{block title}{fg=structure,bg=structure!40}
   \setbeamercolor{block body}{fg=black,bg=structure!10}
   \begin{block}{\bf Travaux pratiques}
}{%
   \end{block}%
}
\newenvironment{exercicecours}[1][]{%
   \setbeamercolor{block title}{fg=structure,bg=structure!40}
   \setbeamercolor{block body}{fg=black,bg=structure!10}
   \begin{block}{{\bf Exercice }#1}
}{%
   \end{block}%
}
\newenvironment{algo}[1][]{%
   \setbeamercolor{block title}{fg=structure,bg=structure!40}
   \setbeamercolor{block body}{fg=black,bg=structure!10}
   \begin{block}{{\bf Algorithme}\hfill{\color{gray}\texttt{#1}}}
}{%
   \end{block}%
}


\setbeamertemplate{proof begin}{
   \setbeamercolor{block title}{fg=black,bg=structure!20}
   \setbeamercolor{block body}{fg=black,bg=structure!5}
   \begin{block}{{\footnotesize Démonstration}}
   \footnotesize
   \smallskip}
\setbeamertemplate{proof end}{%
   \end{block}}
\setbeamertemplate{qed symbol}{\openbox}


\makeatother
\usecolortheme[RGB={51,102,51}]{structure}

% Commande spécifique à ce chapitre
%\newcommand{\alenvers}[1]{\rotatebox[origin=c]{180}{#1}}

%%%%%%%%%%%%%%%%%%%%%%%%%%%%%%%%%%%%%%%%%%%%%%%%%%%%%%%%%%%%%
%%%%%%%%%%%%%%%%%%%%%%%%%%%%%%%%%%%%%%%%%%%%%%%%%%%%%%%%%%%%%


\begin{document}


\title{{\bf \'Equations différentielles}}
\subtitle{Problèmes conduisant à des équations différentielles}

\begin{frame}
  
  \debutmontitre

  \pause

{\footnotesize
\hfill
\setbeamercovered{transparent=50}
\begin{minipage}{0.6\textwidth}
  \begin{itemize}
    \item<3-> Parachutiste
    \item<4-> Demi-vie
    \item<5-> Modèles d'évolution
    \item<6-> Masse attachée à un ressort
  \end{itemize}
\end{minipage}
}

\end{frame}

\setcounter{framenumber}{0}

%%%%%%%%%%%%%%%%%%%%%%%%%%%%%%%%%%%%%%%%%%%%%%%%%%%%%%%%%%%%%%%%
\section*{Parachutiste}

\begin{frame}
\begin{minipage}{0.49\textwidth}
\evidence{Parachutiste}
\uncover<2->{$$\uncover<1-4>{m}\frac{\dd v(t)}{\dd t}= 
\uncover<1-4>{m}g \uncover<4->{- \uncover<1-4>{m} f\cdot v(t)}$$}
\begin{itemize}
  \uncover<6->{\item $g$ constante de gravitation}
  \uncover<7->{\item $f$ coefficient de frottement}
  \uncover<8->{\item vitesse $v(t)$ inconnue}
\end{itemize}
\end{minipage}
\begin{minipage}{0.49\textwidth}
  
\end{minipage}\begin{minipage}{0.49\textwidth}
  \myfigure{1}{
    \tikzinput{fig_equadiff02}
  }  
\end{minipage}

\begin{itemize}
  \uncover<9->{\item \evidence{\'Equation homogène} $v'(t)=-f \cdot v(t)$.}
  \uncover<10->{Solutions : $v(t)= ke^{-ft}$}

  \item \uncover<11->{\evidence{Solution particulière}}
  \begin{itemize}
    \uncover<12->{\item solution particulière sous la forme $v_p(t)=k(t)e^{-ft}$}
    \uncover<13->{\item $v_p'(t) = k'(t)  e^{-ft} -f k(t)e^{-ft}$}
    \uncover<14->{\item $v_p$ solution $\iff$  $k'(t)e^{-ft}=g$}
    \uncover<15->{\item ainsi $k'(t)=ge^{ft}$ donc $k(t)=\frac{g}{f}e^{ft}$}
    \uncover<16->{\item $v_p(t)=\frac{g}{f}$}
  \end{itemize}
  
  \item \uncover<17->{\evidence{Solutions générales} : $v(t)=\frac{g}{f}+k e^{-ft}$, $k\in\Rr$}
  
\end{itemize}

\end{frame}


\begin{frame}
\begin{itemize}
  \item \evidence{Condition initiale.}
  \pause
  Si $v(0)=0$, alors $\displaystyle v(t)=\frac{g}{f}-\frac{g}{f}e^{-ft}$

   \pause\vspace*{-1ex} 
  
  \myfigure{1.2}{
    \tikzinput{fig_equadiff05}
  }

  \pause\vspace*{-1ex}
  
  \item \evidence{Vitesse limite}
  \pause
  \begin{itemize}
    \item lorsque $t\to+\infty$, $v(t) \to v_\infty = \frac{g}{f}$
    \pause
    \item on mesure que $v_\infty$ vaut environ \SI{5}{\meter\per\second} (\SI{20}{\kilo\meter\per\hour})
  \end{itemize}
  
  \pause
  \item \evidence{Position}
  \begin{itemize}
  \pause
    \item $v(t) = \frac{\dd x(t)}{\dd t}$
    \pause
    \item trouver la position $x$ revient à trouver une primitive de $v$
    \pause
    \item $x(t) = \frac{g}{f} t + \frac{g}{f^2}\big(e^{-ft}-1\big)$, $x(0)=0$
  \end{itemize}
\end{itemize}
\end{frame}



%%%%%%%%%%%%%%%%%%%%%%%%%%%%%%%%%%%%%%%%%%%%%%%%%%%%%%%%%%%%%%%%
\section*{Demi-vie}

\begin{frame}

\evidence{Demi-vie}
\pause

\begin{itemize}
  \item Dans un tissu radioactif, la vitesse de désintégration des noyaux radioactifs 
est proportionnelle au nombre de noyaux radioactifs $N(t)$ présents dans le
tissu à l'instant $t$
  \pause
  \item Il existe $\lambda>0$ telle que :
$$N'(t) = -\lambda N(t)$$


\pause
  \item Si $N_0$ désigne le nombre de noyaux à l'instant initial, on a donc :
$$N(t) = N_0 e^{-\lambda t}$$
\end{itemize}
\vspace*{-2ex}\pause
  \myfigure{1}{
    \tikzinput{fig_equadiff06-pres}
  }
\end{frame}


% \begin{frame}
% $$N(t) = N_0 e^{-\lambda t}$$
% 
% \pause
% \begin{itemize}
%   \item Le \defi{temps caractéristique} est $\displaystyle \tau = \frac{1}{\lambda}$
%   \pause
%   \item  Le temps caractéristique $\tau$ est l'abscisse du point d'intersection de la 
%   tangente à l'origine $(T)$ avec l'axe du temps
%   
%   
%   \myfigure{1}{
%     \tikzinput{fig_equadiff06}
%   } 
%   
%   \pause
%   \item \'Equation de $(T)$ : $y = N'(0)t + N(0) = -\lambda N_0 t + N_0$,
%   
%   si $t = \tau$, on a bien $y = 0$
%    
% \end{itemize}
% 
% \end{frame}


\begin{frame}
$$N(t) = N_0 e^{-\lambda t}$$

\pause
\begin{itemize}
  \item La \defi{période de demi-vie} $\tau_{1/2}$ 
  est la période au bout de laquelle la moitié des noyaux se
  sont désintégrés
  
  \pause
  \item 
  \begin{itemize}
    \item on a $N(\tau_{1/2}) = \frac{N_0}{2}$
    \pause
    \item donc $N_0 e^{ -\lambda \tau_{1/2} } =   \frac{N_0}{2}$, d'où 
  $\lambda \tau_{1/2} = \ln 2$ 
    \pause
   \item ainsi \myboxinline{$\tau_{1/ 2} = \frac{\ln 2}{\lambda}$}% $\displaystyle = \tau \ln 2$
    
  \end{itemize}
  
  \pause
  \item $\displaystyle N(t) = N_0 e^{-\lambda t} %= N_0 e^{-\frac{t}{\tau}} 
  \pause =  N_0 e^{-\frac{t}{\tau_{1/2}} \ln 2} 
  \pause = N_0 2^{-\frac{t}{\tau_{1/2}}}$
 \pause
  \item $\tau_{1/ 2}$ est bien le
    temps nécessaire pour que la moitié des noyaux se
  soient désintégrés, quel que soit l'instant initial 
  
  \pause
  \vspace*{-4ex}
  $$N(t+\tau_{1/ 2}) 
  \pause = N_0 2^{-\frac{t+\tau_{1/2}}{\tau_{1/2}}}
  \pause = N_0 2^{-\frac{t}{\tau_{1/2}}-1} 
  \pause = \frac12 N_0 2^{-\frac{t}{\tau_{1/2}}}
  \pause = \frac{N(t)}{2}$$
  
\end{itemize}
 
          
      

\end{frame}


%%%%%%%%%%%%%%%%%%%%%%%%%%%%%%%%%%%%%%%%%%%%%%%%%%%%%%%%%%%%%%%%
\section*{Modèles d'évolution}

\begin{frame}

\evidence{Modèles d'évolutions}


\pause
\begin{itemize}
  \item Culture de bactéries en milieu clos
  \pause
  \item Soit $N_0$ le nombre de bactéries 
  introduites dans la culture à $t = 0$

\end{itemize}

\bigskip

\pause
\evidence{Loi de Malthus}

\pause
\begin{itemize}
  \item Hypothèse : la vitesse d'accroissement des bactéries 
est proportionnelle au nombre de bactéries présentes
  \pause
  \item Le nombre $N(t)$ de bactéries vérifie l'équation différentielle 
  
 \vspace*{-2ex} 
$$y' = ay$$
\vspace*{-2ex}

  \pause
  \item $a>0$ dépend des conditions expérimentales
  \pause
  \item Solution : 
  
  \vspace*{-3ex}
  $$N(t) = N_0 e^{at}$$
  \vspace*{-2ex}
  
  \pause
  \item Le milieu étant limité, ce modèle ne peut pas 
  s'appliquer sur une longue période
\end{itemize}

\end{frame}


\begin{frame}


\evidence{Modèle de Verhulst}

\pause
\begin{itemize}
  \item Hypothèse : le nombre $N(t)$ de bactéries vérifie 
  l'équation différentielle ($a>0$ et $M>0$)

   \vspace*{-3ex}
\begin{equation}
  y' = ay(M - y)
  \label{eq:eqdiffverhulst}
  \tag{$E$}
\end{equation}

  \pause
  \item On cherche solutions $y$ telles que
$y(t)>0$ pour $t\in I = [0,+\infty[$
 
 \pause
  \item \textbf{Changement de fonction}
  \begin{itemize}
  \pause
    \item on pose $z(x) = \frac{1}{y(x)}$
    
    \pause
    \item $\displaystyle z'=-\frac{y'}{y^2} = \frac{ay(y-M)}{y^2} = a - \frac{aM}{y} = a-aMz$

  \end{itemize}
  
  \pause
  \item \textbf{Solutions $z$}
  \begin{itemize}
  \pause
    \item $z' = a-aMz$
    \pause
    \item $z(x) = k e^{-aMx} + \frac1M$, $k \in \Rr$
    
  \end{itemize}

\pause
  \item \textbf{Solutions $y$}

  \pause
  \vspace*{-2ex}
  $$y(x) = \frac{1}{z(x)} = \frac{1}{k e^{-aMx} + \frac1M} = \frac{M}{kM e^{-aMx}+1}$$
  
  \pause
  \item La constante $k$ est déterminée par la condition initiale 
  $y(0) = \frac{M}{kM+1}=N_0$, ainsi $k = \frac{1}{N_0}-\frac{1}{M}$
\end{itemize}  
\end{frame}


\begin{frame} 

\vspace*{-2ex}

$$y' = y(1 - y)$$

\vspace*{-2ex}


\evidence{Exemple}
 
\pause 
\begin{itemize}
  \item $N_0 = 0,01$ (en million de bactéries) et $M=1$, $a=1$
  \pause
  \item Alors $k = \frac{1}{N_0}-\frac{1}{M} = 99$
  \pause
  \item $N(t) = \frac{1}{1 + 99 e^{-t}}$
  \pause
  \item $0<N(t)<1$, pour tout $t\ge0$ 
  \pause et  $N(t) \to 1$ lorsque $t\to+\infty$
  
  \pause
  \item Variations de la fonction $N$ 
  \begin{itemize}
  \pause
    \item $N$ solution de (\ref{eq:eqdiffverhulst}) :   $N'(t) = N(t)(1 - N(t))$
    \pause
    \item ainsi $N'(t)>0$, donc $N$ est croissante
  \end{itemize}
\end{itemize}
\pause

  \myfigure{1.3}{
    \tikzinput{fig_equadiff07}
  }


\end{frame}


%%%%%%%%%%%%%%%%%%%%%%%%%%%%%%%%%%%%%%%%%%%%%%%%%%%%%%%%%%%%%%%%
\section*{Masse attachée à un ressort}

\begin{frame}

\evidence{Masse attachée à un ressort}


\uncover<2->{Quelles sont les forces qui s'appliquent à cette masse ?}

\vspace*{-2ex}

\begin{minipage}{0.45\textwidth}
\begin{itemize}
  \uncover<3->{\item Un poids $\vec P$}
  \uncover<4->{\item une réaction $\vec R=-\vec P$ qui s'oppose au poids}
  \uncover<5->{\item une force de rappel $\vec T$}
  \uncover<6->{\item une force de frottement $\vec F$}
\end{itemize}  
\end{minipage}
\begin{minipage}{0.50\textwidth}
 \vspace*{3ex}
  \myfigure{0.6}{
    \tikzinput{fig_equadiff09}
  }  
\end{minipage}



  
\uncover<7->{\evidence{Principe fondamental de la mécanique}}
 
\uncover<8->{$$\vec P + \vec R + \vec T + \vec F = m\vec a$$}


\uncover<9->{Comme $\vec P+\vec R = \vec 0$, l'équation devient :
$$\vec T + \vec F = m\vec a$$}

\end{frame}


\begin{frame}

\vspace*{3ex}

\evidence{Force de rappel}

\vspace*{-3ex}

\hspace*{-1em}
\begin{minipage}{0.59\textwidth}
\begin{itemize}
  \uncover<2->{\item C'est une force horizontale, nulle 
à la position d'équilibre $x=0$}
  
  \uncover<3->{\item La force de rappel est un vecteur horizontal qui pointe 
vers la position d'équilibre}
  
  \uncover<4->{\item Modélisation $\vec T = -k x \vec i$}
  
  \uncover<5->{\item $x \in \Rr$ est la position de la masse}
  
  \uncover<6->{\item $k>0$ est une constante qui dépend du ressort}
\end{itemize}
\end{minipage}
\begin{minipage}{0.40\textwidth}
  \myfigure{0.45}{
    \tikzinput{fig_equadiff09a}\ \ 
    \tikzinput{fig_equadiff09b}\ \  
    \tikzinput{fig_equadiff09c}\ \      
  }  
\end{minipage}


  
 \end{frame}


\begin{frame} 


\evidence{Oscillations sans frottements}
\pause
\begin{itemize}[<+->]
  \item Pas de frottement : $\vec F = \vec 0$
  
  \item Principe fondamental de la mécanique $-kx(t) = m\frac{\dd^2 x(t)}{\dd t^2}$
  
  \item \'Equation différentielle 
  $$y'' + \frac{k}{m} y = 0$$
  
  \item \'Equation caractéristique est $r^2+\frac{k}{m} = 0$ \quad ($\Delta <0$)
  
  \item $r_1 = +\ii\sqrt{\frac{k}{m}}$ et $r_2 = -\ii\sqrt{\frac{k}{m}}$
  
  \item Solutions de l'équation différentielle
  $$y(x) = e^{\alpha x}\big(\lambda\cos (\beta x)+\mu\sin (\beta x)\big)$$

  \item Ici : 
  $$x(t) = \lambda\cos \left(\sqrt{\tfrac{k}{m}}t\right)
+\mu\sin \left(\sqrt{\tfrac{k}{m}}t\right)
\qquad \lambda, \mu \in \Rr$$
\end{itemize}

\end{frame}


\begin{frame}
\begin{exemple}
\centerline{$x(t) = \lambda\cos \left(\sqrt{\tfrac{k}{m}}t\right)
+\mu\sin \left(\sqrt{\tfrac{k}{m}}t\right)$}

\smallskip

\begin{itemize}
  \pause
  \item On lâche la masse au point d'abscisse $1$, sans vitesse initiale
  \pause
  \item Conditions initiales : $x(0)=1$ et $x'(0)=0$
  \pause
  \item Comme $x(0)=1$ alors $\lambda=1$
  \pause
  \item Comme $x'(0) = 0$ alors $\mu = 0$
  \pause
  \item On trouve une solution périodique :
  
\vspace*{-2ex}  
$$x(t) = \cos \left(\sqrt{\tfrac{k}{m}}t\right)$$
\end{itemize}

\vspace*{-6ex}

  \myfigure{1}{
    \tikzinput{fig_equadiff10}
  }
 
 \vspace*{-5ex}
 
\end{exemple}

\end{frame}



\begin{frame}

\evidence{Oscillations avec faibles frottements}

\pause

\begin{itemize}[<+->]
  \item Force de frottement est proportionnelle 
  à la vitesse et s'oppose au déplacement
  
  \item  $\vec F =  -fm \frac{\dd x(t)}{\dd t}$
  
  \item Principe fondamental de la mécanique $-kx(t)-fm \frac{\dd x(t)}{\dd t} = m\frac{\dd^2 x(t)}{\dd t^2}$
  
  \item \'Equation différentielle : 
   
  \vspace*{-3ex} 
  $$y'' + f y' + \frac{k}{m} y = 0$$

  \item \'Equation caractéristique $r^2+fr+\frac{k}{m}=0$
  
  \item $\Delta = f^2-4\frac{k}{m}$, frottement $f$ faible, $\Delta <0$
  
  \item $r_1 = \alpha + \ii \beta$ et $r_2 = \alpha - \ii \beta$
  
  \item $\alpha = -\frac{f}{2}$, $\beta = \frac{\delta}{2}$, 
  $\delta = \sqrt{|\Delta|}$
  
  \item Solutions de l'équation différentielle
$$y(x) = e^{\alpha x}\big(\lambda\cos (\beta x)+\mu\sin (\beta x)\big)$$

  \item Ici :
  
  \vspace*{-3ex}
$$x(t) = e^{-\frac{f}{2} t}\left(\lambda\cos \left(\tfrac{\delta}{2}t\right)
+\mu\sin \left(\tfrac{\delta}{2}t\right)\right)\qquad \lambda, \mu \in \Rr$$
\end{itemize}

\end{frame}


\begin{frame}

\evidence{Oscillations avec faibles frottements}


$$x(t) = e^{-\frac{f}{2} t}\left(\lambda\cos \left(\tfrac{\delta}{2}t\right)
+\mu\sin \left(\tfrac{\delta}{2}t\right)\right)$$

  
  \myfigure{1}{
    \tikzinput{fig_equadiff11}
  }
\end{frame}



%%%%%%%%%%%%%%%%%%%%%%%%%%%%%%%%%%%%%%%%%%%%%%%%%%%%%%%%%%%%%%%
 \section*{Mini-exercices}

\begin{frame}
\begin{miniexercice}
\begin{enumerate}
  \item Un circuit électrique constitué d'un condensateur 
  de capacité $C$ se décharge dans une résistance $R$.
  Calculer l'évolution de la charge électrique qui vérifie
  $q(t) = -RC \frac{\dd q(t)}{\dd t}$.
  
  \item Calculer et tracer les solutions du système masse-ressort
  pour différents niveaux de frottements.
  
  \item Un tasse de café de température $T_0= \SI{100}{\celsius}$
  est posée dans une pièce de température $T_\infty = \SI{20}{\celsius}$.
  La loi de Newton affirme que la vitesse de décroissance de la température
  $\frac{\dd T(t)}{\dd t}$ est proportionnelle à l'écart entre sa température
  $T(t)$ et la température ambiante $T_\infty$.
  Sachant qu'au bout de \SI{3}{\minute} la température du café 
  est passée à $\SI{80}{\celsius}$, combien de temps faudra-t-il 
  pour avoir un café à $\SI{65}{\celsius}$ ?
\end{enumerate}
\end{miniexercice}
\end{frame}




\end{document}
