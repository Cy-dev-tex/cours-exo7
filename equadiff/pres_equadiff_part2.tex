
%%%%%%%%%%%%%%%%%% PREAMBULE %%%%%%%%%%%%%%%%%%

\documentclass[aspectratio=169,utf8]{beamer}
%\documentclass[aspectratio=169,handout]{beamer}

\usetheme{Boadilla}
%\usecolortheme{seahorse}
\usecolortheme[RGB={245,66,24}]{structure}
\useoutertheme{infolines}

% packages
\usepackage{amsfonts,amsmath,amssymb,amsthm}
\usepackage[utf8]{inputenc}
\usepackage[T1]{fontenc}
\usepackage{lmodern}

\usepackage[francais]{babel}
\usepackage{fancybox}
\usepackage{graphicx}

\usepackage{float}
\usepackage{xfrac}

%\usepackage[usenames, x11names]{xcolor}
\usepackage{tikz}
\usepackage{pgfplots}
\usepackage{datetime}



%-----  Package unités -----
\usepackage{siunitx}
\sisetup{locale = FR,detect-all,per-mode = symbol}

%\usepackage{mathptmx}
%\usepackage{fouriernc}
%\usepackage{newcent}
%\usepackage[mathcal,mathbf]{euler}

%\usepackage{palatino}
%\usepackage{newcent}
% \usepackage[mathcal,mathbf]{euler}



% \usepackage{hyperref}
% \hypersetup{colorlinks=true, linkcolor=blue, urlcolor=blue,
% pdftitle={Exo7 - Exercices de mathématiques}, pdfauthor={Exo7}}


%section
% \usepackage{sectsty}
% \allsectionsfont{\bf}
%\sectionfont{\color{Tomato3}\upshape\selectfont}
%\subsectionfont{\color{Tomato4}\upshape\selectfont}

%----- Ensembles : entiers, reels, complexes -----
\newcommand{\Nn}{\mathbb{N}} \newcommand{\N}{\mathbb{N}}
\newcommand{\Zz}{\mathbb{Z}} \newcommand{\Z}{\mathbb{Z}}
\newcommand{\Qq}{\mathbb{Q}} \newcommand{\Q}{\mathbb{Q}}
\newcommand{\Rr}{\mathbb{R}} \newcommand{\R}{\mathbb{R}}
\newcommand{\Cc}{\mathbb{C}} 
\newcommand{\Kk}{\mathbb{K}} \newcommand{\K}{\mathbb{K}}

%----- Modifications de symboles -----
\renewcommand{\epsilon}{\varepsilon}
\renewcommand{\Re}{\mathop{\text{Re}}\nolimits}
\renewcommand{\Im}{\mathop{\text{Im}}\nolimits}
%\newcommand{\llbracket}{\left[\kern-0.15em\left[}
%\newcommand{\rrbracket}{\right]\kern-0.15em\right]}

\renewcommand{\ge}{\geqslant}
\renewcommand{\geq}{\geqslant}
\renewcommand{\le}{\leqslant}
\renewcommand{\leq}{\leqslant}
\renewcommand{\epsilon}{\varepsilon}

%----- Fonctions usuelles -----
\newcommand{\ch}{\mathop{\text{ch}}\nolimits}
\newcommand{\sh}{\mathop{\text{sh}}\nolimits}
\renewcommand{\tanh}{\mathop{\text{th}}\nolimits}
\newcommand{\cotan}{\mathop{\text{cotan}}\nolimits}
\newcommand{\Arcsin}{\mathop{\text{arcsin}}\nolimits}
\newcommand{\Arccos}{\mathop{\text{arccos}}\nolimits}
\newcommand{\Arctan}{\mathop{\text{arctan}}\nolimits}
\newcommand{\Argsh}{\mathop{\text{argsh}}\nolimits}
\newcommand{\Argch}{\mathop{\text{argch}}\nolimits}
\newcommand{\Argth}{\mathop{\text{argth}}\nolimits}
\newcommand{\pgcd}{\mathop{\text{pgcd}}\nolimits} 


%----- Commandes divers ------
\newcommand{\ii}{\mathrm{i}}
\newcommand{\dd}{\text{d}}
\newcommand{\id}{\mathop{\text{id}}\nolimits}
\newcommand{\Ker}{\mathop{\text{Ker}}\nolimits}
\newcommand{\Card}{\mathop{\text{Card}}\nolimits}
\newcommand{\Vect}{\mathop{\text{Vect}}\nolimits}
\newcommand{\Mat}{\mathop{\text{Mat}}\nolimits}
\newcommand{\rg}{\mathop{\text{rg}}\nolimits}
\newcommand{\tr}{\mathop{\text{tr}}\nolimits}


%----- Structure des exercices ------

\newtheoremstyle{styleexo}% name
{2ex}% Space above
{3ex}% Space below
{}% Body font
{}% Indent amount 1
{\bfseries} % Theorem head font
{}% Punctuation after theorem head
{\newline}% Space after theorem head 2
{}% Theorem head spec (can be left empty, meaning ‘normal’)

%\theoremstyle{styleexo}
\newtheorem{exo}{Exercice}
\newtheorem{ind}{Indications}
\newtheorem{cor}{Correction}


\newcommand{\exercice}[1]{} \newcommand{\finexercice}{}
%\newcommand{\exercice}[1]{{\tiny\texttt{#1}}\vspace{-2ex}} % pour afficher le numero absolu, l'auteur...
\newcommand{\enonce}{\begin{exo}} \newcommand{\finenonce}{\end{exo}}
\newcommand{\indication}{\begin{ind}} \newcommand{\finindication}{\end{ind}}
\newcommand{\correction}{\begin{cor}} \newcommand{\fincorrection}{\end{cor}}

\newcommand{\noindication}{\stepcounter{ind}}
\newcommand{\nocorrection}{\stepcounter{cor}}

\newcommand{\fiche}[1]{} \newcommand{\finfiche}{}
\newcommand{\titre}[1]{\centerline{\large \bf #1}}
\newcommand{\addcommand}[1]{}
\newcommand{\video}[1]{}

% Marge
\newcommand{\mymargin}[1]{\marginpar{{\small #1}}}

\def\noqed{\renewcommand{\qedsymbol}{}}


%----- Presentation ------
\setlength{\parindent}{0cm}

%\newcommand{\ExoSept}{\href{http://exo7.emath.fr}{\textbf{\textsf{Exo7}}}}

\definecolor{myred}{rgb}{0.93,0.26,0}
\definecolor{myorange}{rgb}{0.97,0.58,0}
\definecolor{myyellow}{rgb}{1,0.86,0}

\newcommand{\LogoExoSept}[1]{  % input : echelle
{\usefont{U}{cmss}{bx}{n}
\begin{tikzpicture}[scale=0.1*#1,transform shape]
  \fill[color=myorange] (0,0)--(4,0)--(4,-4)--(0,-4)--cycle;
  \fill[color=myred] (0,0)--(0,3)--(-3,3)--(-3,0)--cycle;
  \fill[color=myyellow] (4,0)--(7,4)--(3,7)--(0,3)--cycle;
  \node[scale=5] at (3.5,3.5) {Exo7};
\end{tikzpicture}}
}


\newcommand{\debutmontitre}{
  \author{} \date{} 
  \thispagestyle{empty}
  \hspace*{-10ex}
  \begin{minipage}{\textwidth}
    \titlepage  
  \vspace*{-2.5cm}
  \begin{center}
    \LogoExoSept{2.5}
  \end{center}
  \end{minipage}

  \vspace*{-0cm}
  
  % Astuce pour que le background ne soit pas discrétisé lors de la conversion pdf -> png
\begin{tikzpicture}
        \fill[opacity=0,green!60!black] (0,0)--++(0,0)--++(0,0)--++(0,0)--cycle; 
\end{tikzpicture}

% toc S'affiche trop tot :
% \tableofcontents[hideallsubsections, pausesections]
}

\newcommand{\finmontitre}{
  \end{frame}
  \setcounter{framenumber}{0}
} % ne marche pas pour une raison obscure

%----- Commandes supplementaires ------

% \usepackage[landscape]{geometry}
% \geometry{top=1cm, bottom=3cm, left=2cm, right=10cm, marginparsep=1cm
% }
% \usepackage[a4paper]{geometry}
% \geometry{top=2cm, bottom=2cm, left=2cm, right=2cm, marginparsep=1cm
% }

%\usepackage{standalone}


% New command Arnaud -- november 2011
\setbeamersize{text margin left=24ex}
% si vous modifier cette valeur il faut aussi
% modifier le decalage du titre pour compenser
% (ex : ici =+10ex, titre =-5ex

\theoremstyle{definition}
%\newtheorem{proposition}{Proposition}
%\newtheorem{exemple}{Exemple}
%\newtheorem{theoreme}{Théorème}
%\newtheorem{lemme}{Lemme}
%\newtheorem{corollaire}{Corollaire}
%\newtheorem*{remarque*}{Remarque}
%\newtheorem*{miniexercice}{Mini-exercices}
%\newtheorem{definition}{Définition}

% Commande tikz
\usetikzlibrary{calc}
\usetikzlibrary{patterns,arrows}
\usetikzlibrary{matrix}
\usetikzlibrary{fadings} 

%definition d'un terme
\newcommand{\defi}[1]{{\color{myorange}\textbf{\emph{#1}}}}
\newcommand{\evidence}[1]{{\color{blue}\textbf{\emph{#1}}}}
\newcommand{\assertion}[1]{\emph{\og#1\fg}}  % pour chapitre logique
%\renewcommand{\contentsname}{Sommaire}
\renewcommand{\contentsname}{}
\setcounter{tocdepth}{2}



%------ Figures ------

\def\myscale{1} % par défaut 
\newcommand{\myfigure}[2]{  % entrée : echelle, fichier figure
\def\myscale{#1}
\begin{center}
\footnotesize
{#2}
\end{center}}


%------ Encadrement ------

\usepackage{fancybox}


\newcommand{\mybox}[1]{
\setlength{\fboxsep}{7pt}
\begin{center}
\shadowbox{#1}
\end{center}}

\newcommand{\myboxinline}[1]{
\setlength{\fboxsep}{5pt}
\raisebox{-10pt}{
\shadowbox{#1}
}
}

%--------------- Commande beamer---------------
\newcommand{\beameronly}[1]{#1} % permet de mettre des pause dans beamer pas dans poly


\setbeamertemplate{navigation symbols}{}
\setbeamertemplate{footline}  % tiré du fichier beamerouterinfolines.sty
{
  \leavevmode%
  \hbox{%
  \begin{beamercolorbox}[wd=.333333\paperwidth,ht=2.25ex,dp=1ex,center]{author in head/foot}%
    % \usebeamerfont{author in head/foot}\insertshortauthor%~~(\insertshortinstitute)
    \usebeamerfont{section in head/foot}{\bf\insertshorttitle}
  \end{beamercolorbox}%
  \begin{beamercolorbox}[wd=.333333\paperwidth,ht=2.25ex,dp=1ex,center]{title in head/foot}%
    \usebeamerfont{section in head/foot}{\bf\insertsectionhead}
  \end{beamercolorbox}%
  \begin{beamercolorbox}[wd=.333333\paperwidth,ht=2.25ex,dp=1ex,right]{date in head/foot}%
    % \usebeamerfont{date in head/foot}\insertshortdate{}\hspace*{2em}
    \insertframenumber{} / \inserttotalframenumber\hspace*{2ex} 
  \end{beamercolorbox}}%
  \vskip0pt%
}


\definecolor{mygrey}{rgb}{0.5,0.5,0.5}
\setlength{\parindent}{0cm}
%\DeclareTextFontCommand{\helvetica}{\fontfamily{phv}\selectfont}

% background beamer
\definecolor{couleurhaut}{rgb}{0.85,0.9,1}  % creme
\definecolor{couleurmilieu}{rgb}{1,1,1}  % vert pale
\definecolor{couleurbas}{rgb}{0.85,0.9,1}  % blanc
\setbeamertemplate{background canvas}[vertical shading]%
[top=couleurhaut,middle=couleurmilieu,midpoint=0.4,bottom=couleurbas] 
%[top=fondtitre!05,bottom=fondtitre!60]



\makeatletter
\setbeamertemplate{theorem begin}
{%
  \begin{\inserttheoremblockenv}
  {%
    \inserttheoremheadfont
    \inserttheoremname
    \inserttheoremnumber
    \ifx\inserttheoremaddition\@empty\else\ (\inserttheoremaddition)\fi%
    \inserttheorempunctuation
  }%
}
\setbeamertemplate{theorem end}{\end{\inserttheoremblockenv}}

\newenvironment{theoreme}[1][]{%
   \setbeamercolor{block title}{fg=structure,bg=structure!40}
   \setbeamercolor{block body}{fg=black,bg=structure!10}
   \begin{block}{{\bf Th\'eor\`eme }#1}
}{%
   \end{block}%
}


\newenvironment{proposition}[1][]{%
   \setbeamercolor{block title}{fg=structure,bg=structure!40}
   \setbeamercolor{block body}{fg=black,bg=structure!10}
   \begin{block}{{\bf Proposition }#1}
}{%
   \end{block}%
}

\newenvironment{corollaire}[1][]{%
   \setbeamercolor{block title}{fg=structure,bg=structure!40}
   \setbeamercolor{block body}{fg=black,bg=structure!10}
   \begin{block}{{\bf Corollaire }#1}
}{%
   \end{block}%
}

\newenvironment{mydefinition}[1][]{%
   \setbeamercolor{block title}{fg=structure,bg=structure!40}
   \setbeamercolor{block body}{fg=black,bg=structure!10}
   \begin{block}{{\bf Définition} #1}
}{%
   \end{block}%
}

\newenvironment{lemme}[0]{%
   \setbeamercolor{block title}{fg=structure,bg=structure!40}
   \setbeamercolor{block body}{fg=black,bg=structure!10}
   \begin{block}{\bf Lemme}
}{%
   \end{block}%
}

\newenvironment{remarque}[1][]{%
   \setbeamercolor{block title}{fg=black,bg=structure!20}
   \setbeamercolor{block body}{fg=black,bg=structure!5}
   \begin{block}{Remarque #1}
}{%
   \end{block}%
}


\newenvironment{exemple}[1][]{%
   \setbeamercolor{block title}{fg=black,bg=structure!20}
   \setbeamercolor{block body}{fg=black,bg=structure!5}
   \begin{block}{{\bf Exemple }#1}
}{%
   \end{block}%
}


\newenvironment{miniexercice}[0]{%
   \setbeamercolor{block title}{fg=structure,bg=structure!20}
   \setbeamercolor{block body}{fg=black,bg=structure!5}
   \begin{block}{Mini-exercices}
}{%
   \end{block}%
}


\newenvironment{tp}[0]{%
   \setbeamercolor{block title}{fg=structure,bg=structure!40}
   \setbeamercolor{block body}{fg=black,bg=structure!10}
   \begin{block}{\bf Travaux pratiques}
}{%
   \end{block}%
}
\newenvironment{exercicecours}[1][]{%
   \setbeamercolor{block title}{fg=structure,bg=structure!40}
   \setbeamercolor{block body}{fg=black,bg=structure!10}
   \begin{block}{{\bf Exercice }#1}
}{%
   \end{block}%
}
\newenvironment{algo}[1][]{%
   \setbeamercolor{block title}{fg=structure,bg=structure!40}
   \setbeamercolor{block body}{fg=black,bg=structure!10}
   \begin{block}{{\bf Algorithme}\hfill{\color{gray}\texttt{#1}}}
}{%
   \end{block}%
}


\setbeamertemplate{proof begin}{
   \setbeamercolor{block title}{fg=black,bg=structure!20}
   \setbeamercolor{block body}{fg=black,bg=structure!5}
   \begin{block}{{\footnotesize Démonstration}}
   \footnotesize
   \smallskip}
\setbeamertemplate{proof end}{%
   \end{block}}
\setbeamertemplate{qed symbol}{\openbox}


\makeatother
\usecolortheme[RGB={51,102,51}]{structure}

%%%%%%%%%%%%%%%%%%%%%%%%%%%%%%%%%%%%%%%%%%%%%%%%%%%%%%%%%%%%%
%%%%%%%%%%%%%%%%%%%%%%%%%%%%%%%%%%%%%%%%%%%%%%%%%%%%%%%%%%%%%


\begin{document}


\title{{\bf \'Equations différentielles}}
\subtitle{\'Equation différentielle linéaire du premier ordre}

\begin{frame}
  
  \debutmontitre

  \pause

{\footnotesize
\hfill
\setbeamercovered{transparent=50}
\begin{minipage}{0.6\textwidth}
  \begin{itemize}
    \item<3-> $y'=ay$
    \item<4-> $y'=a(x)y$
    \item<5-> $y' = a(x)y+b(x)$
    \item<6-> Théorème de Cauchy-Lipschitz
    \item<7-> Courbes intégrales
    \item<8-> Exemples
  \end{itemize}
\end{minipage}
}

\end{frame}

\setcounter{framenumber}{0}

%%%%%%%%%%%%%%%%%%%%%%%%%%%%%%%%%%%%%%%%%%%%%%%%%%%%%%%%%%%%%%%%
\section*{Définition}

\begin{frame}
\begin{mydefinition}
Une équation différentielle \defi{linéaire du premier ordre} est une équation du type : 
\begin{equation}
  y'=a(x)y + b(x)   
  \label{eq:eqdifflinordre1}
  \tag{$E$}
\end{equation}
où $a$ et $b$ sont des fonctions définies sur un intervalle ouvert $I$ de $\Rr$
\end{mydefinition}

\pause

\begin{itemize}
  \item $a$ et $b$ sont des fonctions continues sur $I$
  \pause
  \item Variante $\alpha (x)y'+\beta (x)y=\gamma (x)$
  \pause
  \begin{itemize}
    \item $\alpha (x)\neq 0$ pour tout $x\in I$
    \pause
    \item division par $\alpha$ pour retrouver la forme (\ref{eq:eqdifflinordre1})
  \end{itemize}
  
\end{itemize}
\pause
\begin{enumerate}
  \item $a$ est une constante et $b=0$
  \pause
  \item $a$ est une fonction et $b=0$
  \pause
  \item $a$ et $b$ sont deux fonctions
\end{enumerate}

\end{frame}




%%%%%%%%%%%%%%%%%%%%%%%%%%%%%%%%%%%%%%%%%%%%%%%%%%%%%%%%%%%%%%%%
\section*{$y'=ay$}

\begin{frame}
\begin{theoreme}
\label{th:eqdifflinordre1cst}
Soit $a$ un réel. Soit l'équation différentielle :
\vspace*{-2ex}\begin{equation}
   y' = a y   
  \label{eq:eqdifflinordre1cst}
  \tag{$E$}
\end{equation}
\vspace*{-4ex}

Les solutions de (\ref{eq:eqdifflinordre1cst}), sur $\Rr$, sont les fonctions $y$ définies par :

\pause
\vspace*{-1ex}
\mybox{$y(x) = k e^{ax}$}
\vspace*{-2ex}

\pause
où $k\in \Rr$ est une constante quelconque
\end{theoreme}
\pause
Exemple : \\
\qquad $3y' - 5y = 0 \pause \iff y' = \frac53 y$. \pause Solutions : $y(x) = k e^{\frac53x}$, où $k \!\in\! \Rr$

\medskip

\pause
Preuve rapide  : 
\pause
\begin{itemize}[<+->]
  \item $\frac{y'}{y} =  a$
  \item On intègre : $\ln |y(x)| = ax+b$
  \item On compose par l'exponentielle $|y(x)| = e^{ax+b} = e^{ax}\cdot e^b$
  \item Avec $k = \pm e^b$, on obtient $y(x) = k e^{ax}$
\end{itemize}

\end{frame}


\begin{frame}

~

{
\begin{minipage}{0.73\textwidth}
\mybox{$y(x) = k e^{ax}$}
\end{minipage}
}

\vspace*{-3ex}

  \myfigure{0.5}{
    \hspace*{-3cm}
    \tikzinput{fig_equadiff03a}
    \quad
    \tikzinput{fig_equadiff03b}
  }
  \end{frame}








\begin{frame}
\begin{theoreme}
Les solutions de $y' = a y$ sont les \myboxinline{$y(x) = k e^{ax}$}
où $k\in \Rr$
\end{theoreme}
\begin{proof}
\pause
{\small\begin{enumerate}
  \item Les fonctions proposées sont bien solutions car $y'(x) = ake^{ax} = a y(x)$
 \pause 
  \item Montrons que ce sont les seules solutions
  \pause
  \begin{itemize}[<+->]
    \item Soit $y$ une solution quelconque de $y' = a y$
    
    \item Considérons la fonction $z$ définie par : $z(x) = y(x) e^{-ax}$
    
    \item Alors 
    $z'(x) = y'(x)e^{-ax} +  y(x)\big(-ae^{-ax}\big) =  e^{-ax}\big(y'(x)-ay(x)\big)$
    
    \item Comme $y'(x) - ay(x) = 0$ alors $z'(x) = 0$
    
    \item Il existe une constante $k$ telle que $z(x)=k$ pour tout $x\in \Rr$
 
    \item $z(x)= k \quad\text{donc}\quad  y(x) e^{-ax} = k \quad\text{donc}\quad  
  y(x) = ke^{ax}$
   \qedhere 
  \end{itemize}
 
\end{enumerate}
 }  
  
\end{proof}
\end{frame}



%%%%%%%%%%%%%%%%%%%%%%%%%%%%%%%%%%%%%%%%%%%%%%%%%%%%%%%%%%%%%%%%
\section*{$y'=a(x)y$}

\begin{frame}

\begin{theoreme}
\label{th:eqdifflinordre1}
Soit $a : I \to \Rr$ une fonction continue. Soit $A : I \to \Rr$ une primitive de $a$.
Soit l'équation différentielle :

\vspace*{-2ex}
\begin{equation}
   y' = a(x) y   
  \label{eq:eqdifflinordre1}
  \tag{$E$}
\end{equation}
\vspace*{-2ex}
\pause

Les solutions de (\ref{eq:eqdifflinordre1}) sur $I$ sont les fonctions $y$ définies par :
\mybox{$y(x) = k e^{A(x)}$}
\vspace*{-2ex}
où $k\in \Rr$ est une constante quelconque
\end{theoreme} 
\end{frame}


\begin{frame}
\begin{theoreme}
Les solutions de $y' = a(x) y$ sont les \myboxinline{$y(x) = k e^{A(x)}$}
où $k\in \Rr$ et $A$ est une primitive de $a$
\end{theoreme}
\pause
\vspace*{-2ex}
%Preuve rapide 
\begin{eqnarray*}
&\frac{y'}{y} =  a(x)
\pause
\iff \ln |y(x)| = A(x) + b
\pause
\iff |y(x)| = e^{A(x)+b} \\
\pause
\iff& y(x) = \pm e^b e^{A(x)}
\pause
\iff y(x) = k e^{A(x)} \quad \text{ avec } k = \pm e^b  
\end{eqnarray*}

\pause
\begin{exemple}
$$x^2y'=y$$
\pause
\vspace*{-4ex}
\begin{itemize}
\item Sur $I_+=\,]0,+\infty[$ ou $I_-=\,]-\infty,0[$
\pause
\item L'équation devient $y'= \frac{1}{x^2}y$ (ou encore $\frac{y'}{y}= \frac{1}{x^2}$)
\pause
\item Ici $a(x)=\frac{1}{x^2}$ dont une primitive est
$A(x)=-\frac1x$
\pause
\item Les solutions sont $y(x) = k e^{-\frac1x}$, où $k\in\Rr$
\end{itemize}
\end{exemple}

\end{frame}


%%%%%%%%%%%%%%%%%%%%%%%%%%%%%%%%%%%%%%%%%%%%%%%%%%%%%%%%%%%%%%%%
\section*{$y' = a(x)y+b(x)$}

\begin{frame}
\evidence{\'Equation différentielle linéaire d'ordre $1$ 
avec second membre}
\begin{equation}
   y' = a(x) y  + b(x)
  \label{eq:eqdifflinordre1scnd}
  \tag{$E$}
\end{equation}
où $a : I \to \Rr$ et $b: I \to \Rr$ sont des fonctions continues

\pause
L'équation homogène associée est :
\begin{equation}
   y' = a(x) y
  \label{eq:eqdifflinordre1scndhomo}
  \tag{$E_0$}
\end{equation}

\pause
\begin{proposition}
Si $y_0$ est une solution de (\ref{eq:eqdifflinordre1scnd}), 
alors les solutions de (\ref{eq:eqdifflinordre1scnd}) sont les
fonctions $y : I \to \Rr$ définies par : 

\vspace*{-2ex}
$$y(x)= y_0(x)+k e^{A(x)} \qquad \text{ avec }  k \in \Rr$$
\vspace*{-3ex}

où $x\mapsto A(x)$ est une primitive de $x \mapsto a(x)$
\end{proposition} 

\pause
\evidence{Solution évidente}
\pause
\begin{itemize}
  \item L'équation différentielle $y'=2xy+4x$
  \pause
  \item Solution particulière : $y_0(x)=-2$
  \pause
  \item Solutions :  $y(x) = -2 + ke^{x^2}$, où $k\in\Rr$
\end{itemize}

\end{frame}


\begin{frame}

\evidence{Variation de la constante}

 \pause
\begin{itemize}

  \item (\ref{eq:eqdifflinordre1scnd}) $y' = a(x) y  + b(x)$
   \pause
  \item La solution générale de (\ref{eq:eqdifflinordre1scndhomo}) 
  $y' = a(x) y$, est donnée par $y(x)=ke^{A(x)}$, avec $k\in \Rr$
   \pause
  \item Chercher une solution particulière sous la 
  forme $y_0(x)=k(x)e^{A(x)}$
   \pause
  \item Puisque $A'=a$, on a : 
  $y_0'(x)=a(x)k(x)e^{A(x)} + k'(x)e^{A(x)}= a(x)y_0(x) + k'(x)e^{A(x)}$
   \pause  
  \item $y_0'(x) - a(x)y_0(x) = k'(x)e^{A(x)}$ 
   \pause
  \item Donc $y_0$ est une solution de (\ref{eq:eqdifflinordre1scnd}) si
  et seulement si
 $\begin{array}{rcl}
 k'(x)e^{A(x)}=b(x) 
    \pause&\iff& k'(x)=b(x)e^{-A(x)} \\
    \pause&\iff& k(x)=\int b(x)e^{-A(x)}\dd x
 \end{array}$
  \pause
 \item Ce qui donne une solution particulière 
$y_0(x) = \left(\int b(x)e^{-A(x)}\dd x \right)e^{A(x)}$ de (\ref{eq:eqdifflinordre1scnd}) 
sur $I$
 \pause
  \item La solution générale de (\ref{eq:eqdifflinordre1scnd}) 
  est donnée par $y(x) = y_0(x) + ke^{A(x)}$, $k\in \Rr$
\end{itemize}

\end{frame}


\begin{frame}

\begin{exemple}
\label{ex:eqdifflinordre1scnd}
Résoudre $y'+y = e^x+1$ \quad (\ref{eq:eqdifflinordre1scnd})

\pause
\begin{itemize}
  \item $y'=-y$ (\ref{eq:eqdifflinordre1scndhomo}), solutions : $y(x) = k e^{-x}$, $k\in\Rr$
  \pause
  \item Recherche d'une solution particulière de (\ref{eq:eqdifflinordre1scnd}) sous la forme 
  $y_0(x) = k(x) e^{-x}$
  \pause
  \item On doit trouver $k(x)$ afin que $y_0$ vérifie $y'+y = e^x+1$
  \pause
  \item $\begin{array}{rcl}
        && y_0'+y_0 = e^x+1  \\
 \pause       
 & \iff & \left(k'(x) e^{-x} - k(x) e^{-x} \right) + k(x) e^{-x} = e^x+1 \\
 \pause  
 & \iff & k'(x)e^{-x} = e^x+1 \\
 \pause 
 & \iff & k'(x) = e^{2x}+e^x \\
 \pause 
 & \iff & k(x) = \frac12e^{2x}+e^x + c \\
\end{array}$
  \pause 
  \item $y_0(x) =  k(x) e^{-x} = \left(\frac12e^{2x}+e^x\right)e^{-x} = \frac12e^{x}+1$
  \pause 
  \item Solutions générales de (\ref{eq:eqdifflinordre1scnd}) : $y(x) = \frac12e^{x}+1 + k e^{-x}$, $k\in\Rr$
\end{itemize}

\end{exemple}
\end{frame}



%%%%%%%%%%%%%%%%%%%%%%%%%%%%%%%%%%%%%%%%%%%%%%%%%%%%%%%%%%%%%%%%
\section*{Théorème de Cauchy-Lipschitz}

\begin{frame}

\begin{theoreme}[de Cauchy-Lipschitz]
\label{th:cauchylipschitzord1}
Soit $y'=a(x)y + b(x)$ une équation différentielle linéaire du premier ordre,
où $a,b : I \to \Rr$ sont des fonctions continues sur un intervalle ouvert $I$


Alors, pour tout $x_0 \in I$ et pour tout $y_0 \in \Rr$, il existe une et une 
seule solution $y$ telle que $y(x_0)=y_0$
\end{theoreme}

\bigskip
\pause

Cette solution est :  
$$y(x) = \left(\int_{x_0}^x b(t)e^{-A(t)}\dd t \right)e^{A(x)} + y_0e^{A(x)}$$
où $A$ est la primitive de $a$ s'annulant en $x_0$, et cette solution vérifie bien $y(x_0)=y_0$
\end{frame}


\begin{frame}

\begin{exemple}
Trouver la solution de $y'+y = e^x+1$ vérifiant $y(1)=2$

\pause
\begin{itemize}
  \item Les solutions générales sont $y(x) = \frac12e^{x}+1 + k e^{-x}$
  \pause
  \item Déterminer $k$ afin que $y(1)=2$
  \pause
  \item 
  $\begin{array}{rcl}
   y(1)=2 
   \pause
     & \iff & \frac12e^{1}+1 + k e^{-1} = 2 \\
    \pause 
     & \iff & \frac{k}{e} = 1 - \frac{e}{2} \\
    \pause 
     & \iff & k = e - \frac{e^2}{2}  
   \end{array}
$
\pause
  \item La solution cherchée est $y(x) = \frac12e^{x}+1 + \left(e - \frac{e^2}{2}\right)e^{-x}$
\end{itemize}

\end{exemple}
\end{frame}




%%%%%%%%%%%%%%%%%%%%%%%%%%%%%%%%%%%%%%%%%%%%%%%%%%%%%%%%%%%%%%%%
\section*{Courbes intégrales}

\begin{frame}
\begin{itemize}
  \item Une \defi{courbe intégrale} d'une équation différentielle $(E)$
est le graphe d'une solution de $(E)$

\bigskip
\pause

  \item Théorème de Cauchy-Lipschitz pour les équations 
différentielle linéaire du premier ordre $y'=a(x)y + b(x)$ :
\mybox{
\begin{minipage}{0.7\textwidth}
\center
\og Par chaque point $(x_0,y_0) \in I \times \Rr$
passe une et une seule courbe intégrale \fg  
\end{minipage}
}
\end{itemize}
\end{frame}


\begin{frame}

\begin{itemize}
  \item Exemple : équation différentielle
$y'+y=x$
  \uncover<2->{\item Solutions $y(x) = x-1 + ke^{-x}, \quad k \in\Rr$}
  \uncover<3->{\item Pour chaque point $(x_0,y_0) \in \Rr^2$, il existe une unique solution
$y$, telle que $y(x_0)=y_0$}
\end{itemize}

\vspace*{-3ex}
  \myfigure{0.8}{
   \qquad\qquad\qquad\qquad\tikzinput{fig_equadiff04}
  }  

\end{frame}



%%%%%%%%%%%%%%%%%%%%%%%%%%%%%%%%%%%%%%%%%%%%%%%%%%%%%%%%%%%%%%%%
\section*{Exemples}

\begin{frame}
\begin{exemple}
Résoudre $(E) : \quad x^3y'+(2-3x^2)y=x^3$
\pause

\begin{enumerate}
  \item \evidence{Solution de l'équation homogène} $(E_0)$ : $x^3y'+(2-3x^2)y=0$
  \pause
    \begin{itemize}
      \item pour $x\neq 0$, on a $y'=-\frac{2-3x^2}{x^3}y$
     \pause 
      \item $y(x)=k e^{\int -\frac{2-3x^2}{x^3}\dd x}
=k e^{3\ln |x|}e^{1/x^2}=k|x|^3e^{1/x^2}$
\pause
      \item sur $]0,+\infty[$, $y(x)= k_1 x^3e^{1/x^2}$ \quad ($k_1 \in \Rr$)
     \pause 
      \item sur $]-\infty,0[$, $y(x)= k_2 x^3e^{1/x^2}$ \quad ($k_2 \in \Rr$)
    
    \end{itemize}
  
\pause
  \item \evidence{Solution particulière} 
  \pause
  \begin{itemize}
    \item sous la forme $y(x)=k(x)x^3e^{1/x^2}$
    \pause
    \item on obtient $k'(x) x^3e^{1/x^2} =1$
    \pause
    \item donc $k(x)=\int \frac{e^{-1/x^2}}{x^3}\dd x
    =\frac{1}{2}e^{-1/x^2}+c$
    \pause
    \item solution particulière $y_0(x)=k(x)x^3e^{1/x^2} =\frac{1}{2}x^3$
  \end{itemize}
    
    \pause
    \item \evidence{Solution générale}
    \pause
    \begin{itemize}
      \item sur $]0,+\infty[$, $y(x)=\frac{1}{2}x^3+k_1x^3e^{1/x^2}$ \quad ($k_1 \in \Rr$)
      \pause
      \item sur $]-\infty,0[$, $y(x)=\frac{1}{2}x^3+k_2x^3e^{1/x^2}$ \quad ($k_2 \in \Rr$)
    \end{itemize}

  \end{enumerate}
  
\end{exemple}
\end{frame}


%%%%%%%%%%%%%%%%%%%%%%%%%%%%%%%%%%%%%%%%%%%%%%%%%%%%%%%%%%%%%%%
\section*{Mini-exercices}

\begin{frame}
\begin{miniexercice}
\begin{enumerate} 
  \item Résoudre l'équation différentielle $y'+ y \ln 2 = 0$.
  Tracer les courbes intégrales. Trouver la solution vérifiant $y(1)=\frac12$.
  
  \item Résoudre l'équation différentielle $2y'+3y=5$.
  Trouver la solution vérifiant $y(0)=-\frac13$. Tracer la courbe intégrale.
  
  \item Trouver une solution évidente, puis résoudre l'équation 
  différentielle $2xy'+y=1$. Trouver la solution
  vérifiant $y(1)=2$. Tracer la courbe intégrale.
  Même travail avec l'équation $xy'-y=x^2$.
  
  \item Par la méthode de variation de la constante, trouver une solution
  particulière de l'équation différentielle $y'-2xy=3xe^{x^2}$.
  Même travail avec $y'+2y = \sin(3x)e^{-2x}$.
\end{enumerate}
\end{miniexercice}
\end{frame}




\end{document}
