
%%%%%%%%%%%%%%%%%% PREAMBULE %%%%%%%%%%%%%%%%%%


\documentclass[12pt]{article}

\usepackage{amsfonts,amsmath,amssymb,amsthm}
\usepackage[utf8]{inputenc}
\usepackage[T1]{fontenc}
\usepackage[francais]{babel}


% packages
\usepackage{amsfonts,amsmath,amssymb,amsthm}
\usepackage[utf8]{inputenc}
\usepackage[T1]{fontenc}
%\usepackage{lmodern}

\usepackage[francais]{babel}
\usepackage{fancybox}
\usepackage{graphicx}

\usepackage{float}

%\usepackage[usenames, x11names]{xcolor}
\usepackage{tikz}
\usepackage{datetime}

\usepackage{mathptmx}
%\usepackage{fouriernc}
%\usepackage{newcent}
\usepackage[mathcal,mathbf]{euler}

%\usepackage{palatino}
%\usepackage{newcent}


% Commande spéciale prompteur

%\usepackage{mathptmx}
%\usepackage[mathcal,mathbf]{euler}
%\usepackage{mathpple,multido}

\usepackage[a4paper]{geometry}
\geometry{top=2cm, bottom=2cm, left=1cm, right=1cm, marginparsep=1cm}

\newcommand{\change}{{\color{red}\rule{\textwidth}{1mm}\\}}

\newcounter{mydiapo}

\newcommand{\diapo}{\newpage
\hfill {\normalsize  Diapo \themydiapo \quad \texttt{[\jobname]}} \\
\stepcounter{mydiapo}}


%%%%%%% COULEURS %%%%%%%%%%

% Pour blanc sur noir :
%\pagecolor[rgb]{0.5,0.5,0.5}
% \pagecolor[rgb]{0,0,0}
% \color[rgb]{1,1,1}



%\DeclareFixedFont{\myfont}{U}{cmss}{bx}{n}{18pt}
\newcommand{\debuttexte}{
%%%%%%%%%%%%% FONTES %%%%%%%%%%%%%
\renewcommand{\baselinestretch}{1.5}
\usefont{U}{cmss}{bx}{n}
\bfseries

% Taille normale : commenter le reste !
%Taille Arnaud
%\fontsize{19}{19}\selectfont

% Taille Barbara
%\fontsize{21}{22}\selectfont

%Taille François
%\fontsize{25}{30}\selectfont

%Taille Pascal
%\fontsize{25}{30}\selectfont

%Taille Laura
%\fontsize{30}{35}\selectfont


%\myfont
%\usefont{U}{cmss}{bx}{n}

%\Huge
%\addtolength{\parskip}{\baselineskip}
}


% \usepackage{hyperref}
% \hypersetup{colorlinks=true, linkcolor=blue, urlcolor=blue,
% pdftitle={Exo7 - Exercices de mathématiques}, pdfauthor={Exo7}}


%section
% \usepackage{sectsty}
% \allsectionsfont{\bf}
%\sectionfont{\color{Tomato3}\upshape\selectfont}
%\subsectionfont{\color{Tomato4}\upshape\selectfont}

%----- Ensembles : entiers, reels, complexes -----
\newcommand{\Nn}{\mathbb{N}} \newcommand{\N}{\mathbb{N}}
\newcommand{\Zz}{\mathbb{Z}} \newcommand{\Z}{\mathbb{Z}}
\newcommand{\Qq}{\mathbb{Q}} \newcommand{\Q}{\mathbb{Q}}
\newcommand{\Rr}{\mathbb{R}} \newcommand{\R}{\mathbb{R}}
\newcommand{\Cc}{\mathbb{C}} 
\newcommand{\Kk}{\mathbb{K}} \newcommand{\K}{\mathbb{K}}

%----- Modifications de symboles -----
\renewcommand{\epsilon}{\varepsilon}
\renewcommand{\Re}{\mathop{\text{Re}}\nolimits}
\renewcommand{\Im}{\mathop{\text{Im}}\nolimits}
%\newcommand{\llbracket}{\left[\kern-0.15em\left[}
%\newcommand{\rrbracket}{\right]\kern-0.15em\right]}

\renewcommand{\ge}{\geqslant}
\renewcommand{\geq}{\geqslant}
\renewcommand{\le}{\leqslant}
\renewcommand{\leq}{\leqslant}

%----- Fonctions usuelles -----
\newcommand{\ch}{\mathop{\mathrm{ch}}\nolimits}
\newcommand{\sh}{\mathop{\mathrm{sh}}\nolimits}
\renewcommand{\tanh}{\mathop{\mathrm{th}}\nolimits}
\newcommand{\cotan}{\mathop{\mathrm{cotan}}\nolimits}
\newcommand{\Arcsin}{\mathop{\mathrm{Arcsin}}\nolimits}
\newcommand{\Arccos}{\mathop{\mathrm{Arccos}}\nolimits}
\newcommand{\Arctan}{\mathop{\mathrm{Arctan}}\nolimits}
\newcommand{\Argsh}{\mathop{\mathrm{Argsh}}\nolimits}
\newcommand{\Argch}{\mathop{\mathrm{Argch}}\nolimits}
\newcommand{\Argth}{\mathop{\mathrm{Argth}}\nolimits}
\newcommand{\pgcd}{\mathop{\mathrm{pgcd}}\nolimits} 

\newcommand{\Card}{\mathop{\text{Card}}\nolimits}
\newcommand{\Ker}{\mathop{\text{Ker}}\nolimits}
\newcommand{\id}{\mathop{\text{id}}\nolimits}
\newcommand{\ii}{\mathrm{i}}
\newcommand{\dd}{\mathrm{d}}
\newcommand{\Vect}{\mathop{\text{Vect}}\nolimits}
\newcommand{\Mat}{\mathop{\mathrm{Mat}}\nolimits}
\newcommand{\rg}{\mathop{\text{rg}}\nolimits}
\newcommand{\tr}{\mathop{\text{tr}}\nolimits}
\newcommand{\ppcm}{\mathop{\text{ppcm}}\nolimits}

%----- Structure des exercices ------

\newtheoremstyle{styleexo}% name
{2ex}% Space above
{3ex}% Space below
{}% Body font
{}% Indent amount 1
{\bfseries} % Theorem head font
{}% Punctuation after theorem head
{\newline}% Space after theorem head 2
{}% Theorem head spec (can be left empty, meaning ‘normal’)

%\theoremstyle{styleexo}
\newtheorem{exo}{Exercice}
\newtheorem{ind}{Indications}
\newtheorem{cor}{Correction}


\newcommand{\exercice}[1]{} \newcommand{\finexercice}{}
%\newcommand{\exercice}[1]{{\tiny\texttt{#1}}\vspace{-2ex}} % pour afficher le numero absolu, l'auteur...
\newcommand{\enonce}{\begin{exo}} \newcommand{\finenonce}{\end{exo}}
\newcommand{\indication}{\begin{ind}} \newcommand{\finindication}{\end{ind}}
\newcommand{\correction}{\begin{cor}} \newcommand{\fincorrection}{\end{cor}}

\newcommand{\noindication}{\stepcounter{ind}}
\newcommand{\nocorrection}{\stepcounter{cor}}

\newcommand{\fiche}[1]{} \newcommand{\finfiche}{}
\newcommand{\titre}[1]{\centerline{\large \bf #1}}
\newcommand{\addcommand}[1]{}
\newcommand{\video}[1]{}

% Marge
\newcommand{\mymargin}[1]{\marginpar{{\small #1}}}



%----- Presentation ------
\setlength{\parindent}{0cm}

%\newcommand{\ExoSept}{\href{http://exo7.emath.fr}{\textbf{\textsf{Exo7}}}}

\definecolor{myred}{rgb}{0.93,0.26,0}
\definecolor{myorange}{rgb}{0.97,0.58,0}
\definecolor{myyellow}{rgb}{1,0.86,0}

\newcommand{\LogoExoSept}[1]{  % input : echelle
{\usefont{U}{cmss}{bx}{n}
\begin{tikzpicture}[scale=0.1*#1,transform shape]
  \fill[color=myorange] (0,0)--(4,0)--(4,-4)--(0,-4)--cycle;
  \fill[color=myred] (0,0)--(0,3)--(-3,3)--(-3,0)--cycle;
  \fill[color=myyellow] (4,0)--(7,4)--(3,7)--(0,3)--cycle;
  \node[scale=5] at (3.5,3.5) {Exo7};
\end{tikzpicture}}
}



\theoremstyle{definition}
%\newtheorem{proposition}{Proposition}
%\newtheorem{exemple}{Exemple}
%\newtheorem{theoreme}{Théorème}
\newtheorem{lemme}{Lemme}
\newtheorem{corollaire}{Corollaire}
%\newtheorem*{remarque*}{Remarque}
%\newtheorem*{miniexercice}{Mini-exercices}
%\newtheorem{definition}{Définition}




%definition d'un terme
\newcommand{\defi}[1]{{\color{myorange}\textbf{\emph{#1}}}}
\newcommand{\evidence}[1]{{\color{blue}\textbf{\emph{#1}}}}



 %----- Commandes divers ------

\newcommand{\codeinline}[1]{\texttt{#1}}

%%%%%%%%%%%%%%%%%%%%%%%%%%%%%%%%%%%%%%%%%%%%%%%%%%%%%%%%%%%%%
%%%%%%%%%%%%%%%%%%%%%%%%%%%%%%%%%%%%%%%%%%%%%%%%%%%%%%%%%%%%%



\begin{document}

\debuttexte


%%%%%%%%%%%%%%%%%%%%%%%%%%%%%%%%%%%%%%%%%%%%%%%%%%%%%%%%%%%
\diapo

Les équations différentielles sont des objets parmi les plus importants en mathématiques.

\change
Nous commencerons ce chapitre par une motivation physique 

\change
et quelques exemples mathématiques

\change
avant de présenter la définition d'une équation différentielle,

\change
et de nous concentrer sur les équations différentielles linéaires.



%%%%%%%%%%%%%%%%%%%%%%%%%%%%%%%%%%%%%%%%%%%%%%%%%%%%%%%%%%%
\diapo

Lorsqu'un corps tombe en chute libre sans frottement 

\change
il n'est soumis qu'à son poids $\vec{P}$.

\change

Le cas d'un parachutiste est plus compliqué. 
Le modèle précédent n'est pas applicable

\change
Le parachute fait subir une force de frottement opposée à sa vitesse. 

\change
Le poids est une force verticale dirigée vers le bas, 
et $P = mg$ où $m$ est la masse et $g$ est la constante de gravitation.

\change
La force de frottement est aussi une force verticale mais dirigée vers le haut.
On suppose que le frottement est proportionnel
à la vitesse : $F = -fm v$,

$f$ est le coefficient de frottement.

\change
Ainsi le principe fondamental de la mécanique 
s'écrit $\vec{P}+\vec{F} = m\vec{a}$
où $\vec{a}$ est l'accélération.

\change
Cette équation sur l'axe vertical devient $mg - fmv = ma$, 

\change
L'accélération étant la dérivée de la vitesse par rapport au temps on obtient :

\begin{equation}
\frac{\dd v(t)}{\dd t} = g - kv(t)
\label{eq:para1}
\end{equation}

\change
C'est une relation entre la vitesse $v$ et sa dérivée : 

\change
il s'agit d'une \evidence{équation différentielle}.

\change
Il n'est pas évident de trouver quelle est la fonction $v$ qui convient.
Le but de chapitre est d'apprendre comment déterminer $v(t)$, 

\change
Une fois que l'on aura déterminer la vitesse on saura 
en déduire la position $x(t)$ à tout instant.


%%%%%%%%%%%%%%%%%%%%%%%%%%%%%%%%%%%%%%%%%%%%%%%%%%%%%%%%%%%
\diapo

Qu'est donc une équation différentielle ?

Une équation différentielle est une équation :

\change
dont l'inconnue est une fonction (généralement notée $y(x)$ ou simplement $y$) ;

\change

dans laquelle apparaissent certaines des dérivées de la
  fonction (dérivée première $y'$, ou dérivées d'ordre supérieur $y''$, $y^{(3)},...$).


%%%%%%%%%%%%%%%%%%%%%%%%%%%%%%%%%%%%%%%%%%%%%%%%%%%%%%%%%%%
\diapo

Voici des équations différentielles faciles à résoudre.


Vous devez trouver une fonction $y(x)$
qui vérifie chacune des équations.

Par exemple : quelle fonction $y(x)$
a pour dérivée $sin(x)$. Une solution est $y(x)=-\cos(x)$.

Je vous laisse arrêter la vidéo ici et résoudre les autres exemples.

\change
Vous pourrez ensuite vérifier vos calculs.




%%%%%%%%%%%%%%%%%%%%%%%%%%%%%%%%%%%%%%%%%%%%%%%%%%%%%%%%%%%
\diapo


Il est aussi facile de vérifier qu'une fonction donnée est bien solution d'une équation.

Soit l'équation différentielle $y' = 2xy+4x$.

\change
Je vous laisse vérifier que $y(x)=k\exp(x^2)-2$ est une solution sur $\Rr$, 
ceci quel que soit $k\in \Rr$.

Pour ce faire vous avez juste à dériver ce $y$
  
\change 

Même chose pour l'équation différentielle $x^2y''-2y+2x=0$.

\change
Vérifier que $y(x)=kx^2+x$ est une solution, pour tout $k \in \Rr$.


%%%%%%%%%%%%%%%%%%%%%%%%%%%%%%%%%%%%%%%%%%%%%%%%%%%%%%%%%%%
\diapo



Passons à la définition complète d'une équation différentielle
et surtout d'une solution d'une équation différentielle.


Une \defi{équation différentielle} d'ordre $n$ est une équation de la forme
  \begin{equation}
    F\left(x,y,y',\dots ,y^{(n)}\right)=0 
    \label{eq:eqdiff}
    \tag{$E$}
  \end{equation}
  où $F$ est une fonction de $(n+2)$ variables.
  
 \change 
  
Une \defi{solution} d'une telle équation sur un intervalle $I\subset \Rr$ 
  est une fonction $y :I \to \Rr$ qui est 
  
  (1) $n$ fois dérivable 
  
  et (2) qui vérifie l'équation (\ref{eq:eqdiff}).
  

%%%%%%%%%%%%%%%%%%%%%%%%%%%%%%%%%%%%%%%%%%%%%%%%%%%%%%%%%%%
\diapo

Voici quelques remarques !


C'est la coutume pour les équations différentielles de noter $y$ 
au lieu de $y(x)$, 
  $y'$ au lieu $y'(x)$,\ldots
  
  
\change
On note donc \og$y' = \sin x$\fg\ pour l'équation \og$y'(x) = \sin x$\fg.
  
\change
Il faut s'habituer au changement de nom pour les fonctions et les variables.

\change
  Par exemple $(x'')^3+t(x')^3+(\sin t) x^4=e^t$
  est une équation différentielle d'ordre $2$, 
  
  l'inconnue est une fonction $x$ qui dépend de la variable $t$.
  
  On cherche donc une fonction $x(t)$,
  deux fois dérivable, qui vérifie $(x''(t))^3+t(x'(t))^3+(\sin t) (x(t))^4=e^t$. 

\change

Rechercher une primitive c'est déjà résoudre l'équation différentielle
  $y'= f(x)$. C'est pourquoi on parle souvent \og d'intégrer l'équation différentielle \fg\ 
  pour signifier \og trouver les solutions de l'équation différentielle \fg.
  
 
\change
La notion d'intervalle dans la résolution d'une équation différentielle est fondamentale. 
  Si on change d'intervalle, on peut très bien obtenir d'autres solutions. 

  \change
  Par exemple, si on se place sur l'intervalle $I_1 = ]0, +\infty[$,
  l'équation différentielle $y' = 1/x$ a pour solutions les fonctions $y(x) = \ln(x) + k$.
  
  \change
Alors que sur l'intervalle $I_2 = ]-\infty, 0[$, les solutions sont les fonctions $y(x) = \ln(-x) + k$
($k$ est une constante). 

  \change Si aucune précision n'est donnée sur l'intervalle $I$, 
  on considérera qu'il s'agit de $I = \Rr$.
  

%%%%%%%%%%%%%%%%%%%%%%%%%%%%%%%%%%%%%%%%%%%%%%%%%%%%%%%%%%%
\diapo

[[grand F/petit f]]


Voici un type d'équation différentielle, dites à variables séparées,
qui se résout par intégration.

\change
Une équation différentielle \defi{à variables séparées} est une équation
du type : 
$$y'=g(x)/f(y)$$

\change
ou ce qui revient au même 
$$y'f(y)=g(x)$$

\change
Une telle équation se résout par calcul de primitives.
Si $G(x)$ est une primitive de $g(x)$ 

\change
alors par définition $G'(x)=g(x)$.

\change
Si $F(x)$ est une primitive de $f(x)$, 

alors bien sûr $F'(x)=f(x)$, 


\change
mais surtout par dérivation d'une composition
$\big(F(y(x))\big)' = y'(x) F'(y(x))$

\change
qui vaut donc $y' f(y)$.

\change
Ainsi l'équation différentielle $y'f(y)=g(x)$

\change
se réécrit $\big(F(y(x))\big)' = G'(x)$

\change
ce qui équivaut à une égalité de fonctions :
$F(y(x))=G(x)+c$.  

En effet les deux dérivées étant égales, les fonctions sont égales
à une constante près.

\change
Voici un exemple concret :
$$x^2y' = e^{-y}$$

\change
On commence par *séparer* les variables $x$ d'un côté et $y$ de l'autre :
$y'e^{y} = \frac{1}{x^2}$  (en supposant $x\neq 0$).

\change
On intègre des deux côtés :
$$e^{y} = -\frac{1}{x}+c \quad (c\in\Rr).$$

\change
Ce qui permet d'obtenir $y$ (en supposant $-\frac{1}{x}+c>0$) :
$$y(x) = \ln\left(-\frac{1}{x}+c\right)$$

On a trouvé les solutions.

\change
Il faut quand même terminer la résolution proprement 
en déterminant l'intervalle 
sur lequel cette solution est valide.

Il faut en effet que $x$ soit non nul, et que $-\frac{1}{x}+c>0$.

\change

Cet intervalle dépend de la constante $c$ :

on trouve que si $c<0$, $I=]\frac1c,0[$ ; 

si $c=0$, $I=]-\infty,0[$ ; 

enfin si $c>0$, $I=]\frac1c,+\infty[$.


%%%%%%%%%%%%%%%%%%%%%%%%%%%%%%%%%%%%%%%%%%%%%%%%%%%%%%%%%%%
\diapo


On ne sait pas résoudre toutes les équations différentielles.
On se concentre dans ce chapitre sur deux types d'équations : les équations différentielles
linéaires du premier ordre et celles du second ordre à coefficients constants.

Une équation différentielle d'ordre $n$ est \defi{linéaire} si elle est de la forme
$$a_0(x)y+a_1(x)y'+\dots +a_n(x)y^{(n)} = g(x)$$
où les $a_i$ et $g$ sont des fonctions réelles continues sur un intervalle $I\subset \Rr$.

Le terme linéaire signifie qu'il n'y a pas d'exposant pour les termes $y,y',y'',...$

\change  
Une équation différentielle linéaire est \defi{homogène}, ou \defi{sans second membre},
  si la fonction $g$ ci-dessus est la fonction nulle :
  $$a_0(x)y+a_1(x)y'+\dots +a_n(x)y^{(n)} = 0$$

\change   
Une équation différentielle linéaire est \defi{à coefficients constants} si
  $$a_0y+a_1y'+\dots +a_ny^{(n)} = g(x)$$
  où les $a_i$ sont des constantes réelles (et plus des fonctions) 
  et $g$ reste une fonction continue.
  
  

%%%%%%%%%%%%%%%%%%%%%%%%%%%%%%%%%%%%%%%%%%%%%%%%%%%%%%%%%%%
\diapo

Voici quelques exemples pour assimiler le vocabulaire.

$y' + 5xy = e^x$ est une équation différentielle linéaire du premier ordre avec second membre.

\change
$y' + 5xy = 0$ est l'équation différentielle homogène associée à la précédente.

\change
$2y'' - 3y' + 5y = 0$ est une équation différentielle linéaire du second ordre à coefficients constants,
  sans second membre.
  
\change
$y'^2 - y = x$ ou $y'' \cdot  y' - y = 0$ \emph{ne sont pas} des équations différentielles linéaires.  


%%%%%%%%%%%%%%%%%%%%%%%%%%%%%%%%%%%%%%%%%%%%%%%%%%%%%%%%%%%
\diapo


Pour les équations linéaires le principe de linéarité 
est très important et très utile.

Proposition :

Si $y_1$ et $y_2$ sont solutions de l'équation différentielle linéaire *homogène*
\begin{equation}
  a_0(x)y+a_1(x)y'+\dots +a_n(x)y^{(n)} = 0
  \label{eq:eqdifflin}
 \tag{$E_0$}
\end{equation}
alors, quels que soient $\lambda,\mu \in \Rr$, $\lambda y_1 + \mu y_2$ 
est aussi solution de cette équation.

C'est une simple vérification. 

On peut reformuler cette proposition 
en disant que l'ensemble des solutions
d'une équation différentielle linéaire sans second membre
forme un espace vectoriel.


%%%%%%%%%%%%%%%%%%%%%%%%%%%%%%%%%%%%%%%%%%%%%%%%%%%%%%%%%%%
\diapo

Nous allons appliquer immédiatement le principe de linéarité.

Pour résoudre une équation différentielle linéaire *avec* second membre  cette fois
\begin{equation}
  a_0(x)y+a_1(x)y'+\dots +a_n(x)y^{(n)} = g(x)
  \label{eq:eqdifflinscnd}
 \tag{$E$}
\end{equation}
\change
on décompose souvent la résolution en deux étapes :

(1) on cherche d'abord une solution particulière $y_0$ de cette équation,

\change
(2) ensuite on trouve l'ensemble de toutes les solutions 
$y$ de l'équation *homogène* associée, c-à-d la même équation mais 
avec un second membre nul [montrer].
On note $\mathcal{S}_h$ l'ensemble de ces solutions.

Ces deux étapes permettent de trouver toutes les solutions de (\ref{eq:eqdifflinscnd}) :

\change
En effet on a la principe de superposition :


L'ensemble des solutions $\mathcal{S}$ de (\ref{eq:eqdifflinscnd}) est formé 
des 
$$y_0 + y \quad \text{ avec } \quad  y \in \mathcal{S}_h$$


Autrement dit, on trouve toutes les solutions en ajoutant une solution particulière 
de $(E)$ aux solutions de l'équation homogène $(E_0)$.
C'est une conséquence immédiate du caractère linéaire des équations.

%%%%%%%%%%%%%%%%%%%%%%%%%%%%%%%%%%%%%%%%%%%%%%%%%%%%%%%%%%%
\diapo

Ne partez pas sans chercher ces exercices d’entraînements.


\end{document}
