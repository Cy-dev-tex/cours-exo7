
%%%%%%%%%%%%%%%%%% PREAMBULE %%%%%%%%%%%%%%%%%%


\documentclass[12pt]{article}

\usepackage{amsfonts,amsmath,amssymb,amsthm}
\usepackage[utf8]{inputenc}
\usepackage[T1]{fontenc}
\usepackage[francais]{babel}


% packages
\usepackage{amsfonts,amsmath,amssymb,amsthm}
\usepackage[utf8]{inputenc}
\usepackage[T1]{fontenc}
%\usepackage{lmodern}

\usepackage[francais]{babel}
\usepackage{fancybox}
\usepackage{graphicx}

\usepackage{float}

%\usepackage[usenames, x11names]{xcolor}
\usepackage{tikz}
\usepackage{datetime}

\usepackage{mathptmx}
%\usepackage{fouriernc}
%\usepackage{newcent}
\usepackage[mathcal,mathbf]{euler}

%\usepackage{palatino}
%\usepackage{newcent}


% Commande spéciale prompteur

%\usepackage{mathptmx}
%\usepackage[mathcal,mathbf]{euler}
%\usepackage{mathpple,multido}

\usepackage[a4paper]{geometry}
\geometry{top=2cm, bottom=2cm, left=1cm, right=1cm, marginparsep=1cm}

\newcommand{\change}{{\color{red}\rule{\textwidth}{1mm}\\}}

\newcounter{mydiapo}

\newcommand{\diapo}{\newpage
\hfill {\normalsize  Diapo \themydiapo \quad \texttt{[\jobname]}} \\
\stepcounter{mydiapo}}


%%%%%%% COULEURS %%%%%%%%%%

% Pour blanc sur noir :
%\pagecolor[rgb]{0.5,0.5,0.5}
% \pagecolor[rgb]{0,0,0}
% \color[rgb]{1,1,1}



%\DeclareFixedFont{\myfont}{U}{cmss}{bx}{n}{18pt}
\newcommand{\debuttexte}{
%%%%%%%%%%%%% FONTES %%%%%%%%%%%%%
\renewcommand{\baselinestretch}{1.5}
\usefont{U}{cmss}{bx}{n}
\bfseries

% Taille normale : commenter le reste !
%Taille Arnaud
%\fontsize{19}{19}\selectfont

% Taille Barbara
%\fontsize{21}{22}\selectfont

%Taille François
%\fontsize{25}{30}\selectfont

%Taille Pascal
%\fontsize{25}{30}\selectfont

%Taille Laura
%\fontsize{30}{35}\selectfont


%\myfont
%\usefont{U}{cmss}{bx}{n}

%\Huge
%\addtolength{\parskip}{\baselineskip}
}


% \usepackage{hyperref}
% \hypersetup{colorlinks=true, linkcolor=blue, urlcolor=blue,
% pdftitle={Exo7 - Exercices de mathématiques}, pdfauthor={Exo7}}


%section
% \usepackage{sectsty}
% \allsectionsfont{\bf}
%\sectionfont{\color{Tomato3}\upshape\selectfont}
%\subsectionfont{\color{Tomato4}\upshape\selectfont}

%----- Ensembles : entiers, reels, complexes -----
\newcommand{\Nn}{\mathbb{N}} \newcommand{\N}{\mathbb{N}}
\newcommand{\Zz}{\mathbb{Z}} \newcommand{\Z}{\mathbb{Z}}
\newcommand{\Qq}{\mathbb{Q}} \newcommand{\Q}{\mathbb{Q}}
\newcommand{\Rr}{\mathbb{R}} \newcommand{\R}{\mathbb{R}}
\newcommand{\Cc}{\mathbb{C}} 
\newcommand{\Kk}{\mathbb{K}} \newcommand{\K}{\mathbb{K}}

%----- Modifications de symboles -----
\renewcommand{\epsilon}{\varepsilon}
\renewcommand{\Re}{\mathop{\text{Re}}\nolimits}
\renewcommand{\Im}{\mathop{\text{Im}}\nolimits}
%\newcommand{\llbracket}{\left[\kern-0.15em\left[}
%\newcommand{\rrbracket}{\right]\kern-0.15em\right]}

\renewcommand{\ge}{\geqslant}
\renewcommand{\geq}{\geqslant}
\renewcommand{\le}{\leqslant}
\renewcommand{\leq}{\leqslant}

%----- Fonctions usuelles -----
\newcommand{\ch}{\mathop{\mathrm{ch}}\nolimits}
\newcommand{\sh}{\mathop{\mathrm{sh}}\nolimits}
\renewcommand{\tanh}{\mathop{\mathrm{th}}\nolimits}
\newcommand{\cotan}{\mathop{\mathrm{cotan}}\nolimits}
\newcommand{\Arcsin}{\mathop{\mathrm{Arcsin}}\nolimits}
\newcommand{\Arccos}{\mathop{\mathrm{Arccos}}\nolimits}
\newcommand{\Arctan}{\mathop{\mathrm{Arctan}}\nolimits}
\newcommand{\Argsh}{\mathop{\mathrm{Argsh}}\nolimits}
\newcommand{\Argch}{\mathop{\mathrm{Argch}}\nolimits}
\newcommand{\Argth}{\mathop{\mathrm{Argth}}\nolimits}
\newcommand{\pgcd}{\mathop{\mathrm{pgcd}}\nolimits} 

\newcommand{\Card}{\mathop{\text{Card}}\nolimits}
\newcommand{\Ker}{\mathop{\text{Ker}}\nolimits}
\newcommand{\id}{\mathop{\text{id}}\nolimits}
\newcommand{\ii}{\mathrm{i}}
\newcommand{\dd}{\mathrm{d}}
\newcommand{\Vect}{\mathop{\text{Vect}}\nolimits}
\newcommand{\Mat}{\mathop{\mathrm{Mat}}\nolimits}
\newcommand{\rg}{\mathop{\text{rg}}\nolimits}
\newcommand{\tr}{\mathop{\text{tr}}\nolimits}
\newcommand{\ppcm}{\mathop{\text{ppcm}}\nolimits}

%----- Structure des exercices ------

\newtheoremstyle{styleexo}% name
{2ex}% Space above
{3ex}% Space below
{}% Body font
{}% Indent amount 1
{\bfseries} % Theorem head font
{}% Punctuation after theorem head
{\newline}% Space after theorem head 2
{}% Theorem head spec (can be left empty, meaning ‘normal’)

%\theoremstyle{styleexo}
\newtheorem{exo}{Exercice}
\newtheorem{ind}{Indications}
\newtheorem{cor}{Correction}


\newcommand{\exercice}[1]{} \newcommand{\finexercice}{}
%\newcommand{\exercice}[1]{{\tiny\texttt{#1}}\vspace{-2ex}} % pour afficher le numero absolu, l'auteur...
\newcommand{\enonce}{\begin{exo}} \newcommand{\finenonce}{\end{exo}}
\newcommand{\indication}{\begin{ind}} \newcommand{\finindication}{\end{ind}}
\newcommand{\correction}{\begin{cor}} \newcommand{\fincorrection}{\end{cor}}

\newcommand{\noindication}{\stepcounter{ind}}
\newcommand{\nocorrection}{\stepcounter{cor}}

\newcommand{\fiche}[1]{} \newcommand{\finfiche}{}
\newcommand{\titre}[1]{\centerline{\large \bf #1}}
\newcommand{\addcommand}[1]{}
\newcommand{\video}[1]{}

% Marge
\newcommand{\mymargin}[1]{\marginpar{{\small #1}}}



%----- Presentation ------
\setlength{\parindent}{0cm}

%\newcommand{\ExoSept}{\href{http://exo7.emath.fr}{\textbf{\textsf{Exo7}}}}

\definecolor{myred}{rgb}{0.93,0.26,0}
\definecolor{myorange}{rgb}{0.97,0.58,0}
\definecolor{myyellow}{rgb}{1,0.86,0}

\newcommand{\LogoExoSept}[1]{  % input : echelle
{\usefont{U}{cmss}{bx}{n}
\begin{tikzpicture}[scale=0.1*#1,transform shape]
  \fill[color=myorange] (0,0)--(4,0)--(4,-4)--(0,-4)--cycle;
  \fill[color=myred] (0,0)--(0,3)--(-3,3)--(-3,0)--cycle;
  \fill[color=myyellow] (4,0)--(7,4)--(3,7)--(0,3)--cycle;
  \node[scale=5] at (3.5,3.5) {Exo7};
\end{tikzpicture}}
}



\theoremstyle{definition}
%\newtheorem{proposition}{Proposition}
%\newtheorem{exemple}{Exemple}
%\newtheorem{theoreme}{Théorème}
\newtheorem{lemme}{Lemme}
\newtheorem{corollaire}{Corollaire}
%\newtheorem*{remarque*}{Remarque}
%\newtheorem*{miniexercice}{Mini-exercices}
%\newtheorem{definition}{Définition}




%definition d'un terme
\newcommand{\defi}[1]{{\color{myorange}\textbf{\emph{#1}}}}
\newcommand{\evidence}[1]{{\color{blue}\textbf{\emph{#1}}}}



 %----- Commandes divers ------

\newcommand{\codeinline}[1]{\texttt{#1}}

%%%%%%%%%%%%%%%%%%%%%%%%%%%%%%%%%%%%%%%%%%%%%%%%%%%%%%%%%%%%%
%%%%%%%%%%%%%%%%%%%%%%%%%%%%%%%%%%%%%%%%%%%%%%%%%%%%%%%%%%%%%



\begin{document}

\debuttexte


%%%%%%%%%%%%%%%%%%%%%%%%%%%%%%%%%%%%%%%%%%%%%%%%%%%%%%%%%%%
\diapo

Nous étudions ici l'un des cas les plus simple mais aussi des plus important :
les équations différentielles linéaires du premier ordre.


\change

\change
On commence par les équations du type  $y'=ay$,

\change
puis $y'=a(x)y$

\change
et enfin $y' = a(x)y+b(x)$

\change
On termine par le théorème de Cauchy-Lipschitz,

\change
la notion de courbes intégrales,

\change
et des exemples !

%%%%%%%%%%%%%%%%%%%%%%%%%%%%%%%%%%%%%%%%%%%%%%%%%%%%%%%%%%%
\diapo


Une équation différentielle \defi{linéaire du premier ordre} est une équation du type: 
\begin{equation}
  y'=a(x)y + b(x)   
  \label{eq:eqdifflinordre1}
  \tag{$E$}
\end{equation}
où $a$ et $b$ sont des fonctions, et on se placera ici 
sur un intervalle ouvert $I$ de $\Rr$. 

\change
Dans la suite on supposera que $a$ et $b$ sont 
des fonctions continues sur $I$.

\change
On peut envisager la forme : 
$\alpha (x)y'+\beta (x)y=\gamma (x)$.

\change
On demandera alors que $\alpha (x)\neq 0$ pour tout $x\in I$. 

\change
La division par $\alpha $ permet de retrouver la forme précédente.
 

 \change
Voici le plan de cette vidéo :
On commence par résoudre le cas où $a$ est une constante et $b=0$.

\change
Puis $a$ sera une fonction et $b$ est encore nul.

\change
On termine par le cas général où $a$ et $b$ sont deux fonctions.

%%%%%%%%%%%%%%%%%%%%%%%%%%%%%%%%%%%%%%%%%%%%%%%%%%%%%%%%%%%
\diapo


Soit $a$ un réel quelconque. Considérons 
l'équation différentielle :
$y' = a y $ 

\change

Les solutions de cette équation sont les fonctions $y$ définies par :
$y(x) = k e^{ax}$

$a$ est la constante de l'équation différentielle,

\change
par contre $k\in \Rr$ est une constante quelconque.


\change
Voici un petit exemple :
on souhaite résoudre l'équation différentielle : 
$$3y' - 5y = 0$$

\change
On écrit cette équation sous la forme $y' = \frac53 y$

qui est bien de de cette forme là.

\change
Ses solutions sont les : $y(x) = k e^{\frac53x}$, ceci quelque soit
$k \in \Rr$.  

Cela peut aussi s'interpréter ainsi : 
si $y_0$ est une solution non identiquement nulle de
cette équation différentielle, 
alors toutes les autres solutions $y$ sont des multiples de $y_0$. 
En termes plus savants,
l'ensemble des solutions forme un espace vectoriel de dimension $1$ (une droite vectorielle).

\change

Ce résultat est fondamental. Il est tout aussi fondamental de 
comprendre d'où vient cette formule, via une preuve rapide 
(mais pas tout à fait rigoureuse). 

\change
On réécrit l'équation 
différentielle sous la forme
$$\frac{y'}{y} =  a$$

\change
que l'on intègre à gauche et à droite pour trouver :
$$\ln |y(x)| = ax+b$$

\change
On compose par l'exponentielle des deux côtés pour obtenir :
$$|y(x)| = e^{ax+b} = e^{ax}\times e^b$$

\change
Autrement dit  $y(x) = \pm e^b e^{ax}$.
On posant $k = \pm e^b$ on obtient les solutions 
(non nulles) cherchées.
Nous verrons une preuve plus rigoureuse juste après.



%%%%%%%%%%%%%%%%%%%%%%%%%%%%%%%%%%%%%%%%%%%%%%%%%%%%%%%%%%%
\diapo


L'équation différentielle $y'= a y$
admet donc une infinité de solutions.

Voici les graphes des solutions.

Par exemple ici à gauche $a$ est fixé et est positif.

Chaque valeur de $k$ fournie une solution.


A droite $a$ est fixé et est négatif, 
il y a aussi une infinité de solutions.


La constante $k$ peut être nulle. Dans ce cas, on obtient 
  la \og solution nulle \fg : $y = 0$ sur $\Rr$ qui est une solution
  évidente de l'équation différentielle.

%%%%%%%%%%%%%%%%%%%%%%%%%%%%%%%%%%%%%%%%%%%%%%%%%%%%%%%%%%%
\diapo


On passe à la preuve du théorème que l'on vient de voir :

Les solutions de $y' = a y$ sont les fonctions $y$ définies par :
$y(x) = k e^{ax}$

Voici la démonstration rigoureuse :
  
\change
Tout d'abord  il est facile de vérifier que les fonctions 
proposées sont bien solutions : 
en effet pour $y(x) = k e^{ax}$ alors 
  $y'(x) = ake^{ax} = a y(x).$
  
\change
Montrons qu'il n'y pas pas d'autres solutions. 

\change
On part de $y$ une solution quelconque de l'équation $y' = a y$.

\change
On introduit un nouvelle fonction $z$ définie par : 
$z(x) = y(x) e^{-ax}$ et on va montrer que $z$ est une fonction constante.

\change
Par la formule de dérivation d'un produit :
  $$z'(x) = y'(x)e^{-ax} +  y(x)\big(-ae^{-ax}\big) =  e^{-ax}\big(y'(x)-ay(x)\big)$$
  
\change
  Mais, par hypothèse, $y$ est une solution de l'équation différentielle
  donc $y'(x) - ay(x) = 0$. 
  
  Ce qui implique $z'(x) = 0$, pour tout $x$.
  
  \change
  Ainsi $z$ est une fonction constante sur $\Rr$.
  Autrement dit, il existe une constante $k$ telle que 
  $z(x)=k$ pour tout $x$.
  
  \change
  Maintenant que $z(x)= k$ alors $y(x) e^{-ax} = k$
  et ainsi $y(x) = ke^{ax}.$
  
  Une solution quelconque est bien de la forme annoncée
  et on a fini la preuve.

%%%%%%%%%%%%%%%%%%%%%%%%%%%%%%%%%%%%%%%%%%%%%%%%%%%%%%%%%%%
\diapo

[[petit $a$/grand $A$]]

Le théorème suivant affirme que lorsque $a$ est une fonction,
résoudre l'équation différentielle $y'=a(x)y$ 
revient à déterminer une primitive $A$ de $a$ 
(ce qui n'est pas toujours possible explicitement).


Voici l'énoncé :

Théorème :

Soit $a : I \to \Rr$ une fonction continue. Soit $A : I \to \Rr$ une primitive de $a$.
Soit l'équation différentielle :
$$
   y' = a(x) y   
$$

\change

Alors les solutions cette équation différentielle
sont les fonctions $y$ définies par :
$y(x) = k e^{*A*(x)}$
où $k\in \Rr$ est une constante quelconque.

Notez bien ici que c'est $*A*(x)$ dans l'exponentielle.

On retrouve en prime le théorème précédent : si $a(x)=a$ est 
une fonction constante, alors une primitive est $A(x)=ax$ 
et on obtient bien les solutions $y = k e^{ax}$ du théorème précédent.

%%%%%%%%%%%%%%%%%%%%%%%%%%%%%%%%%%%%%%%%%%%%%%%%%%%%%%%%%%%
\diapo

[[petit $a$/grand $A$]]

Reprenons le théorème : les solutions de $y' = a(x) y$ 
sont les $y(x) = k e^{A(x)}$.

\change
Voici une preuve rapide de ce théorème :

On écrit l'équation différentielle sous la forme 
$\frac{y'}{y} =  a(x) $

\change
on intègre des deux côtés :
$\ln |y(x)| = A(x) + b$


Je vous rappelle que $A(x)$ est une primitive de $a(x)$

\change
on prend l'exponentielle à gauche et à droite
$|y(x)| = e^{A(x)+b}$

\change
ce qui donne $y(x) = \pm e^b e^{A(x)}$

\change
c-à-d $y(x) = k e^{A(x)}$ où $k$ est une constante.

Il faut en plus ajouter la solution nulle pour obtenir le théorème.

\change

Voici un exemple : Comment résoudre 
l'équation différentielle $x^2y'=y$ ?

\change
On se place sur l'intervalle $I_+=]0,+\infty[$ 
ou $I_-=]-\infty,0[$.

\change
L'équation devient $y'= \frac{1}{x^2}y$. 

(ou encore $\frac{y'}{y}= \frac{1}{x^2}$ pour se ramener à cette forme là)

\change
Donc ici $a(x)=\frac{1}{x^2}$, dont on connaît une primitive :
$A(x)=-\frac1x$. 

\change
Ainsi, par le théorème, les solutions cherchées sont 
$y(x) = k e^{-\frac1x}$, où $k\in\Rr$.



%%%%%%%%%%%%%%%%%%%%%%%%%%%%%%%%%%%%%%%%%%%%%%%%%%%%%%%%%%%
\diapo


Il nous reste le cas général de l'équation différentielle linéaire d'ordre $1$ avec second membre :
\begin{equation}
   y' = a(x) y  + b(x)
  \label{eq:eqdifflinordre1scnd}
  \tag{$E$}
\end{equation}
où $a : I \to \Rr$ et $b: I \to \Rr$ sont des fonctions continues.

\change
On aura besoin de l'équation homogène associée :
\begin{equation}
   y' = a(x) y
  \label{eq:eqdifflinordre1scndhomo}
  \tag{$E_0$}
\end{equation}

\change
Il n'y a pas de nouvelle formule à apprendre pour ce cas.
Il suffit d'appliquer le principe de superposition : 
les solutions de (\ref{eq:eqdifflinordre1scnd}) s'obtiennent en
ajoutant à une solution particulière de (\ref{eq:eqdifflinordre1scnd}) les solutions de
(\ref{eq:eqdifflinordre1scndhomo}).
Ce qui donne :


[[petit $a$/grand $A$]]

Proposition : 
Si $y_0$ est une solution de (\ref{eq:eqdifflinordre1scnd}), 
alors les solutions de (\ref{eq:eqdifflinordre1scnd}) sont les
fonctions $y : I \to \Rr$, définie par: 
$$y(x)= y_0(x)+k e^{A(x)} \qquad k \in \Rr$$
où $x\mapsto A(x)$ est une primitive de $x \mapsto a(x)$.

\change

La recherche de la solution générale 
de (\ref{eq:eqdifflinordre1scnd}) se réduit 
donc à la recherche d'une solution particulière. 

Parfois ceci 
se fait en remarquant une solution évidente. 

\change
Par exemple, l'équation différentielle $y'=2xy+4x$

\change
a pour solution particulière la fonction constante $y_0(x)=-2$ ;

\change
donc l'ensemble des solutions de cette équation sont les 
$y(x) = -2 + ke^{x^2}$, où $k\in\Rr$.

%%%%%%%%%%%%%%%%%%%%%%%%%%%%%%%%%%%%%%%%%%%%%%%%%%%%%%%%%%%
\diapo

Voici la méthode de variation de la constante.

Le nom de cette méthode est paradoxal mais justifié !
C'est une méthode générale pour trouver une solution 
particulière en se ramenant à un calcul de primitive.

\change
On souhaite trouver une solution de cette équation $y' = a(x) y  + b(x)$

\change
La solution générale de l'équation sans second membre
$y' = a(x) y$, est donnée 
par $y(x)=ke^{A(x)}$, avec $k\in \Rr$ une constante.

\change
La méthode de la variation de la constante consiste à chercher 
une solution particulière sous la forme $y_0(x)=k(x)e^{A(x)}$, 
où $k$ est maintenant une fonction à déterminer

\change 
On commence par calculer la dérivée de $y_0$.
Puisque $A'=a$, on a :
$$y_0'(x)=a(x)k(x)e^{A(x)} + k'(x)e^{A(x)}= a(x)y_0(x) + k'(x)e^{A(x)}.$$

\change
Que l'on réécrit : $y_0'(x) - a(x)y_0(x) = k'(x)e^{A(x)}$

\change
On veut que $y_0$ soit une solution de l'équation 
(\ref{eq:eqdifflinordre1scnd}) $y' = a(x) y  + b(x)$. 

Donc $y_0$ est une solution de (\ref{eq:eqdifflinordre1scnd}) 
si et seulement si
le terme de gauche est $b(x)$,
c-à-d $k'(x)e^{A(x)}=b(x)$

\change
Autrement dit $k'(x)=b(x)e^{-A(x)}$

\change
ce qui équivaut à $k(x)=\int b(x)e^{-A(x)}\dd x.$

\change
Ce qui donne une solution particulière  de (\ref{eq:eqdifflinordre1scnd}) : 

$y_0(x) = \left(\int b(x)e^{-A(x)}\dd x \right)e^{A(x)}$

\change
Maintenant 
La solution générale de  (\ref{eq:eqdifflinordre1scnd}) est donnée par
la somme de notre solution particulière $y_0(x)$
et des solutions de l'équation sans second membre.


%%%%%%%%%%%%%%%%%%%%%%%%%%%%%%%%%%%%%%%%%%%%%%%%%%%%%%%%%%%
\diapo

Voyons comment fonctionne la méthode de variation de la constante
sur un exemple.

On souhaite résoudre l'équation différentielle $y'+y = e^x+1$.

\change
On commence par résoudre l'équation homogène : $y'=-y$,

les solutions sont les $y(x) = k e^{-x}$, $k\in\Rr$. 

\change
Pour trouver une solution particulière de l'équation avec second membre,

on considère $y_0(x) = k(x) e^{-x}$ où $k$ est maintenant une fonction. 

\change
On doit trouver $k(x)$ afin que $y_0$ vérifie l'équation 
différentielle $y'+y = e^x+1$.

\change
Que $y_0$ soit solution de l'équation différentielle signifie 

$y_0'+y_0 = e^x+1$

\change
$\iff  \left(k'(x) e^{-x} - k(x) e^{-x} \right) + k(x) e^{-x} = e^x+1$

\change

Cela se simplifie --en fait cela doit toujours se simplifier--
$\iff  k'(x)e^{-x} = e^x+1$

\change
$\iff  k'(x) = e^{2x}+e^x$

\change
$\iff  k(x) = \frac12e^{2x}+e^x + c$

\change
On choisit par exemple $c=0$ (n'importe quelle valeur convient)

Nous tenons notre solution particulière !
$$y_0(x) =  k(x) e^{-x} 
= \left(\frac12e^{2x}+e^x\right)e^{-x} 
= \frac12e^{x}+1$$

\change
Les solutions générales de l'équation
$y'+y = e^x+1$ s'obtiennent en additionnant cette solution 
particulière aux solutions de l'équation homogène :
$$y(x) = \frac12e^{x}+1 + k e^{-x}, \qquad k\in\Rr.$$

%%%%%%%%%%%%%%%%%%%%%%%%%%%%%%%%%%%%%%%%%%%%%%%%%%%%%%%%%%%
\diapo


Voici l'énoncé du théorème de Cauchy-Lipschitz dans le cas des 
équations différentielles linéaires du premier ordre.

Théorème : 
Soit $y'=a(x)y + b(x)$ une équation différentielle linéaire du premier ordre,
où $a,b : I \to \Rr$ sont des fonctions continues sur un intervalle ouvert $I$.
Alors pour tout $x_0 \in I$ et pour tout $y_0 \in \Rr$, il existe une et une 
seule solution $y$ telle que $y(x_0)=y_0$.


Il faut comprendre ce théorème ainsi, cette équation différentielle admet une infinité de solution,
mais en fixant une condition initiale, alors il existe une unique solution. On a existence
et unicité.



\change
Nos calculs précédents nous permettent d'écrire explicitement cette solution, même si ce n'est pas très utile :

[[grand $A$/petit $a$]]

$$y(x) = \left(\int_{x_0}^x b(t)e^{-A(t)}\dd t \right)e^{A(x)} + y_0e^{A(x)}$$
où $A$ est la primitive de $a$ s'annulant en $x_0$, 

et cette solution vérifie bien $y(x_0)=y_0$.


%%%%%%%%%%%%%%%%%%%%%%%%%%%%%%%%%%%%%%%%%%%%%%%%%%%%%%%%%%%
\diapo

Illustrons le théorème de Cauchy-Lipschitz par un exemple.


Quelle est la fonction $y$ qui vérifie l'équation différentielle
$y'+y = e^x+1$, et en plus la condition  $y(1)=2$.

\change
Nous avons déjà trouvé toutes les solutions de cette 
équation dans l'exemple précédent.
Ce sont les  $y(x) = \frac12e^{x}+1 + k e^{-x}$
où $k\in\Rr$. 

\change
Nous allons déterminer la constante $k$ afin que la condition initiale
$y(1)=2$ soit vérifiée :

\change \change
$y(1)=2 \iff \frac12e^{1}+1 + k e^{-1} = 2$

\change
$\iff \frac{k}{e} = 1 - \frac{e}{2}$

\change
et donc $k = e - \frac{e^2}{2}$

\change
Ainsi la solution cherchée est 
$y(x) = \frac12e^{x}+1 + \left(e - \frac{e^2}{2}\right)e^{-x}$,
et c'est la seule solution.


%%%%%%%%%%%%%%%%%%%%%%%%%%%%%%%%%%%%%%%%%%%%%%%%%%%%%%%%%%%
\diapo

Il est primordial d'avoir aussi une vision géométrique
des solutions.

Une \defi{courbe intégrale} d'une équation différentielle $(E)$
est le graphe d'une solution de $(E)$.

\change
Le théorème de Cauchy-Lipschitz pour les équations 
différentielle linéaire du premier ordre $y'=a(x)y + b(x)$
se reformule ainsi :

\og Par chaque point $(x_0,y_0) \in I \times \Rr$
passe une et une seule courbe intégrale. \fg


%%%%%%%%%%%%%%%%%%%%%%%%%%%%%%%%%%%%%%%%%%%%%%%%%%%%%%%%%%%
\diapo

Illustrons cette notion de courbe intégrale.

avec l'équation différentielle $y'+y=x$.


\change
On sait résoudre cette équations et on trouve que les solutions
sont les $y(x) = x-1 + ke^{-x}$, avec une solutions pour chaque 
$k \in\Rr$.

Pour chaque $k$, on trace le graphe de la fonction, chaque courbe est une courbe intégrale.

\change

Le théorème de Cauchy-Lipschitz affirme que pour chaque couple 
$(x_0,y_0)$, il existe une unique solution
$y$, telle que $y(x_0)=y_0$. 

\change
Géométriquement cela signifie que pour chaque point 
$(x_0,y_0)$ 

\change
il existe une unique 
courbe intégrale passant par $(x_0,y_0)$.


%%%%%%%%%%%%%%%%%%%%%%%%%%%%%%%%%%%%%%%%%%%%%%%%%%%%%%%%%%%
\diapo

On termine par un exemple qui va nous permettre de récapituler plusieurs
point de cette section.

On va résoudre l'équation différentielle 
$(E) : \quad x^3y'+(2-3x^2)y=x^3$. 


\change
On commence avec l'équation homogène $(E_0)$ :
$x^3y'+(2-3x^2)y=0$.  

\change
Pour $x\neq 0$, on a $y'=-\frac{2-3x^2}{x^3}y$. 

\change
Donc la solution générale s'obtient en trouvant une primitive
de $-\frac{2-3x^2}{x^3}$,

\change
c'est une fraction qui se décompose $1/x^2$ et en $1/x$ et 
et donc après intégration on obtient 
$y(x) =k e^{3\ln |x|}e^{1/x^2}=k|x|^3e^{1/x^2}$. 

\change
Ainsi la solution générale de l'équation homogène 
est  : $y(x)= k_1 x^3e^{1/x^2}$ ; sur l'intervalle $]0,+\infty[$

\change 
et $y(x)= k_2 x^3e^{1/x^2}$ sur $]-\infty,0[$.

Attention ! à part la solution nulle, 
les solutions ne sont pas définies sur $\Rr$.


\change
On revient à l'équation avec second membre.

Il nous suffit de trouver cherche une seule solution de cette équation,

nous allons-la trouver ici par la méthode de variation de la constante.  

\change
On cherche une solution sous la forme $y(x)=k(x)x^3e^{1/x^2}$, 

c-à-d que $k$ qui était une constante dans la solution de l'équation homogène
devient ici une fonction à déterminer.

\change
en dérivant,
en remplaçant dans l'équation différentielle
et en simplifiant, on obtient 
$k'(x) x^3e^{1/x^2} =1. $ 

\change
On intègre pour avoir $k(x)=\frac{1}{2}e^{-1/x^2}+c$. 

\change
Et donc on obtient une solution 
particulière avec par exemple $c=0$
$y_0(x)=k(x)x^3e^{1/x^2}$ ce qui se simplifie en $\frac{1}{2}x^3$. 

On n'oublie pas se rassurer en vérifiant que ceci est bien une solution particulière.

\change
Il ne reste plus qu'à appliquer le principe de superposition 

\change

\change
    
sur $]0,+\infty[$  les solutions sont les : $y(x)=\frac{1}{2}x^3+k_1x^3e^{1/x^2}$.   

sur $]-\infty,0[$ : les solutions sont $y(x)=\frac{1}{2}x^3+k_2x^3e^{1/x^2}$.  

%%%%%%%%%%%%%%%%%%%%%%%%%%%%%%%%%%%%%%%%%%%%%%%%%%%%%%%%%%%
\diapo

Au trvail pour résoudre ces exercices !



\end{document}
