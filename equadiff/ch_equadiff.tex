\documentclass[class=report,crop=false]{standalone}
\usepackage[screen]{../exo7book}

\begin{document}

% Commande ponctuelle
\newcommand{\alenvers}[1]{\rotatebox[origin=c]{180}{#1}}

%====================================================================
\chapitre{Équations différentielles}
%====================================================================

\insertvideo{dkjXofPNMDo}{partie 1. Définition}

\insertvideo{6SfAvnaFhFM}{partie 2. Équation différentielle linéaire du premier ordre}

\insertvideo{vUon9Q7-SZA}{partie 3. Équation différentielle linéaire du second ordre à coefficients constants}

\insertvideo{hcZPh8NbTmw}{partie 4. Problèmes conduisant à des équations différentielles}

\insertfiche{fic00165.pdf}{Équations différentielles}

\bigskip

Lorsqu'un corps tombe en chute libre sans frottement, il n'est soumis qu'à son poids $\vec{P}$.
Par le principe fondamental de la mécanique : $\vec{P} = m\vec{a}$.
Tous les vecteurs sont verticaux donc $mg = ma$, où
$g$ est la constante de gravitation, $a$ l'accélération verticale et $m$ la masse.
On obtient $a=g$. L'accélération étant la dérivée de la vitesse par rapport au temps, on obtient :
\begin{equation}
\frac{\dd v(t)}{\dd t} = g
\label{eq:para1}
\end{equation}


Il est facile d'en déduire la vitesse par intégration :
$v(t) = gt$ (en supposant que la vitesse initiale est nulle), c'est-à-dire
que la vitesse augmente de façon linéaire au cours du temps. Puisque la vitesse est la dérivée de
la position, on a $v(t) = \frac{\dd x(t)}{\dd t}$,
donc par une nouvelle intégration on obtient $x(t) = \frac12 g t^2$
(en supposant que la position initiale est nulle).


  \myfigure{1}{
    \tikzinput{fig_equadiff01}
    \qquad\qquad\qquad\qquad\qquad\qquad
    \tikzinput{fig_equadiff02}
  }


Le cas d'un parachutiste est plus compliqué. Le modèle précédent n'est pas applicable
car il ne tient pas compte des frottements. Le parachute fait
subir une force de frottement opposée à sa vitesse. On suppose que le frottement est proportionnel
à la vitesse : $F = -fm v$ ($f$ est le coefficient de frottement).
Ainsi le principe fondamental de la mécanique devient $mg - fmv = ma$, ce qui conduit à
la relation :
\begin{equation}
\frac{\dd v(t)}{\dd t} = g - fv(t)
\label{eq:para2}
\end{equation}
C'est une relation entre la vitesse $v$ et sa dérivée : il s'agit d'une \evidence{équation différentielle}.
Il n'est pas évident de trouver quelle est la fonction $v$ qui convient.
Le but de ce chapitre est d'apprendre comment déterminer $v(t)$, ce qui nous permettra
d'en déduire la position $x(t)$ à tout instant.

%%%%%%%%%%%%%%%%%%%%%%%%%%%%%%%%%%%%%%%%%%%%%%%%%%%%%%%%%%%%%%%%
\section{Définition}


%---------------------------------------------------------------
\subsection{Introduction}

Une équation différentielle est une équation :
\begin{itemize}
\item dont l'inconnue est une fonction (généralement notée $y(x)$ ou simplement $y$) ;

\item dans laquelle apparaissent certaines des dérivées de la
  fonction (dérivée première $y'$, ou dérivées d'ordres supérieurs $y''$, $y^{(3)},\ldots$).
\end{itemize}

Voici des équations différentielles faciles à résoudre.
\begin{exemple}
De tête, trouver au moins une fonction, solution des
équations différentielles suivantes :
$$\begin{array}{lr}
y' = \sin x & \qquad \qquad \alenvers{$y(x) = -\cos x + k \quad\text{ où } k \in \Rr$} \\
y' = 1 + e^x & \qquad \qquad \alenvers{$y(x) = x + e^x + k \quad\text{ où } k \in \Rr$} \\
y' = y & \qquad \qquad \alenvers{$y(x) = k e^x \quad\text{ où } k \in \Rr$} \\
y' = 3y & \qquad \qquad \alenvers{$y(x) = k e^{3x} \quad\text{ où } k \in \Rr$} \\
y'' = \cos x & \qquad \qquad \alenvers{$y(x) = -\cos x + ax + b \quad\text{ où } a, b \in \Rr$} \\
y'' = y & \qquad \qquad \alenvers{$y(x) = a e^x + b e^{-x} \quad\text{ où }  a, b \in \Rr$}  \\
\end{array}$$
\end{exemple}


Il est aussi facile de vérifier qu'une fonction donnée est bien solution d'une équation.
\begin{exemple}
\sauteligne
\begin{enumerate}
  \item Soit l'équation différentielle $y' = 2xy+4x$.
  Vérifier que $y(x)=k\exp(x^2)-2$ est une solution sur $\Rr$, ceci quel que soit $k\in \Rr$.

  \item Soit l'équation différentielle $x^2y''-2y+2x=0$.
  Vérifier que $y(x)=kx^2+x$ est une solution sur $\Rr$, pour tout $k \in \Rr$.
\end{enumerate}
\end{exemple}


%---------------------------------------------------------------
\subsection{Définition}

Passons à la définition complète d'une équation différentielle
et surtout d'une solution d'une équation différentielle.

\begin{definition}
\sauteligne
\begin{itemize}
  \item Une \defi{équation différentielle}\index{equation differentielle@équation différentielle} d'ordre $n$ est une équation de la forme
  \begin{equation}
    F\left(x,y,y',\dots ,y^{(n)}\right)=0
    \label{eq:eqdiff}
    \tag{$E$}
  \end{equation}
  où $F$ est une fonction de $(n+2)$ variables.

  \item Une \defi{solution} d'une telle équation sur un intervalle $I\subset \Rr$
  est une fonction $y :I \to \Rr$ qui est $n$ fois dérivable
  et qui vérifie l'équation (\ref{eq:eqdiff}).
\end{itemize}
\end{definition}

\begin{remarque*}
\sauteligne
\begin{itemize}
  \item C'est la coutume pour les équations différentielles de noter $y$ au lieu de $y(x)$,
  $y'$ au lieu $y'(x)$,\ldots
  On note donc \og $y' = \sin x$ \fg{} ce qui signifie \og $y'(x) = \sin x$ \fg.

  \item Il faut s'habituer au changement de nom pour les fonctions et les variables.
  Par exemple $(x'')^3+t(x')^3+(\sin t) x^4=e^t$
  est une équation différentielle d'ordre $2$, dont l'inconnue est
  une fonction $x$ qui dépend de la variable $t$. On cherche donc une fonction $x(t)$,
  deux fois dérivable, qui vérifie $(x''(t))^3+t(x'(t))^3+(\sin t) (x(t))^4=e^t$.

  \item Rechercher une primitive, c'est déjà résoudre l'équation différentielle
  $y'= f(x)$. C'est pourquoi on trouve souvent \og intégrer l'équation différentielle \fg{} 
  pour \og trouver les solutions de l'équation différentielle \fg.


  \item La notion d'intervalle dans la résolution d'une équation différentielle est fondamentale.
  Si on change d'intervalle, on peut très bien obtenir d'autres solutions.
  Par exemple, si on se place sur l'intervalle $I_1 =\,]0, +\infty[$,
  l'équation différentielle $y' = 1/x$ a pour solutions les fonctions $y(x) = \ln(x) + k$.
Alors que sur l'intervalle $I_2 =\,]-\infty, 0[$, les solutions sont les fonctions $y(x) = \ln(-x) + k$
($k$ est une constante).

  \item Si aucune précision n'est donnée sur l'intervalle $I$,
  on considérera qu'il s'agit de $I = \Rr$.
\end{itemize}
\end{remarque*}

\begin{exemple}[Équation à variables séparées]
Une équation différentielle 
\defi{à variables séparées}\index{equation differentielle@équation différentielle!a variables separees@à variables séparées}
est une équation
du type :
$$y'=g(x)/f(y) \qquad \text{ ou } \qquad y'f(y)=g(x)$$
Une telle équation se résout par calcul de primitives.
Si $G(x)$ est une primitive de $g(x)$ alors $G'(x)=g(x)$.
Si $F(x)$ est une primitive de $f(x)$ alors
$F'(x)=f(x)$, mais surtout, par dérivation d'une composition,
$\big(F(y(x))\big)' = y'(x) F'(y(x)) = y' f(y)$.
Ainsi l'équation différentielle $y'f(y)=g(x)$
se réécrit $\big(F(y(x))\big)' = G'(x)$
ce qui équivaut à une égalité de fonctions :
$F(y(x))=G(x)+c$.

\bigskip

Voici un exemple concret :
$$x^2y' = e^{-y}$$
On commence par séparer les variables $x$ d'un côté et $y$ de l'autre :
$y'e^{y} = \frac{1}{x^2}$  (en supposant $x\neq 0$).
On intègre des deux côtés :
$$e^{y} = -\frac{1}{x}+c \quad (c\in\Rr)$$
Ce qui permet d'obtenir $y$ (en supposant $-\frac{1}{x}+c>0$) :
$$y(x) = \ln\left(-\frac{1}{x}+c\right)$$
qui est une solution sur chaque intervalle $I$ où elle est définie et dérivable.
Cet intervalle dépend de la constante $c$ :
si $c<0$, $I=\,]\frac1c,0[$ ;
si $c=0$, $I=\,]-\infty,0[$ ; si $c>0$, $I=\,]\frac1c,+\infty[$.

\end{exemple}




%---------------------------------------------------------------
\subsection{Équation différentielle linéaire}

On ne sait pas résoudre toutes les équations différentielles.
On se concentre dans ce chapitre sur deux types d'équations : les équations différentielles
linéaires du premier ordre et celles du second ordre à coefficients constants.

\begin{itemize}
  \item Une équation différentielle d'ordre $n$ est \defi{linéaire} si elle est de la forme
$$a_0(x)y+a_1(x)y'+\dots +a_n(x)y^{(n)} = g(x)$$
où les $a_i$ et $g$ sont des fonctions réelles continues sur un intervalle $I\subset \Rr$.

Le terme linéaire signifie grosso modo qu'il n'y a pas d'exposant pour les termes $y,y',y'',\ldots$

  \item Une équation différentielle linéaire est 
  \defi{homogène}\index{equation differentielle@équation différentielle!homogene@homogène}, ou 
  \defi{sans second membre}\index{equation differentielle@équation différentielle!sans second membre},
  si la fonction $g$ ci-dessus est la fonction nulle :
  $$a_0(x)y+a_1(x)y'+\dots +a_n(x)y^{(n)} = 0$$

  \item Une équation différentielle linéaire est \defi{à coefficients constants} si
  les fonctions $a_i$ ci-dessus sont constantes :
  $$a_0y+a_1y'+\dots +a_ny^{(n)} = g(x)$$
  où les $a_i$ sont des constantes réelles et $g$ une fonction continue.
\end{itemize}

\begin{exemple}
\sauteligne
\begin{enumerate}
  \item $y' + 5xy = e^x$ est une équation différentielle linéaire du premier ordre avec second membre.

  \item $y' + 5xy = 0$ est l'équation différentielle homogène associée à la précédente.

  \item $2y'' - 3y' + 5y = 0$ est une équation différentielle linéaire du second ordre à coefficients constants,
  sans second membre.


  \item $y'^2 - y = x$ ou $y'' \cdot  y' - y = 0$ \emph{ne sont pas} des équations différentielles linéaires.
\end{enumerate}
\end{exemple}



\begin{proposition}[Principe de linéarité]
Si $y_1$ et $y_2$ sont solutions de l'équation différentielle linéaire homogène
\begin{equation}
  a_0(x)y+a_1(x)y'+\dots +a_n(x)y^{(n)} = 0
  \label{eq:eqdifflin}
 \tag{$E_0$}
\end{equation}
alors, quels que soient $\lambda,\mu \in \Rr$, $\lambda y_1 + \mu y_2$ est aussi solution de cette équation.
\end{proposition}

C'est une simple vérification. On peut reformuler la proposition en disant que l'ensemble des solutions
forme un espace vectoriel.


\bigskip

Pour résoudre une équation différentielle linéaire avec second membre
\begin{equation}
  a_0(x)y+a_1(x)y'+\dots +a_n(x)y^{(n)} = g(x),
  \label{eq:eqdifflinscnd}
 \tag{$E$}
\end{equation}
on décompose souvent la résolution en deux étapes :
\begin{itemize}
  \item trouver une solution particulière $y_0$ de l'équation (\ref{eq:eqdifflinscnd}),
  \item trouver l'ensemble $\mathcal{S}_h$ des solutions $y$ de l'équation homogène associée
\begin{equation}
  a_0(x)y+a_1(x)y'+\dots +a_n(x)y^{(n)} = 0
  \label{eq:eqdifflinbis}
 \tag{$E_0$}
\end{equation}
\end{itemize}
ce qui permet de trouver toutes les solutions de (\ref{eq:eqdifflinscnd}) :
\begin{proposition}[Principe de superposition]
L'ensemble des solutions $\mathcal{S}$ de (\ref{eq:eqdifflinscnd}) est formé
des
$$y_0 + y \quad \text{ avec } \quad  y \in \mathcal{S}_h.$$
\end{proposition}
Autrement dit, on trouve toutes les solutions en ajoutant une solution particulière
aux solutions de l'équation homogène.
C'est une conséquence immédiate du caractère linéaire des équations.

\begin{miniexercices}
\sauteligne
\begin{enumerate}
  \item Chercher une solution \og simple \fg{} de l'équation différentielle
  $y'=2y$. Même question avec $y''=-y$ ; $y''+\cos(2x)=0$ ; $xy''=y'$.

  \item Résoudre l'équation différentielle à variables séparées $y'y^2=x$.
  Même question avec $y'= y \ln x$ ; $y' = \frac{1}{y^n}$ ($n\ge1$).

  \item Soit l'équation $y' = y(1 - y)$. Montrer que si
  $y$ est une solution non nulle de cette équation, alors $z = 2y$
  n'est pas solution. Que peut-on en conclure ?

\end{enumerate}
\end{miniexercices}


%%%%%%%%%%%%%%%%%%%%%%%%%%%%%%%%%%%%%%%%%%%%%%%%%%%%%%%%%%%%%%%%
\section{Équation différentielle linéaire du premier ordre}

\begin{definition}
Une équation différentielle \defi{linéaire du premier ordre} est une équation du type:
\begin{equation}
  y'=a(x)y + b(x)
  \label{eq:eqdifflinordre1}
  \tag{$E$}
\end{equation}
où $a$ et $b$ sont des fonctions définies sur un intervalle ouvert $I$ de $\Rr$.
\end{definition}


Dans la suite on supposera que $a$ et $b$ sont des fonctions continues sur $I$.
On peut envisager la forme :
$\alpha (x)y'+\beta (x)y=\gamma (x)$.
On demandera alors que $\alpha (x)\neq 0$ pour tout $x\in I$.
La division par $\alpha $ permet de retrouver la forme (\ref{eq:eqdifflinordre1}).

\bigskip

On va commencer par résoudre le cas où $a$ est une constante et $b=0$.
Puis $a$ sera une fonction (et toujours $b=0$).
On terminera par le cas général où $a$ et $b$ sont deux fonctions.

%---------------------------------------------------------------
\subsection{$y'=ay$}

\begin{theoreme}
\label{th:eqdifflinordre1cst}
Soit $a$ un réel. Soit l'équation différentielle :
\begin{equation}
   y' = a y
  \label{eq:eqdifflinordre1cst}
  \tag{$E$}
\end{equation}
Les solutions de (\ref{eq:eqdifflinordre1cst}), sur $\Rr$, sont les fonctions $y$ définies par :
\mybox{$y(x) = k e^{ax}$}
où $k\in \Rr$ est une constante quelconque.
\end{theoreme}


Ce résultat est fondamental. Il est tout aussi fondamental de
comprendre d'où vient cette formule, via une preuve rapide
(mais pas tout à fait rigoureuse). On réécrit l'équation
différentielle sous la forme
$$\frac{y'}{y} =  a$$
que l'on intègre à gauche et à droite pour trouver :
$$\ln |y(x)| = ax+b$$
On compose par l'exponentielle des deux côtés pour obtenir :
$$|y(x)| = e^{ax+b}$$
Autrement dit  $y(x) = \pm e^b e^{ax}$.
En posant $k = \pm e^b$ on obtient les solutions (non nulles) cherchées.
Nous verrons une preuve rigoureuse juste après.

  \myfigure{0.47}{
    \tikzinput{fig_equadiff03a}
    \quad
    \tikzinput{fig_equadiff03b}
  }
\begin{exemple}
Résoudre l'équation différentielle :
$$3y' - 5y = 0$$
On écrit cette équation sous la forme $y' = \frac53 y$.
Ses solutions, sur $\Rr$, sont donc de la forme :
$y(x) = k e^{\frac53x}$, où $k \in \Rr$.
\end{exemple}

\begin{remarque*}
\sauteligne
\begin{itemize}
  \item L'équation différentielle (\ref{eq:eqdifflinordre1cst})
admet donc une infinité de solutions (puisque l'on a une
infinité de choix de la constante $k$).

  \item La constante $k$ peut être nulle. Dans ce cas, on obtient
  la \og solution nulle \fg{} : $y = 0$ sur $\Rr$, qui est une solution
  évidente de l'équation différentielle.

  \item Le théorème \ref{th:eqdifflinordre1cst} peut aussi s'interpréter ainsi :
  si $y_0$ est une solution non identiquement nulle de
l'équation différentielle (\ref{eq:eqdifflinordre1cst}),
alors toutes les autres solutions $y$ sont des multiples de $y_0$. En termes plus savants,
l'ensemble des solutions forme un espace vectoriel de dimension $1$ (une droite vectorielle).
\end{itemize}
\end{remarque*}


\begin{proof}[Preuve du théorème \ref{th:eqdifflinordre1cst}]
~
\begin{enumerate}
  \item On vérifie que les fonctions proposées sont bien solutions de (\ref{eq:eqdifflinordre1cst}).
  En effet, pour $y(x) = k e^{ax}$, on a
  $$y'(x) = ake^{ax} = a y(x).$$

  \item Montrons que les fonctions proposées sont les seules solutions.
  (C'est-à-dire qu'il n'y en a pas d'un autre type que $y(x)= k e^{ax}$.)
  Soit $y$ une solution quelconque de (\ref{eq:eqdifflinordre1cst}) sur $\Rr$.
  Considérons la fonction $z$ définie par : $z(x) = y(x) e^{-ax}$.
  Alors, par la formule de dérivation d'un produit :
  $$z'(x) = y'(x)e^{-ax} +  y(x)\big(-ae^{-ax}\big) =  e^{-ax}\big(y'(x)-ay(x)\big)$$
  Mais, par hypothèse, $y$ est une solution de (\ref{eq:eqdifflinordre1cst}),
  donc $y'(x) - ay(x) = 0$. On en déduit que $z'(x) = 0$, pour tout réel $x$.
  Ainsi $z$ est une fonction constante sur $\Rr$.
  Autrement dit, il existe une constante $k$ telle que $z(x)=k$ pour tout $x\in \Rr$.
  D'où :
  $$z(x)= k \quad\text{donc}\quad  y(x) e^{-ax} = k \quad\text{donc}\quad
  y(x) = ke^{ax}.$$
  Ce qui termine la preuve du théorème.
\end{enumerate}


\end{proof}

%---------------------------------------------------------------
\subsection{$y'=a(x)y$}

Le théorème suivant affirme que, lorsque $a$ est une fonction,
résoudre l'équation différentielle $y'=a(x)y$
revient à déterminer une primitive $A$ de $a$ (ce qui n'est pas toujours possible explicitement).



\begin{theoreme}
\label{th:eqdifflinordre1}
Soit $a : I \to \Rr$ une fonction continue. Soit $A : I \to \Rr$ une primitive de $a$.
Soit l'équation différentielle :
\begin{equation}
   y' = a(x) y
  \label{eq:eqdifflinordre1bis}
  \tag{$E$}
\end{equation}
Les solutions sur $I$ de (\ref{eq:eqdifflinordre1bis}) sont les fonctions $y$ définies par :
\mybox{$y(x) = k e^{A(x)}$}
où $k\in \Rr$ est une constante quelconque.
\end{theoreme}

Si $a(x)=a$ est une fonction constante, alors une primitive est par exemple $A(x)=ax$ et on retrouve
les solutions du théorème \ref{th:eqdifflinordre1cst}.


Une preuve rapide du théorème \ref{th:eqdifflinordre1} est la suivante :
\begin{eqnarray*}
&\frac{y'}{y} =  a(x)
\iff \ln |y(x)| = A(x) + b
\iff |y(x)| = e^{A(x)+b} \\
\iff& y(x) = \pm e^b e^{A(x)}
\iff y(x) = k e^{A(x)} \quad \text{ avec } k = \pm e^b
\end{eqnarray*}


Une preuve rigoureuse (puisque l'on évite de diviser
par quelque chose qui pourrait être nul) :
\begin{proof}
$$\begin{array}{rl}
     & y(x) \text{ solution de (\ref{eq:eqdifflinordre1bis}) } \\
\iff & y'(x) - a(x)y(x) = 0 \\
\iff & e^{-A(x)}\big(y'(x)-ay(x)\big) = 0 \\
\iff & \big( y(x) e^{-A(x)}\big)' = 0 \\
\iff & \exists k \in \Rr \quad y(x) e^{-A(x)} = k \\
\iff & \exists k \in \Rr \quad y(x) = ke^{A(x)}\\
\end{array}$$
\end{proof}


\begin{exemple}
Comment résoudre l'équation différentielle $x^2y'=y$ ?
On se place sur l'intervalle $I_+=\,]0,+\infty[$ ou $I_-=\,]-\infty,0[$.
L'équation devient $y'= \frac{1}{x^2}y$. Donc $a(x)=\frac{1}{x^2}$, dont une primitive est
$A(x)=-\frac1x$. Ainsi les solutions cherchées sont
$y(x) = k e^{-\frac1x}$, où $k\in\Rr$.
\end{exemple}


%---------------------------------------------------------------
\subsection{$y' = a(x)y+b(x)$}

Il nous reste le cas général de l'équation différentielle linéaire d'ordre $1$ avec second membre :
\begin{equation}
   y' = a(x) y  + b(x)
  \label{eq:eqdifflinordre1scnd}
  \tag{$E$}
\end{equation}
où $a : I \to \Rr$ et $b: I \to \Rr$ sont des fonctions continues.

L'équation homogène associée est :
\begin{equation}
   y' = a(x) y
  \label{eq:eqdifflinordre1scndhomo}
  \tag{$E_0$}
\end{equation}

Il n'y a pas de nouvelle formule à apprendre pour ce cas.
Il suffit d'appliquer le principe de superposition : les solutions de (\ref{eq:eqdifflinordre1scnd}) s'obtiennent en
ajoutant à une solution particulière de (\ref{eq:eqdifflinordre1scnd}) les solutions de
(\ref{eq:eqdifflinordre1scndhomo}).
Ce qui donne :
\begin{proposition}
Si $y_0$ est une solution de (\ref{eq:eqdifflinordre1scnd}),
alors les solutions de (\ref{eq:eqdifflinordre1scnd}) sont les
fonctions $y : I \to \Rr$ définies par:
$$y(x)= y_0(x)+k e^{A(x)} \qquad \text{ avec } k \in \Rr$$
où $x\mapsto A(x)$ est une primitive de $x \mapsto a(x)$.
\end{proposition}


\bigskip
La recherche de la solution générale de (\ref{eq:eqdifflinordre1scnd}) se réduit
donc à la recherche d'une solution particulière. Parfois ceci
se fait en remarquant une solution évidente. Par exemple,
l'équation différentielle $y'=2xy+4x$
a pour solution particulière $y_0(x)=-2$ ;
donc l'ensemble des solutions de cette équation sont les
$y(x) = -2 + ke^{x^2}$, où $k\in\Rr$.

\bigskip
\bigskip

\textbf{Recherche d'une solution particulière : méthode de variation de la constante.}
\index{variation de la constante}

Le nom de cette méthode est paradoxal mais justifié !
C'est une méthode générale pour trouver une solution
particulière en se ramenant à un calcul de primitive.

La solution générale de (\ref{eq:eqdifflinordre1scndhomo}) $y' = a(x) y$ est donnée
par $y(x)=ke^{A(x)}$, avec $k\in \Rr$ une constante.
La méthode de la variation de la constante consiste à chercher
une solution particulière sous la forme $y_0(x)=k(x)e^{A(x)}$,
où $k$ est maintenant une fonction à déterminer pour que
$y_0$ soit une solution de (\ref{eq:eqdifflinordre1scnd}) $y' = a(x) y  + b(x)$.

Puisque $A'=a$, on a :
$$y_0'(x)=a(x)k(x)e^{A(x)} + k'(x)e^{A(x)}= a(x)y_0(x) + k'(x)e^{A(x)}$$
Ainsi :
$$y_0'(x) - a(x)y_0(x) = k'(x)e^{A(x)}$$

Donc $y_0$ est une solution de (\ref{eq:eqdifflinordre1scnd}) si et seulement si
$$k'(x)e^{A(x)}=b(x) \iff k'(x)=b(x)e^{-A(x)} \iff k(x)=\int b(x)e^{-A(x)}\dd x.$$
Ce qui donne une solution particulière
$y_0(x) = \left(\int b(x)e^{-A(x)}\dd x \right)e^{A(x)}$ de (\ref{eq:eqdifflinordre1scnd})
sur $I$.
La solution générale de  (\ref{eq:eqdifflinordre1scnd}) est donnée par
$$y(x) = y_0(x) + ke^{A(x)}, \quad k\in \Rr.$$


\begin{exemple}
\label{ex:eqdifflinordre1scnd}
Soit l'équation $y'+y = e^x+1$.
L'équation homogène est $y'=-y$ dont les solutions sont
les $y(x) = k e^{-x}$, $k\in\Rr$.

Cherchons une solution particulière avec la méthode de variation de la constante :
on note $y_0(x) = k(x) e^{-x}$. On doit trouver $k(x)$ afin que $y_0$ vérifie l'équation
différentielle $y'+y = e^x+1$.
$$\begin{array}{rcl}
        && y_0'+y_0 = e^x+1  \\
 & \iff & \left(k'(x) e^{-x} - k(x) e^{-x} \right) + k(x) e^{-x} = e^x+1 \\
 & \iff & k'(x)e^{-x} = e^x+1 \\
 & \iff & k'(x) = e^{2x}+e^x \\
 & \iff & k(x) = \frac12e^{2x}+e^x + c \\
\end{array}$$
On fixe $c=0$ (n'importe quelle valeur convient) :
$$y_0(x) =  k(x) e^{-x} = \left(\frac12e^{2x}+e^x\right)e^{-x} = \frac12e^{x}+1$$
Nous tenons notre solution particulière ! Les solutions générales de l'équation
$y'+y = e^x+1$ s'obtiennent en additionnant cette solution particulière aux solutions
de l'équation homogène :
$$y(x) = \frac12e^{x}+1 + k e^{-x}, \qquad k\in\Rr.$$
\end{exemple}


%---------------------------------------------------------------
\subsection{Théorème de Cauchy-Lipschitz}


Voici l'énoncé du théorème de Cauchy-Lipschitz dans le cas des
équations différentielles linéaires du premier ordre.
\begin{theoreme}[Théorème de Cauchy-Lipschitz]
\index{theoreme@théorème!de Cauchy-Lipschitz}
\label{th:cauchylipschitzord1}
Soit $y'=a(x)y + b(x)$ une équation différentielle linéaire du premier ordre,
où $a,b : I \to \Rr$ sont des fonctions continues sur un intervalle ouvert $I$.
Alors, pour tout $x_0 \in I$ et pour tout $y_0 \in \Rr$, il existe une et une
seule solution $y$ telle que $y(x_0)=y_0$.
\end{theoreme}

D'après nos calculs précédents cette solution est :
$$y(x) = \left(\int_{x_0}^x b(t)e^{-A(t)}\dd t \right)e^{A(x)} + y_0e^{A(x)}$$
où $A$ est la primitive de $a$ s'annulant en $x_0$, et cette solution vérifie bien $y(x_0)=y_0$.


\begin{exemple}
Trouver la solution de $y'+y = e^x+1$ vérifiant $y(1)=2$.
Nous avons déjà trouvé toutes les solutions de cette équation dans l'exemple
\ref{ex:eqdifflinordre1scnd} : $y(x) = \frac12e^{x}+1 + k e^{-x}$
où $k\in\Rr$. Nous allons déterminer la constante $k$ afin que la condition initiale
$y(1)=2$ soit vérifiée :
$$y(1)=2 \iff \frac12e^{1}+1 + k e^{-1} = 2 \iff \frac{k}{e} = 1 - \frac{e}{2}
\iff k = e - \frac{e^2}{2}$$
Ainsi la solution cherchée est $y(x) = \frac12e^{x}+1 + \left(e - \frac{e^2}{2}\right)e^{-x}$,
et c'est la seule solution.
\end{exemple}

%---------------------------------------------------------------
\subsection{Courbes intégrales}

Une \defi{courbe intégrale}\index{courbe!integrale@intégrale} d'une équation différentielle $(E)$
est le graphe d'une solution de $(E)$.
Le théorème \ref{th:cauchylipschitzord1} pour les équations
différentielles linéaires du premier ordre $y'=a(x)y + b(x)$
se reformule ainsi :
\mybox{
\og Par chaque point $(x_0,y_0) \in I \times \Rr$
passe une et une seule courbe intégrale. \fg}

\begin{exemple}
Les solutions de l'équation différentielle
$y'+y=x$ sont les
$$y(x) = x-1 + ke^{-x} \quad k \in\Rr$$
et sont définies sur $I=\Rr$.
Pour chaque point $(x_0,y_0) \in \Rr^2$, il existe une unique solution
$y$ telle que $y(x_0)=y_0$. Le graphe de cette solution
est la courbe intégrale passant par $(x_0,y_0)$.

  \myfigure{1}{
    \tikzinput{fig_equadiff04}
  }

\end{exemple}


%---------------------------------------------------------------
\subsection{Exemples}

\begin{exemple}
On considère l'équation différentielle
$(E)$ : $x^3y'+(2-3x^2)y=x^3$.

\begin{enumerate}
\item Résoudre l'équation différentielle $(E)$ sur $]0,+\infty[$ et
  $]-\infty,0[$.

\item Peut-on trouver une solution sur $\Rr$ ?

\item Trouver la solution sur $]0,+\infty[$ vérifiant $y(1)=0$.
\end{enumerate}


\bigskip
\textbf{Correction.}

\begin{enumerate}
  \item
  \begin{enumerate}
    \item Résolution de l'équation homogène $(E_0)$ :
$x^3y'+(2-3x^2)y=0$.
Pour $x\neq 0$, on a $y'=-\frac{2-3x^2}{x^3}y$. Donc la solution générale
de $(E_0)$ est $y(x)=k e^{\int -\frac{2-3x^2}{x^3}\dd x}
=k e^{3\ln |x|}e^{1/x^2}=k|x|^3e^{1/x^2}$.
Donc la solution générale de $(E_0)$ sur $]0,+\infty[$ est :
$y(x)= k_1 x^3e^{1/x^2}$ ; et sur $]-\infty,0[$ : $y(x)= k_2 x^3e^{1/x^2}$.

    \item Résolution de l'équation avec second membre $(E)$ par la méthode de variation de
la constante.
On cherche une solution sous la forme $y(x)=k(x)x^3e^{1/x^2}$. En dérivant
et en remplaçant dans l'équation différentielle, on obtient
$k'(x) x^3e^{1/x^2} =1. $ Donc $k(x)=\int \frac{e^{-1/x^2}}{x^3}\dd x
=\frac{1}{2}e^{-1/x^2}+c$. D'où une solution particulière de $(E)$ sur
$]0,+\infty[$ ou $]-\infty,0[$ :
$y_0(x)=k(x)x^3e^{1/x^2} =\frac{1}{2}x^3$.

    \item Solution générale sur $]0,+\infty[$ :
$y(x)=\frac{1}{2}x^3+k_1x^3e^{1/x^2}$.

Solution générale sur $]-\infty,0[$ :
$y(x)=\frac{1}{2}x^3+k_2x^3e^{1/x^2}$.
  \end{enumerate}

  \item  $x^3e^{1/x^2}$ tend vers $+\infty$ (resp. $-\infty$) lorsque $x \to 0^+$ (resp. $0^-$),
donc pour $k_1$ ou $k_2$ non nul, $y$ ne
peut pas être prolongée par continuité en $0$.
Pour $k_1=k_2=0$, $y(x)=\frac{1}{2}x^3$ est continue et dérivable sur
$\Rr$. C'est la seule solution sur $\Rr$.

  \item Si l'on cherche une solution particulière vérifiant $y(1)=0$,
alors on
a $y(x)=\frac{1}{2}x^3+kx^3e^{1/x^2}$, $y(1)=1/2+ke=0$, donc $k=-\frac{1}{2e}$. Donc
$y(x)=\frac{1}{2}x^3-\frac{1}{2e}x^3e^{1/x^2}$.
\end{enumerate}
\end{exemple}


\begin{exemple}
Résoudre $x(1+x)y'-(x+2)y=2x$.

\begin{enumerate}
  \item \textbf{Équation homogène.}

  L'équation homogène est $x(1+x)y'-(x+2)y=0$.
Pour $x\neq 0$ et $x\neq -1$, l'équation s'écrit $y'=\frac{x+2}{x(1+x)}y$.
La décomposition de la fraction en éléments simples est :
$a(x) = \frac{x+2}{x(1+x)}=\frac{2}{x}-\frac{1}{1+x}$.
Une primitive de $a(x)$ est donc
$A(x) = \int \frac{x+2}{x(1+x)} \dd x = 2\ln |x|-\ln|x+1|$.
La solution générale de l'équation homogène est
$y(x)= k e^{A(x)}= k e^{2\ln |x|-\ln|x+1|}
= k e^{\ln \frac{x^2}{|x+1|}}=k\frac{x^2}{|x+1|} =\pm k\frac{x^2}{x+1}$.
Cette solution est bien définie en $x=0$.
On obtient donc la solution générale de l'équation homogène :
$y(x)=k\frac{x^2}{x+1}$ sur $]-\infty, -1[$ ou sur $]-1,+\infty[$.


  \item \textbf{Solution particulière.}

On cherche une solution de l'équation non homogène sous la forme
$y_0(x)=k(x)\frac{x^2}{x+1}$ par la méthode de variation de la constante.
En remplaçant dans l'équation, on
obtient $k'(x)x^3=2x$. Donc pour $x\neq 0$, on a
$k'(x)=\frac{2}{x^2}$, et $k(x)=-\frac{2}{x}$.
D'où la solution générale de l'équation non homogène
$y(x)=-\frac{2x}{x+1} +k\frac{x^2}{x+1}$. Cette solution est définie
sur $]-\infty, -1[$ ou $]-1,+\infty[$.

  \item \textbf{Existe-t-il une solution définie sur $\Rr$ ?}

  On a $y(x)=\frac{x(k x-2)}{x+1}$. Donc pour $k\neq -2$, on ne peut
prolonger $y$ en $-1$.
Pour $k=-2$, on peut prolonger $y$ en $-1$. On obtient une solution
définie sur $\Rr$ : $y=-2x$.

\end{enumerate}

\end{exemple}


%%%%%%%%%%%%%%%%%%%%%%%%%%%%%%%%%%%%%%%%%%%%%%%%%%%%%%%%%%%%%%%%
\begin{miniexercices}
\sauteligne
\begin{enumerate}
  \item Résoudre l'équation différentielle $y'+ y \ln 2 = 0$.
  Tracer les courbes intégrales. Trouver la solution vérifiant $y(1)=\frac12$.

  \item Résoudre l'équation différentielle $2y'+3y=5$.
  Trouver la solution vérifiant $y(0)=-\frac13$. Tracer la courbe intégrale.

  \item Trouver une solution évidente, puis résoudre l'équation
  différentielle $2xy'+y=1$. Trouver la solution
  vérifiant $y(1)=2$. Tracer la courbe intégrale.
  Même travail avec l'équation $xy'-y=x^2$.

  \item Par la méthode de variation de la constante, trouver une solution
  particulière de l'équation différentielle $y'-2xy=3xe^{x^2}$.
  Même travail avec $y'+2y = \sin(3x)e^{-2x}$.
\end{enumerate}
\end{miniexercices}


%%%%%%%%%%%%%%%%%%%%%%%%%%%%%%%%%%%%%%%%%%%%%%%%%%%%%%%%%%%%%%%%
\section{Équation différentielle linéaire du second ordre à coefficients constants}

%---------------------------------------------------------------
\subsection{Définition}


Une équation différentielle linéaire du second ordre, à
 coefficients constants, est une équation de la forme
\begin{equation}
ay''+by'+cy=g(x)
\label{eq:linscd}
\tag{$E$}
\end{equation}
où $a,b,c \in \Rr$, $a \neq 0$ et $g$ est une fonction continue sur
un intervalle ouvert $I$.

L'équation
\begin{equation}
ay''+by'+cy=0
\label{eq:linscdhom}
\tag{$E_0$}
\end{equation}
est appelée l'équation homogène associée à $(E)$.


La structure des solutions de l'équation est très simple :
\begin{theoreme}
\label{th:eqdiffdim}
L'ensemble des solutions de l'équation homogène (\ref{eq:linscdhom}) est
un $\Rr$-espace vectoriel de dimension~$2$.
\end{theoreme}

Nous admettons ce résultat.


%---------------------------------------------------------------
\subsection{Équation homogène}

On cherche une solution de (\ref{eq:linscdhom}) sous la forme
$y(x)=e^{rx}$ où $r \in \Cc$ est une constante à déterminer. On trouve
\begin{eqnarray*}
     & ay''+by'+cy=0 \\
\iff & (ar^2+br+c)e^{rx}=0 \\
\iff & ar^2+br+c=0.
\end{eqnarray*}

\begin{definition}
L'équation $ar^2+br+c=0$ est appelée
\defi{l'équation caractéristique}\index{equation caracteristique@équation caractéristique} associée à (\ref{eq:linscdhom}).
\end{definition}

Soit $\Delta= b^2-4ac$, le discriminant de l'équation
caractéristique associée à (\ref{eq:linscdhom}).


\begin{theoreme}
\sauteligne
\begin{enumerate}

\item Si $\Delta >0$, l'équation caractéristique possède deux racines réelles distinctes
$r_1\neq r_2$ et les solutions de (\ref{eq:linscdhom}) sont les
\mybox{$y(x) = \lambda e^{r_1x}+ \mu e^{r_2x} \quad \text{ où } \lambda, \mu \in \Rr.$}

\item Si $\Delta=0$, l'équation caractéristique possède une racine double $r_0$
et les solutions de (\ref{eq:linscdhom}) sont les
\mybox{$y(x) = (\lambda+\mu x)e^{r_0 x} \quad \text{ où } \lambda, \mu \in \Rr.$}

\item Si $\Delta<0$, l'équation caractéristique possède deux racines complexes
conjuguées $r_1=\alpha+\ii \beta$, $r_2=\alpha-\ii \beta$ et les solutions de (\ref{eq:linscdhom}) sont les
\mybox{$y(x) = e^{\alpha x}\big(\lambda\cos (\beta x)+\mu\sin (\beta x)\big) \quad \text{ où }
\lambda, \mu \in \Rr.$}

\end{enumerate}
\end{theoreme}

\begin{exemple}
\sauteligne
\begin{enumerate}
  \item Résoudre sur $\Rr$ l'équation $y'' - y' - 2y = 0$.

L'équation caractéristique est $r^2 - r - 2 = 0$,  qui s'écrit aussi $(r+1)(r-2) = 0$ ($\Delta >0$).
D'où, pour tout $x \in \Rr$, $y(x) = \lambda e^{-x} + \mu e^{2x}$,
avec $\lambda,\mu \in \Rr$.

  \item Résoudre sur $\Rr$ l'équation $y'' - 4y' + 4y = 0$.

L'équation caractéristique est $r^2 - 4r + 4 = 0$,
soit $(r-2)^2 = 0$ ($\Delta=0$). D'où, pour tout $x \in \Rr$,
$y(x) = (\lambda x + \mu) e^{2x}$, avec $\lambda,\mu \in \Rr$.

  \item Résoudre sur $\Rr$ l'équation $y'' - 2y' + 5y = 0$.

L'équation caractéristique est $r^2-2r+5 = 0$.
Elle admet deux solutions complexes conjuguées :
$r_1 = 1 + 2\ii$ et $r_2 = 1 - 2\ii$ ($\Delta<0$).
D'où, pour tout $x \in \Rr$,
$y(x) = e^x (\lambda \cos(2x) + \mu \sin(2x))$, avec $\lambda, \mu \in \Rr$.
\end{enumerate}
\end{exemple}



\begin{proof}
La preuve consiste à trouver deux solutions linéairement indépendantes,
ce qui permet d'affirmer qu'elles forment une base d'après
le théorème \ref{th:eqdiffdim} (que l'on a admis).
\begin{enumerate}
  \item Si $\Delta>0$, alors l'équation caractéristique a deux racines réelles
distinctes $r_1, r_2$. On obtient ainsi deux solutions $y_1=e^{r_1x},
y_2=e^{r_2x}$ qui sont linéairement indépendantes car $r_1 \neq r_2$.
Comme l'espace des solutions est un espace vectoriel de dimension $2$
(par le théorème \ref{th:eqdiffdim}), alors une base de
l'espace des solutions de (\ref{eq:linscdhom})
est $\big\{e^{r_1x}, e^{r_2x}\big\}$.

La solution générale de  (\ref{eq:linscdhom}) s'écrit
$y(x) = \lambda e^{r_1x} + \mu e^{r_2x},$ où $\lambda,
\mu\in\Rr$.

  \item Si $\Delta=0$, alors l'équation caractéristique a
une racine réelle double $r_0$. On
obtient ainsi une solution $y_1=e^{r_0x}$. On vérifie que $y_2=xe^{r_0x}$
est aussi une solution :
$ay''_2+by'_2+cy_2=(2ar_0+ar_0^2x)e^{r_0x}+(b+br_0x)e^{r_0x}+cxe^{r_0x}
=(2ar_0+b)e^{r_0x}=0$ car $2ar_0+b=P'(r_0)=0$, où $P(r) = ar^2+br+c$.
Ces deux solutions sont linéairement indépendantes.
Une base de l'espace des solutions est $\big\{e^{r_0x}, xe^{r_0x}\big\}$,
et la solution générale de  (\ref{eq:linscdhom}) s'écrit
$y(x) = (\lambda  + \mu x) e^{r_0x}$, où $\lambda,\mu \in\Rr$.

  \item Si $\Delta<0$, alors l'équation caractéristique a deux
racines complexes conjuguées $r_1=\alpha+\ii\beta, r_2=\alpha-\ii\beta$.
On obtient deux solutions complexes $Y_1=e^{(\alpha+\ii\beta)x}=e^{\alpha x}e^{\ii\beta x},
Y_2=e^{(\alpha-\ii\beta)x}=e^{\alpha x}e^{-\ii\beta x}$.
Comme les parties réelles et imaginaires sont des
solutions réelles, on obtient deux solutions réelles
$y_1=e^{\alpha x}\cos(\beta x)$, $y_2=e^{\alpha x}\sin(\beta x)$, qui sont
linéairement indépendantes.
Alors, une base de l'espace des solutions est
$\big\{e^{\alpha x}\cos(\beta x), e^{\alpha x}\sin(\beta x)\big\}$.
La solution générale de (\ref{eq:linscdhom}) s'écrit
$y(x)=e^{\alpha x}(\lambda \cos(\beta x) +
\mu\sin(\beta x))$, où $\lambda,\mu \in \Rr$.
\end{enumerate}
\end{proof}

%---------------------------------------------------------------
\subsection{Équation avec second membre}

Nous passons au cas général d'une équation différentielle linéaire d'ordre $2$, à coefficients constants,
mais avec un second membre $g$ qui est une fonction continue sur
un intervalle ouvert $I \subset \Rr$ :
\begin{equation}
ay''+by'+cy=g(x)
%\label{eq:linscd}
\tag{$E$}
\end{equation}

Pour ce type d'équation, nous admettons le théorème de Cauchy-Lipschitz qui s'énonce ainsi :
\begin{theoreme}[Théorème de Cauchy-Lipschitz]
\index{theoreme@théorème!de Cauchy-Lipschitz}
Pour chaque $x_0\in I$ et chaque couple $(y_0,y_1) \in \Rr^2$,
l'équation (\ref{eq:linscd}) admet une \evidence{unique}
solution $y$ sur $I$ satisfaisant aux conditions initiales :
\mybox{$y(x_0) = y_0$ \quad et \quad $y'(x_0) = y_1$.}
\end{theoreme}


Dans la pratique, pour résoudre une équation différentielle linéaire avec second membre
(avec ou sans conditions initiales), on cherche d'abord une solution de l'équation homogène,
puis une solution particulière de l'équation avec second membre et on applique
le principe de superposition :
\begin{proposition}
Les solutions générales de l'équation (\ref{eq:linscd}) s'obtiennent en
ajoutant les solutions générales de l'équation homogène (\ref{eq:linscdhom})
à une solution particulière de (\ref{eq:linscd}).
\end{proposition}

Il reste donc à déterminer une solution particulière.

%---------------------------------------------------------------
\subsection{Recherche d'une solution particulière}

On donne deux cas particuliers importants et une méthode générale.


\bigskip

\textbf{Second membre du type $e^{\alpha x}P(x)$.}

Si $g(x)=e^{\alpha x}P(x)$, avec $\alpha \in \Rr$ et $P\in \Rr[X]$,
alors on cherche une solution particulière sous la forme
$y_0(x)=e^{\alpha x}x^{m}Q(x)$, où $Q$ est un polynôme de
même degré que $P$ avec :
\begin{itemize}
\item $y_0(x)=e^{\alpha x}Q(x)$ ($m=0$), si $\alpha$ n'est pas une racine de l'équation caractéristique,
\item $y_0(x)=xe^{\alpha x}Q(x)$ ($m=1$), si $\alpha$ est une racine simple de l'équation caractéristique,
\item $y_0(x)=x^2e^{\alpha x}Q(x)$ ($m=2$), si $\alpha$ est une racine double de l'équation caractéristique.
\end{itemize}


\bigskip

\textbf{Second membre du type $e^{\alpha x}\big(P_1(x)\cos (\beta x)+P_2(x)\sin (\beta x)\big)$.}

Si $g(x)=e^{\alpha x} \big(P_1(x)\cos (\beta x)+P_2(x)\sin (\beta x)\big)$,
où $\alpha ,\beta \in \Rr$ et $P_1,P_2\in\Rr[X]$, on cherche une solution particulière sous la forme :
\begin{itemize}
\item $y_0(x)=e^{\alpha x} \big( Q_1(x)\cos (\beta x)+Q_2(x)\sin (\beta x) \big)$,
si $\alpha +\ii \beta$ n'est pas une racine de l'équation caractéristique,

\item $y_0(x)=xe^{\alpha x}  \big( Q_1(x)\cos (\beta x)+Q_2(x)\sin (\beta x) \big)$,
si $\alpha +\ii \beta$ est une racine de l'équation caractéristique.
\end{itemize}
Dans les deux cas, $Q_1$ et $Q_2$ sont deux polynômes
de degré $n=\max\{\deg P_1,\deg P_2\}$.



\begin{exemple}
Résoudre les équations différentielles :
$$(E_0) \  y''-5y'+6y=0 \quad
(E_1) \  y''-5y'+6y=4xe^x \quad
(E_2) \  y''-5y'+6y=4xe^{2x}$$

Trouver la solution de $(E_1)$ vérifiant $y(0)=1$ et $y'(0)=0$.


\begin{enumerate}
  \item \textbf{Équation $(E_0)$.}
  L'équation caractéristique est $r^2-5r+6=(r-2)(r-3)=0$, avec deux
racines distinctes $r_1=2, r_2=3$.
Donc l'ensemble des solutions de $(E_0)$ est $\big\{\lambda e^{2x}+ \mu e^{3x} \mid
\lambda, \mu \in \Rr\big\}$.



  \item \textbf{Équation $(E_1)$.}
  \begin{enumerate}
    \item On cherche une solution particulière à $(E_1)$ sous la forme
$y_0(x)=(ax+b)e^x$. Lorsque l'on injecte $y_0$ dans l'équation $(E_1)$, on obtient :
$$\begin{array}{rl}
& (ax+2a+b)e^x-5(ax+a+b)e^x+6(ax+b)e^x=4xe^x \\
\iff & (a-5a+6a)x+2a+b-5(a+b)+6b=4x \\
\iff & 2a=4 \ \text{ et }\ -3a+2b=0 \\
\iff & a=2 \ \text{ et }\ b=3
\end{array}$$
 Donc $y_0(x)=(2x+3)e^x$.

    \item L'ensemble des solutions de $(E_1)$ est
$\big\{(2x+3)e^x+\lambda e^{2x} + \mu e^{3x} \mid \lambda,\mu\in \Rr\big\}$.

    \item On a $y(x)=(2x+3)e^x+ \lambda e^{2x} + \mu e^{3x}$. On cherche $\lambda$,
$\mu$ tels que $y(0)=1, y'(0)=0$. C'est-à-dire que
$3+\lambda+\mu=1$, $5+2\lambda+3\mu=0$. Donc $\lambda=-1, \mu=-1$, c'est-à-dire que
$y(x)=(2x+3)e^{x}-e^{2x}-e^{3x}$.
  \end{enumerate}
  \item \textbf{Équation $(E_2)$.}
  Comme $2$ est une racine de l'équation caractéristique, on
  cherche une solution particulière sous la forme $y_0(x)=x(ax+b)e^{2x}$.
  On obtient $y_0(x)=x(-2x-4)e^{2x}$.
\end{enumerate}
\end{exemple}

\bigskip


\textbf{Méthode de variation des constantes.}
\index{variation de la constante}

Si $\{y_1,y_2\}$ est une base de solutions de l'équation homogène (\ref{eq:linscdhom}),
on cherche une solution particulière sous la forme
$y_0= \lambda y_1 + \mu y_2$, mais cette fois $\lambda$ et $\mu$ sont deux fonctions
vérifiant :
\mybox{
($S$) \qquad $
\left\{\begin{array}{ccl}
\lambda'y_1+\mu'y_2&=&0\\
\lambda'y'_1+\mu'y'_2&=& \frac{g(x)}{a}.
\end{array}\right.
$
}


Pourquoi cela ?
Si $y_0= \lambda y_1 + \mu y_2$ est une telle fonction, alors :
$$y_0'= \lambda' y_1 + \mu' y_2 + \lambda y_1' + \mu y_2' = \lambda y_1'+ \mu y_2'$$
$$y_0'' = \lambda' y_1'+ \mu' y_2' + \lambda y_1''+ \mu y_2'' = \frac{g(x)}{a} + \lambda y_1''+ \mu y_2''$$
Ainsi l'équation (\ref{eq:linscd}) est vérifiée par $y_0$ :
\begin{eqnarray*}
 ay_0''+by_0'+cy_0
   &=& a\left(\frac{g(x)}{a} + \lambda y_1''+ \mu y_2''\right) + b\big(\lambda y_1'+ \mu y_2'\big) + c\big(\lambda y_1 + \mu y_2\big)\\
   &=& g(x) + \lambda\big(ay_1''+by_1'+cy_1\big) + \mu\big(ay_2''+by_2'+cy_2\big)\\
   &=& g(x)
\end{eqnarray*}
On a utilisé le fait que $y_1$ et $y_2$ sont solutions de l'équation homogène.
Le système ($S$) se résout facilement, ce qui donne $\lambda'$ et $\mu'$,
puis $\lambda$ et $\mu$ par intégration.


\begin{exemple}
Résoudre l'équation suivante, sur l'intervalle $]-\frac\pi2,+\frac\pi2[$ :
$$y'' + y = \frac{1}{\cos x}$$

Les solutions de l'équation homogène $y'' + y =0$ sont
$\lambda \cos x  + \mu \sin x$ où $\lambda,\mu\in\Rr$.

On cherche une solution particulière de l'équation avec second membre sous la forme
$$y_0(x) =\lambda(x) \cos x  + \mu(x) \sin x$$
où cette fois $\lambda(x),\mu(x)$ sont des fonctions à trouver et qui vérifient ($S$) :
$$
\left\{\begin{array}{ccl}
\lambda'y_1+\mu'y_2&=&0\\
\lambda'y'_1+\mu'y'_2&=& \frac{g(x)}{a}
\end{array}\right.
\quad \text{ donc } \quad
\left\{\begin{array}{ccl}
\lambda' \cos x + \mu' \sin x &=&0\\
-\lambda' \sin x + \mu' \cos x &=& \frac{1}{\cos x}.
\end{array}\right.
$$
En multipliant la première ligne par $\sin x$ et la seconde par $\cos x$, on obtient
$$
\left\{\begin{array}{ccl}
\lambda' \cos x \sin x + \mu' (\sin x)^2 &=&0\\
-\lambda' \cos x \sin x + \mu' (\cos x)^2 &=& 1
\end{array}\right.
\quad \text{ donc par somme } \quad
\mu'=1.
$$
Ainsi $\mu(x) = x$ et la première ligne des équations devient
$\lambda' = -\frac{\sin x}{\cos x}$ donc $\lambda(x) = \ln(\cos x)$.

On vérifie pour se rassurer que $y_0(x) = \ln(\cos x) \cos x + x\sin x$ est une solution
de l'équation. Ainsi les fonctions solutions sont de la forme :
$$\lambda \cos x  + \mu \sin x + \ln(\cos x) \cos x + x\sin x$$
quels que soient $\lambda,\mu\in \Rr$.

\end{exemple}


%%%%%%%%%%%%%%%%%%%%%%%%%%%%%%%%%%%%%%%%%%%%%%%%%%%%%%%%%%%%%%%%
\begin{miniexercices}
\sauteligne
\begin{enumerate}
  \item Résoudre l'équation différentielle $y'' + \omega^2y=0$.
  Trouver la solution vérifiant $y(0)=1$ et $y'(0)=1$.
  Tracer la courbe intégrale.
  Résoudre l'équation différentielle $y'' + \omega^2y= \sin(\omega x)$.

  \item Résoudre l'équation différentielle $y'' + y' -6y=0$.
  Trouver la solution vérifiant $y(-1)=1$ et $y'(-1)=0$.
  Tracer la courbe intégrale.
  Résoudre l'équation différentielle $y'' + y' -6y= e^x$.

  \item Résoudre l'équation différentielle $2y'' -2 y' +\frac12 y=0$.
  Trouver la solution ayant une limite finie lorsque $x\to +\infty$.
  Résoudre $2y'' -2 y' +\frac12 y=x-1$.
\end{enumerate}
\end{miniexercices}


%%%%%%%%%%%%%%%%%%%%%%%%%%%%%%%%%%%%%%%%%%%%%%%%%%%%%%%%%%%%%%%%
\section{Problèmes conduisant à des équations différentielles}

%---------------------------------------------------------------
\subsection{Parachutiste}


Revenons sur l'exemple du parachutiste de l'introduction :
sa vitesse verticale vérifie l'équation différentielle
$$\frac{\dd v(t)}{\dd t}= g - f v(t)$$
où $g$ (la constante de gravitation) et $f$ (le coefficient de frottement) sont des
constantes.

  \myfigure{1}{
    \tikzinput{fig_equadiff02}
  }

Nous avons tous les ingrédients pour trouver $v$.

\begin{itemize}
  \item \textbf{Équation homogène.}
  Les solutions de l'équation homogène $v'(t)=-fv(t)$ sont les $v(t)= ke^{-ft}$, $k\in\Rr$.

  \item \textbf{Solution particulière.}
  On cherche une solution particulière $v_p(t)=k(t)e^{-ft}$ de l'équation
  $v'= g - f v$ par la méthode de variation de la constante :
  $v_p'(t) = k'(t)  e^{-ft} -f k(t)e^{-ft}$. Pour que
  $v_p$ soit solution de l'équation différentielle il faut et il suffit donc que
  $k'(t)e^{-ft}=g$. Ainsi $k'(t)=ge^{ft}$ donc, par exemple, $k(t)=\frac{g}{f}e^{ft}$.
  Ainsi $v_p(t)=\frac{g}{f}$.

  \item \textbf{Solutions générales.}
  La solution générale de l'équation est donc $v(t)=\frac{g}{f}+k e^{-ft}$, $k\in\Rr$.

  \item \textbf{Condition initiale.}
  Si à l'instant $t=0$ le parachute se lance avec une vitesse initiale nulle, c'est-à-dire
  $v(0)=0$, alors sa vitesse est :
  $$v(t)=\frac{g}{f}-\frac{g}{f}e^{-ft}.$$

  \myfigure{1.2}{
    \tikzinput{fig_equadiff05}
  }

  \item \textbf{Vitesse limite.}
  Lorsque $t\to+\infty$, $v(t) \to v_\infty = \frac{g}{f}$, qui représente la vitesse limite
  que le parachutiste ne peut dépasser.
  Expérimentalement, on mesure que $v_\infty$ vaut environ \SI{5}{\meter\per\second}
  (soit environ \SI{20}{\kilo\meter\per\hour}), et comme $g \approx \SI{9.81}{\meter\per\second^2}$,
  cela permet de calculer le coefficient de frottement $f$.

  \item \textbf{Position.}
  Comme  $v(t) = \frac{\dd x(t)}{\dd t}$, trouver la position $x$ revient à trouver
  une primitive de $v$ :
  $$x(t) = \frac{g}{f} t + \frac{g}{f^2}\big(e^{-ft}-1\big)$$
  en prenant comme convention $x(0)=0$.
\end{itemize}

Ceci n'est bien sûr qu'un \evidence{modèle} qui ne correspond
pas parfaitement à la réalité, mais permet cependant de mettre en évidence
des propriétés vérifiées par les conditions expérimentales, comme la vitesse limite par exemple.


%---------------------------------------------------------------
\subsection{Demi-vie}

Dans un tissu radioactif, la vitesse de désintégration des noyaux radioactifs
est proportionnelle au nombre de noyaux radioactifs $N(t)$ présents dans le
tissu à l'instant $t$. Il existe donc une constante $\lambda$ strictement positive telle que :
$$N'(t) = -\lambda N(t)$$
Le signe \og $-$\fg{} de cette équation différentielle traduit la
décroissance du nombre de noyaux.
Si $N_0$ désigne le nombre de noyaux à l'instant initial, on a donc :
$$N(t) = N_0 e^{-\lambda t}$$
Dans ce contexte apparaissent souvent deux grandeurs qu'il est bon de
savoir interpréter graphiquement :
\begin{itemize}
  \item Le \defi{temps caractéristique}, noté $\tau$, est défini par :
  $$\tau = \frac{1}{\lambda}$$
  Si $(T)$ désigne la tangente à l'origine de la courbe $(C)$ de la fonction $N$,
  le temps caractéristique $\tau$ est l'abscisse du point d'intersection de la
  droite $(T)$ avec l'axe du temps. En effet, une équation de $(T)$ est :
  $$y = N'(0)t + N(0) = -\lambda N_0 t + N_0$$
On constate que si $t = \tau$, on a bien $y = 0$.
Plus le temps caractéristique est petit,
plus la vitesse de désintégration initiale est élevée.

  \myfigure{1.5}{
    \tikzinput{fig_equadiff06}
  }


  \item La \defi{période de demi-vie}, notée $\tau_{1/2}$,
  est la période au bout de laquelle la moitié des noyaux se
  sont désintégrés. On a donc :
  $$N(\tau_{1/2}) = \frac{N_0}{2}$$
  Donc $N_0 e^{ -\lambda \tau_{1/2} } =   \frac{N_0}{2}$, d'où
  $\lambda \tau_{1/2} = \ln 2$.
  Ainsi :
  $$\tau_{1/ 2} = \frac{\ln 2}{\lambda} = \tau \ln 2$$


  On peut aussi exprimer $N(t)$ en fonction de la période de demi-vie :
  $$N(t) = N_0 e^{-\lambda t} = N_0 e^{-\frac{t}{\tau}}
  =  N_0 e^{-\frac{t}{\tau_{1/2}} \ln 2} = N_0 2^{-\frac{t}{\tau_{1/2}}}$$

    Notez que $\tau_{1/ 2}$ ne dépend pas de $N_0$, et c'est bien le
    temps nécessaire pour que la moitié des noyaux se
  soient désintégrés, ce quel que soit l'instant initial :
  $$N(t+\tau_{1/ 2}) = N_0 2^{-\frac{t+\tau_{1/2}}{\tau_{1/2}}}
  = N_0 2^{-\frac{t}{\tau_{1/2}}-1} = \frac12 N_0 2^{-\frac{t}{\tau_{1/2}}}
  = \frac{N(t)}{2}$$

\end{itemize}



%---------------------------------------------------------------
\subsection{Modèles d'évolution}

On considère une culture de bactéries en milieu clos.
Soit $N_0$ le nombre de bactéries introduites dans la culture à l'instant $t = 0$.

\bigskip

\textbf{Loi de Malthus.}

Un premier modèle est de supposer que la vitesse d'accroissement des bactéries
est proportionnelle au nombre de bactéries en présence.
Cela signifie que le nombre $N(t)$ de bactéries vérifie l'équation différentielle
$$y' = ay,$$
où $a>0$ est une constante dépendant des conditions expérimentales.
Nous savons résoudre cette équation ! Ainsi selon ce modèle
$$N(t) = N_0 e^{at}.$$

Le milieu étant limité (en volume, en éléments nutritifs,\ldots),
le nombre de bactéries ne peut pas croître indéfiniment de façon exponentielle.
Ce modèle ne peut donc s'appliquer sur une longue période.

\bigskip

\textbf{Modèle de Verhulst.}

Pour tenir compte de ces observations, on présente un autre modèle d'évolution.
On suppose que le nombre $N(t)$ de bactéries vérifie l'équation différentielle
\begin{equation}
  y' = ay(M - y),
  \label{eq:eqdiffverhulst}
  \tag{$E$}
\end{equation}
où $a>0$ et $M>0$ sont des constantes.

On cherche les solutions $y$ de (\ref{eq:eqdiffverhulst}) telles que
$y(t)>0$ pour $t\in I = [0,+\infty[$.
Supposons qu'une telle solution $y$ existe.

\begin{itemize}


  \item \textbf{Changement de fonction.}

  On transforme l'équation (\ref{eq:eqdiffverhulst})
  en une équation plus facile à résoudre.
  Pour cela on pose $z(x) = \frac{1}{y(x)}$.
  La fonction $z$ est dérivable sur $I$ et :
  $$z'=-\frac{y'}{y^2} = \frac{ay(y-M)}{y^2} = a - \frac{aM}{y} = a-aMz$$


  \item \textbf{Solutions $z$.}

  Ainsi la fonction $z$ doit vérifier l'équation différentielle
  $$z' = a-aMz,$$
  qui est une équation différentielle linéaire d'ordre $1$
  à coefficients constants avec second membre constant.
  On en déduit que, pour tout $x \in I$,
  $$z(x) = k e^{-aMx} + \frac1M$$
  où $k \in \Rr$ est une constante.

  \item \textbf{Solutions $y$.}

  Cela permet d'obtenir $y$ :
  $$y(x) = \frac{1}{z(x)} = \frac{1}{k e^{-aMx} + \frac1M} = \frac{M}{kM e^{-aMx}+1}$$
  La constante $k$ est déterminée par la condition initiale
  $y(0) = \frac{M}{kM+1}=N_0$, ainsi $k = \frac{1}{N_0}-\frac{1}{M}$.

  \item \textbf{Exemple.}
  On suppose $N_0 = 0,01$ (en million de bactéries) et $M=1$, $a=1$.
  Alors $k = \frac{1}{N_0}-\frac{1}{M} = 99$.
  Ainsi selon ce modèle :
  $$N(t) = \frac{1}{1 + 99 e^{-t}}$$
  Il est clair que $0<N(t)<1$ pour tout $t\ge0$, et
  $N(t) \to 1$ lorsque $t\to+\infty$.

  Pour connaître les variations de la fonction $N$, nul besoin de calculs car on sait déjà que
  $N$ est solution de l'équation différentielle (\ref{eq:eqdiffverhulst}), donc
  $N'(t) = N(t)(1 - N(t))$.
  Ainsi $N'(t)>0$, donc la fonction $N$ est croissante.
\end{itemize}

  \myfigure{1.5}{
    \tikzinput{fig_equadiff07}
  }


Le modèle de Verhulst a l'avantage de
bien faire apparaître un comportement asymptotique particulier :
le nombre de bactéries finit par se stabiliser.




%---------------------------------------------------------------
\subsection{Masse attachée à un ressort}

Une masse est attachée à un ressort.
Quelles sont les forces qui s'appliquent à cette masse ?
\begin{itemize}
  \item Un poids $\vec P$,
  \item une réaction $\vec R=-\vec P$ qui s'oppose au poids,
  \item une force de rappel $\vec T$,
  \item une force de frottement $\vec F$.
\end{itemize}


  \myfigure{0.9}{
    \tikzinput{fig_equadiff09}
  }

\subsubsection*{Principe fondamental de la mécanique}

Le principe fondamental de la mécanique s'écrit :
$$\vec P + \vec R + \vec T + \vec F = m\vec a$$
Il est à noter que la réaction s'opposant au poids, on a $\vec P+\vec R = \vec 0$,
et l'équation devient :
$$\vec T + \vec F = m\vec a$$



\subsubsection*{Force de rappel}

La force de rappel est une force horizontale. Elle est nulle
à la position d'équilibre, qui sera pour nous l'origine $x=0$.
Si on écarte davantage la masse du mur,
la force de rappel est un vecteur horizontal qui pointe
vers la position d'équilibre (vers la gauche sur le dessin).
Si on rapproche la masse du mur, le ressort se comprime,
et la force de rappel est un vecteur horizontal qui pointe
encore vers la position d'équilibre (cette fois vers la droite
sur le dessin). On modélise la force de rappel par
$$\vec T = -k x \vec i$$
où $x$ est la position de la masse (on peut avoir $x\ge0$ ou $x\le 0$), et $k>0$
est une constante qui dépend du ressort.

  \myfigure{0.35}{
    \tikzinput{fig_equadiff09a}\ 
    \tikzinput{fig_equadiff09b}\ 
    \tikzinput{fig_equadiff09c}\ 
  }


\subsubsection*{Oscillations sans frottements}

Dans un premier temps, on suppose qu'il n'y a pas de
frottement : $\vec F = \vec 0$.

Le principe fondamental de la mécanique, considéré uniquement sur l'axe horizontal,
s'écrit alors :
$$-kx(t) = m\frac{\dd^2 x(t)}{\dd t^2}$$

Il s'agit donc de résoudre l'équation différentielle du second ordre :
$$y'' + \frac{k}{m} y = 0$$

L'équation caractéristique est $r^2+\frac{k}{m} = 0$,
dont les solutions sont les nombres complexes
$r_1 = +\ii\sqrt{\frac{k}{m}}$ et $r_2 = -\ii\sqrt{\frac{k}{m}}$.
Nous sommes dans le cas $\Delta = -4\frac{k}{m} <0$.
Les solutions de cette équation caractéristique sont de la forme
$\alpha \pm \ii \beta$ avec $\alpha=0$, $\beta = \sqrt{\frac{k}{m}}$,
ce qui fait que les solutions de l'équation différentielle sont
les :
$$y(x) = e^{\alpha x}\big(\lambda\cos (\beta x)+\mu\sin (\beta x)\big)$$

Dans notre situation (la fonction inconnue est $x$ et la variable $t$) :
$$x(t) = \lambda\cos \left(\sqrt{\tfrac{k}{m}}t\right)
+\mu\sin \left(\sqrt{\tfrac{k}{m}}t\right)
\qquad (\lambda, \mu \in \Rr)$$

\begin{exemple}
On lâche la masse au point d'abscisse $1$, sans vitesse initiale.
Cela nous donne les conditions initiales
$x(0)=1$ et $x'(0)=0$.
Comme $x(0)=1$ alors $\lambda=1$. Comme
$x'(0) = 0$ alors $\mu = 0$.
Ainsi on trouve une solution périodique :
$$x(t) = \cos \left(\sqrt{\tfrac{k}{m}}t\right)$$
  \myfigure{1}{
    \tikzinput{fig_equadiff10}
  }

\end{exemple}


\subsubsection*{Oscillations avec faibles frottements}

On rajoute une force de frottement $\vec F =  -fm \frac{\dd x(t)}{\dd t}$
qui est proportionnelle à la vitesse et s'oppose au déplacement
($f$ est le coefficient de frottement).
Le principe fondamental de la mécanique devient :
$$-kx(t)-fm \frac{\dd x(t)}{\dd t} = m\frac{\dd^2 x(t)}{\dd t^2}$$

Il s'agit donc de résoudre l'équation différentielle :
$$y'' + f y' + \frac{k}{m} y = 0$$

L'équation caractéristique est cette fois $r^2+fr+\frac{k}{m}=0$.
Son discriminant est  $\Delta = f^2-4\frac{k}{m}$.
Supposons que le coefficient de frottement $f$ soit faible, c'est-à-dire
que $\Delta = f^2-4\frac{k}{m} <0$, comme dans le cas sans frottement.
On note $\delta = \sqrt{|\Delta|} = \sqrt{4\frac{k}{m} - f^{2}}$.
Les deux solutions sont
$r_1 = \alpha + \ii \beta$ et $r_2 = \alpha - \ii \beta$ avec
$\alpha = -\frac{f}{2}$ et $\beta = \frac{\delta}{2} = \sqrt{\frac{k}{m} - \frac{f^{2}}{4}}$.
Les solutions de l'équation différentielle sont encore de la forme :
$$y(x) = e^{\alpha x}\big(\lambda\cos (\beta x)+\mu\sin (\beta x)\big)$$
Ce qui donne ici :
$$x(t) = e^{-\frac{f}{2} t}\left(\lambda\cos \left(\tfrac{\delta}{2}t\right)
+\mu\sin \left(\tfrac{\delta}{2}t\right)\right)\qquad (\lambda, \mu \in \Rr)$$

Cette fois la solution n'est plus périodique, mais correspond à un mouvement
oscillant amorti, qui tend vers la position d'équilibre $x=0$.

  \myfigure{1}{
    \tikzinput{fig_equadiff11}
  }


%%%%%%%%%%%%%%%%%%%%%%%%%%%%%%%%%%%%%%%%%%%%%%%%%%%%%%%%%%%%%%%%
\begin{miniexercices}
\sauteligne
\begin{enumerate}
  \item Un circuit électrique constitué d'un condensateur
  de capacité $C$ se décharge dans une résistance $R$.
  Calculer l'évolution de la charge électrique qui vérifie
  $q(t) = -RC \frac{\dd q(t)}{\dd t}$.

  \item Calculer et tracer les solutions du système masse-ressort
  pour différents niveaux de frottements.

  \item Un tasse de café de température $T_0= \SI{100}{\celsius}$
  est posée dans une pièce de température $T_\infty = \SI{20}{\celsius}$.
  La loi de Newton affirme que la vitesse de décroissance de la température
  $\frac{\dd T(t)}{\dd t}$ est proportionnelle à l'écart entre sa température
  $T(t)$ et la température ambiante $T_\infty$.
  Sachant qu'au bout de \SI{3}{\minute} la température du café
  est passée à $\SI{80}{\celsius}$, combien de temps faudra-t-il
  pour avoir un café à $\SI{65}{\celsius}$ ?
\end{enumerate}
\end{miniexercices}

\auteurs{

D'après un cours de Gilles Costantini pour le site \texttt{\href{http://www.bacamaths.net/}{Bacamaths}}
et des cours de Guoting Chen et Abdellah Hanani

Repris et mixés par Arnaud Bodin

Relu par Stéphanie Bodin et Vianney Combet

}


\finchapitre
\end{document}



