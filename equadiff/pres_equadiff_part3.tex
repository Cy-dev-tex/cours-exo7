
%%%%%%%%%%%%%%%%%% PREAMBULE %%%%%%%%%%%%%%%%%%

\documentclass[aspectratio=169,utf8]{beamer}
%\documentclass[aspectratio=169,handout]{beamer}

\usetheme{Boadilla}
%\usecolortheme{seahorse}
\usecolortheme[RGB={245,66,24}]{structure}
\useoutertheme{infolines}

% packages
\usepackage{amsfonts,amsmath,amssymb,amsthm}
\usepackage[utf8]{inputenc}
\usepackage[T1]{fontenc}
\usepackage{lmodern}

\usepackage[francais]{babel}
\usepackage{fancybox}
\usepackage{graphicx}

\usepackage{float}
\usepackage{xfrac}

%\usepackage[usenames, x11names]{xcolor}
\usepackage{tikz}
\usepackage{pgfplots}
\usepackage{datetime}



%-----  Package unités -----
\usepackage{siunitx}
\sisetup{locale = FR,detect-all,per-mode = symbol}

%\usepackage{mathptmx}
%\usepackage{fouriernc}
%\usepackage{newcent}
%\usepackage[mathcal,mathbf]{euler}

%\usepackage{palatino}
%\usepackage{newcent}
% \usepackage[mathcal,mathbf]{euler}



% \usepackage{hyperref}
% \hypersetup{colorlinks=true, linkcolor=blue, urlcolor=blue,
% pdftitle={Exo7 - Exercices de mathématiques}, pdfauthor={Exo7}}


%section
% \usepackage{sectsty}
% \allsectionsfont{\bf}
%\sectionfont{\color{Tomato3}\upshape\selectfont}
%\subsectionfont{\color{Tomato4}\upshape\selectfont}

%----- Ensembles : entiers, reels, complexes -----
\newcommand{\Nn}{\mathbb{N}} \newcommand{\N}{\mathbb{N}}
\newcommand{\Zz}{\mathbb{Z}} \newcommand{\Z}{\mathbb{Z}}
\newcommand{\Qq}{\mathbb{Q}} \newcommand{\Q}{\mathbb{Q}}
\newcommand{\Rr}{\mathbb{R}} \newcommand{\R}{\mathbb{R}}
\newcommand{\Cc}{\mathbb{C}} 
\newcommand{\Kk}{\mathbb{K}} \newcommand{\K}{\mathbb{K}}

%----- Modifications de symboles -----
\renewcommand{\epsilon}{\varepsilon}
\renewcommand{\Re}{\mathop{\text{Re}}\nolimits}
\renewcommand{\Im}{\mathop{\text{Im}}\nolimits}
%\newcommand{\llbracket}{\left[\kern-0.15em\left[}
%\newcommand{\rrbracket}{\right]\kern-0.15em\right]}

\renewcommand{\ge}{\geqslant}
\renewcommand{\geq}{\geqslant}
\renewcommand{\le}{\leqslant}
\renewcommand{\leq}{\leqslant}
\renewcommand{\epsilon}{\varepsilon}

%----- Fonctions usuelles -----
\newcommand{\ch}{\mathop{\text{ch}}\nolimits}
\newcommand{\sh}{\mathop{\text{sh}}\nolimits}
\renewcommand{\tanh}{\mathop{\text{th}}\nolimits}
\newcommand{\cotan}{\mathop{\text{cotan}}\nolimits}
\newcommand{\Arcsin}{\mathop{\text{arcsin}}\nolimits}
\newcommand{\Arccos}{\mathop{\text{arccos}}\nolimits}
\newcommand{\Arctan}{\mathop{\text{arctan}}\nolimits}
\newcommand{\Argsh}{\mathop{\text{argsh}}\nolimits}
\newcommand{\Argch}{\mathop{\text{argch}}\nolimits}
\newcommand{\Argth}{\mathop{\text{argth}}\nolimits}
\newcommand{\pgcd}{\mathop{\text{pgcd}}\nolimits} 


%----- Commandes divers ------
\newcommand{\ii}{\mathrm{i}}
\newcommand{\dd}{\text{d}}
\newcommand{\id}{\mathop{\text{id}}\nolimits}
\newcommand{\Ker}{\mathop{\text{Ker}}\nolimits}
\newcommand{\Card}{\mathop{\text{Card}}\nolimits}
\newcommand{\Vect}{\mathop{\text{Vect}}\nolimits}
\newcommand{\Mat}{\mathop{\text{Mat}}\nolimits}
\newcommand{\rg}{\mathop{\text{rg}}\nolimits}
\newcommand{\tr}{\mathop{\text{tr}}\nolimits}


%----- Structure des exercices ------

\newtheoremstyle{styleexo}% name
{2ex}% Space above
{3ex}% Space below
{}% Body font
{}% Indent amount 1
{\bfseries} % Theorem head font
{}% Punctuation after theorem head
{\newline}% Space after theorem head 2
{}% Theorem head spec (can be left empty, meaning ‘normal’)

%\theoremstyle{styleexo}
\newtheorem{exo}{Exercice}
\newtheorem{ind}{Indications}
\newtheorem{cor}{Correction}


\newcommand{\exercice}[1]{} \newcommand{\finexercice}{}
%\newcommand{\exercice}[1]{{\tiny\texttt{#1}}\vspace{-2ex}} % pour afficher le numero absolu, l'auteur...
\newcommand{\enonce}{\begin{exo}} \newcommand{\finenonce}{\end{exo}}
\newcommand{\indication}{\begin{ind}} \newcommand{\finindication}{\end{ind}}
\newcommand{\correction}{\begin{cor}} \newcommand{\fincorrection}{\end{cor}}

\newcommand{\noindication}{\stepcounter{ind}}
\newcommand{\nocorrection}{\stepcounter{cor}}

\newcommand{\fiche}[1]{} \newcommand{\finfiche}{}
\newcommand{\titre}[1]{\centerline{\large \bf #1}}
\newcommand{\addcommand}[1]{}
\newcommand{\video}[1]{}

% Marge
\newcommand{\mymargin}[1]{\marginpar{{\small #1}}}

\def\noqed{\renewcommand{\qedsymbol}{}}


%----- Presentation ------
\setlength{\parindent}{0cm}

%\newcommand{\ExoSept}{\href{http://exo7.emath.fr}{\textbf{\textsf{Exo7}}}}

\definecolor{myred}{rgb}{0.93,0.26,0}
\definecolor{myorange}{rgb}{0.97,0.58,0}
\definecolor{myyellow}{rgb}{1,0.86,0}

\newcommand{\LogoExoSept}[1]{  % input : echelle
{\usefont{U}{cmss}{bx}{n}
\begin{tikzpicture}[scale=0.1*#1,transform shape]
  \fill[color=myorange] (0,0)--(4,0)--(4,-4)--(0,-4)--cycle;
  \fill[color=myred] (0,0)--(0,3)--(-3,3)--(-3,0)--cycle;
  \fill[color=myyellow] (4,0)--(7,4)--(3,7)--(0,3)--cycle;
  \node[scale=5] at (3.5,3.5) {Exo7};
\end{tikzpicture}}
}


\newcommand{\debutmontitre}{
  \author{} \date{} 
  \thispagestyle{empty}
  \hspace*{-10ex}
  \begin{minipage}{\textwidth}
    \titlepage  
  \vspace*{-2.5cm}
  \begin{center}
    \LogoExoSept{2.5}
  \end{center}
  \end{minipage}

  \vspace*{-0cm}
  
  % Astuce pour que le background ne soit pas discrétisé lors de la conversion pdf -> png
\begin{tikzpicture}
        \fill[opacity=0,green!60!black] (0,0)--++(0,0)--++(0,0)--++(0,0)--cycle; 
\end{tikzpicture}

% toc S'affiche trop tot :
% \tableofcontents[hideallsubsections, pausesections]
}

\newcommand{\finmontitre}{
  \end{frame}
  \setcounter{framenumber}{0}
} % ne marche pas pour une raison obscure

%----- Commandes supplementaires ------

% \usepackage[landscape]{geometry}
% \geometry{top=1cm, bottom=3cm, left=2cm, right=10cm, marginparsep=1cm
% }
% \usepackage[a4paper]{geometry}
% \geometry{top=2cm, bottom=2cm, left=2cm, right=2cm, marginparsep=1cm
% }

%\usepackage{standalone}


% New command Arnaud -- november 2011
\setbeamersize{text margin left=24ex}
% si vous modifier cette valeur il faut aussi
% modifier le decalage du titre pour compenser
% (ex : ici =+10ex, titre =-5ex

\theoremstyle{definition}
%\newtheorem{proposition}{Proposition}
%\newtheorem{exemple}{Exemple}
%\newtheorem{theoreme}{Théorème}
%\newtheorem{lemme}{Lemme}
%\newtheorem{corollaire}{Corollaire}
%\newtheorem*{remarque*}{Remarque}
%\newtheorem*{miniexercice}{Mini-exercices}
%\newtheorem{definition}{Définition}

% Commande tikz
\usetikzlibrary{calc}
\usetikzlibrary{patterns,arrows}
\usetikzlibrary{matrix}
\usetikzlibrary{fadings} 

%definition d'un terme
\newcommand{\defi}[1]{{\color{myorange}\textbf{\emph{#1}}}}
\newcommand{\evidence}[1]{{\color{blue}\textbf{\emph{#1}}}}
\newcommand{\assertion}[1]{\emph{\og#1\fg}}  % pour chapitre logique
%\renewcommand{\contentsname}{Sommaire}
\renewcommand{\contentsname}{}
\setcounter{tocdepth}{2}



%------ Figures ------

\def\myscale{1} % par défaut 
\newcommand{\myfigure}[2]{  % entrée : echelle, fichier figure
\def\myscale{#1}
\begin{center}
\footnotesize
{#2}
\end{center}}


%------ Encadrement ------

\usepackage{fancybox}


\newcommand{\mybox}[1]{
\setlength{\fboxsep}{7pt}
\begin{center}
\shadowbox{#1}
\end{center}}

\newcommand{\myboxinline}[1]{
\setlength{\fboxsep}{5pt}
\raisebox{-10pt}{
\shadowbox{#1}
}
}

%--------------- Commande beamer---------------
\newcommand{\beameronly}[1]{#1} % permet de mettre des pause dans beamer pas dans poly


\setbeamertemplate{navigation symbols}{}
\setbeamertemplate{footline}  % tiré du fichier beamerouterinfolines.sty
{
  \leavevmode%
  \hbox{%
  \begin{beamercolorbox}[wd=.333333\paperwidth,ht=2.25ex,dp=1ex,center]{author in head/foot}%
    % \usebeamerfont{author in head/foot}\insertshortauthor%~~(\insertshortinstitute)
    \usebeamerfont{section in head/foot}{\bf\insertshorttitle}
  \end{beamercolorbox}%
  \begin{beamercolorbox}[wd=.333333\paperwidth,ht=2.25ex,dp=1ex,center]{title in head/foot}%
    \usebeamerfont{section in head/foot}{\bf\insertsectionhead}
  \end{beamercolorbox}%
  \begin{beamercolorbox}[wd=.333333\paperwidth,ht=2.25ex,dp=1ex,right]{date in head/foot}%
    % \usebeamerfont{date in head/foot}\insertshortdate{}\hspace*{2em}
    \insertframenumber{} / \inserttotalframenumber\hspace*{2ex} 
  \end{beamercolorbox}}%
  \vskip0pt%
}


\definecolor{mygrey}{rgb}{0.5,0.5,0.5}
\setlength{\parindent}{0cm}
%\DeclareTextFontCommand{\helvetica}{\fontfamily{phv}\selectfont}

% background beamer
\definecolor{couleurhaut}{rgb}{0.85,0.9,1}  % creme
\definecolor{couleurmilieu}{rgb}{1,1,1}  % vert pale
\definecolor{couleurbas}{rgb}{0.85,0.9,1}  % blanc
\setbeamertemplate{background canvas}[vertical shading]%
[top=couleurhaut,middle=couleurmilieu,midpoint=0.4,bottom=couleurbas] 
%[top=fondtitre!05,bottom=fondtitre!60]



\makeatletter
\setbeamertemplate{theorem begin}
{%
  \begin{\inserttheoremblockenv}
  {%
    \inserttheoremheadfont
    \inserttheoremname
    \inserttheoremnumber
    \ifx\inserttheoremaddition\@empty\else\ (\inserttheoremaddition)\fi%
    \inserttheorempunctuation
  }%
}
\setbeamertemplate{theorem end}{\end{\inserttheoremblockenv}}

\newenvironment{theoreme}[1][]{%
   \setbeamercolor{block title}{fg=structure,bg=structure!40}
   \setbeamercolor{block body}{fg=black,bg=structure!10}
   \begin{block}{{\bf Th\'eor\`eme }#1}
}{%
   \end{block}%
}


\newenvironment{proposition}[1][]{%
   \setbeamercolor{block title}{fg=structure,bg=structure!40}
   \setbeamercolor{block body}{fg=black,bg=structure!10}
   \begin{block}{{\bf Proposition }#1}
}{%
   \end{block}%
}

\newenvironment{corollaire}[1][]{%
   \setbeamercolor{block title}{fg=structure,bg=structure!40}
   \setbeamercolor{block body}{fg=black,bg=structure!10}
   \begin{block}{{\bf Corollaire }#1}
}{%
   \end{block}%
}

\newenvironment{mydefinition}[1][]{%
   \setbeamercolor{block title}{fg=structure,bg=structure!40}
   \setbeamercolor{block body}{fg=black,bg=structure!10}
   \begin{block}{{\bf Définition} #1}
}{%
   \end{block}%
}

\newenvironment{lemme}[0]{%
   \setbeamercolor{block title}{fg=structure,bg=structure!40}
   \setbeamercolor{block body}{fg=black,bg=structure!10}
   \begin{block}{\bf Lemme}
}{%
   \end{block}%
}

\newenvironment{remarque}[1][]{%
   \setbeamercolor{block title}{fg=black,bg=structure!20}
   \setbeamercolor{block body}{fg=black,bg=structure!5}
   \begin{block}{Remarque #1}
}{%
   \end{block}%
}


\newenvironment{exemple}[1][]{%
   \setbeamercolor{block title}{fg=black,bg=structure!20}
   \setbeamercolor{block body}{fg=black,bg=structure!5}
   \begin{block}{{\bf Exemple }#1}
}{%
   \end{block}%
}


\newenvironment{miniexercice}[0]{%
   \setbeamercolor{block title}{fg=structure,bg=structure!20}
   \setbeamercolor{block body}{fg=black,bg=structure!5}
   \begin{block}{Mini-exercices}
}{%
   \end{block}%
}


\newenvironment{tp}[0]{%
   \setbeamercolor{block title}{fg=structure,bg=structure!40}
   \setbeamercolor{block body}{fg=black,bg=structure!10}
   \begin{block}{\bf Travaux pratiques}
}{%
   \end{block}%
}
\newenvironment{exercicecours}[1][]{%
   \setbeamercolor{block title}{fg=structure,bg=structure!40}
   \setbeamercolor{block body}{fg=black,bg=structure!10}
   \begin{block}{{\bf Exercice }#1}
}{%
   \end{block}%
}
\newenvironment{algo}[1][]{%
   \setbeamercolor{block title}{fg=structure,bg=structure!40}
   \setbeamercolor{block body}{fg=black,bg=structure!10}
   \begin{block}{{\bf Algorithme}\hfill{\color{gray}\texttt{#1}}}
}{%
   \end{block}%
}


\setbeamertemplate{proof begin}{
   \setbeamercolor{block title}{fg=black,bg=structure!20}
   \setbeamercolor{block body}{fg=black,bg=structure!5}
   \begin{block}{{\footnotesize Démonstration}}
   \footnotesize
   \smallskip}
\setbeamertemplate{proof end}{%
   \end{block}}
\setbeamertemplate{qed symbol}{\openbox}


\makeatother
\usecolortheme[RGB={51,102,51}]{structure}

%%%%%%%%%%%%%%%%%%%%%%%%%%%%%%%%%%%%%%%%%%%%%%%%%%%%%%%%%%%%%
%%%%%%%%%%%%%%%%%%%%%%%%%%%%%%%%%%%%%%%%%%%%%%%%%%%%%%%%%%%%%


\begin{document}


\title{{\bf \'Equations différentielles}}
\subtitle{\'Equation différentielle linéaire du second ordre\\ à coefficients constants}

\begin{frame}
  
  \debutmontitre

  \pause

{\footnotesize
\hfill
\setbeamercovered{transparent=50}
\begin{minipage}{0.6\textwidth}
  \begin{itemize}
    \item<3-> Définition
    \item<4-> \'Equation homogène
    \item<5-> \'Equation avec second membre
    \item<6-> Recherche d'une solution particulière
  \end{itemize}
\end{minipage}
}

\end{frame}

\setcounter{framenumber}{0}

%%%%%%%%%%%%%%%%%%%%%%%%%%%%%%%%%%%%%%%%%%%%%%%%%%%%%%%%%%%%%%%%
\section*{Définition}

\begin{frame}
\begin{itemize}
  \item Une \defi{équation différentielle linéaire du second ordre, à
 coefficients constants}, est une équation de la forme :
\begin{equation}
ay''+by'+cy=g(x) 
\label{eq:linscd}
\tag{$E$}
\end{equation}
où $a,b,c \in \Rr$, $a \neq 0$ et $g$ est une fonction continue sur
un intervalle ouvert $I$  

\pause

  \item L'équation différentielle : 
\begin{equation}
ay''+by'+cy=0 
\label{eq:linscdhom}
\tag{$E_0$}
\end{equation}
est appelée \defi{l'équation homogène} associée à $(E)$
  
\end{itemize}


\pause

\begin{theoreme}
%\label{th:eqdiffdim}
L'ensemble des solutions de l'équation homogène 
(\ref{eq:linscdhom}) est un $\Rr$-espace vectoriel de dimension~$2$
\end{theoreme}

\end{frame}



%%%%%%%%%%%%%%%%%%%%%%%%%%%%%%%%%%%%%%%%%%%%%%%%%%%%%%%%%%%%%%%%
\section*{\'Equation homogène}

\begin{frame}
\begin{itemize}

  \item $ay''+by'+cy=0$ \quad (\ref{eq:linscdhom})
  \pause
  \item L'équation $ar^2+br+c=0$ est \defi{l'équation caractéristique}
\pause
  \item Soit $\Delta= b^2-4ac$, le discriminant
\end{itemize}
\vspace*{-1ex}\pause
{%\small
\begin{theoreme}
\begin{enumerate}
\item Si $\Delta >0$, l'équation caractéristique possède deux racines réelles distinctes 
$r_1\neq r_2$
\pause et les solutions de (\ref{eq:linscdhom}) sont les
\vspace*{-0.8ex}
\mybox{$y(x) = \lambda e^{r_1x}+ \mu e^{r_2x} \quad \text{ où } \lambda, \mu \in \Rr$}
\vspace*{-1.5ex}\pause
\item Si $\Delta=0$, l'équation caractéristique possède une racine double $r_0$ 
\pause et les solutions de (\ref{eq:linscdhom}) sont les
\vspace*{-0.8ex}
\mybox{$y(x) = (\lambda+\mu x)e^{r_0 x} \quad \text{ où } \lambda, \mu \in \Rr$}
\vspace*{-1.5ex}\pause
\item Si $\Delta<0$, l'équation caractéristique possède deux racines complexes 
$r_1=\alpha+\ii \beta$, $r_2=\alpha-\ii \beta$ 
\pause et les solutions de (\ref{eq:linscdhom}) sont 
\vspace*{-0.8ex}
\mybox{$y(x) = e^{\alpha x}\big(\lambda\cos (\beta x)+\mu\sin (\beta x)\big) \quad \text{ où }
\lambda, \mu \in \Rr$}

\end{enumerate}
\end{theoreme}
}
\end{frame}


\begin{frame}
\begin{exemple}
\begin{enumerate}
  \item Résoudre $y'' - y' - 2y = 0$
  \pause
  \begin{itemize}
    \item équation caractéristique : $r^2 - r - 2 = 0$   
    \pause
    \item $\Delta >0$, $r_1 = -1$, $r_2 = 2$
    \pause
    \item $y(x) = \lambda e^{-x} + \mu e^{2x}$, $\lambda,\mu \in \Rr$
  \end{itemize}

  \pause\bigskip
  
  \item Résoudre $y'' - 4y' + 4y = 0$
  \pause
  \begin{itemize}
    \item $r^2 - 4r + 4 = 0$
    \pause
    \item $\Delta=0$, $r_0 = 2$
    \pause
    \item $y(x) = (\lambda x + \mu) e^{2x}$, $\lambda,\mu \in \Rr$.
  \end{itemize}  
  
  \pause\bigskip
  
  \item Résoudre $y'' - 2y' + 5y = 0$
  \pause
   \begin{itemize}
    \item $r^2-2r+5 = 0$
    \pause
    \item $\Delta<0$, $r_1 = 1 + 2\ii$, $r_2 = 1 - 2\ii$
    \pause
    \item $y(x) = e^x (\lambda \cos(2x) + \mu \sin(2x))$,  $\lambda, \mu \in \Rr$
  \end{itemize} 
\end{enumerate}
\end{exemple}
\end{frame}


\begin{frame}
\begin{itemize}
  \item $ay''+by'+cy=0$ \quad (\ref{eq:linscdhom})
  \pause
  \item On cherche une solution de (\ref{eq:linscdhom}) sous la forme 
$y(x)=e^{rx}$
\pause
  \item $r \in \Cc$ est une constante à déterminer
  \pause
  \item $\begin{array}{rl}
     & ay''+by'+cy=0 \\
     \pause
\iff & (ar^2+br+c)e^{rx}=0 \\
\pause
\iff & ar^2+br+c=0
\end{array}$
\end{itemize}
\pause
\begin{proof}
\pause
\begin{enumerate}
  \item  $\Delta>0$
  \pause
  \begin{itemize}
    \item deux racines réelles distinctes $r_1, r_2$
    \pause
    \item deux solutions de (\ref{eq:linscdhom}) : $y_1=e^{r_1x}, y_2=e^{r_2x}$
    \pause
    \item $y_1,y_2$ sont linéairement indépendantes car $r_1 \neq r_2$
    \pause
    \item solutions forment un espace vectoriel de dimension $2$
    \pause
    \item $(y_1,y_2)$ est donc une base de solutions
    \pause
    \item solution générale : $y(x) = \lambda e^{r_1x} + \mu e^{r_2x},$ où $\lambda,
\mu\in\Rr$
  \end{itemize}

\end{enumerate}
\end{proof}
\end{frame}


%%%%%%%%%%%%%%%%%%%%%%%%%%%%%%%%%%%%%%%%%%%%%%%%%%%%%%%%%%%%%%%%
\section*{\'Equation avec second membre}

\begin{frame}

\begin{equation}
ay''+by'+cy=g(x) 
%\label{eq:linscd}
\tag{$E$}
\end{equation}

\pause

\begin{theoreme}[de Cauchy-Lipschitz]
Pour chaque $x_0\in I$ et chaque couple $(y_0,y_1) \in \Rr^2$,  
l'équation (\ref{eq:linscd}) admet une \evidence{unique}
solution $y(x)$ sur $I$ satisfaisant aux conditions initiales : 
\mybox{$y(x_0) = y_0$ \quad et \quad $y'(x_0) = y_1$}
\end{theoreme}

\pause

\begin{proposition}
Les solutions générales de l'équation (\ref{eq:linscd}) s'obtiennent en
ajoutant les solutions générales de l'équation homogène (\ref{eq:linscdhom}) 
à une solution particulière de (\ref{eq:linscd})
\end{proposition}

\end{frame}


%%%%%%%%%%%%%%%%%%%%%%%%%%%%%%%%%%%%%%%%%%%%%%%%%%%%%%%%%%%%%%%%
\section*{Recherche d'une solution particulière}

\begin{frame}

\evidence{Solution particulière} $ay''+by'+cy=g(x)$ \quad (\ref{eq:linscd})

\pause
\medskip

\evidence{Second membre du type $e^{\alpha x}P(x)$}

\pause
On cherche une solution particulière sous la forme 
$y_0(x)=e^{\alpha x}x^{m}Q(x)$, où $Q$ est un polynôme de 
même degré que $P$ avec :
\pause
\begin{itemize}
\item $y_0(x)=e^{\alpha x}Q(x)$, si $\alpha$ n'est pas une racine de l'équation caractéristique
\pause
\item $y_0(x)=xe^{\alpha x}Q(x)$, si $\alpha$ est une racine simple de l'équation caractéristique
\pause
\item $y_0(x)=x^2e^{\alpha x}Q(x)$, si $\alpha$ est une racine double de l'équation caractéristique
\end{itemize}

\pause
\medskip

\evidence{Second membre du type $e^{\alpha x}\big(P_1(x)\cos (\beta x)+P_2(x)\sin (\beta x)\big)$}

\pause
On cherche une solution particulière sous la forme :
\begin{itemize}
\item $y_0(x)=e^{\alpha x} \big( Q_1(x)\cos (\beta x)+Q_2(x)\sin (\beta x) \big)$, 
si $\alpha +\ii \beta$ n'est pas une racine de l'équation caractéristique
\pause
\item $y_0(x)=xe^{\alpha x}  \big( Q_1(x)\cos (\beta x)+Q_2(x)\sin (\beta x) \big)$, 
si $\alpha +\ii \beta$ est une racine de l'équation caractéristique
\end{itemize}
\pause
où $Q_1$ et $Q_2$ sont deux polynômes de degré $n=\max\{\deg P_1,\deg P_2\}$

\end{frame}

\begin{frame}
\begin{exemple}
\begin{enumerate}
  \item Résoudre $(E_0) \quad y''-5y'+6y=0$
  \pause
  \begin{itemize}
    \item $r^2-5r+6=0$
    \pause
    \item $r_1=2, r_2=3$
    \pause
    \item $\big\{\lambda e^{2x}+ \mu e^{3x} \mid \lambda, \mu \in \Rr\big\}$
  \end{itemize}

  \pause
  \item Résoudre $(E) \quad y''-5y'+6y=4xe^x$
  \pause
  \begin{itemize}
    \item solution particulière sous la forme $y_0(x)=(ax+b)e^x$
    \pause
    
    \item ~ \hspace*{-2em} 
    \begin{minipage}{0.9\textwidth}%\vspace*{-2ex} 
    $\begin{array}{rl}
& (ax+2a+b)e^x-5(ax+a+b)e^x+6(ax+b)e^x=4xe^x \\ 
\pause
\iff & (a-5a+6a)x+2a+b-5(a+b)+6b=4x \\
\pause
\iff & 2a=4 \text{ et } -3a+2b=0 \\
\pause
\iff & a=2 \text{ et } b=3
\end{array}$  
    \end{minipage}

    
    \pause
    \item donc $y_0(x)=(2x+3)e^x$
    \pause
    \item solutions de $(E)$ : $\big\{(2x+3)e^x+\lambda e^{2x} + \mu e^{3x} \mid \lambda,\mu\in \Rr\big\}$ 

  \end{itemize}

  \pause
  \item Trouver la solution de $(E)$ vérifiant $y(0)=1$ et $y'(0)=0$
  \pause
  \begin{itemize}
    \item $y(0)=1 \iff 3+\lambda+\mu=1$
    \pause
    \item $y'(0)=0 \iff 5+2\lambda+3\mu=0$
    \pause
    \item $\lambda=-1, \mu=-1$
    \pause
    \item $y(x)=(2x+3)e^{x}-e^{2x}-e^{3x}$
  \end{itemize}  
  

\end{enumerate}  
\end{exemple}
\end{frame}

\begin{frame}
\evidence{Méthode de variation des constantes} \quad $ay''+by'+cy=g(x)$


\pause
\begin{itemize}
  \item Soit $\{y_1,y_2\}$ est une base de solution de l'équation homogène (\ref{eq:linscdhom})
 \pause 
  \item On cherche une solution particulière sous la forme 
$y_0= \lambda y_1 + \mu y_2$, où $\lambda$ et $\mu$ sont deux fonctions 
vérifiant :
\mybox{
($S$) \qquad $
\left\{\begin{array}{ccl}  
\lambda'y_1+\mu'y_2&=&0\\ 
\lambda'y'_1+\mu'y'_2&=& \frac{g(x)}{a}
\end{array}\right. 
$
}
\pause  
  \item Pourquoi cela ?
Si $y_0= \lambda y_1 + \mu y_2$ est une telle fonction alors
$$y_0'= \lambda' y_1 + \mu' y_2 + \lambda y_1' + \mu y_2' = \lambda y_1'+ \mu y_2'$$
\vspace*{-2ex}
$$y_0'' = \lambda' y_1'+ \mu' y_2' + \lambda y_1''+ \mu y_2'' = \frac{g(x)}{a} + \lambda y_1''+ \mu y_2''$$
\pause\vspace*{-2ex}
  \item Ainsi l'équation (\ref{eq:linscd}) est vérifiée par $y_0$ :
$ay_0''+by_0'+cy_0= a\big(\frac{g(x)}{a} + \lambda y_1''+ \mu y_2''\big) + b\big(\lambda y_1'+ \mu y_2'\big) + c\big(\lambda y_1 + \mu y_2\big)
= g(x) + \lambda\big(ay_1''+by_1'+cy_1\big) + \mu\big(ay_2''+by_2'+cy_2\big)= g(x)$
\pause
  \item On résout le système ($S$), 
  ce qui donne $\lambda'$ et $\mu'$, 
puis $\lambda$ et $\mu$
\end{itemize}







\end{frame}

\begin{frame}
\begin{exemple}
Résoudre $y'' + y = \frac{1}{\cos x}$ sur $]-\frac\pi2,+\frac\pi2[$ 
\pause

\begin{itemize}
  \item Les solutions de l'équation homogène $y'' + y =0$ sont
$\lambda \cos x  + \mu \sin x$ où $\lambda,\mu\in\Rr$
\pause  
  \item Solution particulière de (\ref{eq:linscd}) sous la forme $y_0(x) =\lambda(x) \cos x  + \mu(x) \sin x$
\pause  
  \item $\lambda(x),\mu(x)$ sont des fonctions à trouver et qui vérifient ($S$) :
  \vspace*{-1.5ex}  
\pause$$
\left\{\begin{array}{ccl}  
\lambda'y_1+\mu'y_2&=&0\\ 
\lambda'y'_1+\mu'y'_2&=& \frac{g(x)}{a}
\end{array}\right. 
\pause  \ \text{ donc } \ 
\left\{\begin{array}{ccl}  
\lambda' \cos x + \mu' \sin x &=&0\\ 
-\lambda' \sin x + \mu' \cos x &=& \frac{1}{\cos x}
\end{array}\right. 
$$

\pause\vspace*{-3ex}

  
  \item  ~ \vspace*{-4ex}
$$
\left\{\begin{array}{ccl}  
\lambda' \cos x \sin x + \mu' (\sin x)^2 &=&0\\ 
-\lambda' \cos x \sin x + \mu' (\cos x)^2 &=& 1
\end{array}\right.  
\pause  \  \text{ donc par somme } \ 
\mu'=1
$$

 \pause  \vspace*{-3ex}
 
  \item Ainsi $\mu(x) = x$ et $\lambda' = -\frac{\sin x}{\cos x}$ donc $\lambda(x) = \ln(\cos x)$
 
 \pause  
  \item $y_0(x) = \ln(\cos x) \cos x + x\sin x$ est une solution particulière  
  
  \pause  
  \item Solutions :
$\lambda \cos x  + \mu \sin x + \ln(\cos x) \cos x + x\sin x$
($\forall \lambda,\mu\in \Rr$)
\end{itemize}
 
\end{exemple}

\end{frame}




%%%%%%%%%%%%%%%%%%%%%%%%%%%%%%%%%%%%%%%%%%%%%%%%%%%%%%%%%%%%%%%
 \section*{Mini-exercices}

\begin{frame}
\begin{miniexercice}
\begin{enumerate}
  \item Résoudre l'équation différentielle $y'' + \omega^2y=0$.
  Trouver la solution vérifiant $y(0)=1$ et $y'(0)=1$. 
  Tracer la courbe intégrale.
  Résoudre l'équation différentielle $y'' + \omega^2y= \sin(\omega x)$.
  
  \item Résoudre l'équation différentielle $y'' + y' -6y=0$.
  Trouver la solution vérifiant $y(-1)=1$ et $y'(-1)=0$. 
  Tracer la courbe intégrale.
  Résoudre l'équation différentielle $y'' + y' -6y= e^x$.
  
  \item Résoudre l'équation différentielle $2y'' -2 y' +\frac12 y=0$.
  Trouver la solution ayant une limite finie lorsque $x\to +\infty$.
  Résoudre $2y'' -2 y' +\frac12 y=x-1$.
\end{enumerate}
\end{miniexercice}
\end{frame}




\end{document}
