
%%%%%%%%%%%%%%%%%% PREAMBULE %%%%%%%%%%%%%%%%%%

\documentclass[aspectratio=169,utf8]{beamer}
%\documentclass[aspectratio=169,handout]{beamer}

\usetheme{Boadilla}
%\usecolortheme{seahorse}
\usecolortheme[RGB={245,66,24}]{structure}
\useoutertheme{infolines}

% packages
\usepackage{amsfonts,amsmath,amssymb,amsthm}
\usepackage[utf8]{inputenc}
\usepackage[T1]{fontenc}
\usepackage{lmodern}

\usepackage[francais]{babel}
\usepackage{fancybox}
\usepackage{graphicx}

\usepackage{float}
\usepackage{xfrac}

%\usepackage[usenames, x11names]{xcolor}
\usepackage{tikz}
\usepackage{pgfplots}
\usepackage{datetime}



%-----  Package unités -----
\usepackage{siunitx}
\sisetup{locale = FR,detect-all,per-mode = symbol}

%\usepackage{mathptmx}
%\usepackage{fouriernc}
%\usepackage{newcent}
%\usepackage[mathcal,mathbf]{euler}

%\usepackage{palatino}
%\usepackage{newcent}
% \usepackage[mathcal,mathbf]{euler}



% \usepackage{hyperref}
% \hypersetup{colorlinks=true, linkcolor=blue, urlcolor=blue,
% pdftitle={Exo7 - Exercices de mathématiques}, pdfauthor={Exo7}}


%section
% \usepackage{sectsty}
% \allsectionsfont{\bf}
%\sectionfont{\color{Tomato3}\upshape\selectfont}
%\subsectionfont{\color{Tomato4}\upshape\selectfont}

%----- Ensembles : entiers, reels, complexes -----
\newcommand{\Nn}{\mathbb{N}} \newcommand{\N}{\mathbb{N}}
\newcommand{\Zz}{\mathbb{Z}} \newcommand{\Z}{\mathbb{Z}}
\newcommand{\Qq}{\mathbb{Q}} \newcommand{\Q}{\mathbb{Q}}
\newcommand{\Rr}{\mathbb{R}} \newcommand{\R}{\mathbb{R}}
\newcommand{\Cc}{\mathbb{C}} 
\newcommand{\Kk}{\mathbb{K}} \newcommand{\K}{\mathbb{K}}

%----- Modifications de symboles -----
\renewcommand{\epsilon}{\varepsilon}
\renewcommand{\Re}{\mathop{\text{Re}}\nolimits}
\renewcommand{\Im}{\mathop{\text{Im}}\nolimits}
%\newcommand{\llbracket}{\left[\kern-0.15em\left[}
%\newcommand{\rrbracket}{\right]\kern-0.15em\right]}

\renewcommand{\ge}{\geqslant}
\renewcommand{\geq}{\geqslant}
\renewcommand{\le}{\leqslant}
\renewcommand{\leq}{\leqslant}
\renewcommand{\epsilon}{\varepsilon}

%----- Fonctions usuelles -----
\newcommand{\ch}{\mathop{\text{ch}}\nolimits}
\newcommand{\sh}{\mathop{\text{sh}}\nolimits}
\renewcommand{\tanh}{\mathop{\text{th}}\nolimits}
\newcommand{\cotan}{\mathop{\text{cotan}}\nolimits}
\newcommand{\Arcsin}{\mathop{\text{arcsin}}\nolimits}
\newcommand{\Arccos}{\mathop{\text{arccos}}\nolimits}
\newcommand{\Arctan}{\mathop{\text{arctan}}\nolimits}
\newcommand{\Argsh}{\mathop{\text{argsh}}\nolimits}
\newcommand{\Argch}{\mathop{\text{argch}}\nolimits}
\newcommand{\Argth}{\mathop{\text{argth}}\nolimits}
\newcommand{\pgcd}{\mathop{\text{pgcd}}\nolimits} 


%----- Commandes divers ------
\newcommand{\ii}{\mathrm{i}}
\newcommand{\dd}{\text{d}}
\newcommand{\id}{\mathop{\text{id}}\nolimits}
\newcommand{\Ker}{\mathop{\text{Ker}}\nolimits}
\newcommand{\Card}{\mathop{\text{Card}}\nolimits}
\newcommand{\Vect}{\mathop{\text{Vect}}\nolimits}
\newcommand{\Mat}{\mathop{\text{Mat}}\nolimits}
\newcommand{\rg}{\mathop{\text{rg}}\nolimits}
\newcommand{\tr}{\mathop{\text{tr}}\nolimits}


%----- Structure des exercices ------

\newtheoremstyle{styleexo}% name
{2ex}% Space above
{3ex}% Space below
{}% Body font
{}% Indent amount 1
{\bfseries} % Theorem head font
{}% Punctuation after theorem head
{\newline}% Space after theorem head 2
{}% Theorem head spec (can be left empty, meaning ‘normal’)

%\theoremstyle{styleexo}
\newtheorem{exo}{Exercice}
\newtheorem{ind}{Indications}
\newtheorem{cor}{Correction}


\newcommand{\exercice}[1]{} \newcommand{\finexercice}{}
%\newcommand{\exercice}[1]{{\tiny\texttt{#1}}\vspace{-2ex}} % pour afficher le numero absolu, l'auteur...
\newcommand{\enonce}{\begin{exo}} \newcommand{\finenonce}{\end{exo}}
\newcommand{\indication}{\begin{ind}} \newcommand{\finindication}{\end{ind}}
\newcommand{\correction}{\begin{cor}} \newcommand{\fincorrection}{\end{cor}}

\newcommand{\noindication}{\stepcounter{ind}}
\newcommand{\nocorrection}{\stepcounter{cor}}

\newcommand{\fiche}[1]{} \newcommand{\finfiche}{}
\newcommand{\titre}[1]{\centerline{\large \bf #1}}
\newcommand{\addcommand}[1]{}
\newcommand{\video}[1]{}

% Marge
\newcommand{\mymargin}[1]{\marginpar{{\small #1}}}

\def\noqed{\renewcommand{\qedsymbol}{}}


%----- Presentation ------
\setlength{\parindent}{0cm}

%\newcommand{\ExoSept}{\href{http://exo7.emath.fr}{\textbf{\textsf{Exo7}}}}

\definecolor{myred}{rgb}{0.93,0.26,0}
\definecolor{myorange}{rgb}{0.97,0.58,0}
\definecolor{myyellow}{rgb}{1,0.86,0}

\newcommand{\LogoExoSept}[1]{  % input : echelle
{\usefont{U}{cmss}{bx}{n}
\begin{tikzpicture}[scale=0.1*#1,transform shape]
  \fill[color=myorange] (0,0)--(4,0)--(4,-4)--(0,-4)--cycle;
  \fill[color=myred] (0,0)--(0,3)--(-3,3)--(-3,0)--cycle;
  \fill[color=myyellow] (4,0)--(7,4)--(3,7)--(0,3)--cycle;
  \node[scale=5] at (3.5,3.5) {Exo7};
\end{tikzpicture}}
}


\newcommand{\debutmontitre}{
  \author{} \date{} 
  \thispagestyle{empty}
  \hspace*{-10ex}
  \begin{minipage}{\textwidth}
    \titlepage  
  \vspace*{-2.5cm}
  \begin{center}
    \LogoExoSept{2.5}
  \end{center}
  \end{minipage}

  \vspace*{-0cm}
  
  % Astuce pour que le background ne soit pas discrétisé lors de la conversion pdf -> png
\begin{tikzpicture}
        \fill[opacity=0,green!60!black] (0,0)--++(0,0)--++(0,0)--++(0,0)--cycle; 
\end{tikzpicture}

% toc S'affiche trop tot :
% \tableofcontents[hideallsubsections, pausesections]
}

\newcommand{\finmontitre}{
  \end{frame}
  \setcounter{framenumber}{0}
} % ne marche pas pour une raison obscure

%----- Commandes supplementaires ------

% \usepackage[landscape]{geometry}
% \geometry{top=1cm, bottom=3cm, left=2cm, right=10cm, marginparsep=1cm
% }
% \usepackage[a4paper]{geometry}
% \geometry{top=2cm, bottom=2cm, left=2cm, right=2cm, marginparsep=1cm
% }

%\usepackage{standalone}


% New command Arnaud -- november 2011
\setbeamersize{text margin left=24ex}
% si vous modifier cette valeur il faut aussi
% modifier le decalage du titre pour compenser
% (ex : ici =+10ex, titre =-5ex

\theoremstyle{definition}
%\newtheorem{proposition}{Proposition}
%\newtheorem{exemple}{Exemple}
%\newtheorem{theoreme}{Théorème}
%\newtheorem{lemme}{Lemme}
%\newtheorem{corollaire}{Corollaire}
%\newtheorem*{remarque*}{Remarque}
%\newtheorem*{miniexercice}{Mini-exercices}
%\newtheorem{definition}{Définition}

% Commande tikz
\usetikzlibrary{calc}
\usetikzlibrary{patterns,arrows}
\usetikzlibrary{matrix}
\usetikzlibrary{fadings} 

%definition d'un terme
\newcommand{\defi}[1]{{\color{myorange}\textbf{\emph{#1}}}}
\newcommand{\evidence}[1]{{\color{blue}\textbf{\emph{#1}}}}
\newcommand{\assertion}[1]{\emph{\og#1\fg}}  % pour chapitre logique
%\renewcommand{\contentsname}{Sommaire}
\renewcommand{\contentsname}{}
\setcounter{tocdepth}{2}



%------ Figures ------

\def\myscale{1} % par défaut 
\newcommand{\myfigure}[2]{  % entrée : echelle, fichier figure
\def\myscale{#1}
\begin{center}
\footnotesize
{#2}
\end{center}}


%------ Encadrement ------

\usepackage{fancybox}


\newcommand{\mybox}[1]{
\setlength{\fboxsep}{7pt}
\begin{center}
\shadowbox{#1}
\end{center}}

\newcommand{\myboxinline}[1]{
\setlength{\fboxsep}{5pt}
\raisebox{-10pt}{
\shadowbox{#1}
}
}

%--------------- Commande beamer---------------
\newcommand{\beameronly}[1]{#1} % permet de mettre des pause dans beamer pas dans poly


\setbeamertemplate{navigation symbols}{}
\setbeamertemplate{footline}  % tiré du fichier beamerouterinfolines.sty
{
  \leavevmode%
  \hbox{%
  \begin{beamercolorbox}[wd=.333333\paperwidth,ht=2.25ex,dp=1ex,center]{author in head/foot}%
    % \usebeamerfont{author in head/foot}\insertshortauthor%~~(\insertshortinstitute)
    \usebeamerfont{section in head/foot}{\bf\insertshorttitle}
  \end{beamercolorbox}%
  \begin{beamercolorbox}[wd=.333333\paperwidth,ht=2.25ex,dp=1ex,center]{title in head/foot}%
    \usebeamerfont{section in head/foot}{\bf\insertsectionhead}
  \end{beamercolorbox}%
  \begin{beamercolorbox}[wd=.333333\paperwidth,ht=2.25ex,dp=1ex,right]{date in head/foot}%
    % \usebeamerfont{date in head/foot}\insertshortdate{}\hspace*{2em}
    \insertframenumber{} / \inserttotalframenumber\hspace*{2ex} 
  \end{beamercolorbox}}%
  \vskip0pt%
}


\definecolor{mygrey}{rgb}{0.5,0.5,0.5}
\setlength{\parindent}{0cm}
%\DeclareTextFontCommand{\helvetica}{\fontfamily{phv}\selectfont}

% background beamer
\definecolor{couleurhaut}{rgb}{0.85,0.9,1}  % creme
\definecolor{couleurmilieu}{rgb}{1,1,1}  % vert pale
\definecolor{couleurbas}{rgb}{0.85,0.9,1}  % blanc
\setbeamertemplate{background canvas}[vertical shading]%
[top=couleurhaut,middle=couleurmilieu,midpoint=0.4,bottom=couleurbas] 
%[top=fondtitre!05,bottom=fondtitre!60]



\makeatletter
\setbeamertemplate{theorem begin}
{%
  \begin{\inserttheoremblockenv}
  {%
    \inserttheoremheadfont
    \inserttheoremname
    \inserttheoremnumber
    \ifx\inserttheoremaddition\@empty\else\ (\inserttheoremaddition)\fi%
    \inserttheorempunctuation
  }%
}
\setbeamertemplate{theorem end}{\end{\inserttheoremblockenv}}

\newenvironment{theoreme}[1][]{%
   \setbeamercolor{block title}{fg=structure,bg=structure!40}
   \setbeamercolor{block body}{fg=black,bg=structure!10}
   \begin{block}{{\bf Th\'eor\`eme }#1}
}{%
   \end{block}%
}


\newenvironment{proposition}[1][]{%
   \setbeamercolor{block title}{fg=structure,bg=structure!40}
   \setbeamercolor{block body}{fg=black,bg=structure!10}
   \begin{block}{{\bf Proposition }#1}
}{%
   \end{block}%
}

\newenvironment{corollaire}[1][]{%
   \setbeamercolor{block title}{fg=structure,bg=structure!40}
   \setbeamercolor{block body}{fg=black,bg=structure!10}
   \begin{block}{{\bf Corollaire }#1}
}{%
   \end{block}%
}

\newenvironment{mydefinition}[1][]{%
   \setbeamercolor{block title}{fg=structure,bg=structure!40}
   \setbeamercolor{block body}{fg=black,bg=structure!10}
   \begin{block}{{\bf Définition} #1}
}{%
   \end{block}%
}

\newenvironment{lemme}[0]{%
   \setbeamercolor{block title}{fg=structure,bg=structure!40}
   \setbeamercolor{block body}{fg=black,bg=structure!10}
   \begin{block}{\bf Lemme}
}{%
   \end{block}%
}

\newenvironment{remarque}[1][]{%
   \setbeamercolor{block title}{fg=black,bg=structure!20}
   \setbeamercolor{block body}{fg=black,bg=structure!5}
   \begin{block}{Remarque #1}
}{%
   \end{block}%
}


\newenvironment{exemple}[1][]{%
   \setbeamercolor{block title}{fg=black,bg=structure!20}
   \setbeamercolor{block body}{fg=black,bg=structure!5}
   \begin{block}{{\bf Exemple }#1}
}{%
   \end{block}%
}


\newenvironment{miniexercice}[0]{%
   \setbeamercolor{block title}{fg=structure,bg=structure!20}
   \setbeamercolor{block body}{fg=black,bg=structure!5}
   \begin{block}{Mini-exercices}
}{%
   \end{block}%
}


\newenvironment{tp}[0]{%
   \setbeamercolor{block title}{fg=structure,bg=structure!40}
   \setbeamercolor{block body}{fg=black,bg=structure!10}
   \begin{block}{\bf Travaux pratiques}
}{%
   \end{block}%
}
\newenvironment{exercicecours}[1][]{%
   \setbeamercolor{block title}{fg=structure,bg=structure!40}
   \setbeamercolor{block body}{fg=black,bg=structure!10}
   \begin{block}{{\bf Exercice }#1}
}{%
   \end{block}%
}
\newenvironment{algo}[1][]{%
   \setbeamercolor{block title}{fg=structure,bg=structure!40}
   \setbeamercolor{block body}{fg=black,bg=structure!10}
   \begin{block}{{\bf Algorithme}\hfill{\color{gray}\texttt{#1}}}
}{%
   \end{block}%
}


\setbeamertemplate{proof begin}{
   \setbeamercolor{block title}{fg=black,bg=structure!20}
   \setbeamercolor{block body}{fg=black,bg=structure!5}
   \begin{block}{{\footnotesize Démonstration}}
   \footnotesize
   \smallskip}
\setbeamertemplate{proof end}{%
   \end{block}}
\setbeamertemplate{qed symbol}{\openbox}


\makeatother
\usecolortheme[RGB={142,35,35}]{structure}

%%%%%%%%%%%%%%%%%%%%%%%%%%%%%%%%%%%%%%%%%%%%%%%%%%%%%%%%%%%%%
%%%%%%%%%%%%%%%%%%%%%%%%%%%%%%%%%%%%%%%%%%%%%%%%%%%%%%%%%%%%%

\begin{document}


\title{{\bf Polynômes}}
\subtitle{Définitions}

\begin{frame}
  
  \debutmontitre

  \pause

{\footnotesize
\hfill
\setbeamercovered{transparent=50}
\begin{minipage}{0.6\textwidth}
  \begin{itemize}
    \item<3-> Définitions
    \item<4-> Opérations sur les polynômes
    \item<5-> Vocabulaire
  \end{itemize}
\end{minipage}
}

\end{frame}

\setcounter{framenumber}{0}


%%%%%%%%%%%%%%%%%%%%%%%%%%%%%%%%%%%%%%%%%%%%%%%%%%%%%%%%%%%%%%%%


\section*{Motivation}


\begin{frame}

\begin{itemize}
  \item $aX^2+bX+c=0$
\pause
  \item $aX^3+bX^2+cX+d=0$
\pause
  \begin{itemize}
    \item Tartaglia, \textsc{\romannumeral 16}\textsuperscript{e} siècle
\pause
    \item \og Méthode de Cardan\fg
  \end{itemize}
\end{itemize}

\bigskip 
\pause

\begin{itemize}
  \item Concepts de base

  \item Arithmétique des polynômes

  \item \og Tout polynôme de degré $n$ admet $n$ racines complexes \fg

  \item Fractions rationnelles
\end{itemize}

\end{frame}


%---------------------------------------------------------------
\section{Définitions}

\begin{frame}
\centerline{}


Un \defi{polynôme} à coefficients dans $\Kk$ 
est une expression de la forme
$$P(X) = a_n X^n + a_{n-1} X^{n-1} + \cdots + a_2 X^2 + a_1 X + a_0$$
avec $n\in \Nn$ et $a_0,a_1,\ldots,a_n \in \Kk$

\pause

\begin{itemize}[<+->]
  \item Les $a_i$ sont appelés les \defi{coefficients} du polynôme
  
  \item $\Kk$ désigne le corps $\Qq$, $\Rr$ ou $\Cc$

  \item Si tous les coefficients $a_i$ sont nuls, $P$ est le \defi{polynôme
  nul}, il est noté $0$ 
  
  \item L'ensemble des polynômes est noté $\Kk[X]$
  
\end{itemize} 
  
\end{frame}


\begin{frame}
$$P(X)= \sum_i a_i X^i$$
\begin{itemize}
  \item  Le \defi{degré} de $P$ est le plus grand entier $i$ tel que $a_i\neq0$ ;
on le note $\deg P$

\pause

  \item Un polynôme de la forme $P=a_0$ avec $a_0\in\Kk$ est appelé un 
  \defi{polynôme constant}. Si $a_0 \neq 0$, son degré est $0$

\pause

  \item Le degré du polynôme nul est par convention $-\infty$
\end{itemize}

\pause

\begin{exemple}
\begin{itemize}
  \item $X^3-5X+\frac 34$ est  de degré $3$
  \item $X^n+1$ est  de degré $n$
  \item $2$ est un polynôme constant, de degré $0$ 
\end{itemize}
\end{exemple}
\end{frame}


%---------------------------------------------------------------
\section{Opérations sur les polynômes}

\begin{frame}
\begin{itemize}
  \item \textbf{\'Egalité}
  \begin{itemize}
    \item $P=a_nX^n+a_{n-1}X^{n-1}+\cdots + a_1X+a_0$
    \item $Q=b_nX^n+b_{n-1}X^{n-1}+\cdots+b_1X+b_0$
    \item $P=Q \quad \text{  ssi } \quad a_i=b_i \ \text{ pour tout } i$
  \end{itemize}
  
  
  \pause
  
  
  \item \textbf{Addition} 
$$P+Q = (a_n+b_n)X^n+(a_{n-1}+b_{n-1})X^{n-1}+\cdots
+(a_1+b_1)X+(a_0+b_0)$$

  \pause
  
  \item \textbf{Multiplication}
  \begin{itemize}
    \item $P=a_nX^n+a_{n-1}X^{n-1}+\cdots + a_1X+a_0$
    \item $Q=b_mX^m+b_{m-1}X^{m-1}+\cdots+b_1X+b_0$
    \item $P \times Q= c_rX^r+c_{r-1}X^{r-1}+\cdots +c_1X+c_0 $ \\
    avec $r=n+m$ \ et \myboxinline{$c_k=\sum_{i+j=k}a_ib_j$} pour $k\in \{0,\ldots, r\}$
  \end{itemize}
  
   \pause 
   
  \item \textbf{Multiplication par un scalaire}
Soit $\lambda \in \Kk$,  $\lambda \cdot P$ est le polynôme dont le $i$-ème coefficient
est $\lambda a_i$
\end{itemize}
\end{frame}


\begin{frame}

\begin{exemple}
$P=aX^3+bX^2+cX+d$ \qquad  $Q=\alpha X^2+\beta X + \gamma$
\pause
\begin{itemize}[<+->]
  \item $P+Q =  aX^3+(b+\alpha)X^2+(c+\beta)X+(d+\gamma)$
  \item $P\times Q = (a\alpha)X^5+(a\beta+b\alpha)X^4+(a\gamma+b\beta+c\alpha)X^3
+(b\gamma+c\beta+d\alpha)X^2+(c\gamma+d\beta)X+d\gamma$
  \item $P=Q$ si et seulement si $a=0$, $b=\alpha$, $c=\beta$ et $d=\gamma$
\end{itemize}
\end{exemple}

\end{frame}



\begin{frame}

\begin{proposition}
\begin{itemize}\setlength{\itemsep}{7pt} 
  \item $0+P=P$ \qquad $P+Q=Q+P$ \qquad $(P+Q)+R=P+(Q+R)$
\pause
  \item $1\cdot P = P$ \qquad $P\times Q=Q \times P$ \qquad $(P \times Q) \times R=P \times (Q \times R)$
\pause
  \item $P\times (Q+R)=P\times Q + P \times R$
\end{itemize}
\end{proposition}

\pause

\begin{proposition}
\mybox{$\deg(P\times Q)=\deg P + \deg Q$}
\pause
\mybox{$\deg(P+Q) \le \max(\deg P, \deg Q)$}
\end{proposition}

\bigskip
\pause

\centerline{$\Rr_n[X]= \big\{P \in \Rr[X] \mid \deg P \le n\big\}$}

\end{frame}

%---------------------------------------------------------------
\section{Vocabulaire}

\begin{frame}
\begin{mydefinition} \ 
\begin{itemize}
  \item $a_kX^k$ est un \defi{monôme}
\pause
  \item $P=a_nX^n+a_{n-1}X^{n-1}+\cdots + a_1X+a_0$ avec $a_n\neq0$
\pause  
  \begin{itemize}
    \item $a_nX^n$ est le \defi{terme dominant}
\pause    
    \item $a_n$ est le \defi{coefficient dominant} 
\pause    
    \item si $a_n=1$, $P$ est un \defi{polynôme unitaire}
  \end{itemize}
\end{itemize}
\end{mydefinition}
\pause
\begin{exemple}
\begin{itemize}
  \item $P(X)=(X-1)(X^n+X^{n-1}+\cdots + X+1)$
\pause  
  \item {\small $P(X)= \big(X^{n+1}+X^{n}+\cdots + X^2+X\big) - \big(X^n+X^{n-1}+\cdots + X+1\big) = X^{n+1} - 1$}
  
\pause  
  \item $P(X)$ est un polynôme unitaire, de degré $n+1$
\end{itemize}
\end{exemple}

\end{frame}



%%%%%%%%%%%%%%%%%%%%%%%%%%%%%%%%%%%%%%%%%%%%%%%%%%%%%%%%%%%%%%%%
\section{Mini-exercices}

\begin{frame}

\begin{miniexercice}
\begin{enumerate}
  \item Soit $P(X)=3X^3-2$, $Q(X)=X^2+X-1$, $R(X)=aX+b$.
Calculer $P+Q$, $P\times Q$, $(P+Q)\times R$ et $P\times Q \times R$.
Trouver $a$ et $b$ afin que le degré de $P-QR$ soit le plus petit possible.

  \item Calculer $(X+1)^5-(X-1)^5$.

  \item Déterminer le degré de $(X^2+X+1)^n - aX^{2n}-bX^{2n-1}$ en fonction de $a,b$.

  \item Montrer que si $\deg P \neq \deg Q$ alors $\deg (P+Q)= \max(\deg P,\deg Q)$.
Donner un contre-exemple dans le cas où $\deg P = \deg Q$.

  \item Montrer que si $P(X)=X^n+a_{n-1}X^{n-1}+ \cdots$ alors le coefficient devant $X^{n-1}$ de 
$P(X-\frac{a_{n-1}}{n})$ est nul.
\end{enumerate}
\end{miniexercice}

\end{frame}

\end{document}