
%%%%%%%%%%%%%%%%%% PREAMBULE %%%%%%%%%%%%%%%%%%


\documentclass[12pt]{article}

\usepackage{amsfonts,amsmath,amssymb,amsthm}
\usepackage[utf8]{inputenc}
\usepackage[T1]{fontenc}
\usepackage[francais]{babel}


% packages
\usepackage{amsfonts,amsmath,amssymb,amsthm}
\usepackage[utf8]{inputenc}
\usepackage[T1]{fontenc}
%\usepackage{lmodern}

\usepackage[francais]{babel}
\usepackage{fancybox}
\usepackage{graphicx}

\usepackage{float}

%\usepackage[usenames, x11names]{xcolor}
\usepackage{tikz}
\usepackage{datetime}

\usepackage{mathptmx}
%\usepackage{fouriernc}
%\usepackage{newcent}
\usepackage[mathcal,mathbf]{euler}

%\usepackage{palatino}
%\usepackage{newcent}


% Commande spéciale prompteur

%\usepackage{mathptmx}
%\usepackage[mathcal,mathbf]{euler}
%\usepackage{mathpple,multido}

\usepackage[a4paper]{geometry}
\geometry{top=2cm, bottom=2cm, left=1cm, right=1cm, marginparsep=1cm}

\newcommand{\change}{{\color{red}\rule{\textwidth}{1mm}\\}}

\newcounter{mydiapo}

\newcommand{\diapo}{\newpage
\hfill {\normalsize  Diapo \themydiapo \quad \texttt{[\jobname]}} \\
\stepcounter{mydiapo}}


%%%%%%% COULEURS %%%%%%%%%%

% Pour blanc sur noir :
%\pagecolor[rgb]{0.5,0.5,0.5}
% \pagecolor[rgb]{0,0,0}
% \color[rgb]{1,1,1}



%\DeclareFixedFont{\myfont}{U}{cmss}{bx}{n}{18pt}
\newcommand{\debuttexte}{
%%%%%%%%%%%%% FONTES %%%%%%%%%%%%%
\renewcommand{\baselinestretch}{1.5}
\usefont{U}{cmss}{bx}{n}
\bfseries

% Taille normale : commenter le reste !
%Taille Arnaud
%\fontsize{19}{19}\selectfont

% Taille Barbara
%\fontsize{21}{22}\selectfont

%Taille François
%\fontsize{25}{30}\selectfont

%Taille Pascal
%\fontsize{25}{30}\selectfont

%Taille Laura
%\fontsize{30}{35}\selectfont


%\myfont
%\usefont{U}{cmss}{bx}{n}

%\Huge
%\addtolength{\parskip}{\baselineskip}
}


% \usepackage{hyperref}
% \hypersetup{colorlinks=true, linkcolor=blue, urlcolor=blue,
% pdftitle={Exo7 - Exercices de mathématiques}, pdfauthor={Exo7}}


%section
% \usepackage{sectsty}
% \allsectionsfont{\bf}
%\sectionfont{\color{Tomato3}\upshape\selectfont}
%\subsectionfont{\color{Tomato4}\upshape\selectfont}

%----- Ensembles : entiers, reels, complexes -----
\newcommand{\Nn}{\mathbb{N}} \newcommand{\N}{\mathbb{N}}
\newcommand{\Zz}{\mathbb{Z}} \newcommand{\Z}{\mathbb{Z}}
\newcommand{\Qq}{\mathbb{Q}} \newcommand{\Q}{\mathbb{Q}}
\newcommand{\Rr}{\mathbb{R}} \newcommand{\R}{\mathbb{R}}
\newcommand{\Cc}{\mathbb{C}} 
\newcommand{\Kk}{\mathbb{K}} \newcommand{\K}{\mathbb{K}}

%----- Modifications de symboles -----
\renewcommand{\epsilon}{\varepsilon}
\renewcommand{\Re}{\mathop{\text{Re}}\nolimits}
\renewcommand{\Im}{\mathop{\text{Im}}\nolimits}
%\newcommand{\llbracket}{\left[\kern-0.15em\left[}
%\newcommand{\rrbracket}{\right]\kern-0.15em\right]}

\renewcommand{\ge}{\geqslant}
\renewcommand{\geq}{\geqslant}
\renewcommand{\le}{\leqslant}
\renewcommand{\leq}{\leqslant}

%----- Fonctions usuelles -----
\newcommand{\ch}{\mathop{\mathrm{ch}}\nolimits}
\newcommand{\sh}{\mathop{\mathrm{sh}}\nolimits}
\renewcommand{\tanh}{\mathop{\mathrm{th}}\nolimits}
\newcommand{\cotan}{\mathop{\mathrm{cotan}}\nolimits}
\newcommand{\Arcsin}{\mathop{\mathrm{Arcsin}}\nolimits}
\newcommand{\Arccos}{\mathop{\mathrm{Arccos}}\nolimits}
\newcommand{\Arctan}{\mathop{\mathrm{Arctan}}\nolimits}
\newcommand{\Argsh}{\mathop{\mathrm{Argsh}}\nolimits}
\newcommand{\Argch}{\mathop{\mathrm{Argch}}\nolimits}
\newcommand{\Argth}{\mathop{\mathrm{Argth}}\nolimits}
\newcommand{\pgcd}{\mathop{\mathrm{pgcd}}\nolimits} 

\newcommand{\Card}{\mathop{\text{Card}}\nolimits}
\newcommand{\Ker}{\mathop{\text{Ker}}\nolimits}
\newcommand{\id}{\mathop{\text{id}}\nolimits}
\newcommand{\ii}{\mathrm{i}}
\newcommand{\dd}{\mathrm{d}}
\newcommand{\Vect}{\mathop{\text{Vect}}\nolimits}
\newcommand{\Mat}{\mathop{\mathrm{Mat}}\nolimits}
\newcommand{\rg}{\mathop{\text{rg}}\nolimits}
\newcommand{\tr}{\mathop{\text{tr}}\nolimits}
\newcommand{\ppcm}{\mathop{\text{ppcm}}\nolimits}

%----- Structure des exercices ------

\newtheoremstyle{styleexo}% name
{2ex}% Space above
{3ex}% Space below
{}% Body font
{}% Indent amount 1
{\bfseries} % Theorem head font
{}% Punctuation after theorem head
{\newline}% Space after theorem head 2
{}% Theorem head spec (can be left empty, meaning ‘normal’)

%\theoremstyle{styleexo}
\newtheorem{exo}{Exercice}
\newtheorem{ind}{Indications}
\newtheorem{cor}{Correction}


\newcommand{\exercice}[1]{} \newcommand{\finexercice}{}
%\newcommand{\exercice}[1]{{\tiny\texttt{#1}}\vspace{-2ex}} % pour afficher le numero absolu, l'auteur...
\newcommand{\enonce}{\begin{exo}} \newcommand{\finenonce}{\end{exo}}
\newcommand{\indication}{\begin{ind}} \newcommand{\finindication}{\end{ind}}
\newcommand{\correction}{\begin{cor}} \newcommand{\fincorrection}{\end{cor}}

\newcommand{\noindication}{\stepcounter{ind}}
\newcommand{\nocorrection}{\stepcounter{cor}}

\newcommand{\fiche}[1]{} \newcommand{\finfiche}{}
\newcommand{\titre}[1]{\centerline{\large \bf #1}}
\newcommand{\addcommand}[1]{}
\newcommand{\video}[1]{}

% Marge
\newcommand{\mymargin}[1]{\marginpar{{\small #1}}}



%----- Presentation ------
\setlength{\parindent}{0cm}

%\newcommand{\ExoSept}{\href{http://exo7.emath.fr}{\textbf{\textsf{Exo7}}}}

\definecolor{myred}{rgb}{0.93,0.26,0}
\definecolor{myorange}{rgb}{0.97,0.58,0}
\definecolor{myyellow}{rgb}{1,0.86,0}

\newcommand{\LogoExoSept}[1]{  % input : echelle
{\usefont{U}{cmss}{bx}{n}
\begin{tikzpicture}[scale=0.1*#1,transform shape]
  \fill[color=myorange] (0,0)--(4,0)--(4,-4)--(0,-4)--cycle;
  \fill[color=myred] (0,0)--(0,3)--(-3,3)--(-3,0)--cycle;
  \fill[color=myyellow] (4,0)--(7,4)--(3,7)--(0,3)--cycle;
  \node[scale=5] at (3.5,3.5) {Exo7};
\end{tikzpicture}}
}



\theoremstyle{definition}
%\newtheorem{proposition}{Proposition}
%\newtheorem{exemple}{Exemple}
%\newtheorem{theoreme}{Théorème}
\newtheorem{lemme}{Lemme}
\newtheorem{corollaire}{Corollaire}
%\newtheorem*{remarque*}{Remarque}
%\newtheorem*{miniexercice}{Mini-exercices}
%\newtheorem{definition}{Définition}




%definition d'un terme
\newcommand{\defi}[1]{{\color{myorange}\textbf{\emph{#1}}}}
\newcommand{\evidence}[1]{{\color{blue}\textbf{\emph{#1}}}}



 %----- Commandes divers ------

\newcommand{\codeinline}[1]{\texttt{#1}}

%%%%%%%%%%%%%%%%%%%%%%%%%%%%%%%%%%%%%%%%%%%%%%%%%%%%%%%%%%%%%
%%%%%%%%%%%%%%%%%%%%%%%%%%%%%%%%%%%%%%%%%%%%%%%%%%%%%%%%%%%%%



\begin{document}

\debuttexte

%%%%%%%%%%%%%%%%%%%%%%%%%%%%%%%%%%%%%%%%%%%%%%%%%%%%%%%%%%%
\diapo

\change

Voici un deuxième exemple de groupe fini, également très important,
le groupe des permutations $\mathcal{S}_n$.

\change

Après l'avoir défini,

\change

nous regarderons l'exemple $\mathcal{S}_3$

\change

Et nous terminerons par la décomposition des permutations en 
composition de cycles.


%%%%%%%%%%%%%%%%%%%%%%%%%%%%%%%%%%%%%%%%%%%%%%%%%%%%%%%%%%%
\diapo

Fixons un entier $n\ge 2$.



Nous considérons toutes les bijections qui vont de $\{1,2,\ldots,n\}$
dans $\{1,2,\ldots,n\}$.

Comme l'ensemble de départ et d'arrivée sont les mêmes 
ont peut composer deux bijections.

Alors  l'ensemble de ces bijections, muni de la loi de composition
des fonctions, est un groupe, noté $(\mathcal{S}_n,\circ)$

\change

Une bijection de $\{1,2,\ldots,n\}$
dans $\{1,2,\ldots,n\}$ s'appelle une permutation.

Le groupe $\mathcal{S}_n$ s'appelle le groupe des permutations
ou aussi le groupe symétrique.


%%%%%%%%%%%%%%%%%%%%%%%%%%%%%%%%%%%%%%%%%%%%%%%%%%%%%%%%%%%
\diapo

Le groupe $\mathcal{S}_n$ est un groupe fini
qui contient factorielle $n$ éléments.

\change

La preuve n'est pas dure, nous comptons le nombre de bijections $f$ possibles.

L'image de $1$ peut être n'importe quelle valeur de $1$ à $n$ : il y a donc $n$
choix pour définir $f(1)$.

Pour $f(2)$ un peut choisir toute les valeurs sauf $f(1)$ (car sinon $f$ n'est pas bijective) : 
pour l'image de $2$ il y a donc $n-1$ choix.

Pour $f(3)$ il y a $n-2$ possibilités,

etc.

Pour $f(n)$ il ne reste qu'une seule possibilité.

On résume : $n$ choix pour $f(1)$,
 multiplié par $n-1$ choix pour $f(2)$, multiplié par $n-2$ etc.

Il y a bien $n!$ bijections 

le cardinal du groupe des permutation est $n!$


%%%%%%%%%%%%%%%%%%%%%%%%%%%%%%%%%%%%%%%%%%%%%%%%%%%%%%%%%%%
\diapo

Définir une permutation $f$ équivaut à se donner les images
$f(1)$, $f(2)$, jusqu'à $f(n)$.

Nous noterons une permutations comme un tableau à deux lignes :

première ligne : les éléments de $1$ à $n$,

deuxième ligne : en dessous de $1$, on écrit $f(1)$, en dessous de $2$ on écrit $f(2)$, etc.

\change

Pour cet exemple $n=7$. 
Ce tableau définit bien une permutation $f$ de $\mathcal{S}_7$,

c'est-à-dire une bijection des entiers de $1$ à $7$ à valeur
dans les entiers de $1$ à $7$.

\change

Sur la ligne du haut les éléments de $1$ à $7$ et en bas leur image par $f$

\change

Ici on lit donc $f(1)=3$, $f(2)=7$, $f(3)=5$, $f(4)=4$
$f(5)=6$, $f(6)=1$, $f(7)=2$.

Il est essentiel que l'on retrouve sur la ligne bas, exactement tous les entiers de
$1$ à $7$ (pas nécessairement dans l'ordre bien sûr !) car $f$ est une bijection.



%%%%%%%%%%%%%%%%%%%%%%%%%%%%%%%%%%%%%%%%%%%%%%%%%%%%%%%%%%%
\diapo

Voyons de plus près la structure du groupe $\mathcal{S}_n$.

Son élément neutre est l'application identité, qui à un entier 
$k$ associe $k$.

Par exemple l'élément neutre de $\mathcal{S}_7$ est la permutation suivante,

l'image de $1$ est $1$, l'image de $2$ est $2$, etc.



\change

Ayant deux permutations $f$ et $g$ comment calculer la composition $g\circ f$ ?

\change

C'est facile ! Tout d'abord on récrit la permutation $f$.

\change

De la deuxième à la troisième ligne nous allons écrire la permutation $g$.
Attention, il faut bien associer à chaque élément son image,

donc ici pour la première colonne quelle est l'image de $3$ par $g$, c'est $2$.

\change

Ensuite on continue, l'image de $7$ par $g$ est $6$ 
 


et on complète ainsi la troisième ligne.

\change

nous avons de la première à la deuxième ligne l'application $f$,

de la deuxième ligne à la troisième c'est $g$,

donc de la première ligne à la troisième ligne c'est $g\circ f$.

\change

La deuxième ligne est devenue inutile.

\change

On ne conserve que la première et la dernière ligne, c'est l'écriture de la permutation $g\circ f$.

\change


L'inverse d'une permutation $f$ est encore plus facile à écrire.
Il suffit d'échanger les lignes du haut et du bas.

\change

Comme $f$ se lit du haut vers la bas alors $f^{-1}$ se lit de bas en haut.

\change

Pour écrire la permutation $f^{-1}$ on recopie d'abord la ligne du bas de $f$ 

\change

ensuite on recopie la ligne de haut de $f$

\change

et on préfère réordonner le tableau en commençant par $1$ et son image,
puis $2$ et son image...


%%%%%%%%%%%%%%%%%%%%%%%%%%%%%%%%%%%%%%%%%%%%%%%%%%%%%%%%%%%
\diapo

Nous allons regarder le groupe $\mathcal{S}_3$ de plus près.

Tout d'abord il contient $3!=6$ éléments que nous explicitons :

\change

Tour d'abord l'identité

\change

Ensuite la permutation $\tau_1$ qui envoie $1$ sur $1$,
$2$ sur $3$ et $3$ sur $2$.

\change

De même $\tau_2$ qui fixe $2$ et échange $1$ et $3$.

\change

Et $\tau_3$ qui fixe $3$ et échange $1$ et $2$.

\change

Ensuite notons $\sigma$ la permutation qui envoie $1$ sur $2$,
$2$ sur $3$ et $3$ sur $1$.

\change

et $\sigma^{-1}$ est son inverse qui est la permutation qui envoie $1$ sur $3$,
$2$ sur $1$ et $3$ sur $2$.

\change

Nous avons trouvé les $6$ éléments qui composent $\mathcal{S}_3$.

Maintenant effectuons quelques calculs.

\change

Calculons la composition $\tau_1\circ \sigma$ on écrit $\sigma$ en haut, $\tau_1$ en bas

\change

et on supprime la ligne intermédiaire. 
C'est la permutation qui à $1$ associe $3$,
à $2$ associe $2$, à $3$ associe $1$.

\change

La permutation obtenue est donc $\tau_2$.

\change

Par contre si l'on calcule $\sigma \circ \tau_1$ alors 
la permutation obtenue échange $1$ et $2$ et fixe $3$

\change

c'est donc $\tau_3$.

Ainsi $\tau_1\circ \sigma$ et $\sigma \circ \tau_1$ sont deux permutations différentes.

Le groupe $\mathcal{S}_3$ n'est donc pas un groupe commutatif.

\change

Le même exemple prouve qu'en général, pour $n\ge 3$,
le groupe $\mathcal{S}_n$ n'est pas commutatif.


%%%%%%%%%%%%%%%%%%%%%%%%%%%%%%%%%%%%%%%%%%%%%%%%%%%%%%%%%%%
\diapo

Nous allons établir la table du groupe $\mathcal{S}_3$, c'est-à-dire
calculer toutes les compositions $g \circ f$ pour tous les $f,g$ de 
$\mathcal{S}_3$.

\change

Par exemple nous venons de calculer $\tau_1 \circ \sigma = \tau_2$

\change
 
et aussi  $\sigma \circ \tau_1=\tau_3$

\change

Il y a des cases qui se remplissent facilement : si $g$ égale l'identité
 alors $g \circ f = f$

\change

si c'est $f$ qui est l'identité alors $g\circ f = g$.

\change

Pour compléter les autres cases, il n'y a pas d'autre solution que de faire
les calculs des compositions !

A vous de vérifier la table !

%%%%%%%%%%%%%%%%%%%%%%%%%%%%%%%%%%%%%%%%%%%%%%%%%%%%%%%%%%%
\diapo

Relions le groupe $\mathcal{S}_3$ a un problème géométrique.

Partons de $ABC$ est un triangle équilatéral.

\change

\change

On cherche toutes les isométries du plan qui envoie le triangle sur lui même.

Ce sont les applications bijectives qui conservent les distances et qui envoie
un sommet du triangle sur un sommet (pas nécessairement le même).

\change

Essayons de déterminer ces isométries :

il y a bien sûr l'identité : elle envoie $A$ sur $A$, 
$B$ sur $B$ et $C$ sur $C$.

\change

il y a la rotation $s$ de centre $O$ et d'angle $+2\pi/3$ envoie
$A$ sur $B$, $B$ sur $C$ et $C$ sur $A$.


il y a son inverse $s^{-1}$, qui est la rotation d'angle $-2\pi/3$.

\change


N'oublions pas les réflexions : la réflexions $t_1$ d'axes $\mathcal{D}_1$
envoie $A$ sur $A$ et échange $B$ et $C$.

Il y a aussi les réflexions d'axes $\mathcal{D}_2$ et $\mathcal{D}_3$.

\change

Ces $6$ isométries ont une structure de groupe :

  * l'élément neutre est l'identité,

  * la composition d'une rotation et d'une réflexion est une autre réflexion,
  
  * l'inverse de la rotation d'angle $+2\pi/3$ est la rotation d'angle $-2\pi/3$,
une réflexion est son propre inverse.


\change

Ce qui est encore remarquable : c'est que ce groupe est isomorphe à $\mathcal{S}_3$.

La preuve s'effectue en construisant directement l'isomorphisme :

on envoie la réflexion $t_1$ sur la permutation $\tau_1$,

on envoie $t_2$ sur $\tau_2$, et $t_3$ sur $\tau_3$.

En on envoie la rotation $s$ sur la permutation $\sigma$,
et $s^{-1}$ sur $\sigma^{-1}$.


Cela donne une nouvelle vision du groupe des permutations $\mathcal{S}_3$,
c'est aussi le groupe des isométries d'un triangle équilatéral.



%%%%%%%%%%%%%%%%%%%%%%%%%%%%%%%%%%%%%%%%%%%%%%%%%%%%%%%%%%%
\diapo


Nous allons nous concentrer sur quelques permutations spéciales --les cycles--
qui vont engendrer toutes les permutations.

Un cycle est une permutation $\sigma$ :
 
\begin{itemize}
  \item qui fixe un certain nombre d'éléments ; $i$ est fixe si $\sigma(i)=i$
  \item tous les autres sont obtenus par itération d'un seul éléments c'est-à-dire 
qu'on peut trouver $j$ et les éléments non fixés sont $\sigma(j),\sigma^2(j),\ldots$
\end{itemize}

\change

C'est plus simple à comprendre sur un exemple :

Soit $\sigma$ la permutation suivante.

\change

  - les éléments $1, 3, 6, 7$ sont fixes

\change

  - les autres s'obtiennent comme itération de $2$ 

\change

$2$ s'envoie sur sur $8$ , $8$ s'envoie sur $4$, $4$ sur $5$
et $5$ s'envoie sur notre premier élément $2$.

On aurait aussi bien pu partir de $4$, $5$ ou $8$.

\change

Nous noterons les cycles sous une forme plus condensée :
ici nous l'écrivons : $(2\ 8\ 4\ 5)$.

\change

Il faut lire cela ainsi $2$ s'envoie sur $8$, $8$ sur $4$ , $4$ sur $5$
et $5$ repart à $2$.

Les éléments qui n'apparaissent pas sont fixes : par exemple $3$ n’apparaît
pas donc $\sigma(3)=3$.

\change

On aurait pu  noter ce cycle d'autres façons, cette écriture n'est pas unique :

les autres écritures du même cycle sont $(8\ 4\ 5\ 2)$, $(4\ 5\ 2\ 8)$ ou $(5\ 2\ 8\ 4)$.

\change

Enfin l'inverse d'un cycle est aussi un cycle.

Pour écrire le cycle $\sigma^{-1}$ il suffit de renverser les éléments.

L'inverse du cycle $(2\ 8\ 4\ 5)$ est $(5\ 4\ 8\ 2)$  


%%%%%%%%%%%%%%%%%%%%%%%%%%%%%%%%%%%%%%%%%%%%%%%%%%%%%%%%%%%
\diapo

Encore quelques définitions :

Le support d'un cycle sont les éléments qui ne sont pas fixes 

\change

La longueur (ou l'ordre) d'un cycle est le nombre d'éléments qui ne sont pas fixes

c'est donc le cardinal du support.

\change

Par exemple le support de cycle $(2\ 8\ 4\ 5)$ est donc 
l'ensemble $\{2, 4, 5, 8\}$, ce cycle est de longueur $4$.

\change

Nous avons déjà rencontré des cycles :

par exemple cette permutation $\sigma$ est le cycle $(1\ 2\ 3)$ 
qui est de longueur $3$ 
Il n'a pas d'éléments fixes.

\change

Par contre la permutation suivante 
fixe $1$ et fixe $3$, les éléments $2$ et $4$ sont échangés.

C'est donc le cycle $(2\ 4)$ de longueur $2$.

Les cycles de longueur $2$ s'appelle des transpositions.

\change

Attention ! Toutes les permutations ne sont pas des cycles. 

Par exemple la permutation $f$ suivante n'est pas un cycle.

Partons de l'élément $1$ il s'envoie sur $7$ et $7$ s'envoie sur $1$.

Mais les autres éléments ne sont pas tous fixes.

Par exemple $3$ s'envoie sur $5$  qui s'envoie sur $6$ et retour à $3$

Les éléments $2$ et $4$ eux sont fixes.

\change

On a en fait réalisé $f$ comme la composition de deux cycles, le cycle $(1\ 7)$ et le cycle $(3\ 5\ 6)$.



\change

Les supports de deux cycles $(1\ 7)$ et $(3\ 5\ 6)$ n'ont pas d'éléments communs
alors on a aussi l'égalité

$f = (3\ 5\ 6) \circ (1\ 7)$  

%%%%%%%%%%%%%%%%%%%%%%%%%%%%%%%%%%%%%%%%%%%%%%%%%%%%%%%%%%%
\diapo

[plusieurs prises]

L'exemple précédent est un cas particulier d'un théorème général que nous admettons.

Toute permutation de $\mathcal{S}_n$ se décompose en composition
de cycles à supports disjoints

Autrement dit les cycles engendrent toutes les permutations.


\change

Nous avons l'existence mais nous avons aussi l'unicité :
 
De plus cette décomposition est unique

\change

Il faut comprendre l'unicité ainsi :

on autorise quand même plusieurs écritures pour chacun des cycles

 par exemple $(3\ 5\ 6)$ et $(5\ 6\ 3)$sont deux écritures possibles du même cycle.

\change

on autorise aussi de changer l'ordre de composition des cycles 

par exemple 
les deux écritures $(1\ 7) \circ (3\ 5\ 6)$ et $(3\ 5\ 6) \circ (1\ 7)$ 
désignent la même permutation.


Modulo ces changements d'écriture  la décomposition d'une permutation en composition 
de cycle à supports disjoints est unique.

\change

Voyons comment trouver cette décomposition sur la permutation $f$ que voilà :

on part de $1$ il s'envoie sur $5$ qui s'envoie sur $3$ qui retourne à $1$ :

\change

on donc un premier cycle $(1\ 5\ 3)$

prenons l'élément suivant qui n'est pas encore apparu, c'est $2$

$2$ s'envoie sur $2$, on ne notera pas les éléments fixes.

On prend l'élément suivant c'est $3$ mais il est déjà dans notre premier cycle,

l'élément suivant est $4$ il s'envoie sur $8$ qui revient à $4$.

On compose donc avec la cycle $(4\ 8)$.

\change

Ensuite $6$ s'envoie sur $7$ qui s'envoie sur $6$.

on compose par le cycle $(6\ 7)$.

\change

On a bien écrit $f$ comme une composition de cycles,

par construction les supports sont disjoints.

Le théorème nous certifie que si nous avions fait d'autre choix (par exemple on part
de la fin en cherchant d'abord l'image de $7$, etc. ) alors nous aurions 
obtenu la même décomposition en s'autorisant à écrire les cycles dans l'ordre que l'on souhaite.

\change

Attention ! Le fait que les supports soient disjoints est crucial.

Si l'on écrit des compositions de cycles et que les supports ont des éléments en communs alors 
cela ne commute plus.

\change

Par exemple soit $g$ la permutation définie comme le produit des deux cycles que voilà.

Ceci n'est pas la décomposition du théorème en effet les cycles ont l'élément $2$ en commun.

Calculons cette permutations :

\change

on superpose les permutations correspondant aux deux cycles 

\change

on supprime la ligne du milieu

\change

cette permutation est le cycle $(1\ 2\ 3\ 4)$.

C'est bien $(1\ 2\ 3\ 4)$ qui la décomposition en composition de cycle à support disjoint de $g$.


\change

Même chose avec $h$

\change

\change

Je vous laisse vérifier que $h$ est la cycle $(1\ 3\ 4\ 2)$.

Ainsi $g$ et $h$ sont deux permutations différentes,
par exemple par $g(1)=2$ alors que $h(1)=3$.




%%%%%%%%%%%%%%%%%%%%%%%%%%%%%%%%%%%%%%%%%%%%%%%%%%%%%%%%%%%
\diapo

Comme d'habitude, nous terminons par les mini-exercices.



\end{document}