
%%%%%%%%%%%%%%%%%% PREAMBULE %%%%%%%%%%%%%%%%%%


\documentclass[12pt]{article}

\usepackage{amsfonts,amsmath,amssymb,amsthm}
\usepackage[utf8]{inputenc}
\usepackage[T1]{fontenc}
\usepackage[francais]{babel}


% packages
\usepackage{amsfonts,amsmath,amssymb,amsthm}
\usepackage[utf8]{inputenc}
\usepackage[T1]{fontenc}
%\usepackage{lmodern}

\usepackage[francais]{babel}
\usepackage{fancybox}
\usepackage{graphicx}

\usepackage{float}

%\usepackage[usenames, x11names]{xcolor}
\usepackage{tikz}
\usepackage{datetime}

\usepackage{mathptmx}
%\usepackage{fouriernc}
%\usepackage{newcent}
\usepackage[mathcal,mathbf]{euler}

%\usepackage{palatino}
%\usepackage{newcent}


% Commande spéciale prompteur

%\usepackage{mathptmx}
%\usepackage[mathcal,mathbf]{euler}
%\usepackage{mathpple,multido}

\usepackage[a4paper]{geometry}
\geometry{top=2cm, bottom=2cm, left=1cm, right=1cm, marginparsep=1cm}

\newcommand{\change}{{\color{red}\rule{\textwidth}{1mm}\\}}

\newcounter{mydiapo}

\newcommand{\diapo}{\newpage
\hfill {\normalsize  Diapo \themydiapo \quad \texttt{[\jobname]}} \\
\stepcounter{mydiapo}}


%%%%%%% COULEURS %%%%%%%%%%

% Pour blanc sur noir :
%\pagecolor[rgb]{0.5,0.5,0.5}
% \pagecolor[rgb]{0,0,0}
% \color[rgb]{1,1,1}



%\DeclareFixedFont{\myfont}{U}{cmss}{bx}{n}{18pt}
\newcommand{\debuttexte}{
%%%%%%%%%%%%% FONTES %%%%%%%%%%%%%
\renewcommand{\baselinestretch}{1.5}
\usefont{U}{cmss}{bx}{n}
\bfseries

% Taille normale : commenter le reste !
%Taille Arnaud
%\fontsize{19}{19}\selectfont

% Taille Barbara
%\fontsize{21}{22}\selectfont

%Taille François
%\fontsize{25}{30}\selectfont

%Taille Pascal
%\fontsize{25}{30}\selectfont

%Taille Laura
%\fontsize{30}{35}\selectfont


%\myfont
%\usefont{U}{cmss}{bx}{n}

%\Huge
%\addtolength{\parskip}{\baselineskip}
}


% \usepackage{hyperref}
% \hypersetup{colorlinks=true, linkcolor=blue, urlcolor=blue,
% pdftitle={Exo7 - Exercices de mathématiques}, pdfauthor={Exo7}}


%section
% \usepackage{sectsty}
% \allsectionsfont{\bf}
%\sectionfont{\color{Tomato3}\upshape\selectfont}
%\subsectionfont{\color{Tomato4}\upshape\selectfont}

%----- Ensembles : entiers, reels, complexes -----
\newcommand{\Nn}{\mathbb{N}} \newcommand{\N}{\mathbb{N}}
\newcommand{\Zz}{\mathbb{Z}} \newcommand{\Z}{\mathbb{Z}}
\newcommand{\Qq}{\mathbb{Q}} \newcommand{\Q}{\mathbb{Q}}
\newcommand{\Rr}{\mathbb{R}} \newcommand{\R}{\mathbb{R}}
\newcommand{\Cc}{\mathbb{C}} 
\newcommand{\Kk}{\mathbb{K}} \newcommand{\K}{\mathbb{K}}

%----- Modifications de symboles -----
\renewcommand{\epsilon}{\varepsilon}
\renewcommand{\Re}{\mathop{\text{Re}}\nolimits}
\renewcommand{\Im}{\mathop{\text{Im}}\nolimits}
%\newcommand{\llbracket}{\left[\kern-0.15em\left[}
%\newcommand{\rrbracket}{\right]\kern-0.15em\right]}

\renewcommand{\ge}{\geqslant}
\renewcommand{\geq}{\geqslant}
\renewcommand{\le}{\leqslant}
\renewcommand{\leq}{\leqslant}

%----- Fonctions usuelles -----
\newcommand{\ch}{\mathop{\mathrm{ch}}\nolimits}
\newcommand{\sh}{\mathop{\mathrm{sh}}\nolimits}
\renewcommand{\tanh}{\mathop{\mathrm{th}}\nolimits}
\newcommand{\cotan}{\mathop{\mathrm{cotan}}\nolimits}
\newcommand{\Arcsin}{\mathop{\mathrm{Arcsin}}\nolimits}
\newcommand{\Arccos}{\mathop{\mathrm{Arccos}}\nolimits}
\newcommand{\Arctan}{\mathop{\mathrm{Arctan}}\nolimits}
\newcommand{\Argsh}{\mathop{\mathrm{Argsh}}\nolimits}
\newcommand{\Argch}{\mathop{\mathrm{Argch}}\nolimits}
\newcommand{\Argth}{\mathop{\mathrm{Argth}}\nolimits}
\newcommand{\pgcd}{\mathop{\mathrm{pgcd}}\nolimits} 

\newcommand{\Card}{\mathop{\text{Card}}\nolimits}
\newcommand{\Ker}{\mathop{\text{Ker}}\nolimits}
\newcommand{\id}{\mathop{\text{id}}\nolimits}
\newcommand{\ii}{\mathrm{i}}
\newcommand{\dd}{\mathrm{d}}
\newcommand{\Vect}{\mathop{\text{Vect}}\nolimits}
\newcommand{\Mat}{\mathop{\mathrm{Mat}}\nolimits}
\newcommand{\rg}{\mathop{\text{rg}}\nolimits}
\newcommand{\tr}{\mathop{\text{tr}}\nolimits}
\newcommand{\ppcm}{\mathop{\text{ppcm}}\nolimits}

%----- Structure des exercices ------

\newtheoremstyle{styleexo}% name
{2ex}% Space above
{3ex}% Space below
{}% Body font
{}% Indent amount 1
{\bfseries} % Theorem head font
{}% Punctuation after theorem head
{\newline}% Space after theorem head 2
{}% Theorem head spec (can be left empty, meaning ‘normal’)

%\theoremstyle{styleexo}
\newtheorem{exo}{Exercice}
\newtheorem{ind}{Indications}
\newtheorem{cor}{Correction}


\newcommand{\exercice}[1]{} \newcommand{\finexercice}{}
%\newcommand{\exercice}[1]{{\tiny\texttt{#1}}\vspace{-2ex}} % pour afficher le numero absolu, l'auteur...
\newcommand{\enonce}{\begin{exo}} \newcommand{\finenonce}{\end{exo}}
\newcommand{\indication}{\begin{ind}} \newcommand{\finindication}{\end{ind}}
\newcommand{\correction}{\begin{cor}} \newcommand{\fincorrection}{\end{cor}}

\newcommand{\noindication}{\stepcounter{ind}}
\newcommand{\nocorrection}{\stepcounter{cor}}

\newcommand{\fiche}[1]{} \newcommand{\finfiche}{}
\newcommand{\titre}[1]{\centerline{\large \bf #1}}
\newcommand{\addcommand}[1]{}
\newcommand{\video}[1]{}

% Marge
\newcommand{\mymargin}[1]{\marginpar{{\small #1}}}



%----- Presentation ------
\setlength{\parindent}{0cm}

%\newcommand{\ExoSept}{\href{http://exo7.emath.fr}{\textbf{\textsf{Exo7}}}}

\definecolor{myred}{rgb}{0.93,0.26,0}
\definecolor{myorange}{rgb}{0.97,0.58,0}
\definecolor{myyellow}{rgb}{1,0.86,0}

\newcommand{\LogoExoSept}[1]{  % input : echelle
{\usefont{U}{cmss}{bx}{n}
\begin{tikzpicture}[scale=0.1*#1,transform shape]
  \fill[color=myorange] (0,0)--(4,0)--(4,-4)--(0,-4)--cycle;
  \fill[color=myred] (0,0)--(0,3)--(-3,3)--(-3,0)--cycle;
  \fill[color=myyellow] (4,0)--(7,4)--(3,7)--(0,3)--cycle;
  \node[scale=5] at (3.5,3.5) {Exo7};
\end{tikzpicture}}
}



\theoremstyle{definition}
%\newtheorem{proposition}{Proposition}
%\newtheorem{exemple}{Exemple}
%\newtheorem{theoreme}{Théorème}
\newtheorem{lemme}{Lemme}
\newtheorem{corollaire}{Corollaire}
%\newtheorem*{remarque*}{Remarque}
%\newtheorem*{miniexercice}{Mini-exercices}
%\newtheorem{definition}{Définition}




%definition d'un terme
\newcommand{\defi}[1]{{\color{myorange}\textbf{\emph{#1}}}}
\newcommand{\evidence}[1]{{\color{blue}\textbf{\emph{#1}}}}



 %----- Commandes divers ------

\newcommand{\codeinline}[1]{\texttt{#1}}

%%%%%%%%%%%%%%%%%%%%%%%%%%%%%%%%%%%%%%%%%%%%%%%%%%%%%%%%%%%%%
%%%%%%%%%%%%%%%%%%%%%%%%%%%%%%%%%%%%%%%%%%%%%%%%%%%%%%%%%%%%%


\begin{document}

\debuttexte


%%%%%%%%%%%%%%%%%%%%%%%%%%%%%%%%%%%%%%%%%%%%%%%%%%%%%%%%%%%
\diapo

% Tout sous-espace vectoriel d'un espace vectoriel est lui même 
% un espace vectoriel, et on peut se demander s'il est de dimension finie ou pas.\\
% 
% Prenons l'exemple de l'espace vectoriel  
% des fonctions de $\Rr$ dans $\Rr$ :\\
% 
%  il contient le sous-espace vectoriel 
%   des polynômes de degré $\le n$, qui est de dimension finie ;\\
%   
% mais aussi le sous-espace vectoriel  de toutes les
%   fonctions polynomiales, qui lui est de dimension infinie.
% 
% Ceci vient du fait que l'espace vectoriel  
% des fonctions de $\Rr$ dans $\Rr$ est de dimension infinie.

\change
Nous allons voir dans cette le\c{c}on 

\change
que dans les espaces vectoriels de dimension finie, tout sous-espace vectoriel est également de dimension finie.

\change
nous donnerons quelques exemples de sous-espaces vectoriels et nous calculerons leurs dimensions 

\change
et enfin nous aborderons du théorème des 4 dimensions.


%%%%%%%%%%%%%%%%%%%%%%%%%%%%%%%%%%%%%%%%%%%%%%%%%%%%%%%%%%%
\diapo
Considérons un $\Kk$-espace vectoriel $E$ de dimension finie. Alors on a le théorème suivant

\change
tout sous-espace vectoriel $F$ de $E$ est aussi de dimension finie 

\change
de plus la dimension de $F$ est nécessairement inférieure à la dimension de $E$

\change
en enfin, la dimension de $F$ est égale à la dimension de $E$ si et seulement si $F$ \defi{égal} $E$.


%%%%%%%%%%%%%%%%%%%%%%%%%%%%%%%%%%%%%%%%%%%%%%%%%%%%%%%%%%%
\diapo
Voyons un exemple simple : dans le cas o\`u la dimension de l'espace vectoriel $E$ est $2$.

\change
Quels sont les sous-espaces vectoriels $F$ de $E$?

\change
si le sous-espace vectoriel est  de dimension $0$ : alors c'est  le sous-espace réduit au vecteur nul

\change
  
si le sous-espace vectoriel est  de dimension $1$ : c'est une droite vectorielle, c'est-à-dire 
 un sous-espace engendré par un vecteur $u$ non nul de $E$ 
  
  \change
 si le sous-espace vectoriel est  de dimension $2$ : c'est alors l'espace $E$ tout entier.\\
 
 \change
 Plus généralement, dans un $\Kk$-espace vectoriel $E$ de dimension $n$ 
%  supérieure ou égale à 2, 

\change
tout sous-espace vectoriel de $E$ de dimension $1$ est 
appelé \defi{droite vectorielle} de $E$ 

\change
et tout sous-espace vectoriel de 
$E$ de dimension $2$ est appelé \defi{plan vectoriel} de $E$.

\change
Tout sous-espace vectoriel de $E$ de dimension $n-1$ est 
appelé \defi{hyperplan} de $E$. \\

% Remarquons que 
% pour $n=3$, un hyperplan est un plan vectoriel ; pour $n=2$,
% un hyperplan est une droite vectorielle. 

%%%%%%%%%%%%%%%%%%%%%%%%%%%%%%%%%%%%%%%%%%%%%%%%%%%%%%%%%%%
\diapo
Voyons maintenant un corollaire du théorème sur la dimension des sous-espaces.

Soit $E$ un $\Kk$-espace vectoriel
et
$F$ et $G$ deux sous-espaces vectoriels de $E$.\\

$F$ est supposé de dimension finie
et $G \subset F$.
\\
Alors il y a équivalence entre
$F=G$ et $ \dim F = \dim G$

\change
Voyons un exemple d'application de ce résultat.
Considérons le sous-espace vectoriel $F$ de $\Rr^3$ défini comme l'ensemble des vecteurs de coordonnées $x,y,z$ 
qui vérifient l'équation $2x-3y+z=0$

\change
considérons aussi le sous-espace vectoriel $G$ engendré par les vecteurs $u$ et $v$ suivants

\change
Est-ce que $F=G$ ?

\change
Pour répondre à cette question, on remarque que les vecteurs $u$ et $v$ ne sont pas colinéaires, 
donc $G$ est de dimension $2$.

\change
 De plus $u$ et $v$ appartiennent à $F$ car leurs coordonnées vérifient l'équation qui définit le sous-espace $F$.
 Comme $G$ est engendré par $u$ et $v$, on en déduit que $G$ est contenu dans $F$.
  
\change
Pour trouver la dimension de $F$, on pourrait déterminer une base 
de $F$ et on montrerait alors que la dimension de $F$ est $2$.
Mais il est plus judicieux ici de remarquer que $F$ est contenu strictement dans $\Rr^3$ \\

Par exemple le vecteur de coordonnées $1, 0, 0$  n'est pas dans $F$
donc la dimension de $F$ est strictement plus petite que 3

\change
mais comme  $F$ contient $G$ alors la dimension de $F$ est aussi supérieure ou égal à la dimension de $G$ qui est $2$\\

donc la dimension de $F$ ne peut être que $2$.
  
  \change
  On a donc démontré que $G \subset F$ et que la dimension de $G$ et la dimension de $F$ sont égales à 2. 
  Le corollaire implique alors que  les deux sous-espaces co\"{i}ncident.

%%%%%%%%%%%%%%%%%%%%%%%%%%%%%%%%%%%%%%%%%%%%%%%%%%%%%%%%%%%
\diapo
Attaquons maintenant le théorème le plus important de cette le\c{c}on : le théorème des 4 dimensions. 
Dans ce théorème, $E$ est un espace vectoriel de dimension finie, 
et $F$ et $G$ sont deux sous-espaces vectoriels de $E$. Nous savons déjà que $F$ et $G$ sont de dimension finie, mais de plus on a la relation suivante : \\

la dimension de la somme $F+G$ est égale à la dimension de $F$ plus la dimension de $G$ moins la dimension de l'intersection de $F$ avec $G$. \\

Avant de donner les grandes lignes de la démonstration de ce théorème, nous allons donner deux corollaires et un exemple d'application.

\change
Le premier corollaire s'obtient à partir du théorème lorsque $E$ est somme directe de $F$ et  de $G$. Dans ce cas l'intersection entre $F$ et $G$ est réduite 
au vecteur nul et donc la dimension de l'intersection est 0. 
Ainsi la dimension de $E$ est simplement égale à la dimension de $F$ plus la dimension de $G$.

\change
Le second corollaire découlera de la démonstration du théorème des 4 dimensions. Il dit que tout sous-espace vectoriel d'un espace vectoriel de dimension finie admet un supplémentaire.

%%%%%%%%%%%%%%%%%%%%%%%%%%%%%%%%%%%%%%%%%%%%%%%%%%%%%%%%%%%
\diapo
Voyons maintenant un exemple d'application. Considérons  un espace vectoriel $E$ de dimension $6$, on considère deux sous-espaces $F$ et $G$ avec
$\dim F=3$ et $\dim G=4$. 


\change
Que peut-on dire de $F\cap G$ ? et de la somme $F+G$ ? L'espace $E$ peut-il être somme directe de $F$ et de $G$?

\change 
$F \cap G$ est un sous-espace vectoriel inclus dans $F$, donc 
  $\dim (F \cap G) \le \dim F=3$. Donc les dimensions possibles pour $F \cap G$ sont
  pour l'instant $0,1,2,3$.
  
  \change $F+G$ est un sous-espace vectoriel contenant $G$ et inclus dans $E$, donc la dimension de $F+G$ est comprise entre la dimension de $G$ qui est 4 et la dimension de $E$ qui est 6. Donc les dimensions possibles 
  pour $F + G$ sont $4,5,6$.
  
\change Le théorème  des quatre dimensions nous donne la relation suivante :
  $\dim(F\cap G) =  \dim F + \dim G - \dim(F+G) = 3+4-\dim(F+G) =  7-\dim(F+G)$. 
  Comme $F + G$ est de dimension $4$, $5$ ou $6$, alors la dimension de $F\cap G$ est 
  $3$, $2$ ou $1$. 
  
  \change Conclusion : les dimensions possibles pour $F+G$ sont $4$, $5$ ou $6$ ;
  les dimensions correspondantes pour $F \cap G$ sont alors $3$, $2$ ou $1$.
  
  \change
  Dans tous les cas, la dimension de $F\cap G$ est au moins un, donc l'intersection de $F$ et $G$ n'est pas réduite au vecteur nul.\\
  
  En particulier $F$ et $G$ ne sont jamais en somme directe
  dans $E$.




%%%%%%%%%%%%%%%%%%%%%%%%%%%%%%%%%%%%%%%%%%%%%%%%%%%%%%%%%%%
\diapo
Passons maintenant à la démonstration du théorème des 4 dimensions.


\change
Notons l'analogie de la formule des 4 dimensions avec la formule pour les ensembles finis :
$$\Card (A\cup B) = \Card A + \Card B - \Card (A\cap B).$$ En fait c'est cette formule des cardinaux que nous allons appliquer à des bases bien choisies de $F$ et de $G$ pour obtenir la formule des 4 dimensions.

\change
Nous allons partir d'une base  de $F \cap G$ constituée des vecteurs $\{u_1,\ldots,u_p\}$.

\change
Par le théorème de la base incomplète, on peut compléter cette base  en une base de $F$ en ajoutant des vecteurs $v_{p+1}, \ldots, v_q$ de $F$.

\change
On repart de la base de $F\cap G$ que l'on complète cette fois 
en une base de $G$ en ajoutant des vecteurs $w_{p+1},\ldots,w_r$ de $G$.

  
 \change Nous allons maintenant montrer que la famille formée des $u_i$, $v_j$ et $w_k$  est une base de $F+G$.
  Il est tout d'abord clair que c'est une famille génératrice de $F+G$ (car les $u_i$ et $v_j$ forment une base de $F$ et les $u_i$ et $w_k$ forment une base de $G$)  
  
  
 \change Montrons que cette famille est libre. 
 
 \change
 Considérons une combinaison linéaire nulle de ces vecteurs. On notera $\alpha_i$ les coefficients intervenants devant les vecteurs $u_i$ de $F\cap G$, $\beta_j$ les coefficients devant les vecteurs $v_j$, et enfin $\gamma_k$ seront les coefficients devant les vecteurs $w_k$.
  
 \change
 Considérons le vecteur 
$u$ défini comme la somme des $\alpha_i u_i$, le vecteur $v$ défini comme la somme des $\beta_j v_j$ et enfin le vecteur $w$ défini comme la somme des $\gamma_k w_k$.

\change
Alors la combinaison linéaire nulle s'écrit $u+v+w=0$, donc $u+v = -w$ appartient à  $G$\\


d'autre part  $u+v \in F$ car les vecteurs $u_i$ et $v_i$ forment une base de $F$

\change
Ainsi $u+v \in F\cap G$ et comme  $u$ est dans l'intersection $ F\cap G$, le vecteur $v$ qui s'écrit comme la différence entre $u+v$ et $u$ appartient également à $F\cap G$. Il est donc combinaison linéaire des vecteurs $u_i$ puisque ceux-ci forment une base de l'intersection de $F$ et de $G$. Cela implique $\beta_j=0$ pour tout $j$

\change  La combinaison linéaire nulle initiale devient alors 
  $$\sum_{i=1}^p \alpha_i u_i  + \sum_{k=p+1}^r \gamma_k w_k = 0$$ et on conclut que les $\alpha_i$ et les $\gamma_i$ sont nuls car l'union des $u_i$ et des $w_i$ forment une base de $G$ donc constitue une famille libre de $G$.
  
  \change
  Ainsi la famille constituée des $u_i$ des $v_j$ et des $w_k$ est une base de $F+G$.
  
  Il ne reste plus qu'à compter le nombre de vecteurs dans cette base pour obtenir le théorème des 4 dimensions.


%%%%%%%%%%%%%%%%%%%%%%%%%%%%%%%%%%%%%%%%%%%%%%%%%%%%%%%%%%%
\diapo
Entrainez-vous avec les exercices qui suivent pour vérifier votre compréhension du cours.


\end{document}
