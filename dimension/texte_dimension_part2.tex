
%%%%%%%%%%%%%%%%%% PREAMBULE %%%%%%%%%%%%%%%%%%


\documentclass[12pt]{article}

\usepackage{amsfonts,amsmath,amssymb,amsthm}
\usepackage[utf8]{inputenc}
\usepackage[T1]{fontenc}
\usepackage[francais]{babel}


% packages
\usepackage{amsfonts,amsmath,amssymb,amsthm}
\usepackage[utf8]{inputenc}
\usepackage[T1]{fontenc}
%\usepackage{lmodern}

\usepackage[francais]{babel}
\usepackage{fancybox}
\usepackage{graphicx}

\usepackage{float}

%\usepackage[usenames, x11names]{xcolor}
\usepackage{tikz}
\usepackage{datetime}

\usepackage{mathptmx}
%\usepackage{fouriernc}
%\usepackage{newcent}
\usepackage[mathcal,mathbf]{euler}

%\usepackage{palatino}
%\usepackage{newcent}


% Commande spéciale prompteur

%\usepackage{mathptmx}
%\usepackage[mathcal,mathbf]{euler}
%\usepackage{mathpple,multido}

\usepackage[a4paper]{geometry}
\geometry{top=2cm, bottom=2cm, left=1cm, right=1cm, marginparsep=1cm}

\newcommand{\change}{{\color{red}\rule{\textwidth}{1mm}\\}}

\newcounter{mydiapo}

\newcommand{\diapo}{\newpage
\hfill {\normalsize  Diapo \themydiapo \quad \texttt{[\jobname]}} \\
\stepcounter{mydiapo}}


%%%%%%% COULEURS %%%%%%%%%%

% Pour blanc sur noir :
%\pagecolor[rgb]{0.5,0.5,0.5}
% \pagecolor[rgb]{0,0,0}
% \color[rgb]{1,1,1}



%\DeclareFixedFont{\myfont}{U}{cmss}{bx}{n}{18pt}
\newcommand{\debuttexte}{
%%%%%%%%%%%%% FONTES %%%%%%%%%%%%%
\renewcommand{\baselinestretch}{1.5}
\usefont{U}{cmss}{bx}{n}
\bfseries

% Taille normale : commenter le reste !
%Taille Arnaud
%\fontsize{19}{19}\selectfont

% Taille Barbara
%\fontsize{21}{22}\selectfont

%Taille François
%\fontsize{25}{30}\selectfont

%Taille Pascal
%\fontsize{25}{30}\selectfont

%Taille Laura
%\fontsize{30}{35}\selectfont


%\myfont
%\usefont{U}{cmss}{bx}{n}

%\Huge
%\addtolength{\parskip}{\baselineskip}
}


% \usepackage{hyperref}
% \hypersetup{colorlinks=true, linkcolor=blue, urlcolor=blue,
% pdftitle={Exo7 - Exercices de mathématiques}, pdfauthor={Exo7}}


%section
% \usepackage{sectsty}
% \allsectionsfont{\bf}
%\sectionfont{\color{Tomato3}\upshape\selectfont}
%\subsectionfont{\color{Tomato4}\upshape\selectfont}

%----- Ensembles : entiers, reels, complexes -----
\newcommand{\Nn}{\mathbb{N}} \newcommand{\N}{\mathbb{N}}
\newcommand{\Zz}{\mathbb{Z}} \newcommand{\Z}{\mathbb{Z}}
\newcommand{\Qq}{\mathbb{Q}} \newcommand{\Q}{\mathbb{Q}}
\newcommand{\Rr}{\mathbb{R}} \newcommand{\R}{\mathbb{R}}
\newcommand{\Cc}{\mathbb{C}} 
\newcommand{\Kk}{\mathbb{K}} \newcommand{\K}{\mathbb{K}}

%----- Modifications de symboles -----
\renewcommand{\epsilon}{\varepsilon}
\renewcommand{\Re}{\mathop{\text{Re}}\nolimits}
\renewcommand{\Im}{\mathop{\text{Im}}\nolimits}
%\newcommand{\llbracket}{\left[\kern-0.15em\left[}
%\newcommand{\rrbracket}{\right]\kern-0.15em\right]}

\renewcommand{\ge}{\geqslant}
\renewcommand{\geq}{\geqslant}
\renewcommand{\le}{\leqslant}
\renewcommand{\leq}{\leqslant}

%----- Fonctions usuelles -----
\newcommand{\ch}{\mathop{\mathrm{ch}}\nolimits}
\newcommand{\sh}{\mathop{\mathrm{sh}}\nolimits}
\renewcommand{\tanh}{\mathop{\mathrm{th}}\nolimits}
\newcommand{\cotan}{\mathop{\mathrm{cotan}}\nolimits}
\newcommand{\Arcsin}{\mathop{\mathrm{Arcsin}}\nolimits}
\newcommand{\Arccos}{\mathop{\mathrm{Arccos}}\nolimits}
\newcommand{\Arctan}{\mathop{\mathrm{Arctan}}\nolimits}
\newcommand{\Argsh}{\mathop{\mathrm{Argsh}}\nolimits}
\newcommand{\Argch}{\mathop{\mathrm{Argch}}\nolimits}
\newcommand{\Argth}{\mathop{\mathrm{Argth}}\nolimits}
\newcommand{\pgcd}{\mathop{\mathrm{pgcd}}\nolimits} 

\newcommand{\Card}{\mathop{\text{Card}}\nolimits}
\newcommand{\Ker}{\mathop{\text{Ker}}\nolimits}
\newcommand{\id}{\mathop{\text{id}}\nolimits}
\newcommand{\ii}{\mathrm{i}}
\newcommand{\dd}{\mathrm{d}}
\newcommand{\Vect}{\mathop{\text{Vect}}\nolimits}
\newcommand{\Mat}{\mathop{\mathrm{Mat}}\nolimits}
\newcommand{\rg}{\mathop{\text{rg}}\nolimits}
\newcommand{\tr}{\mathop{\text{tr}}\nolimits}
\newcommand{\ppcm}{\mathop{\text{ppcm}}\nolimits}

%----- Structure des exercices ------

\newtheoremstyle{styleexo}% name
{2ex}% Space above
{3ex}% Space below
{}% Body font
{}% Indent amount 1
{\bfseries} % Theorem head font
{}% Punctuation after theorem head
{\newline}% Space after theorem head 2
{}% Theorem head spec (can be left empty, meaning ‘normal’)

%\theoremstyle{styleexo}
\newtheorem{exo}{Exercice}
\newtheorem{ind}{Indications}
\newtheorem{cor}{Correction}


\newcommand{\exercice}[1]{} \newcommand{\finexercice}{}
%\newcommand{\exercice}[1]{{\tiny\texttt{#1}}\vspace{-2ex}} % pour afficher le numero absolu, l'auteur...
\newcommand{\enonce}{\begin{exo}} \newcommand{\finenonce}{\end{exo}}
\newcommand{\indication}{\begin{ind}} \newcommand{\finindication}{\end{ind}}
\newcommand{\correction}{\begin{cor}} \newcommand{\fincorrection}{\end{cor}}

\newcommand{\noindication}{\stepcounter{ind}}
\newcommand{\nocorrection}{\stepcounter{cor}}

\newcommand{\fiche}[1]{} \newcommand{\finfiche}{}
\newcommand{\titre}[1]{\centerline{\large \bf #1}}
\newcommand{\addcommand}[1]{}
\newcommand{\video}[1]{}

% Marge
\newcommand{\mymargin}[1]{\marginpar{{\small #1}}}



%----- Presentation ------
\setlength{\parindent}{0cm}

%\newcommand{\ExoSept}{\href{http://exo7.emath.fr}{\textbf{\textsf{Exo7}}}}

\definecolor{myred}{rgb}{0.93,0.26,0}
\definecolor{myorange}{rgb}{0.97,0.58,0}
\definecolor{myyellow}{rgb}{1,0.86,0}

\newcommand{\LogoExoSept}[1]{  % input : echelle
{\usefont{U}{cmss}{bx}{n}
\begin{tikzpicture}[scale=0.1*#1,transform shape]
  \fill[color=myorange] (0,0)--(4,0)--(4,-4)--(0,-4)--cycle;
  \fill[color=myred] (0,0)--(0,3)--(-3,3)--(-3,0)--cycle;
  \fill[color=myyellow] (4,0)--(7,4)--(3,7)--(0,3)--cycle;
  \node[scale=5] at (3.5,3.5) {Exo7};
\end{tikzpicture}}
}



\theoremstyle{definition}
%\newtheorem{proposition}{Proposition}
%\newtheorem{exemple}{Exemple}
%\newtheorem{theoreme}{Théorème}
\newtheorem{lemme}{Lemme}
\newtheorem{corollaire}{Corollaire}
%\newtheorem*{remarque*}{Remarque}
%\newtheorem*{miniexercice}{Mini-exercices}
%\newtheorem{definition}{Définition}




%definition d'un terme
\newcommand{\defi}[1]{{\color{myorange}\textbf{\emph{#1}}}}
\newcommand{\evidence}[1]{{\color{blue}\textbf{\emph{#1}}}}



 %----- Commandes divers ------

\newcommand{\codeinline}[1]{\texttt{#1}}

%%%%%%%%%%%%%%%%%%%%%%%%%%%%%%%%%%%%%%%%%%%%%%%%%%%%%%%%%%%%%
%%%%%%%%%%%%%%%%%%%%%%%%%%%%%%%%%%%%%%%%%%%%%%%%%%%%%%%%%%%%%


\begin{document}

\debuttexte


%%%%%%%%%%%%%%%%%%%%%%%%%%%%%%%%%%%%%%%%%%%%%%%%%%%%%%%%%%%
\diapo
Dans la le\c{c}on précédente nous avons introduit la notion de \defi{famille libre} de vecteurs d'un espace vectoriel.
Dans cette le\c{c}on, nous allons définir  ce qu'est une famille \defi{génératrice} de vecteurs. 

\change
A l'aide de ces deux notions - famille libre d'une part, famille génératrice d'autre part, nous pourrons aborder, \defi{dans la prochaine le\c{c}on}, la notion de \defi{base} d'un espace vectoriel,  qui est une notion \defi{cruciale} en algèbre linéaire.

\change
Pour le moment, nous allons dans un premier temps \defi{donner} la définition de \defi{famille génératrice} de vecteurs,

\change
puis nous illustrerons cette définition par des \defi{exemples,} 

\change 
et enfin nous verrons les liens qu'il peut y avoir entre diverses familles génératrices. 



%%%%%%%%%%%%%%%%%%%%%%%%%%%%%%%%%%%%%%%%%%%%%%%%%%%%%%%%%%%
\diapo
Considérons un espace vectoriel sur un corps $\Kk$.

Nous dirons qu'une famille $\{v_1,\dots ,v_p\}$ de vecteurs de $E$ est une \defi{famille génératrice} de$E$ 
si tout vecteur de $E$ est une combinaison linéaire des vecteurs $v_1,\dots ,v_p$.

\change
Autrement dit, si on choisit n'importe quel vecteur $v$ de $E$, 

\change
il est possible de trouver des coefficients 
$ \lambda_1, \ldots,\lambda_p $ du corps de base $\Kk$ 

\change 
tels que $v$ s'écrive
$v=\lambda_1 v_1+\cdots + \lambda_p v_p.$

On dit aussi que la famille $\{v_1,\dots ,v_p\}$ \defi{engendre} l'espace vectoriel $E$. Dans ce cas,
 $E$ \defi{co\"incide} avec le sous-espace engendré par les vecteurs $v_1,\ldots,v_p$.




%%%%%%%%%%%%%%%%%%%%%%%%%%%%%%%%%%%%%%%%%%%%%%%%%%%%%%%%%%%
\diapo
Considérons par exemple les vecteurs $v_1$, $v_2$, $v_3$ de $\Rr^3$ suivants.

\change
Tout vecteur $v$ de $\Rr^3$ s'écrit en coordonnées sous la forme d'un vecteur colonne $x,y,z$

\change
qui peut se décomposer sous la forme 

\change
$x$ x $1er$ vecteur 

\change
+ $y$ x $2eme$ vecteur 

\change
+ $z$ x $3ieme$ vecteur.

Les \defi{coefficients} de $v$ vu comme combinaison linéaire de $v_1$, $v_2$ et $v_3$ sont ici 

\change
$\lambda_1=x$, 

\change
$\lambda_2=y$, 

\change
$\lambda_3=z$.

\change 
La famille $\{v_1,v_2,v_3\}$ est donc une famille \defi{génératrice} de $\Rr^3$.
%%%%%%%%%%%%%%%%%%%%%%%%%%%%%%%%%%%%%%%%%%%%%%%%%%%%%%%%%%%
\diapo

Considérons maintenant les deux vecteurs suivants.

\change
La famille $\{v_1, v_2\}$ est-elle génératrice de $\Rr^3$?

\change
Pour répondre à cette question considérons le vecteur particulier $v$ suivant. Essayons de l'écrire comme combinaisons linéaire de $v_1$ et $v_2$.

\change
Posons $v= \lambda_1 v_1 + \lambda_2 v_2$

\change
L'écriture en coordonnées donne lieu au système suivant.

\change
Ce système n'a \defi{pas} de solution. En effet, la première et la dernière ligne 
impliquent $\lambda_1$ et $\lambda_2=0$, ce qui est incompatible avec la deuxième ligne.

\change
Ainsi les vecteurs $v_1$ et $v_2$ \defi{ne forment pas} une famille génératrice de $\Rr^3$.

%%%%%%%%%%%%%%%%%%%%%%%%%%%%%%%%%%%%%%%%%%%%%%%%%%%%%%%%%%%
\diapo
Considérons maintenant les vecteurs $v_1$ et $v_2$ suivants dans l'espace vectoriel $\Rr^2$.


\change
 $\{v_1, v_2\}$ est une famille génératrice de $\Rr^2$ car tout vecteur de $\Rr^2$ de coordonnées $x$ et $y$ s'écrit comme combinaison linéaire de $v_1$ et $v_2$ avec pour coefficient $x$ $y$.


\change
Maintenant considérons de nouveaux vecteurs $v_1'$ et $v_2'$.

\change
Cette famille $\{v_1', v_2'\}$ est-elle génératrice de $\Rr^2$?
  
  \change
  Pour répondre, considérons un vecteur   $v$ \defi{quelconque} de $\Rr^2$ et essayons de l'écrire comme \defi{combinaison linéaire} de $v_1'$ et $v_2'$. 
  Nous sommes ramenés à étudier le système linéaire suivant 
  
  \change
  o\`u les inconnues sont les coefficients $\lambda$ et $\mu$ de la combinaison linéaire
  
  \change
  Ce système  a pour solution
  $\lambda =x -y$ et $\mu =-x+2y$ 
et ceci, \defi{quels que soient} les réels $x$ et $y$.

\change
Nous pouvons en conclure que $\{v_1', v_2'\}$ est aussi une famille génératrice de $\Rr^2$.
 
 
 \change
Cet exemple prouve qu'il peut exister plusieurs familles finies différentes, 
% non incluses les unes dans les autres, 
génératrices du même espace vectoriel.


%%%%%%%%%%%%%%%%%%%%%%%%%%%%%%%%%%%%%%%%%%%%%%%%%%%%%%%%%%%
\diapo
La proposition suivante est souvent utile :


Si on se donne une famille  $\mathcal{F}$  génératrice de $E$, alors une autre famille 
 $\mathcal{F}'$ est aussi une famille
génératrice de $E$ si et seulement si tout vecteur de $\mathcal{F}$ 
est une combinaison linéaire de vecteurs de $\mathcal{F}'$. 



%%%%%%%%%%%%%%%%%%%%%%%%%%%%%%%%%%%%%%%%%%%%%%%%%%%%%%%%%%%
\diapo
Nous allons maintenant nous intéresser à réduire 
% le nombre de générateurs d'un espace vectoriel, c'est-à-dire réduire 
le nombre d'éléments d'une famille génératrice de vecteurs. 
\\
Pour cela nous pouvons utiliser la proposition suivante.
\\
Si la famille $\mathcal{F} $ constituée des vecteurs   $\{v_1,\ldots,v_p\}$ \defi{engendre} $E$ et si l'un des vecteurs, 
par exemple $v_p$, est combinaison linéaire des autres, alors la famille 
$\mathcal{F}$ \defi{privée} de $ \{v_p\}$
est encore une famille génératrice de $E$.

\change
En effet, comme les vecteurs  $v_1,\dots ,v_p$ engendrent $E$, tout vecteur 
 $v$ de $E$ est combinaison linéaire de $v_1,\dots ,v_p$. Nous notons 
 $\lambda_1, \cdots, \lambda_p$ les coefficients d'une telle combinaison.
 
 \change
Or par  hypothèse $v_p$ est combinaison linéaire des vecteurs  
$v_1,\dots ,v_{p-1}$ ce qui se traduit par l'existence de scalaires 
$\alpha_1, \ldots, \alpha_{p-1}$ tels que
$v_p $ s'écrive comme la somme $\alpha_1 v_1 + \cdots + \alpha_{p-1}v_{p-1}.$

\change
Alors, lorsqu'on remplace le vecteur $v_p$ par son écriture en fonction des autres vecteurs 
nous obtenons une écriture de $v$ comme combinaison linéaire des vecteurs $v_1$ à $v_{p-1}$ :
$v=\left ( \lambda_1+\lambda_{p}\alpha_1\right )v_1+\dots + 
 \left ( \lambda_{p-1}+\lambda_{p}\alpha_{p-1}\right )v_{p-1}$

Ceci achève la démonstration car nous avons montré que tout vecteur de $E$ est en fait combinaison linéaire des $(p-1)$ premiers vecteurs seulement.

% Par ailleurs  il  est clair que si l'on remplace $v_p$ par n'importe lequel des 
% vecteurs de la famille, la démonstration est analogue.  


%%%%%%%%%%%%%%%%%%%%%%%%%%%%%%%%%%%%%%%%%%%%%%%%%%%%%%%%%%%
\diapo
Entrainez-vous à faire les exercices qui correspondent à cette le\c{c}on pour vérifier votre compréhension du cours.

\end{document}
