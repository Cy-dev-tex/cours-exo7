
%%%%%%%%%%%%%%%%%% PREAMBULE %%%%%%%%%%%%%%%%%%

\documentclass[aspectratio=169,utf8]{beamer}
%\documentclass[aspectratio=169,handout]{beamer}

\usetheme{Boadilla}
%\usecolortheme{seahorse}
\usecolortheme[RGB={245,66,24}]{structure}
\useoutertheme{infolines}

% packages
\usepackage{amsfonts,amsmath,amssymb,amsthm}
\usepackage[utf8]{inputenc}
\usepackage[T1]{fontenc}
\usepackage{lmodern}

\usepackage[francais]{babel}
\usepackage{fancybox}
\usepackage{graphicx}

\usepackage{float}
\usepackage{xfrac}

%\usepackage[usenames, x11names]{xcolor}
\usepackage{tikz}
\usepackage{pgfplots}
\usepackage{datetime}



%-----  Package unités -----
\usepackage{siunitx}
\sisetup{locale = FR,detect-all,per-mode = symbol}

%\usepackage{mathptmx}
%\usepackage{fouriernc}
%\usepackage{newcent}
%\usepackage[mathcal,mathbf]{euler}

%\usepackage{palatino}
%\usepackage{newcent}
% \usepackage[mathcal,mathbf]{euler}



% \usepackage{hyperref}
% \hypersetup{colorlinks=true, linkcolor=blue, urlcolor=blue,
% pdftitle={Exo7 - Exercices de mathématiques}, pdfauthor={Exo7}}


%section
% \usepackage{sectsty}
% \allsectionsfont{\bf}
%\sectionfont{\color{Tomato3}\upshape\selectfont}
%\subsectionfont{\color{Tomato4}\upshape\selectfont}

%----- Ensembles : entiers, reels, complexes -----
\newcommand{\Nn}{\mathbb{N}} \newcommand{\N}{\mathbb{N}}
\newcommand{\Zz}{\mathbb{Z}} \newcommand{\Z}{\mathbb{Z}}
\newcommand{\Qq}{\mathbb{Q}} \newcommand{\Q}{\mathbb{Q}}
\newcommand{\Rr}{\mathbb{R}} \newcommand{\R}{\mathbb{R}}
\newcommand{\Cc}{\mathbb{C}} 
\newcommand{\Kk}{\mathbb{K}} \newcommand{\K}{\mathbb{K}}

%----- Modifications de symboles -----
\renewcommand{\epsilon}{\varepsilon}
\renewcommand{\Re}{\mathop{\text{Re}}\nolimits}
\renewcommand{\Im}{\mathop{\text{Im}}\nolimits}
%\newcommand{\llbracket}{\left[\kern-0.15em\left[}
%\newcommand{\rrbracket}{\right]\kern-0.15em\right]}

\renewcommand{\ge}{\geqslant}
\renewcommand{\geq}{\geqslant}
\renewcommand{\le}{\leqslant}
\renewcommand{\leq}{\leqslant}
\renewcommand{\epsilon}{\varepsilon}

%----- Fonctions usuelles -----
\newcommand{\ch}{\mathop{\text{ch}}\nolimits}
\newcommand{\sh}{\mathop{\text{sh}}\nolimits}
\renewcommand{\tanh}{\mathop{\text{th}}\nolimits}
\newcommand{\cotan}{\mathop{\text{cotan}}\nolimits}
\newcommand{\Arcsin}{\mathop{\text{arcsin}}\nolimits}
\newcommand{\Arccos}{\mathop{\text{arccos}}\nolimits}
\newcommand{\Arctan}{\mathop{\text{arctan}}\nolimits}
\newcommand{\Argsh}{\mathop{\text{argsh}}\nolimits}
\newcommand{\Argch}{\mathop{\text{argch}}\nolimits}
\newcommand{\Argth}{\mathop{\text{argth}}\nolimits}
\newcommand{\pgcd}{\mathop{\text{pgcd}}\nolimits} 


%----- Commandes divers ------
\newcommand{\ii}{\mathrm{i}}
\newcommand{\dd}{\text{d}}
\newcommand{\id}{\mathop{\text{id}}\nolimits}
\newcommand{\Ker}{\mathop{\text{Ker}}\nolimits}
\newcommand{\Card}{\mathop{\text{Card}}\nolimits}
\newcommand{\Vect}{\mathop{\text{Vect}}\nolimits}
\newcommand{\Mat}{\mathop{\text{Mat}}\nolimits}
\newcommand{\rg}{\mathop{\text{rg}}\nolimits}
\newcommand{\tr}{\mathop{\text{tr}}\nolimits}


%----- Structure des exercices ------

\newtheoremstyle{styleexo}% name
{2ex}% Space above
{3ex}% Space below
{}% Body font
{}% Indent amount 1
{\bfseries} % Theorem head font
{}% Punctuation after theorem head
{\newline}% Space after theorem head 2
{}% Theorem head spec (can be left empty, meaning ‘normal’)

%\theoremstyle{styleexo}
\newtheorem{exo}{Exercice}
\newtheorem{ind}{Indications}
\newtheorem{cor}{Correction}


\newcommand{\exercice}[1]{} \newcommand{\finexercice}{}
%\newcommand{\exercice}[1]{{\tiny\texttt{#1}}\vspace{-2ex}} % pour afficher le numero absolu, l'auteur...
\newcommand{\enonce}{\begin{exo}} \newcommand{\finenonce}{\end{exo}}
\newcommand{\indication}{\begin{ind}} \newcommand{\finindication}{\end{ind}}
\newcommand{\correction}{\begin{cor}} \newcommand{\fincorrection}{\end{cor}}

\newcommand{\noindication}{\stepcounter{ind}}
\newcommand{\nocorrection}{\stepcounter{cor}}

\newcommand{\fiche}[1]{} \newcommand{\finfiche}{}
\newcommand{\titre}[1]{\centerline{\large \bf #1}}
\newcommand{\addcommand}[1]{}
\newcommand{\video}[1]{}

% Marge
\newcommand{\mymargin}[1]{\marginpar{{\small #1}}}

\def\noqed{\renewcommand{\qedsymbol}{}}


%----- Presentation ------
\setlength{\parindent}{0cm}

%\newcommand{\ExoSept}{\href{http://exo7.emath.fr}{\textbf{\textsf{Exo7}}}}

\definecolor{myred}{rgb}{0.93,0.26,0}
\definecolor{myorange}{rgb}{0.97,0.58,0}
\definecolor{myyellow}{rgb}{1,0.86,0}

\newcommand{\LogoExoSept}[1]{  % input : echelle
{\usefont{U}{cmss}{bx}{n}
\begin{tikzpicture}[scale=0.1*#1,transform shape]
  \fill[color=myorange] (0,0)--(4,0)--(4,-4)--(0,-4)--cycle;
  \fill[color=myred] (0,0)--(0,3)--(-3,3)--(-3,0)--cycle;
  \fill[color=myyellow] (4,0)--(7,4)--(3,7)--(0,3)--cycle;
  \node[scale=5] at (3.5,3.5) {Exo7};
\end{tikzpicture}}
}


\newcommand{\debutmontitre}{
  \author{} \date{} 
  \thispagestyle{empty}
  \hspace*{-10ex}
  \begin{minipage}{\textwidth}
    \titlepage  
  \vspace*{-2.5cm}
  \begin{center}
    \LogoExoSept{2.5}
  \end{center}
  \end{minipage}

  \vspace*{-0cm}
  
  % Astuce pour que le background ne soit pas discrétisé lors de la conversion pdf -> png
\begin{tikzpicture}
        \fill[opacity=0,green!60!black] (0,0)--++(0,0)--++(0,0)--++(0,0)--cycle; 
\end{tikzpicture}

% toc S'affiche trop tot :
% \tableofcontents[hideallsubsections, pausesections]
}

\newcommand{\finmontitre}{
  \end{frame}
  \setcounter{framenumber}{0}
} % ne marche pas pour une raison obscure

%----- Commandes supplementaires ------

% \usepackage[landscape]{geometry}
% \geometry{top=1cm, bottom=3cm, left=2cm, right=10cm, marginparsep=1cm
% }
% \usepackage[a4paper]{geometry}
% \geometry{top=2cm, bottom=2cm, left=2cm, right=2cm, marginparsep=1cm
% }

%\usepackage{standalone}


% New command Arnaud -- november 2011
\setbeamersize{text margin left=24ex}
% si vous modifier cette valeur il faut aussi
% modifier le decalage du titre pour compenser
% (ex : ici =+10ex, titre =-5ex

\theoremstyle{definition}
%\newtheorem{proposition}{Proposition}
%\newtheorem{exemple}{Exemple}
%\newtheorem{theoreme}{Théorème}
%\newtheorem{lemme}{Lemme}
%\newtheorem{corollaire}{Corollaire}
%\newtheorem*{remarque*}{Remarque}
%\newtheorem*{miniexercice}{Mini-exercices}
%\newtheorem{definition}{Définition}

% Commande tikz
\usetikzlibrary{calc}
\usetikzlibrary{patterns,arrows}
\usetikzlibrary{matrix}
\usetikzlibrary{fadings} 

%definition d'un terme
\newcommand{\defi}[1]{{\color{myorange}\textbf{\emph{#1}}}}
\newcommand{\evidence}[1]{{\color{blue}\textbf{\emph{#1}}}}
\newcommand{\assertion}[1]{\emph{\og#1\fg}}  % pour chapitre logique
%\renewcommand{\contentsname}{Sommaire}
\renewcommand{\contentsname}{}
\setcounter{tocdepth}{2}



%------ Figures ------

\def\myscale{1} % par défaut 
\newcommand{\myfigure}[2]{  % entrée : echelle, fichier figure
\def\myscale{#1}
\begin{center}
\footnotesize
{#2}
\end{center}}


%------ Encadrement ------

\usepackage{fancybox}


\newcommand{\mybox}[1]{
\setlength{\fboxsep}{7pt}
\begin{center}
\shadowbox{#1}
\end{center}}

\newcommand{\myboxinline}[1]{
\setlength{\fboxsep}{5pt}
\raisebox{-10pt}{
\shadowbox{#1}
}
}

%--------------- Commande beamer---------------
\newcommand{\beameronly}[1]{#1} % permet de mettre des pause dans beamer pas dans poly


\setbeamertemplate{navigation symbols}{}
\setbeamertemplate{footline}  % tiré du fichier beamerouterinfolines.sty
{
  \leavevmode%
  \hbox{%
  \begin{beamercolorbox}[wd=.333333\paperwidth,ht=2.25ex,dp=1ex,center]{author in head/foot}%
    % \usebeamerfont{author in head/foot}\insertshortauthor%~~(\insertshortinstitute)
    \usebeamerfont{section in head/foot}{\bf\insertshorttitle}
  \end{beamercolorbox}%
  \begin{beamercolorbox}[wd=.333333\paperwidth,ht=2.25ex,dp=1ex,center]{title in head/foot}%
    \usebeamerfont{section in head/foot}{\bf\insertsectionhead}
  \end{beamercolorbox}%
  \begin{beamercolorbox}[wd=.333333\paperwidth,ht=2.25ex,dp=1ex,right]{date in head/foot}%
    % \usebeamerfont{date in head/foot}\insertshortdate{}\hspace*{2em}
    \insertframenumber{} / \inserttotalframenumber\hspace*{2ex} 
  \end{beamercolorbox}}%
  \vskip0pt%
}


\definecolor{mygrey}{rgb}{0.5,0.5,0.5}
\setlength{\parindent}{0cm}
%\DeclareTextFontCommand{\helvetica}{\fontfamily{phv}\selectfont}

% background beamer
\definecolor{couleurhaut}{rgb}{0.85,0.9,1}  % creme
\definecolor{couleurmilieu}{rgb}{1,1,1}  % vert pale
\definecolor{couleurbas}{rgb}{0.85,0.9,1}  % blanc
\setbeamertemplate{background canvas}[vertical shading]%
[top=couleurhaut,middle=couleurmilieu,midpoint=0.4,bottom=couleurbas] 
%[top=fondtitre!05,bottom=fondtitre!60]



\makeatletter
\setbeamertemplate{theorem begin}
{%
  \begin{\inserttheoremblockenv}
  {%
    \inserttheoremheadfont
    \inserttheoremname
    \inserttheoremnumber
    \ifx\inserttheoremaddition\@empty\else\ (\inserttheoremaddition)\fi%
    \inserttheorempunctuation
  }%
}
\setbeamertemplate{theorem end}{\end{\inserttheoremblockenv}}

\newenvironment{theoreme}[1][]{%
   \setbeamercolor{block title}{fg=structure,bg=structure!40}
   \setbeamercolor{block body}{fg=black,bg=structure!10}
   \begin{block}{{\bf Th\'eor\`eme }#1}
}{%
   \end{block}%
}


\newenvironment{proposition}[1][]{%
   \setbeamercolor{block title}{fg=structure,bg=structure!40}
   \setbeamercolor{block body}{fg=black,bg=structure!10}
   \begin{block}{{\bf Proposition }#1}
}{%
   \end{block}%
}

\newenvironment{corollaire}[1][]{%
   \setbeamercolor{block title}{fg=structure,bg=structure!40}
   \setbeamercolor{block body}{fg=black,bg=structure!10}
   \begin{block}{{\bf Corollaire }#1}
}{%
   \end{block}%
}

\newenvironment{mydefinition}[1][]{%
   \setbeamercolor{block title}{fg=structure,bg=structure!40}
   \setbeamercolor{block body}{fg=black,bg=structure!10}
   \begin{block}{{\bf Définition} #1}
}{%
   \end{block}%
}

\newenvironment{lemme}[0]{%
   \setbeamercolor{block title}{fg=structure,bg=structure!40}
   \setbeamercolor{block body}{fg=black,bg=structure!10}
   \begin{block}{\bf Lemme}
}{%
   \end{block}%
}

\newenvironment{remarque}[1][]{%
   \setbeamercolor{block title}{fg=black,bg=structure!20}
   \setbeamercolor{block body}{fg=black,bg=structure!5}
   \begin{block}{Remarque #1}
}{%
   \end{block}%
}


\newenvironment{exemple}[1][]{%
   \setbeamercolor{block title}{fg=black,bg=structure!20}
   \setbeamercolor{block body}{fg=black,bg=structure!5}
   \begin{block}{{\bf Exemple }#1}
}{%
   \end{block}%
}


\newenvironment{miniexercice}[0]{%
   \setbeamercolor{block title}{fg=structure,bg=structure!20}
   \setbeamercolor{block body}{fg=black,bg=structure!5}
   \begin{block}{Mini-exercices}
}{%
   \end{block}%
}


\newenvironment{tp}[0]{%
   \setbeamercolor{block title}{fg=structure,bg=structure!40}
   \setbeamercolor{block body}{fg=black,bg=structure!10}
   \begin{block}{\bf Travaux pratiques}
}{%
   \end{block}%
}
\newenvironment{exercicecours}[1][]{%
   \setbeamercolor{block title}{fg=structure,bg=structure!40}
   \setbeamercolor{block body}{fg=black,bg=structure!10}
   \begin{block}{{\bf Exercice }#1}
}{%
   \end{block}%
}
\newenvironment{algo}[1][]{%
   \setbeamercolor{block title}{fg=structure,bg=structure!40}
   \setbeamercolor{block body}{fg=black,bg=structure!10}
   \begin{block}{{\bf Algorithme}\hfill{\color{gray}\texttt{#1}}}
}{%
   \end{block}%
}


\setbeamertemplate{proof begin}{
   \setbeamercolor{block title}{fg=black,bg=structure!20}
   \setbeamercolor{block body}{fg=black,bg=structure!5}
   \begin{block}{{\footnotesize Démonstration}}
   \footnotesize
   \smallskip}
\setbeamertemplate{proof end}{%
   \end{block}}
\setbeamertemplate{qed symbol}{\openbox}


\makeatother
\usecolortheme[RGB={179,179,12}]{structure}

%%%%%%%%%%%%%%%%%%%%%%%%%%%%%%%%%%%%%%%%%%%%%%%%%%%%%%%%%%%%%
%%%%%%%%%%%%%%%%%%%%%%%%%%%%%%%%%%%%%%%%%%%%%%%%%%%%%%%%%%%%%


\begin{document}


\title{{\bf Développements limités}}
\subtitle{Applications}

\begin{frame}
  
  \debutmontitre

  \pause

{\footnotesize
\hfill
\setbeamercovered{transparent=50}
\begin{minipage}{0.6\textwidth}
  \begin{itemize}
    \item<3-> Calculs de limites
    \item<4-> Position d'une courbe par rapport à sa tangente
    \item<5-> Développement limité en $+\infty$
  \end{itemize}
\end{minipage}
}

\end{frame}

\setcounter{framenumber}{0}


%%%%%%%%%%%%%%%%%%%%%%%%%%%%%%%%%%%%%%%%%%%%%%%%%%%%%%%%%%%%%%%%




%---------------------------------------------------------------
\section{Calculs de limites}

\begin{frame}


Si \quad $f(x) = c_0+c_1(x-a)+\cdots$ \quad alors \quad $\lim_{x\to a} f(x) = c_0$

\pause

\begin{exemple}
Limite en $0$ de $\frac{\ln(1+x)-\tan x+\frac{1}{2}\sin^2x}{3x^2\sin^2x}$ 

\pause

\begin{enumerate}
  \item  \ \\ \vspace*{-3ex}
$\begin{array}{l@{\vrule depth 1.2ex height 3ex width 0mm }}
f(x)=\ln(1+x)-\tan x+\frac{1}{2}\sin^2x \\ \pause
\hspace*{-3em}=\!\big(x-\frac{x^2}{2}+\frac{x^3}{3}-\frac{x^4}{4}+o(x^4)\big)\pause
-\big(x+\frac{x^3}{3}+o(x^4)\big)\pause
 +\frac{1}{2}\big(x-\frac{x^3}{6}+o(x^3)\big)^{\!2} \\ \pause
\hspace*{-3em}=-\frac{x^2}{2}-\frac{x^4}{4}+\frac{1}{2}(x^2-\frac{1}{3}x^4)+o(x^4) \\ \pause
\hspace*{-3em}= -\frac{5}{12}x^4 + o(x^4)
\end{array}$
\pause
  \item $g(x)=3x^2\sin^2x\pause=3x^2\big(x+o(x)\big)^2 \pause=3x^4+o(x^4)$
\pause
  \item $\frac{f(x)}{g(x)}= \frac{-\frac{5}{12}x^4 + o(x^4)}{3x^4+o(x^4)} \pause = \frac{-\frac{5}{12} + o(1)}{3+o(1)}$
\pause
  \item $\lim_{x\to0}\frac{f(x)}{g(x)}=-\frac{5}{36}$
\end{enumerate}
\end{exemple}
  
\end{frame}



%---------------------------------------------------------------
\section{Position d'une courbe par rapport à sa tangente}

\begin{frame}

\begin{proposition}
Soit $f : I \to \Rr$ une fonction admettant un DL en $a$ :
$f(x)=c_0+c_1(x-a)+c_k(x-a)^k+(x-a)^k\epsilon(x)$
\pause
\begin{enumerate}
  \item La tangente à la courbe de $f$ en $a$ est $y=c_0+c_1(x-a)$
\pause
  \item La position de la courbe par rapport à la tangente pour $x$ proche de $a$ est
donnée par le signe $f(x)-y$, i.e. le signe de $c_k(x-a)^k$
\end{enumerate}

\end{proposition}
\pause

\begin{itemize}
  \item Si le signe est positif alors la courbe est au-dessus de la tangente
\pause
  \item Si le signe est négatif alors la courbe est en-dessous de la tangente
\pause
  \item Si le signe change (lorsque l'on passe de $x<a$ à $x>a$) alors la courbe traverse 
la tangente en $a$. \pause C'est un \defi{point d'inflexion}
\pause

Si $a$ est un point d'inflexion alors $f''(a)=0$
\end{itemize}
\end{frame}

\begin{frame}

\myfigure{0.8}{
\tikzinput{fig_dl03} 
\pause
\tikzinput{fig_dl04} 
}  

\pause


\myfigure{0.8}{
\tikzinput{fig_dl05} 
}  

\end{frame}


\begin{frame}
\begin{exemple}
Tangente et position de la courbe $f(x)=x^4-2x^3+1$ en $\frac{1}{2}$
\pause
\begin{itemize}
  \item $f'(x)=4x^3-6x^2$, \pause $f''(x)=12x^2-12x$, donc $f''(\frac{1}{2}) = -3 \neq 0$ et $k=2$ 
\pause
  \item DL de $f$ en $\frac{1}{2}$ par la formule de Taylor-Young \pause
$\begin{array}{rcl@{\vrule depth 1.2ex height 3ex width 0mm }}
 f(x) 
& = & f(\frac12)+f'(\frac12)(x-\frac12)+\frac{f''(\frac12)}{2!}(x-\frac12)^2 +(x-\frac12)^2 \epsilon(x) \\ \pause
& = & \frac{13}{16} -(x- \frac12) -\frac{3}{2}(x-\frac12)^2 + (x-\frac12)^2 \epsilon(x) \\ 
 \end{array}$

\pause
  \item tangente en $\frac12$ est $y= \frac{13}{16} -(x- \frac12)$
\pause
  \item $f(x)-y =  \big(-\frac32 + \epsilon(x)\big)(x-\frac12)^2$ est négatif autour de $x=\frac12$
\pause
  \item le graphe de $f$ est en-dessous de la tangente
\end{itemize}
\end{exemple}
\end{frame}


\begin{frame}
\myfigure{1}{
\tikzinput{fig_dl06a} 
}
\end{frame}


\begin{frame}
\begin{exemple}
Points d'inflexion de $f(x)=x^4-2x^3+1$

\pause

\begin{itemize}
  \item Si $a$ point d'inflexion alors $f''(a)=0$
\pause
  \item Comme $f''(x)=12x^2-12x$ alors $a=0$ ou $a=1$
\pause
  \item 
  \begin{itemize}
     \item DL en $0$ :  $f(x)= 1-2x^3+x^4$
\pause
     \item tangente en $0$ : $y=1$
\pause
     \item $f(x)-y=-2x^3+x^3\epsilon(x)$ change de signe
\pause
     \item $0$ est un point d'inflexion 
  \end{itemize}
\pause
  \item 
  \begin{itemize}
     \item DL en $1$ :  $f(x)=-2(x-1) + 2(x-1)^3+(x-1)^4$
\pause
     \item tangente en $1$ : $y=-2(x-1)$
\pause
     \item $f(x)-y=2(x-1)^3+(x-1)^3\epsilon(x)$ change de signe 
\pause
     \item $1$ est un point d'inflexion 
  \end{itemize}
\end{itemize}



\end{exemple}
\end{frame}

\begin{frame}
\myfigure{1}{
\tikzinput{fig_dl06b} 
}
\end{frame}


%---------------------------------------------------------------
\section{Développement limité en $+\infty$}

\begin{frame}

$f : ]x_0,+\infty[ \to \Rr$ admet un \defi{DL en $+\infty$} à l'ordre $n$
s'il existe des réels $c_0,c_1,\ldots,c_n$ tels que 
$$f(x)=c_0+\frac{c_1}{x}+\cdots+\frac{c_n}{x^n} 
+\frac{1}{x^n}\epsilon\big(\frac{1}{x}\big)$$
\pause
où $\epsilon\big(\frac{1}{x}\big) \to 0$ quand $x\to+\infty$

\pause
\medskip

\mybox{$f(x)$ a un DL en $+\infty$ \quad $\iff$ \quad  $f(\frac{1}{x})$ a un DL en $0^+$}

\pause

\begin{exemple}
\hfill
\begin{minipage}{0.5\textwidth}
\uncover<5->{\myfigure{0.45}{
\tikzinput{fig_dl07} 
}}  
\end{minipage}

\vspace*{-33mm}
\begin{minipage}{0.5\textwidth}
$$
\begin{array}{rcl@{\vrule depth 1.2ex height 3ex width 0mm }}
f(x)
 & = & \ln\big(2+\frac{1}{x}\big) \\ \pause \pause
 & = & \ln2+\ln\big(1+\frac{1}{2x}\big) \\ \pause
 & = & \ln2+\frac{1}{2x}-\frac{1}{8x^2}+\frac{1}{24x^3}+\cdots \\
& & \cdots +(-1)^{n-1}\frac{1}{n2^nx^n} +\frac{1}{x^{n}}\epsilon(\frac{1}{x}) 
\end{array}$$
  
\end{minipage}


\end{exemple}

\end{frame}


\begin{frame}
\begin{proposition} 
Si la  fonction $\frac{f(x)}{x}$ a un DL en $+\infty$
\vspace*{-3mm}
$$\frac{f(x)}{x}= a_0 +\frac{a_1}{x}+\frac{a_k}{x^k}+\frac{1}{x^k}\epsilon(\frac{1}{x})$$
\vspace*{-7mm}
\pause
\begin{enumerate}
  \item $\lim_{x\to+\infty} f(x)-(a_0x+a_1) =0$ : la droite $y= a_0x+a_1$ est une \defi{asymptote} à la courbe de $f$ en
$+\infty$
\uncover<4->{
  \item la position de la courbe par rapport à 
l'asymptote est donnée par le signe de $f(x)-y$, c'est-à-dire le signe de $\frac{a_k}{x^{k-1}}$
}
\end{enumerate}
\end{proposition}
\pause
\myfigure{0.7}{
\tikzinput{fig_dl08} 
}
\end{frame}

\begin{frame}
\begin{exemple}[Asymptote de  $f(x)=\exp{\frac1x} \cdot \sqrt{x^2-1}$ en $+\infty$]
\pause
\myfigure{0.7}{
\tikzinput{fig_dl09} 
}
\pause
\vspace*{-5mm}
$$
\begin{array}{rcl@{\vrule depth 1.2ex height 3ex width 0mm }}
\frac{f(x)}{x} \pause
  & = & \exp{\frac1x} \cdot\frac{\sqrt{x^2-1}}{x} \pause =\exp{\frac1x} \cdot\sqrt{1-\frac{1}{x^2}} \\ \pause
  & = & \Big(1+\frac{1}{x}+\frac{1}{2x^2}+\frac{1}{6x^3} +\frac{1}{x^3}\epsilon(\frac1x)\Big) 
\cdot\Big(1-\frac{1}{2x^2}+\frac{1}{x^3}\epsilon(\frac{1}{x})\Big) \\ \pause
  & = & \cdots = 1+\frac{1}{x}-\frac{1}{3x^3} +\frac{1}{x^3}\epsilon(\frac{1}{x}) \\
\end{array}$$

\pause 

Asymptote de $f$ en $+\infty$ est $y=x+1$ \pause\pause \hfill $f(x)-x-1=-\frac{1}{3x^2}+\frac{1}{x^2}\epsilon(\frac{1}{x})$
\pause
graphe de $f$ est en dessous de l'asymptote quand $x\to+\infty$
\end{exemple}

\end{frame}

%%%%%%%%%%%%%%%%%%%%%%%%%%%%%%%%%%%%%%%%%%%%%%%%%%%%%%%%%%%%%%%%
\section{Mini-exercices}

\begin{frame}

\begin{miniexercice}
\begin{enumerate}
  \item Calculer la limite de $\displaystyle \frac{\sin x - x}{x^3}$ lorsque $x$ tend vers $0$. 
Idem avec $\displaystyle \frac{\sqrt{1+x}-\sh \frac x2}{x^k}$ (pour $k=1,2,3,\ldots$).
  \item Calculer la limite de $\displaystyle \frac{\sqrt x - 1}{\ln x}$ lorsque $x$ tend vers $1$. 
Idem pour $\displaystyle \left(\frac{1-x}{1+x}\right)^{\frac{1}{x}}$, 
puis $\displaystyle \frac{1}{\tan^2 x}-\frac{1}{x^2}$ lorsque $x$ tend vers $0$.
  \item Soit $f(x)=\exp x + \sin x$. Calculer l'équation de la tangente en $x=0$ et la position du graphe.
Idem avec $g(x)= \sh x$.
  \item Calculer le DL en $+\infty$ à l'ordre $5$ de $\frac{x}{x^2-1}$. Idem à l'ordre $2$ pour
$\big(1+\frac 1x\big)^{x}$.
  \item Soit $f(x)=\sqrt{\frac{x^3+1}{x+1}}$. Déterminer l'asymptote en $+\infty$ et la position du graphe
par rapport à cette asymptote.
\end{enumerate}
\end{miniexercice}
\end{frame}

\end{document}